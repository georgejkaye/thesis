\chapter{Some math environments}

\lipsum[5]

A few citations~\cite{SGA1,LaTeX}, a url \url{https://latex.org} and a forward
reference to a result, see~\cref{thm} below.

\section{Definitions, theorems and proofs}

\begin{definition}\index{simplicial!hom-object}\index{simplicial!mor-object}
  A \emph{\(C\)-simplicial hom-object} a lift of a monoid \(B \to C\) and a
  \emph{\(C\)-simplicial mor-object} is...
\end{definition}

\begin{lemma}
  The formula \(\sqrt a \cdot \sqrt b = \sqrt{ab}\) holds.
  %
  \nomenclature[sqrt]{$\sqrt{}$}{square root}
\end{lemma}

\begin{proposition}\index{functor}\index{foo}\index{bar}
  If a functor \(F : \mathcal C \to \mathcal D\) is foo, then it is bar.
  \nomenclature[C]{$\mathcal C$}{a category}
\end{proposition}

\section{Definitions, theorems and proofs}

\lipsum[4]

\subsection{Subsection 1}

\lipsum[5]

\subsection{Subsection 2}

\lipsum[7]

\begin{theorem}\label{thm}
  A map \(f : A \to B\) is foo if and only if all its fibres are bar.
\end{theorem}
\begin{proof}
  \lipsum[7]
  \[
    \int_0^\infty e^{-\alpha x^2} \mathrm{d}x =
    \frac12\sqrt{\int_{-\infty}^\infty e^{-\alpha x^2}}
    \mathrm{d}x\int_{-\infty}^\infty e^{-\alpha y^2}\mathrm{d}y =
    \frac12\sqrt{\frac{\pi}{\alpha}}
  \]
  \lipsum[8]
\end{proof}

\begin{corollary}
  This corollary is some immediate consequence of the theorem.
\end{corollary}

\begin{remark}
  \lipsum[1-2]
\end{remark}

\begin{example}
  \lipsum[5]
\end{example}

See~\cref{table} below.
\begin{table}[h]%
  \centering
  \begin{tabular}{llp{4cm}l}\toprule
    & Name & Affiliation & Email address \\\midrule
    1. & Dr.~Abc Def
       & University of Foo
       & \href{mailto:abc.def@foo.ac.uk}{\texttt{abc.def@foo.ac.uk}} \\
    2. & Prof.~Qwerty Xyz
       & University of Bar
       & \href{mailto:qwerty.xyz@bar.ac.uk}{\texttt{qwerty.xyz@bar.edu}}
  \end{tabular}
  \caption{The caption of a table.}
  \label{table}
\end{table}


%%% Local Variables:
%%% mode: latexmk
%%% TeX-master: "../main"
%%% End:
