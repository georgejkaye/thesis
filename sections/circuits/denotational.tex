\chapter{Denotational semantics}

One can have potentially hours of fun making pretty pictures of circuits in
\(\scircsigma\).
Ultimately though, these circuits are still just \emph{syntax}; they have no
\emph{behaviour}, or more formally \emph{semantics}, assigned to them.
At its core, a semantics for digital circuits relates circuits which have `the
same behaviour given some interpretation'.
However, there is not just one way to construct said relation.
In this thesis we will examine three such ways: a \emph{denotational} semantics,
an \emph{operational} semantics and an \emph{algebraic} semantics.
Each approach comes with advantages and drawbacks; skillful use of all three
will lead to a powerful, fully compositional, perspective on sequential
circuits.

Although the construction of the semantic relation is different, each semantics
must relate the \emph{same} circuits; the behaviour of a circuit should not be
different depending on which lens we are viewing it through.
Formally, each form of semantics must be \emph{sound and complete} with respect
to each other; two sequential circuits \(
    \iltikzfig{strings/category/f}[box=f,colour=seq,dom=m,cod=n]
\) and \(
    \iltikzfig{strings/category/f}[box=g,colour=seq,dom=m,cod=n]
\) are related by one semantics if and only if they are related by the others.

First of all we will define the \emph{denotational semantics} of digital
circuits, which will act as the gold standard against which the other
semantics will be compared.
Denotational semantics is the notion of assigning meaning to structures using
values in some \emph{semantic domain}: a partially ordered set with some
extra structure.
The idea is an old one in computer science, going back to the work of Scott and
Strachey~\cite{scott1970outline,scott1971mathematical}.

\begin{example}
    Consider a language of mathematical expressions defined as
    follows:
    \[
        n \in N ::= \overline{0} \,|\, \overline{1} \,|\, \overline{2} \,|\,
        \cdots
        \qquad
        e \in E ::= add \, e \, e \,|\, mul \, e \, e \,|\,  n
    \]
    To define a denotational semantics for terms \(E\) in this language, we need
    to pick a \emph{semantic domain} for the denotations of terms to belong to.
    An obvious one here is the natural numbers \(\nat\); given a term
    \(e \in E\), we write \(\llbracket{e}\rrbracket\) for its denotation in
    \(\nat\).
    For each \(\overline{n} \in N\), \(\llbracket{\overline{n}}\rrbracket\) is
    the corresponding natural number, and the operations are interpreted as \(
        \llbracket{add \, e_1 \, e_2}\rrbracket
        = \llbracket{e_1}\rrbracket + \llbracket{e_2}\rrbracket
    \) and \(
        \llbracket{mul \, e_1 \, e_2}\rrbracket
        = \llbracket{e_1}\rrbracket \cdot \llbracket{e_2}\rrbracket
    \) respecitvely.
\end{example}

The above example illustrates quite nicely how a denotational semantics should
be \emph{compositional}; the denotations of a composite term should be
constructed by combining the denotations of its components.

\section{Denotational semantics of digital circuits}

With a simple example under our belt, we turn to our true goal: defining a
denotational semantics for digital circuits.
To make things easier for ourselves, we will examine combinational and
sequential circuits individually.

\begin{definition}[Lattice]
    A \emph{lattice} is a poset equipped with binary operations
    \(\ljoin\) (`join') and \(\lmeet\) (`meet') which distribute over each
    other.
\end{definition}

\begin{definition}
    Let \((A, \leq_A)\) and \((B, \leq_B)\) be partial orders.
    A function \(\morph{f}{A}{B}\) is monotone if, for every \(x, y \in A\),
    \(x \leq_A y\) if and only if \(f(x) \leq_B f(y)\).
\end{definition}

\begin{definition}[Interpretation]
    For a signature \(
        \signature = (
            \values, \bullet, \circuitsignaturegates, \circuitsignaturearity,
            \circuitsignaturecoarity
    )\), an \emph{interpretation} of \(\signature\) is a tuple \(
        \interpretation = (\ljoin, \lmeet, \gateinterpretation)
    \) where \((\values, \ljoin, \lmeet)\) is a lattice with \(\bullet\) as the
    infimum, and \(\gateinterpretation\) maps each
    \(p \in \circuitsignaturegates\) to a \(\bot\)-preserving monotone function
    \(
        \valuetuple{\circuitsignaturearity(p)}
        \to
        \valuetuple{\circuitsignaturearity(p)}
    \).
\end{definition}

\subsection{Combinational circuits}

Recall that, in the real world, combinational circuits are effectively
\emph{functions}; they always produce the same outputs given the same inputs.
It therefore makes sense that the denotational semantics of combinational
circuits to be functions.

However, not \emph{all} functions can be modelled by combinational circuits;
there





\subsection{Sequential circuits}