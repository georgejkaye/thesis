\chapter{Denotational semantics}

One can have potentially hours of fun making pretty pictures of circuits in
\(\scircsigma\).
Ultimately though, these circuits are still just \emph{syntax}; they have no
\emph{behaviour}, or more formally \emph{semantics}, assigned to them.
At its core, a semantics for digital circuits relates circuits which have `the
same behaviour given some interpretation'.
However, there is not just one way to construct said relation.
In this thesis we will examine three such ways: a \emph{denotational} semantics,
an \emph{operational} semantics and an \emph{algebraic} semantics.
Each approach comes with advantages and drawbacks; skillful use of all three
will lead to a powerful, fully compositional, perspective on sequential
circuits.

Although the construction of the semantic relation is different, each semantics
must relate the \emph{same} circuits; the behaviour of a circuit should not be
different depending on which lens we are viewing it through.
Formally, each form of semantics must be \emph{sound and complete} with respect
to each other; two sequential circuits \(
    \iltikzfig{strings/category/f}[box=f,colour=seq,dom=m,cod=n]
\) and \(
    \iltikzfig{strings/category/f}[box=g,colour=seq,dom=m,cod=n]
\) are related by one semantics if and only if they are related by the others.

First of all we will define the \emph{denotational semantics} of digital
circuits, which will act as the gold standard against which the other
semantics will be compared.
Denotational semantics is the notion of assigning meaning to structures using
values in some \emph{semantic domain}: a partially ordered set with some
extra structure.
The idea is an old one in computer science, going back to the work of Scott and
Strachey~\cite{scott1970outline,scott1971mathematical}.

\begin{example}
    Consider a language of mathematical expressions defined as
    follows:
    \[
        n \in N :\coloneqq \overline{0} \,|\, \overline{1} \,|\, \overline{2} \,|\,
        \cdots
        \qquad
        e \in E :\coloneqq add \, e \, e \,|\, mul \, e \, e \,|\,  n
    \]
    To define a denotational semantics for terms \(E\) in this language, we need
    to pick a \emph{semantic domain} for the denotations of terms to belong to.
    An obvious one here is the natural numbers \(\nat\); given a term
    \(e \in E\), we write \(\llbracket{e}\rrbracket\) for its denotation in
    \(\nat\).
    For each \(\overline{n} \in N\), \(\llbracket{\overline{n}}\rrbracket\) is
    the corresponding natural number, and the operations are interpreted as \(
        \llbracket{add \, e_1 \, e_2}\rrbracket
        = \llbracket{e_1}\rrbracket + \llbracket{e_2}\rrbracket
    \) and \(
        \llbracket{mul \, e_1 \, e_2}\rrbracket
        = \llbracket{e_1}\rrbracket \cdot \llbracket{e_2}\rrbracket
    \) respectively.
\end{example}

The above example illustrates quite nicely how a denotational semantics should
be \emph{compositional}; the denotations of a composite term should be
constructed by combining the denotations of its components.

\begin{remark}
    The content of this section is a refinement and expansion
    of~\cite[Section 3]{ghica2023compositional}.
\end{remark}

\section{Denotational semantics of digital circuits}\label{sec:circuits}

Now that we comprehend what exactly denotational semantics is, we turn to our
goal of defining a denotational semantics for digital circuits.
We will interpret digital circuits as a certain class of
\emph{stream functions}, functions that operate on infinite sequences of values.
This represents how the output of a circuit may not just operate on the current
input, but all of the previous ones as well.

\begin{remark}
    In \cite{mendler2012constructive}, the semantics of digital circuits with
    delays and cycles are presented using \emph{timed ternary simulation}, an
    algorithm to compute how a sequence of circuit outputs stabilises over time
    given the inputs and value of the current state.
    This differs from our approach as we assign each circuit a concrete stream
    function describing its behaviour, rather than having to solve a system of
    equations in terms of the nodes inside a circuit in order to determine its
    behaviour.
\end{remark}

\subsection{Interpreting circuit components}\label{sec:interpreting-components}

Before assigning stream functions to a given circuit in \(\scircsigma\), we will
first decide how to interpret the individual \emph{components} of a given
circuit signature.
This is crucial because a denotational semantics is defined
\emph{compositionally}; eventually we will need to refer to the interpretations
of particular components.

First we consider the interpretation of the \emph{values} that flow through the
wires in our circuits.
In the denotational semantics the set of values will need to have a bit more
structure, as it must be ordered by how much \emph{information} each value
carries.

\begin{definition}[Partially ordered set]
    A \emph{partially ordered set}, or \emph{poset} for short, is a set \(A\)
    equipped with a reflexive, antisymmetric, and transitive relation \(\leq\),
    i.e.\
    \begin{itemize}
        \item for all \(a \in A\), \(a \leq a\);
        \item for all \(a, b \in A\), if \(a \leq b\) and \(b \leq a\) then
              \(a = b\); and
        \item for all \(a, b, c \in A\), if \(a \leq b\) and \(b \leq c\) then
              \(a \leq c\).
    \end{itemize}
    A poset \((A, leq)\) is called a \emph{finite poset} if \(A\) is finite.
\end{definition}

\begin{definition}[Least and greatest elements]
    In a poset \((A, \leq)\), a \emph{least element} is an element \(b \in A\)
    such that for all elements \(y \in A\), \(b \leq y\).
    Similarly, a \emph{greatest element} is an element \(a \in A\)
    such that for all elements \(x \in A\), \(x \leq a\).
\end{definition}

\begin{example}
    The natural numbers \(\nat\) are a poset ordered in the usual way; they have
    a least element \(0\) but not a greatest element.
    However, any finite subset of the natural numbers \emph{does} have a
    maximal element.
\end{example}

Even in a finite set there is no reason that least and greatest elements should
be \emph{unique}.
However, in our context of digital circuits we would like this to be the case:
the unique least element will represent a complete \emph{lack} of information,
and the unique greatest element will represent \emph{every} piece of information
at once.
To enforce the uniqueness of these elements we must add even more structure to
a poset of values.

\begin{definition}[Lower and upper bounds]
    Given a poset \((A, \leq)\), a \emph{lower bound} of a subset
    \(B \subseteq A\) is an element \(x \in B\) such that for all \(b \in B\),
    \(x \leq b\).
    Similarly, an \emph{upper bound} of \(C \subseteq A\) is an element
    \(y \in C\) such that for all \(b \in B\), \(b \leq y\).
\end{definition}

\begin{definition}[Meet and join]
    Given a poset \((A, \leq)\), a lower bound \(x\) of \(B \subseteq A\) is
    called an \emph{meet} (or infimum, or greatest lower bound) if for all lower
    bounds \(b \in B\), \(b \leq x\).
    Similarly, an upper bound \(y\) of \(B\) is called a \emph{join} (or
    supremum, or least upper bound) if for all upper bounds \(c \in B\),
    \(y \leq c\).
\end{definition}

We usually draw the meet and the join operations as rotated versions of the
order operation.
For example, in the above definitions we have used \(\wedge\) and \(\vee\) for
the order \(\leq\); in subsequent sections we will use \(\lmeet\) and
\(\ljoin\) for the order \(\sqsubseteq\).
In general, the join and meet of a pair of elements in a poset need not exist.
We are interested in the structures in which they \emph{always} exist, which are
known as \emph{lattices}.

\begin{definition}[Lattice]
    A \emph{lattice} is a poset \((A, \leq)\) in which each pair of elements
    \(a,b \in A\) has a meet and join.
    A lattice is called a \emph{finite lattice} if \(A\) is finite, and
    \emph{bounded} if it has a minimum element and a maximum element.
\end{definition}

Much like how every non-empty finite poset has at least one maximimal and
minimal element, every non-empty finite lattice has a join and meet.

\begin{notation}
    For a poset \((A, \leq)\), we write \(\bigwedge A\) for the meet of
    all the elements in \(A\) (if it exists) and \(\bigvee A\) for the join of
    all the elements in \(A\) (again, if it exists).
\end{notation}

\begin{lemma}\label{cref:finite-bounded}
    A non-empty finite lattice \((A, \leq)\) is bounded.
\end{lemma}
\begin{proof}
    As each pair of elements in \(A\) has a meet, and as \(A\) is finite, we
    can define the element \(x\) as \(\bigwedge A\).
    This element is the greatest element in \(A\) by definition of the meet:
    \((a_0 \wedge a_1) \leq a_0\) and \((a_0 \wedge a_1) \leq a_1\),
    \(((a_0 \wedge a_1) \wedge a_2) \leq (a_0 \wedge a_1)\) and
    \(((a_0 \wedge a_1) \wedge a_2) \leq a_2\), and so on.
    The same proof holds in reverse for the join and the greatest element by
    using the join \(\bigvee A\).
\end{proof}

\begin{example}\label{ex:powerset-lattice}
    Let \(A = \{0,1,2\}\), and let \((\mcp A, \subseteq)\) be the poset defined
    as the powerset of \(A\) ordered by subset inclusion.
    This poset can be illustrated by the following \emph{Hasse diagram}, in
    which a line going up from \(a\) to \(b\) indicates that \(a \leq b\).

    \begin{center}
        \begin{tikzcd}
            & \{0,1,2\} & \\
            \{0,1\} \arrow[dash]{ur} &
            \{0,2\} \arrow[dash]{u} &
            \{1,2\} \arrow[dash]{ul} \\
            \{0\} \arrow[dash]{u} \arrow[dash]{ur} &
            \{1\} \arrow[dash]{ul} \arrow[dash]{ur} &
            \{2\} \arrow[dash]{ul} \arrow[dash]{u} \\
            & \{\} \arrow[dash]{ul} \arrow[dash]{u} \arrow[dash]{ur} &
        \end{tikzcd}
    \end{center}

    The diagram clearly illustrates the lattice structure on this poset: the
    join is union and the meet is intersection.
    Subsequently the greatest element \(\top\) is the set \(A\) and the
    least element \(\bot\) is the empty set \(\{\}\).
\end{example}

\begin{remark}
    If \(A\) is a lattice, then for any \(n \in \nat\), \(A^n\) is also a
    lattice by comparing elements pointwise.
    The \(\bot\) is then the word containing \(n\) copies of the \(\bot\)
    element in \(A\), and similarly for the \(\top\) element.

    Recall the set \(A \coloneqq \{0,1,2\}\) from
    \cref{ex:powerset-lattice} and the lattice structure on its powerset
    \(\mathcal{P}A\).
    This induces an ordering on \((\mathcal{P}A)^2\):
    \(\{0,1\}\{1\} \leq \{0,1,2\}\{1,2\}\) because \(\{0,1\} \leq \{0,1,2\}\)
    and \(\{1\} \leq \{1,2\}\), and conversely
    \(\{0,1\}\{1\} \not\leq \{0\}\{1,2\}\) because \(\{0,1\} \not\leq \{0\}\).
    The join and meet are computed by taking the join and meet of each
    component: \(
    \{0,1\}\{1\} \vee \{0,1,2\}\{1,2\} = \{0,1,2\}\{1,2\}
    \) and \(
    \{0,1\}\{1\} \vee \{0\}\{1,2\} = \{0,1\}\{1,2\}
    \).
\end{remark}

Values in a circuit signature are interpreted as elements of a finite
lattice; now the concepts of no information and all information are encoded
as the \(\bot\) and \(\top\) element.
Now the primitives in the signature must be interpreted.
Clearly they should be interpreted as functions between the values, but these
functions must respect the order on the value lattice; one should not be able to
lose information by performing a computation.

\begin{definition}
    Let \((A, \leq_A)\) and \((B, \leq_B)\) be partial orders.
    A function \(\morph{f}{A}{B}\) is \emph{monotone} if, for every \(x, y \in A\),
    \(x \leq_A y\) if and only if \(f(x) \leq_B f(y)\).
\end{definition}

Another condition on the primitives is that when all the inputs to a primitive
are \(\bot\), then it should produce \(\bot\); we cannot produce information
from nothing.

\begin{definition}
    Let \(A,B\) be finite lattices, where \(\bot_A\) is the least element of
    \(A\) and \(\bot_B\) is the least element of \(B\).
    A function \(\morph{f}{A}{B}\) is \emph{\(\bot\)-preserving} if
    \(f(\bot_A) = \bot_B\).
\end{definition}

Assigning interpretations to the combinational components of a circuit sets the
stage for the entire denotational semantics.

\begin{definition}[Interpretation]
    For a signature \(
    \signature = (
    \values, \bullet, \circuitsignaturegates, \circuitsignaturearity,
    \circuitsignaturecoarity
    )\), an \emph{interpretation} of
    \(\signature\) is a tuple \((\sqsubseteq, \gateinterpretation)\) where
    \((\values, \sqsubseteq)\) is a lattice with \(\bullet\) as the least
    element, and \(\gateinterpretation\) maps each
    \(p \in \circuitsignaturegates\) to a \(\bot\)-preserving monotone function
    \(
    \valuetuple{\circuitsignaturearity(p)}
    \to
    \valuetuple{\circuitsignaturecoarity(p)}
    \).
\end{definition}

\begin{example}\label{ex:belnap-interpretation}
    Recall the Belnap signature \(
    \belnapsignature = (
    \belnapvalues, \bot, \belnapgates, \belnaparity, \belnapcoarity
    )
    \) from \cref{ex:belnap-signature}.
    We assign a partial order \(\leq_\mathsf{B}\) to values in
    \(\belnapvalues\) as follows:

    \begin{center}
        \begin{tikzcd}
            & \top & \\
            \belnapfalse \arrow[dash]{ur} & & \belnaptrue \arrow[dash]{ul} \\
            & \bot \arrow[dash]{ul} \arrow[dash]{ur} &
        \end{tikzcd}
    \end{center}

    The gate interpretation \(\belnapgateinterpretation\) has action \(
    \andgate \mapsto \land, \orgate \mapsto \lor, \notgate \mapsto \neg
    \) where \(\land\), \(\lor\) and \(\neg\) are defined by the following
    truth tables~\cite{belnap1977useful}:

    \begin{center}
        \begin{tabular}{|c|cccc|}
            \hline
            \(\land\)        & \(\bot\)         & \(\belnapfalse\) & \(\belnaptrue\)  & \(\top\)         \\
            \hline
            \(\bot\)         & \(\bot\)         & \(\belnapfalse\) & \(\bot\)         & \(\belnapfalse\) \\
            \(\belnapfalse\) & \(\belnapfalse\) & \(\belnapfalse\) & \(\belnapfalse\) & \(\belnapfalse\) \\
            \(\belnaptrue\)  & \(\bot\)         & \(\belnapfalse\) & \(\belnaptrue\)  & \(\top\)         \\
            \(\top\)         & \(\belnapfalse\) & \(\belnapfalse\) & \(\top\)         & \(\top\)         \\
            \hline
        \end{tabular}
        \quad
        \begin{tabular}{|c|cccc|}
            \hline
            \(\lor\)         & \(\bot\)        & \(\belnapfalse\) & \(\belnaptrue\) & \(\top\)        \\
            \hline
            \(\bot\)         & \(\bot\)        & \(\bot\)         & \(\belnaptrue\) & \(\belnaptrue\) \\
            \(\belnapfalse\) & \(\bot\)        & \(\belnapfalse\) & \(\belnaptrue\) & \(\top\)        \\
            \(\belnaptrue\)  & \(\belnaptrue\) & \(\belnaptrue\)  & \(\belnaptrue\) & \(\belnaptrue\) \\
            \(\top\)         & \(\belnaptrue\) & \(\top\)         & \(\belnaptrue\) & \(\top\)        \\
            \hline
        \end{tabular}
        \quad
        \begin{tabular}{|c|c|}
            \hline
            \(\neg\)         &                  \\
            \hline
            \(\bot\)         & \(\bot\)         \\
            \(\belnaptrue\)  & \(\belnapfalse\) \\
            \(\belnapfalse\) & \(\belnaptrue\)  \\
            \(\top\)         & \(\top\)         \\
            \hline
        \end{tabular}
    \end{center}

    The Belnap interpretation is then \(
    (\leq_\mathsf{B}, \belnapgateinterpretation)
    \).
    An online tool for experimenting with the Belnap interpretation can be found
    at \url{https://belnap.georgejkaye.com}.
\end{example}
\subsection{Denotational semantics of combinational circuits}

The semantic domain for \emph{combinational} circuits is straightforward: each
circuit maps to a monotone function.

\begin{definition}
    Let \(\funci\) be the PROP in which the morphisms
    \(m \to n\) are the monotone \(\bot\)-preserving
    functions \(\valuetuple{m} \to \valuetuple{n}\).
\end{definition}

The natural way to map from a PROP of syntax into a PROP of semantics is to
use a functor.

\begin{definition}
    A \emph{PROP morphism} is a strict symmetric monoidal functor between two
    PROPs i.e.\ a functor that preserves the strict symmetric monoidal
    structure.
\end{definition}

\begin{definition}
    Let \(\morph{\circuittofunci}{\ccircsigma}{\funci}\) be the PROP morphism
    with action defined as%
    \vspace{-\abovedisplayskip}
    % \vspace{-\parskip}
    \begin{center}
        \begin{minipage}{0.32\textwidth}
            \centering
            \begin{align*}
                \circuittofunci[
                    \iltikzfig{strings/structure/comonoid/copy}[colour=comb]
                ]
                 & \coloneqq
                (v) \mapsto (v, v)
                \\
                \circuittofunci[
                    \iltikzfig{strings/structure/monoid/merge}[colour=comb]
                ]
                 & \coloneqq
                (v, w) \mapsto v \sqcup w
            \end{align*}
        \end{minipage}
        \quad
        \begin{minipage}{0.25\textwidth}
            \centering
            \begin{align*}
                \circuittofunci[
                    \iltikzfig{strings/structure/comonoid/discard}[colour=comb]
                ]
                 & \coloneqq
                (v) \mapsto ()
                \\
                \circuittofunci[
                    \iltikzfig{strings/structure/monoid/init}[colour=comb]
                ]
                 & \coloneqq
                () \mapsto \bot
            \end{align*}
        \end{minipage}
        \quad
        \begin{minipage}{0.25\textwidth}
            \centering
            \vspace{1.5em}
            \(\circuittofunci[
                \iltikzfig{circuits/components/gates/gate}[gate=p,dom=m,cod=n]
            ]
            \coloneqq
            \gateinterpretation[p]
            \)
        \end{minipage}
    \end{center}
\end{definition}

\begin{remark}
    One might wonder why the fork and the join have different semantics, as they
    would be physically realised by the same wiring.
    This is because digital circuits have a notion of \emph{static causality}:
    outputs can only connect to inputs.
    This is why the semantics of combinational circuits is \emph{functions} and
    not \emph{relations}.

    In real life one could force together two digital devices, but this might
    lead to undefined behaviour in the digital realm.
    This is reflected in the semantics by the use of the join; for example, in
    the Belnap interpretation if one tries to join together \(\belnaptrue\) and
    \(\belnapfalse\) then the overspecified \(\top\) value is produced.
\end{remark}

\subsection{Denotational semantics of sequential circuits}

As one might expect, sequential circuits are slightly trickier to deal with.
In a combinational circuit, the output only depends on the inputs at the current
cycle, but for sequential circuits inputs can affect outputs many cycles after
they occur.

We therefore have to reason with \emph{infinite sequences} of inputs rather than
individual values; these are known as \emph{streams}.

\begin{notation}
    Given a set \(A\), we denote the set of streams (infinite sequences) of
    \(A\) by \(\stream{A}\).
    As a stream can equivalently be viewed as a function \(\nat \to A\), we
    write \(\sigma(i) \in A\) for the \(i\)-th element of stream
    \(\sigma \in \stream{A}\).
    Individual streams are written as \(
    \sigma \in \stream{A}
    \coloneqq
    \sigma(0) \streamcons \sigma(1) \streamcons
    \sigma(2) \streamcons \cdots
    \).
\end{notation}

Streams can be viewed a bit like lists, in that they have a head component and
an (infinite) tail component.

\begin{definition}[Operations on streams]\label{def:stream-operations}
    Given a stream \(\sigma \in \stream{A}\), its \emph{initial value}
    \(\streaminit(\sigma) \in A\) is defined as \(\sigma \mapsto \sigma(0)\)
    and its \emph{stream derivative} \(\streamderv(\sigma) \in \stream{A}\) is
    defined as \(\sigma \mapsto (i \mapsto \sigma(i+1))\).
\end{definition}

\begin{notation}
    For a stream \(\sigma\) with initial value \(a\) and stream derivative
    \(\tau\) we write it as \(\sigma \coloneqq a \streamcons \tau\).
\end{notation}

Streams will serve as the inputs and outputs to circuits, so the denotations of
sequential circuits will be \emph{stream functions}, which consume and produce
streams.
Just like with functions, we cannot claim that all streams are the
denotations of sequential circuits.

\begin{definition}[Causal stream function~\cite{rutten2006algebraic}]
    A stream function \(\morph{f}{\stream{A}}{\stream{B}}\) is \emph{causal} if,
    for all \(i \in \nat\) and \(\sigma,\tau \in \stream{A}\) we have that
    \(\sigma(j) = \tau(j)\) for all \(j \leq i\), then
    \(f(\sigma)(i) = f(\tau)(i)\).
\end{definition}

Causality is a form of continuity; a causal stream function is a stream function
in which the \(i\)-th element of its output stream only depends on elements
\(0\) through \(i\) inclusive of the input stream; it cannot look into the
future.
A neat consequence of causality is that it enables us to lift the stream
operations of initial value and stream derivative to stream \emph{functions}.

\begin{definition}[Initial output~\cite{rutten2006algebraic}]
    For a causal stream function \(\morph{f}{\stream{A}}{\stream{B}}\) and
    \(a \in A\), the \emph{initial output of \(f\) on input \(a\)} is an element
    \(\initialoutput{f}{a} \in A\) defined as
    \(\initialoutput{f}{a} \coloneqq \streaminit(f(a \streamcons \sigma))\) for
    an arbitrary \(\sigma \in \stream{A}\).
\end{definition}

Since \(f\) is causal, the stream \(\sigma\) in the definition of initial
output truly is arbitrary; the \(\streaminit\) function only depends on the
first element of the stream.

\begin{definition}[Functional stream derivative~\cite{rutten2006algebraic}]
    For a stream function \(\morph{f}{\stream{A}}{\stream{B}}\) and
    \(a \in A\), the
    \emph{functional stream derivative of \(f\) on input \(a\)} is a stream
    function \(\streamderivative{f}{a}\) defined as \(
    \streamderivative{f}{a}
    \coloneqq
    \sigma \mapsto \streamderv(f(a \streamcons \sigma))
    \).
\end{definition}

The functional stream derivative \(\streamderivative{f}{a}\) is a new stream
function which acts as \(f\) would `had it seen \(a\) first'.

\begin{remark}
    One intuitive way to view stream functions is to think of them as the states
    of some Mealy machine; the initial output is the output given some input,
    and the functional stream derivative is the transition to a new state.
    As with most things in mathematics, this is no coincidence; there is a
    homomorphism from any Mealy machine to a stream function.
    We will exploit this fact in the next section.
\end{remark}

This leads us to the next property of denotations of sequential circuits.
Although they may have infinitely many inputs and outputs, circuits themselves
are built from a finite number of components.
This means they cannot specify infinite \emph{behaviour}.

\begin{notation}
    Given a finite word \(\listvar{a} \in \freemon{A}\), we abuse notation
    and write \(\streamderivative{f}{\listvar{a}}\) for the repeated
    application of the functional stream derivative for the elements of
    \(\listvar{a}\), i.e.\ \(
    \streamderivative{f}{\varepsilon} \coloneqq f
    \) and \(
    \streamderivative{f}{a \streamcons \listvar{b}} \coloneqq
    \streamderivative{(\streamderivative{f}{a})}{\listvar{b}}
    \).
\end{notation}

\begin{definition}
    Given a stream function \(\morph{f}{\stream{A}}{\stream{B}}\), we say it is
    \emph{finitely specified} if the set \(\{
    \streamderivative{f}{\listvar{a}} \,|\, \listvar{a} \in \freemon{A}
    \}\) is finite.
\end{definition}

As the components of our circuits are monotone and \(\bot\)-preserving, a
denotational semantics for circuits must also be monotone and
\(\bot\)-preserving.
This means we need to lift the order on values to an order on streams.

\begin{notation}
    For a poset \((A, \leq_A)\) and streams \(\sigma,\tau \in \stream{A}\), we
    say that \(\sigma \leq_{\stream{A}} \tau\) if \(\sigma(i) \leq_A \tau(i)\)
    for all \(i \in \nat\).
\end{notation}

For these three properties to be suitable as a denotational semantics for
sequential circuits, we must show that stream functions with these
properties form a category we can map into from \(\scircsigma\).
We will first show that these categories form a symmetric monoidal category, so
we need a suitable candidate for composition and tensor.
There are fairly obvious choices here: for the former we use regular function
composition and for the latter we use the Cartesian product.

\begin{lemma}\label{lem:causality-preserved}
    Causality is preserved by composition and Cartesian product.
\end{lemma}
\begin{proof}
    If the \(i\)-th element of two stream functions \(f\) and \(g\) only depends
    on the first \(i+1\) elements of the input, then so will their composition
    and product.
\end{proof}

\begin{lemma}\label{lem:finitely-specified-preserved}
    Finite specification is preserved by composition and Cartesian
    product.
\end{lemma}
\begin{proof}
    For both the composition and product of two stream functions \(f\) and
    \(g\), the largest the set of stream derivatives could be is the product of
    stream derivatives of \(f\) and \(g\), so this will also be finite.
\end{proof}

\begin{lemma}\label{lem:monotonicity-preserved}
    \(\bot\)-preserving monotonicity is preserved by composition and Cartesian
    product.
\end{lemma}
\begin{proof}
    The composition and product of any monotone function is monotone, and must
    preserve the \(\bot\) element.
\end{proof}

As the categorical operations preserve the desired properties, these stream
functions form a PROP.

\begin{proposition}
    There is a PROP \(\streami\) in which the morphisms \(m \to n\) are the
    causal, finitely specified and \(\bot\)-preserving monotone stream
    functions \(\valuetuplestream{m} \to \valuetuplestream{n}\).
\end{proposition}
\begin{proof}
    Identity is the identity function, the symmetry swaps streams, composition
    is composition of functions, and tensor product on morphisms
    \(\morph{f}{\valuetuplestream{m}}{\valuetuplestream{n}}\) and
    \(\morph{g}{\valuetuplestream{p}}{\valuetuplestream{q}}\) is the Cartesian
    product of functions composed with the components of the isomorphism
    \(\valuetuplestream{m} \times \valuetuplestream{n}
    \cong \valuetuplestream{m+n}\).

    As these constructs satisfy the categorical axioms, and as function
    composition and Cartesian product preserve causality
    (\cref{lem:causality-preserved}),
    finite specification (\cref{lem:finitely-specified-preserved}),
    and monotonicity (\cref{lem:monotonicity-preserved}), this data defines a
    symmetric monoidal category.
\end{proof}

Modelling sequential circuits as stream functions deals with temporal
aspects, but what about feedback?
As the assignment of denotations needs to be compositional, we need
to map the trace on \(\scircsigma\) to a trace on \(\streami\).
A usual candidate for the trace when considering partially ordered settings is
the \emph{least fixed point}.

\begin{definition}[Least fixed point]
    For a poset \((A, \leq)\) and function \(\morph{f}{A}{A}\), the least
    fixed point of \(f\) is a value \(\mu_f\) such that \(f(\mu_f) = f\) and,
    for all values \(v\) such that \(f(v) = v\), \(\mu_f \leq v\).
\end{definition}

Least fixed points are ubiquitious in program semantics, where they are often
used to model \emph{recursion}; since feedback is a form of recursion it seems
apt that we should also follow this route.
As fixed points are so important, they are the subject of many theorems; one
that will come in very useful for us is the \emph{Kleene fixed-point theorem},
which is concerned with a special class of monotone functions.

\begin{definition}[Directed subset]
    For a poset \((A, \leq)\), a non-empty subset \(B \subseteq A\) is called a
    \emph{directed subset} if every pair of elements in \(B\) has an upper bound
    in \(B\).
    If this set has a join \(\bigvee B\) then this element is called a
    \emph{directed join}.
\end{definition}

\begin{notation}[Image]
    For a function \(\morph{f}{A}{B}\) and subset \(C \subseteq A\), we write
    \(f[C] \subseteq B\) for the \emph{image} of \(C\) under \(f\).
\end{notation}

\begin{definition}[Scott continuity]
    Given two posets \((A, \leq_A)\) and \((B, \leq_B)\), a function
    \(\morph{f}{A}{B}\) is \emph{Scott-continuous} if for every directed
    subset \(C \subseteq A\) it holds that \(f(\bigvee_A C) = \bigvee_B(f[C])\)
    i.e.\ \(f\) preserves directed joins.
\end{definition}

\begin{theorem}[Kleene fixed-point theorem~\cite{tarski1955latticetheoretical}]
    Let \((A, \leq)\) be a poset in which each of its directed subsets has a
    join, and let \(\morph{f}{L}{L}\) be a Scott-continuous function.
    Then \(f\) has a least fixed point, defined as \(
    \bot \vee f(\bot) \vee f(f(\bot)) \vee \cdots
    \).
\end{theorem}

\begin{remark}
    The Kleene fixed-point theorem was not actually proved by Kleene, but is
    only named after him!
    The result is often instead attributed to Tarski.
\end{remark}

As we have so far only considered monotone functions, it is useful to get some
intuition for what Scott-continuity brings to the table.

\begin{example}\label{ex:directed-subsets}
    An example of a directed subset of \(\belnapvalues^\omega\) is the set \(T\)
    defined as \(\{\belnaptrue^n\bot \,|\, n \in \nat\}\); the join of this set
    is \(\belnaptrue^\omega\). One Scott-continuous function \(
    \belnapvalues^\omega \to \belnapvalues^\omega
    \) is defined as \(f(\sigma)(0) = \bot\) and
    \(f(\sigma)(i+1) = f(sigma)(i)\); this is Scott-continuous because finding
    the join of a set and then prepending it with \(\bot\) is the same as
    prepending \(\bot\) to each stream in the set and then finding their join.

    An example of a stream function that is monotone but \emph{not}
    Scott-continuous is the function defined as \(
    g(\belnaptrue^n\bot^\omega) \coloneqq \belnapfalse^\omega (n )
    \) and \(g(\belnaptrue^\omega) \coloneqq \top^\omega\) (the other inputs
    do not matter for this example).
    We can show this is not Scott-continuous by considering the set \(T\) above,
    as \(f(\bigsqcup T) = f(\belnaptrue^\omega) = \top^\omega \neq
    \belnapfalse^\omega = \bigsqcup f[T]
    \).
    However, note that this function is \emph{not} causal: as
    \(\belnaptrue^\omega\) has the same prefix as every element in \(T\),
    \(f(\belnaptrue^\omega)\) must also share output prefixes.
\end{example}

So far we have not explicitly enforced Scott-continuity on stream functions; it
turns that it is implied by causality and monotonicity.

\begin{proposition}\label{prop:monotone-causal-scott}
    Let \((A, \leq_A)\) and \((B, \leq_B)\) be finite lattices, and let
    \((\stream{A}, \leq_{\stream{A}})\) and \((\stream{B}, \leq_{\stream{B}})\)
    be the induced lattices on streams.
    Then a causal and monotone function \(\stream{A} \to \stream{B}\) must also
    be Scott-continuous.
\end{proposition}
\begin{proof}
    Consider a directed subset \(D \subseteq \stream{A}\); we need to show that
    for an arbitrary causal, monotone and finitely specified function \(f\) we
    have that \(f\left(\bigvee D\right) = \bigvee f[D]\).

    First consider the case when there is a greatest element in \(D\).
    In this case, \(\bigvee D\) must be the greatest element, and as such
    \(\bigvee D \in D\).
    As \(f\) is monotone then \(f(\bigvee D)\) must be the greatest element in
    \(f[D]\); subsequently, \(f\left(\bigvee D\right) = \bigvee f[D]\).

    Now consider the case where there is no greatest element in \(D\) and
    subsequently \(\bigvee D \not\in D\); if there is no greatest element,
    \(D\) must be infinite.
    Even though it is infinite, \(D\) is a directed subset so each pair of
    elements must have an upper bound, and as \(\leq_{\stream{A}}\) is computed
    pointwise by using \(\leq_A\) we can consider what the upper bounds are
    pointwise too.
    Because \(A\) is finite, there cannot be an infinite chain of upper bounds
    for each element \(i\); there must exist an element \(a_i \in A\) such that
    \(D\) contains infinitely many streams \(\sigma\) such that
    \(\sigma(i) = a_i\).
    This means that \(\left(\bigvee D\right)(i) = a\), so every prefix of
    \(\bigvee D\) must exist as a prefix of a stream in \(D\).
    As \(f\) is causal, for each prefix
    \(f\left(\bigvee D\right)\) there must also exist a \(d \in D\) such that
    \(f(d)\) has that prefix, and as such
    \(\bigvee f[D] = f\left(\bigvee D\right)\).
\end{proof}

This means we can use the Kleene fixed point theorem as a tool to show that the
least fixed point is a trace on \(\streami\).

\begin{lemma}\label{lem:lfp-stream-function}
    Given a morphism \(
    \morph{f}{\valuetuplestream{x+m}}{\valuetuplestream{x+n}}
    \in \streami
    \), and stream \(\sigma \in \valuetuplestream{m}\), the function \(
    \tau \mapsto \proj{0}\mleft(f(\tau,\sigma)\mright)
    \) has a least fixed point.
\end{lemma}
\begin{proof}
    The function \(\tau \mapsto \proj{0}\mleft(f(\tau,\sigma)\mright)\) is
    causal and monotone because \(f\) and the projection function are
    causal and monotone, so it is Scott-continuous by
    \cref{prop:monotone-causal-scott}.
    By the Kleene fixed point theorem, this function has a least fixed point,
    defined as \(
    \proj{0}\mleft(f(\bot^\omega, \sigma)\mright) \ljoin
    \proj{0}\mleft(f(\proj{0}\mleft(f(\bot^\omega, \sigma)\mright), \sigma)\mright) \ljoin
    \cdots
    \).
\end{proof}

We must show that this notion of least fixed point is a trace on \(\streami\).
The first step is to show that taking the least fixed point of a stream function
in \(\streami\) produces another causal, finitely specified,
\(\bot\)-preserving, and monotone stream function.
The original proof idea for this is due to David Sprunger, and relies on an
ordering on stream functions themselves.

\begin{definition}\label{def:state-order}
    Let \(A\) and \(B\) be posets and let
    \(\morph{f, g}{\stream{A}}{\stream{B}}\) be stream functions.
    We say \(f \stateorder g\) if \(f(\sigma) \leq_{\stream{B}} g(\sigma)\)
    for all \(\sigma \in \stream{A}\).
\end{definition}

\begin{theorem}\label{thm:trace-well-defined}
    For a function \(\morph{f}{\valuetuplestream{x+m}}{\valuetuplestream{x+n}}\),
    let \(\mu_f(\sigma)\) be the least fixed point of the function \(
    \tau \mapsto \proj{0}\mleft(f(\tau,\sigma)\mright)
    \).
    Then, the stream function \(
    \sigma \mapsto \proj{1}\mleft(f(\mu_f(\sigma), \sigma)\mright)
    \) is in \(\streami\).
\end{theorem}
\begin{proof}
    To show this, we need to prove that
    \(\sigma \mapsto \proj{1}\mleft(f(\mu_f(\sigma)\sigma)\mright)\) is in
    \(\streami\): it is causal, finitely specified, \(\bot\)-preserving and
    monotone.

    Since \(
    \morph{f}{\valuetuplestream{x+m}}{\valuetuplestream{x+n}}
    \) is a morphism of \(\streami\), it has finitely many stream derivatives.
    For each stream derivative \(\streamderivative{f}{\,\listvar{w}}\), let the
    function \(
    \morph{
        \widehat{\streamderivative{f}{\listvar{w}}}
    }{
        \valuetuplestream{x+m}
    }{
        \valuetuplestream{x}
    }
    \) be defined as \(
    \tau\sigma
    \mapsto
    \proj{0}(\streamderivative{f}{\listvar{w}}(\tau\sigma))
    \).
    Note that each of these functions are causal, \(\bot\)-preserving, and
    monotone, because they are constructed from pieces that are causal
    \(\bot\)-preserving and monotone.

    In particular, \(\mu_f(\sigma)\) is the least fixed point of
    \(\widehat{f_\varepsilon}\left((-)\sigma\right)\).
    Using the Kleene fixed point theorem, the least fixed point of
    \(\widehat{f}((-)\sigma)\) can be obtained by composing
    \(\widehat{f}((-)\sigma)\) repeatedly with itself.
    This means that \(
    \mu_f(\sigma)
    =
    \bigsqcup_{k \in \nat} \widehat{f^k}(\bot^\omega,\sigma)
    \) where \(\widehat{f^k}\) is the \(k\)-fold composition of \(f(-,\sigma)\)
    with itself, i.e.\ \(\widehat{f^0}(\tau\sigma) = \tau\) and \(
    \widehat{f^{k+1}}(\tau\sigma)
    =
    \widehat{f}\left(\left(\widehat{f^{k}}(\sigma, \tau)\right)\sigma\right)
    \).
    That the mapping \(\mu_f\) is causal and monotone is
    straightforward: each of the functions in the join is causal and monotone,
    and join preserves these properties.
    It remains to show that this mapping has finitely many stream derivatives.

    When equipped with \(\stateorder\), the set of functions
    \(\valuetuplestream{x+m} \to \valuetuplestream{x}\)
    is a poset, of which
    \(\{\widehat{f_w} \,|\, w \in (\valuetuple{x+m})^\star\}\)
    is a finite subset.
    Restricting the ordering \(\stateorder\) to this set yields a finite poset.
    Since this poset is finite, the set of strictly increasing sequences in this
    poset is also finite.
    We will now demonstrate a relationship between these sequences and stream
    derivatives of \(\mu_f\).

    Suppose \(
    S = \widehat{f_{\,\listvar{w_0}}} \prec \widehat{f_{\,\listvar{w_1}}} \prec
    \cdots \prec \widehat{f_{\,\listvar{w_{\ell-1}}}}
    \) is a strictly increasing sequence of length \(\ell\) in the set of stream
    functions \(
    \{\widehat{f_w} \,|\, w \in (\valuetuple{x+m})^\star\}
    \).
    We define a function \(
    \morph{g_S}{
        \valuetuplestream{m}
    }{
        \valuetuplestream{x}
    }
    \) as \(
    (\sigma) \mapsto \bigsqcup_{k \in \nat} g_k(\sigma)
    \) where \[
        g_k(\sigma) =
        \begin{cases}
            \bot^\omega                                                              & \text{ if } k = 0              \\
            \widehat{f_{\,\listvar{w_k}}}(\left(g_{k-1}(\sigma)\right)\sigma)        & \text{ if } 1 \leq k \leq \ell \\
            \widehat{f_{\,\listvar{w_{\ell-1}}}}(\left(g_{k-1}(\sigma)\right)\sigma) & \text{ if } \ell < k
        \end{cases}.
    \]
    Let the set \(G \coloneqq \{
    g_S \,|\, S \text{ is a strictly increasing sequence}
    \}\).
    When \(S\) is set to the one-item sequence \(\widehat{f}\), \(g_S\) is
    \(\mu_f\), so \(\mu_f \in G\).
    As \(G\) is finite, this means that if \(G\) is closed under stream
    derivative, \(\mu_f\) has finitely many stream derivatives.
    Any element of \(G\) is either \(\bot^\omega\) or has the form \(
    \sigma
    \mapsto
    \widehat{\streamderivative{f}{\,\listvar{w}}}(g_k(\sigma)\sigma)
    \) for some \(\sigma \in \valuetuplestream{m}\) and
    \(k > 0\).
    As \(\bot^\omega\) is its own stream derivative, we need to show that
    applying stream derivative to the latter produces another element of \(G\).
    \begin{align*}
        \streamderivative{\sigma \mapsto \left(\widehat{\streamderivative{f}{\, \listvar{w}}}(\left(g_{k-1}(\sigma)\right)\sigma)\right)}{ab}
         & = \sigma \mapsto \streamderv\left(\widehat{\streamderivative{f}{\, \listvar{w}}}(ab \streamcons \left(g_{k-1}(\sigma)\right)\sigma)\right)              \\
         & = \sigma \mapsto \streamderv\left(\proj{0}\mleft(\streamderivative{f}{\, \listvar{w}}(ab \streamcons \left(g_{k-1}(\sigma)\right)\sigma)\mright)\right) \\
         & = \sigma \mapsto \proj{0}\mleft(\streamderv\left(\streamderivative{f}{\, \listvar{w}}(ab \streamcons \left(g_{k-1}(\sigma)\right)\sigma)\right)\mright) \\
         & = \sigma \mapsto \proj{0}\mleft(\streamderivative{\left(\streamderivative{f}{\, \listvar{w}}(\left(g_{k-1}(\sigma)\right)\sigma)\right)}{ab}\mright)    \\
         & = \sigma \mapsto \proj{0}\mleft(\streamderivative{f}{\, ab \streamcons \listvar{w}}(\left(g_{k-1}(\sigma)\right)\sigma)\mright)                         \\
         & = \sigma \mapsto \widehat{\streamderivative{f}{\, ab \streamcons \listvar{w}}}(\left(g_{k-1}(\sigma)\right)\sigma)
    \end{align*}
    As \(\proj{0}\mleft(\streamderivative{f}{ab \streamcons \listvar{w}}\mright)\)
    is in \(G\), the latter  is closed under stream derivative.
    Subsequently, \(\mu_f\) has finitely many stream derivatives.

    This means that all the components of
    \(\sigma \mapsto \proj{1}(f(\mu_f(\sigma)\sigma))\) are causal, monotone and
    finitely specified, and as these properties are preserved by composition,
    the composite must also have them, so
    \(\sigma \mapsto \proj{1}(f(\mu_f(\sigma)\sigma))\) is in \(\streami\).
\end{proof}

Even if \(\streami\) is closed under least fixed point, this does not mean that
it is a valid trace.
To verify this we must establish that the categorical axioms of the trace hold.

\begin{theorem}
    A trace on \(\streami\) is given for a function \(
    \morph{f}{\valuetuplestream{x+m}}{\valuetuplestream{x+n}}
    \) by the stream function \(
    \sigma \mapsto \proj{1}(f(\mu_f(\sigma), \sigma))
    \), where \(\mu_f(\sigma)\) is the least fixed point of the function \(
    \tau \mapsto \proj{0}\mleft(f(\tau,\sigma)\mright)
    \) for fixed \(\sigma\).
\end{theorem}
\begin{proof}
    By \cref{thm:trace-well-defined}, \(\streami\) is closed under taking the
    least fixed point, so we just need to show that the axioms of STMCs hold.
    Most of these follow in a straightforward way; the only interesting one is
    the sliding axiom.
    We need to show that for stream functions \(
    \morph{f}{\valuetuplestream{x+m}}{\valuetuplestream{y+n}}
    \) and \(
    \morph{g}{\valuetuplestream{y}}{\valuetuplestream{x}}
    \), we have that \(
    \trace{y}{(\tau, \sigma) \mapsto f(g(\tau), \sigma)}
    =
    \trace{x}{
        (\tau, \sigma)
        \mapsto
        g(\proj{0}\mleft(f(\tau, \sigma)\mright),
        \proj{1}\mleft(f(\tau, \sigma)\mright))
    }
    \).

    Let \(l \coloneqq (\tau, \sigma) \mapsto f(g(\tau), \sigma)\) and
    \(r \coloneqq (\tau, \sigma)
    \mapsto
    g(\proj{0}\mleft(f(\tau, \sigma)\mright),
    \proj{1}\mleft(f(\tau, \sigma)\mright))\); we must apply the candidate
    trace construction to both of these and check they are equal.
    For \(l\), the least fixed point of \(
    \tau \mapsto \proj{0}\mleft(f(g(\tau), \sigma)\mright)
    \) is \[
        \mu_l(\sigma) =
        \proj{0}\mleft(f(g(\bot^\omega), \sigma)\mright) \ljoin
        \proj{0}\mleft(f(g(\proj{0}\mleft(f(g(\bot^\omega), \sigma))\mright), \sigma)\mright) \ljoin
        \cdots.\]
    Plugging this into the candidate trace construction we have that \begin{align*}
         &
        \sigma \mapsto \proj{1}\mleft(f(g(\mu_l^l(\sigma)), \sigma)\mright)
        \\
         & \qquad=
        \sigma \mapsto \proj{1}\mleft(f(g(\proj{0}\mleft(f(g(\dots f(g(\proj{0}\mleft(f(g(\bot^\omega), \sigma)\mright)), \sigma)))\mright)), \sigma)\mright)
    \end{align*}
    For the right-hand side, the least fixed point of \(
    \tau \mapsto g(\proj{0}\mleft(f(\tau, \sigma)\mright))
    \) is \[
        \mu_r(\sigma) =
        g(\proj{0}\mleft(f(\bot^\omega, \sigma)\mright)) \ljoin
        g(\proj{0}\mleft(f(g(\proj{0}\mleft(f(g(\bot^\omega), \sigma)\mright)), \sigma)\mright)) \ljoin
        \cdots
    \]
    When plugged into the candidate trace construction this produces \begin{align*}
         & \sigma \mapsto \proj{1}\mleft(g(\proj{0}(f(\mu_r(\sigma), \sigma))), \proj{1}(f(\mu_r(\sigma), \sigma))\mright)
        \\
         & \qquad =
        \sigma \mapsto \proj{1}(f(\mu_\sigma^r, \sigma))
        \\
         & \qquad=
        \sigma \mapsto \proj{1}(f(
        g(\proj{0}\mleft(f(g(\dots f(g(\proj{0}\mleft(f(g(\bot^\omega), \sigma)\mright)), \sigma)\mright)), \sigma))))
        \\
         & \qquad=
        \sigma \mapsto \proj{1}(f(
        g(\proj{0}\mleft(f(g(\dots f(g(\proj{0}\mleft(f(\bot^\omega, \sigma)\mright)), \sigma)\mright)), \sigma))))
    \end{align*}
    Both the left-hand and the right-hand sides of the sliding equation are
    equal, so the construction is indeed a trace.
\end{proof}

We now have two traced PROPs: a \emph{syntactic} PROP of sequential circuit
terms \(\scircsigma\) and a \emph{semantic} PROP of causal, finitely
specified, monotone stream functions \(\streami\).
It would be straightforward to now define a map from circuits to these stream
functions; indeed, this is the strategy used in~\cite{ghica2024fully}.
Instead, we will first examine another structure with close links to both
circuits and stream functions; that of \emph{Mealy machines}.
The structure of Mealy machines will come in useful when considering the
\emph{completeness} of the denotational semantics.
\section{Monotone Mealy machines}

It is not immediately obvious how to translate back from stream functions in
\(\streami\) to circuits in \(\scircsigma\).
Even though these stream functions have finitely many stream derivatives, how
does one encapsulate this behaviour into a circuit made from finitely many
components while taking into account inputs and state?
Fortunately, we have a secret weapon; the old chestnut that is the
\emph{Mealy machine}~\cite{mealy1955method}.
Working tirelessly in industry as a way of specifying the behaviour of a
circuit without restricting to particular circuitry, Mealy machines also have
a very useful \emph{coalgebraic} viewpoint which we will wield in order to
build a bridge from circuits into stream functions: there is a unique
homomorphism from a Mealy machine to a causal, finitely specified stream
function.

We will do this by assembling a special class of Mealy machines which we dub
\emph{monotone Mealy machines} into another traced PROP.
By constructing PROP morphisms between these two categories that preserve
behaviour of stream functions, we establish that we can work with either
circuit representation.

\begin{definition}[Mealy machine~\cite{mealy1955method}]\label{def:mealy}
    Let \(A\) and \(B\) be finite sets.
    A (finite) \((A,B)\)-\emph{Mealy machine} is a tuple \((S, f)\) where
    \(S\) is a finite set called the \emph{state space},
    \(\morph{f}{S}{(S \times B)^A}\) is the \emph{Mealy function}.
\end{definition}

An \((M,N\)-Mealy machine is parameterised over a set \(A\) of \emph{inputs} and
a set \(B\) of \emph{outputs}.
Each Mealy machine then has a set \(S\) of \emph{states}; the Mealy function
takes in a currrent state and input \((s, a)\) and produces a pair
\(\langle{s^\prime, b}\rangle\) of a transition and output.

\begin{notation}
    We will use the shorthand \(
    \mealyfunctiontransition{f} \coloneqq (s, a) \mapsto \proj{0}(f(s)(a))
\) and \(
    \mealyfunctionoutput{f} \coloneqq (s, a) \mapsto \proj{1}(f(s)(a))
\) for the transition and output component of the Mealy function respectively.
\end{notation}

\begin{example}
    \todo[inline]{Example!}
\end{example}

\subsection{The coalgebraic perspective}

The definition of Mealy machine above is timeless and forms the basis for most
of modern electronics.
Of course, the natural question for the categorist to ask is
\emph{can we make it more categorical?}
And as is often the case, we can, using the notion of a \emph{coalgebra}.

\begin{definition}[Coalgebra]
    Let \(\mcc\) and \(\mcd\) be categories, and let \(\morph{F}{\mcc}{\mcc}\)
    be an endofunctor.
    A \emph{coalgebra} for \(F\), or \(F\)-coalgebra, is an object
    \(A \in \mcc\) along with a morphism \(\morph{\alpha}{A}{FA} \in \mcc\),
    usually written \((A,\alpha)\).
\end{definition}

The light bulb should already have lit up: a Mealy machine is a pair of a set
and a function, so this is clearly a coalgebra in \(\set\).

\begin{definition}
    An \emph{\((A,B)\)-Mealy coalgebra} is a coalgebra of the functor
    \(\morph{Y}{\set}{\set}\) defined as \(S \mapsto (S \times B)^A\).
\end{definition}

\begin{example}
    In \cite{bonsangue2008coalgebraic}, the notation \(
        f(s)(a) = \langle{\initialoutput{s}{a},\streamderivative{s}{a}}\rangle
    \) is used to describe the Mealy function, which coincides with the notation
    used for stream functions; this is not a coincidence!

    Given sets \(A\) and \(B\), let \(\Gamma\) be the set of causal stream
    functions \(\stream{A} \to \stream{B}\), and let
    \(\morph{\nu}{\Gamma}{\Gamma \times B}^A\) be the function defined as \(
        (f, a) \mapsto \langle{\initialoutput{s}{a},\streamderivative{s}{a}}\rangle
    \).
    Then \((\Gamma,\nu)\) is a \((A,B)\)-Mealy coalgebra.
\end{example}

The above example is incredibly important, as it lays the groundwork we will
use extensively to establish connections between circuits, stream functions and
Mealy machines.
If we inspect it a little closer, we find that stream functions are even more
special than just being `a' \((A,B)\)-Mealy coalgebra.

\begin{definition}[Mealy homomorphism]
    For sets \(A\) and \(B\), an \emph{\(A,B\)-Mealy homomorphism} between two
    \((A,B)\)-Mealy coalgebra \((S,f)\) and \((T,g)\) is a function
    \(\morph{h}{S}{T}\) preserving transitions and
    outputs, i.e.\ for a Mealy function \(f\), a Mealy homomorphism \(h\)
    satisfies \(
        h\left(\mealyfunctiontransition{f}(s, a)\right)
        =
        \mealyfunctiontransition{f}(h(s), a)
    \) and \(
        h\left(\mealyfunctionoutput{f}(s, a)\right)
        =
        \mealyfunctionoutput{f}(s, a)
    \).
\end{definition}

The \emph{final} \((M,N)\)-Mealy coalgebra is a \((M,N)\)-Mealy coalgebra to
which every other \((M,N)\)-Mealy coalgebra has a unique homomorphism to.

\begin{remark}
    The coalgebra crowd tend to prefer the use of the word \emph{final} over
    \emph{terminal}; one could quite reasonably call the final coalgebra a
    terminal coalgebra and everything would be alright, but you might get a few
    funny looks.
\end{remark}

\begin{proposition}[\cite{rutten2006algebraic}, Prop. 2.2]
    \label{prop:final-coalgebra}
    For every \((A,B)\)-Mealy coalgebra \((S,f)\), there exists a
    unique \((A,B)\)-Mealy homomorphism \(\morph{{!}}{(S,f)}{(\Gamma,\nu)}\).
\end{proposition}
\begin{proof}
    An \((A,B)\)-Mealy homomorphism \(\morph{g}{(S, f)}{(\Gamma, \nu)}\) is a
    function \(S \to \Gamma\), so for a state \(s_0 \in S\), \(g(s)\) will be a
    stream function \(\stream{A} \to \stream{B}\).
    Let \(\sigma \in \stream{A}\) be an input stream; there is a (unique) series
    of transitions \[
        s_0
        \mealyarrow{\sigma(0)}{b_0}
        s_1
        \mealyarrow{\sigma(1)}{b_1}
        s_2
        \mealyarrow{\sigma(2)}{b_2}
        s_3
        \mealyarrow{\sigma(3)}{b_3}
        \cdots
    \]
    The stream function \(!(s)\) is defined for input \(\sigma\) and
    index \(i \in \nat\) as \(!(s)(\sigma)(i) \coloneqq b_i\).
\end{proof}

For a Mealy coalgebra \((S, f)\) and a start state \(s_0\),
\(!(s)(\sigma)\) maps to the stream of outputs that \((S, f)\) would produce
by applying \(f\) to each element of \(\sigma\), starting from \(s_0\).

When viewed through this perspective, stream functions are nothing more than the
states in another Mealy machine.
However, this is not just any Mealy machine, it is the \emph{minimal} Mealy
machine with the same behaviour as \((S, f)\).

\subsection{Circuits as Mealy machines}

The fact that causal stream functions are the final Mealy coalgebra is the
ingredient that will help us build a bridge between circuits in \(\scircsigma\)
and stream functions in \(\streami\).
The crucial step is to now assemble Mealy machines into another traced PROP
and define PROP morphisms between all three categories in play.

However, not all Mealy machines defined thus far correspond to circuits in
\(\scircsigma\).
As with stream functions, we must refine our notion of Mealy machine in
order to find those that do.
There is an easy check for this: a Mealy machine \((S, f)\) with start state
\(s_0\) specifies the behaviour of a circuit in \(\scircsigma\) if the
stream function \(!(s_0)\) is in \(\streami\).

As established, all stream functions in the image of \(!\) are causal, and since
we only work with \emph{finite} Mealy machines we can also conclude the
following:

\begin{lemma}
    Given a Mealy machine \((S, f)\) and start state \(s_0 \in S\), \(!(s_0)\)
    is finitely specified.
\end{lemma}
\begin{proof}
    \(S\) is finite, and \(\mealytostream\) must preserve transitions.
\end{proof}

Monotonicity is a little less straightforward.
One could set the input, output and state sets to all be posets, and use the
usual definition of monotonicity that way.
However, this overcomplicates things: since each Mealy coalgebra has a
corresponding stream function, we can appeal to this rather than having to
manually order the state set.

First an ordering on stream \emph{functions} is required.

\begin{definition}
    Let \(A\) and \(B\) be posets.
    For stream functions \(\morph{f,g}{\stream{A}}{\stream{B}}\), we say that
    \(f \stateorder g\) if, for all \(\sigma \in \stream{A}\),
    \(f(\sigma) \leq_{\stream{B}}\).
\end{definition}

\begin{definition}[State order]
    Let \(A\) and \(B\) posets.
    For an \((A,B)\)-Mealy machine \((S, f)\) and states \(s,s^\prime \in S\)
    we say that \(s \stateorder s^\prime\) if \(!(s) \stateorder !(s^\prime)\).
\end{definition}

With an ordering on the states of a Mealy machine, monotonicity quickly follows.

\begin{definition}[Monotone Mealy machine]
    Let \(A\) and \(B\) be posets; an \((A,B)\)-Mealy machine is called a
    \emph{monotone} Mealy machine if \(f\) is monotone with respect to the
    appropriate orders.
\end{definition}

This establishes the properties of Mealy machines required to model circuits.
To map to Mealy machines from circuits we need to assemble them into another
PROP, in which the morphisms \(\listvar{m} \to \listvar{n}\) are
\((\valuetuple{m},\valuetuple{n})\)-Mealy machines.
There is one nuance to consider: circuits in \(\scircsigma\) have a `hardcoded'
initial state in the form of the value generators, whereas the Mealy machines
as presented do not have a designated start state per se.
This means that as well as a Mealy machine, each morphism will need to be
equipped with such a start state.

All that remains to define is the composition of Mealy machines, which is
standard.

\begin{definition}[Cascade product~\cite{ginzburg2014algebraic}]
    Given an \((A,B)\)-Mealy machine \((S,f)\) with start state \(s_0\) and
    an \((B,C)\)-Mealy machine \((T,g)\) with state state \((t_0)\), their
    \emph{cascade product} is an \((A,C)\)-Mealy machine defined as \[
        (S \times T, ((s, t), a) \mapsto ((
            \mealyfunctiontransition{f}(s,a),
            \mealyfunctiontransition{g}(t, \mealyfunctionoutput{f}(s, a))
        ),
            \mealyfunctionoutput{g}(t, \mealyfunctionoutput{f}(s, a))
        )).
    \] with start state \((s_0, t_0)\).
\end{definition}

The cascade product of two Mealy machines effectively executes the first on the
inputs, then executes the second on the outputs of the first; the inputs are
`cascaded' through the two Mealy machines.

\begin{example}
    \todo[inline]{Example!}
\end{example}

Tensor product is far more straightforward: it is the direct product of states
and functions.

\begin{definition}
    Let \(\mealyi\) be the PROP in which each morphism
    \(\listvar{m} \to \listvar{n}\) is a pair of a monotone
    \(\valuetuple{\listvar{m}}, \valuetuple{\listvar{n}}\)-Mealy machine
    \((S, f)\) along with a start state \(s_0 \in S\).
    Composition is by cascade product and tensor on morphisms is by
    direct product.
    The identity and the symmetry are the single-state machines that output the
    input and swap the inputs respectively.
\end{definition}

Once again, we must show that this category has a trace.
This can be computed in much the same way as it was for stream functions.

\begin{definition}
    Let \((S, f)\) be a monotone \(
        (\valuetuple{\listvar{xm}}, \valuetuple{\listvar{xn}})
    \)-Mealy machine.
    For a state \(s \in S\) and input \(\listvar{a} \in \listvar{m}\), let
    \(\mu_{s,\listvar{a}}\) be the least fixpoint of \(
        \listvar{r} \mapsto \proj{0}\left(f\left(s, \listvar{ra}\right)\right)
    \).
    The \emph{least fixed point} of \((S, f)\) with start state \(s_0\) is a \(
        (\valuetuple{\listvar{m}}, \valuetuple{\listvar{n}})
    \)-Mealy machine \(
        (S, (s, \listvar{a}) \mapsto f((\mu_{s,\listvar{a}})\listvar{a}))
    \) with start state \(s_0\).
\end{definition}

\begin{proposition}
    The least fixed point is a trace on \(\mealyi\).
\end{proposition}
\begin{proof}
    Let \((S, f)\) be a monotone
    \((\valuetuple{\listvar{xm}}, \valuetuple{\listvar{xn}})\)-Mealy machine.
    Since \(\morph{f}{S \times \listvar{xm}}{S \times \listvar{xn}}\) is
    monotone with regards to the orders on \(S\) and
    \(\valuetuple{\listvar{xm}}\) and \(S \times \listvar{xn}\) is finite, it
    has a least fixed point.
    The function \(
        f^\prime \coloneqq (s, \listvar{a})
        \mapsto
        \proj{1}\left(f((\mu_{s,\listvar{a}})\listvar{a})\right)
    \) is a composition of monotone functions, so it is itself monotone.
    This means \((S, f^\prime)\) is a monotone \(
        (\valuetuple{\listvar{m}}, \valuetuple{\listvar{n}})
    \)-Mealy machine.
    This construction is a trace for the same reason as the trace of
    \(\streami\) is.
\end{proof}

\begin{example}\label{ex:trace-mealy}
    Consider the monotone \((\valuetuple{3},\valuetuple{3})\)-Mealy machine with
    state set \(\belnapvalues\), initial state \(\bot\), and Mealy function \[
        g \coloneqq (s, (v, u, w))
        \mapsto \left\langle
                \neg(u \land v),
                (\neg(s \land w), \neg(u \land v), \neg(s \land w))
        \right\rangle
    ).\]
    To take the trace of this machine, we must first compute the least fixed
    point of \(v \mapsto \neg(s \land w)\), which is clearly just
    \(\neg(s \land w)\).
    Therefore the Mealy function of the traced
    \((\valuetuple{2}, \valuetuple{2})\) machine is \(
        (s, (u, w)) \mapsto g(s, (\neg(s \land w), u, w))
    \).
\end{example}

With monotone Mealy machines in a PROP, we can now represent the unique
homomorphism from a Mealy machine to a set of state functions as a PROP
morphism.

\begin{proposition}
    There is a traced PROP morphism
    \(\morph{\mealytostreami}{\mealyi}{\streami}\) sending a monotone Mealy
    machine \(\left((S, f), s_0\right)\) to \(!(s_0)\), where \(!\) is the
    unique homomorphism \((S,f) \to (\Gamma,\nu)\).
\end{proposition}

We can now interpret both circuits in \(\scircsigma\) and monotone Mealy
machines in \(\mealyi\) as stream functions in \(\streami\).
In essence, given one of these structures, we can determine its \emph{behaviour}
as a stream function.
What is missing, of course, is the links between them.

\section{Between circuits and Mealy machines}

The main result of this section is to establish the links between circuits
and Mealy machines, and use this as a tool to help us assert the completeness of
the denotational semantics.
To do this, we must define PROP morphisms between \(\scircsigma\) and
\(\mealyi\).

Going from \(\scircsigma\) to \(\mealyi\) is straightforward since the former
is freely generated.

\begin{definition}
    Let \(\morph{\circuittomealyi}{\scircsigma}{\mealyi}\) be the traced PROP
    morphism defined on the generators as:
    \begin{align*}
        \circuittomealy[
            \iltikzfig{strings/category/f}[box=F,colour=comb]
        ]{\interpretation}
        &\coloneqq \left(\left(\{()\}, \overline{v} \mapsto \left((), {
            \circuittofunc[
                \iltikzfig{strings/category/f}[box=F,colour=comb]
            ]{\interpretation}(\overline{v})
        }\right)\right), ()\right)
        \\[0.1em]
        \circuittomealy[
            \iltikzfig{circuits/components/values/vs}[val=v]
        ]{\interpretation}
        &\coloneqq
        \left(
            \left(
            \{s_v, s_\bot\},
            \{
                s_v \mapsto \left(s_\bot,v\right),
                s_\bot \mapsto \left(s_\bot,\bot\right)
            \}\right),
            s_v
        \right)
        \\[0.1em]
        \circuittomealy[
            \iltikzfig{circuits/components/waveforms/delay}
        ]{\interpretation}
        &\coloneqq
        \left(
            \left(
            \{ s_v \,|\, v \in \values\},
            (s_v, a) \mapsto \left({v,s_a}\right)\right),
            s_\bot
        \right)
    \end{align*}
    \todo[inline]{Fiddle with spacing}
\end{definition}

\todo[inline]{Draw diagrams for values and delays maybe?}

\begin{example}\label{ex:mealy-translation}
    Applying \(\circuittomealy{\belnapinterpretation}\) to the SR NOR latch from
    \cref{ex:latch} produces the monotone Mealy machine in
    \cref{ex:trace-mealy}.
\end{example}

To check that this is the `correct' translation, it must
\emph{preserve behaviour}: that is to say, if for a sequential circuit \(
    \iltikzfig{strings/category/f}[box=f,colour=seq,dom=m,cod=n]
\), \(
    \circuittostreami[
        \iltikzfig{strings/category/f}[box=f,colour=seq]
    ]
\) is a stream function \(f\), then \(
    \mealytostreami[
        \circuittomealyi[
            \iltikzfig{strings/category/f}[box=f,colour=seq]
        ]
    ]
\) must be the same stream function \(f\).

\begin{theorem}\label{thm:mealy-preserves-behaviour}
    \(
        \circuittostream{\interpretation}
        =
        \mealytostream \circ \circuittomealy{\interpretation}
    \).
\end{theorem}
\begin{proof}
    Since morphisms of \(\mealyi\) and \(\streami\) are both Mealy
    coalgebras, we just need to check that the transitions and outputs of the
    image of \(\circuittomealyi\) and \(\circuittostreami\) agree.
\end{proof}

So one direction is done; now for the other, which as is often the case is
a bit harder.
For regular Mealy machines, there is a standard procedure in circuit
design~\cite{kohavi2009switching} in which each state of a Mealy machine is
\emph{encoded} as a power of values, and combinational logic used to transform
inputs into appropriate outputs.

\begin{example}
    \todo[inline]{Simple boolean examples}
\end{example}

When considering \emph{monotone} Mealy machines and trying to map into
\(\scircsigma\), this procedure must respect monotonicity as the combinational
logic is constructed using monotone components.
This means that an arbitrary encoding cannot be used; we will now show how to
select a suitable encoding.

\begin{definition}
    Let \(S\) be a set equipped with a partial order \(\stateorder\) and a total
    order \(\leq\) such that \(s_0 \leq s_1 \leq \dots s_{k-1}\).
    The \emph{\(\leq\)-encoding} for this assignment is a function
    \(\morph{\gamma_\leq}{S}{\valuetuple{k}}\) defined as
    \(\gamma_\leq(s)(i) \coloneqq \top\) if \(s_i \stateorder s\) and
    \(\gamma_\leq(s)(i) \coloneqq \bot\) otherwise.
\end{definition}

\begin{example}
    Recall the monotone Mealy machine from \cref{ex:mealy-translation}, which
    has state set \(
        \belnapvalues \coloneqq \{\bot,\belnapfalse,\belnaptrue,\top\}
    \).
    We choose the total order on \(\belnapvalues\) as
    \(\bot \leq \belnapfalse \leq \belnaptrue \leq \top\); subsequently, the
    \(\leq\)-encoding is defined as \(
        \bot \mapsto \top\bot\bot\bot, \belnapfalse \mapsto \top\top\bot\bot,
        \belnaptrue \mapsto \top\bot\top\bot, \top \mapsto \top\top\top\top
    \).
\end{example}

It is essential that a \(\leq\)-encoding respects the original ordering of the
states.

\begin{lemma}
    For an ordered state space \((S,\stateorder)\) and a \(\leq\)-encoding
    \(\gamma_\leq\), \(s \stateorder s^\prime\) if and only if
    \(\gamma_\leq(s) \sqsubseteq \gamma_\leq(s^\prime)\).
\end{lemma}
\begin{proof}
    First the \(\onlyifdir\) direction.
    Let \(s_i \stateorder s_j\); we need to show that for every \(l < k\),
    \(s_i(l) \sqsubseteq s_j(l)\).
    The only way this can be violated is if \(s_i(l) = \top\) and
    \(s_j(l) = \bot\).
    But since \(s_i \stateorder s_j\), if \(s_l \stateorder s_i\) then
    \(s_l \stateorder s_j\) also holds.

    Now the \(\ifdir\) direction.
    Assume that \(\gamma_\leq(s_i) \sqsubseteq \gamma_\leq(s_j)\); we need to
    show that \(s_i \stateorder s_j\); i.e.\ that \(\gamma_\leq(s_j)(i) = \top\)
    If \(\gamma_\leq(s_i) \sqsubseteq \gamma_\leq(s_j)\), then for each
    \(l < k\) then \(\gamma_\leq(s_i)(l) \sqsubseteq \gamma_\leq(s_j)(l)\);
    in particular \(\gamma_\leq(s_i)(i) \sqsubseteq \gamma_\leq(s_j)(i)\)
    By definition of \(\gamma_\leq\), \(\gamma_\leq(s_i)(i) = \top\), so if
    \(\gamma_\leq(s_i) \sqsubseteq \gamma_\leq(s_j)\) then
    \(\gamma_\leq(s_j)(i)\) is also \(\top\).
\end{proof}

Using this encoding, we will construct a combinational circuit morphism that,
when interpreted as a function, implements the output and transition function
of the Mealy machine.
There is no reason for such a morphism to exist for an arbitrary interpretation:
why should we expect some hurriedly slapped together collection of gates to be
able to model every function?
The useful interpretations are those that \emph{can} model every function.

\begin{definition}[Functional completeness]
    An interpretation \(\interpretation\) of a signature \(\signature\) is
    \emph{functionally complete} if there exists a map \(
        \morph{\mealytofunc}{\funci}{\scircsigma}
    \) which sends functions \(
        \morph{f}{\valuetuple{\listvar{m}}}{\valuetuple{\listvar{n}}}
    \), to circuits of the form \(
        \iltikzfig{circuits/synthesis/normalised-function}
    \) for some word \(\listvar{v} \in \freemon{\values}\) such that
    \(\circuittostreami[\mealytofunc[f]](\sigma)(i) = f(\sigma(i))\).
\end{definition}

For a particular interpretation there may well be many such maps from functions
to circuits, but for convenience we will assume there is a fixed procedure
\(\mealytofunc\) amd refer to a circuit \(\mealytofunc[f]\) as the
\emph{normalised circuit for \(f\)}.\

\begin{remark}
    Even though \(\mealytofunc\) concerns combinational functions, it maps into
    the category of \emph{sequential} circuits \(\scircsigma\)!
    This is because sometimes instantaneous values must be used to create the
    normalised circuit.
    Despite this, the semantics of the circuit are still combinational since the
    loop enforces that the state is \emph{constant}: it will always produce the
    word \(\listvar{v}\).
    This means the circuit has combinational behaviour despite using sequential
    components; often this is the only way to ensure every function can be
    modelled.
\end{remark}

\begin{example}
    The Belnap interpretation from \cref{ex:belnap-interpretation} is
    functionally complete; for interests of space we postpone the proof for
    \cref{sec:belnap}.
\end{example}

\section{Between circuits and Mealy machines}

We have now established the close links between \(\streami\) and \(\mealyi\).
The main result of this section is to introduce \(\scircsigma\) to the mix.

\subsection{Circuits to monotone Mealy machines}

Circuits have a very natural interpretation as Mealy machines: going from
\(\scircsigma\) to \(\mealyi\) is straightforward since the former is freely
generated.

\begin{definition}
    Let \(\morph{\circuittomealyi}{\scircsigma}{\mealyi}\) be the traced PROP
    morphism defined on the generators as:
    \begin{align*}
        \circuittomealy[
            \iltikzfig{strings/category/f}[box=F,colour=comb]
        ]{\interpretation}
        &\coloneqq \left(\left(\{()\}, \overline{v} \mapsto \left((), {
            \circuittofunc[
                \iltikzfig{strings/category/f}[box=F,colour=comb]
            ]{\interpretation}(\overline{v})
        }\right)\right), ()\right)
        \\[0.1em]
        \circuittomealy[
            \iltikzfig{circuits/components/values/vs}[val=v]
        ]{\interpretation}
        &\coloneqq
        \left(
            \left(
            \{s_v, s_\bot\},
            \{
                s_v \mapsto \left(s_\bot,v\right),
                s_\bot \mapsto \left(s_\bot,\bot\right)
            \}\right),
            s_v
        \right)
        \\[0.1em]
        \circuittomealy[
            \iltikzfig{circuits/components/waveforms/delay}
        ]{\interpretation}
        &\coloneqq
        \left(
            \left(
            \{ s_v \,|\, v \in \values\},
            (s_v, a) \mapsto \left({v,s_a}\right)\right),
            s_\bot
        \right)
    \end{align*}
    \todo[inline]{Fiddle with spacing}
\end{definition}

\todo[inline]{Draw diagrams for values and delays maybe?}

\begin{example}\label{ex:mealy-translation}
    Applying \(\circuittomealy{\belnapinterpretation}\) to the SR NOR latch from
    \cref{ex:latch} produces the monotone Mealy machine in
    \cref{ex:trace-mealy}.
\end{example}

Mealy machines are a reasonable semantics for sequential circuits, but the
image of \(\circuittomealyi\) does not always lead to minimal Mealy machines,
and there are many Mealy machines that may correspond to the same behaviour.
The `purest' semantics of a sequential circuit is a stream function in
\(\streami\).

\begin{definition}
    Let \(\morph{\circuittostreami}{\scircsigma}{\streami}\) be defined as
    \(\mealytostreami \circ \circuittostreami\).
\end{definition}

We have now finally established the \emph{denotation} of a sequential circuit \(
    \iltikzfig{strings/category/f}[box=f,colour=seq,dom=\listvar{m},cod=\listvar{n}]
\): it is the stream function \(
    \morph{\circuittostreami[\iltikzfig{strings/category/f}[box=f,colour=seq]]}{\valuetuplestream{\listvar{m}}}{\valuetuplestream{\listvar{n}}}
\).

\subsection{Monotone Mealy machines to circuits}

So one direction is done; now for the other, which as is often the case is
a bit harder.
For regular Mealy machines, there is a standard procedure in circuit
design~\cite{kohavi2009switching} in which each state of a Mealy machine is
\emph{encoded} as a power of values, and combinational logic used to transform
inputs into appropriate outputs.

\begin{example}
    \todo[inline]{Simple boolean examples}
\end{example}

When considering \emph{monotone} Mealy machines and trying to map into
\(\scircsigma\), this procedure must respect monotonicity as the combinational
logic is constructed using monotone components.
This means that an arbitrary encoding cannot be used; we will now show how to
select a suitable encoding.

\begin{definition}[Encoding]\label{def:encoding}
    Let \(S\) be a set equipped with a partial order \(\stateorder\) and a total
    order \(\leq\) such that \(s_0 \leq s_1 \leq \dots s_{k-1}\).
    The \emph{\(\leq\)-encoding} for this assignment is a function
    \(\morph{\gamma_\leq}{S}{\valuetuple{k}}\) defined as
    \(\gamma_\leq(s)(i) \coloneqq \top\) if \(s_i \stateorder s\) and
    \(\gamma_\leq(s)(i) \coloneqq \bot\) otherwise.
\end{definition}

\begin{example}
    Recall the monotone Mealy machine from \cref{ex:mealy-translation}, which
    has state set \(
        \belnapvalues \coloneqq \{\bot,\belnapfalse,\belnaptrue,\top\}
    \).
    We choose the total order on \(\belnapvalues\) as
    \(\bot \leq \belnapfalse \leq \belnaptrue \leq \top\); subsequently, the
    \(\leq\)-encoding is defined as \(
        \bot \mapsto \top\bot\bot\bot, \belnapfalse \mapsto \top\top\bot\bot,
        \belnaptrue \mapsto \top\bot\top\bot, \top \mapsto \top\top\top\top
    \).
\end{example}

It is essential that a \(\leq\)-encoding respects the original ordering of the
states.

\begin{lemma}
    For an ordered state space \((S,\stateorder)\) and a \(\leq\)-encoding
    \(\gamma_\leq\), \(s \stateorder s^\prime\) if and only if
    \(\gamma_\leq(s) \sqsubseteq \gamma_\leq(s^\prime)\).
\end{lemma}
\begin{proof}
    First the \(\onlyifdir\) direction.
    Let \(s_i \stateorder s_j\); we need to show that for every \(l < k\),
    \(s_i(l) \sqsubseteq s_j(l)\).
    The only way this can be violated is if \(s_i(l) = \top\) and
    \(s_j(l) = \bot\).
    But since \(s_i \stateorder s_j\), if \(s_l \stateorder s_i\) then
    \(s_l \stateorder s_j\) also holds.

    Now the \(\ifdir\) direction.
    Assume that \(\gamma_\leq(s_i) \sqsubseteq \gamma_\leq(s_j)\); we need to
    show that \(s_i \stateorder s_j\); i.e.\ that \(\gamma_\leq(s_j)(i) = \top\)
    If \(\gamma_\leq(s_i) \sqsubseteq \gamma_\leq(s_j)\), then for each
    \(l < k\) then \(\gamma_\leq(s_i)(l) \sqsubseteq \gamma_\leq(s_j)(l)\);
    in particular \(\gamma_\leq(s_i)(i) \sqsubseteq \gamma_\leq(s_j)(i)\)
    By definition of \(\gamma_\leq\), \(\gamma_\leq(s_i)(i) = \top\), so if
    \(\gamma_\leq(s_i) \sqsubseteq \gamma_\leq(s_j)\) then
    \(\gamma_\leq(s_j)(i)\) is also \(\top\).
\end{proof}

Using this encoding, we will construct a combinational circuit morphism that,
when interpreted as a function, implements the output and transition function
of the Mealy machine.
There is no reason for such a morphism to exist for an arbitrary interpretation:
why should we expect some hurriedly slapped together collection of gates to be
able to model every function?
The useful interpretations are those that \emph{can} model every function.

\begin{definition}[Functional completeness]
    An interpretation \(\interpretation\) of a signature \(\signature\) is
    \emph{functionally complete} if there exists a map \(
        \morph{\mealytofunc}{\funci}{\scircsigma}
    \) which sends functions \(
        \morph{f}{\valuetuple{\listvar{m}}}{\valuetuple{\listvar{n}}}
    \), to circuits of the form \(
        \iltikzfig{circuits/synthesis/normalised-function}
    \) for some word \(\listvar{v} \in \freemon{\values}\) such that
    \(\circuittostreami[\mealytofunc[f]](\sigma)(i) = f(\sigma(i))\).
\end{definition}

For a particular interpretation there may well be many such maps from functions
to circuits, but for convenience we will assume there is a fixed procedure
\(\mealytofunc\) amd refer to a circuit \(\mealytofunc[f]\) as the
\emph{normalised circuit for \(f\)}.\

\begin{remark}
    Even though \(\mealytofunc\) concerns combinational functions, it maps into
    the category of \emph{sequential} circuits \(\scircsigma\)!
    This is because sometimes instantaneous values must be used to create the
    normalised circuit.
    However, the loop enforces that the state is \emph{constant}: it will always
    produce the word \(\listvar{v}\).
    This means the circuit has combinational behaviour despite using sequential
    components; often this is the only way to ensure every function can be
    modelled.
\end{remark}

\begin{example}
    The Belnap interpretation from \cref{ex:belnap-interpretation} is
    functionally complete; for interests of space we postpone the proof for
    \cref{sec:belnap}.
    In essence, this is due to a variation of the standard functional
    completeness method for Boolean values, in which a term in disjunctive
    normal form is created by reading lines off a truth table.
\end{example}

With the knowledge that any monotone function has a corresponding circuit
in \(\scircsigma\), we set about encoding the Mealy function.

\begin{definition}[Monotone completion]
    For lattices \(M, N, P\) such that \(M \subseteq N\), and a monotone
    function \(\morph{f}{M}{P}\), let the \emph{monotone completion} of \(f\) be
    the function \(\morph{f_\mathsf{m}}{N}{P}\) recursively defined as \[
        f_\mathsf{m}(v) = \begin{cases}
            f(v)
            &
            \text{if}\ v \in M
            \\
            \bot
            &
            \text{if}\ v = \bot^m, \bot \not\in M
            \\
            \bigsqcup \{ f_\mathsf{m}(w) \,|\, w \leq_N v \}
            &
            \text{otherwise}
        \end{cases}
    \]
\end{definition}

\begin{definition}[Monotone Mealy encoding]\label{def:mealy-encoding}
    For a monotone Mealy machine \((S, f, s_0)\) with \(k\) states and a
    monotone encoding \(\gamma_\leq\), a \emph{monotone Mealy encoding} is a
    function
    \(
        \morph{
            \gamma_\leq(f)
        }{
            \valuetuple{k} \times \valuetuple{m}
        }{
            \valuetuple{k} \times \valuetuple{n}
        }
    \) defined as the monotone completion of the function \(
        (\gamma_\leq(s), \overline{x})
        \mapsto
            (
                \gamma_\leq(\mealyfunctiontransition{f}(s, \overline{x})),
                \mealyfunctionoutput{f}(s, \overline{x})
            )
    \).
\end{definition}

To be able to obtain the syntactic circuit for a monotone Mealy function encoded
in this way, it needs to be a morphism in \(\funci\)!
It is monotone by definition, but we need to make sure it is also
\(\bot\)-preserving.

\begin{lemma}
    A monotone Mealy encoding is in \(\funci\).
\end{lemma}
\begin{proof}
    A Mealy encoding is monotone as it is a monotone completion.
    There cannot be a state encoded as \(\bot^k\), since at least one bit must
    be \(\top\); this means the monotone completion will send the input
    \(\bot^k \concat \bot^m\) to \(\bot^k \concat \bot^n\): it is
    \(\bot\)-preserving.
\end{proof}

The foundations are now set for establishing the image of a PROP morphism from
Mealy machines to circuit terms.
But before we get too ahead of ourselves, there is one more thing to consider:
\cref{def:encoding} depends on some arbitrary total ordering on the states in a
given monotone Mealy machine.
While this may not seem much of an issue (surely one could simply choose?), when
defining a PROP morphism this must be \emph{fixed}, otherwise a Mealy machine
might be mapped to different circuit morphisms depending on the time of day!

\begin{definition}[Chosen state order]
    Let \((S, f, s_0)\) be a monotone Mealy machine with input space
    \(\valuetuple{m}\), and let \(\leq\) be a total order on \(\values\);
    \(\leq\) can be extended to \(\freemon{(\values^m)}\) using the
    lexicographic order.
    Given a state \(s\), let \(t_{s,\leq} \in \freemon{(\values^m)}\) be
    the minimal element of the subset of words that transition from \(s_0\) to
    \(s\), ordered by \(\leq\).
    Then the \emph{chosen state order} \(\leq_S\) is a total order on \(S\)
    defined as \(s \leq_S s^\prime\) if \(t_{s,\leq} \leq t_{s^\prime,\leq}\).
\end{definition}

The PROP morphism from monotone Mealy machines to circuits can then be
parameterised by some ordering on the set of values \(\values\), ensuring that
there is a canonical term in \(\scircsigma\) for each monotone Mealy machine.

\begin{definition}\label{def:mealy-to-circuit}
    For a functionally complete interpretation \(\interpretation\) and total
    order \(\leq\) on \(\values\), let \(
        \morph{
            \mealytocircuiti
        }{
            \mealyi
        }{
            \scircsigma
        }
    \) be the traced PROP morphism with action defined for a monotone Mealy
    machine \((S,f,s)\) as producing \(
        \iltikzfig{circuits/synthesis/mealy-term}
    \).
\end{definition}

Before proceeding to the result that this PROP morphism is behaviour-preserving,
we must show a lemma linking the behaviour circuits in the image of
\(\mealytocircuiti\) to initial outputs and stream derivatives.

\begin{proposition}
    \label{prop:mealy-form-image}
    Given a combinational circuit \(
        \iltikzfig{strings/category/f-2-2}[box=\hat{F},dom1=x,dom2=m,cod1=x,cod2=n,colour=comb]
    \), let \(f\) be the map with action \(
        (\listvar{s}) \mapsto
            \circuittostreami[
                \iltikzfig{circuits/productivity/mealy-form}[core=\hat{F},state=\listvar{s},dom=m,cod=n,delay=x]
            ]
    \) and let \(
        \hat{f}
        :=
        \circuittofunci[
            \iltikzfig{strings/category/f-2-2}[box=\hat{F},colour=comb]
        ]
    \).
    Then, \(
        \mealyoutput{f(\listvar{s})}{\listvar{a}}
        =
        \proj{1}(\hat{f}(\listvar{s}, \listvar{a}))
    \) and \(
        \mealytransition{f(\listvar{s})}{\listvar{a}}
        =
        f(\proj{0}(\hat{f}(\listvar{s}, \listvar{a})))
    \).
\end{proposition}
\begin{proof}
    None of the traced inputs affect the outputs at the
    current cycle, so the initial output is trivial.
    The least fixpoint is reached immediately as \(
        \mu_a := \proj{0}(\hat{f}(\listvar{s}, \listvar{a}))
    \) and the traced input is captured by the delay, so the derivative
    will indeed be \(f(\mu_a)\).
\end{proof}

Now we can show the behaviour-preserving result.

\begin{theorem}\label{thm:mealy-to-circuit}
    \(
        \mealytostream = \circuittostreami \circ \mealytocircuiti
    \).
\end{theorem}
\begin{proof}
    For fixed state set \(S\) and Mealy function \(g\), let
    \(h := (s) \mapsto \mealytostreami[(S, g, s)]\) and \(
        h^\prime := (s) \mapsto \circuittostreami[\mealytocircuiti[(S, g, s)]]
    \).
    We will show that for a state \(s \in S\), there is a bisimulation
    between \(h(s)\) and \(h^\prime(s)\).

    We can transform \(h^\prime(s)\) into a circuit of the form in
    \cref{prop:mealy-form-image} by concatenating the values created by
    \(\mealytofunc\) to the encoded state using axioms of STMCs.
    Then, by finality of \(\mealytostreami\), \cref{def:mealy-to-circuit} and
    \cref{prop:mealy-form-image}, these stream functions both have initial output \(
        \proj{1}(g(s, \listvar{a}))
    \).
    The former has \(
        \streamderivative{h(s)}{\listvar{a}} = h(\proj{0}(g(s, \listvar{a})))
    \) and the latter has \(
        \streamderivative{h^\prime(s)}{\listvar{a}} =
            \gamma(\proj{0}(g(s, \listvar{a})) =
            h^\prime(\proj{0}(g(s, \listvar{a})))
    \).
    As \(h(s)\) and \(h^\prime(s)\) are defined on all states,
    \(\mealytostreami[(S, g, s)]\) and
    \(\circuittostreami[\mealytocircuiti[(S, g, s)]]\) are equal for any Mealy
    machine \((S, g, s)\).
\end{proof}


This brings our foray into Mealy machines to a close: we can translate back and
forth between circuits and \emph{monotone} Mealy machines without fear of
altering their behaviour in terms of their stream functions.

This already is quite nice to have; if we only know the specification of a
circuit in terms of a (monotone) Mealy machine, we can use the PROP morphism
\(\mealytocircuiti\) to generate a circuit rather than attempting to blindly
construct one from primitive logical

Our work is not done yet though: these results on Mealy machines only set the
stage for the final result of this chapter: the completeness of
the denotational semantics.
\section{Completeness of the denotational semantics}

\todo[inline]{Map from streams to Mealy machines}

We conclude by showing that each function in \(\streami\) has at least one
circuit in \(\scircsigma\) with the same behaviour under \(\interpretation\).

\begin{corollary}\label{thm:circuit-stream-correspondence}
    \(
        \circuittostreami
        \circ
        \mealytocircuiti
        \circ
        \streamtomealyi
        =
        \id[\streami]
    \).
\end{corollary}
\begin{proof}
    First we show the former:
    \begin{align*}
        \circuittostream{\interpretation} \circ
        \psi \circ
        \phi
        &=
        \circuittostream{\interpretation} \circ
        \mealytocircuiti \circ
        \streamtomealyi \circ
        \mealytostream \circ
        \circuittomealy{\interpretation}
        \\
        &=
        \mealytostream \circ
        \streamtomealyi \circ
        \mealytostream \circ
        \circuittomealy{\interpretation}
        & \text{\cref{thm:mealy-to-circuit}}
        \\
        &=
        \mealytostream \circ
        \circuittomealy{\interpretation}
        & \text{\cref{cor:minimal-mealy}}
        \\
        &=
        \circuittostream{\interpretation}
        & \text{\cref{thm:mealy-to-circuit}}
    \end{align*}

    Now the latter:
    \begin{align*}
        \phi \circ \psi
        &=
        \mealytostream \circ
        \circuittomealy{\interpretation} \circ
        \mealytocircuiti \circ
        \streamtomealyi
        \\
        &=
        \circuittostream{\interpretation} \circ
        \mealytocircuiti \circ
        \streamtomealyi
        & \text{\cref{thm:mealy-preserves-behaviour}}
        \\
        &=
        \mealytostream \circ
        \streamtomealyi
        & \text{\cref{thm:mealy-to-circuit}}
        \\
        &=
        \id[\streami]
        & \text{\cref{cor:minimal-mealy}}
    \end{align*}
\end{proof}

There is no isomorphism between \(\scircsigma\) and \(\streami\)
as many circuits may have the same semantics but different syntax.

\begin{definition}[Denotational equivalence]
    We say that two sequential circuits are \emph{denotationally equivalent}
    under \(\interpretation\), written \(
        \iltikzfig{strings/category/f}[box=F,colour=seq,dom=m,cod=n]
        \approx_{\interpretation}
        \iltikzfig{strings/category/f}[box=G,colour=seq,dom=m,cod=n]
    \) if \(
        \circuittostream[
            \iltikzfig{strings/category/f}[box=F,colour=seq]
        ]{\interpretation}
        =
        \circuittostream[
            \iltikzfig{strings/category/f}[box=G,colour=seq]
        ]{\interpretation}
    \).
    Let \(\scircsigmai\) be the result of quotienting \(\scircsigma\) by \(
        \approx_{\interpretation}
    \).
\end{definition}

\begin{corollary}
    \(\scircsigmai \cong \streami\).
\end{corollary}

This confirms that \(\streami\), a PROP of monotone causal stream functions
with finitely many stream derivatives, is a suitable semantic domain for
sequential circuits.