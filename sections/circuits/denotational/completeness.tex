\section{Completeness of the denotational semantics}


We conclude by showing that each function in \(\streami\) has at least one
circuit in \(\scircsigma\) with the same behaviour under \(\interpretation\).

\begin{corollary}\label{thm:circuit-stream-correspondence}
    \(
        \circuittostreami
        \circ
        \mealytocircuiti
        \circ
        \streamtomealyi
        =
        \id[\streami]
    \).
\end{corollary}
\begin{proof}
    First we show the former:
    \begin{align*}
        \circuittostream{\interpretation} \circ
        \psi \circ
        \phi
        &=
        \circuittostream{\interpretation} \circ
        \mealytocircuiti \circ
        \streamtomealyi \circ
        \mealytostream \circ
        \circuittomealy{\interpretation}
        \\
        &=
        \mealytostream \circ
        \streamtomealyi \circ
        \mealytostream \circ
        \circuittomealy{\interpretation}
        & \text{\cref{thm:mealy-to-circuit}}
        \\
        &=
        \mealytostream \circ
        \circuittomealy{\interpretation}
        & \text{\cref{cor:minimal-mealy}}
        \\
        &=
        \circuittostream{\interpretation}
        & \text{\cref{thm:mealy-to-circuit}}
    \end{align*}

    Now the latter:
    \begin{align*}
        \phi \circ \psi
        &=
        \mealytostream \circ
        \circuittomealy{\interpretation} \circ
        \mealytocircuiti \circ
        \streamtomealyi
        \\
        &=
        \circuittostream{\interpretation} \circ
        \mealytocircuiti \circ
        \streamtomealyi
        & \text{\cref{thm:mealy-preserves-behaviour}}
        \\
        &=
        \mealytostream \circ
        \streamtomealyi
        & \text{\cref{thm:mealy-to-circuit}}
        \\
        &=
        \id[\streami]
        & \text{\cref{cor:minimal-mealy}}
    \end{align*}
\end{proof}

There is no isomorphism between \(\scircsigma\) and \(\streami\)
as many circuits may have the same semantics but different syntax.

\begin{definition}[Denotational equivalence]
    We say that two sequential circuits are \emph{denotationally equivalent}
    under \(\interpretation\), written \(
        \iltikzfig{strings/category/f}[box=F,colour=seq,dom=m,cod=n]
        \approx_{\interpretation}
        \iltikzfig{strings/category/f}[box=G,colour=seq,dom=m,cod=n]
    \) if \(
        \circuittostream[
            \iltikzfig{strings/category/f}[box=F,colour=seq]
        ]{\interpretation}
        =
        \circuittostream[
            \iltikzfig{strings/category/f}[box=G,colour=seq]
        ]{\interpretation}
    \).
    Let \(\scircsigmai\) be the result of quotienting \(\scircsigma\) by \(
        \approx_{\interpretation}
    \).
\end{definition}

\begin{corollary}
    \(\scircsigmai \cong \streami\).
\end{corollary}

This confirms that, for a functionally complete interpretation
\(\interpretation\), the PROP \(\streami\) of causal, finitely specified and
monotone stream functions, is a \emph{sound and complete} semantic domain for
sequential circuits: every circuit in \(\scircsigma\) has a stream
function in \(\streami\) and every function \(f \in \streami\) has a class of
circuits with \(f\) as their behaviour.