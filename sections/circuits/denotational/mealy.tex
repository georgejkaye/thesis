\section{Monotone Mealy machines}

It is not immediately obvious how to translate back from stream functions in
\(\streami\) to circuits in \(\scircsigma\).
Even though these stream functions have finitely many stream derivatives, how
does one encapsulate this behaviour into a circuit made from finitely many
components while taking into account inputs and state?
Fortunately, we have a secret weapon; the old chestnut that is the
\emph{Mealy machine}~\cite{mealy1955method}.
Working tirelessly in industry as a way of specifying the behaviour of a
circuit without restricting to particular circuitry, Mealy machines also have
a very useful \emph{coalgebraic} viewpoint which we will wield in order to
build a bridge from circuits into stream functions: there is a unique
homomorphism from a Mealy machine to a causal, finitely specified stream
function.

We will do this by assembling a special class of Mealy machines which we dub
\emph{monotone Mealy machines} into another traced PROP.
By constructing PROP morphisms between these two categories that preserve
behaviour of stream functions, we establish that we can work with either
circuit representation.

\begin{definition}[Mealy machine~\cite{mealy1955method}]\label{def:mealy}
    Let \(A\) and \(B\) be finite sets.
    A (finite) \((A,B)\)-\emph{Mealy machine} is a tuple \((S, f)\) where
    \(S\) is a finite set called the \emph{state space},
    \(\morph{f}{S}{(S \times B)^A}\) is the \emph{Mealy function}.
\end{definition}

An \((M,N\)-Mealy machine is parameterised over a set \(A\) of \emph{inputs} and
a set \(B\) of \emph{outputs}.
Each Mealy machine then has a set \(S\) of \emph{states}; the Mealy function
takes in a currrent state and input \((s, a)\) and produces a pair
\(\langle{s^\prime, b}\rangle\)of a transition and output.

\begin{notation}
    We will use the shorthand \(
    \mealyfunctiontransition{f} := (s, a) \mapsto \proj{0}(f(s)(a))
\) and \(
    \mealyfunctionoutput{f} := (s, a) \mapsto \proj{1}(f(s)(a))
\) for the transition and output component of the Mealy function respectively.
\end{notation}

\begin{example}
    \todo[inline]{Example!}
\end{example}

\subsection{The coalgebraic perspective}

The definition of Mealy machine above is timeless and forms the basis for most
of modern electronics.
Of course, the natural question for the categorist to ask is
\emph{can we make it more categorical?}
And as is often the case, we can, using the notion of a \emph{coalgebra}.

\begin{definition}[Coalgebra]
    Let \(\mcc\) and \(\mcd\) be categories, and let \(\morph{F}{\mcc}{\mcc}\)
    be an endofunctor.
    A \emph{coalgebra} for \(F\), or \(F\)-coalgebra, is an object
    \(A \in \mcc\) along with a morphism \(\morph{\alpha}{A}{FA} \in \mcc\),
    usually written \((A,\alpha)\).
\end{definition}

The light bulb should already have lit up: a Mealy machine is a pair of a set
and a function, so this is clearly a coalgebra in \(\set\).

\begin{definition}
    An \emph{\((A,B)\)-Mealy coalgebra} is a coalgebra of the functor
    \(\morph{Y}{\set}{\set}\) defined as \(S \mapsto (S \times B)^A\).
\end{definition}

\begin{example}
    In \cite{bonsangue2008coalgebraic}, the notation \(
        f(s)(a) = \langle{\streaminit{s}{a},\streamderivative{s}{a}}\rangle
    \) is used to describe the Mealy function, which coincides with the notation
    used for stream functions; this is not a coincidence!

    Given sets \(A\) and \(B\), let \(\Gamma\) be the set of causal stream
    functions \(\stream{A} \to \stream{B}\), and let
    \(\morph{\nu}{\Gamma}{\Gamma \times B}^A\) be the function defined as \(
        (f, a) \mapsto \langle{\streaminit{s}{a},\streamderivative{s}{a}}\rangle
    \).
    Then \((\Gamma,\nu)\) is a \((A,B)\)-Mealy coalgebra.
\end{example}

The above example is incredibly important, as it lays the groundwork we will
use extensively to establish connections between circuits, stream functions and
Mealy machines.
If we inspect it a little closer, we find that stream functions are even more
special than just being `a' \((A,B)\)-Mealy coalgebra.

\begin{definition}[Mealy homomorphism]
    A \emph{Mealy homomorphism} between two \((A,B)\)-Mealy coalgebra \((S,f)\)
    and \((T,g)\) is a function \(\morph{h}{S}{T}\) preserving transitions and
    outputs, i.e.\ for a Mealy function \(f\), a Mealy homomorphism \(h\)
    satisfies \(
        h\left(\mealyfunctiontransition{f}(s, a)\right)
        =
        \mealyfunctiontransition{f}(h(s), a)
    \) and \(
        h\left(\mealyfunctionoutput{f}(s, a)\right)
        =
        \mealyfunctionoutput{f}(s, a)
    \).
\end{definition}

The \emph{final} \((M,N)\)-Mealy coalgebra is a \((M,N)\)-Mealy coalgebra to
which every other \((M,N)\)-Mealy coalgebra has a unique homomorphism to.

\begin{remark}
    The coalgebra crowd tend to prefer the use of the word \emph{final} over
    \emph{terminal}; one could quite reasonably call the final coalgebra a
    terminal coalgebra and everything would be alright, but you might get a few
    funny looks.
\end{remark}

\begin{notation}\label{not:transitions}
    For a Mealy coalgebra \(
        (S, f)
    \), state \(s \in S\) and index \(i \in \nat\), let the
    \emph{\(i\)-th transition on \(\sigma\)}, written
    \(\transitions{f}{i}{s}{\sigma}\), be defined as
    \(\transitions{f}{0}{s}{\sigma} := s\) and
    \(\transitions{f}{k+1}{s}{\sigma} :=
        \mealyfunctiontransition{f}(\transitions{f}{k}{s}{\sigma}, \sigma(k))
    \).
\end{notation}

\begin{proposition}[\cite{rutten2006algebraic}, Prop. 2.2]
    \label{prop:final-coalgebra}
    For every Mealy coalgebra \((S,f)\), there exists a
    unique homomorphism \(\morph{{!}}{(S,f)}{(\Gamma,\nu)}\).
\end{proposition}
\begin{proof}
    A homomorphism \(\morph{g}{(S, f)}{(\Gamma, \nu)}\) is a function
    \(S \to \Gamma\), so for a state \(s \in S\), \(g(s)\) will be a stream
    function \(\stream{M} \to \stream{N}\).
    For stream \(\sigma \in \stream{M}\) and \(i \in \nat\), the unique stream
    function is defined as
    \(
        h(s)(\sigma)(i)
        :=
        \mealyfunctionoutput{f}(f^i_1(s, \sigma), \sigma(i))
    \).
\end{proof}

