\section{Partial evaluation}

Partial evaluation~\cite{jones1996introduction} is a paradigm used in software
optimisation in which programs are `evaluated as much as possible' while only
some of the inputs are specified.
For example, it may be the case that a particular input to a program is fixed
for long periods of time; using partial evaluation, we can define a program
specialised for this input.
This program might run significantly faster than the original.

There has been work into partial evaluation for hardware, such as constant
propagation~\cite{singh1996expressing,singh1999partial} and
unfolding~\cite{thompson2006bitlevel}.
However, this has been relatively informal, and can be made rigorous using the
categorical framework.
In this section we will discuss how we could extend the reduction-based
operational semantics to define automatic procedures for applying partial
evaluation to circuits.

\subsection{Tidying up}

When building a circuit, it is often desirable to reduce the number of wires
and components used; this reduces both the physical size of the circuit and its
power consumption.
We can use partial evaluation to transform a circuit into a more minimal form.

\begin{definition}[Tidying rules]
    Let the set of \emph{tidying rules} be defined as the rules in
    \cref{fig:tidy-rules}.
\end{definition}

Most of the tidying rules are self-explanatory; the final rule is necessary in
order to deal with traced circuits with no outputs.
Since all circuits with no outputs have the same behaviour, we are permitted to
`cut' the trace and get a circuit we can apply more tidying rules to.
As non-delay-guarded feedback is already handled by the
\((\instantfeedbackeqn)\) rule, we only need to consider the delay-guarded case.

\begin{figure}
    \centering
    \(
    \iltikzfig{strings/structure/comonoid/unitality-l-lhs}
    \reduction
    \iltikzfig{strings/structure/comonoid/unitality-l-rhs}
    \)
    \quad
    \(
    \iltikzfig{strings/structure/comonoid/unitality-r-lhs}
    \reduction
    \iltikzfig{strings/structure/comonoid/unitality-r-rhs}
    \)
    \quad
    \(
    \iltikzfig{strings/structure/bialgebra/merge-discard-lhs}
    \reduction
    \iltikzfig{strings/structure/bialgebra/merge-discard-rhs}
    \)
    \\[0.5em]
    \(
    \iltikzfig{circuits/axioms/gate-stub-lhs}
    \reduction
    \iltikzfig{circuits/axioms/gate-stub-rhs}
    \)
    \quad
    \(
    \iltikzfig{circuits/axioms/bundler-stub-lhs}
    \reduction
    \iltikzfig{circuits/axioms/bundler-stub-rhs}
    \)
    \quad
    \(
    \iltikzfig{circuits/axioms/stub-lhs}
    \reduction
    \iltikzfig{strings/monoidal/empty}
    \)
    \quad
    \(
    \iltikzfig{circuits/axioms/unobservable-lhs}
    \reduction
    \iltikzfig{circuits/axioms/unobservable-rhs}
    \)
    \quad
    \(
    \iltikzfig{circuits/examples/tidying/redundant-trace-lhs}[box=f]
    \reduction
    \iltikzfig{circuits/examples/tidying/redundant-trace-rhs}[box=f]
    \)
    \caption{Rules for tidying up circuits in Mealy form}
    \label{fig:tidy-rules}
\end{figure}

\begin{proposition}
    Applying the tidying rules to a circuit in Mealy form is confluent and
    terminating.
\end{proposition}
\begin{proof}
    The tidying rules always decrease the size of the circuit.
    The only choice is raised when there is a trace around a combinational
    circuit, but this does not change the internal structure of the subcircuit,
    so rule applications are prevented.
    Moreover, since all this rule does is `cut' a trace, it does not matter if
    this is performed `all in one go', or each feedback loop is cut one by one.
\end{proof}

\subsection{Shortcut rules}

It is often the case that we know that some of the inputs to a circuit is fixed.
This can be modelled by precomposing the relevant input with an
\emph{infinite waveform} \(
\iltikzfig{circuits/components/waveforms/infinite-register}[val=v]
\).
We can propagate these waveforms across a circuit to see if we can reduce it to
something simpler \emph{specialised} for these inputs.

To propagate waveforms across circuits we need to derive a version of the
\((\gateeqn)\) rule for applying waveforms to primitives.
These rules are illustrated in \cref{fig:waveform-rules}.

\begin{figure}
    \centering
    \(
    \iltikzfig{circuits/examples/shortcuts/waveform-primitive-lhs}
    \reduction
    \iltikzfig{circuits/examples/shortcuts/waveform-primitive-rhs}
    \)
    \quad
    \(
    \iltikzfig{circuits/examples/shortcuts/waveform-fork-lhs}
    \reduction
    \iltikzfig{circuits/examples/shortcuts/waveform-fork-rhs}
    \)
    \quad
    \(
    \iltikzfig{circuits/examples/shortcuts/waveform-join-lhs}
    \reduction
    \iltikzfig{circuits/examples/shortcuts/waveform-fork-rhs}
    \)
    \caption{Generalisation of primitive rules to act on infinite waveforms}
    \label{fig:waveform-rules}
\end{figure}

This is not the only way we can partially evaluate with some inputs.
In some interpretations, it may be that we learn something about the output of
a primitive with only some of the inputs specified.

\begin{example}[Belnap shortcuts]
    In the Belnap interpretation \(\belnapinterpretation\), if one applies a
    false value to an \(\andgate\) gate then it will output false regardless of
    the other input.
    Similarly, if one applies a true value to an \(\orgate\) gate it will output
    true.

    Conversely, if one applies a true value to an \(\andgate\)

\end{example}

\begin{figure}
    \centering
    \(
    \iltikzfig{circuits/examples/shortcuts/and-f-infinite-lhs}
    \reduction
    \iltikzfig{circuits/examples/shortcuts/and-f-infinite-rhs}
    \)
    \,\,
    \(
    \iltikzfig{circuits/examples/shortcuts/or-t-infinite-lhs}
    \reduction
    \iltikzfig{circuits/examples/shortcuts/or-t-infinite-rhs}
    \)
    \,\,
    \(
    \iltikzfig{circuits/examples/shortcuts/and-t-infinite-lhs}
    \reduction
    \iltikzfig{circuits/examples/shortcuts/and-t-infinite-rhs}
    \)
    \,\,
    \(
    \iltikzfig{circuits/examples/shortcuts/or-f-infinite-lhs}
    \reduction
    \iltikzfig{circuits/examples/shortcuts/or-f-infinite-rhs}
    \)
    \\[0.5em]
    \(
    \iltikzfig{circuits/examples/shortcuts/and-f-comm-infinite-lhs}
    \reduction
    \iltikzfig{circuits/examples/shortcuts/and-f-infinite-rhs}
    \)
    \,\,
    \(
    \iltikzfig{circuits/examples/shortcuts/or-t-comm-infinite-lhs}
    \reduction
    \iltikzfig{circuits/examples/shortcuts/or-t-infinite-rhs}
    \)
    \,\,
    \(
    \iltikzfig{circuits/examples/shortcuts/and-t-comm-infinite-lhs}
    \reduction
    \iltikzfig{circuits/examples/shortcuts/and-t-infinite-rhs}
    \)
    \,\,
    \(
    \iltikzfig{circuits/examples/shortcuts/or-f-comm-infinite-lhs}
    \reduction
    \iltikzfig{circuits/examples/shortcuts/or-f-infinite-rhs}
    \)
    \caption{Belnap shortcut rules for waveforms}
    \label{fig:shortcut-waveform-rules}
\end{figure}

\begin{example}[Control switches]
    Recall that a \emph{multiplexer} is a circuit component constructed as \(
    \iltikzfig{circuits/components/gates/mux}
    \coloneqq
    \iltikzfig{circuits/components/gates/mux-construction}
    \).
    The first input is a \emph{control} which specifies which of the two other
    inputs is output.
    It is often the case that these control signals will be fixed for long
    periods of time; perhaps they specify some sort of global circuit
    configuration.

    Consider the circuit \(
    \iltikzfig{circuits/examples/control/circuit}
    \), in which the control signal to the multiplexer determines which of two
    subcircuits will become the output.
    We will assume that the control signal is held at false; to use the shortcut
    rules above we expand the definition of the multiplexer so the circuit is \(
    \iltikzfig{circuits/examples/control/circuit-expanded}
    \) and reduce as follows.
    \begin{gather*}
        \iltikzfig{circuits/examples/control/circuit-applied}
        \reduction
        \iltikzfig{circuits/examples/control/circuit-applied-1}
        \reduction
        \iltikzfig{circuits/examples/control/circuit-applied-2}
        \reduction
        \\[1em]
        \iltikzfig{circuits/examples/control/circuit-applied-3}
        \reduction
        \iltikzfig{circuits/examples/control/circuit-applied-4}
    \end{gather*}
\end{example}

\subsection{Shortcut rules}

The reductions defined as part of the operational semantics assume that all the
components in the circuits are fully specified: to apply the \((\gateeqn)\)
reduction we need to know all the inputs.
However, when performing partial evaluation this might not be case; we may have
left some as variables to be specified later.
Fortunately, it is not always necessary to consider all of the inputs to a
primitive in order to determine its behaviour.
For example, for an \(\andgate\) gate, if either of its inputs are false then
the output must also be false, and if either of its inputs are true then the
output is the other input.

Exploiting this, we can define some new `shortcut' reductions shown in
\cref{fig:shortcuts}.
These rules are only sound when the other inputs are also (unreduced)
instantaneous values, as the value will produce \(\bot\) on subsequent cycles,
violating the soundness of the shortcut.

These rule are redundant in a setting where we can fully reduce everything, as
eventually the \((\gateeqn)\) rule would be applied to the \(\andgate\) gate and
the same result found.
However, for partial evaluation they are a crucial way of propagating what
inputs we do have across a circuit.

\begin{figure}
    \centering
    \(
    \iltikzfig{circuits/examples/shortcuts/and-f-instant-lhs}
    \reduction
    \iltikzfig{circuits/examples/shortcuts/and-f-instant-rhs}
    \)
    \quad
    \(
    \iltikzfig{circuits/examples/shortcuts/or-t-instant-lhs}
    \reduction
    \iltikzfig{circuits/examples/shortcuts/or-t-instant-rhs}
    \)
    \quad
    \(
    \iltikzfig{circuits/examples/shortcuts/and-t-instant-lhs}
    \reduction
    \iltikzfig{circuits/examples/shortcuts/and-t-instant-rhs}
    \)
    \quad
    \(
    \iltikzfig{circuits/examples/shortcuts/or-f-instant-lhs}
    \reduction
    \iltikzfig{circuits/examples/shortcuts/or-f-instant-rhs}
    \)
    \\[1em]
    \(
    \iltikzfig{circuits/examples/shortcuts/and-f-infinite-lhs}
    \reduction
    \iltikzfig{circuits/examples/shortcuts/and-f-infinite-rhs}
    \)
    \quad
    \(
    \iltikzfig{circuits/examples/shortcuts/or-t-infinite-lhs}
    \reduction
    \iltikzfig{circuits/examples/shortcuts/or-t-infinite-rhs}
    \)
    \quad
    \(
    \iltikzfig{circuits/examples/shortcuts/and-t-infinite-lhs}
    \reduction
    \iltikzfig{circuits/examples/shortcuts/and-t-infinite-rhs}
    \)
    \quad
    \(
    \iltikzfig{circuits/examples/shortcuts/or-f-infinite-lhs}
    \reduction
    \iltikzfig{circuits/examples/shortcuts/or-f-infinite-rhs}
    \)
    \caption{Examples of some derivable `shortcut rules'}
    \label{fig:shortcuts}
\end{figure}

\begin{example}[Blocking boxes]\label{ex:blocking-boxes}
    Consider the circuit \(
    \iltikzfig{circuits/examples/blocking/circuit}
    \), which contains a `blackbox' combinational circuit \(
    \iltikzfig{strings/category/f}[box=f, colour=comb]
    \) with unknown behaviour.

    Even though we cannot directly reduce the blackbox, if we set the first
    input to false and use the shortcut rule above, we can still produce an
    output value.
    \[
        \iltikzfig{circuits/examples/blocking/applied-false}
        \reduction
        \iltikzfig{circuits/examples/blocking/streamed-false}
        \reduction
        \iltikzfig{circuits/examples/blocking/reduced}
    \]
\end{example}

As well as removing redundant blackboxes, judicious use of shortcut
reductions can dramatically reduce the reductions needed to get the outputs of a
circuit.


\begin{example}[Protocols]
    It may be the case that the inputs to a circuit are known to follow some
    sort of protocol.
    By including preconditions in our equational reasoning, we can observe
    more optimisation potential.
    We write \(=_{\mathsf{A} \in X}\) when we assume the inputs along wire
    \(\mathsf{A}\) will only be members of set \(X \subseteq \valuetuple{m}\).

    For example, \(
    \iltikzfig{circuits/examples/protocol/protocol-rule-lhs}
    =_{\mathsf{A}\in\{\mathsf{t},\mathsf{f}\}}
    \iltikzfig{circuits/examples/protocol/protocol-rule-rhs}
    \), as at least one true values will be applied to the \(\orgate\).
    Consider the larger circuit below and assume that the first two
    inputs can only ever be inverses; it can be shown that this circuit
    exhibits combinational behaviour.
    \begin{gather*}
        \iltikzfig{circuits/examples/protocol/protocol-circuit}
        \Rightarrow
        \iltikzfig{circuits/examples/protocol/circuit-with-protocol}
        =_{\mathsf{A}\in\{\mathsf{t},\mathsf{f}\}}
        \iltikzfig{circuits/examples/protocol/circuit-with-protocol-1}
        =
        \iltikzfig{circuits/examples/protocol/circuit-with-protocol-3}
    \end{gather*}
\end{example}

Since equations can be freely applied without being concerned with the context,
partial evaluation could be automated by applying as many reductions as
possible.