\section{Partial evaluation}

Partial evaluation~\cite{jones1996introduction} is a paradigm used in software
optimisation in which programs are `evaluated as much as possible' while only
some of the inputs are specified.
For example, it may be the case that a particular input to a program is fixed
for long periods of time; using partial evaluation, a program specialised for
this input can be defined.
This program might run significantly faster than the original.

There has been work into partial evaluation for hardware, such as constant
propagation~\cite{singh1996expressing,singh1999partial} and
unfolding~\cite{thompson2006bitlevel}.
However, this has been relatively informal, and can be made rigorous using the
categorical framework.

\subsection{Shortcut rules}

The reductions defined as part of the operational semantics assume that all the
components in the circuits are fully specified: to apply the \((\gateeqn)\)
reduction we need to know all the inputs.
However, when performing partial evaluation this might not be case; we may have
left some as variables to be specified later.
Fortunately, it is not always necessary to consider all of the inputs to a
primitive in order to determine its behaviour.

\begin{example}[Shortcuts for \(\andgate\)]
    A well-known example of a shortcut rule is that when one input to an
    \(\andgate\) gate is false, then the output is always false regardless of
    the other input.
\end{example}


\begin{example}[Blocking boxes]\label{ex:blocking-boxes}
    Consider the circuit \(
    \iltikzfig{circuits/examples/blocking/circuit}
    \), which contains a `blackbox' combinational circuit \(
    \iltikzfig{strings/category/f}[box=f, colour=comb]
    \) with unknown behaviour.

    Even though we cannot directly reduce the blackbox, if we set the first
    input to false and use the shortcut rule above, we can still produce an
    output value.
    \[
        \iltikzfig{circuits/examples/blocking/applied-false}
        \reduction
        \iltikzfig{circuits/examples/blocking/streamed-false}
        \reduction
        \iltikzfig{circuits/examples/blocking/reduced}
    \]
\end{example}

Although the blackbox is blocking, we do know how the \(\andgate\) gate
works; in particular we know that if either of its inputs are false then
the output must also be false.
Exploiting this, we can define a new `shortcut' reduction
\(
\iltikzfig{circuits/examples/blocking/rule-lhs}
\reduction
\iltikzfig{circuits/examples/blocking/rule-rhs}
\), which is clearly sound for the reasons detailed above.
If we could fully reduce the blackbox, this rule would be redundant, as
eventually the \((\gateeqn)\) rule would be applied to the \(\andgate\)
gate and the same result found.
In this context it is useful though as it means we can completely ignore the
second argument to the \(\andgate\) gate and eliminate the unknown blackbox.



As well as removing redundant blackboxes, judicious use of shortcut
reductions could dramatically reduce the number of reductions needed to
get the outputs of a circuit.

The shortcut rule defined above only
which usually only arise after streaming has been applied.
With some

\begin{example}[Control switches]
    Recall that a \emph{multiplexer} is constructed as \(
    \iltikzfig{circuits/components/gates/mux}
    \coloneqq
    \iltikzfig{circuits/components/gates/mux-construction}
    \).

    Among other examples, this could come in useful when considering
    multiplexers with known
\end{example}

\begin{remark}
    These sort of `annihilator' rules in which one input to a primitive forces
    the output to a certain value are not the only shortcut rule.
    There are also `identifier' rules in which one input causes the primitive to
    act as an identity on the remaining inputs; a good example is when one of
    the inputs to an \(\andgate\) gate is set to true.
    These rules could also be helpful for partial evaluation purposes.
\end{remark}


\begin{example}[Protocols]
    It may be the case that the inputs to a circuit are known to follow some
    sort of protocol.
    By including preconditions in our equational reasoning, we can observe
    more optimisation potential.
    We write \(=_{\mathsf{A} \in X}\) when we assume the inputs along wire
    \(\mathsf{A}\) will only be members of set \(X \subseteq \valuetuple{m}\).

    For example, \(
    \iltikzfig{circuits/examples/protocol/protocol-rule-lhs}
    =_{\mathsf{A}\in\{\mathsf{t},\mathsf{f}\}}
    \iltikzfig{circuits/examples/protocol/protocol-rule-rhs}
    \), as at least one true values will be applied to the \(\orgate\).
    Consider the larger circuit below and assume that the first two
    inputs can only ever be inverses; it can be shown that this circuit
    exhibits combinational behaviour.
    \begin{gather*}
        \iltikzfig{circuits/examples/protocol/protocol-circuit}
        \Rightarrow
        \iltikzfig{circuits/examples/protocol/circuit-with-protocol}
        =_{\mathsf{A}\in\{\mathsf{t},\mathsf{f}\}}
        \iltikzfig{circuits/examples/protocol/circuit-with-protocol-1}
        =
        \iltikzfig{circuits/examples/protocol/circuit-with-protocol-3}
    \end{gather*}
\end{example}

Since equations can be freely applied without being concerned with the context,
partial evaluation could be automated by applying as many reductions as
possible.