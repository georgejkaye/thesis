\section{Partial evaluation}

Partial evaluation~\cite{jones1996introduction} is a paradigm used in software
optimisation in which programs are `evaluated as much as possible' before all of
the inputs are specified.
For example, it may be the case that an input to a program is fixed; using
partial evaluation, a program specialised for this input can be defined.
This program might run significantly faster than the original.

There has been work into partial evaluation for hardware, such as constant
propagation~\cite{singh1996expressing,singh1999partial} and
unfolding~\cite{thompson2006bitlevel}.
However, this has been relatively informal, and can be made rigorous using the
categorical framework.
Many of the examples presented in this section will involve some sort of partial
evaluation.

\begin{example}[Protocols]
    It may be the case that the inputs to a circuit are known to follow some
    sort of protocol.
    By including preconditions in our equational reasoning, we can observe
    more optimisation potential.
    We write \(=_{\mathsf{A} \in X}\) when we assume the inputs along wire
    \(\mathsf{A}\) will only be members of set \(X \subseteq \valuetuple{m}\).
    For example, \(
    \iltikzfig{circuits/examples/protocol/protocol-rule-lhs}
    =_{\mathsf{A}\in\{\mathsf{t},\mathsf{f}\}}
    \iltikzfig{circuits/examples/protocol/protocol-rule-rhs}
    \), as at least one true values will be applied to the \(\orgate\).
    Consider the larger circuit below and assume that the first two
    inputs can only ever be inverses; it can be shown that this circuit
    exhibits combinational behaviour.
    \begin{gather*}
        \iltikzfig{circuits/examples/protocol/protocol-circuit}
        \Rightarrow
        \iltikzfig{circuits/examples/protocol/circuit-with-protocol}
        =_{\mathsf{A}\in\{\mathsf{t},\mathsf{f}\}}
        \iltikzfig{circuits/examples/protocol/circuit-with-protocol-1}
        =
        \iltikzfig{circuits/examples/protocol/circuit-with-protocol-3}
    \end{gather*}
\end{example}

Since equations can be freely applied without being concerned with the context,
partial evaluation could be automated by simply applying as many reductions as
possible.