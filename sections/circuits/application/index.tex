\chapter{Potential applications}

So far, we have been concerned largely with \emph{theoretical} concepts; we have
shown how the categorical framework of sequential digital circuits is rigorous
enough for mathematicians to use without breaking anything.

Unfortunately, if one tries to spout category theory at people in industry,
they'll often respond with blank looks before you are swiftly escorted from the
building and instructed never to return.
Instead, they are more concerned with the \emph{practical} benefits of the
framework: what can we do with it that cannot already be done?

Circuit design is already a very well-studied area and existing technologies are
incredibly successful: after all, digital circuits have become the backbone of
our society!
Our framework is therefore not meant to \emph{replace} the existing
technologies, but to \emph{complement} them by highlighting different
perspectives on reasoning with sequential digital circuits.
While the ideas provided in this section are certainly not industry-grade
applications just yet, they are intended to demonstrate the potential of what
the compositional theory can bring to the (already busy) table.

\section{Partial evaluation}

Partial evaluation~\cite{jones1996introduction} is a paradigm used in software
optimisation in which programs are `evaluated as much as possible' while only
some of the inputs are specified.
For example, it may be the case that a particular input to a program is fixed
for long periods of time; using partial evaluation, a program specialised for
this input can be defined.
This program might run significantly faster than the original.

There has been work into partial evaluation for hardware, such as constant
propagation~\cite{singh1996expressing,singh1999partial} and
unfolding~\cite{thompson2006bitlevel}.
However, this has been relatively informal, and can be made rigorous using the
categorical framework.

\subsection{Shortcut rules}

The reductions defined as part of the operational semantics assume that all the
components in the circuits are fully specified: to apply the \((\gateeqn)\)
reduction we need to know all the inputs.
However, when performing partial evaluation this might not be case; we may have
left some as variables to be specified later.
Fortunately, it is not always necessary to consider all of the inputs to a
primitive in order to determine its behaviour.
For example, for an \(\andgate\) gate, if either of its inputs are false then
the output must also be false, and if either of its inputs are true then the
output is the other input.

Exploiting this, we can define some new `shortcut' reductions shown in
\cref{fig:shortcuts}.
Two varieties are provided: one for when the inputs are instantaneous values
(although not necessarily fully reduced), and one for when the input is an
infinite waveform.
These rules are all sound for the reasons detailed above.

In a setting where we can fully reduce everything, these rule are redundant, as
eventually the \((\gateeqn)\) rule would be applied to the \(\andgate\)
gate and the same result found.
However, for partial evaluation they are a crucial way of propagating what
inputs we do have across a circuit.

\begin{figure}
    \centering
    \(
    \iltikzfig{circuits/examples/shortcuts/and-f-instant-lhs}
    \reduction
    \iltikzfig{circuits/examples/shortcuts/and-f-instant-rhs}
    \)
    \quad
    \(
    \iltikzfig{circuits/examples/shortcuts/or-t-instant-lhs}
    \reduction
    \iltikzfig{circuits/examples/shortcuts/or-t-instant-rhs}
    \)
    \quad
    \(
    \iltikzfig{circuits/examples/shortcuts/and-t-instant-lhs}
    \reduction
    \iltikzfig{circuits/examples/shortcuts/and-t-instant-rhs}
    \)
    \quad
    \(
    \iltikzfig{circuits/examples/shortcuts/or-f-instant-lhs}
    \reduction
    \iltikzfig{circuits/examples/shortcuts/or-f-instant-rhs}
    \)
    \caption{Examples of some derivable `shortcut rules'}
    \label{fig:shortcuts}
\end{figure}

\begin{example}[Blocking boxes]\label{ex:blocking-boxes}
    Consider the circuit \(
    \iltikzfig{circuits/examples/blocking/circuit}
    \), which contains a `blackbox' combinational circuit \(
    \iltikzfig{strings/category/f}[box=f, colour=comb]
    \) with unknown behaviour.

    Even though we cannot directly reduce the blackbox, if we set the first
    input to false and use the shortcut rule above, we can still produce an
    output value.
    \[
        \iltikzfig{circuits/examples/blocking/applied-false}
        \reduction
        \iltikzfig{circuits/examples/blocking/streamed-false}
        \reduction
        \iltikzfig{circuits/examples/blocking/reduced}
    \]
\end{example}



As well as removing redundant blackboxes, judicious use of shortcut
reductions could dramatically reduce the number of reductions needed to
get the outputs of a circuit.

The shortcut rule defined above only
which usually only arise after streaming has been applied.
With some

\begin{example}[Control switches]
    Recall that a \emph{multiplexer} is a circuit component constructed as \(
    \iltikzfig{circuits/components/gates/mux}
    \coloneqq
    \iltikzfig{circuits/components/gates/mux-construction}
    \).
    The first input is a \emph{control} which specifies which of the two other
    inputs is output.
    It is often the case that these control signals will be fixed for long
    periods of time; perhaps they specify some sort of global circuit
    configuration.
    \todo[inline]{Finish this off}
\end{example}

\begin{example}[Protocols]
    It may be the case that the inputs to a circuit are known to follow some
    sort of protocol.
    By including preconditions in our equational reasoning, we can observe
    more optimisation potential.
    We write \(=_{\mathsf{A} \in X}\) when we assume the inputs along wire
    \(\mathsf{A}\) will only be members of set \(X \subseteq \valuetuple{m}\).

    For example, \(
    \iltikzfig{circuits/examples/protocol/protocol-rule-lhs}
    =_{\mathsf{A}\in\{\mathsf{t},\mathsf{f}\}}
    \iltikzfig{circuits/examples/protocol/protocol-rule-rhs}
    \), as at least one true values will be applied to the \(\orgate\).
    Consider the larger circuit below and assume that the first two
    inputs can only ever be inverses; it can be shown that this circuit
    exhibits combinational behaviour.
    \begin{gather*}
        \iltikzfig{circuits/examples/protocol/protocol-circuit}
        \Rightarrow
        \iltikzfig{circuits/examples/protocol/circuit-with-protocol}
        =_{\mathsf{A}\in\{\mathsf{t},\mathsf{f}\}}
        \iltikzfig{circuits/examples/protocol/circuit-with-protocol-1}
        =
        \iltikzfig{circuits/examples/protocol/circuit-with-protocol-3}
    \end{gather*}
\end{example}

Since equations can be freely applied without being concerned with the context,
partial evaluation could be automated by applying as many reductions as
possible.
\section{Layers of abstraction}

\cite{lobski2022string}
\section{Refining circuits}\label{sec:refining}

A key part of circuit design comes in \emph{optimising circuits}: making them
run as fast as possible and reduce the \emph{clock cycle}.

\begin{example}[Retiming]
    The clock cycle of a circuit is determined by the longest paths between
    registers. Altering the paths between registers can be achieved using
    \emph{retiming}~\cite{leiserson1991retiming}: moving registers across gates.
    This is modelled by the streaming rule (\cref{lem:streaming});
    forward retiming (streaming left to right) is always possible
    but for backward retiming (streaming right to left), the value
    in the register must be in the image of the gates.
\end{example}

When reasoning equationally, the behaviour of the circuits on either side of the
equation must have exactly the same behaviour.
However, when reasoning with circuits it is sometimes the case that this is too
strict an assertion; we are looking for circuits that output the same outputs
but over a shorter period of time.
This means we may wish to use transformations that only `morally' preserve the
behaviour of a circuit.

\begin{definition}
    For two finite sequences \(
    \listlistvar{v},\listlistvar{w} \in (\valuetuple{m})^k
    \), we say that \(\listlistvar{w}\) is a \emph{stretching} of
    \(\listlistvar{v}\), written \(\listlistvar{v} \ll \listlistvar{w}\), if
    \(\listlistvar{w}\) contains the characters of \(\listlistvar{v}\) but
    possibly repeated or with additional \(\bot\) characters e.g.\ \(
    \belnaptrue\belnapfalse
    \ll
    \bot\bot\belnaptrue\belnaptrue\bot\belnapfalse
    \).
\end{definition}

\begin{definition}
    Given two sequential circuits \(
    \iltikzfig{strings/category/f}[box=f,colour=seq,dom=m,cod=n]
    \) and \(
    \iltikzfig{strings/category/f}[box=g,colour=seq,dom=m,cod=n]
    \) with \(c\) and \(c^\prime\) delay components respectively, we say that \(
    \iltikzfig{strings/category/f}[box=f,colour=seq,dom=m,cod=n]
    \) is \emph{logically equvialent} to \(
    \iltikzfig{strings/category/f}[box=g,colour=seq,dom=m,cod=n]
    \), written \(
    \iltikzfig{strings/category/f}[box=f,colour=seq,dom=m,cod=n]
    \ll
    \iltikzfig{strings/category/f}[box=g,colour=seq,dom=m,cod=n]
    \), if for all sequences \(\listlistvar{v},\listlistvar{w}\) produced by the
    productive operational semantics for inputs of length
    \(\mathsf{max}(c,c^\prime)\),  \(\listlistvar{v} \ll \listlistvar{w}\)
\end{definition}

Including this notion of equivalence in algebraic reasoning allows us to reason
with \emph{inequalities} as well as equalities.
This means that given a circuit, we can use equations as normal, determine that
one component

\begin{example}
    Take the circuit \(
    \iltikzfig{circuits/examples/refinement/circuit}
    \).
    For an input stream \(\sigma\), this circuit produces output stream
    \(
    \bot \land \sigma(0) \streamcons \sigma(1)
    \streamcons \sigma(2) \streamcons \dots
    \).
    By using equalities and logical eqivalence we can obtain a much simpler
    circuit:
    \[
        \iltikzfig{circuits/examples/refinement/circuit}
        =
        \iltikzfig{circuits/examples/refinement/circuit-1}
        \ll
        \iltikzfig{circuits/examples/refinement/circuit-2}
        =
        \iltikzfig{circuits/examples/refinement/circuit-3}
    \]
\end{example}

While this is a somewhat contrived toy example, it is possible that this
technique could be applied to actual circuit optimisation procedures.

\begin{example}[Pipelining]
    \emph{Pipelining}~\cite{parhi1999vlsi} is a technique in which more
    registers are inserted into a circuit to increase throughput.
    This can be emulated in the compositional framework by applying
    transformations locally to registers.
    Ordinarily, such transformations can obfuscate a circuit's behaviour since
    the state space dramatically changes.
    In the compositional model, the structure of the circuit is left relatively
    untouched so this is less of an issue.
\end{example}

Not all circuit transformations are for the purpose of improving performance.
Sometimes additional components must be bolted onto a circuit for \emph{testing}
purposes.

\begin{example}[Scan chains]
    A common way of testing circuits is by using a
    \emph{scan chain}~\cite{mourad2000principles}, a way of forcing the
    inputs to flipflops to test how specific states affect the outputs of the
    circuit.
    Adding a flipflop to a scan chain requires some extra inputs: the
    \(\mathsf{scan}_\mathsf{en}\) wire toggles if the flipflop operates in
    normal mode or if it takes \(\mathsf{scan}_\mathsf{in}\) as its value.
    \[
        \iltikzfig{circuits/examples/scan-chain/flipflop-before-chain}
        \xRightarrow{\text{scan}}
        \iltikzfig{circuits/examples/scan-chain/scan-chain}
    \]
\end{example}

One could factor in these transformations when designing the circuit, but this
can obfuscate the design of the actual logic.
Additionally, applying these transformations where the remaining part of the
circuit is \emph{not} combinational can be quite complex.
With the compositional approach the two tasks can be kept isolated by using
blackboxes, layered explanations, and graphical reasoning.
\section{Implementation}

Throughout this section we have discussed some potential applications for the
compositional theory for digital circuits.
However, the examples have been kept to relatively small toy examples for
ease of presentation and explanation.
For readers less convinced by theoretical results, this might not be enough; how
can the framework be adapted for real-life examples?

Because examples can quickly balloon in size, it becomes impractical to develop
and work through them by hand.
Instead, we turn to our old friend the computer and ask it to do the hard work
for us.
But how do we even communicate such things with a computer?
This will be answered in great detail in the next part of this thesis.




