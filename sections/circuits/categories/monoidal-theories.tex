\section{Monoidal theories}

So far we have only concerned ourselves with \emph{structural} equations:
equations that show how the same term can be constructed using different
combinations of composition, tensor, the identity and symmetry.
However, these only serve as a form of housekeeping: the true `computational
content' of processes comes from equations that show how the generators interact
with \emph{each other}.
These equations are provided by a \emph{monoidal theory}.

\begin{definition}[Monoidal theory]
    A \emph{monoidal theory} is a tuple \((\generators, \equations)\) where
    \(\generators\) is a set of generators and \(\equations\) is a set of
    equations.
\end{definition}

An equation \(f = g\) in a monoidal theory \emph{identifies} the two morphisms
\(f\) and \(g\).
When reasoning with a monoidal theory, we therefore need to work in a category
in which all of the equations are identified in this way.

\begin{definition}[Quotient category]
    Given a category \(\mathbf{C}\) and a set of equations \(\mce\) between
    morphisms in \(\mathbf{C}\) with the same source and target, the
    \emph{quotient category} \(\mathbf{C} / \mce\) is the category in which
    \(\ob{(\mathbf{C} / \mce)} := \ob{\mathbf{C}}\) and \(
            \mor{(\mathbf{C} / \mce)}{X}{Y}
            :=
            \mor{\mathbf{C}}{X}{Y} / \mce
    \): i.e.\ morphisms are the \emph{equivalence classes} of morphisms
    modulo \(\mce\).
\end{definition}

\begin{definition}
    Given a monoidal theory \((\generators, \equations)\), let
    \(\smc{\generators, \equations} := \smc{\generators} / \equations\).
\end{definition}

Note that when \(\equations\) is empty,
\(\smc{\generators, \emptyset} = \smc{\generators}\).

\subsection{Case study: commutative monoids}

Monoidal theories can be used to reason with many structures in mathematics.
One such structure is that of \emph{commutative monoids}.

\begin{definition}[Commutative monoids]\label{def:commutative-monoid}
    The monoidal theory of
    \emph{commutative monoids} is \(
        (\generators[\cmon], \equations[\cmon])
    \), where \(
        \generators[\cmon] := \{
            \iltikzfig{strings/structure/monoid/merge}[colour=white],
            \iltikzfig{strings/structure/monoid/init}[colour=white]
        \}
    \) representing the \emph{multiplication} and the \emph{unit} respectively,
    and \(\equations[\frob]\) comprises the equations
    \begin{center}
        \iltikzfig{strings/structure/monoid/unitality-l-lhs}
        \(=\)
        \iltikzfig{strings/structure/monoid/unitality-l-rhs}
        \quad
        \iltikzfig{strings/structure/monoid/associativity-lhs}
        \(=\)
        \iltikzfig{strings/structure/monoid/associativity-rhs}
        \quad
        \iltikzfig{strings/structure/monoid/commutativity-lhs}
        \(=\)
        \iltikzfig{strings/structure/monoid/commutativity-rhs}
    \end{center}
    We write \(\cmon := \smc{\generators[\cmon], \equations[\cmon]}\).
\end{definition}

The two generators of the theory are a \emph{multiplication} and a \emph{unit}.
The equations describe the properties of the multiplication: it is unital with
respect to the unit; it is associative; and it is commutative.
These equations could be described textually, but the string diagrams provide
intuitive visual interpretations; often it is insightful to reason
\emph{diagrammatically}.

\begin{example}[Right unitality]
    \(
        \iltikzfig{strings/structure/monoid/unitality-r-lhs}
        =
        \iltikzfig{strings/structure/monoid/unitality-r-rhs}
    \) is a valid equation in \(\cmon\).
\end{example}
\begin{proof}
    \(
        \iltikzfig{strings/structure/monoid/unitality-r-lhs}
        \eqaxioms[(\dagger)]
        \iltikzfig{strings/structure/monoid/right-unitality/step-1}
        =
        \iltikzfig{strings/structure/monoid/unitality-l-lhs}
        =
        \iltikzfig{strings/structure/monoid/unitality-l-rhs}
    \)
\end{proof}

Note that the first step \((\dagger)\) of the proof is performed solely by
deforming the string diagram; recall that as connectivity is preserved this is
still valid.
While this proof was fairly simple, the same principles apply to reasoning about
more complex terms.

\begin{example}
    \(
        \iltikzfig{strings/structure/monoid/example/step-0}
        =
        \iltikzfig{strings/structure/monoid/example/step-7}
    \)
    is a valid equation in \(\cmon\).
\end{example}
\begin{proof}
    \(
        \iltikzfig{strings/structure/monoid/example/step-0}
        =
        \iltikzfig{strings/structure/monoid/example/step-1}
        =
        \iltikzfig{strings/structure/monoid/example/step-2}
        =
        \iltikzfig{strings/structure/monoid/example/step-3}
        =
        \\[1em]
        \iltikzfig{strings/structure/monoid/example/step-4}
        =
        \iltikzfig{strings/structure/monoid/example/step-5}
        =
        \iltikzfig{strings/structure/monoid/example/step-6}
        =
        \iltikzfig{strings/structure/monoid/example/step-7}
    \)
\end{proof}