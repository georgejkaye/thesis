\section{Monoidal theories}

So far we have only concerned ourselves with \emph{structural} equations:
equations that show how the same term can be constructed using different
combinations of composition, tensor, the identity and symmetry.
However, these only serve as a form of housekeeping: the true `computational
content' of processes comes from equations that show how the generators interact
with \emph{each other}.
These equations are provided by a \emph{monoidal theory}.

\begin{definition}[Monoidal theory]
    A \emph{monoidal theory} is a tuple \((\generators, \equations)\) where
    \(\generators\) is a set of generators and \(\equations\) is a set of
    equations.
\end{definition}

An equation \(f = g\) in a monoidal theory \emph{identifies} the two morphisms
\(f\) and \(g\).
When reasoning with a monoidal theory, we therefore need to work in a category
in which all of the equations are identified in this way.

\begin{definition}[Quotient category]
    Given a category \(\mathbf{C}\) and a set of equations \(\mce\) between
    morphisms in \(\mathbf{C}\) with the same source and target, the
    \emph{quotient category} \(\mathbf{C} / \mce\) is the category in which
    \(\ob{(\mathbf{C} / \mce)} := \ob{\mathbf{C}}\) and \(
            \mor{(\mathbf{C} / \mce)}{X}{Y}
            :=
            \mor{\mathbf{C}}{X}{Y} / \mce
    \): i.e.\ morphisms are the \emph{equivalence classes} of morphisms
    modulo \(\mce\).
\end{definition}

\begin{definition}
    Given a monoidal theory \((\generators, \equations)\), let
    \(\smc{\generators, \equations} := \smc{\generators} / \equations\).
\end{definition}

Note that when \(\equations\) is empty,
\(\smc{\generators, \emptyset} = \smc{\generators}\).

\subsection{Case study: commutative monoids}

Monoidal theories can be used to reason with many structures in mathematics.
One such structure is that of \emph{commutative monoids}.

\begin{definition}[Commutative monoids]\label{def:commutative-monoid}
    The monoidal theory of
    \emph{commutative monoids} is \(
        (\generators[\cmon], \equations[\cmon])
    \), where \(
        \generators[\cmon] := \{
            \iltikzfig{strings/structure/monoid/merge}[colour=white],
            \iltikzfig{strings/structure/monoid/init}[colour=white]
        \}
    \) representing the \emph{multiplication} and the \emph{unit} respectively,
    and \(\equations[\frob]\) comprises the equations
    \begin{center}
        \iltikzfig{strings/structure/monoid/unitality-l-lhs}
        \(=\)
        \iltikzfig{strings/structure/monoid/unitality-l-rhs}
        \quad
        \iltikzfig{strings/structure/monoid/associativity-lhs}
        \(=\)
        \iltikzfig{strings/structure/monoid/associativity-rhs}
        \quad
        \iltikzfig{strings/structure/monoid/commutativity-lhs}
        \(=\)
        \iltikzfig{strings/structure/monoid/commutativity-rhs}
    \end{center}
    We write \(\cmon := \smc{\generators[\cmon], \equations[\cmon]}\).
\end{definition}

The two generators of the theory are a \emph{multiplication} and a \emph{unit}.
The equations describe the properties of the multiplication: it is unital with
respect to the unit; it is associative; and it is commutative.
These equations could be described textually, but the string diagrams provide
intuitive visual interpretations; often it is insightful to reason
\emph{diagrammatically}.

\begin{example}[Right unitality]
    \(
        \iltikzfig{strings/structure/monoid/unitality-r-lhs}
        =
        \iltikzfig{strings/structure/monoid/unitality-r-rhs}
    \) is a valid equation in \(\cmon\).
\end{example}
\begin{proof}
    \(
        \iltikzfig{strings/structure/monoid/unitality-r-lhs}
        \eqaxioms[(\dagger)]
        \iltikzfig{strings/structure/monoid/right-unitality/step-1}
        =
        \iltikzfig{strings/structure/monoid/unitality-l-lhs}
        =
        \iltikzfig{strings/structure/monoid/unitality-l-rhs}
    \)
\end{proof}

Note that the first step \((\dagger)\) of the proof is performed solely by
deforming the string diagram; as connectivity is preserved this is a valid
proof step.
Deforming the string diagram corresponds to implicitly applying equations
of symmetric monoidal categories.
%
\begin{gather*}
    \iltikzfig{strings/structure/monoid/right-unitality/lhs-unit}
    \eqaxioms[\text{unit of } \seq]
    \iltikzfig{strings/structure/monoid/right-unitality/step-1-1}
    \eqaxioms[\text{self-inverse}]
    \iltikzfig{strings/structure/monoid/right-unitality/step-1-2}
    \\[0.5em]
    \eqaxioms[\text{naturality of } \sigma]
    \iltikzfig{strings/structure/monoid/right-unitality/step-1-3}
    \eqaxioms[\text{unit of } \sigma]
    \iltikzfig{strings/structure/monoid/right-unitality/step-1-4}
    \eqaxioms[\text{unit of } \seq]
    \iltikzfig{strings/structure/monoid/right-unitality/step-1-5}
\end{gather*}
%
Already much more verbose than the simple deformation, this does not
even take into account the repeated applications of associativity of both
compositionality and tensor required if reasoning in the term language!
If we write \(
    \iltikzfig{strings/structure/monoid/merge}[colour=white]
\) as \(\mu\) and \(
    \iltikzfig{strings/structure/monoid/init}[colour=white]
\) as \(\eta\), then the proof on terms becomes:
\begin{align*}
    \id[n] \tensor \eta \seq \mu
    &=
    \id[n] \tensor (\eta \seq \id[n]) \seq \mu
    &
    \text{unit of } \seq
    \\
    &=
    (\id[n] \seq \id[n]) \tensor (\eta \seq \id[n]) \seq \mu
    &
    \text{unit of } \seq
    \\
    &=
    ((\id[n] \tensor \eta) \seq (\id[n] \tensor \id[n])) \seq \mu
    &
    \text{functoriality of } \tensor
    \\
    &=
    ((\id[n] \tensor \eta) \seq (\swap{n}{n} \seq \swap{n}{n})) \seq \mu
    &
    \sigma \text{ is self inverse}
    \\
    &=
    (((\id[n] \tensor \eta) \seq \swap{n}{n}) \seq \swap{n}{n}) \seq \mu
    &
    \text{associativity of } \seq
    \\
    &=
    ((\swap{\varepsilon}{n} \seq (\eta \tensor \id[n])) \seq \swap{n}{n}) \seq \mu
    &
    \text{naturality of } \sigma
    \\
    &=
    ((\id[n] \seq (\eta \tensor \id[n])) \seq \swap{n}{n}) \seq \mu
    &
    \text{unit of } \sigma
    \\
    &=
    ((\eta \tensor \id[n]) \seq \swap{n}{n}) \seq \mu
    &
    \text{unit of } \seq
    \\
    &=
    (\eta \tensor \id[n]) \seq (\swap{n}{n} \seq \mu)
    &
    \text{associativity of} \seq
    \\
    &=
    (\eta \tensor \id[n]) \seq \mu
    &
    \text{commutativity of } \mu
    \\
    &=
    \id[n]
    &
    \text{left unit of } \mu
\end{align*}

It should be clear that reasoning string diagrammatically makes things a lot
easier.
Not only is it less verbose, but the insight required to make a proof step is
often far easier to see in the string diagram notation.