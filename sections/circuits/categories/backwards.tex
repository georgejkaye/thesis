\section{Reversing the wires}

In string diagrams for symmetric monoidal categories, there is a strict notion
of causality: it is not possible to create a cycle from the output of a box to
its input, and outputs may only be joined to outputs.
This enforces an implicit \emph{left-to-right} directionality across the page.

However, this may not always be desirable: one may wish to model a feedback
loop, or perhaps enforce some condition by unifying two outputs.
To do this sort of thing, we must examine symmetric monoidal categories with
some extra structure.

\subsection{Symmetric traced monoidal categories}

First we consider removing the acyclicity condition.

\begin{definition}[Symmetric traced monoidal category \cite{joyal1996traced}]\label{def:stmc}
    A \emph{symmetric traced monoidal category} (STMC) is a symmetric monoidal
    category \(\mathbf{C}\) equipped with a family of functions \(
        \morph{
            \trace{X}[A][B]{-}
        }{
            \mathbf{C}(X \tensor A, X \tensor B)
        }{
            \mathbf{C}(A, B)
        }
    \) for any three objects \(A\), \(B\) and \(X\) subject to the following
    equations:
    \begin{itemize}
        \item \textbf{Tightening (naturality in \(A,B\))}\\\null\qquad
                    \(\trace{X}[A][D]{
                            \id[X] \tensor f \seq g \seq \id[X] \seq h
                        }
                        =
                        f \seq \trace{X}[B][C]{g} \seq h
                    \)
        \item \textbf{Sliding (naturality in \(X\))}\\\null\qquad
                    \(
                        \trace{X}[A][B]{f \seq g \tensor \id[B]}
                        =
                        \trace{Y}[A][B]{g \tensor \id[A] \seq f}
                    \)
        \item \textbf{Yanking}\\\null\qquad
                    \(
                        \trace{X}[X][X]{\swap{X}{X}} = \id[X]
                    \)
        \item \textbf{Vanishing}\\\null\qquad
                    \(
                        \trace{X}[A][B]{\trace{X}[Y \tensor A][Y \tensor B]{f}}
                        =
                        \trace{X \tensor Y}[A][B]{f}
                    \)
        \item \textbf{Superposing}\\\null\qquad
                    \(
                        \trace{X}[A \tensor C][B \tensor F]{f \tensor g}
                        = \trace{X}[A][B]{f} \tensor g
                    \)
    \end{itemize}
\end{definition}

What this means is that in a STMC, if we have a morphism
\(\morph{f}{X \tensor A}{X \tensor B}\) we also have a morphism \(
    \morph{\trace{X}[A][B]{f}}{A}{B}
\).

Traced categories were given the string diagrammatic treatment in
\cite{joyal1996traced}, in which the trace is depicted as a loop.
\begin{gather*}
    \trace{X}[A][B]{
        \iltikzfig{strings/category/f-2-2}[colour=white,box=f,dom1=X,dom2=A,cod1=X,cod2=B]
    }
    =
    \iltikzfig{strings/traced/trace-rhs}[colour=white,box=f,dom=A,cod=B,trace=X]
\end{gather*}
The equations of STMCs then have pleasant graphical interpretations, shown in
\cref{fig:stmc-equations}.
Note that as with the equations of SMCs, these equations amount to deforming the
diagram without altering connections between boxes, so do not need to be
applied explicitly when performing equational reasoning.

Usually we will omit the subscripts from \(\trace{X}[A][B]{f}\) and write it
simply as \(\trace{X}{f}\).

\begin{figure}
    \centering
    \begin{tabular}{c}
        \(\trace{X}[A][D]{
            \id[X] \tensor f \seq g \seq \id[X] \seq h
        }
        =
        f \seq \trace{X}[B][C]{g} \seq h
        \)
        \\[1em]
        \(
        \iltikzfig{strings/traced/naturality-lhs}[colour=white,box1=f,box2=g,box3=h,dom=A,cod=D]
        =
        \iltikzfig{strings/traced/naturality-rhs}[colour=white,box1=f,box2=g,box3=h,dom=A,cod=D]
        \)
    \end{tabular}
    \\[1em]
    \rule[1em]{\textwidth}{0.1mm}
    \\[0.1em]
    \begin{tabular}{ccc}
        \(
        \trace{X}[A][B]{f \seq g \tensor \id[B]}
        =
        \trace{Y}[A][B]{g \tensor \id[A] \seq f}
        \)
         &  &
        \(
        \trace{X}[X][X]{\swap{X}{X}} = \id[X]
        \)
        \\[1em]
        \(
        \iltikzfig{strings/traced/sliding-lhs}[colour=white,box1=f,box2=g,dom=A,cod=B]
        =
        \iltikzfig{strings/traced/sliding-rhs}[colour=white,box1=f,box2=g,dom=A,cod=B]
        \)
         &  &
        \(
        \iltikzfig{strings/traced/yanking-lhs}[obj=X,colour=white]
        =
        \iltikzfig{strings/category/identity}[obj=X,colour=white]
        \)
    \end{tabular}
    \\[1em]
    \rule[1em]{\textwidth}{0.1mm}
    \\[0.1em]
    \begin{tabular}{ccc}
        \(
        \trace{X}[A][B]{\trace{X}[Y \tensor A][Y \tensor B]{f}}
        =
        \trace{X \tensor Y}[A][B]{f}
        \)
         &  &
        \(
        \trace{X}[A \tensor C][B \tensor F]{f \tensor g}
        = \trace{X}[A][B]{f} \tensor g
        \)
        \\[1em]
        \(
        \iltikzfig{strings/traced/vanishing-lhs}[colour=white,box=f,dom=A,cod=B,trace1=X,trace2=Y]
        =
        \iltikzfig{strings/traced/vanishing-rhs}[colour=white,box=f,dom=A,cod=B,trace1=X,trace2=Y]
        \)
         &  &
        \(
        \iltikzfig{strings/traced/superposing-lhs}[colour=white,box1=f,box2=g,dom1=A,cod1=B,dom2=C,cod2=D,trace=X]
        =
        \iltikzfig{strings/traced/superposing-rhs}[colour=white,box1=f,box2=g,dom1=A,cod1=B,dom2=C,cod2=D,trace=X]
        \)
    \end{tabular}
    \caption{Equations of STMCs in string diagram notation}
    \label{fig:stmc-equations}
\end{figure}

\subsection{Compact closed categories}

In a traced category, while wires can flow backwards across the page, regular
left-to-right flow must still be in effect at a wire's \emph{endpoints}.
This means that outputs will always be connected to inputs.
We will now consider another setting in which this is not the case.

\begin{definition}[Compact closed category]
    A \emph{compact closed category} (CCC) is a symmetric monoidal category in
    which every object \(X\) has a \emph{dual} \(\dual{X}\) equipped with
    morphisms called the \emph{unit} \(
        \morph{\ccunit[A]}{I}{\dual{A} \tensor A}
    \) (`cup') and the \emph{counit} \(
        \morph{\cccounit[A]}{A \tensor \dual{A}}{I}
    \) (`cap') satisfying the following \emph{snake equations}:
\end{definition}