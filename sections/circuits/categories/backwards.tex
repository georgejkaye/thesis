\section{Reversing the wires}

In string diagrams for symmetric monoidal categories, there is a strict notion
of causality: it is not possible to create a cycle from the output of a box to
its input, and outputs may only be joined to outputs.
This enforces an implicit \emph{left-to-right} directionality across the page.

However, this may not always be desirable: one may wish to model a feedback
loop, or perhaps enforce some condition by unifying two outputs.
To do this sort of thing, we must examine symmetric monoidal categories with
some extra structure.

\subsection{Symmetric traced monoidal categories}

First we consider removing the acyclicity condition.

\begin{definition}[Symmetric traced monoidal category \cite{joyal1996traced}]\label{def:stmc}
    A \emph{symmetric traced monoidal category} (STMC) is a symmetric monoidal
    category \(\mathbf{C}\) equipped with a family of functions \(
        \morph{
            \trace{X}[A][B]{-}
        }{
            \mathbf{C}(X \tensor A, X \tensor B)
        }{
            \mathbf{C}(A, B)
        }
    \) for any three objects \(A\), \(B\) and \(X\) subject to the following
    equations:
    \begin{itemize}
        \item \textbf{Tightening (naturality in \(A,B\))}\\\null\qquad
                    \(\trace{X}[A][D]{
                            \id[X] \tensor f \seq g \seq \id[X] \seq h
                        }
                        =
                        f \seq \trace{X}[B][C]{g} \seq h
                    \)
        \item \textbf{Sliding (naturality in \(X\))}\\\null\qquad
                    \(
                        \trace{X}[A][B]{f \seq g \tensor \id[B]}
                        =
                        \trace{Y}[A][B]{g \tensor \id[A] \seq f}
                    \)
        \item \textbf{Yanking}\\\null\qquad
                    \(
                        \trace{X}[X][X]{\swap{X}{X}} = \id[X]
                    \)
        \item \textbf{Vanishing}\\\null\qquad
                    \(
                        \trace{X}[A][B]{\trace{X}[Y \tensor A][Y \tensor B]{f}}
                        =
                        \trace{X \tensor Y}[A][B]{f}
                    \)
        \item \textbf{Superposing}\\\null\qquad
                    \(
                        \trace{X}[A \tensor C][B \tensor F]{f \tensor g}
                        = \trace{X}[A][B]{f} \tensor g
                    \)
    \end{itemize}
\end{definition}

What this means is that in a STMC, if we have a morphism
\(\morph{f}{X \tensor A}{X \tensor B}\) we also have a morphism \(
    \morph{\trace{X}[A][B]{f}}{A}{B}
\).

Traced categories were given the string diagrammatic treatment in
\cite{joyal1996traced}, in which the trace is depicted as a loop.
\begin{gather*}
    \trace{X}[A][B]{
        \iltikzfig{strings/category/f-2-2}[colour=white,box=f,dom1=X,dom2=A,cod1=X,cod2=B]
    }
    =
    \iltikzfig{strings/traced/trace-rhs}[colour=white,box=f,dom=A,cod=B,trace=X]
\end{gather*}
The equations of STMCs then have pleasant graphical interpretations, shown in
\cref{fig:stmc-equations}.
Note that as with the equations of SMCs, these equations amount to deforming the
diagram without altering connections between boxes, so do not need to be
applied explicitly when performing equational reasoning.

Usually we will omit the subscripts from \(\trace{X}[A][B]{f}\) and write it
simply as \(\trace{X}{f}\).

\begin{figure}
    \centering
    \begin{tabular}{ccc}
        \(\trace{X}[A][D]{
            \id[X] \tensor f \seq g \seq \id[X] \seq h
        }
        \)
         &
        \(=\)
         &
        \(f \seq \trace{X}[B][C]{g} \seq h\)
        \\[1em]
        \(
        \iltikzfig{strings/traced/naturality-lhs}[colour=white,box1=f,box2=g,box3=h,dom=A,cod=D]
        \)
         &
        \(=\)
         &
        \(
        \iltikzfig{strings/traced/naturality-rhs}[colour=white,box1=f,box2=g,box3=h,dom=A,cod=D]
        \)
    \end{tabular}
    \\[1em]
    \rule[1em]{\textwidth}{0.1mm}
    \\[0.1em]
    \begin{tabular}{ccccccc}
        \(
        \trace{X}[A][B]{f \seq g \tensor \id[B]}
        \)
         &
        \(=\)
         &
        \(
        \trace{Y}[A][B]{g \tensor \id[A] \seq f}
        \)
         &   &
        \(
        \trace{X}[X][X]{\swap{X}{X}}
        \)
         &
        \(=\)
         &
        \(\id[X]\)
        \\[1em]
        \(
        \iltikzfig{strings/traced/sliding-lhs}[colour=white,box1=f,box2=g,dom=A,cod=B]
        \)
         &
        \(=\)
         &
        \(\iltikzfig{strings/traced/sliding-rhs}[colour=white,box1=f,box2=g,dom=A,cod=B]\)
         &   &
        \(
        \iltikzfig{strings/traced/yanking-lhs}[obj=X,colour=white]
        \)
         &
        \(=\)
         &
        \(
        \iltikzfig{strings/category/identity}[obj=X,colour=white]
        \)
    \end{tabular}
    \\[1em]
    \rule[1em]{\textwidth}{0.1mm}
    \\[0.1em]
    \begin{tabular}{ccccccc}
        \(
        \trace{X}[A][B]{\trace{Y}[X \tensor A][X \tensor B]{f}}
        \)
         &
        \(=\)
         &
        \(
        \trace{X \tensor Y}[A][B]{f}
        \)
         &   &
        \(
        \trace{X}[A \tensor C][B \tensor D]{f \tensor g}
        \)
         &
        \(=\)
         &
        \( \trace{X}[A][B]{f} \tensor g\)
        \\[1em]
        \iltikzfig{strings/traced/vanishing-lhs}[colour=white,box=f,dom=A,cod=B,trace1=X,trace2=Y]
         &
        \(=\)
         &
        \iltikzfig{strings/traced/vanishing-rhs}[colour=white,box=f,dom=A,cod=B,trace1=X,trace2=Y]
         &   &
        \iltikzfig{strings/traced/superposing-lhs}[colour=white,box1=f,box2=g,dom1=A,cod1=B,dom2=C,cod2=D,trace=X]
         &
        \(=\)
         &
        \iltikzfig{strings/traced/superposing-rhs}[colour=white,box1=f,box2=g,dom1=A,cod1=B,dom2=C,cod2=D,trace=X]
    \end{tabular}
    \caption{Equations of STMCs in string diagram notation}
    \label{fig:stmc-equations}
\end{figure}

\begin{example}
    The classic example of a symmetric traced monoidal category is the category
    \(\finvectk[k]\) in which the objects are finite dimensional vector spaces
    over a field \(k\) and the morphisms are linear maps.
    The monoidal product is the usual tensor product of vector spaces and the
    trace by an operation known as the `partial trace'.
\end{example}

Traces are often used to model \emph{fixpoints}.


\subsection{Compact closed categories}

In a traced category, while wires can flow backwards across the page, regular
left-to-right flow must still be in effect at a wire's \emph{endpoints}.
This means that outputs will always be connected to inputs.
We will now consider another setting in which this is not the case.

\begin{definition}[Compact closed category]
    A \emph{compact closed category} (CCC) is a symmetric monoidal category in
    which every object \(X\) has a \emph{dual} \(\dual{X}\) equipped with
    morphisms called the \emph{unit} \(
        \morph{\ccunit[A]}{I}{\dual{A} \tensor A}
    \) and the \emph{counit} \(
        \morph{\cccounit[A]}{A \tensor \dual{A}}{I}
    \) satisfying the following \emph{snake equations}:
    \begin{center}
        \includestandalone{figures/category/coherences/snake-1}
        \quad
        \includestandalone{figures/category/coherences/snake-2}
    \end{center}
\end{definition}

In string diagrams, the dual is drawn as a wire flowing from right-to-left
instead of left-to-right; when labelling wires with objects we will drop the
notation for duals and recover the information solely from directionality of the
wires.
The unit and counit `bend' wires: the unit is drawn as \(
    \iltikzfig{strings/compact-closed/cup-self-dual}[obj=A,colour=white]
\) and the counit as \(
    \iltikzfig{strings/compact-closed/cap-self-dual}[obj=A,colour=white]
\).
As a result of this units and counits are often referred to as \emph{cups} and
\emph{caps} respectively.
The snake equations are depicted as in \cref{fig:ccc-equations}, which
sheds some light on their name!

\begin{figure}
    \centering
    \iltikzfig{strings/compact-closed/snake-1}
    \(=\)
    \iltikzfig{strings/category/identity}[colour=white]
    \quad
    \iltikzfig{strings/compact-closed/snake-2}
    \(=\)
    \iltikzfig{strings/category/identity}[colour=white]
    \caption{Equations of CCCs string diagram notation}
    \label{fig:ccc-equations}
\end{figure}

There are some cases where the actual directionality of wires is irrelevant;
we only care about the ability to bend wires.

\begin{definition}[Self-dual compact closed category]
    A compact closed category is \emph{self-dual} if for every object \(A\),
    \(\dual{A} := A\).
\end{definition}

In a self-dual compact closed category, we do not need to label wires with
arrows.

\subsection{Traced vs compact closed}

The graphical notation is particularly suggestive of links between the
trace, the cup and the cap.
This is no coincidence, as there is a well-known result that allows one to
construct a trace in a compact closed setting.

\begin{proposition}[Canonical trace (\cite{joyal1996traced}, Prop. 3.1)]
    \label{prop:canonical-trace}
    Any compact closed category has a trace called the \emph{canonical trace},
    defined as \[
        \trace{X}{f}
        :=
        \ccunit[X] \tensor \id[A]
        \seq
        \id[\dual{X}] \tensor f
        \seq
        (\swap{\dual{X}}{X} \seq \cccounit[X]) \tensor \id[B]
    \]
    \[
        \iltikzfig{strings/compact-closed/canonical-trace}[colour=white,box=f,dom=A,cod=B,trace=X]
    \]
\end{proposition}

In this thesis we are primarily concerned with traced categories, but a plethora
of related work is based in the compact closed realm.
The canonical trace allows us to adapt existing results for our setting as well.

\begin{remark}
    It is also possible to consider the other direction: using the
    \emph{Int-construction}~\cite{joyal1996traced}, given any STMC
    \(\mcc\) one can construct a compact closed category \(\mathsf{Int}(\mcc)\).
    However, this will not be of relevance to us.
\end{remark}