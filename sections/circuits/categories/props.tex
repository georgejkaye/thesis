\subsection{PROPs}

We have described the categorical structure required to interpret five of the
constructors listed in \cref{def:terms}; only the `bundlers'
\(\bundle{\overline{m}}{\overline{n}}\) remain.
However, to give these meaning we must consider a slightly more specialised
setting; a subclass of symmetric monoidal categories known as \emph{PROPs}.

\begin{definition}[Multi-sorted PROP~\cite{maclane1965categorical}]
    Given a set of \emph{sorts} \(\mcc\), a \(\mcc\)-sorted \emph{PROP}
    (category of \emph{PRO}ducts and \emph{P}ermutations) is a strict symmetric
    monoidal category with the objects as words in \(\freemon{\mcc}\) and tensor
    product as concatenation.
\end{definition}

Note that this definition means that the empty word \(\varepsilon\) is the unit
object in any \(\mcc\)-sorted PROP.
As the domain and codomains of generators in \(\generators\) are words of
natural numbers, it is clear to see how a multi-sorted PROP can be used for our
scenario.

\begin{definition}\label{def:freely-generated-prop}
    Given a set of generators \(\generators\), let \(\smc{\generators}\) be the
    \(\natplus\)-sorted PROP where \(
        \smc{\generators}(\overline{m}, \overline{n})
    \) is the set of \(\Sigma\)-terms of type \(\overline{m} \to \overline{n}\).
\end{definition}

The symmetric monoidal structure of \(\smc{\generators}\) means that
\(\generators\)-terms are now subject to the equations listed in
\cref{fig:structural-equations-strings}

\begin{figure}
    \centering
    \begin{gather*}
        \iltikzfig{strings/category/identity-l-lhs}[f][white][\overline{m}][\overline{n}]
        =
        \iltikzfig{strings/category/f}[f][white][\overline{m}][\overline{n}]
        \quad
        \iltikzfig{strings/category/identity-r-lhs}[f][white][\overline{m}][\overline{n}]
        =
        \iltikzfig{strings/category/f}[f][white]
        \quad
        \iltikzfig{strings/category/associativity-lhs}[f][g][h][white][\overline{m}][\overline{p}]
        =
        \iltikzfig{strings/category/associativity-rhs}[f][g][h][white][\overline{m}][\overline{p}]
        \\[0.5em]
        \iltikzfig{strings/monoidal/unit-l-lhs}[f][white][\overline{m}][\overline{n}]
        =
        \iltikzfig{strings/category/f}[f][white][\overline{m}][\overline{n}]
        \quad
        \iltikzfig{strings/monoidal/unit-r-lhs}[f][white][\overline{m}][\overline{n}]
        =
        \iltikzfig{strings/category/f}[f][white][\overline{m}][\overline{n}]
        \quad
        \iltikzfig{strings/monoidal/associativity-lhs}[f][g][h][white]
        =
        \iltikzfig{strings/monoidal/associativity-rhs}[f][g][h][white]
        \\[0.5em]
        \iltikzfig{strings/monoidal/identity-tensor-lhs}[white][\overline{m}][\overline{n}]
        =
        \iltikzfig{strings/category/identity}[white][\overline{mn}]
        \quad
        \iltikzfig{strings/monoidal/interchange-lhs}[f][g][h][k][white]
        =
        \iltikzfig{strings/monoidal/interchange-rhs}[f][g][h][k][white]
        \\[0.5em]
        \iltikzfig{strings/symmetric/naturality-lhs}[f][g][white][\overline{m}][\overline{n}][\overline{p}][\overline{q}]
        =
        \iltikzfig{strings/symmetric/naturality-rhs}[f][g][white][\overline{m}][\overline{n}][\overline{p}][\overline{q}]
        \quad
        \iltikzfig{strings/symmetric/hexagon-lhs}[white][\overline{m}][\overline{n}][\overline{p}]
        =
        \iltikzfig{strings/symmetric/symmetry}[white][\overline{m}][\overline{np}]
        \\[0.5em]
        \iltikzfig{strings/symmetric/unit-l-lhs}[white][\epsilon][\overline{n}]
        =
        \iltikzfig{strings/category/identity}[white][\overline{n}]
        \quad
        \iltikzfig{strings/symmetric/unit-r-lhs}[white][\overline{m}][\epsilon]
        =
        \iltikzfig{strings/category/identity}[white][\overline{m}]
        \quad
        \iltikzfig{strings/symmetric/self-inverse-lhs}[white][\overline{m}][\overline{n}]
        =
        \iltikzfig{strings/category/identity}[white][\overline{mn}]
    \end{gather*}
    \caption{
        Equations of \(\smc{\Sigma}\)
    }
    \label{fig:structural-equations-strings}
\end{figure}