\section{Comparing categories}

Functors can be used to compare categories, and the notions of fullness and
faithfulness show how exactly two categories are related.
We will first consider \emph{subcategories}: a category with `some of the bits
taken out'.

\begin{definition}[Subcategory]
    Given a category \(\mcc\), a \emph{subcategory} of \(\mcc\) is a category
    \(\mcd\) with objects and morphisms subclasses of the objects and morphisms
    in \(\mcc\) subject to the following conditions:
    \begin{itemize}
        \item for every object \(A \in \mcd\), \(\id[A] \in \mor{\mcd}{A}{A}\);
        \item for every morphism \(f \in \mor{\mcd}{A}{B}\), \(A\) and \(B\) are
                in \(\mcd\); and
        \item for morphisms \(f \in \mor{\mcd}{A}{B}\) and
                \(g \in \mor{\mcd}{B}{C}\), \(g \circ f \in \mor{\mcd}{A}{C}\).
    \end{itemize}
\end{definition}

Given a category \(\mcc\) and a subcategory \(\mcd\), there is an obvious
induced functor \(\morph{S}{\mcd}{\mcc}\) mapping objects and morphisms
in \(\mcd\) to the same objects in \(\mcc\); this is called an
\emph{inclusion functor}.
An inclusion functor is clearly faithful, since there cannot be two morphisms in
the subcategory that map to the same morphism in the parent category.
Inclusion functors that are also \emph{full} are of particular interest.

\begin{definition}[Full subcategory]
    A subcategory \(\mcd\) is a \emph{full subcategory} if its inclusion functor
    \(\mcd \to \mcc\) is full and faithful; i.e.\ for all objects
    \(A, B \in \mcd\), the morphisms \(\mor{\mcd}{A}{B} = \mor{\mcc}{A}{B}\).
\end{definition}

\begin{example}
    \(\finset\) is a full subcategory of \(\set\), as every function between
    finite sets is a morphism in both \(\finset\) and \(\set\).
    \(\set\) is a subcategory of \(\rel\) as every function is a relation, but
    it is not a full subcategory because there are more relations \(A \tilde B\)
    than there are functions \(A \to B\).
\end{example}

Sometimes a category is not merely a subcategory of another, but the two
categories are actually (almost) the same!

\begin{definition}
    Two categories \(\mcc\) and \(\mcd\) are \emph{isomorphic} if there exist
    functors \(\morph{F}{\mcc}{\mcd}\) and \(\morph{G}{\mcd}{\mcc}\) such that
    \(G \circ F = \idf[\mcc]\) and \(F \circ G = \idf[\mcd]\).
\end{definition}

It can be inconvenient to construct the functors in both directions; fortunately
isomorphism can be shown by constructing just the functor in one direction, as
long as it has the required properties.

\begin{lemma}[\cite{maclane1978categories}]
    Two categories \(\mcc\) and \(\mcd\) are isomorphic if and only if there
    exists a fully faithful functor \(\morph{F}{\mcc}{\mcd}\) which is also
    bijective on objects.
\end{lemma}

\begin{remark}
    Usually, isomorphism of categories is too restrictive; often the weaker
    notion \emph{equivalence of categories} is used.
    However, in this thesis it turns out that all the results we need really are
    strong enough to be isomorphisms.
\end{remark}