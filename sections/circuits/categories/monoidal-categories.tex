\section{Monoidal categories}

We will now apply the concepts of functors and natural transformations in order
to interpret \emph{parallel} composition \(\tensor\).
A setting with both sequential and parallel composition can be modelled by a
\emph{monoidal category}.

\begin{definition}[Bifunctor]
    A \emph{bifunctor} is a functor with a product category as its domain, i.e.\
    a functor of the form \(\mcc \times \mcd \to \mce\).
\end{definition}

We adapt the notation of functorial boxes to show how bifunctors map from two
categories into one.
\[
    \iltikzfig{strings/category/functors/bif}[f][g][white][F][X][Y][Z][W]
\]
The diagrammatic notation suggests that a bifunctor is exactly what we need to
model parallel composition!

\begin{definition}[Monoidal category]
    \label{def:monoidal-category}
    A \emph{monoidal category} is a category \(\mcc\) equipped with a
    bifunctor \(\morph{{-} \tensor {=}}{\mcc \times \mcc}{\mcc}\) called the
    \emph{tensor product} and an additional object \(I\) called the
    \emph{monoidal unit},
    along with natural isomorphisms
    \begin{itemize}
        \item \(
            \associator{A}{B}{C}
            \colon
            A \tensor (B \tensor C)
            \cong
            (A \tensor B) \tensor C
            \) called the \emph{associator};
        \item \(
            \leftunitor{A}
            \colon
            I \tensor A
            \cong
            A
            \) called the \emph{left unitor}; and
        \item \(
            \rightunitor{A}
            \colon
            A \tensor I
            \cong
            A
            \) called the \emph{right unitor}
    \end{itemize}
    such that the \emph{pentagon} and the \emph{triangle} diagrams below
    commute:
    \begin{center}
        \includestandalone{figures/category/coherences/pentagon}

        \vspace{1em}

        \includestandalone{figures/category/coherences/triangle}
    \end{center}
\end{definition}

\begin{example}
    \(\set\) is a monoidal category, with the tensor product defined as the
    Cartesian product (\(A \tensor B := A \times B\)) and the unit as the
    singleton set (\(I := \mathbbm{1}\)).
\end{example}


We adopt the convention that \(f \tensor g \seq h \tensor k\) should be
bracketed as \((f \tensor g) \seq (h \tensor k)\), i.e.\ \(\tensor\) binds
more strongly than \(\seq\).
As with the case of regular categories, using a string diagrammatic notation
will make this distinction irrelevant.

The definition of monoidal category we have presented is quite general,
particularly with regards to the natural isomorphisms for unitors and
associators.
Most of the time, it is sufficient for these isomorphisms to hold `on the nose'.

\begin{definition}[Strict monoidal category]
    A monoidal category is \emph{strict} if \(\lambda\), \(\rho\) and \(\alpha\)
    are identities.
\end{definition}

In a strict monoidal category, the unitality and associativity of the tensor
hold as equations, as they do for regular composition in a category.
With this in mind, it can be instructive to view the coherences of a monoidal
category in terms of equations.

\begin{figure}
    \begin{gather*}
        \iltikzfig{strings/monoidal/unit-l-lhs}[f][white][A][B]
        =
        \iltikzfig{strings/category/f}[f][white][A][B]
        \quad
        \iltikzfig{strings/monoidal/unit-r-lhs}[f][white][A][B]
        =
        \iltikzfig{strings/category/f}[f][white][A][B]
        \quad
        \iltikzfig{strings/monoidal/associativity-lhs}[f][g][h][white]
        =
        \iltikzfig{strings/monoidal/associativity-rhs}[f][g][h][white]
        \\[0.5em]
        \iltikzfig{strings/monoidal/interchange-lhs}[X][Y][Z][W][white]
        =
        \iltikzfig{strings/monoidal/interchange-rhs}[X][Y][Z][W][white]
        \quad
        \iltikzfig{strings/monoidal/identity-tensor-lhs}[white][A][B]
        =
        \iltikzfig{strings/category/identity}[white][A \tensor B]
    \end{gather*}
    \caption{
        Equations of a strict monoidal category in string diagram notation
    }
    \label{fig:mc-equations}
\end{figure}

\begin{example}
    \(\set\) is a monoidal category, with the tensor product defined as the
    Cartesian product (\(A \tensor B := A \times B\)) and the unit as the
    singleton set (\(I := \mathbbm{1}\)).
\end{example}


We can now construct morphisms by composing them in sequence and in parallel.
The next construct to consider are the \emph{symmetries}.

\begin{definition}[Symmetric monoidal category]
    \label{def:symmetric-monoidal-category}
    A \emph{symmetric monoidal category} (SMC) is a monoidal category \(\mcc\)
    equipped with natural isomorphisms \(
        \swap{A}{B} \colon A \tensor B \cong B \tensor A
    \) such that the following diagrams commute:
    \begin{center}
        \includestandalone{figures/category/coherences/symmetry-unit}
        \includestandalone{figures/category/coherences/symmetry-inverse}

        \vspace{1em}

        \includestandalone[scale=0.95]{figures/category/coherences/hexagon}
    \end{center}
\end{definition}

\begin{example}
    \(\set\) is a symmetric monoidal category, with \(
        \morph{\swap{A}{B}}{A \times B}{B \times A}
    \) defined as the function that swaps elements of a pair.
\end{example}

As with monoidal categories, it is natural to consider \emph{strict} symmetric
monoidal categories and view them in terms of their equations.

\begin{figure}
    \begin{gather*}
        \iltikzfig{strings/symmetric/naturality-lhs}[f][g][white][X][Y][Z][W]
        =
        \iltikzfig{strings/symmetric/naturality-rhs}[f][g][white][X][Y][Z][W]
        \quad
        \iltikzfig{strings/symmetric/hexagon-lhs}[white][X][Y][Z]
        =
        \iltikzfig{strings/symmetric/symmetry}[white][X][Y][Z]
        \\[0.5em]
        \iltikzfig{strings/symmetric/unit-l-lhs}[white][I][X]
        =
        \iltikzfig{strings/category/identity}[white][X]
        \quad
        \iltikzfig{strings/symmetric/unit-r-lhs}[white][X][I]
        =
        \iltikzfig{strings/category/identity}[white][X]
        \quad
        \iltikzfig{strings/symmetric/self-inverse-lhs}[white][X][Y]
        =
        \iltikzfig{strings/category/identity}[white][X \tensor Y]
    \end{gather*}
    \caption{
        Equations of a symmetric monoidal category in string diagram notation
    }
    \label{fig:smc-equations}
\end{figure}