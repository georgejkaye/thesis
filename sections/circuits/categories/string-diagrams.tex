\subsection{String diagrams}

Since composition is associative in a category, we can drop any brackets around
terms like \(
    f \seq ((g \seq h) \seq k)
\) and simply write them as \(
    f \seq g \seq h \seq k
\), which is much easier to read.
However, as we add more structure in the upcoming sections the textual
descriptions will quickly become indecipherable.
Therefore, it is useful to consider an alternative \emph{graphical} notation
known as \emph{string diagrams}~\cite{joyal1991geometry}.
In a string diagram, a morphism \(\morph{f}{A}{B}\) is drawn as a box \(
    \iltikzfig{strings/category/f}[box=f,dom=A,cod=B,colour=white]
\), and the identity \(\morph{\id[A]}{A}{A}\) is drawn as a wire \(
    \iltikzfig{strings/category/identity}[colour=white,obj=A]
\).
Since the identity is so common (indeed, one could infinite extend a wire by
adding exta identities), we generally do not draw a box around it.

Composition is depicted as \emph{horizontal juxtaposition}. \[
    \iltikzfig{strings/category/f}[box=f,dom=A,cod=B,colour=white]
    \seq
    \iltikzfig{strings/category/f}[box=g,dom=B,cod=C,colour=white] :=
    \iltikzfig{strings/category/composition}[box1=f, box2=g, colour=white, dom=A, cod=C]
\]
The power of string diagrams comes from how they `absorb' the equations of a
category, as shown in \cref{fig:c-equations}.

\begin{figure}
    \begin{gather*}
        \iltikzfig{strings/category/identity-l-lhs}[box=f,colour=white,dom=A,cod=B]
        =
        \iltikzfig{strings/category/f}[name=f,dom=A,cod=B,colour=white]
        \quad
        \iltikzfig{strings/category/identity-r-lhs}[box=f,dom=A,cod=B,colour=white]
        =
        \iltikzfig{strings/category/f}[box=f,dom=A,cod=B,colour=white]
        \quad
        \iltikzfig{strings/category/associativity-lhs}[box1=f,box2=g,box3=h,colour=white,dom=A,cod=C]
        =
        \iltikzfig{strings/category/associativity-rhs}[box1=f,box2=g,box3=h,colour=white,dom=A,cod=C]
    \end{gather*}
    \caption{
        Equations of a category in string diagram notation
    }
    \label{fig:c-equations}
\end{figure}

Aside from being convenient from a notational point of view, the diagrams are
also far more intuitive!
The extra structure we will introduce later also has an elegant graphical
interpretation in the string diagram notation.

\begin{remark}
    The direction that the `flow' of string diagrams travels from inputs to
    outputs is a hotly-debated topic; in this thesis we adopt the left-to-right
    approach.
    If you prefer another, you could rotate the document by ninety degrees or
    use a mirror.
\end{remark}