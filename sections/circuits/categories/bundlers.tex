\section{Splitting and combining}

In the context of \(\natplus\)-sorted PROPs, the bundler generators
\(\bundleexpand{\listvar{n}}\) and \(\bundlecontract{\listvar{n}}\) have an
interpretation as morphisms
\(\listvar{m} \to \listvar{n}\) where the source and target words have the
same sum, but constructed in a different way, such as \([1,2] \to [2,1]\).
These constructs were first proposed by Wilson et al in~\cite{wilson2023string}
as a notation for \emph{non-strict categories}.
However, it is also noted that it could be used in the strict case as a way of
grouping and splitting wires.
We present a slight variation of their construction specialised for
\(\natplus\)-sorted PROPs.

\begin{notation}
    We write \(m^n\) for the list of length \(n\) containing only \(m\).
\end{notation}

\begin{definition}
    A \emph{bundled} \(\natplus\)-\emph{PROP} is a \(\natplus\)-sorted PROP
    equipped with natural transformations \(
        \morph{\bundleexpand{n}}{n}{1^n}
    \) and \(
        \morph{\bundlecontract{n}}{1^n}{n}
    \), such that \(\bundleexpand{1^n} = \id[1^n]\), \(\bundlecontract{1^n} = \id[1^n]\),
    and \(\bundleexpand{n}\) and \(\bundlecontract{n}\) are inverses,
    i.e.\ \(
        \bundleexpand{n} \seq \bundlecontract{n} = \id[n]
    \) and \(
        \bundlecontract{n} \seq \bundleexpand{n} = \id[1^n]
    \).
\end{definition}
