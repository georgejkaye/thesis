\section{Divisible PROPs}

In the context of \(\natplus\)-sorted PROPs, the bundler generators
\(\bundleexpand{\listvar{n}}\) and \(\bundlecontract{\listvar{n}}\) have an
interpretation as morphisms
\(\listvar{m} \to \listvar{n}\) where the source and target lists have the
same sum, but constructed in a different way, such as \([1,2] \to [2,1]\).


These constructs were first proposed by Wilson et al in~\cite{wilson2023string}
as a notation for \emph{non-strict categories}.
However, it was noted that a similar idea could also be applied to strict
categories as a way to group and split wires.
Here, we will formalise this notion as \emph{divisible PROPs}.

\begin{notation}
    Given a function \(\morph{f}{X^\star}{\nat}\) and list
    \(\listvar{x} \in X^\star\), we write \(\Sigma{f^\star(\listvar{x})}\) to
    denote the sum of the elements of the list obtained by applying \(f\)
    pointwise to \(\listvar{x}\).
    Given a function \(\morph{f}{X}{\nat}\), we write \(\Sigma f^\star\) to
    denote the sum of the list obtained by applying \(f\) pointwise.
\end{notation}

\begin{definition}[Divisible PROP]
    A \emph{divisble} PROP is a \(\mcc\)-sorted PROP equipped
    with a sort \(b \in \mcc\) called the \emph{base} and a function \(
        \morph{\splitfn}{\mcc}{\natplus}
    \) such that \(\splitfn(u) = 1\), along with natural isomorphisms \(
        \morph{\bundleexpand{A}}{A}{1^{\Sigma\splitfn^\star{A}}}
    \) and \(
        \morph{\bundlecontract{A}}{1^{\Sigma\splitfn^\star{A}}}{A}
    \) such that
    \begin{itemize}
        \item \(\bundleexpand{A}\) and \(\bundlecontract{A}\)
                are \emph{inverses}, i.e.\ \(
                    \bundleexpand{A} \seq \bundlecontract{A} = \id[A]
                \) and \(
                    \bundlecontract{A} \seq \bundleexpand{A} = \id[u^{\Sigma\splitfn A}]
                \);
        \item objects of the form \(b^{n}\) are \emph{annihilators},
                i.e.\ \(
                    \bundleexpand{b^n} = \id[b^n]
                \) and \(
                    \bundlecontract{b^n} = \id[b^n]
                \); and
        \item \(\bundleexpand{A}\) and \(\bundlecontract{A}\) respect the
                tensor, i.e.\ \(
                    \bundleexpand{A} \tensor \bundleexpand{B}
                    =
                    \bundleexpand{A \tensor B}
                \) and \(
                    \bundlecontract{A} \tensor \bundlecontract{B}
                    =
                    \bundlecontract{A \tensor B}
                \).
    \end{itemize}
\end{definition}

In a divisible PROP, every sort can be expressed as a power of the base sort.
The components \(
    \bundleexpand{\listvar{n}}
\), called the \emph{splitters}, turns a list of `thick' wires into a list
solely comprised of the base sort; the components \(
    \bundlecontract{\listvar{n}}
\), called the \emph{cosplitters}, perform the reverse operation.

\begin{example}
    Any \(\natplus\) sorted PROP is divisible with the factor set as \(1\) and
    \(\splitfn\) set as the identity.
\end{example}

In string diagrams, we draw the splitter and cosplitter as an elongated `bubble'
\(
    \iltikzfig{strings/strictifiers/bundle}[colour=white,dom=A,cod=B]
\), with the appropriate number of wires on either side.
The equations of divisible PROPs illustrated as string diagrams are shown in
\cref{fig:divisible-prop-equations}

\begin{example}
    The splitter \(
        \morph{\bundleexpand{[2,3]}}{[2,3]}{1^5}
    \) and is drawn as \(
        \iltikzfig{strings/strictifiers/splitter-example}
    \) and the cosplitter \(
        \morph{\bundlecontract{[1,3,1]}}{1^5}{[1,3,1]}
    \), and drawn as \(
        \iltikzfig{strings/strictifiers/cosplitter-example}
    \).
\end{example}

\begin{figure}
    \begin{gather*}
        \iltikzfig{strings/strictifiers/equations/split-cosplit-inverse-lhs}
        =
        \iltikzfig{strings/strictifiers/equations/split-cosplit-inverse-rhs}
        \quad
        \iltikzfig{strings/strictifiers/equations/cosplit-split-inverse-lhs}
        =
        \iltikzfig{strings/strictifiers/equations/cosplit-split-inverse-rhs}
        \quad
        \iltikzfig{strings/strictifiers/equations/annihilate-lhs}
        =
        \iltikzfig{strings/strictifiers/equations/annihilate-rhs}
        \\[0.5em]
        \iltikzfig{strings/strictifiers/equations/split-tensor-lhs}
        =
        \iltikzfig{strings/strictifiers/equations/split-tensor-rhs}
        \quad
        \iltikzfig{strings/strictifiers/equations/cosplit-tensor-lhs}
        =
        \iltikzfig{strings/strictifiers/equations/cosplit-tensor-rhs}
    \end{gather*}
    \caption{Equations of divisible PROPSs expressed in string diagram notation}
    \label{fig:divisible-prop-equations}
\end{figure}

It is often the case that we do not want to split a wire into \emph{all} of its
constituent bits, but merely rearrange which wires are bundled together.

\begin{definition}[Bundler]
    Given objects \(A,B\) in a divisible PROP, the \emph{bundler}
    \(\morph{\bundle{A}{B}}{A}{B}\) is defined as \(
        \bundleexpand{A} \seq \bundlecontract{B}
    \).
\end{definition}

Abusing notation somewhat, we draw a bundler in the same way as the
(co)splitters.

\begin{example}

    The bundler \(\morph{\bundle{[2,3]}{[1,3,1]}}{[2,3]}{[1,3,1]}\)
    is drawn as \(
        \iltikzfig{strings/strictifiers/bundle-example}
    \).
\end{example}