\subsection{Interpreting the bundlers}

In the context of \(\natplus\)-sorted PROPs, the bundler generators
\(\bundle{\overline{m}}{\overline{n}}\) have an interpretation as morphisms
\(\overline{m} \to \overline{n}\) where the source and target words have the
same sum, but constructed in a different way, such as \([1,2] \to [2,1]\).
These constructs were first proposed by Wilson et al in~\cite{wilson2023string}
as a notation for \emph{non-strict categories}.
However, it is also noted that it could be used in the strict case as a way of
grouping and splitting wires.
We present a slight variation of their construction specialised for
\(\natplus\)-sorted PROPs.

\begin{definition}[Bundled PROP]
    Given a \(\natplus\)-sorted PROP \(\mcc\), its \emph{bundled PROP} is
    a \(\natplus\)-sorted PROP \(\bundled{\mcc}\) with the same objects as
    \(\mcc\) but equipped with additional generators \(
        \morph{\bundleexpand{n}}{n}{1^n}
    \) and \(
        \morph{\bundlecontract{n}}{1^n}{n}
    \),
    for every \(n \in \natplus\), such that \(\bundleexpand{n}\) and
    \(\bundlecontract{n}\) are inverses, i.e.\ \(
        \bundleexpand{n} \seq \bundlecontract{n} = \id[n]
    \) and \(
        \bundlecontract{n} \seq \bundleexpand{n} = \id[1^n]
    \).
\end{definition}

From these primitive generators, we can inductively define bundlers for
larger lists of wires.

\begin{definition}[Composite bundlers]
    For a list \(\overline{m} \in \natplus^\star\), the \emph{composite bundles}
    are defined as follows:
    \begin{gather*}
        \morph{
            \bundleexpand{\overline{m}}
        }{
            \overline{m}
        }{
            1^{\wordsum{\overline{m}}}
        }
        :=
        \bigtensor_{n \in \overline{m}} \bundleexpand{n}
        \quad
        \morph{
            \bundlecontract{\overline{m}}
        }{
            1^{\wordsum{\overline{m}}}
        }{
            \overline{m}
        }
        :=
        \bigtensor_{n \in \overline{m}} \bundlecontract{n}
        \quad
        \morph{
            \bundle{\overline{m}}{\overline{n}}
        }{
            \overline{m}
        }{
            \overline{n}
        }
        :=
        \bundleexpand{\overline{m}} \seq \bundlecontract{\overline{n}}
    \end{gather*}
\end{definition}