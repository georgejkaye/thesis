\subsection{Functors}

It is actually quite rare that we are only making use of one category at a time.
In order to compare categories, we need a notion of \emph{mapping between} them.
A class of maps that enjoy some useful properties are known as \emph{functors}.

\begin{definition}[Functor]
    Given two categories \(\mcc\) and \(\mcd\), a \emph{functor} \(
        \morph{F}{\mcc}{\mcd}
    \) maps objects and morphisms in \(\mcc\) to objects and morphisms in
    \(\mcd\) such that
    \begin{itemize}
        \item \(F(\id[A]) = \id[FA]\) for every \(A \in \ob{\mcc})\); and
        \item \(F(g \circ f) = F(g) \circ F(f)\) for every \(\morph{f}{A}{B}\)
        and \(\morph{g}{B}{C}\).
    \end{itemize}
\end{definition}

A functor \(F\) maps an object \(X\) in one category to an object \(FX\) in
another, and morphisms \(\morph{f}{X}{Y}\) to \(\morph{Ff}{FX}{FY}\).
The two equations are known as the \emph{functoriality} equations; if a map
satisfies these it is \emph{functorial} i.e.\ it is a functor.

Functors have a graphical representation as `functorial
boxes'~\cite{mellies2006functorial}; applying a functor \(\morph{F}{\mcc}{\mcd}\)
to a morphism \(\morph{f}{X}{Y}\) is drawn as
\[
    \iltikzfig{strings/category/functors/f}[box=f,functor=F,colour=white,dom=X,cod=Y]
\]
As always, the wire labels are optional and will be omitted if unambiguous.
The functoriality equations are represented as in
\cref{fig:functoriality-equations}.

\begin{figure}
    \begin{gather*}
        \iltikzfig{strings/category/functors/identity-lhs}[functor=F,object=X]
        =
        \iltikzfig{strings/category/functors/identity-rhs}[functor=F,object=X]
        \quad
        \iltikzfig{strings/category/functors/composition-lhs}[box1=f,box2=g,functor=F,colour=white,dom=X,cod=Z]
        =
        \iltikzfig{strings/category/functors/composition-rhs}[box1=f,box2=g,functor=F,colour=white,dom=X,cod=Z]
    \end{gather*}
    \caption{
        Equations of functoriality in string diagram notation
    }
    \label{fig:functoriality-equations}
\end{figure}

\begin{definition}[Endofunctor]
    An \emph{endofunctor} on a category \(\mcc\) is a functor \(\mcc \to \mcc\).
\end{definition}

\subsubsection{Examples of functors}

\begin{example}[Identity functor]
    A trivial endofunctor for any category \(\mcc\) is the
    \emph{identity functor} \(\morph{\idf}{\mcc}{\mcc}\) which acts as the
    identity on objects and morphisms.
\end{example}

\begin{example}[Powerset functor]
    The notion of powerset can be interpreted as an endofunctor \(
        \morph{\powerset}{\set}{\set}
    \), mapping a set \(X\) to its powerset \(\powerset(X)\) and a morphism
    \(\morph{f}{X}{Y}\) to the function \(\powerset(X) \to \powerset(Y)\) which
    applies \(f\) pointwise.
\end{example}

\begin{example}[List functor]\label{ex:list-functor}
    A functor that crops up frequently in computer science is the
    \emph{list functor} \(\morph{\listf}{\set}{\set}\), which sends a set
    \(X\) to its set of lists \(\freemon{X}\), and sends a function
    \(\morph{f}{X}{Y}\) to the function
    \(\morph{\freemon{f}}{\freemon{X}}{\freemon{Y}}\): which applies \(f\)
    to each element of a list.
\end{example}

\begin{example}[Free monoid]\label{ex:free-monoid}
    When talking about a mathematical structure, there is often a notion
    of its \emph{free construction}, which can be viewed as the most
    `bare-bones' version.
    For example, the set of lists \(\freemon{X}\) is the carrier of the
    \emph{free monoid} on \(X\).
    This means there is a functor \(\morph{F}{\set}{\mon}\) (the
    \emph{free functor}) that acts on objects as \(
        X \mapsto (\freemon{X}, \concat, [])
    \) and sends morphisms \(X \to Y\) to the corresponding monoid homomorphism
    \(\freemon{X} \to \freemon{Y}\).

    There is also a \emph{forgetful} or \emph{underlying} functor
    \(\morph{U}{\mon}{\set}\) which sends a monoid \((X, *, e)\) to its carrier
    set \(X\) and `forgets' the monoid structure.
    These functors form a relationship known as an \emph{adjunction}, but this
    is beyond the scope of this thesis.
\end{example}

\subsection{Properties of functors}

Functors can also have useful properties.

\begin{notation}
    Given a functor \(\morph{F}{\mcc}{\mcd}\), let \(
        \morph{F_{A,B}}{\mor{\mcc}{A}{B}}{\mor{\mcd}{FA}{FB}}
    \) be the induced map sending classes of morphisms \(A \to B\) in \(\mcc\)
    to the classes of morphisms \(FA \to FB\) in \(\mcd\).
\end{notation}

\begin{definition}[Faithful functor~\cite{maclane1978categories}]
    A functor \(\morph{F}{\mcc}{\mcd}\) is \emph{faithful} if \(F_{A,B}\) is
    injective for all \(A,B \in \mcc\).
\end{definition}

A faithful functor \(\morph{F}{\mcc}{\mcd}\) does not coalesce morphisms: every
morphism \(f \in \mor{\mcc}{A}{B}\) has a unique morphism
\(Ff \in \mor{\mcd}{FA}{FB}\).

\begin{example}
    Consider the categories \(\set\) and \(\rel\).
    The functor \(\set \to \rel\) that is the identity on objects and maps
    functions in \(\set\) to the corresponding relation in \(\rel\) is faithful
    every function induces a unique relation.
\end{example}

\begin{definition}[Full functor~\cite{maclane1978categories}]
    A functor \(\morph{F}{\mcc}{\mcd}\) is \emph{full} if \(F_{A,B}\) is
    surjective for all \(A,B \in \mcc\).
\end{definition}

Every morphism \(FA \to FB\) is in the image of a full functor.

\begin{example}
    \todo[inline]{Full functor example}
\end{example}

\begin{definition}[Fully faithful functor~\cite{maclane1978categories}]
    A functor \(\morph{F}{\mcc}{\mcd}\) is \emph{fully faithful} if \(F_{A,B}\)
    is bijective; i.e.\ the functor is full and faithful.
\end{definition}

\begin{example}
    \todo[inline]{Fully faithful functor example}
\end{example}

These definitions of fullness and faithfulness are useful when considering
\emph{subcategories}: categories with `some of the bits taken out'.

\begin{definition}[Subcategory]
    Given a category \(\mcc\), a \emph{subcategory} of \(\mcc\) is a category
    \(\mcd\) with objects and morphisms subclasses of the objects and morphisms
    in \(\mcc\) subject to the following conditions:
    \begin{itemize}
        \item for every object \(A \in \mcd\), \(\id[A] \in \mor{\mcd}{A}{A}\);
        \item for every morphism \(\morph{f}{A}{B} \in \mor{\mcd}{A}{B}\),
                \(A\) and \(B\) are in \(\mcd\); and
        \item for morphisms \(f \in \mor{\mcd}{A}{B}\) and
                \(g \in \mor{\mcd}{B}{C}\), \(g \circ f \in \mor{\mcd}{A}{C}\).
    \end{itemize}
\end{definition}

Given a category \(\mcc\) and a subcategory \(\mcd\), there is an obvious
induced functor \(\morph{S}{\mcd}{\mcc}\) mapping objects and morphisms
in \(\mcd\) to the same objects in \(\mcc\); this is called an
\emph{inclusion functor}.

\begin{example}
    \todo[inline]{Subcategory example}
\end{example}

An inclusion functor is clearly faithful, since there cannot be two morphisms in
the subcategory that map to the same morphism in the parent category.
Inclusions functors that are also \emph{full} are of particular interest.

\begin{definition}[Full subcategory]
    A subcategory \(\mcd\) is a \emph{full subcategory} if its inclusion functor
    \(\mcd \to \mcc\) is full and faithful; i.e.\ for all objects
    \(A, B \in \mcd\), the morphisms \(A \to B \in \mcd\) are precisely the
    morphisms \(A \to B \in \mcc\).
\end{definition}

\begin{example}
    \todo[inline]{Full subcategory example (finite sets and sets)}
\end{example}