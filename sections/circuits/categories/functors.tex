\subsection{Functors}

It is actually quite rare that we are only making use of one category at a time.
In order to compare categories, we need a notion of \emph{mapping between} them.
A class of maps that enjoy some useful properties are known as \emph{functors}.

\begin{definition}[Functor]
    Given two categories \(\mcc\) and \(\mcd\), a \emph{functor} \(
        \morph{F}{\mcc}{\mcd}
    \) maps objects and morphisms in \(\mcc\) to objects and morphisms in
    \(\mcd\) such that
    \begin{itemize}
        \item \(F(\id[A]) = \id[FA]\) for every \(A \in \ob{\mcc})\); and
        \item \(F(g \circ f) = F(g) \circ F(f)\) for every \(\morph{f}{A}{B}\)
        and \(\morph{g}{B}{C}\).
    \end{itemize}
\end{definition}

A functor \(F\) maps an object \(X\) in one category to an object \(FX\) in
another, and morphisms \(\morph{f}{X}{Y}\) to \(\morph{Ff}{FX}{FY}\).
The two equations are known as the \emph{functoriality} equations; if a map
satisfies these it is \emph{functorial} i.e.\ it is a functor.

Functors have a graphical representation as `functorial
boxes'~\cite{mellies2006functorial}; applying a functor \(\morph{F}{\mcc}{\mcd}\)
to a morphism \(\morph{f}{X}{Y}\) is drawn as
\[
    \iltikzfig{strings/category/functors/f}[box=f,functor=F,colour=white,dom=X,cod=Y]
\]
As always, the wire labels are optional and will be omitted if unambiguous.
The functoriality equations are represented as in
\cref{fig:functoriality-equations}.

\begin{figure}
    \begin{gather*}
        \iltikzfig{strings/category/functors/identity-lhs}[functor=F,object=X]
        =
        \iltikzfig{strings/category/functors/identity-rhs}[functor=F,object=X]
        \quad
        \iltikzfig{strings/category/functors/composition-lhs}[box1=f,box2=g,functor=F,colour=white,dom=X,cod=Z]
        =
        \iltikzfig{strings/category/functors/composition-rhs}[box1=f,box2=g,functor=F,colour=white,dom=X,cod=Z]
    \end{gather*}
    \caption{
        Equations of functoriality in string diagram notation
    }
    \label{fig:functoriality-equations}
\end{figure}

\begin{definition}[Endofunctor]
    An \emph{endofunctor} on a category \(\mcc\) is a functor \(\mcc \to \mcc\).
\end{definition}

\subsubsection{Examples of functors}

\begin{example}[Identity functor]
    A trivial endofunctor for any category \(\mcc\) is the
    \emph{identity functor} \(\morph{\idf}{\mcc}{\mcc}\) which acts as the
    identity on objects and morphisms.
\end{example}

\begin{example}[Powerset functor]
    The notion of powerset can be interpreted as an endofunctor \(
        \morph{\powerset}{\set}{\set}
    \), mapping a set \(X\) to its powerset \(\powerset(X)\) and a morphism
    \(\morph{f}{X}{Y}\) to the function \(\powerset(X) \to \powerset(Y)\) which
    applies \(f\) pointwise.
\end{example}

\begin{example}[List functor]\label{ex:list-functor}
    A functor that crops up frequently in computer science is the
    \emph{list functor} \(\morph{\listf}{\set}{\set}\), which sends a set
    \(X\) to its set of lists \(\freemon{X}\), and sends a function
    \(\morph{f}{X}{Y}\) to the function
    \(\morph{\freemon{f}}{\freemon{X}}{\freemon{Y}}\): which applies \(f\)
    to each element of a list.
\end{example}

\begin{example}[Free monoid]\label{ex:free-monoid}
    When talking about a mathematical structure, there is often a notion
    of its \emph{free construction}, which can be viewed as the most
    `bare-bones' version.
    For example, the set of lists \(\freemon{X}\) is the carrier of the
    \emph{free monoid} on \(X\).
    This means there is a functor \(\morph{F}{\set}{\mon}\) (the
    \emph{free functor}) that acts on objects as \(
        X \mapsto (\freemon{X}, \concat, [])
    \) and sends morphisms \(X \to Y\) to the corresponding monoid homomorphism
    \(\freemon{X} \to \freemon{Y}\).

    There is also a \emph{forgetful} or \emph{underlying} functor
    \(\morph{U}{\mon}{\set}\) which sends a monoid \((X, *, e)\) to its carrier
    set \(X\) and `forgets' the monoid structure.
    These functors form a relationship known as an \emph{adjunction}, but this
    is beyond the scope of this thesis.
\end{example}

\subsection{Properties of functors}

Functors can also have useful properties.

\begin{notation}
    Given a functor \(\morph{F}{\mcc}{\mcd}\), let \(
        \morph{F_{A,B}}{\mor{\mcc}{A}{B}}{\mor{\mcd}{FA}{FB}}
    \) be the induced map sending classes of morphisms \(A \to B\) in \(\mcc\)
    to the classes of morphisms \(FA \to FB\) in \(\mcd\).
\end{notation}

\begin{definition}[Faithful functor~\cite{maclane1978categories}]
    A functor \(\morph{F}{\mcc}{\mcd}\) is \emph{faithful} if \(F_{A,B}\) is
    injective for all \(A,B \in \mcc\).
\end{definition}

A faithful functor \(\morph{F}{\mcc}{\mcd}\) does not coalesce morphisms: every
morphism \(f \in \mor{\mcc}{A}{B}\) has a unique morphism
\(Ff \in \mor{\mcd}{FA}{FB}\).

\begin{definition}[Full functor~\cite{maclane1978categories}]
    A functor \(\morph{F}{\mcc}{\mcd}\) is \emph{full} if \(F_{A,B}\) is
    surjective for all \(A,B \in \mcc\).
\end{definition}

Every morphism \(FA \to FB\) is in the image of a full functor.

\begin{definition}[Fully faithful functor~\cite{maclane1978categories}]
    A functor \(\morph{F}{\mcc}{\mcd}\) is \emph{fully faithful} if \(F_{A,B}\)
    is bijective; i.e.\ the functor is full and faithful.
\end{definition}

\begin{example}
    Consider the categories \(\set\) and \(\mon\), and the forgetful functor
    \(\morph{F}{\set}{\mon}\) and \(\morph{U}{\mon}{\set}\).
    \(U\) is faithful as monoid homomorphisms are just functions, but it is not
    full as not all functions are monoid homomorphisms.
    Note that even though \(U\) is faithful, it is not injective on objects
    because there may be many monoids with the same carrier set.
\end{example}