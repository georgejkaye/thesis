\subsection{Functors}

It is actually quite rare that we are only making use of one category at a time.
In order to compare categories, we need a notion of \emph{mapping between} them.
A class of maps that enjoy some useful properties are known as \emph{functors}.

\begin{definition}[Functor]
    Given two categories \(\mcc\) and \(\mcd\), a \emph{functor} \(
        \morph{F}{\mcc}{\mcd}
    \) maps objects and morphisms in \(\mcc\) to objects and morphisms in
    \(\mcd\) such that
    \begin{itemize}
        \item \(F(\id[A]) = \id[FA]\) for every \(A \in \ob{\mcc})\); and
        \item \(F(g \circ f) = F(g) \circ F(f)\) for every \(\morph{f}{A}{B}\)
        and \(\morph{g}{B}{C}\).
    \end{itemize}
\end{definition}

A functor \(F\) maps an object \(X\) in one category to an object \(FX\) in
another, and morphisms \(\morph{f}{X}{Y}\) to \(\morph{Ff}{FX}{FY}\).
The two equations are known as the \emph{functoriality} equations; if a map
satisfies these it is \emph{functorial} i.e.\ it is a functor.

Functors have a graphical representation as `functorial
boxes'~\cite{mellies2006functorial}; applying a functor \(\morph{F}{\mcc}{\mcd}\)
to a morphism \(\morph{f}{X}{Y}\) is drawn as
\[
    \iltikzfig{strings/category/functors/f}[][f][F][white][X][Y]
\]
As always, the wire labels are optional and will be omitted if unambious.
The functoriality equations are represented as in
\cref{fig:functoriality-equations}.

\begin{figure}
    \begin{gather*}
        \iltikzfig{strings/category/functors/identity-lhs}[functor=F,object=X]
        =
        \iltikzfig{strings/category/functors/identity-rhs}[functor=F,object=X]
        \quad
        \iltikzfig{strings/category/functors/composition-lhs}[box1=f,box2=g,functor=F,colour=white,dom=X,cod=Z]
        =
        \iltikzfig{strings/category/functors/composition-rhs}[box1=f,box2=g,functor=F,colour=white,dom=X,cod=Z]
    \end{gather*}
    \caption{
        Equations of functoriality in string diagram notation
    }
    \label{fig:functoriality-equations}
\end{figure}

\begin{definition}[Endofunctor]
    An \emph{endofunctor} on a category \(\mcc\) is a functor \(\mcc \to \mcc\).
\end{definition}

\subsubsection{Examples of functors}

\begin{example}[Identity functor]
    A trivial endofunctor for any category \(\mcc\) is the
    \emph{identity functor} \(\morph{\idf}{\mcc}{\mcc}\) which acts as the
    identity on objects and morphisms.
\end{example}

\begin{example}[Powerset functor]
    The notion of powerset can be interpreted as an endofunctor \(
        \morph{\powerset}{\set}{\set}
    \), mapping a set \(X\) to its powerset \(\powerset(X)\) and a morphism
    \(\morph{f}{X}{Y}\) to the function \(\powerset(X) \to \powerset(Y)\) which
    applies \(f\) pointwise.
\end{example}

\begin{example}[List functor]\label{ex:list-functor}
    A functor that crops up frequently in computer science is the
    \emph{list functor} \(\morph{\listf}{\set}{\set}\), which sends a set
    \(X\) to its set of lists \(\freemon{X}\), and sends a function
    \(\morph{f}{X}{Y}\) to the function
    \(\morph{\freemon{f}}{\freemon{X}}{\freemon{Y}}\): which applies \(f\)
    to each element of a list.
\end{example}

\begin{example}[Free monoid]\label{ex:free-monoid}
    When talking about a mathematical structure \(S\), there is often a notion
    of the \emph{free \(S\)}, which can be viewed as the most `bare-bones'
    version of \(S\).
    For example, the set of lists \(\freemon{X}\) is the carrier of the
    \emph{free monoid} on \(X\).
    This means there is a functor \(\morph{F}{\set}{\mon}\) (the
    \emph{free functor}) that acts on objects as \(
        X \mapsto (\freemon{X}, \concat, [])
    \) and sends morphisms \(X \to Y\) to the corresponding monoid homomorphism
    \(\freemon{X} \to \freemon{Y}\).

    There is also a \emph{forgetful} or \emph{underlying} functor
    \(\morph{U}{\mon}{\set}\) which sends a monoid \((X, *, e)\) to its carrier
    set \(X\) and `forgets' the monoid structure.
    These functors form a relationship known as an \emph{adjunction}, but this
    is beyond the scope of this thesis.
\end{example}