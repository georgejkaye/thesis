\section{Categories}

So far, terms are purely syntax, and terms are only equal if they are
constructed in precisely the same way.
Even without considering the interaction of generators with each other, there
are numerous `structural equations' which need to hold to be useful for
modelling processes.

\begin{example}
    Consider the term \(f \seq \id[n]\), which can be read as `run \(f\) and
    then do nothing'.
    Since we are only concerned with the \emph{input-output} behaviour of
    processes, this should be the same as just \(f\).
    However, as it stands \(f \seq \id[n]\) and \(f\) are not equal.
\end{example}

One could just add all the `right' equations manually, but what are the `right'
equations?
It turns out that an elegant mathematical formalism exists in the form of
\emph{symmetric monoidal categories}.

We begin by defining the structure underlying much of the work to come; that of
a \emph{category}.

\begin{definition}[Categories]
    \label{def:category}
    A \emph{category} \(\mcc\) consists of a class of \emph{objects}
    \(\ob{\mcc}\); a class of \emph{morphisms} \(\mor{\mcc}{A}{B}\)
    for every pair of objects \(A, B \in \ob{\mcc}\); and a \emph{composition}
    operation \(
        \morph{
            {-} \circ {=}
        }{
            \mor{\mcc}{B}{C} \times \mor{\mcc}{A}{B}
        }{
            \mor{\mcc}{A}{C}
        }
    \) such that
    \begin{itemize}
        \item composition is \emph{unital}: for every \(
                    A \in \ob{\mcc}
                \) there exists an \emph{identity morphism} \(
                    \id[A] \in \mor{\mcc}{A}{A}
                \) satisfying \(
                    f \circ \id[A] = f = \id[B] \circ f
                \) for any \(
                    f \in \mor{\mcc}{A}{B}
                \); and
        \item composition is \emph{associative}: for any morphisms \(
                    f \in \mor{\mcc}{A}{B}
                \), \(
                    g \in \mor{\mcc}{B}{C}
                \) and \(h \in \mor{\mcc}{C}{D}\), \(
                    (h \circ g) \circ f = h \circ (g \circ f).
                \)
    \end{itemize}
\end{definition}

A morphism \(f \in \mor{\mcc}{A}{B}\) is also called an \emph{arrow}, and will
often be written \(\morph{f}{A}{B}\) accordingly.
The subscripts on the object and morphism classes are also often omitted.

\subsubsection{Commutative diagrams}

Equations in category theory can be expressed using \emph{commutative diagrams}.
For example, the unitality and associativity of composition can be illustrated
as follows:

\begin{center}
    \includestandalone{figures/category/coherences/unitality}
    \quad
    \includestandalone{figures/category/coherences/associativity}
\end{center}

We say that the above two diagrams \emph{commute} precisely because \(
    \id[B] \circ f = f = f \circ \id[A]
\) and \((h \circ g) \circ f = h \circ (g \circ f)\): no matter which path one
takes, the results are equal.
Using commutative diagrams suggests another notation for composition.

\begin{notation}
    \emph{Diagrammtic order} composition is written as
    \(f \seq g := g \circ f\).
\end{notation}

\subsubsection{Examples of categories}

Categories generalise a plethora of mathematical structures: some examples will
now be provided.

\begin{example}[Preorder]
    A \emph{preorder} is a binary relation \(\leq\) on a set \(X\) which is
    reflexive and transitive.
    Any preorder generates a category \(\mcc_\leq\): the objects are the
    elements of \(X\) and \(\mcc_\leq(x, y)\) contains exactly one morphism if
    \(x \leq y\) and none otherwise.
\end{example}

\begin{example}[Sets]
    The category \(\set\) has sets as objects and functions as morphisms.
    Composition is then just function composition.
    Note that the monomorphisms of \(\set\) are the injective functions and the
    isomorphisms the bijective functions.
    Compare this with the category \(\rel\), which has sets as objects and
    \emph{relations} as morphisms.
\end{example}

\begin{example}[Posets]
    A \emph{partial order} on a set \(X\) is a reflexive, antisymmetric and
    transitive relation \(\leq \subseteq X \times X\).
    A set equipped with a partial order is called a
    \emph{partially ordered set}, or \emph{poset} for short.
    A function \(\morph{f}{X}{Y}\) between posets is called \emph{monotone} if
    \(x \leq y\) implies that \(f(x) \leq f(y)\).

    Much like how sets form a category, posets form the category \(\pos\), where
    \(\ob{\pos}\) are sets and \(\hom{\pos}{X}{Y}\) are the monotone functions
    \(X \to Y\).
\end{example}

\begin{example}[Monoids]
    A \emph{monoid} is a tuple \((X, *, \bullet)\) where \(X\) is a set called
    the \emph{carrier}, \(\morph{*}{X \times X}{X}\) is a binary operation
    called the \emph{multiplication}, and \(\bullet \in X\) is an element called
    the \emph{unit}, such that \(x * \bullet = x = \bullet * x\) for any
    \(x \in X\).
    A \emph{monoid homomorphism} between two monoids \((X, *, e_X)\) and
    \((Y, +, e_Y)\) is a map \(\morph{h}{X}{Y}\) such that
    \(h(x * y) = h(x) + h(y)\) and \(h(e_X) = e_Y\).
    There is a category \(\mon\) with the objects as monoids and morphisms as
    their homomorphisms.
\end{example}

\begin{example}[Product category]
    Given two categories \(\mcc\) and \(\mcd\), their \emph{product category}
    \(\mcc \times \mcd\) is the category with objects are defined as \(
        \ob{(\mcc \times \mcd)} := \ob{\mcc} \times \ob{\mcd}
    \) and the morphisms as \[
        \mor{(\mcc \times \mcd)}{(A, A^\prime)}{(B, B^\prime)}
        :=
        \{
            (f, f^\prime)
            \,|\,
            f \in \mor{\mcc}{A}{B},
            f^\prime \in \mor{\mcd}{A^\prime}{B^\prime}
        \}
    \]
\end{example}

\subsubsection{Properties of morphisms}

When using category theory as a tool for reasoning, often the objects are not
particular interesting: it is the \emph{morphisms} that hold the useful
information.
There are many properties that morphsims can hold; we will detail two which will
be useful in this thesis.

\begin{definition}[Monomorphism]
    A morphism \(\morph{f}{A}{B} \in \mcc\) is called a \emph{monomorphism} (or
    simply \emph{mono} for short) if for any two morphisms
    \(\morph{g_1,g_2}{C}{A}\), if \(f \circ g_1 = f \circ g_2\), then
    \(g_1 = g_2\).
\end{definition}

One can think of monomorphisms are \emph{left-cancellative} morphisms.
There is also a categorical generalisation of \emph{invertible} morphisms.

\begin{definition}[Isomorphism]
    A morphism \(\morph{f}{A}{B} \in \mcc\) is called an \emph{isomorphism} (or
    simply \emph{iso} for short) if there also exists a morphism \(
        \morph{\inverse{f}}{B}{A} \in \mcc
    \) such that \(
        \inverse{f} \circ f = \id[A]
    \) and \(
        f \circ \inverse{f} = \id[B]
    \).
\end{definition}

\begin{example}
    In \(\set\), the monomorphisms are the injective functions and the
    isomorphisms are the surjective functions.
\end{example}