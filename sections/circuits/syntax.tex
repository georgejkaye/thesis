\chapter{Syntax}

It is time to use monoidal theories for our desired use case: that of
\emph{digital circuits}.

\section{Circuit signatures}

\begin{definition}[Circuit signature, value, gate symbol]
    A \emph{circuit signature} \(\circuitsignature\) is a tuple \((
        \circuitsignaturevalues,
        \disconnected,
        \circuitsignaturegates,
        \circuitsignaturearity,
        \circuitsignaturecoarity
    )\) where \(\circuitsignaturevalues\) is a finite set of \emph{values}, \(
        \disconnected \in \circuitsignaturevalues
    \) is a \emph{disconnected} value, \(\circuitsignaturegates\) is a (usually
    finite) set of \emph{gate symbols}, \(
        \morph{\circuitsignaturearity}{\circuitsignaturegates}{\natplus^\star}
    \) is an \emph{arity} function and \(
        \morph{\circuitsignaturecoarity}{\circuitsignaturegates}{\natplus^\star}
    \) is a \emph{coarity} function.
\end{definition}

A particularly important signature is that of gate-level circuits, the most
common level of abstraction for digital circuits.

\begin{example}[Gate-level circuits]\label{ex:sig}
    The signature for \emph{gate-level circuits} is \(
        \belnapsignature = (
            \belnapvalues,
            \belnapnone,
            \belnapgates,
            \belnaparity,
            \belnapcoarity
    )\), where \(
        \belnapvalues := \{\belnapnone, \belnapfalse, \belnaptrue, \belnapboth\}
    \), respectively representing \emph{no} signal, a \emph{false} signal, a
        \emph{true} signal and \emph{both} signals at once, \(
        \belnapgates := \{\andgate,\orgate,\notgate\}
    \), \(
        \belnaparity :=
            \andgate \mapsto [1,1],
            \orgate \mapsto [1,1],
            \notgate \mapsto [1]
    \) and \(
        \belnapcoarity := - \mapsto [1],
    \)
\end{example}

\section{Combinational circuits}

\begin{definition}[Combinational circuit generators]
    Given a circuit signature \(
        \circuitsignature = (
            \circuitsignaturevalues,
            \bullet,
            \circuitsignaturegates,
            \circuitsignaturearity,
            \circuitsignaturecoarity
        )
    \), let the set \(\generators[\ccirc{}]\) of
    \emph{combinational circuit generators} be defined as the set containing \(
        \iltikzfig{circuits/components/gates/gate}[][g][comb][][\circuitsignaturearity(g)]
    \) for each \(g \in \circuitsignaturegates\),
    \(\iltikzfig{strings/structure/monoid/init}[][comb]\),
    \(\iltikzfig{strings/structure/comonoid/copy}[][comb]\),
    \(\iltikzfig{strings/structure/monoid/merge}[][comb]\), and
    \(\iltikzfig{strings/structure/comonoid/discard}[][comb]\).
    We write \(\ccircsigma\) for the freely generated PROP
    \(\smc{\generators[\ccirc{}]}\).
\end{definition}

Rectangular light blue generators are \emph{gates} for each gate symbol in the
signature \(\Sigma\).
The remaining generators are \emph{structural} generators for manipulating
wires: these are present regardless of the signature.
In order, they are for \emph{introducing} wires, \emph{forking}
wires, \emph{joining} wires and \emph{stubbing} wires.

\begin{example}
    The gate generators of \(\ccirc{\belnapsignature}\) are \(
        \iltikzfig{circuits/components/gates/and}
    \), \(
        \iltikzfig{circuits/components/gates/or}
    \), and \(
        \iltikzfig{circuits/components/gates/not}
    \).
\end{example}

When drawing circuits, the coloured backgrounds of generators will often be
omitted in the interests of clarity.
Since the category is freely generated, morphisms are defined by
juxtaposing the generators in a given signature sequentially or in parallel with
each other, the symmetry \iltikzfig{strings/symmetric/symmetry}[][comb] and the
identity \iltikzfig{strings/category/identity}[][comb].
Arbitrary combinational circuit morphisms defined in this way are drawn as light
boxes \iltikzfig{circuits/components/circuits/f}[][f][comb][m][n].

\begin{notation}\label{not:arbitrary-width-structure}
    It is a simple exercise to define the structural generators, the identity
    and symmetry on arbitrary bit inputs and outputs using the axioms of
    symmetric monoidal categories.
    In diagrams, these are drawn as their single-bit
    counterparts:
    \[
        \iltikzfig{strings/structure/monoid/init}[][comb][n]
        \quad
        \iltikzfig{strings/structure/comonoid/copy}[][comb][n]
        \quad
        \iltikzfig{strings/structure/monoid/merge}[][comb][n]
        \quad
        \iltikzfig{strings/structure/comonoid/discard}[][comb][n]
        \quad
        \iltikzfig{strings/category/identity}[][comb][n]
        \quad
        \iltikzfig{strings/symmetric/symmetry}[][comb][m][n]
    \]
\end{notation}

\section{Sequential circuits}

Combinational circuits compute functions of their inputs, but have no internal
state.
Real-world circuits often involve \emph{delay} and \emph{feedback}: these are
known as \emph{sequential circuits}.
To model feedback, extra structure must be added to the category of
combinational circuits in the form of a \emph{trace}.

\begin{definition}[Symmetric traced monoidal category]
    \cite{joyal1996traced}
    A \emph{symmetric traced monoidal category}, often abbreviated as STMC, is a
    symmetric monoidal category \(\mcc\) equipped with a family of functions
    \(
        \morph{
            \trace{X}[A][B]{-}
        }{
            \mcc(X \tensor A, X \tensor B)
        }{
            \mcc(X, Y)
        }
    \) satisfying the axioms of STMCs:
\end{definition}

In string diagrams, the trace is represented by joining some of the
inputs of a circuit to its outputs.

\begin{center}
    \(
        \trace{x}[m][n]{\iltikzfig{strings/traced/trace-lhs}[f][comb][x][m][n]}
        \stackrel{\text{def}}{=}
        \iltikzfig{strings/traced/trace-rhs}[][f][comb][m][n]
    \)
\end{center}