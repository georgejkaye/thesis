
\section{Bisimulation equations}

The optimistic reader might thing that once two circuits have been brought to
the same initial state, we might stumble across a way of translating one into
the other through some combination of using the normalisation equations and
sliding components around the trace.
Unfortunately, these is simply \emph{not sufficient} for
translating between circuits with the same semantics.

\begin{example}
    Consider the following circuit in \(\scirc{\belnapsignature}\): \[
        \iltikzfig{circuits/examples/state-change/circuit}
        \quad
        \iltikzfig{circuits/examples/state-change/circuit-simpler}
    \]
    The registers will \emph{always} contain
    \(\belnapfalse\) and \(\belnaptrue\) and both circuits will
    produce a constant \(\belnapfalse\) output, so a complete equational theory
    should be able to translate between them.
    To this end, we assemble these into Mealy forms with the same initial states:
    \[
        \iltikzfig{circuits/examples/state-change/circuit-mealy}
        \quad
        \iltikzfig{circuits/examples/state-change/circuit-simpler-mealy}
    \]
    The combinational cores do \emph{not} have the same semantics!
    They only act the same because they receive certain inputs
    from \(\belnapvalues^{3}\).
\end{example}

\todo[inline]{Segue to bot top thing}

The circuit on the right hand side of the encoding equation has a state
containing just \(\bot\) and \(\top\) values.
This is the `pseudo-normal form' for sequential circuits; depending on the
state order picked there are likely to be multiple circuits of this form.
Fortunately, since these circuits are also in normalised Mealy form, the
encoding equation can be used to translate between pseudo-normal forms as well.

\begin{definition}
    For an interpretation \(\interpretation\), let
    \(\mce_{\interpretation}\) be defined as \(
    \mealyequations +
    \instantfeedbackeqn +
    \cartesianequations +
    \normalisingequations +
    \encodingequation
    \), and let \(\scircsigmae\) be defined as
    \(\scircsigma / \mce_{\interpretation}\).
\end{definition}

\begin{theorem}
    For a functionally complete interpretation \(\interpretation\), \(
    \iltikzfig{strings/category/f}[box=f,colour=seq,dom=\listvar{m},cod=\listvar{n}]
    =
    \iltikzfig{strings/category/f}[box=g,colour=seq,dom=\listvar{m},cod=\listvar{n}]
    \) in \(\scircsigmae\) if and only if \(
    \circuittostreami[
        \iltikzfig{strings/category/f}[box=f,colour=seq,dom=\listvar{m},cod=\listvar{n}]
    ]
    =
    \circuittostreami[
        \iltikzfig{strings/category/f}[box=g,colour=seq,dom=\listvar{m},cod=\listvar{n}]
    ]
    \).
\end{theorem}
\begin{proof}
    All the equations are sound, so we only need to consider the \(\ifdir\)
    direction.
    By \cref{lem:normalised-mealy}, the two circuits can be brought to
    normalised Mealy form using
    \(
    \mealyequations +
    \instantfeedbackeqn +
    \cartesianequations +
    \normalisingequations
    \).
    By \cref{def:mealy-to-circuit} and
    \cref{thm:circuit-stream-correspondence} there must exist a circuit \(
    \iltikzfig{strings/category/f}[box=h,colour=seq]
    \coloneqq
    \iltikzfig{circuits/algebraic/encoding}[transition={h_0},output={h_1},state={\listvar{s}}]
    \) such that \(
    \circuittostream[
        \iltikzfig{strings/category/f}[box=f,colour=seq]
    ]{\interpretation}
    =
    \circuittostream[
        \iltikzfig{strings/category/f}[box=h,colour=seq]
    ]{\interpretation}
    =
    \circuittostream[
        \iltikzfig{strings/category/f}[box=g,colour=seq]
    ]{\interpretation}
    \).
    The circuit \(
    \iltikzfig{strings/category/f}[box=H,colour=seq]
    \) is encoded such that the state words are in the image of
    \(\gamma_\leq\).
    This means that applying the encoding equation with \(\leq\) to the
    normalised Mealy forms obtained above will yield the circuit \(
    \iltikzfig{strings/category/f}[box=h,colour=seq]
    \).
\end{proof}

As always, the soundness and completeness of the algebraic semantics means we
can establish another isomorphism of PROPs.

\begin{corollary}
    \(\scircsigmai \cong \scircsigmae\).
\end{corollary}

This brings our jaunt into algebraic semantics to a close; given a circuit \(
\iltikzfig{strings/category/f}[box=f,colour=seq,dom=\listvar{m},cod=\listvar{n}]
\) we know that we can translate it into another circuit \(
\iltikzfig{strings/category/f}[box=g,colour=seq,dom=\listvar{m},cod=\listvar{n}]
\) with the same behaviour by only using equations in \(\mce_\interpretation\).

Of course, the procedure of using the normalisation equations to translate a
circuit into normalised Mealy form before using the encoding equation (possibly
multiple times depending on how lucky one gets with their orderings) may be
tedious; one might wonder how this is beneficial to the operational approach in
the previous chapter.
But the beauty of the \emph{algebraic} semantics is that we \emph{don't} need to
do this every time!
Equations can be proven as lemmas and then used repeatedly in the future as
`shortcuts', possibly saving many reasoning steps.
In time, the algebraicist will build up a powerful repertoire of equations and
wield them to bend circuits to their will.