\section{Isolation equations}

So far, we have equations that can translate a combinational circuit into a
canonical representative; this means we can just normalise a circuit and then
use some sort of lookup table to retrieve the corresponding function.

The next step is to determine how to translate a circuit with an arbitrary
set of state words into the `one-hot' encoding of circuits in the image of
\(\mealytocircuiti\), in the state words have the same width as the number of
states in the circuit.
This means we will need some way of computing the number of states in the
circuit.
To do this, we add some equations for \emph{isolating} the transition and output
components of the combinational core.

\begin{definition}[Normalised Mealy form]
    A sequential circuit is in \emph{normalised Mealy form} if it is in the form
    \(
        \iltikzfig{circuits/axioms/encoding}[transition={f_0},output={f_1},dom=m,cod=n,delay=x]
    \) where \(
        \iltikzfig{strings/category/f-2-1}[box=f_0,colour=seq]
    \) and \(
        \iltikzfig{strings/category/f-2-1}[box=f_1,colour=seq]
    \) are normalised circuits.
    Given such a circuit, we write \(
        \morph{\tilde{f}_{0}}{\valuetuple{x+m}}{{\valuetuple{x}}}
    \) and \(
        \morph{\tilde{f}_{1}}{\valuetuple{x+m}}{{\valuetuple{n}}}
    \) for the functions that satisfy \(
        \mealytofunc[\tilde{f}_{0}] = \iltikzfig{strings/category/f-2-1}[box=f_0,colour=seq]
    \) and \(
        \mealytofunc[\tilde{f}_{1}] = \iltikzfig{strings/category/f-2-1}[box=f_1,colour=seq]
    \).
\end{definition}

While it may be obvious which components of the circuit contribute to the
transition and which to the output, it is more likely that the components will
be tightly intertwined, with lots of sharing going on.
This means that to obtain a version of the circuit with the two parts separated
we may need to \emph{copy} components.
For those in the know, this can mean only one thing.

\begin{definition}[Cartesian category~\cite{fox1976coalgebras}]
    A category is \emph{Cartesian} if its tensor product is given by the Cartesian
    product, or, equivalently, if it satisfies the equations in
    \cref{fig:cartesian-axioms}.
\end{definition}

\begin{figure}
    \centering
    \(
        \equationdisplay{
            \iltikzfig{strings/structure/comonoid/unitality-l-lhs}
        }{
            \iltikzfig{strings/structure/comonoid/unitality-l-rhs}
        }{
            \comonoiduniteqnletter
        }
    \)
    %
    \;
    %
    \(
        \equationdisplay{
            \iltikzfig{strings/structure/comonoid/associativity-lhs}
        }{
            \iltikzfig{strings/structure/comonoid/associativity-rhs}
        }{
            \comonoidassoceqnletter
        }
    \)
    %
    \;
    %
    \(
        \equationdisplay{
            \iltikzfig{strings/structure/comonoid/commutativity-lhs}
        }{
            \iltikzfig{strings/structure/comonoid/commutativity-rhs}
        }{
            \comonoidcommeqnletter
        }
    \)
    \;
    \(
        \equationdisplay{
            \iltikzfig{strings/structure/comonoid/coherence-counit-lhs}[obj1=A,obj2=B]
        }{
            \iltikzfig{strings/structure/comonoid/coherence-counit-rhs}[obj1=A,obj2=B]
        }{
            \counitcoheqn
        }
    \)

    \vspace{1em}

    \(
        \equationdisplay{
            \iltikzfig{strings/structure/comonoid/coherence-comonoid-lhs}[obj1=A,obj2=B]
        }{
            \iltikzfig{strings/structure/comonoid/coherence-comonoid-rhs}[obj1=A,obj2=B]
        }{
            \comonoidcoheqn
        }
    \)
    %
    \;
    \(
        \equationdisplay{
            \iltikzfig{strings/structure/cartesian/naturality-copy-lhs}[box=f,colour=white,dom=A,cod=B]
        }{
            \iltikzfig{strings/structure/cartesian/naturality-copy-rhs}[box=f,colour=white,dom=A,cod=B]
        }{
            \cartnatcopyeqn
        }
    \)
    \;
    \(
        \equationdisplay{
            \iltikzfig{strings/structure/cartesian/naturality-discard-lhs}[box=f,colour=white,dom=A]
        }{
            \iltikzfig{strings/structure/cartesian/naturality-discard-rhs}[dom=m]
        }{
            \cartnatdisceqn
        }
    \)
    \caption{Equations that hold in any Cartesian category}
    \label{fig:cartesian-axioms}
\end{figure}

We could quotient \(\scircsigma\) by the equations in
\cref{fig:cartesian-axioms}, but we prefer to use a finite number of equations.

\begin{definition}
    Let \(\cartesianequations\) be defined as the set of equations in
    \cref{fig:local-cartesian-equations}.
\end{definition}

\begin{remark}
    Note that \((\forkjoineqn)\) and \((\joinforkeqn)\) induce equations on the
    `composite' forks and joins from \cref{not:arbitrary-width-structure}.
\end{remark}

\begin{figure*}
    \centering
    \(
        \equationdisplay{
            \iltikzfig{circuits/axioms/fork-gate-lhs}[gate=p,dom=m,cod=n]
        }{
            \iltikzfig{circuits/axioms/fork-gate-rhs}[gate=p,dom=m,cod=n]
        }{
            \gateforkeqn
        }
        \quad
        \equationdisplay{
            \iltikzfig{circuits/axioms/fork-lhs}
        }{
            \iltikzfig{circuits/axioms/fork-rhs}
        }{
            \forkeqnshort
        }
        \quad
        \equationdisplay{
            \iltikzfig{circuits/axioms/delay-fork-lhs}
        }{
            \iltikzfig{circuits/axioms/delay-fork-rhs}
        }{
            \delayforkeqn
        }
        \quad
        \equationdisplay{
            \iltikzfig{circuits/axioms/stub-lhs}
        }{
            \iltikzfig{strings/monoidal/empty}
        }{
            \stubeqnshort
        }
        \quad
        \equationdisplay{
            \iltikzfig{circuits/axioms/gate-stub-lhs}[gate=p,dom=m]
        }{
            \iltikzfig{circuits/axioms/gate-stub-rhs}[dom=m]
        }{
            \gatestubeqn
        }
        \quad
        \equationdisplay{
            \iltikzfig{circuits/axioms/trace-discard-lhs}[box=F,dom=m,trace=x]
        }{
            \iltikzfig{circuits/axioms/trace-discard-rhs}[dom=m]
        }{
            \tracediscardeqn
        }
    \)
    \\[0.5em]
    \(
        \equationdisplay{
            \iltikzfig{strings/structure/comonoid/unitality-l-lhs}
        }{
            \iltikzfig{strings/structure/comonoid/unitality-l-rhs}
        }{
            \comonoidunitleqn
        }
        \quad
        \equationdisplay{
            \iltikzfig{strings/structure/comonoid/associativity-lhs}
        }{
            \iltikzfig{strings/structure/comonoid/associativity-rhs}
        }{
            \comonoidassoceqn
        }
        \quad
        \equationdisplay{
            \iltikzfig{strings/structure/comonoid/commutativity-lhs}
        }{
            \iltikzfig{strings/structure/comonoid/commutativity-rhs}
        }{
            \comonoidcommeqn
        }
        \quad
        \equationdisplay{
            \iltikzfig{strings/structure/frobenius/copy-merge-lhs}
        }{
            \iltikzfig{strings/structure/frobenius/copy-merge-rhs}
        }{
            \forkjoineqn
        }
        \quad
        \equationdisplay{
            \iltikzfig{strings/structure/bialgebra/merge-copy-lhs}
        }{
            \iltikzfig{strings/structure/bialgebra/merge-copy-rhs}
        }{
            \joinforkeqn
        }
        \quad
        \equationdisplay{
            \iltikzfig{strings/structure/bialgebra/merge-discard-lhs}
        }{
            \iltikzfig{strings/structure/bialgebra/merge-discard-rhs}
        }{
            \joinstubeqn
        }
    \)
    \caption{
       Set \(\cartesianequations\) of \emph{local Cartesian} equations
    }
    \label{fig:local-cartesian-equations}
\end{figure*}

\begin{lemma}\label{lem:join-fork-inverse}
    For any sequential circuit \(
        \iltikzfig{strings/category/f-1-2}[box=F,colour=seq,dom=m,cod1=n,cod2=p]
    \), \(
        \iltikzfig{circuits/axioms/join-fork-inverses-lhs}[box=f,dom=\listvar{m},cod1=\listvar{n},cod2=\listvar{p}]
        =
        \iltikzfig{circuits/axioms/join-fork-inverses-rhs}[box=f,dom=\listvar{m},cod1=\listvar{n},cod2=\listvar{p}]
    \) holds in \(\scircsigma / \mcc\).
\end{lemma}
\begin{proof}
    This is by induction over the structure of the circuit \(
        \iltikzfig{strings/category/f-1-2}[box=F,colour=seq]
    \).
    First we must check the base cases.
    For \(
        \iltikzfig{circuits/components/gates/gate}[gate=g,colour=comb]
    \):
    \begin{gather*}
        \iltikzfig{circuits/productivity/join-fork-inverse/gate-partial-step-0}
        \eqaxioms[(\gateforkeqn)]
        \iltikzfig{circuits/productivity/join-fork-inverse/gate-partial-step-1}
        \eqaxioms[(\forkjoineqn)]
        \iltikzfig{circuits/productivity/join-fork-inverse/gate-partial-step-2}
        \eqaxioms[(\gateforkeqn)]
        \iltikzfig{circuits/productivity/join-fork-inverse/gate-partial-step-3}
    \end{gather*}
    For \(
        \iltikzfig{strings/structure/comonoid/copy}[colour=comb]
    \) we first check if only one of the outputs are joined:
    \begin{gather*}
        \iltikzfig{circuits/productivity/join-fork-inverse/fork-single-step-0}
        \eqaxioms[(\comonoidassoceqnletter), (\comonoidcommeqnletter)]
        \iltikzfig{circuits/productivity/join-fork-inverse/fork-single-step-1}
        \eqaxioms[(\forkjoineqn)]
        \iltikzfig{circuits/productivity/join-fork-inverse/fork-single-step-2}
        \eqaxioms[(\comonoidassoceqnletter), (\comonoidcommeqnletter)]
        \iltikzfig{circuits/productivity/join-fork-inverse/fork-single-step-3}
    \end{gather*}
    And now if both outputs are joined:
    \begin{gather*}
        \iltikzfig{circuits/productivity/join-fork-inverse/fork-double-step-0}
        \eqaxioms[(\joinforkeqn)]
        \iltikzfig{circuits/productivity/join-fork-inverse/fork-double-step-1}
        \eqaxioms[(\forkjoineqn)]
        \iltikzfig{circuits/productivity/join-fork-inverse/fork-double-step-2}
        \eqaxioms[(\comonoidassoceqnletter), (\comonoidcommeqnletter)]
        \iltikzfig{circuits/productivity/join-fork-inverse/fork-double-step-3}
    \end{gather*}
    For \(
        \iltikzfig{strings/structure/monoid/merge}[colour=comb]
    \):
    \[
        \iltikzfig{circuits/productivity/join-fork-inverse/join-step-0}
        \eqaxioms[(\joinforkeqn)]
        \iltikzfig{circuits/productivity/join-fork-inverse/join-step-1}
        \eqaxioms[(\forkjoineqn)]
        \iltikzfig{circuits/productivity/join-fork-inverse/join-step-2}
        \eqaxioms[(\joinforkeqn)]
        \iltikzfig{circuits/productivity/join-fork-inverse/join-step-3}
    \]
    For \(
        \iltikzfig{circuits/components/waveforms/delay}
    \): \[
        \iltikzfig{circuits/productivity/join-fork-inverse/delay-step-0}
        \eqaxioms[(\delayforkeqn)]
        \iltikzfig{circuits/productivity/join-fork-inverse/delay-step-1}
        \eqaxioms[(\forkjoineqn)]
        \iltikzfig{circuits/productivity/join-fork-inverse/delay-step-2}
        \eqaxioms[(\delayforkeqn)]
        \iltikzfig{circuits/productivity/join-fork-inverse/delay-step-3}
    \]
    The proofs for \(
        \iltikzfig{strings/structure/monoid/init}[colour=comb]
    \), \(
        \iltikzfig{strings/category/identity}[colour=comb]
    \), \(
        \iltikzfig{strings/symmetric/symmetry}[colour=comb]
    \) and \(
        \iltikzfig{circuits/components/values/vs}
    \) are trivial.

    For composition, we need to show that \(
        \iltikzfig{circuits/axioms/join-fork-inverses-composition-lhs}[box1=f,box2=g,dom=\listvar{m},cod1=\listvar{n},cod2=\listvar{p}]
        =
        \iltikzfig{circuits/axioms/join-fork-inverses-composition-rhs}[box1=f,box2=g,dom=\listvar{m},cod1=\listvar{n},cod2=\listvar{p}]
    \).
    By inductive hypothesis, \(
        \iltikzfig{circuits/axioms/join-fork-inverses-lhs-simple}[box=f]
        =
        \iltikzfig{circuits/axioms/join-fork-inverses-rhs-simple}[box=f]
    \) so we have that \[
        \iltikzfig{circuits/axioms/join-fork-inverses-composition-lhs}[box1=f,box2=g]
        =
        \iltikzfig{circuits/productivity/join-fork-inverse/composition-step-1}[box1=f,box2=g]
        =
        \iltikzfig{circuits/productivity/join-fork-inverse/composition-step-2}[box1=f,box2=g]
        =
        \iltikzfig{circuits/axioms/join-fork-inverses-composition-rhs}[box1=f,box2=g].
    \]

    For tensor, we just need to apply the inductive hypothesis twice as the
    there are two circuits which do not interact.

    For trace we need to show that \(
        \iltikzfig{circuits/productivity/join-fork-inverse/trace-lhs}[box=f]
        =
        \iltikzfig{circuits/productivity/join-fork-inverse/trace-rhs}[box=f]
    \).

\end{proof}