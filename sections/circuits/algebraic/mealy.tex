\section{Mealy equations}

Although we have now left operational semantics behind in the previous chapter,
the notion of Mealy form will come in very handy for defining a sound and
complete algebraic semantics.
We could just naively port across the Mealy rules by changing the squiggly arrow
into an equals sign and hoping nobody notices, but we are better people than
that.
Instead we will show how Mealy form can be derived using a finite set of
equations, rather than resorting to equations parameterised over all circuits.

\begin{figure}
    \centering
    \(
        \equationdisplay{
            \iltikzfig{strings/structure/monoid/unitality-l-lhs}
        }{
            \iltikzfig{strings/structure/monoid/unitality-l-rhs}
        }{
            \monoidunitleqn
        }
        \quad
        \equationdisplay{
            \iltikzfig{strings/structure/monoid/unitality-r-lhs}
        }{
            \iltikzfig{strings/structure/monoid/unitality-r-rhs}
        }{
            \monoidunitreqn
        }
        \quad
        \equationdisplay{
            \iltikzfig{circuits/axioms/bottom-delay-lhs}
        }{
            \iltikzfig{circuits/axioms/bottom-delay-rhs}
        }{
            \bottomdelayeqn
        }
        \quad
        \equationdisplay{
            \iltikzfig{circuits/instant-feedback/equation-lhs}
        }{
            \iltikzfig{circuits/instant-feedback/equation-rhs}
        }{
            \instantfeedbackeqn
        }
    \)
    \caption{
        Set of Mealy equations
        \(\mealyequations\).
    }
    \label{fig:mealy-equations}
\end{figure}

\begin{definition}
    The set \(\mealyequations\) of \emph{Mealy equations} in
    \cref{fig:mealy-equations} are sound.
\end{definition}
\begin{proof}
    The first two rules hold as the join is a monoid in the stream semantics.
    The \((\bottomdelayeqn)\) holds because the semantics of the delay
    component are to output a \(\bot\) value first and then the (delayed)
    inputs: as the semantics of the \(
        \iltikzfig{strings/structure/monoid/init}[colour=comb]
    \) component are to \emph{always} produce \(\bot\), then it does not make a
    difference how delayed it is.
    The final equation is just the instant feedback rule from the previous
    chapter, which is sound by \cref{prop:instant-feedback}.
\end{proof}

\begin{proposition}
    Given a sequential circuit \(
        \iltikzfig{strings/category/f}[box=F,colour=seq]
    \), there exists a circuit in Mealy form such that \(
        \iltikzfig{strings/category/f}[box=F,colour=seq]
        =
        \iltikzfig{circuits/productivity/mealy-form}
    \) in \(\scircsigma / \mealyequations\).
\end{proposition}
\begin{proof}
    Any circuit can be assembled into global trace-delay form solely using the
    axioms of STMCs, which hold by default in \(\scircsigma\).
    From this, a circuit in pre-Mealy form can be obtained by translating
    delays into registers using \(
        \iltikzfig{circuits/axioms/delay-to-register/step-0}
        \eqaxioms[(\monoidunitleqn)]
        \iltikzfig{circuits/axioms/delay-to-register/step-1}
    \) and translating values into registers using \(
        \iltikzfig{circuits/axioms/value-to-register/step-0}
        \eqaxioms[(\monoidunitreqn)]
        \iltikzfig{circuits/axioms/value-to-register/step-1}
        \eqaxioms[(\bottomdelayeqn)]
        \iltikzfig{circuits/axioms/value-to-register/step-2}
    \).
    Subsequently a circuit in Mealy form can be obtained by applying the
    \((\instantfeedbackeqn)\).
\end{proof}

\(\scircsigma / \mealyequations\) is a category in which all circuits are equal
to at least one circuit in Mealy form.
In general, there will be many Mealy forms depending on the ordering one picks
for the delays and values; while each Mealy form circuit will have the same
semantics, there is no reason that their combinational cores must as well.