\section{Normalising circuits}

We can now translate any circuit into a denotationally-equal Mealy form using
equations in \(\mealyeqn\); what now?
As mentioned in the previous section, there is no guarantee that the
combinational cores of Mealy forms have the same semantics, even if the state
word is the same.
Even so, one might think that through some combination of sliding components
around the trace and across the delay, one might stumble across a way of
translating one circuit into another.
Unfortunately, these is simply \emph{not sufficient} for
translating between circuits with the same semantics.

\begin{example}
    Consider the following circuit in \(\scirc{\belnapsignature}\): \[
        \iltikzfig{circuits/examples/state-change/circuit}
        \quad
        \iltikzfig{circuits/examples/state-change/circuit-simpler}
    \]
    The registers will \emph{always} contain
    \(\belnapfalse\) and \(\belnaptrue\) and both circuits will
    produce a constant \(\belnapfalse\) output, so a complete equational theory
    should be able to translate between them.
    To this end, we assemble these into Mealy forms with the same initial states:
    \[
        \iltikzfig{circuits/examples/state-change/circuit-mealy}
        \quad
        \iltikzfig{circuits/examples/state-change/circuit-simpler-mealy}
    \]
    The combinational cores do \emph{not} have the same semantics!
    They only act the same because they receive certain inputs
    from \(\valuetuple{3}\).
\end{example}

We must take \emph{context} into account when defining
equations; unfortunately equations that only deal with the interactions between
individual generators are not enough.

Instead, our strategy is to translate circuits in Mealy form into some sort of
`normal form', from which it is clearer which circuits have the same behaviour.

\todo[inline]{I don't like the previous paragraph}

The first step is to consider what exactly this normal form should look like.
Mealy form itself is already a sort of normal form, but there is no restriction
on how the combinational core is defined.
For functionally complete interpretations, one possibility for a normal form
for combinational circuits immediately presents itself in the form of the map
\(\mealytofunc\) from \(\funci\) to \(\scircsigma\), which defines a canonical
circuit for any (combinational) function.

\begin{remark}
    Recall that even though \(\mealytofunc\) maps into \(\scircsigma\), every
    circuit in its image has combinational behaviour; the image of
    \(\mealytofunc\) is constrained in such a way that the instantaneous values
    can be used as constants.
\end{remark}

\begin{definition}[Normalising equations]
    For a functionally complete interpretation \(\interpretation\), a set of
    equations \(\normalisingequations\) is \emph{normalising} if any
    compositional circuit \(
        \iltikzfig{strings/category/f}[box=f,colour=comb,dom=m,cod=1]
    \) is equal to a circuit in the image of \(\mealytofunc\) by equations in
    \(\normalisingequations\).
\end{definition}

The aim of the normalising equations for an interpretation translate a
combinational core into a form from which it is easy to read off a truth table.
Note that we only require the normalising equations to translate circuits with a
\emph{single} output; this means we will need a way to express each output of
a combinational core as an independent circuit.
This is also important as we need to distinguish which outputs of the core
contribute to the next state, and which to the outputs of the entire circuit.

\begin{definition}[Normalised Mealy form]
    A sequential circuit is in \emph{normalised Mealy form} if it is in the form
    \(
        \iltikzfig{circuits/axioms/encoding}[transition={f_0},output={f_1},dom=m,cod=n,delay=x]
    \) where \(
        \iltikzfig{strings/category/f-2-1}[box=f_0,colour=seq]
    \) and \(
        \iltikzfig{strings/category/f-2-1}[box=f_1,colour=seq]
    \) are normalised circuits.
    Given such a circuit, we write \(
        \morph{\tilde{f}_{0}}{\valuetuple{x+m}}{{\valuetuple{x}}}
    \) and \(
        \morph{\tilde{f}_{1}}{\valuetuple{x+m}}{{\valuetuple{n}}}
    \) for the functions that satisfy \(
        \mealytofunc[\tilde{f}_{0}] = \iltikzfig{strings/category/f-2-1}[box=f_0,colour=seq]
    \) and \(
        \mealytofunc[\tilde{f}_{1}] = \iltikzfig{strings/category/f-2-1}[box=f_1,colour=seq]
    \).
\end{definition}


\section{Case study: The Belnap interpretation}

\todo[inline]{The normalising equations for the Belnap interpretation}