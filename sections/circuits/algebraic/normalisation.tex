\section{Normalising circuits}

We can now translate any circuit into a denotationally-equal Mealy form using
equations in \(\mealyeqn\); what now?
As mentioned in the previous section, there is no guarantee that the
combinational cores of Mealy forms have the same semantics, even if the state
word is the same.
Even so, one might think that through some combination of sliding components
around the trace and across the delay, one might stumble across a way of
translating one circuit into another.
Unfortunately, these is simply \emph{not sufficient} for
translating between circuits with the same semantics.

\begin{example}
    Consider the following circuit in \(\scirc{\belnapsignature}\): \[
        \iltikzfig{circuits/examples/state-change/circuit}
        \quad
        \iltikzfig{circuits/examples/state-change/circuit-simpler}
    \]
    The registers will \emph{always} contain
    \(\belnapfalse\) and \(\belnaptrue\) and both circuits will
    produce a constant \(\belnapfalse\) output, so a complete equational theory
    should be able to translate between them.
    To this end, we assemble these into Mealy forms with the same initial states:
    \[
        \iltikzfig{circuits/examples/state-change/circuit-mealy}
        \quad
        \iltikzfig{circuits/examples/state-change/circuit-simpler-mealy}
    \]
    The combinational cores do \emph{not} have the same semantics!
    They only act the same because they receive certain inputs
    from \(\valuetuple{3}\).
\end{example}

We must take \emph{context} into account when defining
equations; unfortunately equations that only deal with the interactions between
individual generators are not enough.

