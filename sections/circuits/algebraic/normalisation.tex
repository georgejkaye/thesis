\section{Normalising circuits}

We can now translate any circuit into a denotationally-equal Mealy form using
equations in \(\mealyeqn\); what now?
As mentioned in the previous section, there is no guarantee that the
combinational cores of Mealy forms have the same semantics, even if the state
word is the same.
Even so, one might think that through some combination of sliding components
around the trace and across the delay, one might stumble across a way of
translating one circuit into another.
Unfortunately, these is simply \emph{not sufficient} for
translating between circuits with the same semantics.

\begin{example}
    Consider the following circuit in \(\scirc{\belnapsignature}\): \[
        \iltikzfig{circuits/examples/state-change/circuit}
        \quad
        \iltikzfig{circuits/examples/state-change/circuit-simpler}
    \]
    The registers will \emph{always} contain
    \(\belnapfalse\) and \(\belnaptrue\) and both circuits will
    produce a constant \(\belnapfalse\) output, so a complete equational theory
    should be able to translate between them.
    To this end, we assemble these into Mealy forms with the same initial states:
    \[
        \iltikzfig{circuits/examples/state-change/circuit-mealy}
        \quad
        \iltikzfig{circuits/examples/state-change/circuit-simpler-mealy}
    \]
    The combinational cores do \emph{not} have the same semantics!
    They only act the same because they receive certain inputs
    from \(\belnapvalues^{3}\).
\end{example}

We must take \emph{context} into account when defining
equations; unfortunately equations that only deal with the interactions between
individual generators are not enough.

Instead, our strategy is to translate circuits in Mealy form into some sort of
`normal form', from which it is clearer which circuits have the same behaviour.
We take inspiration from a form we have seen already: the image of
\(\mealytocircuiti\), the map from Mealy machines to circuit morphisms.
Recall that circuits in the image of \(\mealytocircuiti\) are in the form \(
    \iltikzfig{circuits/synthesis/mealy-term-spaced}[trace=\listvar{x},dom=\listvar{m},cod=\listvar{n}]
\), where \(\gamma_\leq\) is an \emph{encoding} of the state set with respect
to some total order \(\leq\).
This encoding maps states to words containing only \(\bot\) and \(\top\) where
the width of each word is equal to the number of states in the original Mealy
machine.

From the completeness procedure for the denotational semantics, we know that any
circuit is guaranteed to be denotationally equivalent to a circuit in this form;
if we have equations to translate any circuits with the same behaviour to this
circuit, then we have a complete algebraic semantics.

The map from Mealy machines to circuits also enforces the structure of the
Mealy core: it must be in the image of the map \(\mealytofunc\) from \(\funci\)
to \(\scircsigma\).
For a given interpretation this map essentially specifies the `canonical'
version of any combinational circuit.
So as not to work with arbitrary cores, it is preferable to first translate
combinational circuits into this canonical circuit; this allows us to restrict
the class of circuits any future equations must apply to.

\begin{definition}[Normalised circuit]
    A sequential circuit \(
        \iltikzfig{strings/category/f}[box=f,colour=seq,dom=\listvar{m},cod=\listvar{n}]
    \) is \emph{normalised} if it is in the image of \(\mealytofunc\).
\end{definition}

\begin{remark}
    Recall that even though \(\mealytofunc\) maps into \(\scircsigma\), every
    circuit in its image has combinational behaviour; the image of
    \(\mealytofunc\) is constrained in such a way that the instantaneous values
    can be used as constants.
\end{remark}

As the normalised version of a given circuit is interpretation-dependent, there
is no standard set of equations for normalising a circuit.
Instead, these must be specified on a interpretation-by-interpretation basis.

\begin{definition}[Normalising equations]
    For a functionally complete interpretation \(\interpretation\), a set of
    equations \(\normalisingequations\) is \emph{normalising} if any
    compositional circuit \(
        \iltikzfig{strings/category/f}[box=f,colour=comb,dom=\listvar{m},cod=\listvar{n}]
    \) is equal to a circuit in the image of \(\mealytofunc\) by equations in
    \(\normalisingequations\).
\end{definition}

The aim of the normalising equations for an interpretation translate a
combinational core into a form from which it is easy to read off a truth table.

\subsection{Case study: The Belnap interpretation}\label{sec:algebraic-case-study}

\todo[inline]{The normalising equations for the Belnap interpretation}
