\section{Normalising circuits}

How does one start when trying to define a complete set of equations for some
framework?
Usually the strategy is to define enough equations to bring any term to some
sort of (pseudo-)\emph{normal form}; the theory is then complete if terms with
the same semantics have the same normal form.

We have already seen something that looks a bit like a normal form: the
\emph{Mealy form} from the previous section.
This is by no means a true normal form, as there are many different Mealy forms
that represent the same behaviour.
Nevertheless, it is a useful starting point so we will need equations to bring
circuits to Mealy form in our theory.

We could just naively port across the Mealy rules from the operational semantics
by changing the squiggly arrow into an equals sign and hoping nobody notices,
but we are better people than that.
Instead we will show how Mealy form can be derived using a finite set of
equations, rather than with an equation parameterised over all circuits.

\begin{figure}
    \centering
    \(
    \equationdisplay{
        \iltikzfig{strings/structure/monoid/unitality-l-lhs}
    }{
        \iltikzfig{strings/structure/monoid/unitality-l-rhs}
    }{
        \joinunitleqn
    }
    \qquad
    \equationdisplay{
        \iltikzfig{strings/structure/monoid/unitality-r-lhs}
    }{
        \iltikzfig{strings/structure/monoid/unitality-r-rhs}
    }{
        \joinunitreqn
    }
    \)
    \\[0.25em]
    \rule{\textwidth}{0.1mm}
    \\[0.5em]
    \(
    \equationdisplay{
        \iltikzfig{circuits/axioms/bottom-delay-lhs}
    }{
        \iltikzfig{circuits/axioms/bottom-delay-rhs}
    }{
        \bottomdelayeqn
    }
    \qquad
    \equationdisplay{
        \iltikzfig{circuits/instant-feedback/equation-lhs}[box=f]
    }{
        \iltikzfig{circuits/instant-feedback/equation-rhs}[box=f]
    }{
        \instantfeedbackeqn
    }
    \)
    \\[0.25em]
    \rule{\textwidth}{0.1mm}
    \caption{
        Set of Mealy equations
        \(\mealyequations\).
    }
    \label{fig:mealy-equations}
\end{figure}

\begin{definition}
    The set \(\mealyequations\) of \emph{Mealy equations} in
    \cref{fig:mealy-equations} are sound.
\end{definition}
\begin{proof}
    The first two rules hold as the join is a monoid in the stream semantics.
    The \((\bottomdelayeqn)\) holds because the semantics of the delay
    component are to output a \(\bot\) value first and then the (delayed)
    inputs: as the semantics of the \(
    \iltikzfig{strings/structure/monoid/init}[colour=comb]
    \) component are to \emph{always} produce \(\bot\), then it does not make a
    difference how delayed it is.
    The final equation is the instant feedback rule, which is sound by
    \cref{prop:instant-feedback}.
\end{proof}

\begin{proposition}\label{prop:mealy-equations}
    Given a sequential circuit \(
    \iltikzfig{strings/category/f}[box=F,colour=seq]
    \), there exists a circuit in Mealy form such that \(
    \iltikzfig{strings/category/f}[box=F,colour=seq]
    =
    \iltikzfig{circuits/productivity/mealy-form}
    \) in \(\scircsigma / \mealyequations\).
\end{proposition}
\begin{proof}
    Any circuit can be assembled into global trace-delay form solely using the
    axioms of STMCs, which hold by default in \(\scircsigma\).
    From this, a circuit in pre-Mealy form can be obtained by translating
    delays into registers using \(
    \iltikzfig{circuits/axioms/delay-to-register/step-0}
    \eqaxioms[(\monoidunitleqn)]
    \iltikzfig{circuits/axioms/delay-to-register/step-1}
    \) and translating values into registers using \(
    \iltikzfig{circuits/axioms/value-to-register/step-0}
    \eqaxioms[(\monoidunitreqn)]
    \iltikzfig{circuits/axioms/value-to-register/step-1}
    \eqaxioms[(\bottomdelayeqn)]
    \iltikzfig{circuits/axioms/value-to-register/step-2}
    \).
    Subsequently a circuit in Mealy form can be obtained by applying the
    \((\instantfeedbackeqn)\) rule.
\end{proof}

\(\scircsigma / \mealyequations\) is a category in which all circuits are equal
to at least one circuit in Mealy form.
In general, there will be many Mealy forms depending on the ordering one picks
for the delays and values.
Our task is to provide equations to map any two denotationally equivalent
circuits to the \emph{same} Mealy form.

Even if the combinational cores of two Mealy forms have the same behaviour, they
may not have the same structure.
To reduce the number of cores we have to consider, we will first establish
equations for translating any combinational circuit into some canonical circuit.
We already met a method for determining what this canonical circuit is: the
functional completeness map \(\mealytofunc\) from \(\funci\) to \(\scircsigma\).

\begin{definition}[Normalised circuit]
    A sequential circuit \(
    \iltikzfig{strings/category/f}[box=f,colour=seq,dom=\listvar{m},cod=\listvar{n}]
    \) is \emph{normalised} if it is in the image of \(\mealytofunc\).
\end{definition}

Recall that even though \(\mealytofunc\) maps into \(\scircsigma\), every
circuit in its image has combinational behaviour.
This is quite an important distinction to make, so we will give it a proper
name.

\begin{lemma}[Essentially combinational]
    A sequential circuit is \emph{essentially combinational} if it is in the
    form \(
    \iltikzfig{circuits/synthesis/normalised-function}
    \).
\end{lemma}

Essentially combinational circuits are sequential circuits that exhibit
combinational behaviour: any value components are only used to introduce
constants which do not alter over time.

As the normalised version of a given circuit is interpretation-dependent, there
is no standard set of equations for normalising a circuit.
Instead, these must be specified on a interpretation-by-interpretation basis.

\begin{definition}[Normalising equations]
    For a functionally complete interpretation \(\interpretation\), a set of
    equations \(\normalisingequations\) is \emph{normalising} if any
    essentially combinational circuit \(
    \iltikzfig{strings/category/f}[box=f,colour=seq,dom=\listvar{m},cod=\listvar{n}]
    \) is equal to a circuit in the image of \(\mealytofunc\) by equations in
    \(\normalisingequations\).
\end{definition}

The normalising equations for a given interpretation can be used to translate a
combinational core into a circuit from which it is easy to read off a truth
table.