\section{Encoding equations}

While the core of a circuit in normalised Mealy form is the canonical
representation of that circuit's Mealy function, there is no reason that
circuits with the same behaviour should have the same Mealy function.
Even if the state words are the same, they may be jumbled up in different
orders.
This is why checking equivalence of Mealy machines is usually performed by
constructing a \emph{bisimulation}; what we need is a way to `hot-swap' the
states and cores of circuits in normalised Mealy form with bisimilar ones.

Of course, there are also (infinitely) many bisimilar Mealy machines and state
words and no clear way to map between them, so we need to fix a particular
class of functions and find a deterministic way to map any circuit to this
class.
The obvious choice, to which we have previously alluded, is the `one-hot'
encoding used to synthesise a circuit morphism from a Mealy machine.
To do this we must compute the number of states a circuit might assume, so we
know the width of word we will need to encode to; this is where isolating the
transition and output of the Mealy core will come in useful.

\begin{definition}[Circuit states]
    Given a circuit \(
    \iltikzfig{strings/category/f}[box=f,colour=seq,dom=\listvar{m},cod=\listvar{n}]
    \) in normalised Mealy form such that \(
    \iltikzfig{strings/category/f}[box=f,colour=seq]
    \coloneqq
    \iltikzfig{circuits/productivity/mealy-form}[core=|f|,colour=seq,delay=\listvar{x}]
    \), let the set of \emph{circuit states of \(f\) from \(\listvar{s}\)},
    written \(S(f,\listvar{s})\), be the smallest set containing \(\listvar{s}\)
    and closed under \(
    \listvar{r} \mapsto \proj{x}\left(\tilde{f}_0(\listvar{r},\listvar{v})\right)
    \) for any \(\listvar{v} \in \valuetuple{\listvar{m}}\).
\end{definition}

The set of circuit states is essential in determining valid equations between
circuits in normalised Mealy form.
The first thing to consider when comparing two circuits in normalised Mealy
form is that the circuits may not have the same states, or even have state words
of equal width.
One of our families of equations will be based around \emph{encoding} states
of a circuit into different representations.

\begin{definition}[Encoder]\label{def:encoder}
    Let \(S \subseteq \valuetuple{x}\) and \(T \subseteq \valuetuple{y}\) be two
    sets; an \emph{(S,T)-encoder} is two functions \(
    \morph{\mathsf{enc}_{S,T}}{S}{T}
    \) and \(
    \morph{\mathsf{dec}_{S,T}}{T}{S}
    \) such that \(dec_{S,T} \circ enc_{S,T}\).
\end{definition}

\begin{example}
    Given a set of states \(S \subseteq \valuetuple{x}\), a \(\leq\)-encoding
    \(\morph{\gamma_\leq}{S}{\valuetuple{k}}\) used when defining the map from
    Mealy machines to circuits (\cref{def:encoding}) is an example of a
    \((S,\valuetuple{k})\)-encoder; the decoder
    \(\mathsf{dec}_{S,\valuetuple{x}}\) is defined as the monotone completion
    of \(\gamma_\leq^{-1}\).
\end{example}

Using a decoder and an encoder, we can change the state words that a circuit
operates with.

\begin{proposition}[Encoding equation]\label{prop:encoding-equation}
    Let \(\iltikzfig{strings/category/f}[box=f,colour=seq]\) be a circuit in
    normalised Mealy form such that \(
    \iltikzfig{strings/category/f}[box=f,colour=seq]
    \coloneqq
    \iltikzfig{circuits/productivity/mealy-form}[core=|\hat{f}|,colour=seq,delay=\listvar{x}]
    \) and let \(\mathsf{enc},\mathsf{dec}\) be the two functions of a
    \((S(\hat{f},\listvar{s})),T\)-encoder as defined above.
    Then the \emph{encoding equation} \((\encodingequation)\) in
    \cref{fig:encoding-equation} is sound.
\end{proposition}
\begin{proof}
    The interpretations of \(
    \iltikzfig{strings/category/f-2-1}[box={\transitiontranslation{f}},colour=seq,dom1=\listvar{x},dom2=\listvar{m},cod=\listvar{x}]
    \) and \(
    \iltikzfig{strings/category/f-2-1}[box={\outputtranslation{f}},colour=seq,dom1=\listvar{x},dom2=\listvar{m},cod=\listvar{n}]
    \) are as the behaviours of \(
    \iltikzfig{strings/category/f-2-1}[box={f},colour=seq,dom1=\listvar{x},dom2=\listvar{m},cod=\listvar{x}]
    \) and \(
    \iltikzfig{strings/category/f-2-1}[box={g},colour=seq,dom1=\listvar{x},dom2=\listvar{m},cod=\listvar{x}]
    \) wrapped in appropriate decodings and encodings.
\end{proof}

\begin{notation}[Restriction]
    Given a function \(\morph{f}{A}{B}\), and a set
    \(C \subseteq A\), let \(\morph{f|C}{C}{B}\) be the \emph{restriction} of
    \(f\) to act on elements of \(C\), i.e.\ for all \(c \in C\),
    \(f|C(c) = f(c)\).
\end{notation}

\begin{definition}
    Let \(
    \iltikzfig{circuits/productivity/mealy-form}[core=|f|,colour=seq,delay=\listvar{x}]
    \) be a circuit in normalised Mealy form with state set \(S(f, \listvar{s})\).
    Then, for a total order \(\leq\) on \(S(f, \listvar{s})\), the
    \emph{\(\leq\)-encoding circuit} is a
\end{definition}

\begin{definition}[State change equation]

\end{definition}



\begin{definition}[Restriction equations]
    Given two circuits in normalised Mealy form \(
    \iltikzfig{circuits/productivity/mealy-form}[core=|f|,colour=seq]
    \) and \(
    \iltikzfig{circuits/productivity/mealy-form}[core=|g|,colour=seq]
    \)  a \emph{restriction equation}
\end{definition}

Since the transition subcircuit of a circuit in normalised Mealy form is itself
a normalised function, we can easily perform a lookup to find out what the
behaviour of \(\tilde{f}\) is.
The size of the set of circuit states determines the width of the encoded state.

\begin{definition}[Translations]
    For a set of states \(S\), let \(\morph{\gamma_\leq}{S}{\valuetuple{y}}\) be
    a \(\leq\)-encoding.
    For a circuit \(
    \iltikzfig{strings/category/f-2-1}[box=f, colour=seq, dom1=\listvar{x}, dom2=\listvar{m}, cod=\listvar{x}]
    \coloneqq
    \mealytofunc[\hat{f}]
    \), its \emph{\(\gamma_\leq\)-transition} is a circuit \(
    \iltikzfig{strings/category/f-2-1}[box=\transitiontranslation{f}, colour=seq, dom1=\listvar{x}, dom2=\listvar{m}, cod=\listvar{x}]
    \) defined as \(
    \mealytofunc[
        \left(\gamma_\leq(\listvar{s}),\listvar{v}\right)
        \mapsto \gamma_\leq(\hat{f}(\listvar{s}, \listvar{v}))
    ]
    \), and for a circuit \(
    \iltikzfig{strings/category/f-2-1}[box=g, colour=seq, dom1=\listvar{x}, dom2=\listvar{m}, cod=\listvar{x}]
    \coloneqq
    \mealytofunc[\hat{g}]
    \), its \emph{\(\gamma_\leq\)-output} is a circuit \(
    \iltikzfig{strings/category/f-2-1}[box=\outputtranslation{f}, colour=seq, dom1=\listvar{x}, dom2=\listvar{m}, cod=\listvar{x}]
    \) defined as \(
    \mealytofunc[
        \left(\gamma_\leq(\listvar{s}),\listvar{v}\right)
        \mapsto \hat{g}(\listvar{s}, \listvar{v})
    ].
    \)
\end{definition}

These two translations are the the essential components of our final equation,
the \emph{encoding} equation, which is parameterised by an order \(\leq\) on the
states which specifies how to map a circuit in normalised Mealy form to
its corresponding \(\leq\)-encoded form.

\begin{proposition}[Encoding equation]\label{prop:encoding-equation}
    Given two normalised circuits \(
    \iltikzfig{strings/category/f-2-1}[box=f,colour=seq,dom1=\listvar{x},dom2=\listvar{m},cod=\listvar{x}]
    \) and \(
    \iltikzfig{strings/category/f-2-1}[box=f,colour=seq,dom1=\listvar{x},dom2=\listvar{m},cod=\listvar{n}]
    \) along with word \(\listvar{s} \in \valuetuple{\listvar{x}}\), let
    \(\leq\) be an ordering on \(S(f,\listvar{s})\).
    Then the \emph{encoding equation} \((\encodingequation)\) in
    \cref{fig:encoding-equation} is sound.
\end{proposition}
\begin{proof}
    The interpretations of \(
    \iltikzfig{strings/category/f-2-1}[box={\transitiontranslation{f}},colour=seq,dom1=\listvar{x},dom2=\listvar{m},cod=\listvar{x}]
    \) and \(
    \iltikzfig{strings/category/f-2-1}[box={\outputtranslation{f}},colour=seq,dom1=\listvar{x},dom2=\listvar{m},cod=\listvar{n}]
    \) are as the behaviours of \(
    \iltikzfig{strings/category/f-2-1}[box={f},colour=seq,dom1=\listvar{x},dom2=\listvar{m},cod=\listvar{x}]
    \) and \(
    \iltikzfig{strings/category/f-2-1}[box={g},colour=seq,dom1=\listvar{x},dom2=\listvar{m},cod=\listvar{x}]
    \) wrapped in appropriate decodings and encodings.
\end{proof}

\begin{figure}
    \centering
    \(
    \equationdisplay{
        \iltikzfig{circuits/productivity/mealy-form}[core=|f|,delay=x, colour=seq]
    }{
        \iltikzfig{circuits/algebraic/state-encoding}[core=|f|,delay=x]
    }{
        \encodingequation
    }
    \)
    \\[0.25em]
    \rule{\textwidth}{0.1mm}
    \\[0.5em]
    \(
    \equationdisplay{
        \iltikzfig{circuits/axioms/gate-lhs}
    }{
        \iltikzfig{circuits/axioms/gate-rhs}
    }{
        \gateeqn
    }
    \quad
    \equationdisplay{
        \iltikzfig{circuits/axioms/fork-lhs}
    }{
        \iltikzfig{circuits/axioms/fork-rhs}
    }{
        \forkeqn
    }
    \quad
    \equationdisplay{
        \iltikzfig{circuits/axioms/join-lhs}
    }{
        \iltikzfig{circuits/axioms/join-rhs}
    }{
        \joineqn
    }
    \)
    \\[0.25em]
    \rule{\textwidth}{0.1mm}
    \\[0.5em]
    \(
    \equationdisplay{
        \iltikzfig{circuits/axioms/stub-lhs}
    }{
        \iltikzfig{strings/monoidal/empty}
    }{
        \stubeqn
    }
    \quad
    \equationdisplay{
        \iltikzfig{circuits/axioms/delay-fork-lhs}
    }{
        \iltikzfig{circuits/axioms/delay-fork-rhs}
    }{
        \delayforkeqn
    }
    \quad
    \equationdisplay{
        \iltikzfig{circuits/axioms/bottom-delay-lhs}
    }{
        \iltikzfig{circuits/axioms/bottom-delay-rhs}
    }{
        \bottomdelayeqn
    }
    \quad
    \equationdisplay{
        \iltikzfig{circuits/axioms/streaming-lhs}
    }{
        \iltikzfig{circuits/axioms/streaming-rhs}[gate=p]
    }{
        \streamingeqn
    }
    \)
    \\[0.25em]
    \rule{\textwidth}{0.1mm}
    \\[0.5em]
    \(
    \equationdisplay{
        \iltikzfig{strings/structure/comonoid/unitality-l-lhs}
    }{
        \iltikzfig{strings/structure/comonoid/unitality-l-rhs}
    }{
        \comonoiduniteqnletter
    }
    \quad
    \equationdisplay{
        \iltikzfig{strings/structure/monoid/associativity-lhs}
    }{
        \iltikzfig{strings/structure/monoid/associativity-rhs}
    }{
        \monoidassoceqnletter
    }
    \quad
    \equationdisplay{
        \iltikzfig{strings/structure/monoid/commutativity-lhs}
    }{
        \iltikzfig{strings/structure/monoid/commutativity-rhs}
    }{
        \monoidcommeqnletter
    }
    \quad
    \equationdisplay{
        \iltikzfig{strings/structure/bialgebra/merge-copy-lhs}
    }{
        \iltikzfig{strings/structure/bialgebra/merge-copy-rhs}
    }{
        \joinforkeqn
    }
    \)
    \caption{
        Equations for encoding circuit states
    }
    \label{fig:encoding-equation}
\end{figure}

\begin{remark}
    The encoding equation is really a \emph{family} of equations parameterised
    over the circuits \(
    \iltikzfig{strings/category/f}[box=f,colour=seq]
    \) and \(
    \iltikzfig{strings/category/f}[box=g,colour=seq]
    \).
    Since we have the precondition that these circuits are normalised, this is
    not quite every circuit in the universe but there will still be an equation
    for each possible function \(
    \valuetuple{\listvar{m}} \to \valuetuple{\listvar{n}}
    \), of which there are infinitely many.
    any old circuit.
    What is important is that given the two subcircuits and state word of a
    circuit in normalised Mealy form, one can compute the set of circuit states
    and subsequently the encoded circuit.
\end{remark}

The circuit on the right hand side of the encoding equation has a state
containing just \(\bot\) and \(\top\) values.
This is the `pseudo-normal form' for sequential circuits; depending on the
state order picked there are likely to be multiple circuits of this form.
Fortunately, since these circuits are also in normalised Mealy form, the
encoding equation can be used to translate between pseudo-normal forms as well.

\begin{definition}
    For an interpretation \(\interpretation\), let
    \(\mce_{\interpretation}\) be defined as \(
    \mealyequations +
    \instantfeedbackeqn +
    \cartesianequations +
    \normalisingequations +
    \encodingequation
    \), and let \(\scircsigmae\) be defined as
    \(\scircsigma / \mce_{\interpretation}\).
\end{definition}

\begin{theorem}
    For a functionally complete interpretation \(\interpretation\), \(
    \iltikzfig{strings/category/f}[box=f,colour=seq,dom=\listvar{m},cod=\listvar{n}]
    =
    \iltikzfig{strings/category/f}[box=g,colour=seq,dom=\listvar{m},cod=\listvar{n}]
    \) in \(\scircsigmae\) if and only if \(
    \circuittostreami[
        \iltikzfig{strings/category/f}[box=f,colour=seq,dom=\listvar{m},cod=\listvar{n}]
    ]
    =
    \circuittostreami[
        \iltikzfig{strings/category/f}[box=g,colour=seq,dom=\listvar{m},cod=\listvar{n}]
    ]
    \).
\end{theorem}
\begin{proof}
    All the equations are sound, so we only need to consider the \(\ifdir\)
    direction.
    By \cref{lem:normalised-mealy}, the two circuits can be brought to
    normalised Mealy form using
    \(
    \mealyequations +
    \instantfeedbackeqn +
    \cartesianequations +
    \normalisingequations
    \).
    By \cref{def:mealy-to-circuit} and
    \cref{thm:circuit-stream-correspondence} there must exist a circuit \(
    \iltikzfig{strings/category/f}[box=h,colour=seq]
    \coloneqq
    \iltikzfig{circuits/algebraic/encoding}[transition={h_0},output={h_1},state={\listvar{s}}]
    \) such that \(
    \circuittostream[
        \iltikzfig{strings/category/f}[box=f,colour=seq]
    ]{\interpretation}
    =
    \circuittostream[
        \iltikzfig{strings/category/f}[box=h,colour=seq]
    ]{\interpretation}
    =
    \circuittostream[
        \iltikzfig{strings/category/f}[box=g,colour=seq]
    ]{\interpretation}
    \).
    The circuit \(
    \iltikzfig{strings/category/f}[box=H,colour=seq]
    \) is encoded such that the state words are in the image of
    \(\gamma_\leq\).
    This means that applying the encoding equation with \(\leq\) to the
    normalised Mealy forms obtained above will yield the circuit \(
    \iltikzfig{strings/category/f}[box=h,colour=seq]
    \).
\end{proof}

As always, the soundness and completeness of the algebraic semantics means we
can establish another isomorphism of PROPs.

\begin{corollary}
    \(\scircsigmai \cong \scircsigmae\).
\end{corollary}

This brings our jaunt into algebraic semantics to a close; given a circuit \(
\iltikzfig{strings/category/f}[box=f,colour=seq,dom=\listvar{m},cod=\listvar{n}]
\) we know that we can translate it into another circuit \(
\iltikzfig{strings/category/f}[box=g,colour=seq,dom=\listvar{m},cod=\listvar{n}]
\) with the same behaviour by only using equations in \(\mce_\interpretation\).

Of course, the procedure of using the normalisation equations to translate a
circuit into normalised Mealy form before using the encoding equation (possibly
multiple times depending on how lucky one gets with their orderings) may be
tedious; one might wonder how this is beneficial to the operational approach in
the previous chapter.
But the beauty of the \emph{algebraic} semantics is that we \emph{don't} need to
do this every time!
Equations can be proven as lemmas and then used repeatedly in the future as
`shortcuts', possibly saving many reasoning steps.
In time, the algebraicist will build up a powerful repertoire of equations and
wield them to bend circuits to their will.