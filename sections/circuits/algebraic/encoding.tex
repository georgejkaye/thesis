\section{Encoding equations}

While the core of a circuit in normalised Mealy form is the canonical
representation of that circuit's Mealy function, there is no reason that
circuits with the same behaviour should have the same Mealy function.
Even if the state words are the same, they may be jumbled up in different
orders.
This is why checking equivalence of Mealy machines is usually performed by
constructing a \emph{bisimulation}; what we need is a way to `hot-swap' the
states and cores of circuits in normalised Mealy form with bisimilar ones.

Of course, there are also (infinitely) many bisimilar Mealy machines and state
words and no clear way to map between them, so we need to fix a particular
class of functions and find a deterministic way to map any circuit to this
class.
The obvious choice, to which we have previously alluded, is the `one-hot'
encoding used to synthesise a circuit morphism from a Mealy machine.
To do this we must compute the number of states a circuit might assume, so we
know the width of word we will need to encode to; this is where isolating the
transition and output of the Mealy core will come in useful.

\begin{definition}[Circuit states]
    Given a circuit in normalised Mealy form \(
        \iltikzfig{circuits/axioms/encoding}[transition={f_0},output={f_1},dom=\listvar{m},cod=\listvar{n},delay=\listvar{x}]
    \), let the set of \emph{circuit states}, written \(S(f_0),\listvar{s}\),
    be the smallest set containing \(\listvar{s}\) and closed under \(
        \listvar{r} \mapsto \tilde{f}_0(\listvar{r},\listvar{v})
    \) for any \(\listvar{v} \in \valuetuple{\listvar{m}}\).
\end{definition}

Since the transition subcircuit of a circuit in normalised Mealy form is itself
a normalised function, we can easily perform a lookup to find out what the
behaviour of \(\tilde{f}\) is.
The size of the set of circuit states determines the width of the encoded state.

\begin{definition}[Translations]
    For a set of states \(S\), let \(\morph{\gamma_\leq}{S}{\valuetuple{y}}\) be
    a \(\leq\)-encoding.
    For a circuit \(
        \iltikzfig{strings/category/f-2-1}[box=f, colour=seq, dom1=\listvar{x}, dom2=\listvar{m}, cod=\listvar{x}]
        \coloneqq
        \mealytofunc[\hat{f}]
    \), its \emph{\(\gamma_\leq\)-transition} is a circuit \(
        \iltikzfig{strings/category/f-2-1}[box=f^\leq_0, colour=seq, dom1=\listvar{x}, dom2=\listvar{m}, cod=\listvar{x}]
    \) defined as \(
        \mealytofunc[
            \left(\gamma_\leq(\listvar{s}),\listvar{v}\right)
            \mapsto \gamma_\leq(f(\listvar{s}, \listvar{v}))
        ]
    \), and for a circuit \(
        \iltikzfig{strings/category/f-2-1}[box=g, colour=seq, dom1=\listvar{x}, dom2=\listvar{m}, cod=\listvar{x}]
        \coloneqq
        \mealytofunc[\hat{g}]
    \), its \emph{\(\gamma_\leq\)-output} is a circuit \(
        \iltikzfig{strings/category/f-2-1}[box=g^\leq_1, colour=seq, dom1=\listvar{x}, dom2=\listvar{m}, cod=\listvar{x}]
    \) defined as \(
        \mealytofunc[
            \left(\gamma_\leq(\listvar{s}),\listvar{v}\right)
            \mapsto \gamma_\leq(())
        ].
    \)
\end{definition}

