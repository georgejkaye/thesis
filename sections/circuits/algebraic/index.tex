\chapter{Algebraic semantics}

Testing that the behaviour of two circuits is equivalent by testing that every
input produces the same output for each circuit is a perfectly reasonable
strategy.
But this is somewhat `nuclear'; rather than using what we know
about the structure of a circuit's subcomponents, we just blast away
exhaustively trying all the inputs to find a contradiction.

A more elegant method of reasoning is by defining a set \emph{equations} between
subcircuits and \emph{quotienting} \(\scircsigma\) by these equations.
A proof of equivalence between two circuits is then presented using algebraic
reasoning: applying equations to translate one circuit into the other.
This is often \emph{far more efficient} than having to test every input!
This is the focus of our final perspective on semantics for sequential
circuits, \emph{algebraic semantics}.

\begin{example}\label{ex:expressions-algebraic}
    For the last time we return to the language of arithmetical expressions from
    \cref{ex:expressions-denotational}.
    An algebraic semantics for this language can be defined using a set of
    equations: the familiar equations of associativity, commutativity,
    unitality, annihiliation and distributivity, along with equations for
    actually performing arithmetic.
    \begin{gather*}
        add\,(add\,e_1\,e_2)\,e_3 = add\,e_1\,(add\,e_2\,e_3)
        \qquad
        mul\,(mul\,e_1\,e_2)\,e_3 = mul\,e_1\,(mul\,e_2\,e_3)
        \\
        add\,e_1\,e_2 = add\,e_2\,e_1
        \qquad
        mul\,e_1\,e_2 = mul\,e_2\,e_1
        \\
        add\,e_1\,\overline{0} = e_1
        \qquad
        mul\,e_1\,\overline{1} = e_1
        \qquad
        mul\,e_1\,\overline{0} = \overline{0}
        \\
        mul\,e_1\,(add\,e_2\,e_3) = add\,(mul\,e_1\,e_2)\,(mul\,e_1\,e_3)
        \\
        add\,\overline{n_1}\,\overline{n_2} = add\,\overline{n_1+n_2}
        \qquad
        add\,\overline{n_1}\,\overline{n_2} = \overline{n_1 \cdot n_2}
    \end{gather*}
    If everything is specified concretely as values then one could easily just
    use the last two equations to compare two expressions by reducing
    two expressions to values as in the operational semantics.
    The power of the algebraic semantics comes from the fact we can reason
    abstractly with expressions containing \emph{blackboxes}.
    Take the following example, containing some arbitrary component \(e\).
    \begin{align*}
        mul\,(add\,e\,(mul\,e\,\overline{3}))\,\overline{2}
        &=
        mul\,(add\,(mul\,e\,1)\,(mul\,e\,\overline{3}))\,\overline{2}
        \\
        &=
        mul\,(mul\,e\, add(\overline{1}\,\overline{3}))\,\overline{2}
        \\
        &=
        mul\,(mul \,e\,\overline{4})\,\overline{2}
        \\
        &=
        mul\,e\,(mul \,\overline{4}\,\overline{2})
        \\
        &=
        mul\,e\,\overline{8}
        \\
        &=
        mul\,\overline{8}\,e
    \end{align*}
    Despite not specifying the structure of \(e\), we have
    shown how the expression is equal to a slightly simpler one.
    This process creates new general equations which can be used as `shortcuts'
    in future reasoning, potentially saving many steps.
\end{example}

As with the operational semantics, we are especially interested in defining a
\emph{sound and complete} algebraic semantics for sequential digital circuits
with respect to the denotational semantics.
That is to say, for each equation \(
    \iltikzfig{strings/category/f}[box=f,colour=seq]
    =
    \iltikzfig{strings/category/f}[box=g,colour=seq]
\) then \(
    \circuittostreami[\iltikzfig{strings/category/f}[box=f,colour=seq]]
    =
    \circuittostreami[\iltikzfig{strings/category/f}[box=g,colour=seq]]
\), and there must be enough equations such that if \(
    \circuittostreami[\iltikzfig{strings/category/f}[box=f,colour=seq]]
    =
    \circuittostreami[\iltikzfig{strings/category/f}[box=g,colour=seq]]
\) then there exists a series of equations identifying \(
    \circuittostreami[\iltikzfig{strings/category/f}[box=f,colour=seq]]
\) and \(
    \circuittostreami[\iltikzfig{strings/category/f}[box=f,colour=seq]]
\).

\begin{remark}
    An `equational theory' for sequential circuits was one of the first things
    presented in the original circuits
    paper~\cite{ghica2016categorical,ghica2017diagrammatic}.
    In that paper, equations that `seemed right' were used to quotient the
    syntax, with the ultimate aim of showing that the resulting category was
    \emph{Cartesian}.
    This was done quite informally, and was made more confusing as
    the categories of circuits were subsequently quotiented by some notion of
    `extensional equivalence', an attempt to rectify the fact that the
    equations only dealt with closed circuits.
    Soundness and completeness of the equational theory was not considered
    because there was no other semantics to compare it against!

    In essence, the previous work was almost `the wrong way round': equations
    were defined and semantics drawn from them.
    In the more recent version of the work~\cite[Sec. 5]{ghica2024fully}, which
    forms the basis for this chapter, the equations are derived from the
    denotational semantics.
    Not only does this give us a formal way of verifying that these equations
    are sound, it sets the backdrop against which we can test if the algebraic
    semantics are sufficient: are any two denotationally equal circuits
    identified by equations?

    Fans of the original work should fear not, because what came before has not
    been entirely discarded.
    Most of the original equations have found a home as rules of the operational
    semantics in the previous chapter, and the fact that categories of circuits
    should be Cartesian will crop up later on; stay tuned!
\end{remark}

\section{Mealy equations}

Although we have now left operational semantics behind in the previous chapter,
the notion of Mealy form will come in very handy for defining a sound and
complete algebraic semantics.
We could just naively port across the Mealy rules by changing the squiggly arrow
into an equals sign and hoping nobody notices, but we are better people than
that.
Instead we will show how Mealy form can be derived using a finite set of
equations, rather than resorting to equations parameterised over all circuits.

\begin{figure}
    \centering
    \(
        \equationdisplay{
            \iltikzfig{strings/structure/monoid/unitality-l-lhs}
        }{
            \iltikzfig{strings/structure/monoid/unitality-l-rhs}
        }{
            \monoidunitleqn
        }
        \quad
        \equationdisplay{
            \iltikzfig{strings/structure/monoid/unitality-r-lhs}
        }{
            \iltikzfig{strings/structure/monoid/unitality-r-rhs}
        }{
            \monoidunitreqn
        }
        \quad
        \equationdisplay{
            \iltikzfig{circuits/axioms/bottom-delay-lhs}
        }{
            \iltikzfig{circuits/axioms/bottom-delay-rhs}
        }{
            \bottomdelayeqn
        }
        \quad
        \equationdisplay{
            \iltikzfig{circuits/instant-feedback/equation-lhs}
        }{
            \iltikzfig{circuits/instant-feedback/equation-rhs}
        }{
            \instantfeedbackeqn
        }
    \)
    \caption{
        Set of Mealy equations
        \(\mealyequations\).
    }
    \label{fig:mealy-equations}
\end{figure}

\begin{definition}
    The set \(\mealyequations\) of \emph{Mealy equations} in
    \cref{fig:mealy-equations} are sound.
\end{definition}
\begin{proof}
    The first two rules hold as the join is a monoid in the stream semantics.
    The \((\bottomdelayeqn)\) holds because the semantics of the delay
    component are to output a \(\bot\) value first and then the (delayed)
    inputs: as the semantics of the \(
        \iltikzfig{strings/structure/monoid/init}[colour=comb]
    \) component are to \emph{always} produce \(\bot\), then it does not make a
    difference how delayed it is.
    The final equation is just the instant feedback rule from the previous
    chapter, which is sound by \cref{prop:instant-feedback}.
\end{proof}

\begin{proposition}
    Given a sequential circuit \(
        \iltikzfig{strings/category/f}[box=F,colour=seq]
    \), there exists a circuit in Mealy form such that \(
        \iltikzfig{strings/category/f}[box=F,colour=seq]
        =
        \iltikzfig{circuits/productivity/mealy-form}
    \) in \(\scircsigma / \mealyequations\).
\end{proposition}
\begin{proof}
    Any circuit can be assembled into global trace-delay form solely using the
    axioms of STMCs, which hold by default in \(\scircsigma\).
    From this, a circuit in pre-Mealy form can be obtained by translating
    delays into registers using \(
        \iltikzfig{circuits/axioms/delay-to-register/step-0}
        \eqaxioms[(\monoidunitleqn)]
        \iltikzfig{circuits/axioms/delay-to-register/step-1}
    \) and translating values into registers using \(
        \iltikzfig{circuits/axioms/value-to-register/step-0}
        \eqaxioms[(\monoidunitreqn)]
        \iltikzfig{circuits/axioms/value-to-register/step-1}
        \eqaxioms[(\bottomdelayeqn)]
        \iltikzfig{circuits/axioms/value-to-register/step-2}
    \).
    Subsequently a circuit in Mealy form can be obtained by applying the
    \((\instantfeedbackeqn)\) rule.
\end{proof}

\(\scircsigma / \mealyequations\) is a category in which all circuits are equal
to at least one circuit in Mealy form.
In general, there will be many Mealy forms depending on the ordering one picks
for the delays and values; while each Mealy form circuit will have the same
semantics, there is no reason that their combinational cores must as well.
\section{Normalising circuits}

How does one start when trying to define a complete set of equations for some
framework?
Usually the strategy is to define enough equations to bring any term to some
sort of (pseudo-)\emph{normal form}; the theory is then complete if terms with
the same semantics have the same normal form.

We have already seen something that looks a bit like a normal form: the
\emph{Mealy form} from the previous section.
This is by no means a true normal form, as there are many different Mealy forms
that represent the same behaviour.
Nevertheless, it is a useful starting point so we will need equations to bring
circuits to Mealy form in our theory.

We could just naively port across the Mealy rules from the operational semantics
by changing the squiggly arrow into an equals sign and hoping nobody notices,
but we are better people than that.
Instead we will show how Mealy form can be derived using a finite set of
equations, rather than resorting to equations parameterised over all circuits.

\begin{figure}
    \centering
    \(
        \equationdisplay{
            \iltikzfig{strings/structure/monoid/unitality-l-lhs}
        }{
            \iltikzfig{strings/structure/monoid/unitality-l-rhs}
        }{
            \monoidunitleqn
        }
        \quad
        \equationdisplay{
            \iltikzfig{strings/structure/monoid/unitality-r-lhs}
        }{
            \iltikzfig{strings/structure/monoid/unitality-r-rhs}
        }{
            \monoidunitreqn
        }
        \quad
        \equationdisplay{
            \iltikzfig{circuits/axioms/bottom-delay-lhs}
        }{
            \iltikzfig{circuits/axioms/bottom-delay-rhs}
        }{
            \bottomdelayeqn
        }
        \quad
        \equationdisplay{
            \iltikzfig{circuits/instant-feedback/equation-lhs}
        }{
            \iltikzfig{circuits/instant-feedback/equation-rhs}
        }{
            \instantfeedbackeqn
        }
    \)
    \caption{
        Set of Mealy equations
        \(\mealyequations\).
    }
    \label{fig:mealy-equations}
\end{figure}

\begin{definition}
    The set \(\mealyequations\) of \emph{Mealy equations} in
    \cref{fig:mealy-equations} are sound.
\end{definition}
\begin{proof}
    The first two rules hold as the join is a monoid in the stream semantics.
    The \((\bottomdelayeqn)\) holds because the semantics of the delay
    component are to output a \(\bot\) value first and then the (delayed)
    inputs: as the semantics of the \(
    \iltikzfig{strings/structure/monoid/init}[colour=comb]
    \) component are to \emph{always} produce \(\bot\), then it does not make a
    difference how delayed it is.
    The final equation is just the instant feedback rule from the previous
    chapter, which is sound by \cref{prop:instant-feedback}.
\end{proof}

\begin{proposition}\label{prop:mealy-equations}
    Given a sequential circuit \(
    \iltikzfig{strings/category/f}[box=F,colour=seq]
    \), there exists a circuit in Mealy form such that \(
    \iltikzfig{strings/category/f}[box=F,colour=seq]
    =
    \iltikzfig{circuits/productivity/mealy-form}
    \) in \(\scircsigma / \mealyequations\).
\end{proposition}
\begin{proof}
    Any circuit can be assembled into global trace-delay form solely using the
    axioms of STMCs, which hold by default in \(\scircsigma\).
    From this, a circuit in pre-Mealy form can be obtained by translating
    delays into registers using \(
    \iltikzfig{circuits/axioms/delay-to-register/step-0}
    \eqaxioms[(\monoidunitleqn)]
    \iltikzfig{circuits/axioms/delay-to-register/step-1}
    \) and translating values into registers using \(
    \iltikzfig{circuits/axioms/value-to-register/step-0}
    \eqaxioms[(\monoidunitreqn)]
    \iltikzfig{circuits/axioms/value-to-register/step-1}
    \eqaxioms[(\bottomdelayeqn)]
    \iltikzfig{circuits/axioms/value-to-register/step-2}
    \).
    Subsequently a circuit in Mealy form can be obtained by applying the
    \((\instantfeedbackeqn)\) rule.
\end{proof}

\(\scircsigma / \mealyequations\) is a category in which all circuits are equal
to at least one circuit in Mealy form.
In general, there will be many Mealy forms depending on the ordering one picks
for the delays and values.

One subclass of circuits in Mealy form which is particularly special is the
image of \(\mealytocircuiti\), the map from Mealy machines to circuit morphisms.
Recall that circuits in the image of \(\mealytocircuiti\) are in the form \(
\iltikzfig{circuits/synthesis/mealy-term-spaced}[trace=\listvar{x},dom=\listvar{m},cod=\listvar{n}]
\), where \(\gamma_\leq\) is an \emph{encoding} of the state set with respect
to some total order \(\leq\).
This encoding maps states to words containing only \(\bot\) and \(\top\) where
the width of each word is equal to the number of states in the original Mealy
machine.

From the completeness procedure for the denotational semantics, we know that any
circuit is guaranteed to be denotationally equivalent to a circuit in this form;
if we have equations to translate any circuits with the same behaviour to this
circuit, then we have a complete algebraic semantics.

The map from Mealy machines to circuits also enforces the structure of the
Mealy core: it must be in the image of the map \(\mealytofunc\) from \(\funci\)
to \(\scircsigma\).
For a given interpretation this map essentially specifies the `canonical'
version of any combinational circuit.
So as not to work with arbitrary cores, it is preferable to first translate
combinational circuits into this canonical circuit; this allows us to restrict
the class of circuits any future equations must apply to.

\begin{definition}[Normalised circuit]
    A sequential circuit \(
    \iltikzfig{strings/category/f}[box=f,colour=seq,dom=\listvar{m},cod=\listvar{n}]
    \) is \emph{normalised} if it is in the image of \(\mealytofunc\).
\end{definition}

\begin{remark}
    Recall that even though \(\mealytofunc\) maps into \(\scircsigma\), every
    circuit in its image has combinational behaviour; the image of
    \(\mealytofunc\) is constrained in such a way that the instantaneous values
    can be used as constants.
\end{remark}

As the normalised version of a given circuit is interpretation-dependent, there
is no standard set of equations for normalising a circuit.
Instead, these must be specified on a interpretation-by-interpretation basis.

\begin{definition}[Normalising equations]
    For a functionally complete interpretation \(\interpretation\), a set of
    equations \(\normalisingequations\) is \emph{normalising} if any
    compositional circuit \(
    \iltikzfig{strings/category/f}[box=f,colour=comb,dom=\listvar{m},cod=\listvar{n}]
    \) is equal to a circuit in the image of \(\mealytofunc\) by equations in
    \(\normalisingequations\).
\end{definition}

The aim of the normalising equations for an interpretation translate a
combinational core into a form from which it is easy to read off a truth table.

\subsection{Case study: The Belnap interpretation}\label{sec:algebraic-case-study}

\todo[inline]{The normalising equations for the Belnap interpretation}

\section{Encoding equations}

A circuit in Mealy form is a syntactic representation of a Mealy machine: it
has a combinational core acting as the Mealy machine. and registers containing
the initial state of the machine.
As we saw in the operational semantics, as a circuit is applied to inputs these
registers are updated with new values as the circuit transitions to new states.
When reasoning algebraically, we cannot evaluate these states as we have no
inputs to compute with.
However, the states a circuit \emph{might} assume will still be important as
they dictate whether an equation is valid.

\begin{definition}[States]
    Let \(\morph{f}{\valuetuple{\listvar{xm}}}{\valuetuple{\listvar{xn}}}\) be a
    monotone function and let \(\listvar{s} \in  \valuetuple{x}\) be an
    initial state.
    Then the \emph{states of \(f\) from \(\listvar{s}\)}, denoted
    \(S_{f,\listvar{s}}\), is the smallest set containing \(\listvar{s}\) and
    closed under \(
    \listvar{r}
    \mapsto
    \proj{x}\left(\tilde{f}_0(\listvar{r},\listvar{v})\right)
    \) for any \(\listvar{v} \in \valuetuple{\listvar{m}}\).
\end{definition}

\begin{example}\label{ex:circuit-states}
    Consider the circuit \(
    \iltikzfig{circuits/examples/bottrue/circuit}
    \).
    The semantics of the combinational core are clearly
    \(sr \mapsto (s \lor r)s(s \lor r)\), where the first two characters are the
    next state and the third is the output.
    The initial state is \(\bot\belnaptrue\), so the subsequent states are
    \((\bot \lor \belnaptrue)\bot = \belnaptrue\bot\) and
    \((\belnaptrue \lor \bot)\belnaptrue = \belnaptrue\belnaptrue\).
    As \((\belnaptrue \lor \belnaptrue)\belnaptrue = \belnaptrue\belnaptrue\),
    there are no more circuit states and the complete set is
    \(\{\bot\belnaptrue,\belnaptrue\bot,\belnaptrue\belnaptrue\}\).

    Note that as the output of the circuit is computed as \(s \lor r\), for each
    circuit state the output is \(\belnaptrue\).
    This means that the circuit is denotationally equivalent to \(
    \iltikzfig{circuits/components/waveforms/infinite-register}[val=\belnaptrue]
    \), but this circuit only has a single state \(\belnaptrue\).
\end{example}

We need to \emph{encode} the states of one circuit as another; we have already
encountered this notion using \emph{Mealy homomorphisms}
(\cref{def:mealy-homomorphism});
functions between the state sets that preserve transitions and outputs.
While two `inverse' homomorphisms may not be isomorphisms, the round
trip will always map to a state with the same behaviour.

\begin{lemma}
    For Mealy homomorphisms \(\morph{h}{(S,f)}{(T,g)}\) and
    \(\morph{h^\prime}{(T,g)}{(S,f)}\), any state \(s \in S\) and input
    \(a \in A\), \(
    \mealyfunctiontransition{f}(s, a)
    =
    \mealyfunctiontransition{f}(h^\prime(h(s)), a)
    \).
\end{lemma}
\begin{proof}
    Immediate as Mealy homomorphisms preserve outputs.
\end{proof}

Pairs of `inverse' Mealy homomorphisms as described above will act as state
encoders and decoders between circuits.
To create circuits representing these homomorphisms, we once again use the
functional completeness map for an interpretation, which assigns a syntactic
circuit to any monotone (combinational) function.

\begin{lemma}
    For partial orders \(S\) and \(T\) and monotone Mealy coalgebra
    \((S,f)\) and \((T,g)\), any Mealy homomorphism \(\morph{h}{(S,f)}{(T,g)}\)
    is monotone.
\end{lemma}
\begin{proof}
    In a monotone Mealy coalgebra, the functions \(f\) and \(g\) are monotone,
    and for \(h\) to be a Mealy homomorphism, \(
    \mealyfunctionoutput{f}(s)
    =
    \mealyfunctionoutput{g}(h(s))
    \).
    Assume states \(s,r \in S\); subsequently we have \(
    \mealyfunctionoutput{g}(h(s), a)
    =
    \mealyfunctionoutput{f}(s, a)
    \leq
    \mealyfunctionoutput{f}(r, a)
    =
    \mealyfunctionoutput{g}(h(r), a)
    \).
    This means that the function \(
    s \mapsto \mealyfunctionoutput{g}(h(s), a)
    \) is monotone; as \(\mealyfunctionoutput{g}\) is monotone, \(h\) must
    also be monotone.
\end{proof}

For two circuits \(
\iltikzfig{circuits/productivity/mealy-form}[core=|f|,state=\listvar{s},colour=seq,dom=\listvar{m},cod=\listvar{n},delay=\listvar{x}]
\) and \(
\iltikzfig{circuits/productivity/mealy-form}[core=|g|,state=\listvar{t},colour=seq,dom=\listvar{m},cod=\listvar{n},delay=\listvar{y}]
\), the encoders and decoders we will use for circuits will be Mealy homomorphisms
\(\morph{h}{(S_{f,\listvar{s}},f)}{(S_{g,\listvar{t}},g)}\)
and
\(\morph{h}{(S_{g,\listvar{t}},g)}{(S_{f,\listvar{s}},f)}\).


\begin{proposition}
    For two denotationally equivalent circuits \(
    \iltikzfig{circuits/productivity/mealy-form}[core=|f|,state=\listvar{s},colour=seq,dom=\listvar{m},cod=\listvar{n},delay=\listvar{x}]
    \) and \(
    \iltikzfig{circuits/productivity/mealy-form}[core=|g|,state=\listvar{t},colour=seq,dom=\listvar{m},cod=\listvar{n},delay=\listvar{y}]
    \), there exists at least one Mealy homomorphism \(
    \morph{h}{(S_{f,\listvar{s}},f)}{(S_{g,\listvar{t}},g)}
    \).
\end{proposition}
\begin{proof}
    As the two circuits are denotationally equivalent, at least one valid
    mapping of states can be obtained by sending states in the former to the
    state in the latter obtained after inputting the same word.
    However, there will be more than one valid homomorphism if the second
    circuit has more states than the first, as there will be multiple states in
    the second that act like a single state in the first.
\end{proof}

Note that these are Mealy homomorphisms on the subset of states that a circuit
can assume, \emph{not} the entire set of words that can fit into the state!
This means that encoding and decoding circuits cannot be inserted arbitrarily
but only in the context of a larger circuit in Mealy form.

\begin{proposition}[Encoding equation]\label{prop:encoding-equation}
    For a normalised circuit \(
    \iltikzfig{strings/category/f-2-2}[box=|f|,colour=seq,dom1=\listvar{x},dom2=\listvar{m},cod1=\listvar{x},cod2=\listvar{y}]
    \) and initial state \(\listvar{s} \in \valuetuple{\listvar{x}}\), let
    \(\morph{\mathsf{enc}}{S_{f,\listvar{s}}}{\valuetuple{y}}\) and
    \(\morph{\mathsf{dec}}{\valuetuple{y}}{S_{f, \listvar{s}}}\) be Mealy
    homomorphisms.
    Then the \emph{encoding equation} \((\encodingequation)\) in
    \cref{fig:encoding-equation} is sound, where
    \(\mathsf{enc}_\mathsf{m},\mathsf{dec}_\mathsf{m}\) are monotone completions
    as defined in \cref{def:monotone-completion}.
\end{proposition}
\begin{proof}
    Let \(g\) be the map \(\listvar{r} \mapsto
    \circuittostreami[\iltikzfig{circuits/algebraic/state-encoding}[core=|f|,delay=x,state=\listvar{r}]]
    \); by \cref{prop:mealy-form-image} we know that \(
    \mealyoutput{g(\listvar{t})}{\listvar{v}}
    =
    \proj{1}(f(\mathsf{dec}(\mathsf{enc}(\listvar{t})), \listvar{v}))
    \) and \(
    \mealytransition{g(\listvar{t})}{\listvar{v}}
    =
    g(\proj{0}(f(\mathsf{dec}(\mathsf{enc}(\listvar{t})), \listvar{v})))
    \).
    When \(\listvar{t} \in S_{f, \listvar{s}}\) we have that \(
    \mathsf{dec}(\mathsf{enc}(\listvar{s}))\) is an equivalent state to
    \(\listvar{s}\),
    so
    \(
    \mealyoutput{g(\listvar{t})}{\listvar{v}}
    =
    \proj{1}(f(\listvar{t}), \listvar{v})
    \) and \(
    \mealytransition{g(\listvar{t})}{\listvar{v}}
    \) is equivalent to \(
    g(\proj{0}(f(\listvar{t})), \listvar{v})
    \) by definition of encoders.
    As \(
    \iltikzfig{circuits/algebraic/state-encoding}[core=|f|,delay=x,state=\listvar{s}]
    \coloneqq
    g(\listvar{s})
    \) and \(\listvar{s} \in S_{f,\listvar{s}}\),
    every subsequent stream derivative will also be of the form
    \(g(\listvar{t})\) where \(\listvar{t} \in S_{f,\listvar{s}}\), so the
    equation is sound.
\end{proof}

\begin{remark}
    The encoding equation is an equation \emph{schema}: this is required because
    the width of a circuit state can be arbitrarily large, and each extra bit
    adds a whole new set of Mealy homomorphisms to consider.
\end{remark}

\begin{figure}
    \centering
    \(
    \equationdisplay{
        \iltikzfig{circuits/productivity/mealy-form}[core=|f|,delay=x, colour=seq]
    }{
        \iltikzfig{circuits/algebraic/state-encoding}[core=|f|,delay=x]
    }{
        \encodingequation
    }
    \)
    \\[0.25em]
    \rule{\textwidth}{0.1mm}
    \\[0.5em]
    \(
    \equationdisplay{
        \iltikzfig{circuits/axioms/gate-lhs}
    }{
        \iltikzfig{circuits/axioms/gate-rhs}
    }{
        \gateeqn
    }
    \quad
    \equationdisplay{
        \iltikzfig{circuits/axioms/fork-lhs}
    }{
        \iltikzfig{circuits/axioms/fork-rhs}
    }{
        \forkeqn
    }
    \quad
    \equationdisplay{
        \iltikzfig{circuits/axioms/join-lhs}
    }{
        \iltikzfig{circuits/axioms/join-rhs}
    }{
        \joineqn
    }
    \)
    \\[0.25em]
    \rule{\textwidth}{0.1mm}
    \\[0.5em]
    \(
    \equationdisplay{
        \iltikzfig{circuits/axioms/stub-lhs}
    }{
        \iltikzfig{strings/monoidal/empty}
    }{
        \stubeqn
    }
    \quad
    \equationdisplay{
        \iltikzfig{circuits/axioms/delay-fork-lhs}
    }{
        \iltikzfig{circuits/axioms/delay-fork-rhs}
    }{
        \delayforkeqn
    }
    \quad
    \equationdisplay{
        \iltikzfig{circuits/axioms/bottom-delay-lhs}
    }{
        \iltikzfig{circuits/axioms/bottom-delay-rhs}
    }{
        \bottomdelayeqn
    }
    \quad
    \equationdisplay{
        \iltikzfig{circuits/axioms/streaming-lhs}
    }{
        \iltikzfig{circuits/axioms/streaming-rhs}[gate=p]
    }{
        \streamingeqn
    }
    \)
    \\[0.25em]
    \rule{\textwidth}{0.1mm}
    \\[0.5em]
    \(
    \equationdisplay{
        \iltikzfig{strings/structure/comonoid/unitality-l-lhs}
    }{
        \iltikzfig{strings/structure/comonoid/unitality-l-rhs}
    }{
        \comonoiduniteqnletter
    }
    \quad
    \equationdisplay{
        \iltikzfig{strings/structure/monoid/associativity-lhs}
    }{
        \iltikzfig{strings/structure/monoid/associativity-rhs}
    }{
        \monoidassoceqnletter
    }
    \quad
    \equationdisplay{
        \iltikzfig{strings/structure/monoid/commutativity-lhs}
    }{
        \iltikzfig{strings/structure/monoid/commutativity-rhs}
    }{
        \monoidcommeqnletter
    }
    \quad
    \equationdisplay{
        \iltikzfig{strings/structure/bialgebra/merge-copy-lhs}
    }{
        \iltikzfig{strings/structure/bialgebra/merge-copy-rhs}
    }{
        \joinforkeqn
    }
    \)
    \caption{
        Equations for encoding circuit states
    }
    \label{fig:encoding-equation}
\end{figure}

After applying the encoding equation, the state has not actually changed; we
have just inserted a pair of an encoder and a decoder circuit.
To translate the state we recycle some of the rules from the operational
semantics.
As with the Mealy equations, it is desirable to express these transformations in
terms of smaller components

\begin{lemma}
    The equations on the bottom three rows of \cref{fig:encoding-equation} are
    sound.
\end{lemma}
\begin{proof}
    It is a straightforward exercise to compare the stream functions.
\end{proof}

Using the equations in \cref{fig:encoding-equation}, we can derive some larger
equations.

\begin{lemma}\label{lem:generalised-streaming}
    For a combinational circuit \(
    \iltikzfig{strings/category/f}[box=f,colour=comb]
    \), \(
    \iltikzfig{circuits/axioms/generalised-streaming-lhs}[box=f]
    =
    \iltikzfig{circuits/axioms/generalised-streaming-rhs}[box=f]
    \) by the equations in \cref{fig:encoding-equation}.
\end{lemma}
\begin{proof}
    This is by induction on the structure of \(
    \iltikzfig{strings/category/f}[box=f,colour=comb]
    \).
    First the base cases.
    The case for the gate is immediate by \((\streamingeqn)\).
    For \(\iltikzfig{strings/structure/comonoid/copy}[colour=comb]\) we have
    that \(
    \iltikzfig{circuits/productivity/generalised-streaming/fork-step-0}
    \eqaxioms[(\joinforkeqn)]
    \iltikzfig{circuits/productivity/generalised-streaming/fork-step-1}
    \eqaxioms[(\delayforkeqn)]
    \iltikzfig{circuits/productivity/generalised-streaming/fork-step-2}
    \).
    For \(\iltikzfig{strings/structure/monoid/merge}[colour=comb]\):
    \begin{gather*}
        \iltikzfig{circuits/productivity/generalised-streaming/join-step-2}
        \eqaxioms[(\monoidassoceqnletter)]
        \iltikzfig{circuits/productivity/generalised-streaming/join-step-3}
        \eqaxioms[(\monoidassoceqnletter)]
        \iltikzfig{circuits/productivity/generalised-streaming/join-step-4}
        \eqaxioms[(\monoidcommeqnletter)]
        \iltikzfig{circuits/productivity/generalised-streaming/join-step-5}
        =
        \\
        \iltikzfig{circuits/productivity/generalised-streaming/join-step-6}
        \eqaxioms[(\monoidassoceqnletter)]
        \iltikzfig{circuits/productivity/generalised-streaming/join-step-7}
        \eqaxioms[(\monoidassoceqnletter)]
        \iltikzfig{circuits/productivity/generalised-streaming/join-step-8}
        \eqaxioms[(\monoidassoceqnletter)]
        \iltikzfig{circuits/productivity/generalised-streaming/join-step-9}
    \end{gather*}
    The case for \(\iltikzfig{strings/structure/comonoid/discard}[colour=comb]\) is
    trivial, and the case for \(\iltikzfig{strings/structure/monoid/init}[colour=comb]\)
    follows by \((\comonoiduniteqnletter)\) and \((\bottomdelayeqn)\).
    The cases for \(\iltikzfig{strings/category/identity}[colour=comb]\) and
    \(\iltikzfig{strings/symmetric/symmetry}[colour=comb]\) follow by axioms of STMCs.
    Since the underlying circuit is combinational, for the inductive cases we just
    need to check composition and tensor, which are also trivial.
\end{proof}

\begin{lemma}\label{lem:unroll-waveform}
    For any value \(v \in \values\), \(
    \iltikzfig{circuits/algebraic/unroll-waveform/step-0}[value=v]
    =
    \iltikzfig{circuits/algebraic/unroll-waveform/step-6}[value=v]
    \) by equations in \cref{fig:encoding-equation}.
\end{lemma}
\begin{proof}
    \begin{gather*}
        \iltikzfig{circuits/algebraic/unroll-waveform/step-0}[value=v]
        \coloneqq
        \iltikzfig{circuits/algebraic/unroll-waveform/step-1}[value=v]
        \eqaxioms[(\joinforkeqn)]
        \iltikzfig{circuits/algebraic/unroll-waveform/step-2}[value=v]
        \eqaxioms[(\forkeqn)]
        \iltikzfig{circuits/algebraic/unroll-waveform/step-3}[value=v]
        \eqaxioms[(\delayforkeqn)]
        \\
        \iltikzfig{circuits/algebraic/unroll-waveform/step-4}[value=v]
        \coloneqq
        \iltikzfig{circuits/algebraic/unroll-waveform/step-5}[value=v]
        =
        \iltikzfig{circuits/algebraic/unroll-waveform/step-6}[value=v]
    \end{gather*}
\end{proof}

We will now show that the equations in \cref{fig:encoding-equation} allow us to
translate a circuit into one with an encoded state.

\begin{theorem}
    For a normalised circuit \(
    \iltikzfig{strings/category/f-2-2}[box=|f|,colour=seq,dom1=\listvar{x},dom2=\listvar{m},cod1=\listvar{x},cod2=\listvar{y}]
    \) and initial state \(\listvar{s} \in \valuetuple{\listvar{x}}\), the
    equation \(
    \iltikzfig{circuits/productivity/mealy-form}[core=|f|, colour=seq]
    =
    \iltikzfig{circuits/algebraic/state-encoded}[core=|f|,state=\listvar{s}]
    \) is derivable by the equations in \cref{fig:encoding-equation}.
\end{theorem}
\begin{proof}
    By the \((\encodingequation)\) equation we have that \(
    \iltikzfig{circuits/productivity/mealy-form}[core=|f|,delay=x, colour=seq]
    =
    \iltikzfig{circuits/algebraic/state-encoding}[core=|f|,delay=x,state=\listvar{s}]
    \); we need to `push' the encoder \(
    \iltikzfig{circuits/algebraic/encoder}
    \) through the state.
    Although the encoder is sequential, by the definition of \(\lvert-\rvert\),
    it must be of the form \(
    \iltikzfig{circuits/synthesis/normalised-function}[box=g]
    \) by definition of functional completeness; this means we can perform a
    variant of streaming.
    \begin{gather*}
        \iltikzfig{circuits/algebraic/encoding-state/step-0a}[box=g]
        \coloneqq
        \iltikzfig{circuits/algebraic/encoding-state/step-0}[box=g]
        \eqaxioms[\text{\cref{lem:unroll-waveform}}]
        \iltikzfig{circuits/algebraic/encoding-state/step-1}[box=g]
        \eqaxioms[\text{\cref{lem:generalised-streaming}}]
        \\[0.5em]
        \iltikzfig{circuits/algebraic/encoding-state/step-2}[box=g]
        \eqaxioms[\text{\cref{def:encoder}}]
        \iltikzfig{circuits/algebraic/encoding-state/step-3}[box=g]
        \coloneqq
        \\[0.5em]
        \iltikzfig{circuits/algebraic/encoding-state/step-4}[box=g]
        \coloneqq
        \iltikzfig{circuits/algebraic/encoding-state/step-5}[box=g]
    \end{gather*}
    The proof is completed by sliding the encoder around the trace.
\end{proof}

Before proceeding to our next set of equations, we will show how the encoding
equations can be used to encode the states of any circuit as those of a
denotationally equivalent one.
We are now ready to show that the encoding equations allow us to translate the
states of a circuit into those of a denotationally equivalent circuit.

\begin{corollary}
    For denotationally equivalent circuits \(
    \iltikzfig{circuits/productivity/mealy-form}[core=|f|,state=\listvar{s},colour=seq,dom=\listvar{m},cod=\listvar{n},delay=\listvar{x}]
    \) and \(
    \iltikzfig{circuits/productivity/mealy-form}[core=|g|,state=\listvar{t},colour=seq,dom=\listvar{m},cod=\listvar{n},delay=\listvar{y}]
    \), there exists \(
    \iltikzfig{circuits/productivity/mealy-form}[core=|h|,state=\listvar{t},colour=seq,dom=\listvar{m},cod=\listvar{n},delay=\listvar{x}]
    \) such that \(
    \iltikzfig{circuits/productivity/mealy-form}[core=|f|,state=\listvar{s},colour=seq,dom=\listvar{m},cod=\listvar{n},delay=\listvar{x}]
    =
    \iltikzfig{circuits/productivity/mealy-form}[core=|h|,state=\listvar{t},colour=seq,dom=\listvar{m},cod=\listvar{n},delay=\listvar{x}]
    \) by equations in \cref{fig:encoding-equation}.
\end{corollary}

If we have the right encoders, we can translate the initial state of a circuit
into a different word, and end up with a new circuit in a `Mealy form' with a
sequential core.
However, since the encoders, decoders, and core are essentially combinational,
these can be normalised once again to obtain a new normalised core.
