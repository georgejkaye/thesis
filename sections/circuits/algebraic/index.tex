\chapter{Algebraic semantics}

Testing that the behaviour of two circuits is equivalent by testing that every
input produces the same output for each circuit is a perfectly reasonable
strategy.
But this is somewhat `nuclear'; rather than using what we know
about the structure of a circuit's subcomponents, we just blast away
exhaustively trying all the inputs to find a contradiction.

A more elegant method of reasoning is by defining a set \emph{equations} between
subcircuits and \emph{quotienting} \(\scircsigma\) by these equations.
A proof of equivalence between two circuits is then presented using algebraic
reasoning: applying equations to translate one circuit into the other.
This is often \emph{far more efficient} than having to test every input!
This is the focus of our final perspective on semantics for sequential
circuits, \emph{algebraic semantics}.

\begin{example}\label{ex:expressions-algebraic}
    For the last time we return to the language of arithmetical expressions from
    \cref{ex:expressions-denotational}.
    An algebraic semantics for this language can be defined using a set of
    equations: the familiar equations of associativity, commutativity,
    unitality, annihiliation and distributivity, along with equations for
    actually performing arithmetic.
    \begin{gather*}
        add\,(add\,e_1\,e_2)\,e_3 = add\,e_1\,(add\,e_2\,e_3)
        \qquad
        mul\,(mul\,e_1\,e_2)\,e_3 = mul\,e_1\,(mul\,e_2\,e_3)
        \\
        add\,e_1\,e_2 = add\,e_2\,e_1
        \qquad
        mul\,e_1\,e_2 = mul\,e_2\,e_1
        \\
        add\,e_1\,\overline{0} = e_1
        \qquad
        mul\,e_1\,\overline{1} = e_1
        \qquad
        mul\,e_1\,\overline{0} = \overline{0}
        \\
        mul\,e_1\,(add\,e_2\,e_3) = add\,(mul\,e_1\,e_2)\,(mul\,e_1\,e_3)
        \\
        add\,\overline{n_1}\,\overline{n_2} = add\,\overline{n_1+n_2}
        \qquad
        add\,\overline{n_1}\,\overline{n_2} = \overline{n_1 \cdot n_2}
    \end{gather*}
    If everything is specified concretely as values then one could easily just
    use the last two equations to compare two expressions by reducing
    two expressions to values as in the operational semantics.
    The power of the algebraic semantics comes from the fact we can reason
    abstractly with expressions containing \emph{blackboxes}.
    Take the following example, containing some arbitrary component \(e\).
    \begin{align*}
        mul\,(add\,e\,(mul\,e\,\overline{3}))\,\overline{2}
        &=
        mul\,(add\,(mul\,e\,1)\,(mul\,e\,\overline{3}))\,\overline{2}
        \\
        &=
        mul\,(mul\,e\, add(\overline{1}\,\overline{3}))\,\overline{2}
        \\
        &=
        mul\,(mul \,e\,\overline{4})\,\overline{2}
        \\
        &=
        mul\,e\,(mul \,\overline{4}\,\overline{2})
        \\
        &=
        mul\,e\,\overline{8}
        \\
        &=
        mul\,\overline{8}\,e
    \end{align*}
    Despite not specifying the structure of \(e\), we have
    shown how the expression is equal to a slightly simpler one.
    This process creates new general equations which can be used as `shortcuts'
    in future reasoning, potentially saving many steps.
\end{example}

As with the operational semantics, we are especially interested in defining a
\emph{sound and complete} algebraic semantics for sequential digital circuits
with respect to the denotational semantics.
That is to say, for each equation \(
    \iltikzfig{strings/category/f}[box=f,colour=seq]
    =
    \iltikzfig{strings/category/f}[box=g,colour=seq]
\) then \(
    \circuittostreami[\iltikzfig{strings/category/f}[box=f,colour=seq]]
    =
    \circuittostreami[\iltikzfig{strings/category/f}[box=g,colour=seq]]
\), and there must be enough equations such that if \(
    \circuittostreami[\iltikzfig{strings/category/f}[box=f,colour=seq]]
    =
    \circuittostreami[\iltikzfig{strings/category/f}[box=g,colour=seq]]
\) then there exists a series of equations identifying \(
    \circuittostreami[\iltikzfig{strings/category/f}[box=f,colour=seq]]
\) and \(
    \circuittostreami[\iltikzfig{strings/category/f}[box=f,colour=seq]]
\).

\begin{remark}
    An `equational theory' for sequential circuits was one of the first things
    presented in the original circuits
    paper~\cite{ghica2016categorical,ghica2017diagrammatic}.
    In that paper, equations that `seemed right' were used to quotient the
    syntax, with the ultimate aim of showing that the resulting category was
    \emph{Cartesian}.
    This was done quite informally, and was made more confusing as
    the categories of circuits were subsequently quotiented by some notion of
    `extensional equivalence', an attempt to rectify the fact that the
    equations only dealt with closed circuits.
    Soundness and completeness of the equational theory was not considered
    because there was no other semantics to compare it against!

    In essence, the previous work was almost `the wrong way round': equations
    were defined and semantics drawn from them.
    In the more recent version of the work~\cite[Sec. 5]{ghica2024fully}, which
    forms the basis for this chapter, the equations are derived from the
    denotational semantics.
    Not only does this give us a formal way of verifying that these equations
    are sound, it sets the backdrop against which we can test if the algebraic
    semantics are sufficient: are any two denotationally equal circuits
    identified by equations?

    Fans of the original work should fear not, because what came before has not
    been entirely discarded.
    Most of the original equations have found a home as rules of the operational
    semantics in the previous chapter, and the fact that categories of circuits
    should be Cartesian will crop up later on; stay tuned!
\end{remark}

\section{Mealy equations}

Although we have now left operational semantics behind in the previous chapter,
the notion of Mealy form will come in very handy for defining a sound and
complete algebraic semantics.
We could just naively port across the Mealy rules by changing the squiggly arrow
into an equals sign and hoping nobody notices, but we are better people than
that.
Instead we will show how Mealy form can be derived using a finite set of
equations, rather than resorting to equations parameterised over all circuits.

\begin{figure}
    \centering
    \(
    \equationdisplay{
        \iltikzfig{strings/structure/monoid/unitality-l-lhs}
    }{
        \iltikzfig{strings/structure/monoid/unitality-l-rhs}
    }{
        \joinunitleqn
    }
    \qquad
    \equationdisplay{
        \iltikzfig{strings/structure/monoid/unitality-r-lhs}
    }{
        \iltikzfig{strings/structure/monoid/unitality-r-rhs}
    }{
        \joinunitreqn
    }
    \)
    \\[0.25em]
    \rule{\textwidth}{0.1mm}
    \\[0.5em]
    \(
    \equationdisplay{
        \iltikzfig{circuits/axioms/bottom-delay-lhs}
    }{
        \iltikzfig{circuits/axioms/bottom-delay-rhs}
    }{
        \bottomdelayeqn
    }
    \qquad
    \equationdisplay{
        \iltikzfig{circuits/instant-feedback/equation-lhs}[box=f]
    }{
        \iltikzfig{circuits/instant-feedback/equation-rhs}[box=f]
    }{
        \instantfeedbackeqn
    }
    \)
    \\[0.25em]
    \rule{\textwidth}{0.1mm}
    \caption{
        Set of Mealy equations
        \(\mealyequations\).
    }
    \label{fig:mealy-equations}
\end{figure}

\begin{definition}
Let \(\mealyequations\) be the set of equations in \cref{fig:mealy-equations}.
\end{definition}

\begin{proposition}
    Given a sequential circuit \(
        \iltikzfig{strings/category/f}[box=F,colour=seq]
    \), there exists a circuit in Mealy form such that \(
        \iltikzfig{strings/category/f}[box=F,colour=seq]
        =
        \iltikzfig{circuits/productivity/mealy-form}
    \) in \(\scircsigma / \mealyequations\).
\end{proposition}
\begin{proof}
    Any circuit can be assembled into global trace-delay form solely using the
    axioms of STMCs, which hold by default in \(\scircsigma\).
    From this, a circuit in pre-Mealy form can be obtained by translating
    delays into registers using \(
        \iltikzfig{circuits/axioms/delay-to-register/step-0}
        \eqaxioms[(\monoidunitleqn)]
        \iltikzfig{circuits/axioms/delay-to-register/step-1}
    \) and translating values into registers using \(
        \iltikzfig{circuits/axioms/value-to-register/step-0}
        \eqaxioms[(\monoidunitreqn)]
        \iltikzfig{circuits/axioms/value-to-register/step-1}
        \eqaxioms[(\bottomdelayeqn)]
        \iltikzfig{circuits/axioms/value-to-register/step-2}
    \).
    Subsequently a circuit in Mealy form can be obtained by applying the
    \((\instantfeedback)\).
\end{proof}
\section{Normalising circuits}

We can now translate any circuit into a denotationally-equal Mealy form using
equations in \(\mealyeqn\); what now?
As mentioned in the previous section, there is no guarantee that the
combinational cores of Mealy forms have the same semantics, even if the state
word is the same.
Even so, one might think that through some combination of sliding components
around the trace and across the delay, one might stumble across a way of
translating one circuit into another.
Unfortunately, these is simply \emph{not sufficient} for
translating between circuits with the same semantics.

\begin{example}
    Consider the following circuit in \(\scirc{\belnapsignature}\): \[
        \iltikzfig{circuits/examples/state-change/circuit}
        \quad
        \iltikzfig{circuits/examples/state-change/circuit-simpler}
    \]
    The registers will \emph{always} contain
    \(\belnapfalse\) and \(\belnaptrue\) and both circuits will
    produce a constant \(\belnapfalse\) output, so a complete equational theory
    should be able to translate between them.
    To this end, we assemble these into Mealy forms with the same initial states:
    \[
        \iltikzfig{circuits/examples/state-change/circuit-mealy}
        \quad
        \iltikzfig{circuits/examples/state-change/circuit-simpler-mealy}
    \]
    The combinational cores do \emph{not} have the same semantics!
    They only act the same because they receive certain inputs
    from \(\valuetuple{3}\).
\end{example}

We must take \emph{context} into account when defining
equations; unfortunately equations that only deal with the interactions between
individual generators are not enough.


\section{Encoding equations}

While the core of a circuit in normalised Mealy form is the canonical
representation of that circuit's Mealy function, there is no reason that
circuits with the same behaviour should have the same Mealy function.
Even if the state words are the same, they may be jumbled up in different
orders.
This is why checking equivalence of Mealy machines is usually performed by
constructing a \emph{bisimulation} rather than basic equality.
To bring this notion into the syntactic realm, we must consider the states

\begin{remark}
    Rather than equations between concrete generators, the equations in this
    section will be \emph{families} of equations parameterised by normalised
    combinational cores and initial states.
    While this does lead to infinite equations, we are working with normalised
    cores so there will be exactly one equation for each combination of
    `behaviour' and state.
\end{remark}

\begin{definition}[States]
    Let \(\morph{f}{\valuetuple{\listvar{xm}}}{\valuetuple{\listvar{xn}}}\) be a
    monotone function and let \(\listvar{s} \in  \valuetuple{x}\) be an
    initial state.
    Then the \emph{states of \(f\) from \(\listvar{s}\)} is the smallest set
    containing \(\listvar{s}\) and closed under \(
    \listvar{r}
    \mapsto
    \proj{x}\left(\tilde{f}_0(\listvar{r},\listvar{v})\right)
    \) for any \(\listvar{v} \in \valuetuple{\listvar{m}}\).
\end{definition}

The set of circuit states is essential in determining valid equations between
circuits in normalised Mealy form.
The first thing to consider when comparing two circuits in normalised Mealy
form is that the circuits may not have the same states, or even have state words
of equal width.
One of our families of equations will be based around \emph{encoding} states
of a circuit into different representations.

\begin{definition}[Encoder]\label{def:encoder}
    Let \(S \subseteq \valuetuple{x}\) and \(T \subseteq \valuetuple{y}\) be two
    sets; an \emph{(S,T)-encoder} is two functions \(
    \morph{\mathsf{enc}_{S,T}}{S}{T}
    \) and \(
    \morph{\mathsf{dec}_{S,T}}{T}{S}
    \) such that \(dec_{S,T} \circ enc_{S,T}\).
\end{definition}

\begin{example}
    Given a set of states \(S \subseteq \valuetuple{x}\), a \(\leq\)-encoding
    \(\morph{\gamma_\leq}{S}{\valuetuple{k}}\) used when defining the map from
    Mealy machines to circuits (\cref{def:encoding}) is an example of a
    \((S,\valuetuple{k})\)-encoder; the decoder
    \(\mathsf{dec}_{S,\valuetuple{x}}\) is defined as the monotone completion
    of \(\gamma_\leq^{-1}\).
\end{example}

Using a decoder and an encoder, we can change the state words that a circuit
operates with.

\begin{proposition}[Encoding equation]\label{prop:encoding-equation}
    Let \(\iltikzfig{strings/category/f}[box=f,colour=seq]\) be a circuit in
    normalised Mealy form such that \(
    \iltikzfig{strings/category/f}[box=f,colour=seq]
    \coloneqq
    \iltikzfig{circuits/productivity/mealy-form}[core=|\hat{f}|,colour=seq,delay=\listvar{x}]
    \) and let \(\mathsf{enc},\mathsf{dec}\) be the two functions of a
    \((S(\hat{f},\listvar{s})),T\)-encoder as defined above.
    Then the \emph{encoding equation} \((\encodingequation)\) in
    \cref{fig:encoding-equation} is sound.
\end{proposition}
\begin{proof}
    The interpretations of \(
    \iltikzfig{strings/category/f-2-1}[box={\transitiontranslation{f}},colour=seq,dom1=\listvar{x},dom2=\listvar{m},cod=\listvar{x}]
    \) and \(
    \iltikzfig{strings/category/f-2-1}[box={\outputtranslation{f}},colour=seq,dom1=\listvar{x},dom2=\listvar{m},cod=\listvar{n}]
    \) are as the behaviours of \(
    \iltikzfig{strings/category/f-2-1}[box={f},colour=seq,dom1=\listvar{x},dom2=\listvar{m},cod=\listvar{x}]
    \) and \(
    \iltikzfig{strings/category/f-2-1}[box={g},colour=seq,dom1=\listvar{x},dom2=\listvar{m},cod=\listvar{x}]
    \) wrapped in appropriate decodings and encodings.
\end{proof}

\begin{notation}[Restriction]
    Given a function \(\morph{f}{A}{B}\), and a set
    \(C \subseteq A\), let \(\morph{f|C}{C}{B}\) be the \emph{restriction} of
    \(f\) to act on elements of \(C\), i.e.\ for all \(c \in C\),
    \(f|C(c) = f(c)\).
\end{notation}

\begin{definition}
    Let \(
    \iltikzfig{circuits/productivity/mealy-form}[core=|f|,colour=seq,delay=\listvar{x}]
    \) be a circuit in normalised Mealy form with state set \(S(f, \listvar{s})\).
    Then, for a total order \(\leq\) on \(S(f, \listvar{s})\), the
    \emph{\(\leq\)-encoding circuit} is a
\end{definition}

\begin{definition}[State change equation]

\end{definition}



\begin{definition}[Restriction equations]
    Given two circuits in normalised Mealy form \(
    \iltikzfig{circuits/productivity/mealy-form}[core=|f|,colour=seq]
    \) and \(
    \iltikzfig{circuits/productivity/mealy-form}[core=|g|,colour=seq]
    \)  a \emph{restriction equation}
\end{definition}

Since the transition subcircuit of a circuit in normalised Mealy form is itself
a normalised function, we can easily perform a lookup to find out what the
behaviour of \(\tilde{f}\) is.
The size of the set of circuit states determines the width of the encoded state.

\begin{definition}[Translations]
    For a set of states \(S\), let \(\morph{\gamma_\leq}{S}{\valuetuple{y}}\) be
    a \(\leq\)-encoding.
    For a circuit \(
    \iltikzfig{strings/category/f-2-1}[box=f, colour=seq, dom1=\listvar{x}, dom2=\listvar{m}, cod=\listvar{x}]
    \coloneqq
    \mealytofunc[\hat{f}]
    \), its \emph{\(\gamma_\leq\)-transition} is a circuit \(
    \iltikzfig{strings/category/f-2-1}[box=\transitiontranslation{f}, colour=seq, dom1=\listvar{x}, dom2=\listvar{m}, cod=\listvar{x}]
    \) defined as \(
    \mealytofunc[
        \left(\gamma_\leq(\listvar{s}),\listvar{v}\right)
        \mapsto \gamma_\leq(\hat{f}(\listvar{s}, \listvar{v}))
    ]
    \), and for a circuit \(
    \iltikzfig{strings/category/f-2-1}[box=g, colour=seq, dom1=\listvar{x}, dom2=\listvar{m}, cod=\listvar{x}]
    \coloneqq
    \mealytofunc[\hat{g}]
    \), its \emph{\(\gamma_\leq\)-output} is a circuit \(
    \iltikzfig{strings/category/f-2-1}[box=\outputtranslation{f}, colour=seq, dom1=\listvar{x}, dom2=\listvar{m}, cod=\listvar{x}]
    \) defined as \(
    \mealytofunc[
        \left(\gamma_\leq(\listvar{s}),\listvar{v}\right)
        \mapsto \hat{g}(\listvar{s}, \listvar{v})
    ].
    \)
\end{definition}

These two translations are the the essential components of our final equation,
the \emph{encoding} equation, which is parameterised by an order \(\leq\) on the
states which specifies how to map a circuit in normalised Mealy form to
its corresponding \(\leq\)-encoded form.

\begin{proposition}[Encoding equation]\label{prop:encoding-equation}
    Given two normalised circuits \(
    \iltikzfig{strings/category/f-2-1}[box=f,colour=seq,dom1=\listvar{x},dom2=\listvar{m},cod=\listvar{x}]
    \) and \(
    \iltikzfig{strings/category/f-2-1}[box=f,colour=seq,dom1=\listvar{x},dom2=\listvar{m},cod=\listvar{n}]
    \) along with word \(\listvar{s} \in \valuetuple{\listvar{x}}\), let
    \(\leq\) be an ordering on \(S(f,\listvar{s})\).
    Then the \emph{encoding equation} \((\encodingequation)\) in
    \cref{fig:encoding-equation} is sound.
\end{proposition}
\begin{proof}
    The interpretations of \(
    \iltikzfig{strings/category/f-2-1}[box={\transitiontranslation{f}},colour=seq,dom1=\listvar{x},dom2=\listvar{m},cod=\listvar{x}]
    \) and \(
    \iltikzfig{strings/category/f-2-1}[box={\outputtranslation{f}},colour=seq,dom1=\listvar{x},dom2=\listvar{m},cod=\listvar{n}]
    \) are as the behaviours of \(
    \iltikzfig{strings/category/f-2-1}[box={f},colour=seq,dom1=\listvar{x},dom2=\listvar{m},cod=\listvar{x}]
    \) and \(
    \iltikzfig{strings/category/f-2-1}[box={g},colour=seq,dom1=\listvar{x},dom2=\listvar{m},cod=\listvar{x}]
    \) wrapped in appropriate decodings and encodings.
\end{proof}

\begin{figure}
    \centering
    \begin{minipage}{0.45\textwidth}
        \centering
        \(
            \equationdisplay{
                \iltikzfig{circuits/algebraic/encoding}[state=\listvar{s},transition=f,output=g,delay=x]
            }{
                \iltikzfig{circuits/algebraic/encoding-longstate}[state=\gamma_{\leq}(\listvar{s}),transition={\transitiontranslation{f}},output={\outputtranslation{g}},delay=y]
            }{
                \encodingequation
            }
        \)
    \end{minipage}
    \begin{minipage}{0.1\textwidth}
        \centering
        \vspace{0.5em}
        \begin{tabular}{c}
            \(
                \iltikzfig{strings/category/f}[box=f,colour=seq],
                \iltikzfig{strings/category/f}[box=g,colour=seq]
            \)
            \\[0.25em]
            normalised
        \end{tabular}
    \end{minipage}
    \caption{
        The encoding equation
    }
    \label{fig:encoding-equation}
\end{figure}

\begin{remark}
    The encoding equation is really a \emph{family} of equations parameterised
    over the circuits \(
    \iltikzfig{strings/category/f}[box=f,colour=seq]
    \) and \(
    \iltikzfig{strings/category/f}[box=g,colour=seq]
    \).
    Since we have the precondition that these circuits are normalised, this is
    not quite every circuit in the universe but there will still be an equation
    for each possible function \(
    \valuetuple{\listvar{m}} \to \valuetuple{\listvar{n}}
    \), of which there are infinitely many.
    any old circuit.
    What is important is that given the two subcircuits and state word of a
    circuit in normalised Mealy form, one can compute the set of circuit states
    and subsequently the encoded circuit.
\end{remark}

The circuit on the right hand side of the encoding equation has a state
containing just \(\bot\) and \(\top\) values.
This is the `pseudo-normal form' for sequential circuits; depending on the
state order picked there are likely to be multiple circuits of this form.
Fortunately, since these circuits are also in normalised Mealy form, the
encoding equation can be used to translate between pseudo-normal forms as well.

\begin{definition}
    For an interpretation \(\interpretation\), let
    \(\mce_{\interpretation}\) be defined as \(
    \mealyequations +
    \instantfeedbackeqn +
    \cartesianequations +
    \normalisingequations +
    \encodingequation
    \), and let \(\scircsigmae\) be defined as
    \(\scircsigma / \mce_{\interpretation}\).
\end{definition}

\begin{theorem}
    For a functionally complete interpretation \(\interpretation\), \(
    \iltikzfig{strings/category/f}[box=f,colour=seq,dom=\listvar{m},cod=\listvar{n}]
    =
    \iltikzfig{strings/category/f}[box=g,colour=seq,dom=\listvar{m},cod=\listvar{n}]
    \) in \(\scircsigmae\) if and only if \(
    \circuittostreami[
        \iltikzfig{strings/category/f}[box=f,colour=seq,dom=\listvar{m},cod=\listvar{n}]
    ]
    =
    \circuittostreami[
        \iltikzfig{strings/category/f}[box=g,colour=seq,dom=\listvar{m},cod=\listvar{n}]
    ]
    \).
\end{theorem}
\begin{proof}
    All the equations are sound, so we only need to consider the \(\ifdir\)
    direction.
    By \cref{lem:normalised-mealy}, the two circuits can be brought to
    normalised Mealy form using
    \(
    \mealyequations +
    \instantfeedbackeqn +
    \cartesianequations +
    \normalisingequations
    \).
    By \cref{def:mealy-to-circuit} and
    \cref{thm:circuit-stream-correspondence} there must exist a circuit \(
    \iltikzfig{strings/category/f}[box=h,colour=seq]
    \coloneqq
    \iltikzfig{circuits/algebraic/encoding}[transition={h_0},output={h_1},state={\listvar{s}}]
    \) such that \(
    \circuittostream[
        \iltikzfig{strings/category/f}[box=f,colour=seq]
    ]{\interpretation}
    =
    \circuittostream[
        \iltikzfig{strings/category/f}[box=h,colour=seq]
    ]{\interpretation}
    =
    \circuittostream[
        \iltikzfig{strings/category/f}[box=g,colour=seq]
    ]{\interpretation}
    \).
    The circuit \(
    \iltikzfig{strings/category/f}[box=H,colour=seq]
    \) is encoded such that the state words are in the image of
    \(\gamma_\leq\).
    This means that applying the encoding equation with \(\leq\) to the
    normalised Mealy forms obtained above will yield the circuit \(
    \iltikzfig{strings/category/f}[box=h,colour=seq]
    \).
\end{proof}

As always, the soundness and completeness of the algebraic semantics means we
can establish another isomorphism of PROPs.

\begin{corollary}
    \(\scircsigmai \cong \scircsigmae\).
\end{corollary}

This brings our jaunt into algebraic semantics to a close; given a circuit \(
\iltikzfig{strings/category/f}[box=f,colour=seq,dom=\listvar{m},cod=\listvar{n}]
\) we know that we can translate it into another circuit \(
\iltikzfig{strings/category/f}[box=g,colour=seq,dom=\listvar{m},cod=\listvar{n}]
\) with the same behaviour by only using equations in \(\mce_\interpretation\).

Of course, the procedure of using the normalisation equations to translate a
circuit into normalised Mealy form before using the encoding equation (possibly
multiple times depending on how lucky one gets with their orderings) may be
tedious; one might wonder how this is beneficial to the operational approach in
the previous chapter.
But the beauty of the \emph{algebraic} semantics is that we \emph{don't} need to
do this every time!
Equations can be proven as lemmas and then used repeatedly in the future as
`shortcuts', possibly saving many reasoning steps.
In time, the algebraicist will build up a powerful repertoire of equations and
wield them to bend circuits to their will.