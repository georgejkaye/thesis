\section{Feedback}

One of the major issues that comes with trying to reduce circuits in
\(\scircsigma\) is the presence of feedback.
Without proper attention, one could end up infinitely unfolding and we never
produce any output values!
The first portion of our operational semantics will revolve around defining some
\emph{global transformations} to make a circuit suitable for reductions.

The first observation we make does not even need anything new to be defined as
it follows immediately from axioms of STMCs.

\begin{lemma}[Global trace-delay form]\label{lem:trace-delay}
    For a sequential circuit \(
        \iltikzfig{strings/category/f}[box=f,colour=seq,dom=m,cod=n]
    \) there exists a combinational circuit \(
        \iltikzfig{circuits/productivity/pre-mealy-core}
    \) and \(\overline{v} \in \valuetuple{z}\) such that \(
        \iltikzfig{strings/category/f}[box=f,colour=seq,dom=m,cod=n]
        =
        \iltikzfig{circuits/productivity/trace-delay}[core=F,state=\listvar{v},dom=m,cod=n,delay=y,trace=x,valwidth=z]
    \) by axioms of STMCs.
\end{lemma}
\begin{proof}
    By applying the axioms of traced categories; any trace can be `pulled'
    to the outside of a term by superposing and tightening.
    For the delays, a trace can be introduced using yanking and then the
    same procedure as above followed.
\end{proof}

\begin{example}
    The SR NOR latch circuit from \cref{ex:sr-latch} is assembled into global
    trace-delay form in \cref{fig:sr-latch-global-trace-delay-form}.
\end{example}

\begin{figure}
    \centering
    \iltikzfig{circuits/examples/sr-latch/circuit-global-trace-delay}
    \(\reduction[(\mealyeqn)]\)
    \iltikzfig{circuits/examples/sr-latch/circuit-premealy}
    \caption{
        The SR NOR latch from \cref{fig:sr-latch} assembled into global
        trace-delay form, and transformed into pre-Mealy form with
        \((\mealyeqn)\).
    }
    \label{fig:sr-latch-global-trace-delay}
\end{figure}

This form is evocative of what we saw when mapping from Mealy machines to
circuits in the previous section, but rather than the state being determined by
one word, the instantaneous values and the delays are kept separate.
Our first global translation will remedy this.

\begin{definition}[Pre-Mealy form]\label{def:pre-mealy}
    A sequential circuit is in \emph{pre-Mealy form} if it is in the form \(
        \iltikzfig{circuits/productivity/pre-mealy-form}[core=\hat{f},state=\listvar{s},dom=m,cod=n,trace=x,delay=y]
    \).
\end{definition}

\begin{lemma}\label{lem:mealy-rule}
    The rule \(
        \iltikzfig{circuits/productivity/trace-delay}[core=f,state=\listvar{v},dom=m,cod=n,delay=y,trace=x,valwidth=z]
        \reduction[(\mealyeqn)]
        \iltikzfig{circuits/productivity/mealy-rule}[core=f,trace=x,delays=y,values=z]
    \) is sound.
\end{lemma}
\begin{proof}
    It is a simple exercise to check the corresponding stream functions.
\end{proof}

By assembling a circuit into global trace-delay form and
applying the \((\mealyeqn)\) rule, we can establish a word \(\listvar{s}\)
representing the initial state of a circuit.
This word is by no means unique: it depends on how the circuit is put into
global trace-delay form.
What matters most is that we \emph{can} do it!

\begin{corollary}
    For any sequential circuit \(
        \iltikzfig{strings/category/f}[box=f, colour=seq, dom=m, cod=n]
    \), there exists at least one valid application of the Mealy rule.
\end{corollary}

\begin{example}
    The \((\mealyeqn)\) rule is applied to the global trace-delay form
    SR NOR latch in \cref{fig:sr-latch-global-trace-delay}.
    Here the initial state word is just \(\bot\).
\end{example}

The result of applying the \((\mealyeqn)\) reduction still differs from the
image of \(\mealytocircuiti\) as it may have a trace with no delay on it: an
instance of \emph{non-delay-guarded feedback}.

\begin{remark}
    The mere mention of non-delay-guarded feedback may trigger alarm bells in
    the minds of those well-acquainted with circuit design!
    It is often common in industry to enforce that circuits have no
    non-delay-guarded feedback; one might ask if we should also enforce this
    tenet in order to stick to `well-behaved' circuits.
    However, not only would this violate the categorical setting of a STMC (the
    `yanking' equation would no longer hold), careful use of non-delay-guarded
    feedback can still result in useful circuits as a clever way of sharing
    resources~\cite{malik1994analysis,riedel2004cyclic,mendler2012constructive}.
    The minimal circuit to implement a function often \emph{must} be
    constructed using cycles~\cite{rivest1977necessity,riedel2003synthesis}!
\end{remark}

\begin{example}\label{ex:cyclic-combinational}
    A particularly famous circuit~\cite{malik1994analysis} which is useful
    despite the presence of non-delay-guarded feedback is shown in
    \cref{fig:cyclic-combinational}, where \(
        \iltikzfig{strings/category/f}[box=f,colour=comb]
    \) and \(
        \iltikzfig{strings/category/f}[box=G,colour=comb]
    \) are arbitrary combinational circuits.
    The trapezoidal gate is a \emph{multiplexer}; it has a vertical
    \emph{control} input and two horizontal \emph{data} inputs.
    The multiplexer is defined as \(
        \iltikzfig{circuits/components/gates/mux}
        \coloneqq
        \iltikzfig{circuits/components/gates/mux-construction}
    \).
    The multiplexer is a particularly common circuit component as it is used as
    a rudimentary if statement: when the control is \(\belnapfalse\) the output
    is the first data input and when it is \(\belnaptrue\) the output is the
    second data input.

    The circuit in \cref{fig:cyclic-combinational} has no state and its trace is
    global so it is already in pre-Mealy form, and has
    non-delay-guarded feedback.
    Despite this, it produces useful output when the control signal is \(0\)
    or \(1\):
    \begin{gather*}
        \circuittostream[
            \iltikzfig{strings/category/f-2-1}[box=f,colour=seq]
        ]{\interpretation_\star}(\sigma)(i)
        =
        \circuittofunc[
                \iltikzfig{circuits/examples/cyclic-combinational/reduced-false}
        ]{\interpretation_\star}
        (\proj{1}(\sigma(i)))
        \text{ if } \proj{0}(\sigma(i)) = \belnapfalse
        \\
        \circuittostream[
            \iltikzfig{strings/category/f-2-1}[box=f,colour=seq]
        ]{\interpretation_\star}(\sigma)(i)
        =
        \circuittofunc[
                \iltikzfig{circuits/examples/cyclic-combinational/reduced-true}
        ]{\interpretation_\star}
        (\proj{1}(\sigma(i)))
        \text{ if } \proj{0}(\sigma(i)) = \belnaptrue
    \end{gather*}
    The feedback is just used as a clever way to share circuit components: the
    multiplexers and control signal simply determine which order the two
    subcircuits are applied in.
\end{example}

\begin{figure}
    \centering
    \iltikzfig{circuits/examples/cyclic-combinational/circuit}
    \quad
    \iltikzfig{circuits/examples/cyclic-combinational/circuit-scirc}
    \caption{
        A useful cyclic combinational circuit
        \cite[Fig. 1]{mendler2012constructive}, and a possible interpretation in
        \(\scircsigma\).
    }
    \label{fig:cyclic-combinational}
\end{figure}

A combinational circuit surrounded by non-delay-guarded feedback still
implements a function, as there are no delay components.
Nevertheless, non-delay-guarded feedback does still block our path to future
transformations, so it must be eliminated.
Using a methodology also employed by~\cite{riedel2012cyclic}, we turn to the
Kleene fixed-point theorem.

\begin{lemma}\label{lem:monotone-fixpoint}
    For a monotone function \(\morph{f}{\values^{n+m}}{\valuetuple{n}}\) and
    \(i \in \nat\), let \(\morph{f^i}{\valuetuple{m}}{\valuetuple{n}}\) be
    defined as \(f^0(x)  = f(\bot^n,x)\) and \(f^{k+1}(x) = f(f^k(x), x)\).
    Let \(c\) be the length of the longest chain in the lattice
    \(\valuetuple{n}\).
    Then, for \(j > c\), \(f^c(x) = f^{j}(x)\).
\end{lemma}
\begin{proof}
    Since \(f\) is monotone, it has a least fixed point by the Kleene
    fixed-point theorem.
    This will either be some value \(v\) or, since \(\values\) is finite, the
    \(\top\) element.
    The most iterations of \(f\) it would take to obtain this fixpoint is \(c\),
    i.e.\ the function produces a value one step up the lattice each time.
\end{proof}

\begin{definition}[Iteration]\label{def:iteration}
    For a combinational circuit \(
        \iltikzfig{strings/category/f-2-2}[box=f,colour=comb,dom1=x,dom2=m,cod1=x,cod2=n],
    \)
    its \emph{\(n\)th iteration} \(
        \iltikzfig{strings/category/f-1-2}[box=f^n,colour=comb,dom=m,cod1=x,cod2=n]
    \) is defined inductively over \(n\) as \(
        \iltikzfig{circuits/instant-feedback/f0-box}[dom=m,trace=x,cod=n]
        \coloneqq
        \iltikzfig{circuits/instant-feedback/f0-definition}[dom=m,trace=x,cod=n]
    \) and \(
        \iltikzfig{circuits/instant-feedback/fkp1-box}[dom=m,trace=x,cod=n]
        \coloneqq
        \iltikzfig{circuits/instant-feedback/fkp1-definition}[dom=m,trace=x,cod=n]
    \).
\end{definition}

The trace in \(\streami\) is by the least fixed point, computed by repeatedly
applying \(f\) to itself starting from \(\bot\).
The above lemma gives a fixed upper bound for the number of times we need to
apply \(f\) to reach this fixed point, based on the size of the lattice.
We can port this over to the syntactic setting.

\begin{definition}[Unrolling]\label{def:unrolling}
    For an interpretation with values \(\values\), the \emph{unrolling}
    of a combinational circuit \(
        \iltikzfig{strings/category/f-2-2}[box=f,colour=comb,dom1=x,dom2=m,cod1=x,cod2=n]
    \), written \(
        \iltikzfig{strings/category/f-1-2}[box=f^\dagger,colour=comb,dom=m,cod1=x,cod2=n]
    \), is defined as \(
        \iltikzfig{circuits/instant-feedback/fc-box}[dom=m,trace=x,cod=n]
    \) where \(c\) is the length of the longest chain in \(\valuetuple{x}\).
\end{definition}

Using these constructs we can replace a combinational circuit wrapped in
non-delay-guarded feedback with a behaviourally equivalent circuit with no
feedback at all!

\begin{proposition}\label{prop:instant-feedback}
    The instant feedback rule \(
        \iltikzfig{circuits/instant-feedback/equation-lhs}
        \reduction[(\instantfeedbackeqn)]
        \iltikzfig{circuits/instant-feedback/equation-rhs}
    \) is sound.
\end{proposition}
\begin{proof}
    By \cref{lem:monotone-fixpoint}, applying the circuit \(c\)
    times reaches a fixpoint.
    The circuit is combinational so each element of the output
    \(\circuittostreami[
        \iltikzfig{strings/category/f}[box=f,colour=comb]
    ](\sigma)(i)\) is a function; this means that \cref{lem:monotone-fixpoint}
    can be applied to each element.
\end{proof}

\begin{example}
    The \((\instantfeedbackeqn)\) rule is applied to the cyclic combinational
    circuit from \cref{fig:cyclic-combinational} in
    \cref{fig:cyclic-combinational-unrolled}.
\end{example}

\begin{figure}
    \centering
    \scalebox{0.8}{
        \iltikzfig{circuits/examples/cyclic-combinational/circuit-scirc-unrolled}
    }
    \caption{
        The \((\instantfeedbackeqn)\) rule applied to the circuit from
        \cref{fig:cyclic-combinational}
    }
    \label{fig:cyclic-combinational-unrolled}
\end{figure}