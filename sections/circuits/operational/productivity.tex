\section{Productivity}

For an open circuit it is not possible to reduce an entire circuit down to some
output values, as there will still be open wires awaiting the next inputs.
Nevertheless, if we precompose a circuit with some inputs we can provide some
rules for propagating these inputs across the circuit.

Formally, for sequential circuit \(
    \iltikzfig{strings/category/f}[box=f,colour=seq,dom=\listvar{m},cod=\listvar{n}]
\) and values \(
    \listvar{v} \in \valuetuplestream{\listvar{m}}
\), this corresponds to finding reductions such that \(
    \iltikzfig{circuits/productivity/productive-goal-lhs}[box=f,input=\listvar{v},dom=\listvar{m},cod=\listvar{n}]
    \reductions
    \iltikzfig{circuits/productivity/productive-goal-rhs}[box=g,output=\listvar{w},dom=\listvar{m},cod=\listvar{n}]
\).
We first consider the combinational case, with our final global transformation.

\begin{lemma}[Streaming]\label{lem:streaming}
    The \emph{streaming rule} \(
        \iltikzfig{circuits/axioms/generalised-streaming-lhs}[box=f]
        \reduction[(\streamingeqn)]
        \iltikzfig{circuits/axioms/generalised-streaming-rhs}[box=f]
    \) is sound.
\end{lemma}
\begin{proof}
    Once again this can be shown by considering the stream semantics.
    First note that by unfolding the notation, \(
        \iltikzfig{circuits/axioms/generalised-streaming-lhs}[box=f]
        \coloneqq
        \iltikzfig{circuits/axioms/generalised-streaming-lhs-verbose}[box=f]
    \).
    The streaming rule is then effectively `pushing' the combinational circuit
    \(\iltikzfig{strings/category/f}[box=f,colour=comb]\) across the join.

    The join is \emph{not} a natural transformation so this does not hold in
    general, but because one argument is an instantaneous value and the other
    is a delay, at least one of the inputs to the join will be \(\bot\) for a
    given circuit.
    As the interpretations of combinational circuits must be
    \(\bot\)-preserving, the circuit can safely be pushed across the join and
    delay.
\end{proof}

The streaming rule shows that when a combinational circuit is applied to an
input with an instantaneous and a delayed component, the circuit can be copied
so that one copy handles what is happening `now' and the other handles what is
happening `later'.

\begin{example}
    \cref{fig:sr-latch-streamed} shows how the SR NOR latch in Mealy form from
    \cref{fig:sr-latch-unrolled} is transformed by the \((\streamingeqn)\) rule.

\end{example}

As there is a delay on the bottom argument of the join, the output of a streamed
circuit at the current tick is now contained entirely in the top argument of the
join.
The final rules we present will reduce this copy to values, as desired.

\begin{lemma}[Value rules]
    The following \emph{value rules} are sound:
    \begin{gather*}
        \iltikzfig{circuits/axioms/fork-lhs}[val=v]
        \reduction[(\forkeqn)]
        \iltikzfig{circuits/axioms/fork-rhs}[val=v]
        \quad
        \iltikzfig{circuits/axioms/join-lhs}[val1=v,val2=w]
        \reduction[(\joineqn)]
        \iltikzfig{circuits/axioms/join-rhs}[val1=v,val2=w]
        \quad
        \iltikzfig{circuits/axioms/stub-lhs}[val=v]
        \reduction[(\stubeqn)]
        \iltikzfig{strings/monoidal/empty}
        \quad
        \iltikzfig{circuits/axioms/gate-lhs}[gate=p,input=\listvar{v}]
        \reduction[(\gateeqn)]
        \iltikzfig{circuits/axioms/gate-rhs}[gate=p,input=\listvar{v}]
        \\
        \iltikzfig{circuits/axioms/unbundle-lhs}
        \reduction[(\unbundleeqn)]
        \iltikzfig{circuits/axioms/unbundle-rhs}
        \quad
        \iltikzfig{circuits/axioms/bundle-lhs}
        \reduction[(\bundleeqn)]
        \iltikzfig{circuits/axioms/bundle-rhs}
    \end{gather*}
\end{lemma}
\begin{proof}
    These follow immediately from \cref{lem:combinational-streams} and the
    interpretation of combinational circuits as function.
\end{proof}

Reducing the `now' core is the only time in which exhaustive application is
required, as more is involved than than just copying circuit components.

\begin{lemma}\label{lem:reduce-core-confluent}
    Applying the value rules is confluent.
\end{lemma}
\begin{proof}
    There are no overlaps between the rules.
\end{proof}

\begin{lemma}\label{lem:reduce-core-terminating}
    For a combinational circuit \(
        \iltikzfig{strings/category/f}[box=f,colour=comb,dom=\listvar{m},cod=\listvar{n}]
    \) and \(\listvar{v} \in \valuetuple{\listvar{m}}\), there exists a word
    \(\listvar{w} \in \valuetuple{\listvar{n}}\) such that applying the value
    rules exhaustively to \(
        \iltikzfig{circuits/components/circuits/f-applied}[box=F,colour=comb]
    \) terminates at \(
        \iltikzfig{circuits/components/values/vs}[val=\listvar{w}]
    \).
\end{lemma}

These rules are all we need to propagate input values across a circuit.

\begin{corollary}\label{cor:mealy-form-productivity}
    For circuit \(
        \iltikzfig{circuits/productivity/mealy-form-applied}[core=F,dom=m,cod=n,delay=x]
    \) there exist \(
        \listvar{t} \in \valuetuple{x}
    \) and \(
        \listvar{w} \in \valuetuple{n}
    \) such that \(
        \iltikzfig{circuits/productivity/mealy-form-applied}[core=F,dom=m,cod=n,delay=x]
        \reductions
        \iltikzfig{circuits/productivity/mealy-form-produced}[core=F,dom=m,cod=n,delay=x]
    \) by applying \(\streamingeqn\) once followed by the value rules
    exhaustively.
\end{corollary}

\begin{example}\label{ex:productivity}
    \cref{fig:sr-latch-streamed} shows how the `now' copy of the transformed SR
    latch circuit from \cref{fig:sr-latch-unrolled} for inputs
    \(\belnaptrue\belnapfalse\) (a `reset' pulse) is reduced by the
    combinational rules.
    The next state is \(\belnaptrue\), the first output is \(\belnapfalse\) and
    the second is \(\bot\).
    The first output (false) is what we would expect given a reset pulse, but
    the second may raise an eyebrow.
    This arises due to the delay; recall that this models inertial delay in the
    wires rather than an actual memory element.
    Subsequently, it will take another cycle to produce the expected output
    \(\belnapfalse\belnaptrue\).
\end{example}

\begin{figure*}
    \centering
    \scalebox{0.6}{\iltikzfig{circuits/examples/sr-latch/circuit-streamed-full}}
    \caption{
        Applying \(\streamingeqn\) with inputs \(\belnaptrue\belnapfalse\) to
        the circuit from \cref{fig:sr-latch-unrolled}, before using the value
        rules to reduce to its output values.
    }
    \label{fig:sr-latch-streamed}
\end{figure*}

By now putting together all the components in this section and the previous,
we have a productive strategy for processing inputs to \emph{any} sequential
circuit.

\begin{corollary}[Productivity]\label{cor:productivity}
    For sequential circuit \(
        \iltikzfig{strings/category/f}[box=F,colour=seq,dom=m,cod=n]
    \) and inputs \(\listvar{v} \in \valuetuple{m}\), there exists
    \(\listvar{w} \in \valuetuple{n}\) such that \(
        \iltikzfig{circuits/productivity/productive-goal-lhs}[box=F,input=\listvar{v},dom=m,cod=n]
        \reductions
        \iltikzfig{circuits/productivity/productive-goal-rhs}[box=G,output=\listvar{w},dom=m,cod=n]
    \) by applying \(\mealyeqn\), \(\instantfeedbackeqn\) and \(\streamingeqn\)
    once successively followed by the value rules exhaustively.
\end{corollary}

\begin{remark}
    As we saw in \cref{cor:mealy-form-productivity}, applying
    \((\streamingeqn)\) followed by the value rules to a circuit in Mealy form
    produces another circuit in Mealy form.
    This means that for one circuit and a whole stream of inputs,
    \((\mealyeqn)\) and \((\instantfeedbackeqn)\) need only be applied
    \emph{once} to get a circuit into Mealy form at the beginning before
    processing values.
    As we shall see late on this is very handy for implementing these reductions
    mechanically, as very few large circuit transformations are required.
\end{remark}