\chapter{Monoidal categories and theories}

\section{Category}

We begin by recalling some basic notions from category theory.

\begin{definition}[Category]
    A \emph{category} \(\mcc\) consists of a class of \emph{objects}
    \(\ob{\mcc}\); a class of \emph{morphisms} \(\mor{\mcc}{A}{B}\)
    for every pair of objects \(A, B \in \ob{\mcc}\); and a \emph{composition}
    operation \(
        \morph{
            {-} \circ {=}
        }{
            \mor{\mcc}{B}{C} \times \mor{\mcc}{A}{B}
        }{
            \mor{\mcc}{A}{C}
        }
    \) such that
    \begin{itemize}
        \item composition is \emph{unital}: for every \(
                    A \in \ob{\mcc}
                \) there exists an \emph{identity morphism} \(
                    \id[A] \in \mor{\mcc}{A}{A}
                \) satisfying \(
                    f \circ \id[A] = f = \id[B] \circ f
                \) for any \(
                    f \in \mor{\mcc}{A}{B}
                \); and
        \item composition is \emph{associative}: for any morphisms \(
                    f \in \mor{\mcc}{A}{B}
                \), \(
                    g \in \mor{\mcc}{B}{C}
                \) and \(h \in \mor{\mcc}{C}{D}\), \(
                    (h \circ g) \circ f = h \circ (g \circ f).
                \)
    \end{itemize}
\end{definition}

A morphism \(f \in \mor{\mcc}{A}{B}\) is also called an \emph{arrow}, and will
often be written \(\morph{f}{A}{B}\) accordingly.
The subscripts on the object and morphism classes are also often omitted.
Equations in category theory can be expressed using \emph{commutative diagrams}.
For example, the unitality and associativity of composition can be illustrated
as follows:

\begin{center}
    \includestandalone{figures/category/coherences/unitality}
    \quad
    \includestandalone{figures/category/coherences/associativity}
\end{center}

We say that the above two diagrams \emph{commute} precisely because \(
    \id[B] \circ f = f \circ \id[A]
\): the results of following two paths that diverge then intersect are equal.
When expressing conditions using commutative diagrams, it becomes natural to
also consider an alternative notation for composition.

\begin{notation}
    \emph{Diagrammatic-order} composition is defined as
    \(f \seq g := g \circ f\).
\end{notation}

We are particularly interested in using categories to model \emph{processes},
in which the morphisms are used to transform source objects into target objects.

\begin{example}
    We define a category of...
\end{example}

There is a particularly important class of morphisms which have a notion of
\emph{invertability}.

\begin{definition}[Isomorphism]
    A morphism \(\morph{f}{A}{B} \in \mcc\) is called an \emph{isomorphism} if
    there also exists morphism \(\morph{\inverse{f}}{B}{A} \in \mcc\) such
    that \(\inverse{f} \circ f = \id[A]\) and \(f \circ \inverse{f} = \id[B]\).
\end{definition}

\section{Functors}

It is rare that one only ever works with a single category at a time.
Consequently we require a way to map between them.

\begin{definition}[Functor]
    Given two categories \(\mcc\) and \(\mcd\), a \emph{functor} \(
        \morph{F}{\mcc}{\mcd}
    \) maps objects and morphisms in \(\mcc\) to objects and morphisms in
    \(\mcd\) such that
    \begin{itemize}
        \item \(F(\id[A]) = \id[FA]\) for every \(A \in \ob{\mcc})\); and
        \item \(F(g \circ f) = F(g) \circ F(f)\) for every \(\morph{f}{A}{B}\)
        and \(\morph{g}{B}{C}\).
    \end{itemize}
\end{definition}

One may even wish to compare functors themselves.

\begin{definition}[Natural transformation]
    Given two functors \(\morph{F, G}{\mcc}{\mcd}\), a
    \emph{natural transformation} \(\morph{\eta}{F}{G}\) is a family of
    morphisms \(
        \eta_A \in \mor{\mcd}{FA}{GA}
    \) called \emph{components} for each \(A \in \ob{\mcc}\) such that \(
        \eta_B \circ Ff = Gf \circ \eta_A
    \), i.e.\ the following diagram commutes:
    \begin{center}
        \includestandalone{figures/category/coherences/natural}
    \end{center}
    A natural transformation is called a \emph{natural isomorphism} if
    every component is an isomorphism.
\end{definition}

\begin{remark}
    There exists a category \(\cat\) in which `small' categories are the objects
    and functors are the morphisms.
    As natural transformations are morphisms between functors, in \(\cat\) they
    are morphisms between morphisms!
    A category with this structure is known as a \(2\)-category in higher
    category lingo, but will keep our feet firmly in the realm of
    \(1\)-categories in this thesis.
\end{remark}

\section{Monoidal categories}

As we have seen, categories allow us to string together multiple processes to
create a composite process.
However, these processes are not particularly interesting: processes cannot have
more than one input or output, and there is no notion of parallelism.
These issues can be allayed by adding structure to elevate a category into a
\emph{monoidal} category, which has a notion of \emph{parallel} composition in
addition to the already defined \emph{sequential} composition.

The parallel composition structure can be added to a regular category by using
a special kind of functor.

\begin{definition}[Product category]
    Given two categories \(\mcc\) and \(\mcd\), their \emph{product category}
    \(\mcc \times \mcd\) is the category in which the objects are defined as \(
        \ob{(\mcc \times \mcd)} := \ob{\mcc} \times \ob{\mcd}
    \) and the morphisms as \[
        \mor{(\mcc \times \mcd)}{(A, A^\prime)}{(A, A^\prime)}
        :=
        \{
            (f, f^\prime)
            \,|\,
            f \in \mor{\mcc}{A}{B},
            f^\prime \in \mor{\mcd}{A^\prime}{B^\prime}
        \}
    \]
\end{definition}

\begin{definition}[Bifunctor]
    A \emph{bifunctor} is a functor with a product category as its domain, i.e.\
    a functor of the form \(\mcc \times \mcd \to \mce\).
\end{definition}

\begin{definition}[Monoidal category]
    A \emph{monoidal category} is a category \(\mcc\) equipped with a
    bifunctor \(\morph{{-} \tensor {=}}{\mcc \times \mcc}{\mcc}\) called the
    \emph{tensor product} and an additional object \(I\) called the
    \emph{monoidal unit},
    along with natural isomorphisms
    \begin{itemize}
        \item \(
            \associator{A}{B}{C}
            \colon
            A \tensor (B \tensor C)
            \cong
            (A \tensor B) \tensor C
            \) called the \emph{associator};
        \item \(
            \leftunitor{A}
            \colon
            I \tensor A
            \cong
            A
            \) called the \emph{left unitor}; and
        \item \(
            \rightunitor{A}
            \colon
            A \tensor I
            \cong
            A
            \) called the \emph{right unitor}
    \end{itemize}
    such that the \emph{pentagon} and the \emph{triangle} diagrams below
    commute:
    \begin{center}
        \includestandalone{figures/category/coherences/pentagon}
        \includestandalone{figures/category/coherences/triangle}
    \end{center}
\end{definition}

\section{Symmetric monoidal categories}

\section{Symmetric traced monoidal categories}