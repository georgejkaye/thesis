\chapter{Monoidal categories and theories}

\section{Category}

We begin by recalling some basic notions from category theory.

\begin{definition}[Category]
    A \emph{category} \(\mcc\) consists of a class of \emph{objects}
    \(\ob{\mcc}\); a class of \emph{morphisms} \(\mor{\mcc}{X}{Y}\)
    for every pair of objects \(X, Y \in \ob{\mcc}\); and a \emph{composition}
    operation \(
        \morph{
            {-} \circ {=}
        }{
            \mor{\mcc}{Y}{Z} \times \mor{\mcc}{X}{Y}
        }{
            \mor{\mcc}{X}{Z}
        }
    \) such that
    \begin{itemize}
        \item composition is \emph{unital}: for every \(
                    X \in \ob{\mcc}
                \) there exists an \emph{identity morphism} \(
                    \id[X] \in \mor{\mcc}{X}{X}
                \) satisfying \(
                    f \circ \id[X] = f = \id[Y] \circ f
                \) for any \(
                    f \in \mor{\mcc}{X}{Y}
                \); and
        \item composition is \emph{associative}: for any morphisms \(
                    f \in \mor{\mcc}{X}{Y}
                \), \(
                    g \in \mor{\mcc}{Y}{Z}
                \) and \(h \in \mor{\mcc}{Z}{W}\), \(
                    (h \circ g) \circ f = h \circ (g \circ f).
                \)
    \end{itemize}
\end{definition}

A morphism \(f \in \mor{\mcc}{X}{Y}\) is also called an \emph{arrow}, and will
often be written \(\morph{f}{X}{Y}\) accordingly.
The subscripts on the object and morphism classes are also often omitted.

Properties in category theory are often expressed using
\emph{commuting diagrams} to aid clarity.
For example, the two properties of composition can be expressed as follows:

\begin{center}
    Commuting diagrams
\end{center}

\begin{notation}
    We also often work with so-called \emph{diagrammatic-order} composition \(
        \morph{
            {-} \seq {=}
        }{
            \mor{\mcc}{X}{Y} \times \mor{\mcc}{Y}{Z}
        }{
            \mor{\mcc}{X}{Z}
        }
    \) which is defined as \(f \seq g := g \circ f\).
\end{notation}

We are particularly interested in using categories to model \emph{processes},
in which the morphisms are used to transform source objects into target objects.

\begin{example}
    We define a category of...
\end{example}

It is rare that one only ever works with a single category at a time.
Consequently we require a way to map between them.

\begin{definition}[Functor]
    Given two categories \(\mcc\) and \(\mcd\), a \emph{functor} \(
        \morph{F}{\mcc}{\mcd}
    \) maps objects and morphisms in \(\mcc\) to objects and morphisms in
    \(\mcd\) such that
    \begin{itemize}
        \item \(F(\id[X]) = \id[FX]\) for every \(X \in \ob{\mcc})\); and
        \item \(F(g \circ f) = F(g) \circ F(f)\) for every \(\morph{f}{X}{Y}\)
        and \(\morph{g}{Y}{Z}\).
    \end{itemize}
\end{definition}

\begin{definition}[Natural transformation]
    Given two functors \(\morph{F, G}{\mcc}{\mcd}\), a
    \emph{natural transformation} \(\eta\) is a family of morphisms \(
        \eta_X \in \mor{\mcd}{FX}{GX}
    \) for each \(X \in \ob{\mcc}\) such that \(
        \eta_Y \circ Ff = Gf \circ \eta_X
    \).
\end{definition}

\section{Monoidal categories}

As we have seen, categories allow us to string together multiple processes to
create a composite process.
However, these processes are not particularly interesting: processes cannot have
more than one input or output, and there is no notion of parellism.
These issues can be allayed by adding structure to elevate a category into a
\emph{monoidal} category.

\begin{definition}[Product category]
    Given two categories \(\mcc\) and \(\mcd\), their \emph{product category}
    \(\mcc \times \mcd\) is the category in which \(
        \ob{(\mcc \times \mcd)} := \ob{\mcc} \times \ob{\mcd}
    \) and \[
        \mor{(\mcc \times \mcd)}{(X, X^\prime)}{(Y, Y^\prime)}
        :=
        \{
            (f, f^\prime)
            \,|\,
            f \in \mor{\mcc}{X}{Y},
            f^\prime \in \mor{\mcd}{X^\prime}{Y^\prime}
        \}
    \]
\end{definition}

\begin{definition}[Bifunctor]
    A \emph{bifunctor} is a functor with a product category as its domain, i.e.\
    a functor of the form \(\mcc \times \mcd \to \mce\).
\end{definition}


\begin{definition}[Monoidal category]
    A \emph{monoidal category} is a tuple \((\mcc, \tensor, I)\) where
    \begin{itemize}
        \item \(\mcc\) is a category;
        \item \(\morph{- \tensor -}{\mcc \times \mcc}{\mcc}\) is a bifunctor.
        \item \(I \in \ob{\mcc}\) is a
    \end{itemize}
\end{definition}

\section{Symmetric monoidal categories}

\section{Symmetric traced monoidal categories}