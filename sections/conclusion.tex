\chapter{Conclusion}

This brings us to the conclusion of this work; in this final chapter we will sum
up our contributions and look ahead to the future.

\section{Summary of contributions}

Our contributions can be divided into two main topics: the development of a
fully compositional theory of sequential circuits, and the application of this
framework to graph rewriting.

\subsection{Semantics of Digital Circuits}

The first major contribution of this thesis was to take the existing informal
work on categorical semantics of sequential
circuits~\cite{ghica2016categorical,ghica2017diagrammatic} and develop it into
a rigorous mathematical theory.

\subsubsection{Syntax of sequential circuits}

In the original work on categorical semantics for digital
circuits~\cite{ghica2016categorical}, the semantics of circuits were defined as
part of the base categories of circuits.
Not only was this a confusing presentation, it also `hardcoded' a particular
approach to semantics, which made later developments more fiddly.
In \cref{chap:syntax}, we built the foundations for a different approach, in
which circuits are first constructed as morphisms in a PROP \(\scircsigma\) of
\emph{syntactic} circuits with no associated behaviour.
Behaviour of circuits can be assigned by quotienting this category by some
semantic relation; this makes the framework as a whole more modular and
`semantic-agnostic'.

\subsubsection{Denotational semantics}

The semantics of circuit components was previously defined as part of
equations on the syntactic category; in \cref{chap:denotational} we defined the
notion of an \emph{interpretation} \(\interpretation\) of components as
morphisms in a PROP of monotone functions \(\funci\) in order to keep syntax and
semantics separate.
This interpretation parameterises the PROP \(\streami\) in which morphisms are
causal, monotone, finitely specified stream functions; these are the denotations
of sequential circuits.
The major contribution of this section is that this PROP has a \emph{trace}: the
least fixed point.

In order to map from circuits in \(\scircsigma\) to stream functions in
\(\streami\), we made use of \emph{Mealy machines} in \cref{sec:mealy}.
We defined a traced PROP \(\mealyi\) of \emph{monotone} Mealy machines, and
showed how we could use existing work on their \emph{coalgebraic} properties
to map between Mealy machines and stream functions using two PROP morphisms.

The most novel contributions of this chapter arise by relating Mealy machines
and stream functions back to circuits in \(\scircsigma\).
In \cref{sec:synthesis}, we defined PROP morphisms between circuits and
Mealy machines, in one direction by freely mapping circuit components to
primitive Mealy machines, and in the other by \emph{encoding} states of a
Mealy machine while preserving monotonicity.
Using these PROP morphisms, we showed in \cref{sec:denotational-completeness}
that \(\streami\) is a sound and complete denotational semantics for sequential
circuits.

\subsubsection{Operational semantics}

Previous work had defined an restricted operational semantics for closed
circuits without non-delay-guarded feedback~\cite{ghica2017diagrammatic}.
The major contribution of \cref{chap:operational} was the introduction of the
`instant feedback' rule in \cref{sec:feedback}: eliminating non-delay-guarded
feedback by iterating a circuit a certain number of times.
This rule played a key part in the productive reduction strategy presented in
\cref{sec:productivity}, which is integral to the formalism of a notion
of \emph{observational equivalence} using a relation \(\sim_\interpretation\).
We showed this relation is sound and complete with respect to the
denotational semantics, and that it is the
\emph{largest adequate congruence}~\cite{morris1969lambdacalculus,gordon1980denotational}
on \(\scircsigma\).

\subsubsection{Algebraic semantics}

In \cref{chap:algebraic} we presented a sound and complete algebraic semantics
for sequential circuits, which can be divided into three broad classes.
The first class (\cref{sec:normalising}) contains equations for
\emph{normalising} circuits into a Mealy form with a canonical (essentially
combinational) core.
The second class (\cref{sec:encoding}) contains equations for \emph{encoding}
initial states of circuits in Mealy form using Mealy homomorphisms.
The final class (\cref{sec:restriction}) contains equations for translating
between circuits which act the same way when \emph{restricted} to the
accessible internal states.
In \cref{sec:algebraic-completeness} we showed how these equations are enough to
translate between any two denotationally equivalent circuits, thus exhibiting
that this is a sound and complete algebraic semantics.

\subsection{Graph rewriting for digital circuits}

Viewing sequential circuits through the categorical lens opens up new ways of
reasoning with circuits, such as by applying recent work on string diagram
rewriting~\cite{bonchi2022string,bonchi2022stringa} using terms interpreted as
morphisms in a category \(\cspdhyp\) of cospans of hyperrgaphs.
Our second major contribution is to extend this for settings with a
\emph{traced comonoid} structure, of which \(\scircsigma\) is an example.

\subsubsection{String diagrams as hypergraphs}

In \cref{chap:hypergraphs} we defined two sub-PROPs of \(\cspdhyp\):
the PROP \(\pmcspdhyp\) of \emph{partial monogamous} cospans and
the PROP \(\plmcspdhyp\) of \emph{partial left-monogamous} cospans.
We showed the former are in correspondence with \emph{traced} terms and the
latter are in correspondence with such terms equipped with a
\emph{cocommutative comonoid} structure: terms equal by equations of STMCs or
cocommutative comonoids are mapped to ismorphic cospans of hypergraphs.

\subsubsection{Graph rewriting}

The primary reason for interpreting terms as hypergraphs is to perform automatic
reasoning with them via \emph{graph rewriting}.
In \cref{chap:rewriting} we characterised valid graph rewrites in a traced or
traced comonoid setting using \emph{traced boundary complements} and
\emph{traced left-boundary complements} respectively.
While there may be multiple valid graph rewrites, we showed that every graph
rewrite performed in this way corresponds to a valid term rewrite, so the
rewriting system is sound and complete.

\subsection{Case studies in Belnap logic}

The framework presented in this thesis is parameterised over a
\emph{signature} specifying signals and components, and an
\emph{interpretation} mapping them to behaviour.
Throughout the thesis we have considered a particular instantiation in terms of
\emph{Belnap's four-valued logic}.
In \cref{sec:denotational-belnap} we showed how the Belnap interpretation is
\emph{functionally complete} in that all monotone functions between Belnap
values can be expressed in terms of the three Belnap operations, and in
\cref{sec:algebraic-belnap} we defined the equations required to bring any
essentially combinational Belnap circuit into a normal form.
We envision that similar strategies could be applied for reasoning
about transistor-level circuits to more abstract viewpoints.

\subsection{Implementations}

The developments of this thesis have been supported with some small toy
examples, including a running example depicting an SR NOR latch.
Unfortunately, any circuit larger than this quickly explodes in size and scope,
and as such is not suitable for rendering in a book.

For experimenting with the Belnap interpretation, we developed a small tool
(\url{https://belnap.georgejkaye.com}) which can generate the
corresponding circuits given Belnap functions and truth tables.
On a larger scale, we also developed a \emph{hardware description language} for
designing and evaluating with circuits using the operational semantics,
presented in \cref{sec:hdl}.

\section{Future work}

The work presented in this thesis acts as a major milestone in the project to
develop a fully compositional theory of sequential circuits,
but there are several ways in which the work can be continued to create an
even more thorough package of categorical methods for reasoning with digital
circuits.

\subsection{Theoretical extensions}

The categorical framework for digital circuits may be sound and complete, but
it only models circuits that are fully specified with concrete values; in
\cref{chap:semantics-applications} we presented some ideas for how we could
extend the framework.
We proposed some alternate reduction rules to automate the tidying-up of
circuits (\cref{sec:tidy}) or to handle persistent inputs modelled as infinite
waveforms (\cref{sec:shortcut}).
We also examined how we could model inputs that follow protocols by adding new
components modelling uncertain values (\cref{sec:uncertain}).
At a meta-level, we also discussed how we could apply work on
\emph{layered PROPs}~\cite{lobski2022string} to view circuits at different
levels of abstraction (\cref{sec:abstraction}), or implement new inequality
relations to compare circuits that output the same signals but over different
timespans (\cref{sec:refining}).

We are particularly keen to investigate the ways that our framework can be
applied to the \emph{partial evaluation} of sequential circuits; while this is a
topic that has been examined to some
extent~\cite{singh1999partial,mckay1998dynamic,thompson2006bitlevel}, we believe
our rigorous foundations will provide a new perspective on the way forward.

\subsection{More applications}

In \cref{chap:semantics-applications} we presented some ideas for how
the categorical framework could be applied to real-world digital circuits.
It would be interesting to actually develop these ideas into proper
industry-grade techniques for working with circuits.
We could then compare \emph{benchmarks} with existing procedures, allowing us to
see whether our work has practical benefit in addition to bringing theoretical
clarity.

On the topic of implementation, our hardware description language is still
quite open-ended.
While circuits can currently be designed and (partially) evaluated, it would be
useful to add a native way of \emph{verifying} circuits rather than merely
inspecting the outputs of two circuits manually, perhaps with some built-in
verification language.
Moreover, \emph{synthesis} of circuits to more traditional circuit design
languages such as VHDL or Verilog, or even to a language suitable for printing
on silicon (so-called `netlists'), would allow for the benefits of our tool to
be combined with the experienced power of the traditional methods.

\subsection{Beyond the abstraction}

Our categorical framework focuses on a commonly-used abstraction
of \emph{sequential synchronous circuits}.
Crucially we operate in a \emph{discrete} setting, both in terms of the signals
used and the notion of time.
A potential future research direction could be to see if it is possible to adapt
the techniques used here to a more \emph{continuous} settings.

A concept important in circuit design that is not present in our abstraction is
that of \emph{fan-out}, the idea that there are only so many times one can fork
a wire before the signal it carries becomes unstable.
To model this, one would need to work with continuous signals that degrade by
some factor on each fork.
One immediate issue that would arise is that the fork would no longer be
coassociativity of the comonoid, as different branches would carry different
`strengths' of signals.

When working with potentially degraded signals, circuit designers use
\emph{amplifiers} to restore the strength of signals.
Which signal is restored can be nondeterministic, so this could add an element
of \emph{probability} to digital circuits.
Care would have to be taken, as nondeterministic computations breaks the
naturality of the copy: rolling a die once and copying the outcome is not the
same as duplicating the die and rolling each of them separately.

Our notion of delay is quite primitive, as it delays all inputs by one step.
In reality, the \emph{propagation delay} can differ depending on the change in
signal; for example, the transition from high to low may be quicker than the
transition from low to high.
This could be implemented by parameterising the \(\shiftstream{\bot}\) stream
function with different delays for different inputs; how this might affect the
rest of the framework remains to be seen.
The propagation delay can also be affected by fan-out; the higher the fan-out,
the higher the propagation delay.

The model of delay could be altered even further; rather than modelling
it as a series of discrete timesteps, it could be modelled \emph{continuously}
to handle asynchronous circuits.
This would be a significant change; since stream
functions have discrete elements, what would the denotation of circuits over
continuous time be?
It is possible that modelling asynchronous circuits would require a completely
different way of reasoning.