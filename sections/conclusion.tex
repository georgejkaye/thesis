\chapter{Conclusion}

This brings us to the conclusion of this work; in this final chapter we will sum
up our contributions and look ahead to the future.

\section{Summary of contributions}

\subsection{Syntax of sequential circuits}

In the original work on categorical semantics for digital
circuits~\cite{ghica2016categorical},

Not only was this a confusing presentation, it also `hardcoded' a particular
approach to semantics, which made later developments more fiddly.
In \cref{chap:syntax}, we built the foundations for a different approach, in
which circuits are first constructed as morphisms in a PROP \(\scircsigma\) of
\emph{syntactic} circuits with no associated behaviour.
Behaviour of circuits can be assigned by quotienting this category by some
semantic relation; this makes the framework as a whole more modular and
`semantic-agnostic'.

\subsection{Denotational semantics}

While circuits and streams had been linked in previous
work~\cite{ghica2017diagrammatica}, this was not expressed formally.
In \cref{chap:denotational} we presented a rigorous \emph{denotational}
semantics in terms of certain types of stream functions.

Previously, the semantics of circuit components was defined as part of
equations on the syntactic category; in \cref{sec:circuits} we use a notion of
an \emph{interpretation} \(\interpretation\) of components as morphisms in a
PROP of monotone functions \(\funci\) in order to keep syntax and semantics
separate.
This interpretation parameterises the \(\streami\) in which morphisms are
causal, monotone, finitely specified stream functions; these are the denotations
of sequential circuits (\cref{sec:circuits}).
The major contribution of this section is that this PROP has a \emph{trace}: the
least fixed point.

In order to map from circuits in \(\scircsigma\) to stream functions in
\(\streami\), we made use of \emph{Mealy machines} in \cref{sec:mealy}.
We defined a traced PROP \(\mealyi\) of \emph{monotone} Mealy machines, and
showed how we could use existing work on their \emph{coalgebraic} properties
to map between Mealy machines and stream functions using two PROP morphisms
\(\morph{\mealytostreami}{\mealyi}{\streami}\) and
\(\morph{\streamtomealyi}{\streami}{\mealyi}\).

The most novel contributions of this chapter arise by relating Mealy machines
and stream functions back to circuits in \(\scircsigma\).
In \cref{sec:synthesis}, we defined PROP morphisms
\(\morph{\circuittomealyi}{\scircsigma}{\mealyi}\) and
\(\morph{\mealytocircuiti}{\mealyi}{\scircsigma}\); the former by assigning
Mealy machines to components of freely generated morphisms, and the latter by
adapting the procedure of \emph{encoding} circuit states to be compatible with
monotonicity.
As a result, in \cref{sec:denotational-completeness}, we showed that
\(\streami\) is a sound and complete denotational semantics for sequential
circuits: every morphism in \(\scircsigma\) has a corresponding stream function
in \(\streami\), and conversely every stream function in \(\streami\) has
a circuit in \(\scircsigma\) with the same behaviour as the original stream
function.
This gives us a notion of \emph{denotational equivalence}
\(\approx_\interpretation\); quotienting \(\scircsigma\) by this relation
produces a PROP isomorphic to \(\streami\).

\subsection{Operational semantics}

A denotational semantics for sequential circuits is an important tool in our
arsenal, but in some situations it is not ideal because it obscures how the
\emph{structure} of a circuit affects its behaviour.
In \cref{chap:operational} we defined a reduction-based
\emph{operational semantics} for showing how inputs applied to a circuit are
propagated across components and transformed into next states and outputs.

Previous work had defined an restricted operational semantics for closed
circuits without non-delay-guarded feedback~\cite{ghica2017diagrammatic}.
The major contribution of this chapter was the introduction of the
`instant feedback' rule in \cref{sec:feedback}: eliminating non-delay-guarded
feedback by iterating a circuit a certain number of times.
This rule played a key part in the productive reduction strategy presented in
\cref{sec:productivity}; while a similar strategy had been defined previously
in \emph{loc.\ cit.\ }, this is refined presentation that drops the requirement
to `unfold' a circuit and copy the components: here we only need to use the
streaming equation followed by reductions detailing how values interact with
combinational components.

A final contribution in \cref{sec:observational} was the formalism of a notion
of \emph{observational equivalence} using a relation \(\sim_\interpretation\).
We showed that this relation is sound and complete with respect to the
denotational semantics in that two circuits have the same denotation if and only
if their observational behaviour is the same.
Following Morris~\cite{morris1969lambdacalculus,gordon1980denotational}, we
showed that \(\sim_\interpretation\) is the \emph{largest adequate congruence}
on \(\scircsigma\), so it is indeed the most appropriate notion of observational
equivalence.

\subsection{Algebraic semantics}

Operational semantics provides an intuitive way of reasoning step-by-step
about circuits, but computing observational equivalence for large circuits can
be intractible.
In \cref{chap:algebraic} we presented an
\emph{algebraic semantics for sequential circuits}.
An equational theory for circuits was presented in~\cite{ghica2016categorical}
in terms of `natural laws', but this was not complete nor was it possible
to show it was was without a formal denotational semantics.

We presented our new algebraic theory as three distinct classes of equations.
The first class (\cref{sec:normalising}) contains equations for
\emph{normalising} circuits into a Mealy form with a canonical (essentially
combinational) core.
The second class (\cref{sec:encoding}) contains equations for \emph{encoding}
initial states of circuits in Mealy form using Mealy homomorphisms.
The final class (\cref{sec:restriction}) contains equations for translating
between circuits which act the same way when \emph{restricted} to the
accessible internal states.
In \cref{sec:algebraic-completeness} we showed how these equations are enough to
translate between any two denotationally equivalent circuits, thus exhibiting
that this is a sound and complete algebraic semantics.

\section{Graph rewriting for digital circuits}

These three semantic relations are all sound and complete with respect to one
another, so if two circuits are related by one then they are related by the
others.

As an extended example of how the categorical perspective opens up new avenues,
we showed how existing work on string diagram rewriting could be adapted for
settings equipped with a \emph{traced comonoid structure}, of which
\(\scircsigma\) is an example.
This lets us use graph rewriting to reason about the semantics of digital
circuits \emph{automatically}.

\section{Case studies in Belnap logic}

The framework presented in this thesis is parameterised over a
\emph{signature} specifying signals and components, and an
\emph{interpretation} mapping them to behaviour.
Throughout the thesis we have considered a particular instantiation in terms of
\emph{Belnap's four-valued logic}, showing how one can derive the required
properties and equations.
In \cref{sec:denotational-belnap} we showed how the Belnap interpretation is
\emph{functionally complete} in that all monotone functions between Belnap
values can be expressed in terms of the three Belnap operations, and in
\cref{sec:algebraic-belnap} we defined the equations required to bring any
essentially combinational Belnap circuit into a normal form.

The Belnap interpretation is but one possible setting in which one could apply
our work; we envision that similar strategies could be applied for reasoning
about transistor-level circuits all the way up to more abstract viewpoints.

\section{Future work}

\subsection{Beyond the abstraction}

\todo[inline]{Fan-out}

\todo[inline]{Asynchronous circuits}

\subsection{More applications}

This thesis has been primarily theoretical in nature.
In \cref{chap:semantics-applications} we presented some ideas for how
the categorical framework could be applied to real-world digital circuits, but
these were just that: ideas.
It would be interesting to actually develop these ideas into proper
industry-grade techniques for working with circuits.
We could then compare \emph{benchmarks} with existing procedures, allowing us to
see whether our work has practical benefit in addition to bringing theoretical
clarity.

On the topic of implementation, our hardware description language is still
quite open-ended.
While circuits can currently be designed and (partially) evaluated, it would be
useful to add a native way of \emph{verifying} circuits rather than merely
inspecting the outputs of two circuits manually, perhaps with some built-in
verification language.
Moreover, \emph{synthesis} of circuits to more traditional circuit design
languages such as VHDL or Verilog, or even to a language suitable for printing
on silicon (so-called `netlists'), would allow for the benefits of our tool to
be combined with the experienced power of the traditional methods.