\chapter{Conclusion}

This brings us to the conclusion of this work; in this final chapter we will sum
up our contributions and look ahead to the future.

\section{Summary of contributions}

We presented the category of syntactic circuits \(\scircsigma\) and defined
three semantic relations on top of it.

\todo[inline]{Actually specify contributions}

The first is a \emph{denotational} semantics of certain types of stream
functions.

\todo[inline]{Denotational semantics}

The second is an \emph{operational semantics} based on some global
circuit transformations and local reductions.

\todo[inline]{Operational semantics}

The third is an
\emph{algebraic semantics} defined in terms of normalising, encoding, and
restriction equations.

\todo[inline]{Algebraic semantics}

These three semantic relations are all sound and complete with respect to one
another, so if two circuits are related by one then they are related by the
others.

As an extended example of how the categorical perspective opens up new avenues,
we showed how existing work on string diagram rewriting could be adapted for
settings equipped with a \emph{traced comonoid structure}, of which
\(\scircsigma\) is an example.
This lets us use graph rewriting to reason about the semantics of digital
circuits \emph{automatically}.

\todo[inline]{Graph rewriting}


The framework presented in this thesis is parameterised over a
\emph{signature} specifying signals and components, and an
\emph{interpretation} mapping them to behaviour.
Throughout the thesis we have considered a particular instantiation in terms of
\emph{Belnap's four-valued logic}, showing how one can derive the required
properties and equations.
This is but one possible setting in which one could use our work; we envision
that similar strategies could be applied for reasoning about transistor-level
circuits all the way up to more abstract viewpoints.

\section{Future work}

\subsection{Beyond the abstraction}

\todo[inline]{Fan-out}

\todo[inline]{Asynchronous circuits}

\subsection{More applications}

This thesis has been primarily theoretical in nature.
In \cref{chap:semantics-applications} we presented some ideas for how
the categorical framework could be applied to real-world digital circuits, but
these were just that: ideas.
It would be interesting to actually develop these ideas into proper
industry-grade techniques for working with circuits.
We could then compare \emph{benchmarks} with existing procedures, allowing us to
see whether our work has practical benefit in addition to bringing theoretical
clarity.

On the topic of implementation, our hardware description language is still
quite open-ended.
While circuits can currently be designed and (partially) evaluated, it would be
useful to add a native way of \emph{verifying} circuits rather than merely
inspecting the outputs of two circuits manually, perhaps with some built-in
verification language.
Moreover, \emph{synthesis} of circuits to more traditional circuit design
languages such as VHDL or Verilog, or even to a language suitable for printing
on silicon (so-called `netlists'), would allow for the benefits of our tool to
be combined with the experienced power of the traditional methods.