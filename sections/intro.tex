\chapter{Introduction}

Today's society has become dependent on digital
circuits~\cite{katz2005contemporary}, which run our computers, homes, vehicles
and much more.
These days, digital circuits are so common that one may doubt that there are
\emph{any} gaps in our theoretical understanding of them.

But while the \emph{design} of faster and more efficient circuits is a
well-trodden area, it is relatively self-contained.
We wish to support the existing techniques and procedures with alternatives
successfully applied in other fields, such as that of programming languages.
To see where the parallels lie between digital circuits and other areas,
we need a foundational, mathematically rigorous theory.
Although there has been previous work on such a
theory~\cite{lafont2003algebraic,ghica2017diagrammatic,ghica2018structural}, it
is the goal of this thesis to bring this project to its ultimate conclusion:
a \emph{fully compositional theory of synchronous sequential circuits}.
The first point of order is to unpack exactly what this means.

\section{Synchronous sequential circuits}

The term `\emph{circuit}' (or `\emph{network}') is often used for any system
constructed by connecting wires between primitive components.
Two common constructs are the ability to \emph{fork} wires or \emph{join} them
together, traditionally drawn using black dots as shown in \cref{fig:forkjoin}.
These circuits are not particularly interesting as they are structural in
nature; it is the other components that add meaning.

\begin{figure}
    \centering
    \iltikzfig{circuits/forkjoin}
    \caption{
        A circuit of forks and joins with no other primitives
    }
    \label{fig:forkjoin}
\end{figure}

When it comes to picking a set of components, there is a plethora of choices
for various applications; even when restricting to \emph{electronic} circuits
there are different flavours to consider.
One variety is \emph{analog} circuits constructed from components such
as resistors, capacitors and inductors, such as in the left of
\cref{fig:circuits}.
Reasoning with analog circuits requires manipulating equations relating
quantities such as voltage, current and resistance; in a parallel line of work
to our own, analog circuits have already been given a compositional mathematical
treatment~\cite{boisseau2022string}.

\begin{figure}
    \centering
    \raisebox{-2.2em}{
        \scalebox{0.75}{
            \begin{circuitikz}
                \draw (0,0) to[inductor] (2.5,0);
                \draw (2.5,0) to[short, *-] (4,0);
                \draw (4,0) to[resistor] (4,-2);
                \draw (2.5,0) to[capacitor] (2.5,-2);
                \draw (2.5,-2) to[short, *-] (4,-2);
                \draw (-2,-2) to[american voltage source] (2.5,-2);
                \draw (-2,0) to [voltmeter] (0,0);
                \draw (-2,-2) to (-2,0);
            \end{circuitikz}
        }
    }
    \qquad
    \iltikzfig{circuits/examples/sr-latch/real-circuit-no-text}
    \caption{
        An analog circuit, using a \emph{voltmeter}, an \emph{inductor}, a
        \emph{capacitor}, a \emph{resistor}, and a \emph{voltage source} as
        primitive components; and a digital circuit, using
        \emph{\(\nandgate\) gates} as primitive components
    }
    \label{fig:circuits}
\end{figure}

We are concerned with \emph{digital} circuits that operate over a finite
number of \emph{discrete} values.
Here there is a much stricter notion of causality linking input and output;
signals provided as input to a digital circuit propagate across the components,
producing outputs and updating state.
Digital circuits are constructed by connecting \emph{logic gates} together with
wires, to the inputs of other logic gates as illustrated in the right of
\cref{fig:circuits}.
A key point to note is that when designing a digital circuit, one may create
\emph{cycles}: paths from a component to itself.
At a low level the components of a digital circuit are still constructed using
analog parts, but the higher-level \emph{abstraction} to digital components and
discrete signals makes it far easier to design and reuse them.

Digital circuits can further be divided into classes based on their components.
A \emph{combinational} circuit is a circuit that only contains logical
operations for computing functions.
Such circuits have no memory; the outputs at each tick of the clock only depend
on the inputs received at that moment.
A far more useful class of circuits is that of \emph{sequential} circuits, which
have \emph{delay} and \emph{feedback} in addition to logic gates.
Most sequential circuits can be divided into two parts:
the \emph{combinational logic} for performing logical functions, and the
\emph{state registers} for storing data.

A typical digital circuit operates by using the combinational logic to perform
some function on the inputs and the current state, and use the results of this
computation to produce output signals and update the state with new data.
When circuits are used in practice, it is usually desired for them to run as
fast as possible: the time between providing the inputs and updating the
output and state should be minimised.

With this in mind, there are two different ways one can design a sequential
circuit.
A \emph{synchronous} circuit is one in which the state only changes in time with
some global clock signal, whereas in \emph{asynchronous} circuits the state
changes as soon as the inputs do.
The latter type of circuits is useful when speed is of the essence, but are
harder to design because small differences in how quickly components process
inputs can lead to the circuit assuming an unexpected state.
For this reason, most practical circuits are restricted to the synchronous
kind, and so this is where we are focusing our mathematical theory.

\section{Compositionality and category theory}

A (synchronous sequential) circuit can be viewed as a component with some
input and output wires; these wires can be connected to other components to
create a bigger and more complex circuits.
This is called \emph{composing} circuits, and some ways of doing this are
shown in \cref{fig:composition}.
We can compose circuits \emph{horizontally} (if the outputs of the first match
with the inputs of the second), \emph{vertically}, or even by connecting an
output wire to an input wire, creating a \emph{feedback loop}.

\begin{figure}
    \centering
    \iltikzfig{strings/category/composition}
    \qquad
    \iltikzfig{strings/monoidal/tensor}
    \qquad
    \iltikzfig{strings/traced/trace-rhs}
    \caption{
        Three types of composition: \emph{sequential}, \emph{parallel} (tensor)
        and using a \emph{trace operator}
    }
    \label{fig:composition}
\end{figure}

Our goal is to define a \emph{fully compositional} theory of synchronous
sequential circuits.
Here, we take full compositionality to mean that we can compose circuits solely
on the basis of their interfaces: we should not have to perform any sort of
semantic check or `peek inside' a circuit to find out if composition is
permitted.

Compositionality is an appealing paradigm to follow because it means we can
repeatedly split complicated circuits into simpler parts until we reach some
indivisible atomic components.
If one defines the behaviour of these components and how they interact
with composition, it is possible to establish the behaviour of some
larger circuit \emph{inductively} by breaking it down into its constituent
parts.
For example, some circuit \(f\) might be decomposed as in
\cref{fig:decomposition}.
To prove that \(f\) has some property, we can prove it holds for \(g\) and \(k\)
and verify if composition \emph{preserves} these properties.

\begin{figure}
    \centering
    \iltikzfig{strings/compositional/example-0}
    \(=\)
    \iltikzfig{strings/compositional/example-1}
    \(=\)
    \iltikzfig{strings/compositional/example-2}
    \caption{
        Decomposing a large circuit \(f\) into smaller circuits \(g\)
        and \(k\)
    }
    \label{fig:decomposition}
\end{figure}

While it is possible to work with this surface-level notion of
compositionality, it is important to establish a mathematical
foundation to determine the meaning and properties of composition; for this, we
turn to \emph{category theory}~\cite{maclane1978categories}.
A \emph{category} is made up of \emph{objects} and \emph{morphisms} (`arrows')
between them.
Any two arrows \(\morph{f}{A}{B}\) and \(\morph{g}{B}{C}\) can be
\emph{composed} to  make a new morphism \(\morph{g \circ f}{A}{C}\), and every
object \(A\) has a unique \emph{identity} morphism \(\id[A]\).
Composition is \emph{associative} and \emph{unital}; that is to say, the
following equations hold:
\[
    (h \circ g) \circ f = h \circ (g \circ f)
    \qquad
    f \circ \id[A] = f
    \qquad
    \id[B] \circ f = f
\]
It turns out that this simple concept paves the way to an array of
theorems that can generalise many areas of computer science, mathematics, and
beyond.

It is straightforward to see how a compositional process with inputs of type
\(A\) and outputs of type \(B\) could be modelled as a morphism
\(\morph{f}{A}{B}\).
But to model more complex processes one needs to work with a class of categories
known as
\emph{freely generated symmetric monoidal categories}~\cite{maclane1963natural},
categories equipped with an additional notion of \emph{parallel} composition
\(\tensor\) along with a family of morphisms
\(\morph{\swap{A}{B}}{A\tensor B}{B \tensor A}\) known as \emph{symmetries} to
swap over inputs.
Using a set of primitive components as \emph{generators}, the two composition
operators can be used in combination with the identities and symmetries to build
up more complicated terms.
These terms can be written as \emph{term strings}, such as the following:
\[
    \id[1]
    \tensor
    (f\ \tensor\ g
    \seq
    \swap{1}{1}
    )
    \seq
    (h\ \tensor\ \id[1] \seq f)
    \tensor
    \id[1]
    \seq
    g
\]

Terms described in this way are quite unintuitive, as the two different types of
composition are compressed into a one-dimensional text string.
Fortunately, symmetric monoidal categories admit an intuitive graphical notation
known as \emph{string diagrams}, in which generators are drawn as boxes
connected by wires, with the identity depicted as an empty wire and the symmetry
as swapping over wires.
For example, the term described above can be depicted diagrammatically as in
\cref{fig:string}, which is far easier to comprehend than the original term
string.

\begin{figure}
    \centering
    \iltikzfig{strings/example-smc}
    \caption{Example of a string diagram}
    \label{fig:string}
\end{figure}

String diagrams do not add any more computational power to one-dimensional
reasoning, but they are
immensely beneficial because the categorical axioms of associativity and
unitality are `absorbed' by the notation: two morphisms are equal if and only if
their string diagrams share the same connectivity between
boxes~\cite{kelly1980coherence,kissinger2014abstract,selinger2011survey}.
This makes proofs far less bureaucratic, as one can focus on the non-trivial
steps without having to constantly rearrange the bracketing of a term.
String diagrams also make the work more approachable and easier to explain to
non-mathematicians; using the diagrammatic approach it is possible to give talks
about category theory without mentioning categories at all.
There have even been books written with this
philosophy~\cite{coecke2018picturing}.


\section{Compositionality and sequential circuits}

One might argue that composition is already widespread in sequential circuit
design, and indeed it is: circuits are constructed by connecting lots of very
common primitive components together to make something more complex.
But this is done informally, as the behaviour of a circuit is usually tested
by \emph{simulating} it and seeing what happens.
We can simulate the subcomponents, but what does this mean for their composite?
Without a guarantee of full compositionality, we have no reason to
believe that connecting two well-behaved circuits together will result in
another well-behaved circuit.

Progress towards full compositionality for sequential circuits has been hampered
by the presence of the dreaded \emph{non-delay-guarded feedback loop}; a cycle
that does not pass through any state registers.
Non-delay-guarded feedback can lead to undefined behaviour; for example, if we
assume that the rightmost circuit \(f\) in \cref{fig:composition} contains no
registers, then it is not immediately obvious how to compute the first input, as
it would depend on itself.

Some approaches try to nullify this by considering only some `safe' subset of
circuits which will always be well-behaved~\cite{christensen2021wire}, by
introducing some sort of `type system' on wires so that components may only be
connected if they are guaranteed to have well-defined behaviour at the same
points in time~\cite{nigam2023modular}, or by only considering certain kinds of
composition~\cite{alekseyev2014compositional}.
While these are useful perspectives, they still shy away from true full
compositionality for sequential circuits.

Even though there are indeed times when non-delay-guarded feedback can lead to
unwanted behaviour, careful use can still result in useful circuits, so it
should not be excluded from our mathematical theory of sequential circuits.
Consider the circuit below, where \(f\) and \(g\) are inverses, i.e.\
\(
\iltikzfig{circuits/examples/yanking-rhs}
=
\iltikzfig{circuits/examples/yanking-simple}
\).
If we were to enforce that every loop in our circuits is
somehow delay-guarded, the line of equational reasoning in \cref{fig:yanking}
is forbidden; `yanking out' the otherwise trivial loop of wires would implicitly
delete a delay and alter the outputs of the circuit over time.
\begin{figure}
    \centering
    \iltikzfig{circuits/examples/yanking-lhs}
    \(=\)
    \iltikzfig{circuits/examples/yanking-rhs}
    \(=\)
    \iltikzfig{circuits/examples/yanking-simple}
    \caption{
        Example of diagrammatic equational reasoning by `yanking' a feedback
        loop
    }
    \label{fig:yanking}
\end{figure}

Non-delay-guarded feedback can also be used in clever ways to create sequential
circuits that exhibit combinational behaviour.
The circuit in \cref{fig:cyclic} is a classic example~\cite{malik1994analysis}
in which the feedback is just used to share the two subcircuits \(
\iltikzfig{strings/category/f}[box=f]
\) and \(
\iltikzfig{strings/category/f}[box=g]
\); the circuit acts as \(g \circ f\) or
\(f \circ g\) depending on the control input \(\mathsf{c}\).
Such circuits, while not following conventional design methodology, are more
efficient in terms of circuit size and power consumption.
This once more illustrates how working in a setting where not all loops are
delay-guarded may be beneficial to us.

\begin{figure}
    \centering
    \iltikzfig{circuits/examples/cyclic-combinational/circuit}
    \caption{
        Example of a circuit with non-delay-guarded feedback that can
        exhibit combinational behaviour~\cite{malik1994analysis}
    }
    \label{fig:cyclic}
\end{figure}

\section{Theories of digital circuits}

While this thesis details a fully compositional
\emph{categorical} theory of digital circuits, this is by no means the first
time sequential circuits have been given the mathematical treatment.
\emph{Mealy machines}~\cite{mealy1955method} are the \emph{de facto}
mathematical structure for specifying the behaviour of sequential circuits, and
it is well
known how they should be composed~\cite{ginzburg2014algebraic}.
In more recent times Mealy machines have been given a categorical treatment as
certain kinds of \emph{coalgebra}~\cite{rutten2006algebraic,bonsangue2008coalgebraic}.
However, Mealy machines abstract away from the \emph{components} of the circuit;
we are keen to preserve the link between structure and behaviour.

While not explicitly categorical, the idea of representing circuits as
mathematical expressions built up from primitive components was studied in the
80s by Gordon, who worked on
\emph{denotational semantics for sequential machines}~\cite{gordon1980denotational}
and used this idea to present
\emph{a model of register transfer systems}~\cite{gordon1982model}.
Gordon subsequently noted that \emph{higher order logic} would make a good fit
for a hardware description language~\cite{gordon1985why}, and this has become
a ubiquitous concept in formal verification of hardware~\cite{gupta1992formal}.

The first steps towards a categorical theory of digital circuits took place
after the turn of the millennium, when Lafont presented an
\emph{algebraic theory of boolean circuits}~\cite{lafont2003algebraic}.
This work already bears a great resemblance to the framework presented in this
thesis; circuits are presented as morphisms in a symmetric monoidal category
freely generated over a set of primitive logic gates.
However, Lafont's work only considered circuits for \emph{Boolean functions};
these circuits did not have any notion of delay or feedback.
Lafont also made use of a diagrammatic language (shown in \cref{fig:history}),
but the equations of monoidal categories still had to be applied explicitly.

\begin{figure}
    \centering
    \iltikzfig{circuits/examples/lafont}
    \qquad
    \iltikzfig{circuits/examples/sr-latch/circuit}
    \caption{
        Two representations of digital circuits, the first by
        Lafont~\cite{lafont2003algebraic} and the second by Ghica, Jung, and
        Lopez~\cite{ghica2016categorical,ghica2017diagrammatic}
    }
    \label{fig:history}
\end{figure}

It was not until 2016 that \emph{sequential} circuits were given the
categorical treatment by Ghica and Jung~\cite{ghica2016categorical}, who were
later joined by Lopez when considering how to use this for a graph-rewriting
based operational semantics~\cite{ghica2017diagrammatic}.
In this line of work, sequential circuits were modelled as morphisms in a
\emph{symmetric traced monoidal category}, a symmetric monoidal category
extended with a \emph{trace operator}.
In the context of sequential circuits, the trace operator models
\emph{feedback}: \(
\trace{}{
    \iltikzfig{strings/category/f-2-2}[colour=white,box=f]
}
=
\iltikzfig{strings/traced/trace-rhs}[colour=white,box=f]
\).

This marks a departure from many other recent works on compositional processes
such as the work on string diagrammatic
\emph{signal flow theory}~\cite{bonchi2021survey} or analog
circuits~\cite{boisseau2022string}, which operate in a setting with a
\emph{Frobenius} structure.
In addition to any application-specific components, these settings also contain
the four structural components shown in \cref{fig:frob}, forming what is known
mathematically as a \emph{commutative monoid} and a
\emph{cocommutative comonoid}.
These components are used for model the forks and joins we alluded to earlier in
this chapter; as also illustrated in \cref{fig:frob}, one can use them to create
a feedback loop.

\begin{figure}
    \centering
    \iltikzfig{strings/structure/monoid/merge}
    \quad
    \iltikzfig{strings/structure/monoid/init}
    \quad
    \iltikzfig{strings/structure/comonoid/copy}
    \quad
    \iltikzfig{strings/structure/comonoid/discard}
    \qquad
    \iltikzfig{strings/traced/trace-from-product}
    \caption{
        The four generators of a Frobenius structure, and the construction of
        a trace using them
    }
    \label{fig:frob}
\end{figure}

So why not opt for such a route for sequential circuits?
The issue arises in the form of \emph{copying}: we would like the leftmost
equation in \cref{fig:copy} circuits to hold for any circuit \(
\iltikzfig{strings/category/f}[box=f]
\).
What this means is that running \(f\) and then copying the outputs
should be exactly the same as copying the inputs and running two copies of
\(f\) in parallel.
But this seemingly innocent equation causes the construction of the feedback
loop using smaller components to fall apart.
This is because if we instantiate the circuit \(
\iltikzfig{strings/category/f}[box=f]
\) to the \(
\iltikzfig{strings/structure/monoid/init}
\) component, it can be propagated over the `end' of the trace and break the
loop, also illustrated in \cref{fig:copy}.

\begin{figure}
    \centering
    \iltikzfig{strings/structure/cartesian/naturality-copy-lhs}[box=f]
    \(=\)
    \iltikzfig{strings/structure/cartesian/naturality-copy-rhs}[box=f]
    \qquad
    \iltikzfig{strings/structure/bialgebra/init-copy-lhs}
    \(=\)
    \iltikzfig{strings/structure/bialgebra/init-copy-rhs}
    \qquad
    \iltikzfig{strings/traced/trace-from-product}
    \(=\)
    \iltikzfig{strings/traced/trace-from-product-1}
    \caption{The copying equation, and its implications}
    \label{fig:copy}
\end{figure}

This is known as the \emph{no-cloning theorem}; any setting with a Frobenius
structure cannot also admit copying.
For this reason, we need to model circuits in a \emph{traced} category in which
the feedback loop is built as one piece so it cannot be broken.

\section{Contributions}

The contributions of this thesis are split into two parts,
\emph{Semantics of Digital Circuits}, and
\emph{Graph Rewriting for Digital Circuits}.
These sections respectively correspond to two papers:
\emph{%
    A Fully Compositional Theory of Sequential Digital Circuits:
    Denotational, Operational and Algebraic Semantics%
}~\cite{ghica2024fully}, and \emph{%
    Rewriting Modulo Traced Monoidal Structure%
}~\cite{ghica2023rewriting}, which was published in
\emph{Formal Structures for Computation and Deduction (FSCD) 2023}.

\subsection{Semantics of Digital Circuits}

The first part of this thesis sets about finishing the project on categorical
semantics for digital
circuits~\cite{ghica2016categorical,ghica2017diagrammatic} in a methodical,
rigorous fashion.
We first define the categorical syntax of sequential circuits and follow this up
with three sound and complete semantic frameworks: denotational, operational,
and algebraic.
Each of these frameworks has their own benefits and intended uses; together they
form a comprehensive examination of semantics of digital circuits.
The framework is sufficiently general to encompass circuits constructed from
all manner of components ranging from the level of transistors to the level of
logic gates and beyond, but to provide some intuition we
include an extended case study into circuits constructed from
\emph{Belnap logic gates}~\cite{belnap1977useful}, an extension of traditional
Boolean logic containing the usual \(\andgate\), \(\orgate\) and \(\notgate\)
gates.
An overview of the categories involved can be seen in \cref{fig:circuits-map}.

\subsubsection{\cref{chap:syntax}: Syntax of sequential circuits}

Previously, the categories of circuits were immediately quotiented by some
`natural laws'; this made it difficult to define maps from circuits to other
categories, as equations on circuits also had to be considered.
We take a more modular approach, in which we first define the syntactic
categories \(\ccircsigma\) of \emph{combinational} circuits and \(\scircsigma\)
of \emph{sequential} circuits.
These are categories in which we can construct circuit morphisms; the three
semantic theories that we will present next provide different ways of
\emph{quotienting} these categories in order to identify circuits with the same
behaviour under some interpretation.

\subsubsection{\cref{chap:denotational}: Denotational semantics}

While the previous circuits work discussed assigning semantics to circuits in
terms of streams, this model was not constructed in great detail.
It was not even deemed important enough to appear in the conference version of
the paper, only being examined in detail in the arXiv
preprint~\cite{ghica2017diagrammatica}.
Here we present a category \(\streami\) of \emph{stream functions} with certain
properties, which serve as the denotational semantics for sequential circuits.
This denotational semantics is sound and complete in that every syntactic
circuit can be expressed as one of these stream functions, and every such stream
function can be mapped to a syntactic circuit which has the original stream
function as its behaviour.

We also define a category \(\mealyi\) of Mealy machines lifted to work on
lattices, as a `bridge' between circuits and stream functions.
As well as being essential for showing the soundness and completeness of the
denotational semantics, having this category of Mealy machines is nice to have
in its own right, as it shows how existing circuit
methodologies~\cite{kohavi2009switching} are compatible with our rigorous
mathematic framework.

Circuits that map to the same stream function are called
\emph{denotationally equivalent}.
Quotienting \(\scircsigma\) by denotational equivalence we obtain a category
\(\scircsigmai\); this is the category against which we will compare our next
two semantic theories.


\begin{figure}
    \centering
    \includestandalone{figures/circuits/map}
    \caption{Categories of digital circuits}
    \label{fig:circuits-map}
\end{figure}

\subsubsection{\cref{chap:operational}: Operational semantics}

The original motivation for a categorical theory of circuits was to create an
\emph{operational semantics} for digital circuits, bringing techniques from
software to hardware.
While such a system was presented in~\cite{ghica2017diagrammatic}, this only
worked on \emph{closed} circuits with no \emph{non-delay-guarded feedback}.
One of the main contributions of this chapter is to lift this restriction using
a novel reduction rule for eliminating non-delay-guarded feedback inspired by
the Kleene fixpoint theorem.
Combined with a generalisation of the previous reduction procedure to
work on open circuits, this means that any circuit applied to some inputs can be
reduced in order to determine its outputs and next state.

As a result of this, we also present a new formal notion of
\emph{observational equivalence} on sequential circuits, and show that it is the
correct one using the well-known universal property that it is the largest
adequate congruence relation~\cite{gordon1998operational}.
Quotienting \(\scircsigma\) by observational equivalence gives us another
semantic category of circuits \(\scircsigmaobs\).
By establishing an isomorphism between \(\scircsigmai\) and \(\scircsigmaobs\)
we show that the operational semantics is also sound and complete: two circuits
have the same behaviour as stream functions if and only if they reduce to the
same outputs for all inputs.

\subsubsection{\cref{chap:algebraic}: Algebraic semantics}

The previous framework of digital circuits was presented as an
\emph{algebraic semantics}: the category of circuits was quotiented by certain
`natural laws', which were stated as axioms rather than being derived from any
mathematical model.
These equations were not actually enough to show the desired results, so
additional quotients of `extensional equivalence' were used to add in the
remaining equalities.

In this thesis our equational theory is guided by the stream semantics,
building up to an algebraic semantics for circuits without having to add any
arbitrary quotients.
We try to stick to standard equations on algebraic structures and small `local'
equations detailing the interactions on individual generators, but the nature of
digital circuits means that some larger equations including \emph{context} are
necessary to include.
Ultimately, we define a set of equations \(\mce_\interpretation\) and show that
these equations suffice to bring any circuit to a pseudo-normal form.

Quotienting \(\scircsigma\) by these equations gives us our last semantic
category \(\scircsigmae\).
Establishing an isomorphism between \(\scircsigmai\)
and \(\scircsigmaobs\) shows that the algebraic semantics is sound and complete:
two circuits have the same behaviour as stream functions if and only if they
can be translated into each other using the equations.

\subsubsection{\cref{chap:semantics-applications}: Potential applications}

We conclude the first part of the thesis by examining some potential
applications for the categorical framework of digital circuits.
In particular, we show how one could use the framework for
\emph{partial evaluation} of circuits, and how we can use (in)equational
reasoning to develop more efficient circuits.

\subsection{Graph Rewriting for Digital Circuits}

\begin{figure}
    \centering
    \includestandalone{figures/graphs/roadmap}
    \caption{Categories of terms and cospans of hypergraphs}
    \label{fig:hypergraphs-map}
\end{figure}

While this work marks the first time the semantics of sequential digital
circuits have been given a rigorous mathematical treatment, it is not really
feasible to apply the techniques to anything more than toy circuits by hand;
trying to manually apply the techniques to actual, practical, circuits would
quickly become impractical.
Instead it is desirable to have a computer deal with all the hard work
for us and reason \emph{automatically}.
To do this, we need to represent circuits \emph{combinatorially} as graphs.

Representing the categorical syntax of digital circuits in this way was
considered in~\cite{ghica2017diagrammatic} using
Kissinger's \emph{framed point graphs}~\cite{kissinger2012pictures}.
These had several drawbacks: for example, many framed point graphs correspond to
the same string diagram term, and the category of such graphs is not fully
\emph{adhesive}, a property that provides nice properties for graph rewriting.
In the second part of the thesis, we extend more recent work on
\emph{hypergraph string diagram rewriting}~\cite{bonchi2022string,bonchi2022stringa,bonchi2022stringb}
so we can apply it to sequential digital circuits.
An overview of the categories involved can be seen in
\cref{fig:hypergraphs-map}.

\subsubsection{\cref{chap:hypergraphs}: String diagrams as hypergraphs}

Previous work on string diagram rewriting using hypergraphs showed how terms
in a freely generated symmetric monoidal category \(\smcsigma\) equipped with a
\emph{special commutative Frobenius structure} \(\frob\) correspond to morphisms
in a category of \emph{cospans of hypergraphs} \(\cspdhyp\), and terms without a
Frobenius structure correspond to morphisms in a category of
cospans of \emph{monogamous acyclic} \(\macspdhyp\).

We extend this work to terms in a freely generated symmetric traced monoidal
category \(\stmcsigma\); since these terms occupy the space in between regular
symmetric monoidal terms and those with a Frobenius structure, the
interpretation as cospans accordingly sits between \(\macspdhyp\) and
\(\cspdhyp\) in the form of the category of \emph{partial monogamous} cospans of
hypergraphs \(\pmcspdhyp\).

We furthermore extend this to the case where the traced terms are additionally
equipped with a cocommutative comonoid structure \(\ccomon\), leading to a
category of \emph{partial left-monogamous} cospans of hypergraphs
\(\plmcspdhyp\).

\subsubsection{\cref{chap:rewriting}: Graph rewriting}

Reasoning on terms interpreted as cospans of hypergraphs is performed using
\emph{double pushout (DPO) graph rewriting}.
The main computational step during this procedure is identifying a valid
\emph{pushout complement}: the context of the rewrite step.
Bonchi et al.\ showed that for Frobenius terms, any pushout complement
corresponds to a term rewrite~\cite{bonchi2022string}, and for symmetric
monoidal terms exactly one pushout complement corresponds to a term
rewrite~\cite{bonchi2022stringa}: the \emph{boundary complement}.

For the traced case and the traced comonoid case some pushout complements are
valid and some are not.
We characterise those that are as \emph{traced boundary complements} for the
former and \emph{traced left-boundary complements} for the latter.

\subsubsection{\cref{chap:graphs-applications}: Applications of graph rewriting}

Interpreting circuits as hypergraphs introduces the opportunity to evaluate them
\emph{automatically} using graph rewriting.
As a first case study, we show how graph rewriting modulo traced comonoid
structure can aid reasoning in settings with a \emph{Cartesian} structure, of
which the semantic categories of digital circuits are an example.

Subsequently, we illustrate how graph rewriting can be used as a combinatorial
implemnetation of the operational semantics for digital circuits.
This culminates in the presentation of a graph-rewriting-based hardware
description language based on the work throughout the thesis, with which one can
design and (partially) evaluate circuits in a step-by-step manner.