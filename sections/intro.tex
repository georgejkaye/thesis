\chapter{Introduction}

Today's society has become dependent on digital circuits, which are the brains
of our computers, homes, vehicles and much more.
These days, digital circuits are so common that one may doubt that there are
\emph{any} gaps in our theoretical understanding of them.

But while the \emph{design} of increasingly more efficient circuits is a
well-trodden area, it is relatively self-contained.
We wish to reframe our view of digital circuits in such a way that techniques
that have been successfully applied in other fields, such as that of programming
languages, can be applied here also.
To see where the parallels lie between digital circuits and other areas,
we need a foundational, mathematically rigorous theory.
Unfortunately, such a theory has thus far eluded us.

\section{Synchronous sequential circuits}

To define a mathematical theory for digital circuits we must be precise about
what we mean.
The term \emph{circuit} (or \emph{network}) is often used as a blanket term
for a system constructed by connecting wires between primitive components.
Even when restricting to \emph{electronic} circuits there are different flavours
to consider.
One might consider \emph{analog} circuits constructed from components such as
resistors, capacitors and inductors.
Reasoning with analog circuits requires manipulating equations relating
quantities such as voltage, current and resistance.
In a parallel line of work to our own, analog circuits have already been given
the compositional mathematical treatment~\cite{boisseau2022string}.

\begin{center}
    \scalebox{0.5}{
        \begin{circuitikz}
            \draw (0,0) to[inductor] (2.5,0);
            \draw (2.5,0) to[short, *-] (4,0);
            \draw (4,0) to[resistor] (4,-2);
            \draw (2.5,0) to[capacitor] (2.5,-2);
            \draw (2.5,-2) to[short, *-] (4,-2);
            \draw (-2,-2) to[american voltage source] (2.5,-2);
            \draw (-2,0) to [voltmeter] (0,0);
            \draw (-2,-2) to (-2,0);
        \end{circuitikz}
    }
\end{center}

We are concerned with \emph{digital} circuits that operate over a finite
number of \emph{discrete} values.
Here there is much stricter notion of causality linking input and output; the
input signals change and these propagate across the circuit, producing outputs
and updating state.
While every digital circuit could just be viewed as an analog circuit, the
discrete style of reasoning makes it far easier to design and reuse them.

\begin{center}
    \iltikzfig{circuits/examples/sr-latch/real-circuit-no-text}
\end{center}

Digital circuits can further be divided based on their components and how they
process input; our work focuses on \emph{synchronous sequential circuits}.
In a \emph{sequential} circuit, the output at a given time can depend on any of
the inputs received since the circuit was switched on; succinctly, it is a
circuit with \emph{memory}.
Most sequential circuits can be divided into two parts:
\emph{combinational logic} for performing logical functions, and
\emph{state registers} for storing data.
While purely combinational circuits do exist, most useful circuits need
at least some memory.

How exactly these state registers change over time leads to another way
to classify sequential circuits.
A \emph{synchronous} circuit is one in which the state only changes in time with
some global clock signal, whereas in \emph{asynchronous} circuits the state
changes as soon as the inputs do.
The latter type of circuits are useful when speed is of the essence, but are
harder to design because of race conditions between circuit components: if one
gate is slightly faster then the circuit could temporarily assume an unexpected
state and produce the incorrect output.
For this reason, most practical circuits are restricted to the synchronous
kind.

\section{Compositionality and category theory}

Our goal is a \emph{fully compositional} theory of synchronous sequential
circuits.
But `fully compositional' is also a phrase thrown around without much thought to
its precise meaning.
Here, we take full compositionality to mean that we can compose circuits
without fear that we will somehow create something degenerate.
We should not have to perform any sort of semantic check or `peek inside' a
circuit to find out if composition is permitted; we should be able to treat
everything as a black box.

Compositionality is an appealing paradigm to follow because it means we can
split complicated and unwieldy circuits into simpler components.
These components can themselves be split up into simpler components, and so on
until we reach some indivisible atomic components.
If one defines the behaviour of these atomic components and how they interact
with composition, it is possible to quickly establish the behaviour of some
larger circuit \emph{inductively} by piecing things together.
\[
    \iltikzfig{strings/compositional/example-0}
    =
    \iltikzfig{strings/compositional/example-1}
    =
    \iltikzfig{strings/compositional/example-2}
\]

In the above example, we can define the behaviour of the large component \(f\)
by defining the behaviour of the two small components \(g\) and \(k\), along
with what it means to compose horizontally and vertically.
To prove properties about \(f\), we can prove properties about \(g\) and \(k\)
and verify if composition \emph{preserves} these properties.

Composition is not even restricted to the mathematical realm; it is generally
considered good practice to write programs in terms of small, easily reusable
pieces that can be pieced together by composing functions, which is why
functional programming paradigms are starting to creep into even the most
imperative of languages.

While it is possible to work with this abstract, surface-level notion of
compositionality, it is important to establish a rigorous mathematical
foundation to determine what it means to compose and the properties that
arise from it; to do this, we turn to the field of
\emph{category theory}~\cite{maclane1978categories}.
A \emph{category} is made up of \emph{objects} and \emph{morphisms} (`arrows')
between them; the only required properties are that any two morphisms \(f\) and
\(g\) can be composed to make a new morphism \(f \seq g\) in an
associative manner (\((f \seq g) \seq h = f \seq (g \seq h)\)), and that every
object \(A\) has a special \emph{identity} morphism \(\id[A]\) that is
\emph{unital} (\(\id \seq f = f = f \seq \id\)).
It turns out that these simple concepts pave the way to an array of
theorems that can generalise many areas of computer science, mathematics, and
beyond.

Compositional processes go hand-in-hand with a class of categories known as
\emph{freely generated symmetric monoidal categories}~\cite{maclane1963natural},
categories equipped with a notion of \emph{parallel} composition \(\tensor\) in
addition to the ordinary categorical \emph{sequential} composition \(\seq\),
along with morphisms known as \emph{symmetries} \(\sigma\) to swap over inputs.
In a symmetric monoidal category, a process that takes something of shape \(A\)
as input and outputs something of shape \(B\) is modelled as a morphism
\(A \to B\).
Using a set of primitive components as \emph{generators}, the two composition
operators can be used in combination with the identities and symmetries to build
up more complicated terms.
\[
    \id[1]
    \tensor
    (f\ \tensor\ g
    \seq
    \swap{1}{1}
    )
    \seq
    (h\ \tensor\ \id[1] \seq f)
    \tensor
    \id[1]
    \seq
    g
\]

While complex terms can be built this way, the one-dimensional
syntactic notation quickly becomes unwieldy and confusing.
Fortunately, symmetric monoidal categories admin an intuitive graphical notation
known as \emph{string diagrams}, in which generators are drawn as boxes connected
by wires; the identity is then depicted as an empty wire and the symmetry as
swapping over wires.
\[
    \iltikzfig{strings/example-smc}
\]

String diagrams do not add any more computational power to
standard one-dimensional reasoning, but they are
immensely beneficial because the categorical axioms of associativity and
unitality are `absorbed' by the notation: two morphisms are equal if and only if
their string diagrams are isomorphic~\cite{kelly1980coherence,kissinger2014abstract}.
This makes proofs far less bureaucratic, as one can focus on the non-trivial
steps without having to constantly rearrange the bracketing of a term.
String diagrams also make the work more approachable and easier to explain to
non-mathematicians.
Using the diagrammatic approach it is possible to give talks about category
theory without mentioning categories at all; there have even been books written
with this philosophy~\cite{coecke2018picturing}.


\section{Compositionality and sequential circuits}

One might argue that composition is already widespread in sequential circuit
design, and indeed it is: circuits are constructed by connecting lots of very
common primitive components together to make something more complex.
But this is done informally, as the behaviour of a circuit is usually tested
by \emph{simulating} it and seeing what happens.
We can simulate the subcomponents, but what does this mean for their composite?
Without a guarantee of full compositionality, we have no reason to
believe that connecting two well-behaved circuits together will result in
another well-behaved circuit.

Progress towards full compositionality for sequential circuits has been hampered
by the presence of the dreaded \emph{non-delay-guarded feedback loop}; a cycle
that does not pass through any memory elements.
Non-delay-guarded feedback can lead to undefined behaviour; for example, what is
the meaning of the following circuit?
\[
    \iltikzfig{circuits/instant-feedback/trand-no-value}
\]

Some approaches try to nullify this by considering only some `safe' subset of
circuits which will always be well-behaved~\cite{christensen2021wire}, by
introducing some sort of `type system' on wires so that components may only be
connected if their timings match up~\cite{nigam2023modular}, or by only
considering certain kinds of composition~\cite{alekseyev2014compositional}.
While these are useful perspectives, they still shy away from true full
compositionality for sequential circuits.

Even though there are indeed times when non-delay-guarded feedback can lead to
unwanted behaviour, careful use can still result in useful circuits, so it
should not be excluded from our mathematical theory of sequential circuits.
For example, the following equation on circuits can only be performed
if such loops are permitted.
\[
    \iltikzfig{circuits/examples/yanking-rhs}
    =
    \iltikzfig{circuits/examples/yanking-lhs}
\]
In larger circuits such an equation could unlock more optimisation potential.

Non-delay-guarded feedback can also be used in clever ways to create sequential
circuits that exhibit combinational behaviour.
The following is a classic example~\cite{malik1994analysis} in which the
feedback is just used to share resources; the circuit acts as \(f \seq g\) or
\(g \seq f\) depending on the control input.
\[
    \iltikzfig{circuits/examples/cyclic-combinational/circuit}
\]
Such circuits, while not following conventional design methodology, are more
efficient in terms of circuit size, and hence power consumption.

\section{Theories of digital circuits}

While this thesis details a fully compositional
\emph{categorical} theory of digital circuits, this is by no means the first
time sequential circuits have been given the mathematical treatment.
\emph{Mealy machines}~\cite{mealy1955method}, the \emph{de facto} structure for
specifying the behaviour of sequential circuits, are ancient and it is well
known how they should be composed~\cite{ginzburg2014algebraic}.
In more recent times Mealy machines have been given a categorical treatment as
certain kinds of \emph{coalgebra}~\cite{rutten2006algebraic,bonsangue2008coalgebraic}.
However, Mealy machines abstract away from the \emph{components} of the circuit;
we are keen to preserve the link between structure and behaviour.

While not explicitly categorical, the idea of representing circuits as
mathematical expressions built up from primitive components was studied in the
80s by Gordon, who worked on
\emph{denotational semantics for sequential machines}~\cite{gordon1980denotational}
and used this idea to present
\emph{a model of register transfer systems}~\cite{gordon1982model}.
Gordon subsequently noted that \emph{higher order logic} would make a good fit
for a hardware description language~\cite{gordon1985why}, and this has become
a ubiquitous concept in formal verification of hardware~\cite{gupta1992formal}.

The first steps towards a categorical theory of digits took place at the turn of
the millenium, when Lafont presented an
\emph{algebraic theory of boolean circuits}~\cite{lafont2003algebraic}.
This work already bears a great resemblance to the framework presented in this
thesis; circuits
are presented as morphisms in a symmetric monoidal category freely generated
over a set of primitive logic gates.
However, Lafont's work only considered circuits for \emph{Boolean functions};
these circuits did not have any notion of delay or feedback.
\[
    \iltikzfig{circuits/examples/lafont}
\]
It was not until 2016 that \emph{sequential} circuits were given the
categorical treatment by Ghica and Jung~\cite{ghica2016categorical}, who were
later joined by Lopez when considering how to use this for a graph-rewriting
based operational semantics~\cite{ghica2017diagrammatic}.
\[
    \iltikzfig{circuits/examples/sr-latch/circuit}
\]
Here, sequential circuits were modelled as morphisms in a
\emph{symmetric traced monoidal category}, a symmetric monoidal category
extended with a \emph{trace operator}.
In the context of sequential circuits, the trace operator models
\emph{feedback}.
\[
    \trace{}{
        \iltikzfig{strings/category/f-2-2}[colour=white,box=f]
    }
    =
    \iltikzfig{strings/traced/trace-rhs}[colour=white,box=f]
\]
This marks a departure from many other recent works on compositional processes,
such as the work on string diagrammatic
\emph{signal flow theory}~\cite{bonchi2021survey} or analog
circuits~\cite{boisseau2022string}.
These works operate in a setting with a \emph{Frobenius} structure equipped with
a commutative monoid \((
\iltikzfig{strings/structure/monoid/merge},
\iltikzfig{strings/structure/monoid/init}
)\) and a cocommutative comonoid \((
\iltikzfig{strings/structure/comonoid/copy},
\iltikzfig{strings/structure/comonoid/discard}
)\).
The primary difference in such settings is that wires are implicitly
bidirectional and can be `bent' by using these primitive components.
\[
    \iltikzfig{strings/compact-closed/cup}
    \coloneqq
    \iltikzfig{strings/structure/frobenius/cup}
    \quad
    \iltikzfig{strings/compact-closed/cap}
    \coloneqq
    \iltikzfig{strings/structure/frobenius/cap}
\]
The Frobenius setting is appealing because a feedback loop can be
constructed by piecing together tiles rather than having to use an explicit
operation.
\[
    \iltikzfig{strings/traced/trace-from-product}
\]
So why not opt for such a route for sequential circuits?
One might think that the bidirectional nature of the wires already raises a
problem, as digital circuits have a much more rigorous notion of input-output
causality: they model \emph{functions} rather than \emph{relations}.
But even in a Frobenius setting it is possible to restrict to the circuits with
`left-to-right' flow, as illustrated in~\cite{bonchi2021survey}.

The actual problem arises from the construction of the trace from smaller
components; since circuits have a notion of \emph{copying}, this is
fundamentally incompatible with the Frobenius structure.
\[
    \iltikzfig{strings/traced/trace-from-product}
    =
    \iltikzfig{strings/traced/trace-from-product-1}
    =
    \iltikzfig{strings/traced/trace-from-product-2}
    =
    \iltikzfig{strings/traced/trace-from-product-bialg}
\]
For this reason, any feedback loop on circuits needs to be constructed as one
operation.

\section{Contributions}

This contributions of this thesis are split into two parts,
\emph{Semantics of Digital Circuits}, and
\emph{Graph Rewriting for Digital Circuits}.
These sections respectively correspond to two papers:
\emph{%
    A Fully Compositional Theory of Sequential Digital Circuits:
    Denotational, Operational and Algebraic Semantics%
}~\cite{ghica2024fully}, and \emph{%
    Rewriting Modulo Traced Monoidal Structure%
}~\cite{ghica2023rewriting}, which was published in
\emph{Formal Structures for Computation and Deduction (FSCD) 2023}.

\subsection{Semantics of Digital Circuits}

In the previous work, semantics of digital circuits was defined in a sometimes
confusing manner, intermingling equations on circuits with other quotients.
In the first part of this thesis we set about fixing this: we first define the
categorical syntax of sequential digital circuits and follow this up with three
sound and complete semantic frameworks: denotational, operational, and
algebraic.
Each of these frameworks has their own benefits and intended uses; together they
make a part of a comprehensive examination of semantics of digital circuits.
The framework is sufficiently general to encompass circuits constructed from
all manner of components ranging from the level of transistors to the level of
logic gates and beyond to higher abstraction, but to provide some intuition we
include an extended case study into circuits constructed from
\emph{Belnap logic gates}~\cite{belnap1977useful}, an extension of traditional
Boolean logic containing the usual \(\andgate\), \(\orgate\) and \(\notgate\)
gates.
An overview of the categories involved can be seen in \cref{fig:circuits-map}.

\begin{figure}
    \centering
    \includestandalone{figures/circuits/map}
    \caption{Categories of digital circuits}
    \label{fig:circuits-map}
\end{figure}

\subsubsection{Syntax of sequential circuits}

Previously, the syntax and semantics of digital circuits were confusingly
intermingled, with components, equations, and extensional equivalence defined
all in one go.
We take a more modular approach, in which we first define the syntactic
categories \(\ccircsigma\) of \emph{combinational} circuits and \(\scircsigma\)
of \emph{sequential} circuits.
These are categories in which we can construct circuit morphisms; the three
semantic theories that we will present next provide different ways of
\emph{quotienting} this category in order to identify circuits with the same
behaviour under some interpretation.

\subsubsection{Denotational semantics}

While the previous circuits work discussed assigning semantics to circuits in
terms of streams, soundness and completeness of this model was not considered.
It was not even deemed important enough to appear in the conference version of
the paper, only being examined in detail in the arXiv
preprint~\cite{ghica2017diagrammatica}.
Here we present a denotational semantics for sequential circuits as
\emph{stream functions} with certain properties.
This denotational semantics is sound and complete in that every syntactic
circuit can be expressed as one of these stream functions, and every such stream
function can be mapped to a syntactic circuit which has the original stream
function as its behaviour.

Along the way, we also define a category \(\mealyi\)  of Mealy machines lifted
to work on lattices as a `bridge' between circuits and stream functions.
As well as being essential for showing the soundness and completeness of the
denotational semantics, having this category of Mealy machines is nice to have
in its own right, as it shows how existing circuit
methodologies~\cite{kohavi2009switching} are compatible with our rigorous
mathematic framework.

Circuits that map to the same stream function are called
\emph{denotationally equivalent}.
By quotienting \(\scircsigma\) by denotational equivalence we obtain a category
\(\scircsigmai\); this is the category against which we will compare our next
two semantic theories.

\subsubsection{Operational semantics}

The original motivation for a categorical theory of circuits was to create an
\emph{operational semantics} for digital circuits, bringing techniques from
software to hardware.
While such a system was presented in~\cite{ghica2017diagrammatic}, this only
worked on \emph{closed} circuits with no \emph{non-delay-guarded feedback}.
One of the main contributions of this section is a novel reduction rule for
eliminating non-delay-guarded feedback inspired by the Kleene fixpoint theorem,
which, combined with a generalisation of the previous reduction procedure to
work on open circuits, means that any circuit applied to some inputs can be
reduced in order to determine its outputs and next state.

As a result of this, we also present a new formal notion of
\emph{observational equivalence} on sequential circuits, and show that it is the
correct one using the well-known universal property that it is the largest
adequate congruence relation~\cite{gordon1998operational}.
By quotienting \(\scircsigma\) by observational equivalence we obtain another
semantic category of circuits \(\scircsigmaobs\).
By establishing an isomorphism between \(\scircsigmai\) and \(\scircsigmae\)
we show that the operational semantics is also sound and complete: two circuits
have the same behaviour as stream functions if and only if they reduce to the
same outputs.

\subsubsection{Algebraic semantics}

The previous framework of digital circuits was presented as an
\emph{algebraic semantics}: the category of circuits was quotiented by certain
`natural laws', which were stated as axioms rather than being derived from any
mathematical model.
These equations were not actually enough to show the desired results, so
additional quotients of `extensional equivalence' were used to add in the
remaining equalities.

In this thesis our equational theory is guided by the stream semantics,
building up to an algebraic semantics for circuits without having to add any
arbitrary quotients.
We try to stick to standard equations on algebraic structures and small `local'
equations detailing the interactions on individual generators, but the nature of
digital circuits means that some larger circuits including \emph{context} are
necessary to include.
The main result is showing the equations suffice to bring any circuit to a
pseudo-normal form.

Quotienting \(\scircsigma\) by these equations gives us our last semantic
category \(\scircsigmae\); establishing an isomorphism between \(\scircsigmai\)
and \(\scircsigmaobs\) shows that the algebraic semantics is sound and complete:
two circuits have the same behaviour as stream functions if and only if they
can be translated into each other using the equations.

\subsection{Graph Rewriting for Digital Circuits}

\begin{figure}
    \centering
    \includestandalone{figures/graphs/roadmap}
    \caption{Categories of terms and cospans of hypergraphs}
    \label{fig:hypergraphs-map}
\end{figure}

While this work marks the first time the semantics of sequential digital
circuits have been given a rigorous mathematical treatment, it is not really
feasible to apply the techniques to anything more than toy circuits by hand;
trying to manually apply the techniques to actual, practical, circuits would
quickly become impractical.
Instead it is desirable to have a computer deal with all the hard work
for us and reason \emph{automatically}.
To do this, we need to represent circuits \emph{combinatorially} as graphs.

Representing the categorical syntax of digital circuits in this way was
considered in~\cite{ghica2017diagrammatic} using
Kissinger's \emph{framed point graphs}~\cite{kissinger2012pictures}.
These had several drawbacks, such as the need to reason modulo
\emph{wire homeomorphism} and the fact that framed point graphs were not fully
adhesive.
In the second part of the thesis, we extend more recent work on
\emph{hypergraph string diagram rewriting}~\cite{bonchi2022string,bonchi2022stringa,bonchi2022stringb}
so we can apply it to sequential digital circuits.
An overview of the categories involved can be seen in
\cref{fig:hypergraphs-map}.

The existing work showed how terms in a freely generated symmetric monoidal
category \(\smcsigma\) equipped with a
\emph{special commutative Frobenius structure} \(\frob\) could be interpreted as
morphisms in a category of
\emph{cospans of (monogamous) hypergraphs} \(\cspdhyp\).
We extend this work for terms in a freely generated symmetric \emph{traced}
monoid category \(\stmcsigma\) equipped with a
\emph{commutative comonoid structure} \(\ccomon\).
Using graph rewriting, we can express the operational and algebraic semantics
in a language that is easy for computers can understand and reason with, opening
up the possibility for automatic reasoning about complex, actually useful,
digital sequential circuits.
