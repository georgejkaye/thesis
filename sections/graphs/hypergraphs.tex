\chapter{Graphs}

String diagrams are an appealing way of reasoning with pen and paper: they bring
intuition to confusing one-dimensional text strings, can often shed light on the
next step of a proof, and most importantly, they look very pretty.
Unfortunately, they do have drawbacks: they take up a lot of time and space, and
if one is not careful the scribbles on a whiteboard might all start to merge
into one, resulting in a mess worse than the original term representation.

Instead it is desirable to perform reasoning with string diagrams
\emph{computationally}.
Although computers do not deal well with topological objects like string
diagrams `as is', they are very well acquainted with combinatorial objects: for
a computer to reason effectively with string diagrams, they must first be
interpreted as \emph{graphs}.

\section{String diagram rewriting}

String diagram graph rewriting is a relatively new field, first appearing at the
turn of the 2010s with \emph{string graphs}~\cite{%
    dixon2010open,dixon2013opengraphs,kissinger2012pictures%
}.
In string graphs, string diagrams are represented as graphs with two classes of
nodes for \emph{boxes} and \emph{wires}; it should be obvious which nodes
correspond to which element of string diagram design.
Crucially, one wire in a string diagram could be represented by arbitrarily
many wire nodes connected together; all of these different depictions are
identified by a notion of \emph{wire homeomorphism}, in which adjacent wire
nodes can be collapsed into one.

\todo[inline]{String graph example?}

String graphs modulo wire homeomorphism is a suitable setting for modelling
traced or compact closed categories.
More recently, there has been a flurry of work on string diagram rewriting
modulo \emph{Frobenius structure} using \emph{hypergraphs}~\cite{%
    bonchi2016rewriting,zanasi2017rewriting,bonchi2017confluence,%
    bonchi2018rewriting,bonchi2022string,bonchi2022stringa,bonchi2022stringb%
}.
Hypergraphs are a generalisation of graphs in which edges can have arbitrarily
many sources and targets, rather than just one each.
With hypergraphs, generators are represented as hyperedges, and connections
between generators indicated by if their sources and targets overlap.
The beauty of the hypergraph formalism is that there is no restriction for
vertices to only be incident on a single source and target, so one can easily
model structures such as monoids or comonoids.
It turns out when modelling string diagrams as hypergraphs, the equations of
a \emph{special commutative Frobenius algebra} are `absorbed': string diagrams
equal by Frobenius are interpreted as isomorphic hypergraphs.
This means that, in some sense, rewriting using hypergraphs can be even more
advantageous than using string diagrams!

\todo[inline]{Hypergraph example?}

Naturally, there have also been variations on this work where the complete
Frobenius structure is not present.
Suitable restrictions on hypergraphs and the graph rewriting process are also
identified in~\cite{bonchi2016rewriting} for rewriting
\emph{symmetric monoidal structure}.
Research followed on rewriting modulo
\emph{(co)monoid structure}~\cite{milosavljevic2023string} (`half a Frobenius')
and our work~\cite{ghica2023rewriting} on rewriting modulo
\emph{traced comonoid structure}.
The latter is the basis for this part of the thesis.

\section{Hypergraphs}

Hypergraphs are formally defined as a functor category.

\todo[inline]{Blackboard 1}

\begin{definition}[Hypergraph]
    Let \(\mathbf{X}\) be the category with set of objects
    \((\nat \times \nat) + 1\) and \(k + l\) morphisms from an object
    \((k, l)\) to \(\star\).
    The category of hypergraphs \(\hyp\) is the functor category
    \([\mathbf{X}, \set]\).
\end{definition}

One can think of the category \(\mathbf{X}\) as a `template' for the structure
of a hypergraph: the object \(\star\) represents the nodes and each object
\((k, l)\) represents hyperedges with \(k\) sources and \(l\)
targets; each such edge must pick \(k\) source nodes and \(l\) target nodes
from \(\star\).

Each object in \(\hyp\) is a functor that instantiates each object in
\(\mathbf{X}\) to a concrete set.
Evoking this notation of applying a functor to objects in \(\mathbf{X}\), for a
hypergraph \(F \in \hyp\) we write \(\vertices{F}\) for its set of
nodes and \(\edges{F}{k}{l}\) for the set of edges with \(k\) sources and \(l\)
targets.

\todo[inline]{Definition of hypergraph morphisms}

A morphism in \(\hyp\) between hypergraphs \(F\) and \(G\) consists of functions
\(\morph{\vertices{f}}{\vertices{F}}{\vertices{G}}\) and
\(\morph{\edges{f}{k}{l}}{\edges{F}{k}{l}}{\edges{G}{k}{l}}\) preserving sources
and targets in the obvious way.

Much like with regular graphs, it is much more intuitive to draw out hypergraphs
rather than look at their combinatorial representation.
We draw nodes as black dots, and hyperedges as `bubbles' with ordered tentacles
on the left and right that connect to source and target nodes respectively.

\begin{example}
    \todo[inline]{Do hypergraph example}
\end{example}


\subsubsection{Labelled hypergraphs}

From the example drawn above, it should be clear to see how hypergraphs are a
suitable representation of string diagrams: generators correspond to hyperedges
and wires to the nodes between them.
However, the hyperedges are currently not \emph{labelled}.

\begin{definition}[Slice category~\cite{lawvere1963functorial}]
    Given a category \(\mathbf{C}\) and an object \(C \in \mathbf{C}\), the
    \emph{slice category} \(\mathbf{C} / C\) has objects the morphisms of
    \(\mathbf{C}\) with target \(C\) and morphisms
    \((\morph{f}{X}{C}) \to (\morph{g}{X^\prime}{C})\) the morphisms
    \(\morph{g}{X}{X^\prime} \in \mathbf{C}\) such that \(f^\prime circ g = f\).
\end{definition}



\todo[inline]{Adapt from FSCD paper section 2}
