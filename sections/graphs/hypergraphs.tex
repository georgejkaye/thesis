\chapter{Graphs}

String diagrams are an appealing way of reasoning with pen and paper: they bring
intuition to confusing one-dimensional text strings, can often shed light on the
next step of a proof, and most importantly, they look very pretty.
Unfortunately, they do have drawbacks: they take up a lot of time and space, and
if one is not careful the scribbles on a whiteboard might all start to merge
into one, resulting in a mess worse than the original term representation.

Instead it is desirable to perform reasoning with string diagrams
\emph{computationally}.
Although computers do not deal well with topological objects like string
diagrams `as is', they are very well acquainted with combinatorial objects: for
a computer to reason effectively with string diagrams, they must first be
interpreted as \emph{graphs}.

\section{String diagram rewriting}

String diagram graph rewriting is a relatively new field, first appearing at the
turn of the 2010s with \emph{string graphs}~\cite{%
    dixon2010open,dixon2013opengraphs,kissinger2012pictures%
}.
In string graphs, string diagrams are represented as graphs with two classes of
nodes for \emph{boxes} and \emph{wires}; it should be obvious which nodes
correspond to which element of string diagram design.
Crucially, one wire in a string diagram could be represented by arbitrarily
many wire nodes connected together; all of these different depictions are
identified by a notion of \emph{wire homeomorphism}, in which adjacent wire
nodes can be collapsed into one.

\todo[inline]{String graph example?}

String graphs modulo wire homeomorphism is a suitable setting for modelling
traced or compact closed categories.
More recently, there has been a flurry of work on string diagram rewriting
modulo \emph{Frobenius structure} using \emph{hypergraphs}~\cite{%
    bonchi2016rewriting,zanasi2017rewriting,bonchi2017confluence,%
    bonchi2018rewriting,bonchi2022string,bonchi2022stringa,bonchi2022stringb%
}.
Hypergraphs are a generalisation of graphs in which edges can have arbitrarily
many sources and targets, rather than just one each.
With hypergraphs, generators are represented as hyperedges, and connections
between generators indicated by if their sources and targets overlap.
The beauty of the hypergraph formalism is that there is no restriction for
nodes to only be incident on a single source and target, so one can easily
model structures such as monoids or comonoids.
It turns out when modelling string diagrams as hypergraphs, the equations of
a \emph{special commutative Frobenius algebra} are `absorbed': string diagrams
equal by Frobenius are interpreted as isomorphic hypergraphs.
This means that, in some sense, rewriting using hypergraphs can be even more
advantageous than using string diagrams!

\todo[inline]{Hypergraph example?}

Naturally, there have also been variations on this work where the complete
Frobenius structure is not present.
Suitable restrictions on hypergraphs and the graph rewriting process are also
identified in~\cite{bonchi2016rewriting} for rewriting
\emph{symmetric monoidal structure}.
Research followed on rewriting modulo
\emph{(co)monoid structure}~\cite{milosavljevic2023string} (`half a Frobenius')
and our work~\cite{ghica2023rewriting} on rewriting modulo
\emph{traced comonoid structure}.
The latter is the basis for this part of the thesis.

\section{Hypergraphs}

We will begin by defining the categories of hypergraphs required, following the
pattern detailed in \cite{bonchi2022string}.
Hypergraphs are formally defined as a functor category.

\begin{definition}[Hypergraph~\cite{bonchi2016rewriting}]
    Let \(\mathbf{X}\) be the category with object set
    \((\nat \times \nat) + \star\) and morphisms
    \(\morph{\sources{i}}{(k,l)}{\star}\) for each \(i < k\)
    and \(\morph{\targets{j}}{(k,l)}{\star}\) for each \(j < l\).
    The category of hypergraphs \(\hyp\) is the functor category
    \([\mathbf{X}, \set]\).
\end{definition}

One can think of the category \(\mathbf{X}\) as a `template' for the structure
of a hypergraph: the object \(\star\) represents the nodes and each object
\((k, l)\) represents hyperedges with \(k\) sources and \(l\) targets; each such
edge must pick \(k\) sources and \(l\) targets from \(\star\).

Objects in \(\hyp\) are functors that instantiates each object in \(\mathbf{X}\)
to a concrete set.
Subsequently, for a hypergraph \(F \in \hyp\) we write \(\vertices{F}\) for its
set of nodes and \(\edges{F}{k}{l}\) for the set of edges with \(k\) sources and
\(l\) targets.
Since it is a functor category, the morphisms in \(\hyp\) are natural
transformations: structure-preserving maps between hypergraphs.

\begin{definition}[Hypergraph homomorphism]
    Given two hypergraphs \(F, G \in \hyp\), \emph{hypergraph homomorphism}
    \(F \to G\) consists of functions
    \(\morph{\vertices{f}}{\vertices{F}}{\vertices{G}}\) and
    \(\morph{\edges{f}{k}{l}}{\edges{F}{k}{l}}{\edges{G}{k}{l}}\) such that the
    following diagrams commute:
    \begin{center}
    \begin{tikzcd}
        \edges{F}{k}{l}
        \arrow{r}{\edges{f}{k}{l}}
        \arrow{d}{\sources{i}}
        &
        \edges{G}{k}{l}
        \arrow{d}{\sources{i}}
        \\
        \vertices{F}
        \arrow{r}{\vertices{f}}
        &
        \vertices{G}
    \end{tikzcd}
    \qquad
    \begin{tikzcd}
        \edges{F}{k}{l}
        \arrow{r}{\edges{f}{k}{l}}
        \arrow{d}{\targets{i}}
        &
        \edges{G}{k}{l}
        \arrow{d}{\targets{i}}
        \\
        \vertices{F}
        \arrow{r}{\vertices{f}}
        &
        \vertices{G}
    \end{tikzcd}
\end{center}
\end{definition}

Much like with regular graphs, it is much more intuitive to draw out hypergraphs
rather than look at their combinatorial representation.
We draw nodes as black dots, and hyperedges as `bubbles' with ordered tentacles
on the left and right that connect to source and target nodes respectively.

\begin{example}
    \todo[inline]{Do hypergraph example}
\end{example}

\subsection{Labelled hypergraphs}

From the example drawn above, it should be clear to see how hypergraphs are a
suitable representation of string diagrams: generators correspond to hyperedges
and wires to the nodes between them.
However, the hyperedges are currently not \emph{labelled} with generator
symbols.
To do this, we must first translate the notion of signature to hypergraphs.

\begin{definition}[Hypergraph signature~\cite{bonchi2016rewriting}]
    For a set of generators \(\signature\) and sorts \(\mathcal{S}\) as defined
    in \cref{def:generators}, the \emph{hypergraph signature}
    \(\hypsignature{(\Sigma, S)} \in \hyp\) is defined as follows:
    \begin{gather*}
        \vertices{\hypsignature{\Sigma}} := \{ v_n \,|\, s \in S\}
        \quad
        \edges{\hypsignature{\Sigma}}{k}{l} := \{ e_g \,|\, g \in \signature\}
        \\
        \sources{i}(e_g) := v_{\dom[e_g](i)}
        \quad
        \targets{j}(e_g) := v_{\cod[e_g](j)}
    \end{gather*}
\end{definition}

\begin{example}
    \todo[inline]{Do hypergraph signature example}
\end{example}

Labels can then be assigned to the edges of a hypergraph \(F\) using a
homomorphism \(F \to \hypsignature{\Sigma}\).
To do this to \emph{all} the hypergraphs in \(\hyp\) and create a category of
\emph{labelled} hypergraphs, we make use of some more categorical machinery.

\begin{definition}[Slice category~\cite{lawvere1963functorial}]
    For a category \(\mathbf{C}\) and an object \(C \in \mathbf{C}\), the
    \emph{slice category} \(\mathbf{C} \slice C\) has objects the morphisms of
    \(\mathbf{C}\) with target \(C\) and morphisms
    \((\morph{f}{X}{C}) \to (\morph{g}{X^\prime}{C})\) the morphisms
    \(\morph{g}{X}{X^\prime} \in \mathbf{C}\) such that \(f^\prime\circ g = f\).
\end{definition}

\begin{definition}[Labelled hypergraphs~\cite{bonchi2016rewriting}]
    Let \(\hypsigmas\) be the category of hypergraphs labelled over a set of
    generators \(\Sigma\) and sorts \(\mathcal{S}\), defined as the slice category
    \(\hyp \slice \hypsignature{(\Sigma,C)}\).
\end{definition}

\begin{example}
    \todo[inline]{Do labelled hypergraph example}
\end{example}

\subsection{Cospans of hypergraphs}

String diagrams also have \emph{input} and \emph{output} interfaces.
(Labelled) hypergraphs may have suggestively dangling nodes in the pictures,
but and are not ordered, and moreover we may wish to set a non-dangling node
as an interface.
To set the interfaces of a hypergraph, hypergraph homomorphisms are used
to `pick' the appropriate nodes.

\begin{definition}[Cospan]
    A \emph{cospan} in a category \(\mcc\) is a pair of morphisms \(X \to A\)
    and \(X \to B\) in \(\mcc\), usually written \(\cospan{X}{A}{Y}\).

    A \emph{cospan morphism} \(
        (\cospan{X}[f]{A}[g]{Y}) \to (\cospan{X}[h]{B}[k]{Y})
    \) is a morphism \(\morph{\alpha}{A}{B}\) in \(\mathbf{C}\)
    such that the following diagram commutes:
    %
    \begin{center}
        \includestandalone{figures/category/diagrams/cospan-morphism}
    \end{center}
%
    Two cospans \(\cospan{X}{A}{Y}\) and \(\cospan{X}{B}{Y}\) are
    \emph{isomorphic} if there exists a morphism of cospans as above where
    \(\alpha\) is an isomorphism.
\end{definition}


As with all the constructions so far, cospans must be assembled into a category
to be useful for our purpose.
This means a notion of \emph{composition} of cospans is required.

\begin{definition}[Composition of cospans]
    \label{def:cospan-composition}
    In a category \(\mcc\) with pushouts, the composition of cospans
    \(\cospan{X}[f]{A}[g]{Y}\) and \(\cospan{Y}[h]{B}[k]{Z}\) is by pushout:
    \begin{center}
        \includestandalone{figures/category/diagrams/cospan-composition}
    \end{center}
\end{definition}

\begin{definition}[Categories of cospans]
    Let \(\mcc\) be a category with pushouts and an initial object.
    The category of cospans over \(\mathbf{C}\), denoted \(\csp{\mathbf{C}}\),
    has as objects the objects of \(\mathbf{C}\) and as morphisms \(A \to B\)
    the isomorphism classes of cospans \(\cospan{A}{X}{B}\) for some
    \(X \in \mcc\).
    Composition is by pushout as detailed in \cref{def:cospan-composition} and
    the identity is \(X \xrightarrow{\id[X]} X \xleftarrow{\id[X]} X\).

    This category is symmetric monoidal with tensor given by the coproduct in
    \(\mathbf{C}\), unit the initial object \(0 \in \mathbf{C}\), and symmetry
    by \(\cospan{A+B}{A+B}{B+A}\).
\end{definition}

Interfaces are assigned to a hypergraph \(F\) by having it occupy the `apex' of
a cospan and having the `legs' on either side pick inputs and outputs
respectively.

\begin{definition}[Discrete hypergraph]
    A \emph{discrete hypergraph} is a hypergraph in which each edge set is
    empty.
\end{definition}

A discrete hypergraph with \(n\) nodes is written as \(n\) when clear from
context.
Discrete hypergraphs can represent the \emph{number} of interface nodes; all
that remains is the \emph{order}.

\begin{theorem}[\cite{bonchi2022string}, Thm. 3.6]
    Let \(\mathbb{X}\) be a PROP whose monoidal product is a coproduct,
    \(\mathbf{C}\) a category with pushouts and an initial object, and \(
        \morph{F}{\mathbb{X}}{\mathbf{C}}
    \) a coproduct-preserving functor.
    Then there exists a PROP \(\csp[F]{\mathbf{C}}\) whose arrows \(m \to n\)
    are isomorphism classes of \(\mathbf{C}\) cospans \(\cospan{Fm}{C}{Fn}\).
\end{theorem}


\begin{theorem}[\cite{bonchi2022string}, Thm. 3.8]
    \label{thm:cospan-homomorphism}
    Let \(\mathbb{X}\) be a PROP whose monoidal product is a coproduct,
    \(\mathbf{C}\) a category with a pushout and an initial object, and
    \(\morph{F}{\mathbb{X}}{\mathbf{C}}\) a coproduct-preserving functor.
    Then there is a homomorphism of PROPs \(
        \morph{\tilde{F}}{\csp{\mathbb{X}}}{\csp[F]{\mathbf{C}}}
    \) that sends \(\cospan{m}[f]{X}[g]{n}\) to \(\cospan{Fm}[Ff]{FX}[Fg]{Fn}\).
    If \(F\) is full and faithful, then \(\tilde{F}\) is faithful.
\end{theorem}

\begin{definition}
    Let \(\finset\) be the PROP with morphisms \(m \to n\) the functions
    between finite sets \([m] \to [n]\).
\end{definition}

Since we are working in a setting with a set of sorts \(C\), we need to work
with \emph{multi-sorted} finite sets; these can be obtained by using the slice
category \(\finset \slice C\).

\begin{definition}[\cite{bonchi2022string}]
    Let \(C\) be a set of sorts, and let
    \(\morph{D}{\finset \slice C}{\hypsigma}\) be the faithful,
    coproduct-preserving functor that sends each object
    \(\listvar{x} \in \finset\) to the discrete hypergraph \(m \in \hypsigma\)
    and each morphism to the induced homomorphism of discrete hypergraphs.
\end{definition}

\todo[inline]{Adapt from FSCD paper section 2}
