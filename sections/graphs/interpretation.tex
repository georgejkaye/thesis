\chapter{String diagrams as hypergraphs}

It is now time to establish the correspondence between cospans of hypergraphs
and string diagrams.
We will first recount the constructions used by Bonchi et al for the
Frobenius~\cite{bonchi2022string} and symmetric
monoidal~\cite{bonchi2022stringa} cases, befofe showing how we can use their
ingredients for a \emph{traced} setting either with or without a comonoid
structure.

\section{Frobenius terms}

As we shall soon see, it is actually simpler to start with the more complicated
Frobenius terms.
This is because \emph{every} Frobenius term corresponds to a cospan of
hypergraphs and vice versa.

\begin{definition}[\cite{bonchi2022string}]\label{def:hyp-morphisms}
    Let \(\morph{\termtohypsigmas}{\smcsigmas}{\cspdshyp}\) be a PROP
    morphism defined as
    \begin{gather*}
        \termtohyp[\iltikzfig{strings/category/generator}[box=\phi,colour=white,dom=m,cod=n]]{\Sigma}
        :=
        \cospan{m}{\iltikzfig{graphs/terms/generator}
        }{n}
        \\
        \termtohyp[\iltikzfig{strings/category/identity}[colour=white,obj=n]]{\Sigma}
        :=
        \cospan{n}[\id]{n}[\id]{n}
        \qquad
        \termtohyp[\iltikzfig{strings/symmetric/symmetry}[colour=white,obj1=m,obj2=n]]{\Sigma}
        :=
        \cospan{m+n}[[\id,\id]]{m+n}[[\id,\id]]{n+m}
    \end{gather*}
\end{definition}

\begin{proposition}[\cite{bonchi2022string}]
    \(\termtohypsigmas\) is faithful.
\end{proposition}

This handles the regular symmetric monoidal generators; we also need to map from
the Frobenius generators.
Recall the definition of a \emph{hypergraph category} from
\cref{def:hypergraph-category}; perhaps confusingly, the category of hypergraphs
\(\hyp\) is \emph{not} a hypergraph category but the category of \emph{cospans}
of hypergraphs is!

\begin{proposition}[\cite{carboni1987cartesian,bonchi2022string}]
    \label{prop:frobenius-map}
    \(\cspdshyp\) is a hypergraph category.
\end{proposition}
\begin{proof}
    A Frobenius structure can be defined on \(\cspdhyp\) for each \(n \in \nat\)
    as follows:
    \begin{gather*}
        \iltikzfig{strings/structure/monoid/merge}[colour=white, obj=n]
        :=
        \cospan{n + n}{n}{n}
        \quad
        \iltikzfig{strings/structure/monoid/init}[colour=white, obj=n]
        :=
        \cospan{0}{n}{n}
        \\
        \iltikzfig{strings/structure/comonoid/copy}[colour=white, obj=n]
        :=
        \cospan{n}{n}{n+n}
        \quad
        \iltikzfig{strings/structure/comonoid/discard}[colour=white, obj=n]
        :=
        \cospan{n}{n}{0}
        \qedhere
    \end{gather*}
\end{proof}

\begin{definition}
    Let \(\morph{\frobtohypsigmas}{\frobs}{\cspdshyp}\) be a PROP morphism
    defined as in \cref{prop:frobenius-map}.
\end{definition}

This means we can map from both generators of symmetric monoidal terms and the
Frobenius structure to cospans of hypergraphs.
Therefore to map from Frobenius terms, we simply put them together.

\begin{definition}
    Let \(
        \morph{\termandfrobtohypsigmas}{\smcsigmas + \frobs}{\cspdshyp}
    \) be the copairing of \(\termtohypsigmas\) and
    \(\frobtohypsigmas\).
\end{definition}

\begin{proposition}[\cite{bonchi2022string}, ]

\end{proposition}

\begin{proposition}[\cite{bonchi2022string}]\label{prop:tohyp-faithful}
    \(\termtohypsigmas\) and \(\frobtohypsigmas\) are faithful
\end{proposition}
\begin{proof}

\end{proof}

\todo[inline]{Adapt from FSCD paper section 2}

\section{Traced terms}

We now turn our attention to \emph{traced} terms.
Recall that every compact closed category has a \emph{canonical trace} by
\cref{thm:canonical-trace}, and that every hypergraph category is self-dual
compact closed.
Using this, we can reuse the PROP morphisms from the previous section for our
scenario.
Since we have a map from \(\smcsigma + \frob\) to cospans of hypergraphs, we
need to interpret every term in \(\stmcsigma\) in terms of components from
either \(\smcsigma\) or \(\frob\).

% TODO this might be already done in the circuits section
\begin{lemma}\label{lem:smc-core}
    Let \(\iltikzfig{strings/category/f}[box=f,colour=white,dom=m,cod=n]\) be a
    term in \(\stmc{\Sigma}\).
    Then there exists at least one \(
        \iltikzfig{strings/category/f-2-2}[box=g,colour=white,dom1=x,dom2=m,cod1=x,cod2=n]
        \in \smc{\Sigma}
    \) such that \(
        \trace{x}{\iltikzfig{strings/category/f-2-2}[box=g, colour=white]}
        =
        \iltikzfig{strings/category/f}[box=f,colour=white]
    \).
\end{lemma}
\begin{proof}
    Any trace can be made a `global trace' by tightening and superposing.
\end{proof}

\begin{proposition}
    There exists a faithful PROP morphism \(
        \morph{\tracedtosymandfrobsigmas}{\stmc{\Sigma}}{\smc{\Sigma} + \frobs}
    \).
\end{proposition}
\begin{proof}
    \cref{lem:smc-core} is used to isolate a term in \(\smc{\Sigma}\).
    The corresponding term in \(\smc{\Sigma} + \frob\) is then the canonical
    trace of this term.
    There may be many such terms in \(\smc{\Sigma}\), but the canonical trace
    being a trace means that any possible outcomes post-trace are all equal.
    The equations of \(\frob\) do not merge any morphisms
    since the only use of the generators of \(\frob\) is in the canonical trace,
    to which the Frobenius equations do not apply.
\end{proof}

This means that a term in \(\stmcsigma\) is translated into a cospan of
hypergraphs by applying
\(\termandfrobtohypsigmas \circ \tracedtosymandfrobsigmas\).

\begin{corollary}
    \(\termandfrobtohypsigmas \circ \tracedtosymandfrobsigmas\) is faithful.
\end{corollary}
\begin{proof}
    \(\tracedtosymandfrobsigmas\) is faithful by definition and
    \(\termandfrobtohypsigmas\) is faithful by \cref{prop:tohyp-faithful}.
\end{proof}

This means that every distinct traced term has a \emph{unique} cospan of
partial monogamous hypergraphs up to isomorphism.
All that remains to show is that every

\begin{definition}
    Let \(\permsprop\) be the sub-PROP of \(\finsetprop\) containing only the
    bijective functions.
\end{definition}

\begin{lemma}\label{lem:symmetries-prop}
    \(\smc{} \cong \permsprop\).
\end{lemma}
\begin{proof}
    The morphism \(\morph{\phi}{\smc{}}{\permsprop}\) is defined over
    generators in \(\smc{}\) as \[
        \phi(\iltikzfig{strings/monoidal/empty}) = \{\}
        \quad
        \phi(\iltikzfig{strings/category/identity}[colour=white])
        =
        \{0 \mapsto 0\}
        \quad
        \phi(\iltikzfig{strings/symmetric/symmetry}[colour=white])
        =
        \{0 \mapsto 1, 1 \mapsto 0\}
    \]
    Since any term in \(\smc{}\) can be expressed using these generators,
    this defines the complete transformation.

    The reverse morphism \(\morph{\psi}{\finsetprop}{\smc{}}\) is inductively
    over the size of \(m\).
    For the base case \(\morph{f}{[0]}{[0]}\), let \(
        \phi(f) := \iltikzfig{strings/monoidal/empty}
    \).
    For \(
        \morph{f}{[k+1]}{[k+1]}
    \), let \(i\) such that \(f(i) = k+1\), and define the function \(
        \morph{f^\prime}{\nat_{k}}{\nat_{k}}
    \) as the function such that \(
        f^\prime(j) = f(j)
    \) if \(j < i\), and \(f(j+1)\) otherwise.
    Then \[
        \psi(f) := \iltikzfig{strings/symmetric/f-construction}.
    \]

    These are shown to be inverses by a simple induction in both directions.
\end{proof}

\begin{lemma}\label{lem:monog-discrete-cospan}
    Given a monogamous cospan \(\cospan{m}[f]{m}[g]{m}\), there exists a unique
    term \(
        \iltikzfig{strings/category/f}[box=h,colour=white,cod=m,dom=m]
        \in \smc{}
    \) up to the axioms of SMCs such that \(
        \termtohyp[\iltikzfig{strings/category/f}[box=h,colour=white]]{\Sigma}
        =
        \cospan{m}[f]{m}[g]{m}
    \).
\end{lemma}
\begin{proof}
    Since the cospan is monogamous, \(f\) and \(g\) are mono.
    As the cospan is also discrete, there exists a (unique) bijective
    function \(\morph{h^\prime}{[m]}{[m]}\) such that \(h^\prime(i) = j\) if
    \(f(i) = g(j)\).
    By \cref{lem:symmetries-prop}, there is a corresponding term \(
        \iltikzfig{strings/category/f}[box=h,colour=white,cod=m,dom=m]
        \in \smc{}
    \) that is unique up to SMC axioms: a simple induction shows that \(
        \termtohyp[\iltikzfig{strings/category/f}[box=h,colour=white]]{\Sigma}
        =
        \cospan{m}[f]{m}[g]{m}
    \).
\end{proof}

These cospans are used to construct a term in \(\stmcsigma\) for a given cospan
of partial monogamous hypergraphs, establishing partial monogamy as the
defining characteristic of cospans in the image of \(
    \termandfrobtohypsigma \circ \tracedtosymandfrob{\Sigma}
\).

\begin{theorem}\label{thm:termtohyp-image}
    A cospan \(\cospan{m}{F}{n}\) is in the image of \(
        \termandfrobtohypsigma \circ \tracedtosymandfrob{\Sigma}\) if
    and only if it is partial monogamous.
\end{theorem}
\begin{proof}
    To show that \(\tracedtosymandfrob[\termtohyp[f]{\Sigma}]{\Sigma}\) is partial
    monogamous for any \(f \in \stmc{\Sigma}\) we use induction on the structure of
    \(f\).
    Generators, identities and symmetries are partial monogamous, as
    semi-monogamicity is preserved by composition, tensor and trace.
    So \(\termtohyp[f]{\Sigma}\) is partial monogamous.

    Now we show that any partial monogamous cospan \(
        \cospan{m}[f]{F}[g]{n}
    \)
    must be in the image of \(
        \termtohyp{\Sigma} \circ \tracedtosymandfrob{\Sigma}
    \).
    To do this, we will now construct a cospan that is isomorphic to
    \(\cospan{m}[f]{F}[g]{n}\), but from which it is possible to read off a
    unique term in \(\stmc{\Sigma}\).
    The components of the new cospan are as follows:
    \begin{itemize}
        \item let \(L\) be the hypergraph containing vertices with degree
                \((0,0)\) that are not in the image of \(f\) or \(g\);
        \item let \(E\) be the hypergraph containing hyperedges of \(F\) and
                their source and target vertices, but disconnected;
        \item let \(V\) be the discrete hypergraph containing all the
                vertices of \(F\); and
        \item let \(S\) and \(T\) be the discrete hypergraphs containing
                the source and target vertices of hyperedges in \(F\)
                respectively, with the ordering determined by some order
                \(e_1,e_2,\cdots,e_n\) on the edges in \(F\).
    \end{itemize}

    These parts can be composed and a trace applied to obtain the follow
    cospan:
    \begin{gather}
        \trace{T}{
            \cospan{T + m}[\id + f]{V}[\id + g]{S + n}
            \,\seq\,
            \cospan{\emptyset + S + n}[\id]{L + E + n}[\id]{\emptyset + T + n}
        }
        \label{gat:cospan}
    \end{gather}

    This can be checked to be isomorphic to the original cospan
    \(\cospan{m}[f]{F}[g]{n}\) by applying the pushouts.
    From this we can read off a term in \(\stmc{\Sigma}\):
    Since the first cospan is monogamous, it corresponds to a term \(
        \iltikzfig{strings/category/f-2-2}[box=f,colour=white,dom1={|\vertices{T}|},dom2=m,cod1={|\vertices{S}|},cod2=n]
    \) by \cref{lem:monog-discrete-cospan}.
    The second cospan corresponds to \(
        \iltikzfig{strings/category/f}[box=g,colour=white,dom={|\vertices{S}|},cod={|\vertices{T}|}]
        :=
        \bigtensor_{v \in \vertices{L}}
        \iltikzfig{strings/traced/trace-id}
        \tensor
        \bigtensor_{e \in 0 \leq i \leq n}
        \iltikzfig{graphs/isomorphism/label-e}
        \tensor
        \iltikzfig{strings/category/identity}[colour=white,obj=n]
    \), where \(\elabel{}(e)\) is the generator in \(\generators\) that \(e\) is
    labelled with.
    Putting this all together yields \(
        h := \termtohypsigma[\iltikzfig{graphs/isomorphism/construction}]
    \).
    While there may be multiple orderings on the edges, the possible terms
    are equal by sliding and the naturality of symmetry, so there is one
    unique term \(
        \iltikzfig{strings/category/f}[box=h,colour=white]
    \) that corresponds to cospan (\ref{gat:cospan}).
    It is clear by definition that \(
        \termtohypsigma[\iltikzfig{strings/category/f}[box=h,colour=white]]
    \) produces (\ref{gat:cospan}), which is isomorphic to the original
    cospan \(\cospan{m}[f]{F}[g]{n}\), so it is in the image of
    \(\termtohypsigma \circ \tracedtosymandfrob{\Sigma}\).
\end{proof}

This shows that \(
    \termandfrobtohypsigma \circ \tracedtosymandfrob{\Sigma}
\) is a \emph{full} mapping from \(\stmcsigma\) to \(\pmcspdhyp\).
As \(\tracedtosymandfrob{\Sigma}\) is faithful by definition and
\(\termandfrobtohypsigma\) are faithful by \cref{prop:tohyp-faithful}, the
entire mapping is also faithful: \(\stmcsigma\) is mapped to a \emph{unique}
cospan of hypergraphs up to isomorphism.

\begin{corollary}\label{cor:stmc-graph-iso}
    \(\stmc{\Sigma} \cong \pmcspfihyp\).
\end{corollary}