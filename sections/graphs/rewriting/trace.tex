\section{Rewriting with traced structure}

Say we have a rewrite rule and a term as illustrated below: \[
    \iltikzfig{strings/category/f}[colour=white,box=l]
    =
    \iltikzfig{strings/category/f}[colour=white,box=r]
    \qquad
    \iltikzfig{strings/rewriting/rewrite-trace-frob-l-nobox}
\]
Clearly the rule should be valid in a traced setting, but when assembling the
term into the form of \cref{def:term-rewriting} not all of the pieces are traced
terms.
\[
    \iltikzfig{strings/rewriting/rewrite-trace-frob-l-nobox}
    =
    \iltikzfig{strings/rewriting/rewrite-trace-frob}
\]
Fortunately, by tweaking the layout of the terms this term can be put into a
form in which we can isolate the instance of a rewrite rule and the remaining
context such that all of the pieces are valid traced monoidal terms.

\begin{definition}[Traced rewriting system]\label{def:term-rewriting}
    \index{traced rewriting system}
    \index{rewriting system!traced}
    A \emph{rewriting system} \(\mcr\) for a traced PROP \(\mcc\)
    consists is a set of \emph{rewrite rules} \(
    \rrule{
        \iltikzfig{strings/category/f}[box=l,colour=white,dom=i,cod=j]
    }{
        \iltikzfig{strings/category/f}[box=r,colour=white,dom=i,cod=j]
    }
    \).
    Given terms \(
    \iltikzfig{strings/category/f}[box=g,colour=white,dom=m,cod=n]
    \) and \(
    \iltikzfig{strings/category/f}[box=h,colour=white,dom=m,cod=n]
    \) in \(\stmcsigma\) we write \(
    \iltikzfig{strings/category/f}[box=g,colour=white]
    \rewrite[\mcr]
    \iltikzfig{strings/category/f}[box=h,colour=white]
    \) if there exists a rewrite rule \(\rrule{
        \iltikzfig{strings/category/f}[box=l,colour=white,dom=i,cod=j]
    }{
        \iltikzfig{strings/category/f}[box=r,colour=white,dom=i,cod=j]
    }\) in \(\mcr\) and \(
    \iltikzfig{strings/category/f-2-2}[box=c,colour=white,dom1=j,dom2=m,cod1=i,cod2=n]
    \) in \(\stmcsigma\) such that \[
        \iltikzfig{strings/category/f}[box=g,colour=white]
        =
        \iltikzfig{strings/rewriting/rewrite-l}
        \quad
        \text{and}
        \quad
        \iltikzfig{strings/category/f}[box=h,colour=white]
        =
        \iltikzfig{strings/rewriting/rewrite-r}
    \] by axioms of STMCs.
    We write \(
    \iltikzfig{strings/category/f}[box=l,colour=white]
    \rewrites[\mcr]
    \iltikzfig{strings/category/f}[box=r,colour=white]
    \) for a sequence of such rules.
\end{definition}


In the Frobenius setting, every pushout complement is a valid rewrite, but there
is no reason for the same to be the case for traced or traced comonoid
rewriting.
Bonchi et al showed in~\cite{bonchi2022stringa} that \emph{exactly one} pushout
complement corresponds to a valid rewrite in the symmetric monoidal case by
characterising it as a \emph{boundary complement}.

\begin{definition}[Boundary complement {\cite[Def.\ 30]{bonchi2022stringa}}]
    \index{boundary complement}
    For monogamous cospans \(
    \cospan{i}[a_1]{L}[a_2]{j}
    \) and \(
    \cospan{m}[b_1]{G}[b_2]{n}
    \) and a monomorphism \(\morph{f}{L}{G}\), a pushout complement as below
    \begin{center}
        \begin{tikzcd}[column sep=large]
            L \arrow[swap]{d}{f}
            &
            i + j
            \arrow[swap]{l}{a \coloneqq [a_1, a_2]}
            \arrow{d}{c \coloneqq [c_1, c_2]}
            \\
            G
            \arrow["\urcorner"{anchor=center, pos=0.125}, draw=none]{ur}
            &
            C
            \arrow{l}{g}
            \\
            &
            m + n
            \arrow{ul}{[b_1,b_2]}
            \arrow[swap]{u}{d \coloneqq [d_1,d_2]}
        \end{tikzcd}
    \end{center}
    is called a \emph{boundary complement} if the morphisms \(c_1\) and \(c_2\) are
    mono and \(
    \cospan{j + m}[[c_2,d_1]]{C}[[d_2,c_1]]{n + i}
    \) is a monogamous cospan.
\end{definition}

\begin{proposition}[\cite{bonchi2022stringa}, Prop.\ 31]
    When boundary complements exist in \(\hypsigma\), they are unique.
\end{proposition}

For rewriting in a traced setting we weaken boundary complements, replacing
references to monogamy with partial monogamy.

\begin{definition}[Traced boundary complement]
    \index{traced boundary complement}
    \label{def:traced-boundary-complement}
    For partial monogamous cospans \(
    \cospan{i}[a_1]{L}[a_2]{j}
    \) and \(
    \cospan{m}[b_1]{G}[b_2]{n}
    \), a pushout complement as below
    \begin{center}
        \begin{tikzcd}[column sep=large]
            L \arrow[swap]{d}{f}
            &
            i + j
            \arrow[swap]{l}{a \coloneqq [a_1, a_2]}
            \arrow{d}{c \coloneqq [c_1, c_2]}
            \\
            G
            \arrow["\urcorner"{anchor=center, pos=0.125}, draw=none]{ur}
            &
            C
            \arrow{l}{g}
            \\
            &
            m + n
            \arrow{ul}{[b_1,b_2]}
            \arrow[swap]{u}{d \coloneqq [d_1,d_2]}
        \end{tikzcd}
    \end{center}
    is called a \emph{traced boundary complement} if the morphisms \(c_1\) and
    \(c_2\) are mono and \(
    \cospan{j + m}[[c_2,d_1]]{C}[[d_2,c_1]]{n + i}
    \) is a partial monogamous cospan.
\end{definition}

By restricting to traced boundary complements, DPO rewriting can be formulated
for terms in a traced setting.

\begin{definition}[Traced DPO]
    \index{traced DPO}
    \index{DPO!traced}
    For morphisms \(G \leftarrow m + n\) and \(H \leftarrow m + n\) in
    \(\hypsigma\), there is a traced rewrite \(G \trgrewrite[\mcr] H\) if there
    exists a rule \(
    \spann{L}{i + j}{R} \in \mcr
    \) and cospan \(
    \cospan{i + j}{C}{m + n} \in \hypsigma
    \) such that the diagram in \cref{def:dpo-rewriting} commutes and \(i + j \to C\)
    is a traced boundary complement.
\end{definition}

In a traced boundary complement, the matching \(L \to G\) is not required to
be mono; permitting the matching to merge vertices means that incidences of a
rewrite rule can be found inside a trace.

\begin{example}
    Consider the rule \(
    \rrule{
        \iltikzfig{graphs/dpo/traced-example/rule-lhs}
    }{
        \iltikzfig{graphs/dpo/traced-example/rule-rhs}
    }
    \) and the term \(
    \iltikzfig{graphs/dpo/traced-example/term}
    \), in which there is clearly an instance of the rule.
    The interpretation of this as a DPO derivation with a valid traced boundary
    complement is illustrated below.
    \begin{center}
        \includestandalone{figures/graphs/dpo/traced-example/rewrite}
    \end{center}
\end{example}

A key feature of rewriting modulo traced structure is the \emph{yanking} axiom,
which can lead to some non-obvious rewrites.

\begin{example}
    Consider the rule \(
    \rrule{
        \iltikzfig{graphs/dpo/split-loop/rule-lhs}
    }{
        \iltikzfig{graphs/dpo/split-loop/rule-rhs}
    }
    \).
    The interpretation of this as a DPO rule in a valid traced boundary
    complement is illustrated below.
    \begin{center}
        \includestandalone{figures/graphs/dpo/split-loop/rewrite}
    \end{center}
    This corresponds to a valid term rewrite:
    \[
        \iltikzfig{graphs/dpo/split-loop/derivation-1}
        =
        \iltikzfig{graphs/dpo/split-loop/derivation-2}
        =
        \iltikzfig{graphs/dpo/split-loop/derivation-3}
        =
        \iltikzfig{graphs/dpo/split-loop/derivation-4}
    \]
\end{example}

Use of yanking is also what can lead to multiple boundary complements, and hence
a choice in rewrites.

\begin{example}
    Consider the rule \(
    \rrule{
        \iltikzfig{graphs/dpo/non-unique/rule-lhs}
    }{
        \iltikzfig{graphs/dpo/non-unique/rule-rhs}
    }
    \).
    Below are two valid traced boundary complements involving a matching of this
    rule.
    \begin{center}
        \scalebox{0.95}{\includestandalone{figures/graphs/dpo/non-unique/rewrite-1}}
        \quad
        \scalebox{0.95}{\includestandalone{figures/graphs/dpo/non-unique/rewrite-2}}
    \end{center}
    \vspace{-1em}
    These two derivations arise through yanking:
    \begin{gather*}
        \iltikzfig{graphs/dpo/non-unique/derivation-1}
        =
        \iltikzfig{graphs/dpo/non-unique/derivation-2}
        =
        \iltikzfig{graphs/dpo/non-unique/derivation-3a}
        =
        \iltikzfig{graphs/dpo/non-unique/derivation-4a}
        =
        \iltikzfig{graphs/dpo/non-unique/derivation-5a}
        \\
        \iltikzfig{graphs/dpo/non-unique/derivation-1}
        =
        \iltikzfig{graphs/dpo/non-unique/derivation-2}
        =
        \iltikzfig{graphs/dpo/non-unique/derivation-3b}
        =
        \iltikzfig{graphs/dpo/non-unique/derivation-4b}
        =
        \iltikzfig{graphs/dpo/non-unique/derivation-5b}
    \end{gather*}
\end{example}

Another condition on symmetric monoidal graph rewriting is that the matching
must be \emph{convex}: any path between vertices must also be captured.
Again thanks to yanking, this is not necessary in the traced case.

\begin{example}
    Consider the rule \(
    \rrule{
        \iltikzfig{graphs/dpo/convex/example-l}
    }{
        \iltikzfig{graphs/dpo/convex/example-r}
    }
    \) and the term \iltikzfig{graphs/dpo/convex/example-g}.
    Although it is not immediately obvious, there is in fact
    a matching of the former in the latter.
    Performing the DPO procedure yields the following:
    %
    \begin{gather*}
        \includestandalone{figures/graphs/dpo/convex/rewrite}
    \end{gather*}
    In a non-traced setting this is an invalid rule, but it is possible with
    yanking.
    \begin{gather*}
        \iltikzfig{graphs/dpo/convex/example-g}
        =
        \iltikzfig{graphs/dpo/convex/rewrite-2}
        =
        \iltikzfig{graphs/dpo/convex/rewrite-4}
        =\\[1em]
        \iltikzfig{graphs/dpo/convex/rewrite-5}
        =
        \iltikzfig{graphs/dpo/convex/example-h}
    \end{gather*}
\end{example}

For traced DPO to be sound, the rewritten graph must correspond to a traced
term.
First we prove a lemma to show how using the compact closed structure of
\(\smcsigma + \frob\) to reorganise interfaces corresponds to switching the
cospan maps in \(\cspdhyp\).

\begin{lemma}\label{lem:switch-interfaces}
    Let \(
    \iltikzfig{strings/category/f-2-2}[box=c,colour=white,dom1=m,dom2=n,cod1=p,cod2=q]
    \) be a term in \(\smcsigma + \frob\).
    Then if \(
    \termandfrobtohypsigmac[
        \iltikzfig{strings/category/f-2-2}[box=c,colour=white]
    ]
    =
    \cospan{m + n}[[f_1,f_2]]{F}[[g_1,g_2]]{p + q}
    \) then \(
    \termandfrobtohypsigmac[
        \iltikzfig{strings/rewriting/c-folded}
    ]
    =
    \cospan{p + m}[[g_1,f_1]]{F}[[f_2,g_2]]{n + q}
    \).
\end{lemma}
\begin{proof}
    By definition of cups and caps in \(\cspdhyp\).
\end{proof}

We need to show that rewriting a term with a rule \(\rrule{
    \iltikzfig{strings/category/f}[box=l,colour=white]
}{
    \iltikzfig{strings/category/f}[box=r,colour=white]
}\) coincides with graph rewriting on the hypergraph interpretations of
this rule.

\begin{notation}
    For a traced rule \(
    \rrule{
        \iltikzfig{strings/category/f}[box=l,colour=white,dom=i,cod=j]
    }{
        \iltikzfig{strings/category/f}[box=r,colour=white,dom=i,cod=j]
    } \in \stmcsigma
    \), its DPO rule is defined as \(
    \termandfrobtohypsigma[
        \tracedtosymandfrobsigma[
            \rrule{
                \iltikzfig{strings/category/f}[box=l,colour=white]
            }{
                \iltikzfig{strings/category/f}[box=r,colour=white]
            }
        ]
    ]
    \coloneqq
    \spann{
        \termandfrobtohypsigma[
            \foldinterfaces[
                \tracedtosymandfrobsigma[
                    \iltikzfig{strings/category/f}[box=l,colour=white]
                ]
            ]
        ]
    }{i+j}{
        \termandfrobtohypsigma[
            \foldinterfaces[
                \tracedtosymandfrobsigma[
                    \iltikzfig{strings/category/f}[box=r,colour=white]
                ]
            ]
        ]
    }
    .\)
\end{notation}

\begin{theorem}\label{thm:traced-rewrite-correspondence}
    For a rule \(r \in \stmcsigma\), we have that \(
    \iltikzfig{strings/category/f}[box=g,colour=white]
    \rewrite[r]
    \iltikzfig{strings/category/f}[box=h,colour=white]
    \) if and only if \(
    \termandfrobtohypsigma[
        \foldinterfaces[
            \tracedtosymandfrobsigma[
                \iltikzfig{strings/category/f}[box=g,colour=white]
            ]
        ]
    ]
    \grewrite[
        \termandfrobtohypsigma[
            \tracedtosymandfrobsigma[r]
        ]
    ]
    \termandfrobtohypsigma[
        \foldinterfaces[
            \tracedtosymandfrobsigma[
                \iltikzfig{strings/category/f}[box=h,colour=white]
            ]
        ]
    ].\)
\end{theorem}
\begin{proof}
    First the \((\Rightarrow)\) direction.
    If \(
    \iltikzfig{strings/category/f}[box=g,colour=white]
    \rewrite[\mcr]
    \iltikzfig{strings/category/f}[box=h,colour=white]
    \) then we have \(
    \iltikzfig{strings/category/f}[box=g,colour=white]
    =
    \iltikzfig{strings/rewriting/rewrite-l}
    \) and \(
    \iltikzfig{strings/rewriting/rewrite-r}
    =
    \iltikzfig{strings/category/f}[box=h,colour=white]
    \); we must derive the DPO diagram in \(\hypsigma\).
    First we give names to the following cospans:
    \begin{alignat*}{3}
        \cospan{0}{L}{i + j}
         & \coloneqq
        \termandfrobtohypsigma[
            \foldinterfaces[
                \tracedtosymandfrobsigma[
                    \iltikzfig{strings/category/f}[box=l,colour=white]
                ]
            ]
        ]
         &           & =
        \termandfrobtohypsigma[
            \iltikzfig{strings/rewriting/l-folded}
        ]
        \\
        \cospan{0}{R}{i + j}
         & \coloneqq
        \termandfrobtohypsigma[
            \foldinterfaces[
                \tracedtosymandfrobsigma[
                    \iltikzfig{strings/category/f}[box=r,colour=white]
                ]
            ]
        ]
         &           & =
        \termandfrobtohypsigma[\iltikzfig{strings/rewriting/r-folded}]
        \\
        \cospan{0}{G}{m + n}
         & \coloneqq
        \termandfrobtohypsigma[
            \foldinterfaces[
                \tracedtosymandfrobsigma[
                    \iltikzfig{strings/category/f}[box=g,colour=white]
                ]
            ]
        ]
         &           & =
        \termandfrobtohypsigma[\iltikzfig{strings/rewriting/lc-folded-shifted}]
        \\
        \cospan{0}{H}{m + n}
         & \coloneqq
        \termandfrobtohypsigma[
            \foldinterfaces[
                \tracedtosymandfrobsigma[
                    \iltikzfig{strings/category/f}[box=h,colour=white]
                ]
            ]
        ]
         &           & =
        \termandfrobtohypsigma[\iltikzfig{strings/rewriting/rc-folded-shifted}]
    \end{alignat*}

    Moving into \(\smcsigma + \frob\), we have that \(
    \iltikzfig{strings/rewriting/lc-folded-shifted}
    =
    \iltikzfig{strings/rewriting/lc-folded}
    \); so by functoriality \(
    \termandfrobtohypsigma[
        \foldinterfaces[
            \tracedtosymandfrobsigma[
                \iltikzfig{strings/category/f}[box=g,colour=white]
            ]
        ]
    ]
    =
    \termandfrobtohypsigma[\iltikzfig{strings/rewriting/l-folded}]
    \seq
    \termandfrobtohypsigma[\iltikzfig{strings/rewriting/c-folded}]
    \), i.e.\ \(
    \cospan{0}{G}{m + n} =
    \cospan{0}{L}{i + j}
    \seq
    \cospan{i + j}{C}{m + n}
    \).
    Cospan composition is by pushout, so \(\cospan{L}{G}{C}\) is a pushout.
    Using the same reasoning, \(\cospan{R}{G}{C}\) is also a pushout; this
    gives us the DPO diagram.
    All that remains is to check that the aforementioned pushouts are traced
    boundary complements; this follows by \cref{lem:switch-interfaces} as \(
    \termandfrobtohypsigma[
        \tracedandcomonoidtofrobsigma[
            \iltikzfig{strings/category/f-2-2}[box=c,colour=white]
        ]
    ]
    \) is partial monogamous.

    Now for the \(\ifdir\) direction: we assume we have a traced DPO
    rewrite, so there exist cospans \(
    \cospan{0}{L}{i + j},
    \cospan{0}{R}{i + j},
    \cospan{i + j}{C}{m + n}
    \) as above such that the DPO diagram commutes and
    \(i + j \to C \to G\) is a traced boundary complement.
    We must show that \(
    \iltikzfig{strings/category/f}[box=g,colour=white]
    =
    \iltikzfig{strings/rewriting/rewrite-l}
    \) and \(
    \iltikzfig{strings/category/f}[box=h,colour=white]
    =
    \iltikzfig{strings/rewriting/rewrite-r}
    \).

    We have that \(
    \cospan{0}{G}{m + n} =
    \cospan{0}{L}{i + j} \seq
    \cospan{i + j}[[c_1,c_2]]{C}[[d_1,d_2]]{m + n}
    \) as cospan composition is by pushout.
    Let \(
    \iltikzfig{strings/category/f-2-2}[box=c^\prime, colour=white,dom1=i,dom2=j,cod1=m,cod2=n]
    \) be the term in \(\smcsigma + \frob\) such that \(
    \termandfrobtohypsigma[
        \iltikzfig{strings/category/f-2-2}[box=c^\prime, colour=white]
    ]
    =
    \cospan{i + j}[[c_1,c_2]]{C}[[d_1,d_2]]{m + n}
    \), which exists as \(\termandfrobtohypsigma\) is full.


    The cospan \(\cospan{j + m}[[c_2,d_1]]{C}[[c_1,d_2]]{i + n}\)
    is partial monogamous because \(i + j \to C \to G\) is a traced
    boundary complement.
    Let \(
    \iltikzfig{strings/category/f-2-2}[box=c, colour=white,dom1=j,dom2=m,cod1=i,cod2=n]
    \)  be the term in \(\smcsigma + \frob\) such that \(
    \termandfrobtohypsigma[
        \iltikzfig{strings/category/f-2-2}[box=c, colour=white]
    ]
    =
    \cospan{j + m}[[c_2,d_1]]{C}[[c_1,d_2]]{i + n}
    \).
    Using \cref{lem:switch-interfaces}, we have that \(
    \termandfrobtohypsigma[
        \iltikzfig{strings/rewriting/c-folded}
    ]
    =
    \cospan{i + j}[[c_1,c_2]]{C}[[d_1,d_2]]{m + n}
    \).

    So we have that \(
    \termandfrobtohypsigma[
        \foldinterfaces[
            \iltikzfig{strings/category/f}[box=g,colour=white]
        ]
    ]
    =
    \termandfrobtohypsigma[
        \foldinterfaces[
            \iltikzfig{strings/category/f}[box=l,colour=white]
        ]
    ]
    \seq
    \termandfrobtohypsigma[
        \iltikzfig{strings/category/f-2-2}[box=c^\prime, colour=white]
    ]
    \); by fullness we derive that \(
    \iltikzfig{strings/rewriting/g-folded-box}
    =
    \iltikzfig{strings/rewriting/lc}
    =
    \iltikzfig{strings/rewriting/lc-folded}
    =
    \iltikzfig{strings/rewriting/lc-folded-shifted}
    \).
    This means that \(
    \foldinterfaces[
        \iltikzfig{strings/category/f}[box=g,colour=white]
    ]
    =
    \iltikzfig{strings/rewriting/lc-folded-shifted}
    \) so `unfolding' the interface gives us \(
    \iltikzfig{strings/category/f}[box=g,colour=white]
    =
    \iltikzfig{strings/rewriting/rewrite-l}
    \).
    Since \(
    \termandfrobtohypsigma[
        \iltikzfig{strings/category/f-2-2}[box=c, colour=white]
    ]
    \) is partial monogamous, \(
    \iltikzfig{strings/category/f-2-2}[box=c, colour=white]
    \) is in \(\stmcsigma\).
    As the trace in \(\stmcsigma\) is the canonical trace, the entire term is in
    \(\stmcsigma\), completing the proof.
    The same procedure holds for rewriting from the other direction.
\end{proof}

This gives us a sound and complete graph rewriting system for terms in
\(\stmcsigma\), and can be generalised to the coloured setting as well.

\begin{notation}
    For a rewrite rule \(
    \rrule{
        \iltikzfig{strings/category/f}[box=l,colour=white,dom=\listvar{i},cod=\listvar{j}]
    }{
        \iltikzfig{strings/category/f}[box=r,colour=white,dom=\listvar{i},cod=\listvar{j}]
    } \in \stmcsigmac
    \), its DPO rule is \(
    \termandfrobtohypsigmac[
        \tracedtosymandfrobsigmac[
            \rrule{
                \iltikzfig{strings/category/f}[box=l,colour=white]
            }{
                \iltikzfig{strings/category/f}[box=r,colour=white]
            }
        ]
    ]
    \), defined as \[
        \spann{
            \termandfrobtohypsigmac[
                \foldinterfacesc[
                    \tracedtosymandfrobsigmac[
                        \iltikzfig{strings/category/f}[box=l,colour=white]
                    ]
                ]
            ]
        }{\listvar{ij}}{
            \termandfrobtohypsigmac[
                \foldinterfacesc[
                    \tracedtosymandfrobsigmac[
                        \iltikzfig{strings/category/f}[box=r,colour=white]
                    ]
                ]
            ]
        }
        .\]
\end{notation}

\begin{theorem}\label{thm:traced-rewrite-correspondence-coloured}
    For a rule \(r \in \stmcsigma\), we have that \(
    \iltikzfig{strings/category/f}[box=g,colour=white]
    \rewrite[r]
    \iltikzfig{strings/category/f}[box=h,colour=white]
    \) if and only if \(
    \termandfrobtohypsigmac[
        \foldinterfacesc[
            \tracedtosymandfrobsigmac[
                \iltikzfig{strings/category/f}[box=g,colour=white]
            ]
        ]
    ]
    \grewrite[
        \termandfrobtohypsigmac[
            \tracedtosymandfrobsigmac[r]
        ]
    ]
    \termandfrobtohypsigmac[
        \foldinterfacesc[
            \tracedtosymandfrobsigmac[
                \iltikzfig{strings/category/f}[box=g,colour=white]
            ]
        ]
    ].\)
\end{theorem}