\section{Rewriting with commutative comonoid structure}

Hypergraphs were originally chosen as the combinatorial domain for rewriting
terms in \(\stmcsigmac + \frobc\) because they allow rewriting modulo
\emph{Frobenius structure}.
In the traced rewriting system described in the previous section we did not
fully exploit this, only using the combinatorial structure to absorb instances
of yanking.
We will now show how we can take more advantage of the hypergraph language by
rewriting  modulo \emph{commutative comonoid} structure.

It is not difficult to see how one might extend the results of the previous
section for this aim; the partial monogamy conditions need to be weakened to
partial \emph{left-}monogamy conditions.

\begin{definition}[Traced left-boundary complement]
    \label{def:traced-left-boundary-complement}
    For partial left-monogamous cospans \(
    \cospan{i}[a_1]{L}[a_2]{j}
    \) and \(
    \cospan{n}[b_1]{G}[b_2]{m} \in \hypsigma
    \), a pushout complement as in \cref{def:traced-boundary-complement}
    is called a \emph{traced left-boundary complement} if \(c_2\)
    is mono and \(
    \cospan{\listvar{jm}}[[c_2,d_1]]{C}[[c_1,d_2]]{{\listvar{in}}}
    \) is a partial left-monogamous cospan.
\end{definition}

\begin{definition}[Traced comonoid DPO]
    For morphisms \(G \leftarrow m+n\) and \(H \leftarrow m+n\) in
    \(\hypsigma\), there is a traced comonoid rewrite \(G \trgrewrite[\mcr] H\)
    if there exists a rule \(
    \spann{L}{\listvar{ij}}{G} \in \mcr
    \) and cospan \(
    \cospan{\listvar{ij}}{C}{\listvar{mn}} \in \hypsigma
    \) such that diagram in \cref{def:dpo-rewriting} commutes and
    \(\listvar{ij} \to C \to G\) is a
    traced left-boundary complement.
\end{definition}

Just like traced rewriting, there may be multiple traced left-boundary
complements.
In fact, since partial left-monogamy is a weaker restriction than partial
monogamy, there are even more potential rewrites.
These arise because we are now absorbing the equations of a commutative
comonoid.

\begin{example}
    Consider the following rule and its interpretation.
    \begin{gather*}
        \rrule{
            \iltikzfig{graphs/dpo/non-unique-comonoid/rule-lhs}
        }{
            \iltikzfig{graphs/dpo/non-unique-comonoid/rule-rhs}
        }
        \qquad
        \raisebox{-2em}{\includestandalone{figures/graphs/dpo/non-unique-comonoid/rule}}
    \end{gather*}
    Two valid rewrites are as follows:
    \begin{center}
        \includestandalone{figures/graphs/dpo/non-unique-comonoid/rewrite-1}
        \quad
        \includestandalone{figures/graphs/dpo/non-unique-comonoid/rewrite-2}
    \end{center}
    The first rewrite is the `obvious' one, but the second also holds by
    cocommutativity:
    \begin{gather*}
        \iltikzfig{graphs/dpo/non-unique-comonoid/rewrite-1}
        =
        \iltikzfig{graphs/dpo/non-unique-comonoid/rewrite-2a}
        \qquad
        \iltikzfig{graphs/dpo/non-unique-comonoid/rewrite-1}
        =
        \iltikzfig{graphs/dpo/non-unique-comonoid/rewrite-2b}
        =
        \iltikzfig{graphs/dpo/non-unique-comonoid/rewrite-3b}
    \end{gather*}
\end{example}

To show the soundness and completeness of traced comonoid rewriting, we follow
the same procedure as in the traced setting.

\begin{theorem}
    For a rule \(\rrule{
        \iltikzfig{strings/category/f}[box=l,colour=white]
    }{
        \iltikzfig{strings/category/f}[box=r,colour=white]
    }\) in \(
    \stmcsigmac + \ccomonc
    \), \(
    \iltikzfig{strings/category/f}[box=g,colour=white]
    \rewrite[\rrule{
            \iltikzfig{strings/category/f}[box=l,colour=white]
        }{
            \iltikzfig{strings/category/f}[box=r,colour=white]
        }]
    \iltikzfig{strings/category/f}[box=h,colour=white]
    \) if and only if \(
    \termandfrobtohypsigmac[
        \foldinterfaces[
            \tracedandcomonoidtofrobsigmac[
                \iltikzfig{strings/category/f}[box=g,colour=white]
            ]
        ]
    ]
    \grewrite[
        \termandfrobtohypsigmac[
            \rrule{
                \tracedandcomonoidtofrobsigmac[
                    \iltikzfig{strings/category/f}[box=l,colour=white]
                ]
            }{
                \tracedandcomonoidtofrobsigmac[
                    \iltikzfig{strings/category/f}[box=r,colour=white]
                ]
            }
        ]
    ]
    \termandfrobtohypsigmac[
        \foldinterfaces[
            \iltikzfig{strings/category/f}[box=g,colour=white]
        ]
    ]\).
\end{theorem}
\begin{proof}
    This is as \cref{thm:traced-rewrite-correspondence}, but replacing partial
    monogamy with partial left-monogamy, and traced boundary complements with
    traced left-boundary complements.
\end{proof}

This means that we can also safely perform rewriting modulo traced comonoid
structure by building on the machinery used for the traced case.