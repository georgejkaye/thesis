\section{Double pushout rewriting}

As a backdrop, let us first consider how equations in a monoidal theory are
applied to terms without any sort of special graph interpretation.

\begin{definition}[Term rewriting]\label{def:term-rewriting}
    A \emph{rewriting system} \(\mcr\) for a traced PROP \(\stmcsigmac\)
    consists of a set of \emph{rewrite rules} \(
        \rrule{
            \iltikzfig{strings/category/f}[box=l,colour=white,dom=i,cod=j]
        }{
            \iltikzfig{strings/category/f}[box=r,colour=white,dom=i,cod=j]
        }
    \).
    Given terms \(
        \iltikzfig{strings/category/f}[box=g,colour=white,dom=m,cod=n]
    \) and \(
        \iltikzfig{strings/category/f}[box=h,colour=white,dom=m,cod=n]
    \) in \(\stmc{\generators}\) we write \(
        \iltikzfig{strings/category/f}[box=g,colour=white]
        \rewrite[\mcr]
        \iltikzfig{strings/category/f}[box=h,colour=white]
    \) if there exists rewrite rule \((
        \iltikzfig{strings/category/f}[box=l,colour=white,dom=i,cod=j],
        \iltikzfig{strings/category/f}[box=r,colour=white,dom=i,cod=j]
    )\) in \(\mcr\) and \(
        \iltikzfig{strings/category/f-2-2}[box=c,colour=white,dom1=j,dom2=m,cod1=i,cod2=n]
    \) in \(\stmc{\Sigma}\) such that \(
        \iltikzfig{strings/category/f}[box=g,colour=white]
        =
        \iltikzfig{strings/rewriting/rewrite-l}
    \) and \(
        \iltikzfig{strings/category/f}[box=h,colour=white]
        =
        \iltikzfig{strings/rewriting/rewrite-r}
    \) by axioms of STMCs.
\end{definition}