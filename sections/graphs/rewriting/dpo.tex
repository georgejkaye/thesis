\section{Double pushout rewriting}

As a backdrop, let us first consider how equations in a monoidal theory are
applied to terms without any sort of special graph interpretation.

\begin{definition}[Rewriting system]\label{def:term-rewriting}
    A \emph{rewriting system} \(\mcr\) for a traced PROP \(\mathbb{A}\)
    consists of a set of \emph{rewrite rules} \(
        \rrule{
            \iltikzfig{strings/category/f}[box=l,colour=white,dom=\listvar{i},cod=\listvar{j}]
        }{
            \iltikzfig{strings/category/f}[box=r,colour=white,dom=\listvar{i},cod=\listvar{j}]
        }
    \).
    Given terms \(
        \iltikzfig{strings/category/f}[box=g,colour=white,dom=\listvar{m},cod=\listvar{n}]
    \) and \(
        \iltikzfig{strings/category/f}[box=h,colour=white,dom=\listvar{m},cod=\listvar{n}]
    \) in \(\stmc{\generators}\) we write \(
        \iltikzfig{strings/category/f}[box=g,colour=white]
        \rewrite[\mcr]
        \iltikzfig{strings/category/f}[box=h,colour=white]
    \) if there exists rewrite rule \(\rrule{
        \iltikzfig{strings/category/f}[box=l,colour=white,dom=\listvar{i},cod=\listvar{j}]
    }{
        \iltikzfig{strings/category/f}[box=r,colour=white,dom=\listvar{i},cod=\listvar{j}]
    }\) in \(\mcr\) and \(
        \iltikzfig{strings/category/f-2-2}[box=c,colour=white,dom1=\listvar{j},dom2=\listvar{m},cod1=\listvar{i},cod2=\listvar{n}]
    \) in \(\stmc{\Sigma}\) such that \(
        \iltikzfig{strings/category/f}[box=g,colour=white]
        =
        \iltikzfig{strings/rewriting/rewrite-l}
    \) and \(
        \iltikzfig{strings/category/f}[box=h,colour=white]
        =
        \iltikzfig{strings/rewriting/rewrite-r}
    \) by axioms of STMCs.
\end{definition}

The difference between a rewriting system and an equational theory is that in
the former the rules are \emph{directed} whereas equations are not.
Of course, it is straightforward to derive a rewriting system from an equational
theory by adding the reductions for `both ways' of each equation.

\begin{definition}[\cite{bonchi2022stringa}, Sec. 2.4]
    Given a coloured monoidal theory \((C,\generators,\equations)\), let
    \(\mcr_{\equations}\) be the rewriting theory defined as \(\{
        \rrule{
            \iltikzfig{strings/category/f}[box=l,colour=white,dom=\listvar{i},cod=\listvar{j}]
        }{
            \iltikzfig{strings/category/f}[box=r,colour=white,dom=\listvar{i},cod=\listvar{j}]
        }
        \,|\,
        \iltikzfig{strings/category/f}[box=l,colour=white,dom=\listvar{i},cod=\listvar{j}]
        =
        \iltikzfig{strings/category/f}[box=r,colour=white,dom=\listvar{i},cod=\listvar{j}]
        \in
        \mce
    \} \cup \{
        \rrule{
            \iltikzfig{strings/category/f}[box=r,colour=white,dom=\listvar{i},cod=\listvar{j}]
        }{
            \iltikzfig{strings/category/f}[box=l,colour=white,dom=\listvar{i},cod=\listvar{j}]
        }
        \,|\,
        \iltikzfig{strings/category/f}[box=l,colour=white,dom=\listvar{i},cod=\listvar{j}]
        =
        \iltikzfig{strings/category/f}[box=r,colour=white,dom=\listvar{i},cod=\listvar{j}]
        \in
        \mce
    \}\).
\end{definition}


\begin{proposition}[\cite{bonchi2022stringa}, Prop. 2.18]
    Given two terms \(
        \iltikzfig{strings/category/f}[box=g,colour=white],
        \iltikzfig{strings/category/f}[box=h,colour=white]
        \in \stmc{C,\generators,\equations}
    \), \(
        \iltikzfig{strings/category/f}[box=g,colour=white]
        =
        \iltikzfig{strings/category/f}[box=h,colour=white]
    \) if and only if \(
        \iltikzfig{strings/category/f}[box=g,colour=white]
        \rewrites[\mcr_{\equations}]
        \iltikzfig{strings/category/f}[box=h,colour=white]
    \).
\end{proposition}