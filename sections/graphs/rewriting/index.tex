\chapter{Graph rewriting}

The results of the previous chapter have given us a combinatorial setting in
which string diagrams equal by topological deformations or by the equations of
commutative comonoids are translated into isomorphic cospans of hypergraphs.
This already gives us a straightforward way to interpret these structures
computationally.

However, it is very rare that we will only be working modulo the equations of
traced comonoids; we will often also have the equations of a
\emph{monoidal theory} to worry about.
Since an equation \(
    \iltikzfig{strings/category/f}[box=f,colour=white]
    =
    \iltikzfig{strings/category/f}[box=g,colour=white]
\) may actually change the components of the term, there is no reason for the
interpretations \(
    \termandfrobtohypsigmac[
        \tracedandcomonoidtofrobsigmac[
            \iltikzfig{strings/category/f}[box=f,colour=white]
        ]
    ]
\) and \(
    \termandfrobtohypsigmac[
        \tracedandcomonoidtofrobsigmac[
            \iltikzfig{strings/category/f}[box=g,colour=white]
        ]
    ]
\) to be isomorphic.
Instead, to reason about cospans of hypergraphs modulo the equations of some
monoidal theory we must \emph{rewrite} the components of the graph.

Performing rewriting in this way is common in computer science and accordingly
there is a wealth of knowledge on both
term~\cite{knuth1970simple,gadduchi1996algebraic} and graph~\cite{}

Rewriting for string diagrams has its roots in Burroni's work on
polygraphs~\cite{burroni}

This approach even briefly entered our world of digital circuits in Lafont's
work on Boolean circuits~\cite{lafont2003algebraic}, which used a rudimentary
form of graphical syntactic rewriting.
However, in this approach even the axioms of SMCs needed to be applied
explicitly, greatly hampering the tractability of the system.

It is only more recently that \emph{graph rewriting} for string diagrams has
been studied, first with the aforementioned \emph{string graphs} of Dixon and
Kissinger~\cite{dixon2010open,kissinger2012pictures,dixon2013opengraphs}.
This only considered rewriting in the presence of a trace rather than
considering any additional structure, but nevertheless was successfully
implemented in a proof assistant called
Quantomatic~\cite{kissinger2015quantomatic}.
Instead, our work continues to closely follow that of Bonchi et
al~\cite{bonchi2022string,bonchi2022stringa}, which has several advantages
over string graphs.
Unlike the category of string graphs, the category of hypergraphs is
\emph{adhesive}~\cite{lack2004adhesive} which affords it some nice rewriting
properties.
Moreover, rewriting modulo Frobenius structure (or even just monoid or comonoid
structure) can reveal rewrites `hidden' behind the absorbed equations that might
not be seen with the naked eye.

\begin{remark}
    The content of this section is a revised and expanded version
    of~\cite[Sec. 5]{ghica2023rewriting}.
\end{remark}

\section{Double pushout rewriting}

As a backdrop, let us first consider how equations in a monoidal theory are
applied to terms without any sort of special graph interpretation.

\begin{definition}[Term rewriting]\label{def:term-rewriting}
    A \emph{rewriting system} \(\mcr\) for a traced PROP \(\stmcsigmac\)
    consists of a set of \emph{rewrite rules} \(
        \rrule{
            \iltikzfig{strings/category/f}[box=l,colour=white,dom=i,cod=j]
        }{
            \iltikzfig{strings/category/f}[box=r,colour=white,dom=i,cod=j]
        }
    \).
    Given terms \(
        \iltikzfig{strings/category/f}[box=g,colour=white,dom=m,cod=n]
    \) and \(
        \iltikzfig{strings/category/f}[box=h,colour=white,dom=m,cod=n]
    \) in \(\stmc{\generators}\) we write \(
        \iltikzfig{strings/category/f}[box=g,colour=white]
        \rewrite[\mcr]
        \iltikzfig{strings/category/f}[box=h,colour=white]
    \) if there exists rewrite rule \((
        \iltikzfig{strings/category/f}[box=l,colour=white,dom=i,cod=j],
        \iltikzfig{strings/category/f}[box=r,colour=white,dom=i,cod=j]
    )\) in \(\mcr\) and \(
        \iltikzfig{strings/category/f-2-2}[box=c,colour=white,dom1=j,dom2=m,cod1=i,cod2=n]
    \) in \(\stmc{\Sigma}\) such that \(
        \iltikzfig{strings/category/f}[box=g,colour=white]
        =
        \iltikzfig{strings/rewriting/rewrite-l}
    \) and \(
        \iltikzfig{strings/category/f}[box=h,colour=white]
        =
        \iltikzfig{strings/rewriting/rewrite-r}
    \) by axioms of STMCs.
\end{definition}
\section{Rewriting with traced structure}

In the Frobenius setting, every pushout complement is a valid rewrite, but there
is no reason for the same to be the case for traced or traced comonoid
rewriting.
Bonchi et al showed in~\cite{bonchi2022stringa} that \emph{exactly one} pushout
complement corresponds to a valid rewrite in the symmetric monoidal case by
characterising it as a \emph{boundary complement}.

\begin{definition}[Boundary complement (\cite{bonchi2022stringa}, Def. 30)]
    For monogamous cospans \(
    \cospan{\listvar{i}}[a_1]{L}[a_2]{\listvar{j}}
    \) and \(
    \cospan{\listvar{m}}[b_1]{G}[b_2]{\listvar{n}}
    \) and a monomorphism \(\morph{f}{L}{G}\), a pushout complement as below
    \begin{center}
        \begin{tikzcd}[column sep=large]
            L \arrow[swap]{d}{f}
            &
            \listvar{ij}
            \arrow[swap]{l}{a := [a_1, a_2]}
            \arrow{d}{c := [c_1, c_2]}
            \\
            G
            \arrow["\urcorner"{anchor=center, pos=0.125}, draw=none]{ur}
            &
            C
            \arrow{l}{g}
            \\
            &
            \listvar{mn}
            \arrow{ul}{[b_1,b_2]}
            \arrow[swap]{u}{d := [d_1,d_2]}
        \end{tikzcd}
    \end{center}
    is called a \emph{boundary complement} if \(c_1\) and \(c_2\) are
    mono and \(
    \cospan{\listvar{jm}}[[c_2,d_1]]{C}[[d_2,c_1]]{\listvar{ni}}
    \) is a boundary monogamous cospan.
\end{definition}

What is particularly special about boundary complements is that for morphisms
\(\listvar{ij} \to L \to G\) there is always \emph{exactly one} pushout
complement which is also a boundary complement!

\begin{proposition}[\cite{bonchi2022stringa}, Prop. 31]
    When boundary complements exist in \(\hypsigmac\), they are unique.
\end{proposition}

For rewriting in a traced setting, boundary complements are too strong, so we
will weaken them to \emph{traced} boundary complements, replacing references to
monogamy with partial monogamy.

\begin{definition}[Traced boundary complement]
    \label{def:traced-boundary-complement}
    For partial monogamous cospans \(
    \cospan{\listvar{i}}[a_1]{L}[a_2]{\listvar{j}}
    \) and \(
    \cospan{\listvar{m}}[b_1]{G}[b_2]{\listvar{n}}
    \), a pushout complement as below
    \begin{center}
        \begin{tikzcd}[column sep=large]
            L \arrow[swap]{d}{f}
            &
            \listvar{ij}
            \arrow[swap]{l}{a := [a_1, a_2]}
            \arrow{d}{c := [c_1, c_2]}
            \\
            G
            \arrow["\urcorner"{anchor=center, pos=0.125}, draw=none]{ur}
            &
            C
            \arrow{l}{g}
            \\
            &
            \listvar{mn}
            \arrow{ul}{[b_1,b_2]}
            \arrow[swap]{u}{d := [d_1,d_2]}
        \end{tikzcd}
    \end{center}
    is called a \emph{traced boundary complement} if \(c_1\) and \(c_2\) are
    mono and \(
    \cospan{\listvar{jm}}[[c_2,d_1]]{C}[[d_2,c_1]]{\listvar{ni}}
    \) is a partial monogamous cospan.
\end{definition}

By restricting to traced boundary complements, DPO rewriting can be formulated
for terms in a traced setting.

\begin{definition}[Traced DPO]
    For morphisms \(G \leftarrow \listvar{mn}\) and \(H \leftarrow \listvar{mn}\) in
    \(\hypsigma\), there is a traced rewrite \(G \trgrewrite[\mcr] H\) if there
    exists a rule \(
    \spann{L}{\listvar{ij}}{G} \in \mcr
    \) and cospan \(
    \cospan{\listvar{ij}}{C}{\listvar{mn}} \in \hypsigma
    \) such that diagram in \cref{def:dpo-rewriting} commutes and \(\listvar{ij} \to C\)
    is a traced boundary complement.
\end{definition}

Traced boundary complements are not much more powerful than regular boundary
complements; \(c_1\) is still not permitted to merge vertices in the inputs and
the same for \(c_2\) and the outputs.
The real power of traced DPO, and what increases the number of pushout
complements, is the fact that the matching \(L \to G\) is no longer required to
be mono.

Since \(\cospan{\listvar{m}}{G}{\listvar{n}}\) is partial monogamous, vertices
in \(L\) can only be merged if their degrees sum to no more than \((1,1)\).
Merging vertices in this way corresponds to using the trace to find a match.

\begin{example}
    Consider the rule \(
    \rrule{
        \iltikzfig{graphs/dpo/traced-example/rule-lhs}
    }{
        \iltikzfig{graphs/dpo/traced-example/rule-rhs}
    }
    \) and the term \(
    \iltikzfig{graphs/dpo/traced-example/term}
    \), in which there is clearly an instance of the rule.
    The interpretation of this as a DPO derivation with a valid traced boundary
    complement is illustrated below.
    \begin{center}
        \includestandalone{figures/graphs/dpo/traced-example/rewrite}
    \end{center}
\end{example}

A key feature of rewriting modulo traced structure is the \emph{yanking} axiom,
which can lead to some non-obvious rewrites.

\begin{example}
    Consider the rule \(
    \rrule{
        \iltikzfig{graphs/dpo/split-loop/rule-lhs}
    }{
        \iltikzfig{graphs/dpo/split-loop/rule-rhs}
    }
    \).
    The interpretation of this as a DPO rule in a valid traced boundary
    complement is illustrated below.
    \begin{center}
        \includestandalone{figures/graphs/dpo/split-loop/rewrite}
    \end{center}
    This corresponds to a valid term rewrite:
    \[
        \iltikzfig{graphs/dpo/split-loop/derivation-1}
        =
        \iltikzfig{graphs/dpo/split-loop/derivation-2}
        =
        \iltikzfig{graphs/dpo/split-loop/derivation-3}
        =
        \iltikzfig{graphs/dpo/split-loop/derivation-4}
    \]

    Note that applying yanking is required in the term setting because
    the traced wire is flowing from right to left, whereas applying the rule
    requires all wires flowing left to right.
\end{example}

Use of yanking is also what can lead to multiple boundary complements, and hence
a choice in rewrites.

\begin{example}
    Consider the rule \(
    \rrule{
        \iltikzfig{graphs/dpo/non-unique/rule-lhs}
    }{
        \iltikzfig{graphs/dpo/non-unique/rule-rhs}
    }
    \).
    Below are two valid traced boundary complements involving a matching of this
    rule.

    \begin{center}
        \scalebox{0.95}{\includestandalone{figures/graphs/dpo/non-unique/rewrite-1}}
        \quad
        \scalebox{0.95}{\includestandalone{figures/graphs/dpo/non-unique/rewrite-2}}
    \end{center}

    These two derivations arise through yanking:
    \begin{gather*}
        \iltikzfig{graphs/dpo/non-unique/derivation-1}
        =
        \iltikzfig{graphs/dpo/non-unique/derivation-2}
        =
        \iltikzfig{graphs/dpo/non-unique/derivation-3a}
        =
        \iltikzfig{graphs/dpo/non-unique/derivation-4a}
        =
        \iltikzfig{graphs/dpo/non-unique/derivation-5a}
        \\
        \iltikzfig{graphs/dpo/non-unique/derivation-1}
        =
        \iltikzfig{graphs/dpo/non-unique/derivation-2}
        =
        \iltikzfig{graphs/dpo/non-unique/derivation-3b}
        =
        \iltikzfig{graphs/dpo/non-unique/derivation-4b}
        =
        \iltikzfig{graphs/dpo/non-unique/derivation-5b}
    \end{gather*}
\end{example}

There is another condition on graph rewriting modulo symmetric monoidal
structure in that the matching must be \emph{convex}: any path between vertices
must also be captured.
Luckily for us, this is not necessary in the traced case.

\begin{example}
    Consider the rule \(
    \rrule{
        \iltikzfig{graphs/dpo/convex/example-l}
    }{
        \iltikzfig{graphs/dpo/convex/example-r}
    }
    \) and the term \iltikzfig{graphs/dpo/convex/example-g}.
    Although it is not immediately obvious, there is in fact
    a matching of the former in the latter.
    Performing the DPO procedure yields the following:
    %
    \begin{gather*}
        \includestandalone{figures/graphs/dpo/convex/rewrite}
    \end{gather*}
    In a non-traced setting this is an invalid rule!
    However, it is possible with yanking.
    \begin{gather*}
        \iltikzfig{graphs/dpo/convex/example-g}
        =
        \iltikzfig{graphs/dpo/convex/rewrite-2}
        =
        \iltikzfig{graphs/dpo/convex/rewrite-4}
        =
        \iltikzfig{graphs/dpo/convex/rewrite-5}
        =
        \iltikzfig{graphs/dpo/convex/example-h}
    \end{gather*}
    This shows that convexity is not a required component of traced rewriting.
\end{example}

\todo[inline]{Refine after here}

We are almost ready to show the soundness and completeness of this DPO rewriting
system.
The final prerequisite is a decomposition lemma akin to that used
in~\cite{bonchi2022string}.

\begin{lemma}[Traced decomposition]\label{lem:traced-decomposition}
    Given partial monogamous cospans \(
    \cospan{m}[d_1]{G}[d_2]{n}
    \) and \(
    \cospan{i}[a_1]{L}[a_2]{j}
    \), and a morphism \(
    L \xrightarrow{f} G
    \) such that \(\listvar{ij} \rightarrow L \rightarrow G\) satisfies the no-dangling
    and no-identification conditions, then there exists a partial monogamous
    cospan \(
    \cospan{j+m}[[c_2,d_1]]{C}[[c_1,d_2]]{i+n}
    \) such that \(
    \cospan{m}{G}{n}
    \) can be factored as
    \begin{gather*}
        \trace{i}{
            \begin{array}{cc}
                \cospan{i}[a_1]{L}[a_2]{j} \\
                \tensor                    \\
                \cospan{m}{m}{m}
            \end{array}
            \seq
            \cospan{j+m}[[c_2,d_1]]{C}[[c_1,d_2]]{i+n}
        }
    \end{gather*}
    where \(
    \cospan{j+m}[c_2,d_1]{C}[c_1,d_2]{i+n}
    \) is a traced boundary complement.
\end{lemma}
\begin{proof}
    Let \(
    i + j \xrightarrow{[c_1, c_2]} C \xleftarrow{[d_1, d_2]} \listvar{mn}
    \) be defined as a traced boundary complement of \(
    \listvar{ij} \xrightarrow{[a_1,a_2]} L \xrightarrow{f} G
    \), which exists as the no-dangling and no-identification condition is
    satisfied.
    We assign names to the various cospans in play, and reason string
    diagrammatically:
    \begin{align*}
        \iltikzfig{strings/category/f}[box=l,colour=white,dom=i,cod=j] & := \cospan{i}{L}{j}
                                                                       &
        \iltikzfig{strings/category/f-0-2}[box=\hat{l},colour=white,cod1=i,cod2=j]
                                                                       & :=
        \cospan{0}{L}{\listvar{ij}}
        \\
        \iltikzfig{strings/category/f-2-2}[box=c,colour=white,dom1=j,dom2=m,cod1=i,cod2=n]
                                                                       & :=
        \cospan{j+m}[[c_2, d_1]]{C}[[c_1, d_2]]{i+n}
                                                                       &
        \iltikzfig{strings/category/f-2-2}[box=\hat{c},colour=white,dom1=i,dom2=j,cod1=m,cod2=n]
                                                                       & :=
        \cospan{\listvar{ij}}[[c_1, c_2]]{C}[[d_1, d_2]]{\listvar{mn}}
        \\
        \iltikzfig{strings/category/f}[box=g,colour=white,dom=m,cod=n]
                                                                       & :=
        \cospan{m}{G}{n}
                                                                       &
        \iltikzfig{strings/category/f-0-2}[box=\hat{g},colour=white,cod1=m,cod2=n]
                                                                       & :=
        \cospan{0}{G}{\listvar{mn}}
    \end{align*}
    Note that the cospans in the left column are partial monogamous by definition
    of rewrite rules and traced boundary complements.
    We will show that  \(
    \iltikzfig{strings/category/f}[box=g,colour=white]
    \) can be decomposed into a form using the two cospans \(
    \iltikzfig{strings/category/f}[box=l,colour=white]
    \) and \(
    \iltikzfig{strings/category/f-2-2}[box=c,colour=white]
    \), along with identities.

    By using the compact closed structure of \(\cspdhyp\), we have the following:
    \begin{gather*}
        \iltikzfig{strings/category/f}[box=g,colour=white,dom=m,cod=n]
        =
        \iltikzfig{graphs/dpo/g-bent}
        \qquad
        \iltikzfig{strings/category/f-2-2}[box=\hat{c},colour=white,dom1=i,dom2=j,cod1=m,cod2=n]
        =
        \iltikzfig{graphs/dpo/cprime-as-c}
        \qquad
        \iltikzfig{strings/category/f-0-2}[box=\hat{l},colour=white,cod1=i,cod2=j]
        =
        \iltikzfig{strings/compact-closed/f-bent-input}[box=l,colour=white,cod=i,dom=j]
    \end{gather*}
    Since \(G\) is the pushout of \(
    L \xleftarrow{[a_1, a_2]} \listvar{ij} \xrightarrow{[c_1, c_2]} C
    \) and pushout is cospan composition, we also have that \(
    \iltikzfig{strings/category/f-0-2}[box=\hat{g},colour=white,cod1=m,cod2=n]
    =
    \iltikzfig{graphs/dpo/lctilde}
    \).
    Putting this all together we can show that
    \begin{gather*}
        \iltikzfig{strings/category/f}[box=g,colour=white,dom=m,cod=n]
        =
        \iltikzfig{graphs/dpo/g-bent}
        =
        \iltikzfig{graphs/dpo/l-c-bent}
        =
        \iltikzfig{graphs/dpo/l-c-bent-1}
        =
        \iltikzfig{graphs/dpo/lc-bent-2}
        =
        \iltikzfig{strings/rewriting/rewrite-l}[dom=m,cod=n]
    \end{gather*}
    Since the `loop' is constructed in the same manner as the canonical trace on
    \(\cspdhyp\) (and is therefore identical in the graphical notation), this is a
    term in the form of (\ref{gath:decomposition}).
\end{proof}

We need to show that, term rewriting with a set of rules \(\mcr\)
coincides with graph rewriting on the hypergraph interpretations of these rules.
Recall that DPO rules have only one interface so the rules need to be `bent'
using \(\foldinterfaces\);
we write \(
\foldinterfaces[\tracedtosymandfrob[\mcr]{\Sigma}]
\) for the pointwise map \(
(
\iltikzfig{strings/category/f}[box=l,colour=white],
\iltikzfig{strings/category/f}[box=r,colour=white]
)
\mapsto
(
\foldinterfaces[
    \tracedtosymandfrob[
    \iltikzfig{strings/category/f}[box=l,colour=white]
]{\Sigma}
],
\foldinterfaces[
    \tracedtosymandfrob[
    \iltikzfig{strings/category/f}[box=r,colour=white]
]{\Sigma}
]
).
\)

\begin{theorem}\label{thm:traced-rewriting}
    Let \(\mcr\) be a rewriting system on \(\stmcsigma\).
    Then,
    \begin{gather*}
        \iltikzfig{strings/category/f}[box=g,colour=white,dom=m,cod=n]
        \rewrite[\mcr]
        \iltikzfig{strings/category/f}[box=h,colour=white,dom=m,cod=n]
        \quad
        \text{if and only if}
        \quad
        \termandfrobtohypsigma[
            \foldinterfaces[
                \tracedtosymandfrob[
                    \iltikzfig{strings/category/f}[box=g,colour=white]
                ]{\Sigma}
            ]
        ]
        \grewrite[
            \termandfrobtohypsigma[
                \foldinterfaces[
                    \tracedtosymandfrob[\mcr]{\Sigma}
                ]
            ]
        ]
        \termandfrobtohypsigma[
            \foldinterfaces[
                \tracedtosymandfrob[
                    \iltikzfig{strings/category/f}[box=h,colour=white]
                ]{\Sigma}
            ]
        ].
    \end{gather*}
\end{theorem}
\begin{proof}
    First the \((\Rightarrow)\) direction.
    If \(
    \iltikzfig{strings/category/f}[box=g,colour=white]
    \rewrite[\mcr]
    \iltikzfig{strings/category/f}[box=h,colour=white]
    \) then we have \(
    \iltikzfig{strings/category/f}[box=g,colour=white]
    =
    \iltikzfig{strings/rewriting/rewrite-l}
    \) and \(
    \iltikzfig{strings/rewriting/rewrite-r}
    =
    \iltikzfig{strings/category/f}[box=h,colour=white].
    \)
    Define the following cospans:
    \begin{alignat}{3}
        \label{gath:l-cospan}
        \cospan{0}{L}{\listvar{ij}}
         & :=
        \termandfrobtohypsigma[\foldinterfaces[\iltikzfig{strings/category/f}[box=l,colour=white]]]
         &    & =
        \termandfrobtohypsigma[\iltikzfig{strings/rewriting/l-folded}]
        \\
        \cospan{0}{R}{\listvar{ij}}
         & :=
        \termandfrobtohypsigma[\foldinterfaces[\iltikzfig{strings/category/f}[box=r,colour=white]]]
         &    & =
        \termandfrobtohypsigma[\iltikzfig{strings/rewriting/r-folded}]
        \\
        \cospan{0}{G}{\listvar{mn}}
         & :=
        \termandfrobtohypsigma[\foldinterfaces[\iltikzfig{strings/category/f}[box=f,colour=white]]]
         &    & =
        \termandfrobtohypsigma[\iltikzfig{strings/rewriting/lc-folded}]
        \\
        \label{gath:h-cospan}
        \cospan{0}{H}{\listvar{mn}}
         & :=
        \termandfrobtohypsigma[\foldinterfaces[\iltikzfig{strings/category/f}[box=h,colour=white]]]
         &    & =
        \termandfrobtohypsigma[\iltikzfig{strings/rewriting/rc-folded}]
        \\
        \cospan{\listvar{ij}}{C}{\listvar{mn}}
         & :=
        \termandfrobtohypsigma[\iltikzfig{strings/rewriting/c-folded}]
         &    &
    \end{alignat}
    By functoriality, since \(
    \foldinterfaces[\iltikzfig{strings/category/f}[box=f,colour=white]]
    =
    \iltikzfig{strings/rewriting/l-folded}
    \seq
    \iltikzfig{strings/rewriting/c-folded}
    \) then \[
        \cospan{0}{G}{\listvar{mn}} = \cospan{0}{L}{\listvar{ij}} \seq \cospan{\listvar{ij}}{C}{\listvar{mn}}.
    \]
    Cospan composition is pushout, so \(\cospan{L}{G}{C}\) is a pushout.
    Using the same reasoning, \(\cospan{R}{G}{C}\) is also a pushout: this
    gives us the DPO diagram.
    All that remains is to check that the aforementioned pushouts are traced
    boundary complements: this follows by inspecting components.

    Now the \(\ifdir\) direction: assume we have a DPO diagram where
    \(L \leftarrow i + j\), \(i + j \rightarrow R\), \(m + n \to G\) and
    \(m + n \to H\) are defined as in (\ref{gath:l-cospan}-\ref{gath:h-cospan})
    above.
    We must show that \(
    \iltikzfig{strings/category/f}[box=f,colour=white]
    =
    \iltikzfig{strings/rewriting/rewrite-l}
    \) and \(
    \iltikzfig{strings/category/f}[box=h,colour=white]
    =
    \iltikzfig{strings/rewriting/rewrite-r}
    \).
    By definition of traced boundary complement \(\cospan{j+m}{C}{i+n}\) is a
    partial monogamous cospan, so by fullness of
    \(\termandfrobtohypsigma \circ \tracedtosymandfrobsigma\), there exists a term \(
    \iltikzfig{strings/category/f-2-2}[box=c,colour=white,dom1=j,dom2=m,cod1=i,cod2=n]
    \in \stmcsigma
    \) such that \(
    \termandfrobtohypsigma[
        \tracedtosymandfrob[
            \iltikzfig{strings/category/f-2-2}[box=c,colour=white]
        ]{\Sigma}
    ]
    =
    \cospan{j+m}{C}{i+n}
    \).
    By traced decomposition (\autoref{lem:traced-decomposition}), we have that for any
    traced boundary complement \(\cospan{\listvar{ij}}{C}{\listvar{mn}}\) and morphism
    \(L \to G\), \(\cospan{m}{G}{n}\) can be factored as in
    (\ref{gath:decomposition}), i.e.\ \[
        \termandfrobtohypsigma[\iltikzfig{strings/category/f}[box=f,colour=white]]
        =
        \trace{j}{\termandfrobtohypsigma[\iltikzfig{strings/category/f}[box=l,colour=white]]
            \tensor
            \id[n]
            \seq
            \termandfrobtohypsigma[\iltikzfig{strings/category/f-2-2}[box=c,colour=white]]}.
    \]
    By functoriality, we have \(
    \iltikzfig{strings/category/f}[box=f,colour=white]
    =
    \iltikzfig{strings/rewriting/rewrite-l}
    \).
    The same follows for \(
    \iltikzfig{strings/category/f}[box=h,colour=white]
    =
    \iltikzfig{strings/rewriting/rewrite-r}
    \).
\end{proof}
\section{Rewriting with commutative comonoid structure}