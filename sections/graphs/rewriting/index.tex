\chapter{Graph rewriting}

The results of the previous chapter have given us a combinatorial setting in
which string diagrams equal by topological deformations or by the equations of
commutative comonoids are translated into isomorphic cospans of hypergraphs.
This already gives us a straightforward way to interpret these structures
computationally.

However, it is very rare that we will only be working modulo the equations of
traced comonoids; the equations of some \emph{monoidal theory} will also be in
play.
Since an equation \(
\iltikzfig{strings/category/f}[box=f,colour=white]
=
\iltikzfig{strings/category/f}[box=g,colour=white]
\) may actually change the components of the term, there is no reason for the
interpretations \(
\termandfrobtohypsigmac[
    \tracedandcomonoidtofrobsigmac[
        \iltikzfig{strings/category/f}[box=f,colour=white]
    ]
]
\) and \(
\termandfrobtohypsigmac[
    \tracedandcomonoidtofrobsigmac[
        \iltikzfig{strings/category/f}[box=g,colour=white]
    ]
]
\) to be isomorphic.
Instead, to reason about cospans of hypergraphs modulo the equations of some
monoidal theory the graphs must be \emph{rewritten}.

Traditional term rewriting is inherently one-dimensional, which can be somewhat
restrictive as it enforces that each generator must have a coarity of \(1\).
On the other hand, string diagrams are an example of \emph{higher dimensional}
rewriting, which its roots in Burroni's work on
polygraphs~\cite{burroni1993higherdimensional}.
This higher-dimensional approach briefly entered the world of digital
circuits in Lafont's work on Boolean circuits~\cite{lafont2003algebraic}, which
used a rudimentary form of graphical syntactic rewriting.
However, in this approach even the axioms of SMCs needed to be applied
explicitly, greatly hampering the tractability of the system.

It is only more recently that rewriting string diagrams modulo the axioms of
SMCs using \emph{graph rewriting} has been studied, starting with the
aforementioned \emph{string graphs} of Dixon and
Kissinger~\cite{dixon2010open,kissinger2012pictures,dixon2013opengraphs}.
This only considered rewriting in the presence of a trace rather than
considering any additional structure, but nevertheless was successfully
implemented in a proof assistant called
Quantomatic~\cite{kissinger2015quantomatic}.
Instead, our work continues to closely follow that of Bonchi et
al~\cite{bonchi2022string,bonchi2022stringa}, which has several advantages
over string graphs.
Unlike the category of string graphs, the category of hypergraphs is
\emph{adhesive}~\cite{lack2004adhesive} which affords it some nice rewriting
properties.
Moreover, rewriting modulo Frobenius structure (or even just monoid or comonoid
structure) can reveal rewrites `hidden' behind the absorbed equations that might
not be seen with the naked eye.

In this chapter, we will explore how to extend this work to the traced and
traced comonoid case, allowing us to perform graph rewriting on the cospans
defined in the previous chapter.

\begin{remark}
    The content of this section is a revised version
    of~\cite[Sec. 5]{ghica2023rewriting}.
\end{remark}

\section{Double pushout rewriting}

As a backdrop, let us first consider how equations in a monoidal theory are
applied to terms without any sort of special graph interpretation.

\begin{definition}[Rewriting system]\label{def:term-rewriting}
    A \emph{rewriting system} \(\mcr\) for a traced PROP \(\mathbb{A}\)
    consists of a set of \emph{rewrite rules} \(
        \rrule{
            \iltikzfig{strings/category/f}[box=l,colour=white,dom=\listvar{i},cod=\listvar{j}]
        }{
            \iltikzfig{strings/category/f}[box=r,colour=white,dom=\listvar{i},cod=\listvar{j}]
        }
    \).
    Given terms \(
        \iltikzfig{strings/category/f}[box=g,colour=white,dom=\listvar{m},cod=\listvar{n}]
    \) and \(
        \iltikzfig{strings/category/f}[box=h,colour=white,dom=\listvar{m},cod=\listvar{n}]
    \) in \(\stmc{\generators}\) we write \(
        \iltikzfig{strings/category/f}[box=g,colour=white]
        \rewrite[\mcr]
        \iltikzfig{strings/category/f}[box=h,colour=white]
    \) if there exists rewrite rule \(\rrule{
        \iltikzfig{strings/category/f}[box=l,colour=white,dom=\listvar{i},cod=\listvar{j}]
    }{
        \iltikzfig{strings/category/f}[box=r,colour=white,dom=\listvar{i},cod=\listvar{j}]
    }\) in \(\mcr\) and \(
        \iltikzfig{strings/category/f-2-2}[box=c,colour=white,dom1=\listvar{j},dom2=\listvar{m},cod1=\listvar{i},cod2=\listvar{n}]
    \) in \(\stmc{\Sigma}\) such that \(
        \iltikzfig{strings/category/f}[box=g,colour=white]
        =
        \iltikzfig{strings/rewriting/rewrite-l}
    \) and \(
        \iltikzfig{strings/category/f}[box=h,colour=white]
        =
        \iltikzfig{strings/rewriting/rewrite-r}
    \) by axioms of STMCs.
\end{definition}

The difference between a rewriting system and an equational theory is that in
the former the rules are \emph{directed} whereas equations are not.
Of course, it is straightforward to derive a rewriting system from an equational
theory by adding the reductions for `both ways' of each equation.

\begin{definition}[\cite{bonchi2022stringa}, Sec. 2.4]
    Given a coloured monoidal theory \((C,\generators,\equations)\), let
    \(\mcr_{\equations}\) be the rewriting theory defined as \(\{
        \rrule{
            \iltikzfig{strings/category/f}[box=l,colour=white,dom=\listvar{i},cod=\listvar{j}]
        }{
            \iltikzfig{strings/category/f}[box=r,colour=white,dom=\listvar{i},cod=\listvar{j}]
        }
        \,|\,
        \iltikzfig{strings/category/f}[box=l,colour=white,dom=\listvar{i},cod=\listvar{j}]
        =
        \iltikzfig{strings/category/f}[box=r,colour=white,dom=\listvar{i},cod=\listvar{j}]
        \in
        \mce
    \} \cup \{
        \rrule{
            \iltikzfig{strings/category/f}[box=r,colour=white,dom=\listvar{i},cod=\listvar{j}]
        }{
            \iltikzfig{strings/category/f}[box=l,colour=white,dom=\listvar{i},cod=\listvar{j}]
        }
        \,|\,
        \iltikzfig{strings/category/f}[box=l,colour=white,dom=\listvar{i},cod=\listvar{j}]
        =
        \iltikzfig{strings/category/f}[box=r,colour=white,dom=\listvar{i},cod=\listvar{j}]
        \in
        \mce
    \}\).
\end{definition}

\begin{proposition}[\cite{bonchi2022stringa}, Prop. 2.18]
    Given two terms \(
        \iltikzfig{strings/category/f}[box=g,colour=white],
        \iltikzfig{strings/category/f}[box=h,colour=white]
        \in \stmc{C,\generators,\equations}
    \), \(
        \iltikzfig{strings/category/f}[box=g,colour=white]
        =
        \iltikzfig{strings/category/f}[box=h,colour=white]
    \) if and only if \(
        \iltikzfig{strings/category/f}[box=g,colour=white]
        \rewrites[\mcr_{\equations}]
        \iltikzfig{strings/category/f}[box=h,colour=white]
    \).
\end{proposition}

The equivalent for graphs is, unsurprisingly, \emph{graph rewriting}.
There has been a wealth of research into various graph rewriting techniques;
we use one known as \emph{double pushout (DPO)} rewriting.
DPO was introduced in the early 70s by Ehrig, Pfender, and
Schneider~\cite{ehrig1973graphgrammars} as one of the first \emph{algebraic}
approaches to graph rewriting.
Although first defined in the category of graphs, it has since been generalised
to encompass a variety of combinatorial structures; later on we will see the
precise categorical properties required for DPO rewriting to be well-defined.
Since we are working with hypergraphs we will present the definitions in terms
of hypergraphs, but one can see how they could be adapted in terms of other
structures.

The core element of a DPO rewriting system is a \emph{DPO rule}, which specifies
a possible rewrite as a span of hypergraphs.

\begin{definition}[DPO rule]
    Given interfaced hypergraphs \(
        \cospan{\listvar{i}}[a_1]{L}[a_2]{\listvar{j}}
    \) and \(
        \cospan{\listvar{i}}[b_1]{R}[b_2]{\listvar{j}}
    \) in \(\cspdchyp\), their \emph{DPO rule} is a span in \(\hypsigmac\)
    defined as \(
        \spann{L}[[a_1,a_2]]{\listvar{ij}}[[b_1,b_2]]{R}
    \).
\end{definition}

Note that the DPO rule is a span in the category of hypergraphs \(\hypsigmac\),
\emph{not} the category of interfaced hypergraphs \(\cspdchyp\)!
This is because identifying the occurrence of a DPO rule in some larger
hypergraph will be performed using a \emph{hypergraph homomorphism}.

Although we could press on now and define `normal' DPO rewriting, we will
actually use an extension, known as
\emph{double pushout rewriting with interfaces}
(DPOI rewriting)~\cite{bonchi2017confluence}.
This framework enjoys the \emph{Knuth-Bendix property}~\cite{knuth1970simple};
graph rewriting is confluent when all \emph{critical pairs are joinable}.
Put more simply, this means that a rewriting system is confluent if, whenever
there is an overlap of rules such that \(G \grewrite H\) and
\(G \grewrite H^\prime\), then there exists another graph \(K\) such that
\(H \grewrites K\) and \(H^\prime \grewrites K\).

\begin{definition}[DPOI rewriting]
    Let \(\mcr\) be a set of DPO rules.
    Then, for morphisms \(G \leftarrow \listvar{mn}\) and
    \(H \leftarrow \listvar{mn}\) in
    \(\hypsigmac\), there is a rewrite \(G \trgrewrite[\mcr] H\) if there
    exist a rule \(
        \spann{L}{\listvar{ij}}{R} \in \mcr
    \) and cospan \(
        \cospan{\listvar{ij}}{C}{\listvar{mn}} \in \hypsigma
    \) such that the following diagram commutes.
    \begin{center}
        \begin{tikzcd}[row sep = small, column sep = small]
            L \arrow{d}
            & \listvar{ij} \arrow{l}\arrow{r}\arrow{d}
            & R \arrow{d} \\
            G \arrow["\urcorner"{anchor=center, pos=0.125}, draw=none]{ur}
            & C \arrow{l}\arrow{r}
            & H \arrow["\ulcorner"{anchor=center, pos=0.125}, draw=none]{ul} \\
            & \listvar{mn} \arrow{ul}\arrow{u}\arrow{ur}
        \end{tikzcd}
    \end{center}
\end{definition}

The DPO diagram above may look a little intimidating a first glance, so we will
break it down and describe what happens during a typical graph rewrite.
The first thing to note is that the graphs all have a \emph{single} interface
\(G \leftarrow \listvar{mn}\) rather than the cospans
\(\cospan{\listvar{m}}{G}{\listvar{n}}\) we are used to.
This means that to perform graph rewriting on graphs in \(\hypsigmac\),
interfaces of terms in \(\smcsigmac + \frobc\) must be `folded' into one using
the compact closed structure.

\begin{definition}[\cite{bonchi2022stringa}]
    Let \(\morph{\foldinterfaces}{\smcsigmac + \frobc}{\smcsigmac + \frobc}\)
    be defined as having action \(
        \iltikzfig{strings/category/f}[box=f,colour=white,dom=\listvar{m},cod=\listvar{n}]
        \mapsto
        \iltikzfig{strings/rewriting/folding}[box=f,colour=white,dom=\listvar{m},cod=\listvar{n}]
    \).
\end{definition}

The image of \(\foldinterfaces\) is not in the image of
\(\tracedtosymandfrobsigmac\) or \(\tracedandcomonoidtofrobsigmac\) any more,
as inputs of generators may now connect to outputs of the term.
Fortunately this is not an issue, as long as we `unfold' the interfaces once
rewriting is completed.

\begin{proposition}[\cite{bonchi2022string}, Prop. 4.8]
    If \(
        \termandfrobtohypsigma[\iltikzfig{strings/category/f}[box=f,colour=white,dom=m,cod=n]]
        =
        \cospan{\listvar{m}}[i]{F}[o]{\listvar{n}}
    \) then \(
        \termandfrobtohypsigma[
            \foldinterfaces[
                \iltikzfig{strings/category/f}[box=f,colour=white]
            ]
        ]
    \) is isomorphic to \(
        \cospan{0}[]{F}[i+o]{\listvar{mn}}
    \).
\end{proposition}
\begin{proof}
    Straightforward by definition of the cup using the Frobenius structure.
\end{proof}

We are now ready to begin rewriting.
Say we have a DPO rule \(\spann{L}{\listvar{ij}}{R}\) and a larger cospan of
hypergraphs \(\cospan{\listvar{m}}{G}{\listvar{n}}\).
We suggestively assemble them as follows:

\begin{center}
    \begin{tikzcd}[row sep = small, column sep = small]
        L
        & \listvar{ij} \arrow{l}\arrow{r}
        & R \\
        G
        &
        & \\
        & \listvar{mn} \arrow{ul}
    \end{tikzcd}
\end{center}

As mentioned above, to identify an occurrence of \(L\) in \(G\), we use a
hypergraph homomorphism \(L \to G\) to identify the components that will be
rewritten.

\begin{center}
    \begin{tikzcd}[row sep = small, column sep = small]
        L \arrow{d}
        & \listvar{ij} \arrow{l}\arrow{r}
        & R \\
        G
        &
        & \\
        & \listvar{mn} \arrow{ul}
    \end{tikzcd}
\end{center}

We now need to identify the \emph{context} in which the rewrite will occur in.
Essentially, the context is the `graph \(G\) with \(L\) cut out', which can be
formally defined with what is known as a \emph{pushout complement}, a sort of
`reverse pushout'.

\begin{definition}[Pushout complement]\label{def:pushout-complement}
    Let \(\listvar{ij} \to L \to G \rightarrow \listvar{mn}\) be morphisms in
    \(\hypsigmac\); their \emph{pushout complement} is an object \(C\)
    with morphisms \(\listvar{ij} \to C \to G\) such that \(\cospan{L}{G}{C}\) is a
    pushout and the diagram below commutes.
    \begin{center}
        \begin{tikzcd}[column sep=large]
            L \arrow[swap]{d}{f}
            &
            \listvar{ij}
            \arrow[swap]{l}{a := [a_1, a_2]}
            \arrow{d}{c := [c_1, c_2]}
            \\
            G
            \arrow["\urcorner"{anchor=center, pos=0.125}, draw=none]{ur}
            &
            C
            \arrow{l}{g}
            \\
            &
            \listvar{mn}
            \arrow{ul}{[b_1,b_2]}
            \arrow[swap]{u}{d := [d_1,d_2]}
        \end{tikzcd}
    \end{center}
\end{definition}

Is a pushout complement always guaranteed to exist for any morphism \(L \to G\)?
The answer is no, but fortunately the existence conditions are fairly standard.

\begin{definition}[No-dangling-hyperedges condition~\cite{bonchi2022string}, Def.\ 3.16]
    Given morphisms \(\listvar{ij} \xrightarrow{a} L \xrightarrow{f} G\) in
    \(\hypsigma\), they satisfy the \emph{no-dangling} condition if, for every
    hyperedge not in
    the image of \(f\), each of its source and target vertices is either not in
    the image of \(f\) or are in the image of \(f \circ a\).
\end{definition}

\begin{definition}[No-identification condition~\cite{bonchi2022string}, Def.\ 3.16]
    Given morphisms \(\listvar{ij} \xrightarrow{a} L \xrightarrow{f} G\) in
    \(\hypsigma\), they satisfy the \emph{no-identification} condition if any
    two distinct elements merged by \(f\) are also in the image of \(f \circ a\).
\end{definition}

\begin{proposition}[\cite{bonchi2022string}, Prop. 3.17]
    \label{prop:pushout-complement}
    The morphisms \(\listvar{ij} \to L \to G\) have at least one pushout complement if
    and only if they satisfy the no-dangling and no-identification conditions.
\end{proposition}

\begin{definition}
    A morphism \(\listvar{ij} \to L \to G\) is called a \emph{matching} if it
    has at least one pushout complement.
\end{definition}

A pushout complement specifies the rewriting context, leaving us a hole in which
the other side of the rewrite rule can be glued in.

\begin{center}
    \begin{tikzcd}[row sep = small, column sep = small]
        L \arrow{d}
        & \listvar{ij} \arrow{l}\arrow{r}\arrow{d}
        & R \\
        G \arrow["\urcorner"{anchor=center, pos=0.125}, draw=none]{ur}
        & C \arrow{l}
        & \\
        & \listvar{mn} \arrow{ul}\arrow{u}
    \end{tikzcd}
\end{center}

To actually compute the rewritten graph, we perform another pushout to retrieve
the complete DPO diagram.

\begin{center}
    \begin{tikzcd}[row sep = small, column sep = small]
        L \arrow{d}
        & \listvar{ij} \arrow{l}\arrow{r}\arrow{d}
        & R \arrow{d} \\
        G \arrow["\urcorner"{anchor=center, pos=0.125}, draw=none]{ur}
        & C \arrow{l}
        & H \\
        & \listvar{mn} \arrow{ul}\arrow{u}\arrow{ur}
    \end{tikzcd}
\end{center}

\begin{example}
    \todo[inline]{Do a concrete example}
\end{example}
\section{Pushout complements}

The reader should now have a good grasp on how DPO rewriting
works; find a matching, compute the pushout complement, and perform one more
pushout to complete the rewrite.
While the final step is a deterministic procedure, finding the pushout
complement is a little more subtle.
Since the pushout complement uniquely defines the final rewrite, one might be
cautious about how we choose it.
First of all, a pushout complement may not even exist for arbitrary morphisms
\(i+j \to L \to G\).

\begin{definition}
    A morphism \(i+j \to L \to G\) is called a \emph{matching} if it
    has at least one pushout complement.
\end{definition}

If there is no pushout complement, there is no possible rewrite, so it is
important to know when one exists.
Fortunately, the existence conditions are well-known.

\begin{definition}[No-dangling-hyperedges condition~\cite{corradini1997algebraic}, Prop. 3.3.4]
    Given morphisms \(i+j \xrightarrow{a} L \xrightarrow{f} G\) in
    \(\hypsigma\), they satisfy the \emph{no-dangling} condition if, for every
    hyperedge not in
    the image of \(f\), each of its source and target vertices is either not in
    the image of \(f\) or are in the image of \(f \circ a\).
\end{definition}

\begin{example}
    The following pair of morphisms does not satisfy the no-dangling-hyperedges
    condition.
    \[
        \iltikzfig{graphs/dpo/no-dangling/k}
        \xrightarrow{a}
        \iltikzfig{graphs/dpo/no-dangling/l}
        \xrightarrow{f}
        \iltikzfig{graphs/dpo/no-dangling/g}
    \]
    To obtain the pushout complement we `cut out' any vertices in the
    rightmost graph which are in the image of \(f\) but not the image of
    \(f \circ a\), as the latter are the interfaces of the rule.
    However, if we cut out the vertices labelled \(2\) and \(3\), the edge
    \(e_2\) will be left with `dangling' tentacles connected to no vertices, a
    malformed hypergraph.
    \begin{center}
        \begin{tikzcd}
            \iltikzfig{graphs/dpo/no-dangling/l}
            \arrow{d}{f}
            &
            \iltikzfig{graphs/dpo/no-dangling/k}
            \arrow{l}{a}
            \arrow{d}
            \\
            \iltikzfig{graphs/dpo/no-dangling/g}
            &
            \iltikzfig{graphs/dpo/no-dangling/c}
            \arrow{l}
        \end{tikzcd}
    \end{center}
\end{example}

\begin{definition}[No-identification condition~\cite{corradini1997algebraic}, Prop. 3.3.4]
    Given morphisms \(i+j \xrightarrow{a} L \xrightarrow{f} G\) in
    \(\hypsigma\), they satisfy the \emph{no-identification} condition if any
    two distinct elements merged by \(f\) are also in the image of \(f \circ a\).
\end{definition}

\begin{example}
    The following morphisms do not satisfy the no-identification
    condition.
    \[
        \iltikzfig{graphs/dpo/no-identification/k}
        \xrightarrow{a}
        \iltikzfig{graphs/dpo/no-identification/l}
        \xrightarrow{f}
        \iltikzfig{graphs/dpo/no-identification/g}
    \]
    When trying to construct a pushout complement, the edge \(e_2\) will be
    removed.
    However, since vertices \(2\) and \(3\) are not mapped from the rule
    interfaces, there is no reason that a pushout would glue them together so
    that they are merged in the final graph.
    Therefore no pushout complement can exist.
    \begin{center}
        \begin{tikzcd}
            \iltikzfig{graphs/dpo/no-identification/l}
            \arrow{d}{f}
            &
            \iltikzfig{graphs/dpo/no-identification/k}
            \arrow{l}{a}
            \arrow{d}
            \\
            \iltikzfig{graphs/dpo/no-identification/g}
            &
            \iltikzfig{graphs/dpo/no-identification/c}
            \arrow{l}
        \end{tikzcd}
    \end{center}
\end{example}

With these two conditions, we can establish when pushout complements exist at
all and subsequently when there is an opportunity for a rewrite.

\begin{proposition}[\cite{corradini1997algebraic}, Prop. 3.3.4]
    \label{prop:pushout-complement}
    The morphisms \(i+j \to L \to G\) have at least one pushout
    complement if and only if they satisfy the no-dangling and no-identification
    conditions.
\end{proposition}

It is all very well knowing if there is at least one pushout complement, but
what about when there is \emph{exactly one} pushout complement?
When this is the case, the rewrite is uniquely specified for a given rule and
matching.
To answer this question, we must examine a class of categories to which
\(\hypsigmac\) belongs to, known as \emph{adhesive} categories.
One can think of these as categories in which graph rewriting `plays nicely'.

\begin{definition}[van Kampen square~(\cite{lack2005adhesive}, Def. 2.1)]
    Let there be a commutative cube as drawn below.
    \begin{center}
        \begin{tikzcd}
            G \arrow{rrr} \arrow{ddd} &
            &
            &
            H \arrow{ddd} \\
            &
            E \arrow{ul} \arrow{r} \arrow{d} &
            F \arrow{ur} \arrow{d} &
            \\
            &
            A \arrow{dl} \arrow{r} &
            B \arrow{dr} &
            \\
            C \arrow{rrr} &
            &
            &
            D
        \end{tikzcd}
    \end{center}
    The bottom face of the cube (\(ABCD\)) is called a
    \emph{van Kampen (VK) square} if it is a pushout and, when the back and
    left faces (\(EFAB\) and \(GECA\)) are pullbacks, the front and right faces
    (\(GHCD\) and \(FHBD\)) are pullbacks if and only if the top face (\(GHEF\))
    is a pushout.
\end{definition}

\begin{definition}
    Given a span \(\spann{A}[f]{B}[g]{C}\) with a pushout \(\cospan{B}{D}{C}\),
    the pushout is called a \emph{pushout along a monomorphism} if \(f\) or
    \(g\) is a monomorphism.
\end{definition}

\begin{definition}[Adhesive category~(\cite{lack2005adhesive}, Def. 3.1)]
    A category \(\mcc\) is \emph{adhesive} if
    \begin{itemize}
        \item \(\mcc\) has pushouts along monomorphisms;
        \item \(\mcc\) has pullbacks; and
        \item pushouts along monomorphisms are VK squares.
    \end{itemize}
\end{definition}

The definition of van Kampen square and subsequently adhesive categories may
look a bit confusing to the uninitiated.
At a high level, an adhesive category is one in which one in which objects can
be `split apart' and `glued together' by using pushouts and pullbacks.

\begin{example}
    A natural example of an adhesive category is \(\set\), which has pushouts
    and pullbacks.
    To get some idea how the van Kampen condition holds, consider the
    following commutative cube in \(\set\), adapted from
    \cite[Sec. 4.3]{kissinger2012pictures}.
    \begin{center}
        \begin{tikzcd}
            A^\prime
            \arrow{rrr}
            \arrow{ddd}{f_A}
            &
            &
            &
            X
            \arrow{ddd}{f}
            \\
            &
            A^\prime \cap B^\prime
            \arrow{r}
            \arrow{d}
            \arrow{ul}
            &
            B^\prime
            \arrow{ur}
            \arrow{d}{f_B}
            &
            \\
            &
            A \cap B
            \arrow{r}
            \arrow{dl}
            &
            B
            \arrow{dr}
            &
            \\
            A
            \arrow{rrr}
            &
            &
            &
            A \cup B
        \end{tikzcd}
    \end{center}
    The bottom face is a pushout, and since \(A \cup B \to A\) and
    \(A \cup B \to B\) are monomorphisms, this is a pushout along a
    monomorphism.
    Furthermore, the left and back faces are pullbacks because \(f_A\) and
    \(f_B\) agree on the intersection of \(A^\prime\) and \(B^\prime\).

    Now we must show that the front and right faces are pullbacks if and only if
    the top face is a pushout.
    For the front and right faces to be pullbacks, \(f\) must restrict to
    \(f_A\) and \(f_B\) along \(A\) and \(B\).
    This can only be the case if and only if \(X = A^\prime \cap B^\prime\), in
    which case the top square must be a pushout.
\end{example}

Regardless of intuition, proving the necessary van Kampen condition might be
tricky.
Fortunately, adhesivity is preserved by several categorical constructions, so
by using \(\set\) as a base it is straightforward to show that more complicated
categories are also adhesive.

\begin{proposition}[\cite{lack2005adhesive}, Prop. 3.5]
    For an adhesive category \(\mcc\) and object \(C\) of \(\mcc\),
    \(\mcc \slice C\) is adhesive.
    For another category \(\mcx\), \([\mcx, \mcc]\) is also adhesive.
\end{proposition}

\begin{corollary}
    \(\hypsigma\) and \(\hypsigmac\) are adhesive.
\end{corollary}
\begin{proof}
    \(\hypsigma\) and \(\hypsigmac\) are defined as the slice of a functor
    category over \(\set\), so they are adhesive.
\end{proof}

The notion of splitting and gluing is very useful when it comes to graph
rewriting, as it corresponds to how we extract and insert various subgraphs
during the rewriting process.
In particular, for certain DPO rules in an adhesive category a pushout
complement is \emph{uniquely} defined for a given matching.

\begin{definition}[Left-linear rules]
    A DPO rule \(\spann{L}[f]{i+j}{R}\) is called \emph{left-linear} if \(f\)
    is mono.
\end{definition}

\begin{theorem}[\cite{lack2005adhesive}, Lem. 4.5]
    In an adhesive category, if a pushout complement exists for morphisms
    \(I \xrightarrow{m} L \to G\) and \(m\) is a monomorphism, then it is unique
    up to isomorphism.
\end{theorem}
\begin{proof}
    The proof relies on several non-trivial lemmas that hold in adhesive
    categories in addition to some other results about pushouts.
    We refer the interested reader to
    \cite[Lems. 4.3.6 - 4.3.9]{kissinger2012pictures} for the grisly details.
\end{proof}

This means that, for a large class of rewrite rules, if we find a matching we
can be sure that rewriting is defined uniquely.
However, there may be useful rewrite rules which are \emph{not} left-linear.

\begin{example}\label{ex:non-left-linear}
    Consider the following rule.
    \[
        \rrule{
            \iltikzfig{strings/category/identity}
        }{
            \iltikzfig{strings/category/generator}[box=e]
        }
        \qquad
        \iltikzfig{graphs/dpo/non-left-linear/l}
        \leftarrow
        \iltikzfig{graphs/dpo/non-left-linear/k}
        \to
        \iltikzfig{graphs/dpo/non-left-linear/r}
    \]
    A rule where a generator is inserted into an identity wire is perfectly
    reasonable, but its interpretation as a span of hypergraphs is not
    left-linear.
    This means that the pushout complement may not be unique for a given
    matching.
    For example, consider trying to use the above rule in the term
    \(\iltikzfig{graphs/dpo/non-left-linear/g-term}\) using the following pair
    of morphisms on graphs:
    \[
        \iltikzfig{graphs/dpo/non-left-linear/k}
        \to
        \iltikzfig{graphs/dpo/non-left-linear/l}
        \to
        \iltikzfig{graphs/dpo/non-left-linear/g}
    \]
    This matching yields the following pushout complements and rewrites:
    \begin{gather*}
        \iltikzfig{graphs/dpo/non-left-linear/c1}
        \to
        \iltikzfig{graphs/dpo/non-left-linear/h1}
        \qquad
        \iltikzfig{graphs/dpo/non-left-linear/c2}
        \to
        \iltikzfig{graphs/dpo/non-left-linear/h2}
        \\[0.5em]
        \iltikzfig{graphs/dpo/non-left-linear/c3}
        \to
        \iltikzfig{graphs/dpo/non-left-linear/h3}
        \qquad
        \iltikzfig{graphs/dpo/non-left-linear/c4}
        \to
        \iltikzfig{graphs/dpo/non-left-linear/h4}
        \\[0.5em]
        \iltikzfig{graphs/dpo/non-left-linear/c5}
        \to
        \iltikzfig{graphs/dpo/non-left-linear/h5}
    \end{gather*}
\end{example}

One might think this is undesirable, but these multiple rewrites actually arise
due to the presence of the Frobenius algebra.

\begin{example}
    Each of the five complements and rewrites in \cref{ex:non-left-linear}
    corresponds to a valid application of equations on terms, perhaps modulo
    Frobenius equations.
    The first rewrite \(
    \iltikzfig{graphs/dpo/non-left-linear/h1}
    \) is the `obvious' one:
    \begin{gather*}
        \iltikzfig{graphs/dpo/non-left-linear/g-term}
        =
        \iltikzfig{graphs/dpo/non-left-linear/g-term-rewrite-1-0}
        =
        \iltikzfig{graphs/dpo/non-left-linear/g-term-rewrite-1-1}
        =
        \iltikzfig{graphs/dpo/non-left-linear/g-term-rewrite-1-2}
    \end{gather*}
    The second rewrite \(
    \iltikzfig{graphs/dpo/non-left-linear/h2}
    \) uses the compact closed structure:
    \begin{gather*}
        \iltikzfig{graphs/dpo/non-left-linear/g-term}
        =
        \iltikzfig{graphs/dpo/non-left-linear/g-term-rewrite-2-0}
        =
        \iltikzfig{graphs/dpo/non-left-linear/g-term-rewrite-2-1}
        =
        \iltikzfig{graphs/dpo/non-left-linear/g-term-rewrite-2-2}
        =
        \iltikzfig{graphs/dpo/non-left-linear/g-term-rewrite-2-3}
    \end{gather*}
    The third rewrite \(
    \iltikzfig{graphs/dpo/non-left-linear/h3}
    \) uses the unitality of the monoid:
    \begin{gather*}
        \iltikzfig{graphs/dpo/non-left-linear/g-term}
        =
        \iltikzfig{graphs/dpo/non-left-linear/g-term-rewrite-3-0}
        =
        \iltikzfig{graphs/dpo/non-left-linear/g-term-rewrite-3-1}
        =
        \iltikzfig{graphs/dpo/non-left-linear/g-term-rewrite-3-2}
        =
        \iltikzfig{graphs/dpo/non-left-linear/g-term-rewrite-3-3}
    \end{gather*}
    Similarly, the third rewrite \(
    \iltikzfig{graphs/dpo/non-left-linear/h4}
    \) uses the counitality of the comonoid:
    \begin{gather*}
        \iltikzfig{graphs/dpo/non-left-linear/g-term}
        =
        \iltikzfig{graphs/dpo/non-left-linear/g-term-rewrite-4-0}
        =
        \iltikzfig{graphs/dpo/non-left-linear/g-term-rewrite-4-1}
        =
        \iltikzfig{graphs/dpo/non-left-linear/g-term-rewrite-4-2}
        =
        \iltikzfig{graphs/dpo/non-left-linear/g-term-rewrite-4-3}
    \end{gather*}
    The fifth rewrite \(
    \iltikzfig{graphs/dpo/non-left-linear/h5}
    \) uses the Frobenius equations:
    \begin{gather*}
        \iltikzfig{graphs/dpo/non-left-linear/g-term}
        =
        \iltikzfig{graphs/dpo/non-left-linear/g-term-rewrite-5-0}
        =
        \iltikzfig{graphs/dpo/non-left-linear/g-term-rewrite-5-1}
        =
        \iltikzfig{graphs/dpo/non-left-linear/g-term-rewrite-5-2}
        =
        \iltikzfig{graphs/dpo/non-left-linear/g-term-rewrite-5-3}
        =
        \\
        \iltikzfig{graphs/dpo/non-left-linear/g-term-rewrite-5-4}
        =
        \iltikzfig{graphs/dpo/non-left-linear/g-term-rewrite-5-5}
        =
        \iltikzfig{graphs/dpo/non-left-linear/g-term-rewrite-5-6}
        =
        \iltikzfig{graphs/dpo/non-left-linear/g-term-rewrite-5-7}
    \end{gather*}
    All of the rewrites other than the first were `hidden' behind the Frobenius
    equations, but were exposed automatically by the graph rewriting framework.
\end{example}

Even if every rewrite is correct, there is still the question of how to find
them all.
For hypergraphs this problem has already been
tackled~\cite{heumuller2011construction}; the pushout complements can be
enumerated as quotients of an `exploded' context.

\begin{definition}[Exploded context~(\cite{heumuller2011construction}, Const. 1)]
    Let \(i+j \to L \xrightarrow{f} G\) be a pair of morphisms in
    \(\hypsigma\).
    Then the \emph{exploded context} for these morphisms is a graph
    \(\listvar{ij} + \tilde{G}\) where \(\tilde{G}\) is constructed as follows:
    \begin{enumerate}
        \item for each vertex \(v \in G\) not in the image of \(f\), add one
              vertex to \(\tilde{G}\);
        \item for each hyperedge \(e \in G\) not in the image of \(f\), add one
              hyperedge to \(\tilde{G}\);
        \item for each hyperedge \(e \in \tilde{G}\), let the \(i\)-th source
              \(\tilde{s_i}(e)\) be defined as \(s_i(h)\) if
              \(s_i(h) \in \tilde{G}\) or a new, fresh vertex otherwise;
        \item repeat the above for the targets.
    \end{enumerate}
\end{definition}

Pushout complements can then be computed as quotients of this exploded
context.

\begin{proposition}[\cite{heumuller2011construction} (Props. 3-4), \cite{bonchi2022string}]
    For a pair of morphisms \(i+j \to L \to G\) in \(\hypsigma\), let
    \(i + j + \tilde{G}\) be its exploded context.
    Define a map \(\morph{q}{i + j + \tilde{G}}{G}\) sending elements in
    \(\tilde{G}\) from \(G\) to themselves, and sending vertices from
    \(i + j\) to their image under \(i + j \to L \to G\).
    Then a pushout complement \(i + j \to C \to G\) is valid if and only
    \(C\) is obtained as a quotient on the exploded context
    that only identifies vertices in the image of \(q^{-1}(v)\) for each vertex
    \(v \in G\).
\end{proposition}

This means that given a DPO rule and an incidence of this rule inside a
larger graph such that the no-dangling and no-identification conditions are
satisfied, we can enumerate all the possible pushout complements.
It can be shown that each of these pushout complements correspond to a valid
rewrite in a Frobenius setting.

\begin{notation}
    Given a rewrite rule \(
    \rrule{
        \iltikzfig{strings/category/f}[box=l,colour=white,dom=i,cod=j]
    }{
        \iltikzfig{strings/category/f}[box=r,colour=white,dom=i,cod=j]
    }
    \), its interpretation as a DPO rule is \(
    \termandfrobtohypsigma[
        \rrule{
            \iltikzfig{strings/category/f}[box=l,colour=white]
        }{
            \iltikzfig{strings/category/f}[box=r,colour=white]
        }
    ]
    \coloneqq
    \spann{
        \termandfrobtohypsigma[
            \foldinterfaces[
                \iltikzfig{strings/category/f}[box=l,colour=white]
            ]
        ]
    }{i+j}{
        \termandfrobtohypsigma[
            \foldinterfaces[
                \iltikzfig{strings/category/f}[box=r,colour=white]
            ]
        ]
    }
    \).
\end{notation}

\begin{theorem}[\cite{bonchi2022string}, Thm. 4.9]
    Let  \(\rrule{
        \iltikzfig{strings/category/f}[box=l,colour=white]
    }{
        \iltikzfig{strings/category/f}[box=r,colour=white]
    }\) be a rewrite rule in \(
    \smcsigma + \frob
    \).
    Then, \(
    \iltikzfig{strings/category/f}[box=g,colour=white]
    \rewrite[\rrule{
            \iltikzfig{strings/category/f}[box=l,colour=white]
        }{
            \iltikzfig{strings/category/f}[box=r,colour=white]
        }]
    \iltikzfig{strings/category/f}[box=h,colour=white]
    \) if and only if \(
    \termandfrobtohypsigma[
        \foldinterfaces[
            \iltikzfig{strings/category/f}[box=g,colour=white]
        ]
    ]
    \grewrite[
        \termandfrobtohypsigma[
            \rrule{
                \iltikzfig{strings/category/f}[box=l,colour=white]
            }{
                \iltikzfig{strings/category/f}[box=r,colour=white]
            }
        ]
    ]
    \termandfrobtohypsigma[
        \foldinterfaces[
            \iltikzfig{strings/category/f}[box=g,colour=white]
        ]
    ]\).
\end{theorem}

\subsection{Multicoloured rewriting}

The results generalise in the obvious way to the coloured setting.

\begin{notation}
    For rule \(
    \rrule{
        \iltikzfig{strings/category/f}[box=l,colour=white,dom=\listvar{i},cod=\listvar{j}]
    }{
        \iltikzfig{strings/category/f}[box=r,colour=white,dom=\listvar{i},cod=\listvar{j}]
    }
    \) in \(\smcsigmac + \frobc\), its interpretation as a DPO rule is written
    as \(
    \termandfrobtohypsigmac[
        \rrule{
            \iltikzfig{strings/category/f}[box=l,colour=white]
        }{
            \iltikzfig{strings/category/f}[box=r,colour=white]
        }
    ]
    \coloneqq
    \spann{
        \termandfrobtohypsigmac[
            \foldinterfacesc[
                \iltikzfig{strings/category/f}[box=l,colour=white]
            ]
        ]
    }{\listvar{ij}}{
        \termandfrobtohypsigmac[
            \foldinterfacesc[
                \iltikzfig{strings/category/f}[box=r,colour=white]
            ]
        ]
    }
    \).
\end{notation}

\begin{definition}[\cite{bonchi2022string}]
    Let \(\morph{\foldinterfacesc}{\smcsigmac + \frobc}{\smcsigmac + \frobc}\)
    be defined as having action \(
    \iltikzfig{strings/category/f}[box=f,colour=white,dom=\listvar{m},cod=\listvar{n}]
    \mapsto
    \iltikzfig{strings/rewriting/folding}[box=f,colour=white,dom=\listvar{m},cod=\listvar{n}]
    \).
\end{definition}

\begin{theorem}[\cite{bonchi2022string}, Prop. 4.10]
    For a rule \(\rrule{
        \iltikzfig{strings/category/f}[box=l,colour=white]
    }{
        \iltikzfig{strings/category/f}[box=r,colour=white]
    }\) in \(
    \smcsigmac + \frobc
    \), \(
    \iltikzfig{strings/category/f}[box=g,colour=white]
    \rewrite[\rrule{
            \iltikzfig{strings/category/f}[box=l,colour=white]
        }{
            \iltikzfig{strings/category/f}[box=r,colour=white]
        }]
    \iltikzfig{strings/category/f}[box=h,colour=white]
    \) if and only if \(
    \termandfrobtohypsigmac[
        \foldinterfacesc[
            \iltikzfig{strings/category/f}[box=g,colour=white]
        ]
    ]
    \grewrite[
        \termandfrobtohypsigmac[
            \rrule{
                \iltikzfig{strings/category/f}[box=l,colour=white]
            }{
                \iltikzfig{strings/category/f}[box=r,colour=white]
            }
        ]
    ]
    \termandfrobtohypsigmac[
        \foldinterfacesc[
            \iltikzfig{strings/category/f}[box=g,colour=white]
        ]
    ]\).
\end{theorem}
\section{Rewriting with traced structure}

In the Frobenius setting, every pushout complement is a valid rewrite, but there
is no reason for the same to be the case for traced or traced comonoid
rewriting.
Bonchi et al showed in~\cite{bonchi2022stringa} that \emph{exactly one} pushout
complement corresponds to a valid rewrite in the symmetric monoidal case by
characterising it as a \emph{boundary complement}.

\begin{definition}[Boundary complement (\cite{bonchi2022stringa}, Def. 30)]
    For monogamous cospans \(
    \cospan{\listvar{i}}[a_1]{L}[a_2]{\listvar{j}}
    \) and \(
    \cospan{\listvar{m}}[b_1]{G}[b_2]{\listvar{n}}
    \) and a monomorphism \(\morph{f}{L}{G}\), a pushout complement as below
    \begin{center}
        \begin{tikzcd}[column sep=large]
            L \arrow[swap]{d}{f}
            &
            \listvar{ij}
            \arrow[swap]{l}{a := [a_1, a_2]}
            \arrow{d}{c := [c_1, c_2]}
            \\
            G
            \arrow["\urcorner"{anchor=center, pos=0.125}, draw=none]{ur}
            &
            C
            \arrow{l}{g}
            \\
            &
            \listvar{mn}
            \arrow{ul}{[b_1,b_2]}
            \arrow[swap]{u}{d := [d_1,d_2]}
        \end{tikzcd}
    \end{center}
    is called a \emph{boundary complement} if \(c_1\) and \(c_2\) are
    mono and \(
    \cospan{\listvar{jm}}[[c_2,d_1]]{C}[[d_2,c_1]]{\listvar{ni}}
    \) is a boundary monogamous cospan.
\end{definition}

What is particularly special about boundary complements is that for morphisms
\(\listvar{ij} \to L \to G\) there is always \emph{exactly one} pushout
complement which is also a boundary complement!

\begin{proposition}[\cite{bonchi2022stringa}, Prop. 31]
    When boundary complements exist in \(\hypsigmac\), they are unique.
\end{proposition}

For rewriting in a traced setting, boundary complements are too strong, so we
will weaken them to \emph{traced} boundary complements, replacing references to
monogamy with partial monogamy.

\begin{definition}[Traced boundary complement]
    \label{def:traced-boundary-complement}
    For partial monogamous cospans \(
    \cospan{\listvar{i}}[a_1]{L}[a_2]{\listvar{j}}
    \) and \(
    \cospan{\listvar{m}}[b_1]{G}[b_2]{\listvar{n}}
    \), a pushout complement as below
    \begin{center}
        \begin{tikzcd}[column sep=large]
            L \arrow[swap]{d}{f}
            &
            \listvar{ij}
            \arrow[swap]{l}{a := [a_1, a_2]}
            \arrow{d}{c := [c_1, c_2]}
            \\
            G
            \arrow["\urcorner"{anchor=center, pos=0.125}, draw=none]{ur}
            &
            C
            \arrow{l}{g}
            \\
            &
            \listvar{mn}
            \arrow{ul}{[b_1,b_2]}
            \arrow[swap]{u}{d := [d_1,d_2]}
        \end{tikzcd}
    \end{center}
    is called a \emph{traced boundary complement} if \(c_1\) and \(c_2\) are
    mono and \(
    \cospan{\listvar{jm}}[[c_2,d_1]]{C}[[d_2,c_1]]{\listvar{ni}}
    \) is a partial monogamous cospan.
\end{definition}

By restricting to traced boundary complements, DPO rewriting can be formulated
for terms in a traced setting.

\begin{definition}[Traced DPO]
    For morphisms \(G \leftarrow \listvar{mn}\) and \(H \leftarrow \listvar{mn}\) in
    \(\hypsigma\), there is a traced rewrite \(G \trgrewrite[\mcr] H\) if there
    exists a rule \(
    \spann{L}{\listvar{ij}}{R} \in \mcr
    \) and cospan \(
    \cospan{\listvar{ij}}{C}{\listvar{mn}} \in \hypsigma
    \) such that diagram in \cref{def:dpo-rewriting} commutes and \(\listvar{ij} \to C\)
    is a traced boundary complement.
\end{definition}

Traced boundary complements are not much more powerful than regular boundary
complements; \(c_1\) is still not permitted to merge vertices in the inputs and
the same for \(c_2\) and the outputs.
The real power of traced DPO, and what increases the number of pushout
complements, is the fact that the matching \(L \to G\) is no longer required to
be mono.

Since \(\cospan{\listvar{m}}{G}{\listvar{n}}\) is partial monogamous, vertices
in \(L\) can only be merged if their degrees sum to no more than \((1,1)\).
Merging vertices in this way corresponds to using the trace to find a match.

\begin{example}
    Consider the rule \(
    \rrule{
        \iltikzfig{graphs/dpo/traced-example/rule-lhs}
    }{
        \iltikzfig{graphs/dpo/traced-example/rule-rhs}
    }
    \) and the term \(
    \iltikzfig{graphs/dpo/traced-example/term}
    \), in which there is clearly an instance of the rule.
    The interpretation of this as a DPO derivation with a valid traced boundary
    complement is illustrated below.
    \begin{center}
        \includestandalone{figures/graphs/dpo/traced-example/rewrite}
    \end{center}
\end{example}

A key feature of rewriting modulo traced structure is the \emph{yanking} axiom,
which can lead to some non-obvious rewrites.

\begin{example}
    Consider the rule \(
    \rrule{
        \iltikzfig{graphs/dpo/split-loop/rule-lhs}
    }{
        \iltikzfig{graphs/dpo/split-loop/rule-rhs}
    }
    \).
    The interpretation of this as a DPO rule in a valid traced boundary
    complement is illustrated below.
    \begin{center}
        \includestandalone{figures/graphs/dpo/split-loop/rewrite}
    \end{center}
    This corresponds to a valid term rewrite:
    \[
        \iltikzfig{graphs/dpo/split-loop/derivation-1}
        =
        \iltikzfig{graphs/dpo/split-loop/derivation-2}
        =
        \iltikzfig{graphs/dpo/split-loop/derivation-3}
        =
        \iltikzfig{graphs/dpo/split-loop/derivation-4}
    \]

    Note that applying yanking is required in the term setting because
    the traced wire is flowing from right to left, whereas applying the rule
    requires all wires flowing left to right.
\end{example}

Use of yanking is also what can lead to multiple boundary complements, and hence
a choice in rewrites.

\begin{example}
    Consider the rule \(
    \rrule{
        \iltikzfig{graphs/dpo/non-unique/rule-lhs}
    }{
        \iltikzfig{graphs/dpo/non-unique/rule-rhs}
    }
    \).
    Below are two valid traced boundary complements involving a matching of this
    rule.

    \begin{center}
        \scalebox{0.95}{\includestandalone{figures/graphs/dpo/non-unique/rewrite-1}}
        \quad
        \scalebox{0.95}{\includestandalone{figures/graphs/dpo/non-unique/rewrite-2}}
    \end{center}

    These two derivations arise through yanking:
    \begin{gather*}
        \iltikzfig{graphs/dpo/non-unique/derivation-1}
        =
        \iltikzfig{graphs/dpo/non-unique/derivation-2}
        =
        \iltikzfig{graphs/dpo/non-unique/derivation-3a}
        =
        \iltikzfig{graphs/dpo/non-unique/derivation-4a}
        =
        \iltikzfig{graphs/dpo/non-unique/derivation-5a}
        \\
        \iltikzfig{graphs/dpo/non-unique/derivation-1}
        =
        \iltikzfig{graphs/dpo/non-unique/derivation-2}
        =
        \iltikzfig{graphs/dpo/non-unique/derivation-3b}
        =
        \iltikzfig{graphs/dpo/non-unique/derivation-4b}
        =
        \iltikzfig{graphs/dpo/non-unique/derivation-5b}
    \end{gather*}
\end{example}

There is another condition on graph rewriting modulo symmetric monoidal
structure in that the matching must be \emph{convex}: any path between vertices
must also be captured.
Luckily for us, this is not necessary in the traced case.

\begin{example}
    Consider the rule \(
    \rrule{
        \iltikzfig{graphs/dpo/convex/example-l}
    }{
        \iltikzfig{graphs/dpo/convex/example-r}
    }
    \) and the term \iltikzfig{graphs/dpo/convex/example-g}.
    Although it is not immediately obvious, there is in fact
    a matching of the former in the latter.
    Performing the DPO procedure yields the following:
    %
    \begin{gather*}
        \includestandalone{figures/graphs/dpo/convex/rewrite}
    \end{gather*}
    In a non-traced setting this is an invalid rule!
    However, it is possible with yanking.
    \begin{gather*}
        \iltikzfig{graphs/dpo/convex/example-g}
        =
        \iltikzfig{graphs/dpo/convex/rewrite-2}
        =
        \iltikzfig{graphs/dpo/convex/rewrite-4}
        =
        \iltikzfig{graphs/dpo/convex/rewrite-5}
        =
        \iltikzfig{graphs/dpo/convex/example-h}
    \end{gather*}
    This shows that convexity is not a required component of traced rewriting.
\end{example}

\todo[inline]{Refine after here}

We are almost ready to show the soundness and completeness of this DPO rewriting
system.
The final prerequisite is a decomposition lemma akin to that used
in~\cite{bonchi2022string}.

\begin{lemma}[Traced decomposition]\label{lem:traced-decomposition}
    For partial monogamous cospans \(
    \cospan{\listvar{m}}[d_1]{G}[d_2]{\listvar{n}}
    \) and \(
    \cospan{\listvar{i}}[a_1]{L}[a_2]{\listvar{j}}
    \), and a morphism \(
    L \xrightarrow{f} G
    \) such that \(\listvar{ij} \rightarrow L \rightarrow G\) satisfies the no-dangling
    and no-identification conditions, then there exists a partial monogamous
    cospan \(
    \cospan{\listvar{jm}}[[c_2,d_1]]{C}[[c_1,d_2]]{\listvar{in}}
    \) such that \(
    \cospan{m}{G}{n}
    \) can be factored as
    \begin{gather*}
        \trace{\listvar{i}}{
            \begin{array}{cc}
                \cospan{\listvar{i}}[a_1]{L}[a_2]{\listvar{j}} \\
                \tensor                                        \\
                \cospan{\listvar{m}}{\listvar{m}}{\listvar{m}}
            \end{array}
            \seq
            \cospan{\listvar{jm}}[[c_2,d_1]]{C}[[c_1,d_2]]{\listvar{in}}
        }
    \end{gather*}
    where \(
    \cospan{\listvar{jm}}[c_2,d_1]{C}[c_1,d_2]{\listvar{in}}
    \) is a traced boundary complement.
\end{lemma}
\begin{proof}
    First let us assign names to the cospans in the decomposed form.
    \begin{gather*}
        \iltikzfig{strings/category/f}[box=l,colour=white,dom=\listvar{i},cod=\listvar{j}]
        \coloneqq
        \cospan{\listvar{i}}{L}{\listvar{j}}
        \qquad
        \iltikzfig{strings/category/f}[box=g,colour=white,dom=\listvar{m},cod=\listvar{n}]
        \coloneqq
        \cospan{\listvar{m}}{G}{\listvar{n}}
        \\
        \iltikzfig{strings/category/f-2-2}[box=c,colour=white,dom1=\listvar{j},dom2=m,cod1=\listvar{i},cod2=\listvar{n}]
        \coloneqq
        \cospan{\listvar{jm}}[[c_2, d_1]]{C}[[c_1, d_2]]{\listvar{in}}
    \end{gather*}
    We must show that  \(
    \iltikzfig{strings/category/f}[box=g,colour=white,dom=\listvar{m},cod=\listvar{n}]
    =
    \iltikzfig{strings/rewriting/rewrite-l}[dom=\listvar{m},cod=\listvar{n}]
    \).

    Let \(
    \listvar{ij} \xrightarrow{[c_1, c_2]} C \xleftarrow{[d_1, d_2]} \listvar{mn}
    \) be defined as a traced boundary complement of \(
    \listvar{ij} \xrightarrow{[a_1,a_2]} L \xrightarrow{f} G
    \), which exists as the no-dangling and no-identification conditions are
    satisfied.
    Note that these cospans in column are partial monogamous by definition
    of rewrite rules and traced boundary complements.
    Note that these cospans are similarly defined to those above; but the legs
    are different; we name these as
    \begin{gather*}
        \iltikzfig{strings/category/f-0-2}[box=\hat{l},colour=white,cod1=\listvar{i},cod2=\listvar{j}]
        \coloneqq
        \cospan{\varepsilon}{L}{\listvar{ij}}
        \quad
        \iltikzfig{strings/category/f-0-2}[box=\hat{g},colour=white,cod1=\listvar{m},cod2=\listvar{n}]
        \coloneqq
        \cospan{\varepsilon}{G}{\listvar{mn}}
        \\
        \iltikzfig{strings/category/f-2-2}[box=\hat{c},colour=white,dom1=\listvar{i},dom2=\listvar{j},cod1=\listvar{m},cod2=\listvar{n}]
        \coloneqq
        \cospan{\listvar{ij}}[[c_1, c_2]]{C}[[d_1, d_2]]{\listvar{mn}}
    \end{gather*}
    Using the compact closed structure of \(\cspdhyp\), we have the following:
    \begin{gather*}
        \iltikzfig{strings/category/f}[box=g,colour=white,dom=\listvar{m},cod=\listvar{n}]
        =
        \iltikzfig{graphs/dpo/g-bent}
        \qquad\quad
        \iltikzfig{strings/category/f-2-2}[box=\hat{c},colour=white,dom1=\listvar{i},dom2=\listvar{j},cod1=\listvar{m},cod2=\listvar{n}]
        =
        \iltikzfig{graphs/dpo/cprime-as-c}
        \qquad\quad
        \iltikzfig{strings/category/f-0-2}[box=\hat{l},colour=white,cod1=\listvar{i},cod2=\listvar{j}]
        =
        \iltikzfig{strings/compact-closed/f-bent-input}[box=l,colour=white,cod=\listvar{i},dom=\listvar{j}]
    \end{gather*}


    Since \(G\) is the pushout of \(
    L \xleftarrow{[a_1, a_2]} \listvar{ij} \xrightarrow{[c_1, c_2]} C
    \) and pushout is cospan composition, we also have that \(
    \iltikzfig{strings/category/f-0-2}[box=\hat{g},colour=white,cod1=\listvar{m},cod2=\listvar{n}]
    =
    \iltikzfig{graphs/dpo/lctilde}
    \).
    Putting this all together we can show that
    \begin{gather*}
        \iltikzfig{strings/category/f}[box=g,colour=white,dom=\listvar{m},cod=\listvar{n}]
        =
        \iltikzfig{graphs/dpo/g-bent}
        =
        \iltikzfig{graphs/dpo/l-c-bent}
        =
        \iltikzfig{graphs/dpo/l-c-bent-1}
        =
        \iltikzfig{graphs/dpo/lc-bent-2}
        =
        \iltikzfig{strings/rewriting/rewrite-l}[dom=\listvar{m},cod=\listvar{n}]
    \end{gather*}
    Since the `loop' is constructed in the same manner as the canonical trace on
    \(\cspdhyp\) (and is therefore identical in the graphical notation), this is a
    term in the form of \cref{lem:traced-decomposition}.
\end{proof}

\begin{lemma}\label{lem:switch-interfaces}
    \todo[inline]{Lemma that we can show swapping interface sides with cups and caps}
\end{lemma}
\begin{proof}
    \todo[inline]{The proof}
\end{proof}

We need to show that term rewriting with a set of rules \(\mcr\)
coincides with graph rewriting on the hypergraph interpretations of these rules.

\begin{theorem}\label{thm:traced-rewriting}
    For a rewrite rule \(\rrule{
        \iltikzfig{strings/category/f}[box=l,colour=white]
    }{
        \iltikzfig{strings/category/f}[box=r,colour=white]
    }\) in \(
    \stmcsigmac
    \), \(
    \iltikzfig{strings/category/f}[box=g,colour=white]
    \rewrite[\rrule{
            \iltikzfig{strings/category/f}[box=l,colour=white]
        }{
            \iltikzfig{strings/category/f}[box=r,colour=white]
        }]
    \iltikzfig{strings/category/f}[box=h,colour=white]
    \) if and only if \(
    \termandfrobtohypsigmac[
        \foldinterfaces[
            \tracedtosymandfrobsigmac[
                \iltikzfig{strings/category/f}[box=g,colour=white]
            ]
        ]
    ]
    \grewrite[
        \termandfrobtohypsigmac[
            \rrule{
                \tracedtosymandfrobsigmac[
                    \iltikzfig{strings/category/f}[box=l,colour=white]
                ]
            }{
                \tracedtosymandfrobsigmac[
                    \iltikzfig{strings/category/f}[box=r,colour=white]
                ]
            }
        ]
    ]
    \termandfrobtohypsigmac[
        \foldinterfaces[
            \iltikzfig{strings/category/f}[box=g,colour=white]
        ]
    ]\).
\end{theorem}
\begin{proof}
    First the \((\Rightarrow)\) direction.
    If \(
    \iltikzfig{strings/category/f}[box=g,colour=white]
    \rewrite[\mcr]
    \iltikzfig{strings/category/f}[box=h,colour=white]
    \) then we have \(
    \iltikzfig{strings/category/f}[box=g,colour=white]
    =
    \iltikzfig{strings/rewriting/rewrite-l}
    \) and \(
    \iltikzfig{strings/rewriting/rewrite-r}
    =
    \iltikzfig{strings/category/f}[box=h,colour=white]
    \); we must derive the DPO diagram in \(\hypsigmac\).

    Define the following cospans:
    \begin{alignat*}{3}
        \cospan{\varepsilon}{L}{\listvar{ij}}
         & :=
        \termandfrobtohypsigmac[
            \foldinterfaces[
                \tracedtosymandfrobsigmac[
                    \iltikzfig{strings/category/f}[box=l,colour=white]
                ]
            ]
        ]
         &    & =
        \termandfrobtohypsigmac[
            \iltikzfig{strings/rewriting/l-folded}
        ]
        \\
        \cospan{\varepsilon}{R}{\listvar{ij}}
         & :=
        \termandfrobtohypsigmac[
            \foldinterfaces[
                \tracedtosymandfrobsigmac[
                    \iltikzfig{strings/category/f}[box=r,colour=white]
                ]
            ]
        ]
         &    & =
        \termandfrobtohypsigmac[\iltikzfig{strings/rewriting/r-folded}]
        \\
        \cospan{\varepsilon}{G}{\listvar{mn}}
         & :=
        \termandfrobtohypsigmac[
            \foldinterfaces[
                \tracedtosymandfrobsigmac[
                    \iltikzfig{strings/category/f}[box=f,colour=white]
                ]
            ]
        ]
         &    & =
        \termandfrobtohypsigmac[\iltikzfig{strings/rewriting/lc-folded}]
        \\
        \cospan{\varepsilon}{H}{\listvar{mn}}
         & :=
        \termandfrobtohypsigmac[
            \foldinterfaces[
                \tracedtosymandfrobsigmac[
                    \iltikzfig{strings/category/f}[box=h,colour=white]
                ]
            ]
        ]
         &    & =
        \termandfrobtohypsigmac[\iltikzfig{strings/rewriting/rc-folded}]
        \\
        \cospan{\listvar{ij}}{C}{\listvar{mn}}
         & :=
        \termandfrobtohypsigmac[\iltikzfig{strings/rewriting/c-folded}]
         &    &
    \end{alignat*}

    By functoriality, we have that \(
    \termandfrobtohypsigmac[
        \foldinterfaces[
            \tracedtosymandfrobsigmac[
                \iltikzfig{strings/category/f}[box=f,colour=white]
            ]
        ]
    ]
    =
    \termandfrobtohypsigmac[\iltikzfig{strings/rewriting/l-folded}]
    \seq
    \termandfrobtohypsigmac[\iltikzfig{strings/rewriting/c-folded}]
    \), i.e.\ \(
    \cospan{\varepsilon}{G}{\listvar{mn}} =
    \cospan{\varepsilon}{L}{\listvar{ij}}
    \seq
    \cospan{\listvar{ij}}{C}{\listvar{mn}}.
    \).
    Cospan composition is pushout, so \(\cospan{L}{G}{C}\) is a pushout.
    Using the same reasoning, \(\cospan{R}{G}{C}\) is also a pushout; this
    gives us the DPO diagram.
    All that remains is to check that the aforementioned pushouts are traced
    boundary complements; this follows by inspecting components.

    Now for the \(\ifdir\) direction: we assume that we have a traced DPO
    rewrite \(
    \termandfrobtohypsigmac[
        \foldinterfaces[
            \tracedtosymandfrobsigmac[
                \iltikzfig{strings/category/f}[box=g,colour=white]
            ]
        ]
    ]
    \grewrite[
        \termandfrobtohypsigmac[
            \rrule{
                \tracedtosymandfrobsigmac[
                    \iltikzfig{strings/category/f}[box=l,colour=white]
                ]
            }{
                \tracedtosymandfrobsigmac[
                    \iltikzfig{strings/category/f}[box=r,colour=white]
                ]
            }
        ]
    ]
    \termandfrobtohypsigmac[
        \foldinterfaces[
            \iltikzfig{strings/category/f}[box=g,colour=white]
        ]
    ]
    \).
    This means there exists \(
    \cospan{\varepsilon}{L}{\listvar{ij}},
    \cospan{\varepsilon}{R}{\listvar{ij}},
    \cospan{\listvar{ij}}{C}{\listvar{mn}}
    \) as defined above such that the DPO diagram commutes and
    \(\listvar{ij} \to C \to G\) is a traced boundary complement;
    we must show that \(
    \iltikzfig{strings/category/f}[box=g,colour=white]
    =
    \iltikzfig{strings/rewriting/rewrite-l}
    \) and \(
    \iltikzfig{strings/category/f}[box=h,colour=white]
    =
    \iltikzfig{strings/rewriting/rewrite-r}
    \).

    As cospan composition is pushout, we have that \(
    \cospan{\varepsilon}{G}{\listvar{mn}} =
    \cospan{\varepsilon}{L}{\listvar{ij}} \seq
    \cospan{\listvar{ij}}[[c_1,c_2]]{C}[d_1,d_2]{\listvar{mn}}
    \).

    Note that \(\cospan{\listvar{ij}}{C}{\listvar{mn}}\) is \emph{not} in the
    image of \(
    \termandfrobtohypsigmac \circ \tracedtosymandfrobsigmac
    \) as it is not necessarily partial monogamous.
    For now, let \(
    \iltikzfig{strings/category/f-2-2}[box=c^\prime, colour=white,dom1=\listvar{i},dom2=\listvar{j},cod1=\listvar{m},cod2=\listvar{n}]
    \) be the term in \(\smcsigmac + \frobc\) such that \(
    \termandfrobtohypsigmac[
        \iltikzfig{strings/category/f-2-2}[box=c^\prime, colour=white]
    ]
    =
    \cospan{\listvar{ij}}[c_1,c_2]{C}[d_1,d_2]{\listvar{mn}}
    \), which exists as \(\termandfrobtohypsigmac\) is full.
    As we have that \(
    \termandfrobtohypsigmac[
        \foldinterfaces[
            \iltikzfig{strings/category/f}[box=g,colour=white]
        ]
    ]
    =
    \termandfrobtohypsigmac[
        \foldinterfaces[
            \iltikzfig{strings/category/f}[box=l,colour=white]
        ]
    ]
    \seq
    \termandfrobtohypsigmac[
        \iltikzfig{strings/category/f-2-2}[box=c^\prime, colour=white]
    ]
    \), we also have by fullness that \(
    \iltikzfig{strings/rewriting/g-folded-box}
    =
    \iltikzfig{strings/rewriting/lc}
    \) in \(\smcsigmac + \frobc\).

    Although \(\cospan{\listvar{ij}}{C}{\listvar{mn}}\) is not partial
    monogamous, \(\cospan{\listvar{jm}}[c_2,d_1]{C}[c_1,d_2]{\listvar{in}}\)
    \emph{is} because \(\listvar{ij} \to C \to G\) is a traced boundary
    complement.
    Let \(
    \iltikzfig{strings/category/f-2-2}[box=c, colour=white,dom1=\listvar{j},dom2=\listvar{m},cod1=\listvar{i},cod2=\listvar{n}]
    \)  be the term in \(\smcsigmac + \frobc\) such that \(
    \termandfrobtohypsigmac[
        \iltikzfig{strings/category/f-2-2}[box=c, colour=white]
    ]
    =
    \cospan{\listvar{jm}}[c_2,d_1]{C}[c_1,d_2]{\listvar{in}}
    \).
    By \cref{lem:switch-interfaces}, we have that if \(
    \termandfrobtohypsigmac[
        \iltikzfig{strings/category/f-2-2}[box=c^\prime, colour=white]
    ]
    =
    \cospan{\listvar{ij}}[c_1,c_2]{C}[d_1,d_2]{\listvar{mn}}
    \) and \(
    \termandfrobtohypsigmac[
        \iltikzfig{strings/category/f-2-2}[box=c^\prime, colour=white]
    ]
    =
    \cospan{\listvar{ij}}[c_1,c_2]{C}[d_1,d_2]{\listvar{mn}}
    \), then \(
    \iltikzfig{strings/rewriting/c-folded}
    =
    \cospan{\listvar{ij}}[c_1,c_2]{C}[d_1,d_2]{\listvar{mn}}
    \).
    Subsequently, \(
    \iltikzfig{strings/rewriting/g-folded-box}
    =
    \iltikzfig{strings/rewriting/lc-folded}
    =
    \iltikzfig{strings/rewriting/lc-folded-shifted}
    \).

    Returning to the perspective from \(\stmcsigmac\), we have that \(
    \foldinterfaces[
        \iltikzfig{strings/category/f}[box=g,colour=white]
    ]
    =
    \iltikzfig{strings/rewriting/lc-folded-shifted}
    \).
    Since \(
    \termandfrobtohypsigmac[
        \iltikzfig{strings/category/f-2-2}[box=c^\prime, colour=white]
    ]
    \) is partial monogamous, \(
    \iltikzfig{strings/category/f-2-2}[box=c^\prime, colour=white]
    \) is in \(\stmcsigma\) and we can conclude that \(
    \iltikzfig{strings/category/f}[box=g,colour=white]
    =
    \iltikzfig{strings/rewriting/rewrite-l}
    \), completing the proof.

    The same procedure holds for rewriting from the other direction.
\end{proof}
\section{Rewriting with commutative comonoid structure}