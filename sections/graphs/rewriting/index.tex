\chapter{Graph rewriting}

The results of the previous chapter have given us a combinatorial setting in
which string diagrams equal by topological deformations or by the equations of
commutative comonoids are translated into isomorphic cospans of hypergraphs.
This already gives us a straightforward way to interpret these structures
computationally.

However, it is very rare that we will only be working modulo the equations of
traced comonoids; the equations of some \emph{monoidal theory} will also be in
play.
Since an equation \(
\iltikzfig{strings/category/f}[box=f,colour=white]
=
\iltikzfig{strings/category/f}[box=g,colour=white]
\) may actually change the components of the term, there is no reason for the
interpretations \(
\termandfrobtohypsigmac[
    \tracedandcomonoidtofrobsigmac[
        \iltikzfig{strings/category/f}[box=f,colour=white]
    ]
]
\) and \(
\termandfrobtohypsigmac[
    \tracedandcomonoidtofrobsigmac[
        \iltikzfig{strings/category/f}[box=g,colour=white]
    ]
]
\) to be isomorphic.
Instead, to reason about cospans of hypergraphs modulo the equations of some
monoidal theory the graphs must be \emph{rewritten}.

Traditional term rewriting is inherently one-dimensional, which can be somewhat
restrictive as it enforces that each generator must have a coarity of \(1\).
On the other hand, string diagrams are an example of \emph{higher dimensional}
rewriting, which its roots in Burroni's work on
polygraphs~\cite{burroni1993higherdimensional}.
This higher-dimensional approach briefly entered the world of digital
circuits in Lafont's work on Boolean circuits~\cite{lafont2003algebraic}, which
used a rudimentary form of graphical syntactic rewriting.
However, in this approach even the axioms of SMCs needed to be applied
explicitly, greatly hampering the tractability of the system.

It is only more recently that rewriting string diagrams modulo the axioms of
SMCs using \emph{graph rewriting} has been studied, starting with the
aforementioned \emph{string graphs} of Dixon and
Kissinger~\cite{dixon2010open,kissinger2012pictures,dixon2013opengraphs}.
This only considered rewriting in the presence of a trace rather than
considering any additional structure, but nevertheless was successfully
implemented in a proof assistant called
Quantomatic~\cite{kissinger2015quantomatic}.
Our work continues to follow that of Bonchi et
al~\cite{bonchi2022string,bonchi2022stringa}, which has several advantages
over string graphs.
Unlike string graphs, the category of hypergraphs is
\emph{adhesive}~\cite{lack2004adhesive} which affords it some nice rewriting
properties.
Moreover, rewriting modulo Frobenius structure (or even just monoid or comonoid
structure) can reveal `hidden' rewrites that may not be seen with the naked eye.
In this chapter, we will explore how to extend this work to the traced and
traced comonoid case.

\begin{remark}
    The content of this section is a revised version
    of~\cite[Sec. 5]{ghica2023rewriting}.
\end{remark}

\section{Double pushout rewriting}

As a backdrop, let us first consider how equations in a monoidal theory are
applied to terms without any sort of special graph interpretation.

\begin{definition}[Term rewriting]\label{def:term-rewriting}
    A \emph{rewriting system} \(\mcr\) for a traced PROP \(\stmcsigmac\)
    consists of a set of \emph{rewrite rules} \(
        \rrule{
            \iltikzfig{strings/category/f}[box=l,colour=white,dom=i,cod=j]
        }{
            \iltikzfig{strings/category/f}[box=r,colour=white,dom=i,cod=j]
        }
    \).
    Given terms \(
        \iltikzfig{strings/category/f}[box=g,colour=white,dom=m,cod=n]
    \) and \(
        \iltikzfig{strings/category/f}[box=h,colour=white,dom=m,cod=n]
    \) in \(\stmc{\generators}\) we write \(
        \iltikzfig{strings/category/f}[box=g,colour=white]
        \rewrite[\mcr]
        \iltikzfig{strings/category/f}[box=h,colour=white]
    \) if there exists rewrite rule \((
        \iltikzfig{strings/category/f}[box=l,colour=white,dom=i,cod=j],
        \iltikzfig{strings/category/f}[box=r,colour=white,dom=i,cod=j]
    )\) in \(\mcr\) and \(
        \iltikzfig{strings/category/f-2-2}[box=c,colour=white,dom1=j,dom2=m,cod1=i,cod2=n]
    \) in \(\stmc{\Sigma}\) such that \(
        \iltikzfig{strings/category/f}[box=g,colour=white]
        =
        \iltikzfig{strings/rewriting/rewrite-l}
    \) and \(
        \iltikzfig{strings/category/f}[box=h,colour=white]
        =
        \iltikzfig{strings/rewriting/rewrite-r}
    \) by axioms of STMCs.
\end{definition}
\section{Pushout complements}

\index{pushout complement}
While computing the rewritten graph from a context is a deterministic procedure,
finding the pushout complement specifying said context
is a little more subtle.

\begin{definition}
    \index{matching}
    A morphism \(i+j \to L \to G\) is called a \emph{matching} if it
    has at least one pushout complement.
\end{definition}

If there is no pushout complement, there is no possible rewrite, so it is
important to know when one exists.
Fortunately, there are two well-known conditions for existence of pushout
complements when rewriting with hypergraphs.
The first ensures that all the sources and targets of a hyperedge are present
in a candidate context.

\begin{definition}[No-dangling-hyperedges condition~\cite{corradini1997algebraic}, Prop.\ 3.3.4]
    \index{no-dangling-hyperedges condition}
    Given morphisms \(i+j \xrightarrow{a} L \xrightarrow{f} G\) in
    \(\hypsigma\), they satisfy the \emph{no-dangling} condition if, for every
    hyperedge not in
    the image of \(f\), each of its source and target vertices is either not in
    the image of \(f\) or are in the image of \(f \circ a\).
\end{definition}

\begin{example}
    The following pair of morphisms does not satisfy the no-dangling-hyperedges
    condition.
    \[
        \iltikzfig{graphs/dpo/no-dangling/k}
        \xrightarrow{a}
        \iltikzfig{graphs/dpo/no-dangling/l}
        \xrightarrow{f}
        \iltikzfig{graphs/dpo/no-dangling/g}
    \]
    To obtain the pushout complement we `cut out' any vertices in the
    rightmost graph which are in the image of \(f\) but not the image of
    \(f \circ a\), as the latter are the interfaces of the rule.
    However, if we cut out the vertices labelled \(2\) and \(3\), the edge
    \(e_2\) will be left with `dangling' tentacles connected to no vertices, a
    malformed hypergraph.
    \begin{center}
        \begin{tikzcd}
            \iltikzfig{graphs/dpo/no-dangling/l}
            \arrow{d}{f}
            &
            \iltikzfig{graphs/dpo/no-dangling/k}
            \arrow{l}{a}
            \arrow{d}
            \\
            \iltikzfig{graphs/dpo/no-dangling/g}
            &
            \iltikzfig{graphs/dpo/no-dangling/c}
            \arrow{l}
        \end{tikzcd}
    \end{center}
\end{example}

The second condition enforces that merging of vertices is well-defined.

\begin{definition}[No-identification condition~\cite{corradini1997algebraic}, Prop.\ 3.3.4]
    \index{no-identification condition}
    Given morphisms \(i+j \xrightarrow{a} L \xrightarrow{f} G\) in
    \(\hypsigma\), they satisfy the \emph{no-identification} condition if any
    two distinct elements merged by \(f\) are also in the image of \(f \circ a\).
\end{definition}

\begin{example}
    The following does not satisfy the no-identification
    condition.
    \[
        \iltikzfig{graphs/dpo/no-identification/k}
        \xrightarrow{a}
        \iltikzfig{graphs/dpo/no-identification/l}
        \xrightarrow{f}
        \iltikzfig{graphs/dpo/no-identification/g}
    \]
    When trying to construct a pushout complement, the edge \(e_2\) will be
    removed.
    However, since vertices \(2\) and \(3\) are not mapped from the rule
    interfaces, there is no reason that a pushout would glue them together so
    that they are merged in the final graph.
    Therefore no pushout complement can exist.
    \begin{center}
        \begin{tikzcd}
            \iltikzfig{graphs/dpo/no-identification/l}
            \arrow{d}{f}
            &
            \iltikzfig{graphs/dpo/no-identification/k}
            \arrow{l}{a}
            \arrow{d}
            \\
            \iltikzfig{graphs/dpo/no-identification/g}
            &
            \iltikzfig{graphs/dpo/no-identification/c}
            \arrow{l}
        \end{tikzcd}
    \end{center}
\end{example}

With these two conditions, we can establish when pushout complements exist for
a pair of hypergraph homomorphisms.
If there is a pushout complement, then there is an opportunity for a rewrite.

\begin{proposition}[\cite{corradini1997algebraic}, Prop.\ 3.3.4]
    \label{prop:pushout-complement}
    The morphisms \(i+j \to L \to G \in \hypsigma\) have at least one pushout
    complement if and only if they satisfy the no-dangling and no-identification
    conditions.
\end{proposition}

It is all very well knowing if there is at least one pushout complement, but
what about when there is \emph{exactly one} pushout complement?
When this is the case, the rewrite is uniquely specified for a given rule and
matching.
To answer this question, we must examine a class of categories to which
\(\hypsigma\) belongs, known as \emph{adhesive} categories.
One can think of these as categories in which graph rewriting `plays nicely'.

To define what an adhesive category is, we must first define a special kind of
pushout that interacts in a particular way with other pushouts and pullbacks.

\begin{definition}[van Kampen square {\cite[Def.\ 2.1]{lack2005adhesive}}]
    \index{van Kampen square}
    \index{VK square}
    Let there be a commutative cube as drawn below.
    \begin{center}
        \begin{tikzcd}
            G \arrow{rrr} \arrow{ddd} &
            &
            &
            H \arrow{ddd} \\
            &
            E \arrow{ul} \arrow{r} \arrow{d} &
            F \arrow{ur} \arrow{d} &
            \\
            &
            A \arrow{dl} \arrow{r} &
            B \arrow{dr} &
            \\
            C \arrow{rrr} &
            &
            &
            D
        \end{tikzcd}
    \end{center}
    The bottom face of the cube (\(ABCD\)) is called a
    \emph{van Kampen (VK) square} if it is a pushout and, when the back and
    left faces (\(EFAB\) and \(GECA\)) are pullbacks, the front and right faces
    (\(GHCD\) and \(FHBD\)) are pullbacks if and only if the top face (\(GHEF\))
    is a pushout.
\end{definition}

In an adhesive category, VK squares arise when performing pushouts of
\emph{monomorphisms}.
This is important for graph rewriting because monomorphisms in categories of
graphs generally correspond to \emph{embeddings}; graph homomorphisms where
subgraphs can be `cut out' of a graph without causing the rest of the graph to
become degenerate.

\begin{definition}
    \index{pushout!along a monomorphism}
    Given a span \(\spann{A}[f]{B}[g]{C}\) with a pushout \(\cospan{B}{D}{C}\),
    the pushout is called a \emph{pushout along a monomorphism} if \(f\) or
    \(g\) is a monomorphism.
\end{definition}

\begin{definition}[Adhesive category {\cite[Def.\ 3.1]{lack2005adhesive}}]
    \index{adhesive category|textbf}
    \index{category!adhesive}
    A category is \emph{adhesive} if
    \begin{itemize}
        \item it has pushouts along monomorphisms;
        \item it has pullbacks; and
        \item pushouts along monomorphisms are VK squares.
    \end{itemize}
\end{definition}

The definition of van Kampen square and subsequently adhesive categories may
look a bit confusing to the uninitiated.
Succinctly, objects in an adhesive category can
be `split apart' and `glued together' by using pushouts and pullbacks.

\begin{example}
    A natural example of an adhesive category is \(\set\), which has pushouts
    and pullbacks.
    To get some idea how the van Kampen condition holds, consider the
    following commutative cube in \(\set\), adapted from
    \cite[Sec.\ 4.3]{kissinger2012pictures}.
    \begin{center}
        \begin{tikzcd}
            A^\prime
            \arrow{rrr}
            \arrow{ddd}{f_A}
            &
            &
            &
            X
            \arrow{ddd}{f}
            \\
            &
            A^\prime \cap B^\prime
            \arrow{r}
            \arrow{d}
            \arrow{ul}
            &
            B^\prime
            \arrow{ur}
            \arrow{d}{f_B}
            &
            \\
            &
            A \cap B
            \arrow{r}
            \arrow{dl}
            &
            B
            \arrow{dr}
            &
            \\
            A
            \arrow{rrr}
            &
            &
            &
            A \cup B
        \end{tikzcd}
    \end{center}
    The bottom face is a pushout, and since \(A \cup B \to A\) and
    \(A \cup B \to B\) are monomorphisms, this is a pushout along a
    monomorphism.
    Furthermore, the left and back faces are pullbacks because \(f_A\) and
    \(f_B\) agree on the intersection of \(A^\prime\) and \(B^\prime\).

    Now we must show that the front and right faces are pullbacks if and only if
    the top face is a pushout.
    For the front and right faces to be pullbacks, \(f\) must restrict to
    \(f_A\) and \(f_B\) along \(A\) and \(B\).
    This can only be the case if and only if \(X = A^\prime \cap B^\prime\), in
    which case the top square must be a pushout.
\end{example}

Proving the necessary van Kampen condition might be tricky.
Fortunately, adhesivity is preserved by several categorical constructions, so
by using \(\set\) as a base it is straightforward to show that more complicated
categories are also adhesive.

\begin{proposition}[\cite{lack2005adhesive}, Prop.\ 3.5]
    For an adhesive category \(\mcc\) and object \(C\) of \(\mcc\),
    \(\mcc \slice C\) is adhesive.
    Given another category \(\mcx\), \([\mcx, \mcc]\) is also adhesive.
\end{proposition}

\begin{corollary}
    \(\hypsigma\) and \(\hypsigmac\) are adhesive.
\end{corollary}
\begin{proof}
    \(\hypsigma\) and \(\hypsigmac\) are defined as the slice of a functor
    category over \(\set\), so they are adhesive.
\end{proof}

The key property adhesive categories enjoy is that, for certain DPO rules, a
pushout complement is \emph{uniquely} defined for a given matching.

\begin{definition}[Left-linear rules]
    A DPO rule \(\spann{L}[f]{i+j}{R}\) is called \emph{left-linear} if \(f\)
    is mono.
\end{definition}

\begin{theorem}[\cite{lack2005adhesive}, Lem.\ 4.5]
    In an adhesive category, if a pushout complement exists for morphisms
    \(I \xrightarrow{m} L \to G\) and \(m\) is a monomorphism, then it is unique
    up to isomorphism.
\end{theorem}
\begin{proof}
    The proof relies on several non-trivial lemmas that hold in adhesive
    categories in addition to some other results about pushouts.
    We refer the interested reader to
    \cite[Lems. 4.3.6 - 4.3.9]{kissinger2012pictures} for the grisly details.
\end{proof}

However, there may be useful rewrite rules which are \emph{not} left-linear.

\begin{example}\label{ex:non-left-linear}
    Consider the following (reasonable) rule.
    \[
        \rrule{
            \iltikzfig{strings/category/identity}
        }{
            \iltikzfig{strings/category/generator}[box=e]
        }
        \qquad
        \iltikzfig{graphs/dpo/non-left-linear/l}
        \leftarrow
        \iltikzfig{graphs/dpo/non-left-linear/k}
        \to
        \iltikzfig{graphs/dpo/non-left-linear/r}
    \].
    Now consider applying the above rule to the term
    \(\iltikzfig{graphs/dpo/non-left-linear/g-term}\) using the following pair
    of morphisms:
    \[
        \iltikzfig{graphs/dpo/non-left-linear/k}
        \to
        \iltikzfig{graphs/dpo/non-left-linear/l}
        \to
        \iltikzfig{graphs/dpo/non-left-linear/g}
    \]
    This matching yields the following pushout complements and rewrites:
    \begin{gather*}
        \scalebox{0.9}{\iltikzfig{graphs/dpo/non-left-linear/c1}}
        \to
        \scalebox{0.9}{\iltikzfig{graphs/dpo/non-left-linear/h1}}
        \\[0.5em]
        \scalebox{0.9}{\iltikzfig{graphs/dpo/non-left-linear/c2}}
        \to
        \scalebox{0.9}{\iltikzfig{graphs/dpo/non-left-linear/h2}}
        \\[0.5em]
        \scalebox{0.9}{\iltikzfig{graphs/dpo/non-left-linear/c3}}
        \to
        \scalebox{0.9}{\iltikzfig{graphs/dpo/non-left-linear/h3}}
        \,\,
        \scalebox{0.9}{\iltikzfig{graphs/dpo/non-left-linear/c4}}
        \to
        \scalebox{0.9}{\iltikzfig{graphs/dpo/non-left-linear/h4}}
        \\
        \scalebox{0.9}{\iltikzfig{graphs/dpo/non-left-linear/c5}}
        \to
        \scalebox{0.9}{\iltikzfig{graphs/dpo/non-left-linear/h5}}
    \end{gather*}
\end{example}

One might think this is undesirable, but these multiple rewrites actually arise
due to the presence of the Frobenius algebra.

\begin{example}
    Each of the five complements and rewrites in \cref{ex:non-left-linear}
    corresponds to a valid application of equations on terms, perhaps modulo
    Frobenius equations.
    The first complement \(
    \scalebox{0.9}{\iltikzfig{graphs/dpo/non-left-linear/h1}}
    \) is the `obvious' one:
    \begin{gather*}
        \iltikzfig{graphs/dpo/non-left-linear/g-term}
        =
        \iltikzfig{graphs/dpo/non-left-linear/g-term-rewrite-1-2}
    \end{gather*}
    The complement \(
    \scalebox{0.9}{\iltikzfig{graphs/dpo/non-left-linear/h2}}
    \) uses the compact closed structure:
    \begin{gather*}
        \iltikzfig{graphs/dpo/non-left-linear/g-term}
        =
        \iltikzfig{graphs/dpo/non-left-linear/g-term-rewrite-2-0}
        =
        \iltikzfig{graphs/dpo/non-left-linear/g-term-rewrite-2-3}
    \end{gather*}
    The complement \(
    \scalebox{0.9}{\iltikzfig{graphs/dpo/non-left-linear/h3}}
    \) uses the unitality of the monoid:
    \begin{gather*}
        \iltikzfig{graphs/dpo/non-left-linear/g-term}
        =
        \iltikzfig{graphs/dpo/non-left-linear/g-term-rewrite-3-0}
        =
        \iltikzfig{graphs/dpo/non-left-linear/g-term-rewrite-3-3}
    \end{gather*}
    The complement \(
    \scalebox{0.9}{\iltikzfig{graphs/dpo/non-left-linear/h4}}
    \) uses the counitality of the comonoid:
    \begin{gather*}
        \iltikzfig{graphs/dpo/non-left-linear/g-term}
        =
        \iltikzfig{graphs/dpo/non-left-linear/g-term-rewrite-4-0}
        =
        \iltikzfig{graphs/dpo/non-left-linear/g-term-rewrite-4-3}
    \end{gather*}
    The complement \(
    \scalebox{0.9}{\iltikzfig{graphs/dpo/non-left-linear/h5}}
    \) uses the Frobenius equations:
    \begin{gather*}
        \iltikzfig{graphs/dpo/non-left-linear/g-term}
        =
        \iltikzfig{graphs/dpo/non-left-linear/g-term-rewrite-5-0}
        =
        \iltikzfig{graphs/dpo/non-left-linear/g-term-rewrite-5-1}
        =
        \iltikzfig{graphs/dpo/non-left-linear/g-term-rewrite-5-2}
        =
        \\
        \iltikzfig{graphs/dpo/non-left-linear/g-term-rewrite-5-3}
        =
        \iltikzfig{graphs/dpo/non-left-linear/g-term-rewrite-5-4}
        =
        \iltikzfig{graphs/dpo/non-left-linear/g-term-rewrite-5-7}
    \end{gather*}
\end{example}

The problem of finding each possible pushout complement has already been
tackled for hypergraphs~\cite{heumuller2011construction}; they can be enumerated
as quotients of an `exploded' context.

\begin{definition}[{\cite[Const.\ 1]{heumuller2011construction}}]
    \index{exploded context}
    For morphisms \(i+j \to L \xrightarrow{f} G\) in \(\hypsigma\),
    their \emph{exploded context} is the graph
    \(i + j + \tilde{G}\) where \(\tilde{G}\) constructed as follows:
    \begin{enumerate}
        \item for each vertex \(v \in G\) not in the image of \(f\), add one
              vertex to \(\tilde{G}\);
        \item for each hyperedge \(e \in G\) not in the image of \(f\), add one
              hyperedge to \(\tilde{G}\);
        \item for each hyperedge \(e \in \tilde{G}\), let the \(i\)-th source
              \(\tilde{s_i}(e)\) be defined as \(s_i(h)\) if
              \(s_i(h) \in \tilde{G}\) or a new, fresh vertex otherwise;
        \item repeat the above for the targets.
    \end{enumerate}
\end{definition}

Pushout complements can then be computed as quotients of this exploded
context.

\begin{proposition}[\cite{heumuller2011construction} (Props. 3-4), \cite{bonchi2022string}]
    For a pair of morphisms \(i+j \to L \to G\) in \(\hypsigma\), let
    \(i + j + \tilde{G}\) be its exploded context.
    Define a map \(\morph{q}{i + j + \tilde{G}}{G}\) sending elements in
    \(\tilde{G}\) from \(G\) to themselves, and sending vertices from
    \(i + j\) to their image under \(i + j \to L \to G\).
    Then a pushout complement \(i + j \to C \to G\) is valid if and only the
    context \(C\) is a quotient on the exploded context that only identifies
    vertices in the image of \(q^{-1}(v)\) for each vertex \(v \in G\).
\end{proposition}

Given a DPO rule and matching, we can enumerate all pushout complements; each of
these corresponds to a valid rewrite in a Frobenius setting.

\begin{notation}
    For a rule \(
    \rrule{
        \iltikzfig{strings/category/f}[box=l,colour=white,dom=i,cod=j]
    }{
        \iltikzfig{strings/category/f}[box=r,colour=white,dom=i,cod=j]
    } \in \smcsigma + \frob
    \), its DPO rule is defined as \(
    \termandfrobtohypsigma[
        \rrule{
            \iltikzfig{strings/category/f}[box=l,colour=white]
        }{
            \iltikzfig{strings/category/f}[box=r,colour=white]
        }
    ]
    \coloneqq
    \spann{
        \termandfrobtohypsigma[
            \foldinterfaces[
                \iltikzfig{strings/category/f}[box=l,colour=white]
            ]
        ]
    }{i+j}{
        \termandfrobtohypsigma[
            \foldinterfaces[
                \iltikzfig{strings/category/f}[box=r,colour=white]
            ]
        ]
    }
    \).
\end{notation}

\begin{theorem}[{\cite[Thm.\ 4.9]{bonchi2022string}}]
    For rule \(r \in \smcsigma + \frob\), we have that \(
    \iltikzfig{strings/category/f}[box=g,colour=white]
    \rewrite[r]
    \iltikzfig{strings/category/f}[box=h,colour=white]
    \) if and only if \(
    \termandfrobtohypsigma[
        \foldinterfaces[
            \iltikzfig{strings/category/f}[box=g,colour=white]
        ]
    ]
    \grewrite[\termandfrobtohypsigma[r]
    ]
    \termandfrobtohypsigma[
        \foldinterfaces[
            \iltikzfig{strings/category/f}[box=g,colour=white]
        ]
    ].\)
\end{theorem}

\subsection{Multicoloured rewriting}

The results generalise in the obvious way to the coloured setting.

\begin{notation}
    For a term rewrite rule \(
    \rrule{
        \iltikzfig{strings/category/f}[box=l,colour=white,dom=\listvar{i},cod=\listvar{j}]
    }{
        \iltikzfig{strings/category/f}[box=r,colour=white,dom=\listvar{i},cod=\listvar{j}]
    }
    \) in \(\smcsigmac + \frobc\), its interpretation as a DPO rule is defined
    as \[
        \termandfrobtohypsigmac[
            \rrule{
                \iltikzfig{strings/category/f}[box=l,colour=white]
            }{
                \iltikzfig{strings/category/f}[box=r,colour=white]
            }
        ]
        \coloneqq
        \spann{
            \termandfrobtohypsigmac[
                \foldinterfacesc[
                    \iltikzfig{strings/category/f}[box=l,colour=white]
                ]
            ]
        }{\listvar{ij}}{
            \termandfrobtohypsigmac[
                \foldinterfacesc[
                    \iltikzfig{strings/category/f}[box=r,colour=white]
                ]
            ]
        }
        .\]
\end{notation}

\begin{definition}[\cite{bonchi2022string}]
    \nomenclature{\(\foldinterfacesc\)}{`folding' \(C\)-coloured PROP morphism \(\smcsigmac + \frobc \to \smcsigmac + \frobc\)}
    Let \(\morph{\foldinterfacesc}{\smcsigmac + \frobc}{\smcsigmac + \frobc}\)
    be defined as having action \(
    \iltikzfig{strings/category/f}[box=f,colour=white,dom=\listvar{m},cod=\listvar{n}]
    \mapsto
    \iltikzfig{strings/rewriting/folding}[box=f,colour=white,dom=\listvar{m},cod=\listvar{n}]
    \).
\end{definition}

\begin{theorem}[{\cite[Prop.\ 4.10]{bonchi2022string}}]
    For rewrite rule \(r \in \smcsigmac + \frobc\), we have that \(
    \iltikzfig{strings/category/f}[box=g,colour=white]
    \rewrite[r]
    \iltikzfig{strings/category/f}[box=h,colour=white]
    \) if and only if \(
    \termandfrobtohypsigmac[
        \foldinterfacesc[
            \iltikzfig{strings/category/f}[box=g,colour=white]
        ]
    ]
    \grewrite[\termandfrobtohypsigmac[r]]
    \termandfrobtohypsigmac[
        \foldinterfacesc[
            \iltikzfig{strings/category/f}[box=g,colour=white]
        ]
    ].\)
\end{theorem}
\section{Rewriting with traced structure}

In the Frobenius setting, every pushout complement is a valid rewrite, but there
is no reason for the same to be the case for traced or traced comonoid
rewriting.
Bonchi et al showed in~\cite{bonchi2022stringa} that \emph{exactly one} pushout
complement corresponds to a valid rewrite in the symmetric monoidal case by
characterising it as a \emph{boundary complement}.

\begin{definition}[Boundary complement (\cite{bonchi2022stringa}, Def. 30)]
    For monogamous cospans \(
    \cospan{\listvar{i}}[a_1]{L}[a_2]{\listvar{j}}
    \) and \(
    \cospan{\listvar{m}}[b_1]{G}[b_2]{\listvar{n}}
    \) and a monomorphism \(\morph{f}{L}{G}\), a pushout complement as below
    \begin{center}
        \begin{tikzcd}[column sep=large]
            L \arrow[swap]{d}{f}
            &
            \listvar{ij}
            \arrow[swap]{l}{a := [a_1, a_2]}
            \arrow{d}{c := [c_1, c_2]}
            \\
            G
            \arrow["\urcorner"{anchor=center, pos=0.125}, draw=none]{ur}
            &
            C
            \arrow{l}{g}
            \\
            &
            \listvar{mn}
            \arrow{ul}{[b_1,b_2]}
            \arrow[swap]{u}{d := [d_1,d_2]}
        \end{tikzcd}
    \end{center}
    is called a \emph{boundary complement} if \(c_1\) and \(c_2\) are
    mono and \(
    \cospan{\listvar{jm}}[[c_2,d_1]]{C}[[d_2,c_1]]{\listvar{ni}}
    \) is a boundary monogamous cospan.
\end{definition}

What is particularly special about boundary complements is that for morphisms
\(\listvar{ij} \to L \to G\) there is always \emph{exactly one} pushout
complement which is also a boundary complement!

\begin{proposition}[\cite{bonchi2022stringa}, Prop. 31]
    When boundary complements exist in \(\hypsigmac\), they are unique.
\end{proposition}

For rewriting in a traced setting, boundary complements are too strong, so we
will weaken them to \emph{traced} boundary complements, replacing references to
monogamy with partial monogamy.

\begin{definition}[Traced boundary complement]
    \label{def:traced-boundary-complement}
    For partial monogamous cospans \(
    \cospan{\listvar{i}}[a_1]{L}[a_2]{\listvar{j}}
    \) and \(
    \cospan{\listvar{m}}[b_1]{G}[b_2]{\listvar{n}}
    \), a pushout complement as below
    \begin{center}
        \begin{tikzcd}[column sep=large]
            L \arrow[swap]{d}{f}
            &
            \listvar{ij}
            \arrow[swap]{l}{a := [a_1, a_2]}
            \arrow{d}{c := [c_1, c_2]}
            \\
            G
            \arrow["\urcorner"{anchor=center, pos=0.125}, draw=none]{ur}
            &
            C
            \arrow{l}{g}
            \\
            &
            \listvar{mn}
            \arrow{ul}{[b_1,b_2]}
            \arrow[swap]{u}{d := [d_1,d_2]}
        \end{tikzcd}
    \end{center}
    is called a \emph{traced boundary complement} if \(c_1\) and \(c_2\) are
    mono and \(
    \cospan{\listvar{jm}}[[c_2,d_1]]{C}[[d_2,c_1]]{\listvar{ni}}
    \) is a partial monogamous cospan.
\end{definition}

By restricting to traced boundary complements, DPO rewriting can be formulated
for terms in a traced setting.

\begin{definition}[Traced DPO]
    For morphisms \(G \leftarrow \listvar{mn}\) and \(H \leftarrow \listvar{mn}\) in
    \(\hypsigma\), there is a traced rewrite \(G \trgrewrite[\mcr] H\) if there
    exists a rule \(
    \spann{L}{\listvar{ij}}{G} \in \mcr
    \) and cospan \(
    \cospan{\listvar{ij}}{C}{\listvar{mn}} \in \hypsigma
    \) such that diagram in \cref{def:dpo-rewriting} commutes and \(\listvar{ij} \to C\)
    is a traced boundary complement.
\end{definition}

Traced boundary complements are not much more powerful than regular boundary
complements; \(c_1\) is still not permitted to merge vertices in the inputs and
the same for \(c_2\) and the outputs.
The real power of traced DPO, and what increases the number of pushout
complements, is the fact that the matching \(L \to G\) is no longer required to
be mono.

Since \(\cospan{\listvar{m}}{G}{\listvar{n}}\) is partial monogamous, vertices
in \(L\) can only be merged if their degrees sum to no more than \((1,1)\).
Merging vertices in this way corresponds to using the trace to find a match.

\begin{example}
    Consider the rule \(
    \rrule{
        \iltikzfig{graphs/dpo/traced-example/rule-lhs}
    }{
        \iltikzfig{graphs/dpo/traced-example/rule-rhs}
    }
    \) and the term \(
    \iltikzfig{graphs/dpo/traced-example/term}
    \), in which there is clearly an instance of the rule.
    The interpretation of this as a DPO derivation with a valid traced boundary
    complement is illustrated below.
    \begin{center}
        \includestandalone{figures/graphs/dpo/traced-example/rewrite}
    \end{center}
\end{example}

A key feature of rewriting modulo traced structure is the \emph{yanking} axiom,
which can lead to some non-obvious rewrites.

\begin{example}
    Consider the rule \(
    \rrule{
        \iltikzfig{graphs/dpo/split-loop/rule-lhs}
    }{
        \iltikzfig{graphs/dpo/split-loop/rule-rhs}
    }
    \).
    The interpretation of this as a DPO rule in a valid traced boundary
    complement is illustrated below.
    \begin{center}
        \includestandalone{figures/graphs/dpo/split-loop/rewrite}
    \end{center}
    This corresponds to a valid term rewrite:
    \[
        \iltikzfig{graphs/dpo/split-loop/derivation-1}
        =
        \iltikzfig{graphs/dpo/split-loop/derivation-2}
        =
        \iltikzfig{graphs/dpo/split-loop/derivation-3}
        =
        \iltikzfig{graphs/dpo/split-loop/derivation-4}
    \]

    Note that applying yanking is required in the term setting because
    the traced wire is flowing from right to left, whereas applying the rule
    requires all wires flowing left to right.
\end{example}

Use of yanking is also what can lead to multiple boundary complements, and hence
a choice in rewrites.

\begin{example}
    Consider the rule \(
    \rrule{
        \iltikzfig{graphs/dpo/non-unique/rule-lhs}
    }{
        \iltikzfig{graphs/dpo/non-unique/rule-rhs}
    }
    \).
    Below are two valid traced boundary complements involving a matching of this
    rule.

    \begin{center}
        \scalebox{0.95}{\includestandalone{figures/graphs/dpo/non-unique/rewrite-1}}
        \quad
        \scalebox{0.95}{\includestandalone{figures/graphs/dpo/non-unique/rewrite-2}}
    \end{center}

    These two derivations arise through yanking:
    \begin{gather*}
        \iltikzfig{graphs/dpo/non-unique/derivation-1}
        =
        \iltikzfig{graphs/dpo/non-unique/derivation-2}
        =
        \iltikzfig{graphs/dpo/non-unique/derivation-3a}
        =
        \iltikzfig{graphs/dpo/non-unique/derivation-4a}
        =
        \iltikzfig{graphs/dpo/non-unique/derivation-5a}
        \\
        \iltikzfig{graphs/dpo/non-unique/derivation-1}
        =
        \iltikzfig{graphs/dpo/non-unique/derivation-2}
        =
        \iltikzfig{graphs/dpo/non-unique/derivation-3b}
        =
        \iltikzfig{graphs/dpo/non-unique/derivation-4b}
        =
        \iltikzfig{graphs/dpo/non-unique/derivation-5b}
    \end{gather*}
\end{example}

There is another condition on graph rewriting modulo symmetric monoidal
structure in that the matching must be \emph{convex}: any path between vertices
must also be captured.
Luckily for us, this is not necessary in the traced case.

\begin{example}
    Consider the rule \(
    \rrule{
        \iltikzfig{graphs/dpo/convex/example-l}
    }{
        \iltikzfig{graphs/dpo/convex/example-r}
    }
    \) and the term \iltikzfig{graphs/dpo/convex/example-g}.
    Although it is not immediately obvious, there is in fact
    a matching of the former in the latter.
    Performing the DPO procedure yields the following:
    %
    \begin{gather*}
        \includestandalone{figures/graphs/dpo/convex/rewrite}
    \end{gather*}
    In a non-traced setting this is an invalid rule!
    However, it is possible with yanking.
    \begin{gather*}
        \iltikzfig{graphs/dpo/convex/example-g}
        =
        \iltikzfig{graphs/dpo/convex/rewrite-2}
        =
        \iltikzfig{graphs/dpo/convex/rewrite-4}
        =
        \iltikzfig{graphs/dpo/convex/rewrite-5}
        =
        \iltikzfig{graphs/dpo/convex/example-h}
    \end{gather*}
    This shows that convexity is not a required component of traced rewriting.
\end{example}

\todo[inline]{Refine after here}

We are almost ready to show the soundness and completeness of this DPO rewriting
system.
The final prerequisite is a decomposition lemma akin to that used
in~\cite{bonchi2022string}.

\begin{lemma}[Traced decomposition]\label{lem:traced-decomposition}
    Given partial monogamous cospans \(
    \cospan{m}[d_1]{G}[d_2]{n}
    \) and \(
    \cospan{i}[a_1]{L}[a_2]{j}
    \), and a morphism \(
    L \xrightarrow{f} G
    \) such that \(\listvar{ij} \rightarrow L \rightarrow G\) satisfies the no-dangling
    and no-identification conditions, then there exists a partial monogamous
    cospan \(
    \cospan{j+m}[[c_2,d_1]]{C}[[c_1,d_2]]{i+n}
    \) such that \(
    \cospan{m}{G}{n}
    \) can be factored as
    \begin{gather*}
        \trace{i}{
            \begin{array}{cc}
                \cospan{i}[a_1]{L}[a_2]{j} \\
                \tensor                    \\
                \cospan{m}{m}{m}
            \end{array}
            \seq
            \cospan{j+m}[[c_2,d_1]]{C}[[c_1,d_2]]{i+n}
        }
    \end{gather*}
    where \(
    \cospan{j+m}[c_2,d_1]{C}[c_1,d_2]{i+n}
    \) is a traced boundary complement.
\end{lemma}
\begin{proof}
    Let \(
    i + j \xrightarrow{[c_1, c_2]} C \xleftarrow{[d_1, d_2]} \listvar{mn}
    \) be defined as a traced boundary complement of \(
    \listvar{ij} \xrightarrow{[a_1,a_2]} L \xrightarrow{f} G
    \), which exists as the no-dangling and no-identification condition is
    satisfied.
    We assign names to the various cospans in play, and reason string
    diagrammatically:
    \begin{align*}
        \iltikzfig{strings/category/f}[box=l,colour=white,dom=i,cod=j] & := \cospan{i}{L}{j}
                                                                       &
        \iltikzfig{strings/category/f-0-2}[box=\hat{l},colour=white,cod1=i,cod2=j]
                                                                       & :=
        \cospan{0}{L}{\listvar{ij}}
        \\
        \iltikzfig{strings/category/f-2-2}[box=c,colour=white,dom1=j,dom2=m,cod1=i,cod2=n]
                                                                       & :=
        \cospan{j+m}[[c_2, d_1]]{C}[[c_1, d_2]]{i+n}
                                                                       &
        \iltikzfig{strings/category/f-2-2}[box=\hat{c},colour=white,dom1=i,dom2=j,cod1=m,cod2=n]
                                                                       & :=
        \cospan{\listvar{ij}}[[c_1, c_2]]{C}[[d_1, d_2]]{\listvar{mn}}
        \\
        \iltikzfig{strings/category/f}[box=g,colour=white,dom=m,cod=n]
                                                                       & :=
        \cospan{m}{G}{n}
                                                                       &
        \iltikzfig{strings/category/f-0-2}[box=\hat{g},colour=white,cod1=m,cod2=n]
                                                                       & :=
        \cospan{0}{G}{\listvar{mn}}
    \end{align*}
    Note that the cospans in the left column are partial monogamous by definition
    of rewrite rules and traced boundary complements.
    We will show that  \(
    \iltikzfig{strings/category/f}[box=g,colour=white]
    \) can be decomposed into a form using the two cospans \(
    \iltikzfig{strings/category/f}[box=l,colour=white]
    \) and \(
    \iltikzfig{strings/category/f-2-2}[box=c,colour=white]
    \), along with identities.

    By using the compact closed structure of \(\cspdhyp\), we have the following:
    \begin{gather*}
        \iltikzfig{strings/category/f}[box=g,colour=white,dom=m,cod=n]
        =
        \iltikzfig{graphs/dpo/g-bent}
        \qquad
        \iltikzfig{strings/category/f-2-2}[box=\hat{c},colour=white,dom1=i,dom2=j,cod1=m,cod2=n]
        =
        \iltikzfig{graphs/dpo/cprime-as-c}
        \qquad
        \iltikzfig{strings/category/f-0-2}[box=\hat{l},colour=white,cod1=i,cod2=j]
        =
        \iltikzfig{strings/compact-closed/f-bent-input}[box=l,colour=white,cod=i,dom=j]
    \end{gather*}
    Since \(G\) is the pushout of \(
    L \xleftarrow{[a_1, a_2]} \listvar{ij} \xrightarrow{[c_1, c_2]} C
    \) and pushout is cospan composition, we also have that \(
    \iltikzfig{strings/category/f-0-2}[box=\hat{g},colour=white,cod1=m,cod2=n]
    =
    \iltikzfig{graphs/dpo/lctilde}
    \).
    Putting this all together we can show that
    \begin{gather*}
        \iltikzfig{strings/category/f}[box=g,colour=white,dom=m,cod=n]
        =
        \iltikzfig{graphs/dpo/g-bent}
        =
        \iltikzfig{graphs/dpo/l-c-bent}
        =
        \iltikzfig{graphs/dpo/l-c-bent-1}
        =
        \iltikzfig{graphs/dpo/lc-bent-2}
        =
        \iltikzfig{strings/rewriting/rewrite-l}[dom=m,cod=n]
    \end{gather*}
    Since the `loop' is constructed in the same manner as the canonical trace on
    \(\cspdhyp\) (and is therefore identical in the graphical notation), this is a
    term in the form of (\ref{gath:decomposition}).
\end{proof}

We need to show that, term rewriting with a set of rules \(\mcr\)
coincides with graph rewriting on the hypergraph interpretations of these rules.
Recall that DPO rules have only one interface so the rules need to be `bent'
using \(\foldinterfaces\);
we write \(
\foldinterfaces[\tracedtosymandfrob[\mcr]{\Sigma}]
\) for the pointwise map \(
(
\iltikzfig{strings/category/f}[box=l,colour=white],
\iltikzfig{strings/category/f}[box=r,colour=white]
)
\mapsto
(
\foldinterfaces[
    \tracedtosymandfrob[
    \iltikzfig{strings/category/f}[box=l,colour=white]
]{\Sigma}
],
\foldinterfaces[
    \tracedtosymandfrob[
    \iltikzfig{strings/category/f}[box=r,colour=white]
]{\Sigma}
]
).
\)

\begin{theorem}\label{thm:traced-rewriting}
    Let \(\mcr\) be a rewriting system on \(\stmcsigma\).
    Then,
    \begin{gather*}
        \iltikzfig{strings/category/f}[box=g,colour=white,dom=m,cod=n]
        \rewrite[\mcr]
        \iltikzfig{strings/category/f}[box=h,colour=white,dom=m,cod=n]
        \quad
        \text{if and only if}
        \quad
        \termandfrobtohypsigma[
            \foldinterfaces[
                \tracedtosymandfrob[
                    \iltikzfig{strings/category/f}[box=g,colour=white]
                ]{\Sigma}
            ]
        ]
        \grewrite[
            \termandfrobtohypsigma[
                \foldinterfaces[
                    \tracedtosymandfrob[\mcr]{\Sigma}
                ]
            ]
        ]
        \termandfrobtohypsigma[
            \foldinterfaces[
                \tracedtosymandfrob[
                    \iltikzfig{strings/category/f}[box=h,colour=white]
                ]{\Sigma}
            ]
        ].
    \end{gather*}
\end{theorem}
\begin{proof}
    First the \((\Rightarrow)\) direction.
    If \(
    \iltikzfig{strings/category/f}[box=g,colour=white]
    \rewrite[\mcr]
    \iltikzfig{strings/category/f}[box=h,colour=white]
    \) then we have \(
    \iltikzfig{strings/category/f}[box=g,colour=white]
    =
    \iltikzfig{strings/rewriting/rewrite-l}
    \) and \(
    \iltikzfig{strings/rewriting/rewrite-r}
    =
    \iltikzfig{strings/category/f}[box=h,colour=white].
    \)
    Define the following cospans:
    \begin{alignat}{3}
        \label{gath:l-cospan}
        \cospan{0}{L}{\listvar{ij}}
         & :=
        \termandfrobtohypsigma[\foldinterfaces[\iltikzfig{strings/category/f}[box=l,colour=white]]]
         &    & =
        \termandfrobtohypsigma[\iltikzfig{strings/rewriting/l-folded}]
        \\
        \cospan{0}{R}{\listvar{ij}}
         & :=
        \termandfrobtohypsigma[\foldinterfaces[\iltikzfig{strings/category/f}[box=r,colour=white]]]
         &    & =
        \termandfrobtohypsigma[\iltikzfig{strings/rewriting/r-folded}]
        \\
        \cospan{0}{G}{\listvar{mn}}
         & :=
        \termandfrobtohypsigma[\foldinterfaces[\iltikzfig{strings/category/f}[box=f,colour=white]]]
         &    & =
        \termandfrobtohypsigma[\iltikzfig{strings/rewriting/lc-folded}]
        \\
        \label{gath:h-cospan}
        \cospan{0}{H}{\listvar{mn}}
         & :=
        \termandfrobtohypsigma[\foldinterfaces[\iltikzfig{strings/category/f}[box=h,colour=white]]]
         &    & =
        \termandfrobtohypsigma[\iltikzfig{strings/rewriting/rc-folded}]
        \\
        \cospan{\listvar{ij}}{C}{\listvar{mn}}
         & :=
        \termandfrobtohypsigma[\iltikzfig{strings/rewriting/c-folded}]
         &    &
    \end{alignat}
    By functoriality, since \(
    \foldinterfaces[\iltikzfig{strings/category/f}[box=f,colour=white]]
    =
    \iltikzfig{strings/rewriting/l-folded}
    \seq
    \iltikzfig{strings/rewriting/c-folded}
    \) then \[
        \cospan{0}{G}{\listvar{mn}} = \cospan{0}{L}{\listvar{ij}} \seq \cospan{\listvar{ij}}{C}{\listvar{mn}}.
    \]
    Cospan composition is pushout, so \(\cospan{L}{G}{C}\) is a pushout.
    Using the same reasoning, \(\cospan{R}{G}{C}\) is also a pushout: this
    gives us the DPO diagram.
    All that remains is to check that the aforementioned pushouts are traced
    boundary complements: this follows by inspecting components.

    Now the \(\ifdir\) direction: assume we have a DPO diagram where
    \(L \leftarrow i + j\), \(i + j \rightarrow R\), \(m + n \to G\) and
    \(m + n \to H\) are defined as in (\ref{gath:l-cospan}-\ref{gath:h-cospan})
    above.
    We must show that \(
    \iltikzfig{strings/category/f}[box=f,colour=white]
    =
    \iltikzfig{strings/rewriting/rewrite-l}
    \) and \(
    \iltikzfig{strings/category/f}[box=h,colour=white]
    =
    \iltikzfig{strings/rewriting/rewrite-r}
    \).
    By definition of traced boundary complement \(\cospan{j+m}{C}{i+n}\) is a
    partial monogamous cospan, so by fullness of
    \(\termandfrobtohypsigma \circ \tracedtosymandfrobsigma\), there exists a term \(
    \iltikzfig{strings/category/f-2-2}[box=c,colour=white,dom1=j,dom2=m,cod1=i,cod2=n]
    \in \stmcsigma
    \) such that \(
    \termandfrobtohypsigma[
        \tracedtosymandfrob[
            \iltikzfig{strings/category/f-2-2}[box=c,colour=white]
        ]{\Sigma}
    ]
    =
    \cospan{j+m}{C}{i+n}
    \).
    By traced decomposition (\autoref{lem:traced-decomposition}), we have that for any
    traced boundary complement \(\cospan{\listvar{ij}}{C}{\listvar{mn}}\) and morphism
    \(L \to G\), \(\cospan{m}{G}{n}\) can be factored as in
    (\ref{gath:decomposition}), i.e.\ \[
        \termandfrobtohypsigma[\iltikzfig{strings/category/f}[box=f,colour=white]]
        =
        \trace{j}{\termandfrobtohypsigma[\iltikzfig{strings/category/f}[box=l,colour=white]]
            \tensor
            \id[n]
            \seq
            \termandfrobtohypsigma[\iltikzfig{strings/category/f-2-2}[box=c,colour=white]]}.
    \]
    By functoriality, we have \(
    \iltikzfig{strings/category/f}[box=f,colour=white]
    =
    \iltikzfig{strings/rewriting/rewrite-l}
    \).
    The same follows for \(
    \iltikzfig{strings/category/f}[box=h,colour=white]
    =
    \iltikzfig{strings/rewriting/rewrite-r}
    \).
\end{proof}
\section{Rewriting with commutative comonoid structure}