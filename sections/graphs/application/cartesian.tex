\section{Cartesian structure}\label{sec:cartesian-structure}

One important class of categories with a traced comonoid structure are
\emph{traced Cartesian}, or \emph{dataflow},
categories~\cite{cazanescu1990new}.

\begin{definition}[Cartesian category~\cite{fox1976coalgebras}]
    A monoidal category is \emph{Cartesian} if its tensor is given by the
    category-theoretic product.
\end{definition}

As the tensor is a product, every object in a Cartesian category is
automatically equipped with a comonoid structure.
An alternative way of viewing a Cartesian category is as a category with such
comonoid structure in which the comonoid and the counit are \emph{natural}:
morphisms can be `pushed' through them.

\begin{theorem}[\cite{fox1976coalgebras}]
    A category \(\mcc\) equipped with a commutative comonoid structure is
    Cartesian if and only if the equations in \cref{fig:cartesian-equations}
    hold for all \(\iltikzfig{strings/category/f}[box=f,dom=m,cod=n]\).
\end{theorem}

\begin{figure}
    \centering
    \(
    \equationdisplay{
        \iltikzfig{strings/structure/comonoid/unitality-l-lhs}
    }{
        \iltikzfig{strings/structure/comonoid/unitality-l-rhs}
    }{
        \comonoidunitleqn
    }
    \)
    %
    \;
    \(
    \equationdisplay{
        \iltikzfig{strings/structure/comonoid/unitality-r-lhs}
    }{
        \iltikzfig{strings/structure/comonoid/unitality-r-rhs}
    }{
        \comonoidunitreqn
    }
    \)
    \\[0.25em]
    \rule{\textwidth}{0.1mm}
    \\[0.5em]
    \(
    \equationdisplay{
        \iltikzfig{strings/structure/comonoid/associativity-lhs}
    }{
        \iltikzfig{strings/structure/comonoid/associativity-rhs}
    }{
        \comonoidassoceqn
    }
    \)
    %
    \;
    %
    \(
    \equationdisplay{
        \iltikzfig{strings/structure/comonoid/commutativity-lhs}
    }{
        \iltikzfig{strings/structure/comonoid/commutativity-rhs}
    }{
        \comonoidcommeqn
    }
    \)
    \\[0.25em]
    \rule{\textwidth}{0.1mm}
    \\[0.5em]
    \(
    \equationdisplay{
        \iltikzfig{strings/structure/comonoid/coherence-comonoid-lhs}[obj1=A,obj2=B]
    }{
        \iltikzfig{strings/structure/comonoid/coherence-comonoid-rhs}[obj1=A,obj2=B]
    }{
        \comonoidcoheqn
    }
    \)
    %
    \;
    %
    \(
    \equationdisplay{
        \iltikzfig{strings/structure/comonoid/coherence-counit-lhs}[obj1=A,obj2=B]
    }{
        \iltikzfig{strings/structure/comonoid/coherence-counit-rhs}[obj1=A,obj2=B]
    }{
        \counitcoheqn
    }
    \)
    \\[0.25em]
    \rule{\textwidth}{0.1mm}
    \\[0.5em]
    \(
    \equationdisplay{
        \iltikzfig{strings/structure/cartesian/naturality-copy-lhs}[box=f,colour=white,dom=A,cod=B]
    }{
        \iltikzfig{strings/structure/cartesian/naturality-copy-rhs}[box=f,colour=white,dom=A,cod=B]
    }{
        \cartnatcopyeqn
    }
    \)
    %
    \quad
    %
    \(
    \equationdisplay{
        \iltikzfig{strings/structure/cartesian/naturality-discard-lhs}[box=f,colour=white,dom=A]
    }{
        \iltikzfig{strings/structure/cartesian/naturality-discard-rhs}[dom=m]
    }{
        \cartnatdisceqn
    }
    \)
    \\[0.25em]
    \rule{\textwidth}{0.1mm}
    \caption{Equations that hold in any Cartesian category}
    \label{fig:cartesian-equations}
\end{figure}

The two equations at the bottom of \cref{fig:cartesian-equations} show how
morphisms in Cartesian categories can be \emph{copied} and \emph{discarded}.
Sequential circuits have a natural notion of copying and discarding data, so it
makes sense that the semantic categories of circuits should be Cartesian.
In \cite{ghica2016categorical}, the equational theory is used to show that this
is the case, but with the stream semantics we have a much more elementary proof.

\begin{theorem}
    \(\streami\) is Cartesian.
\end{theorem}
\begin{proof}
    The tensor in \(\streami\) is defined to be the Cartesian product.
\end{proof}

As the three semantic categories are isomorphic to \(\streami\) they are also
Cartesian.

\begin{corollary}
    \(\scircsigmai\), \(\scircsigmaobs\) and \(\scircsigmae\) are Cartesian.
\end{corollary}

We can express the data of a Cartesian category as an extension of the
monoidal theory of commutative comonoids.

\begin{definition}
    The Cartesian monoidal theory \(
    (\generators[\mathbf{Cart}], \equations[\mathbf{Cart}])
    \) is defined as \(
    \generators[\mathbf{Cart}] \coloneqq \generators[\ccomon]
    \) and \(
    \equations[\mathbf{Cart}]
    \coloneqq
    \generators[\ccomon] + (\cartnatcopyeqn) + (\cartnatdisceqn)
    \) where \((\cartnatcopyeqn)\) and \((\cartnatdisceqn)\) are defined as in
    \cref{fig:cartesian-theory-equations}.
\end{definition}

\begin{remark}
    Note that we do not need the two `coherence' Cartesian equations when
    considering monoidal theories, because they follow immediately from the
    construction of multiple-bit structures
    (\cref{not:arbitrary-width-structure}).
\end{remark}

\begin{figure}
    \centering
    \(
    \equationdisplay{
        \iltikzfig{strings/structure/cartesian/naturality-copy-lhs}[box=f,colour=white,dom=m,cod=n]
    }{
        \iltikzfig{strings/structure/cartesian/naturality-copy-rhs}[box=f,colour=white,dom=m,cod=n]
    }{
        \cartnatcopyeqn
    }
    \)
    \qquad
    \(
    \equationdisplay{
        \iltikzfig{strings/structure/cartesian/naturality-discard-lhs}[box=f,colour=white,dom=m]
    }{
        \iltikzfig{strings/structure/cartesian/naturality-discard-rhs}[dom=m]
    }{
        \cartnatdisceqn
    }
    \)
    \caption{
        Equations of the monoidal theory \(\mathbf{Cart}_\mathcal{C}\),
        for generator \(f\)
    }
    \label{fig:cartesian-theory-equations}
\end{figure}

The hypergraph interpretations of these rules is shown in
\cref{fig:cartesian-graphs}.

\begin{remark}
    The combination of Cartesian equations with the underlying compact closed
    structure of \(\cspdhyp\) may prompt alarm bells, as a compact closed
    category in which the tensor is the Cartesian product is trivial.
    However, it is important to note that \(\cspdhyp\) is \emph{not} subject to
    these equations: it is only a setting for performing graph
    rewrites.
\end{remark}

\begin{figure}
    \centering
    \includestandalone{figures/graphs/dpo/cartesian/copy/rule}
    \qquad
    \raisebox{1em}{\includestandalone{figures/graphs/dpo/cartesian/discard/rule}}
    \caption{
        Interpretations of equations in \(\mathbf{Cart}\) for
        generator \(e\).
    }
    \label{fig:cartesian-graphs}
\end{figure}

Reasoning about Cartesian categories by using the hypergraph interpretation is
beneficial because the equations of the comonoid no longer have to be applied
explicitly, and one can focus on the two naturality equations
\((\cartnatcopyeqn)\) and \((\cartnatdisceqn)\).

As an example, we will consider \emph{fixpoint operators}, which are closely
linked to traced Cartesian categories.
Several equivalent axiomatisations for fixpoint operators
exist~\cite{hasegawa1997recursion,simpson2000complete}; we use the equations
presented in~\cite{hasegawa2009traced}.

\begin{definition}[\cite{hasegawa2009traced}]
    Let \(\mcc\) be a Cartesian category.
    A \emph{Conway fixpoint operator} in \(\mcc\) is a family of functions \(
    \morph{
        (-)^{\dagger_{A,X}}
    }{
        \mcc(A \times X, X)
    }{
        \mcc(A, X)
    }
    \) drawn as \[
        \left(\iltikzfig{strings/category/f-2-1}[colour=white,box=f,dom1=X,dom2=A,cod=X]\right)
        \coloneqq
        \iltikzfig{strings/structure/cartesian/fixpoint}
    \] subject to the following equations:
    \begin{gather*}
        \iltikzfig{strings/structure/cartesian/fixpoint-lhs}
        =
        \iltikzfig{strings/structure/cartesian/fixpoint-rhs}
        \qquad
        \iltikzfig{strings/structure/cartesian/dinaturality-fixpoint-lhs}
        =
        \iltikzfig{strings/structure/cartesian/dinaturality-fixpoint-rhs}
        \\[0.5em]
        \iltikzfig{strings/structure/cartesian/naturality-fixpoint-lhs}
        =
        \iltikzfig{strings/structure/cartesian/naturality-fixpoint-rhs}
        \qquad
        \iltikzfig{strings/structure/cartesian/diagonal-fixpoint-lhs}
        =
        \iltikzfig{strings/structure/cartesian/diagonal-fixpoint-rhs}
    \end{gather*}
\end{definition}

The notation we use for fixpoint operator is already evocative of a trace, and
the two are indeed equivalent.

\begin{theorem}[\cite{hasegawa1997recursion}, Thm.\ 3.1, \cite{hasegawa1999models}]
    A category is traced if and only if it has a Conway operator.
\end{theorem}
\begin{proof}
    This can be shown by constructing a one operator using the other.
    \begin{gather*}
        (\iltikzfig{strings/category/f-2-1}[box=f,colour=white])^{\dagger_{X,A}}
        =
        \iltikzfig{strings/structure/cartesian/fixpoint-from-trace}
        \qquad
        \trace{X}[A][B]{\iltikzfig{strings/category/f-2-2}[box=f,colour=white]}
        =
        \iltikzfig{strings/structure/cartesian/trace-from-fixpoint}
    \end{gather*}
    As these constructions are inverses, the conditions are equivalent.
\end{proof}

In addition to Hasegawa's proof, the above result was independently observed by
Bloom and Ésik \cite{bloom1993iteration},
Ștefănescu \cite{stefanescu2000network}, and Hyland, the former two before the
notion of traced monoidal categories were formalised.

\begin{example}[Unfolding]
    Reasoning about fixpoints in a traced category can be performed using the
    \emph{unfolding} rule.
    \begin{gather*}
        \iltikzfig{strings/traced/trace-rhs}[box=f,colour=white,dom=m,cod=n,trace=x]
        =
        \iltikzfig{circuits/examples/reasoning/unfolding/unfolding-1}[box=f,colour=white,dom=m,cod=n,trace=x]
        =
        \iltikzfig{circuits/examples/reasoning/unfolding/unfolding-2}[box=f,colour=white,dom=m,cod=n,trace=x]
        =
        \iltikzfig{circuits/examples/reasoning/unfolding/unfolding-3}[box=f,colour=white,dom=m,cod=n,trace=x]
    \end{gather*}
    In the syntactic setting, this requires the application of multiple
    equations: the two counitality equations followed by the copy equation and
    optionally some axioms of STMCs for housekeeping.
    If we use the hypergraph interpretation, the comonoid equations are absorbed
    into the notation so only one rule needs to be applied.

    \begin{center}
        \includestandalone{figures/graphs/dpo/unfolding/rewrite-1}
    \end{center}
\end{example}

The dual notion of traced \emph{cocartesian}
categories~\cite{bainbridge1976feedback} are also important in computer science:
a trace in a traced cocartesian category corresponds to \emph{iteration} in the
context of \emph{control flow}.
The details of this section could also be applied to the cocartesian case by
flipping all the directions and working with partial \emph{right}-monogamous
cospans.

However, attempting to combine the product and coproduct approaches for settings
with a \emph{biproduct} would simply yield the category \(\cspdhyp\), a
hypergraph category subject to the Frobenius
equations in \cref{fig:frobenius-equations}.
A category with biproducts is not necessarily subject to such equations, so this
would not be a suitable approach.