\section{Sequential circuits}

Of course, our intended application for graph rewriting modulo traced
comonoid structure is that of sequential circuits.

\begin{example}
    Recall the SR NOR latch circuit from \cref{ex:sr-latch}.
    This is interpreted as a partial left-monogamous cospan of hypergraphs as
    follows, where the \(\wedge\), \(\neg\) and \(\delta\) edges respectively
    represent the \(\orgate\) gate, \(\notgate\) gate, and delay.
    \begin{center}
        \begin{tikzcd}[bezier bounding box=true]
            \iltikzfig{graphs/circuits/sr-latch/inputs}
            \arrow{r}
            &
            \iltikzfig{graphs/circuits/sr-latch/latch}
            &
            \arrow{l}
            \iltikzfig{graphs/circuits/sr-latch/outputs}
        \end{tikzcd}
    \end{center}
\end{example}

When performing the instant feedback reduction on terms, it can lead to a
complicated term with the inputs forked multiple times in order to pass them to
each of the copies of the original circuit.
In the hypergraph representation, the forks are all absorbed into one.

\begin{example}
    \begin{center}
        \includestandalone{figures/graphs/circuits/unroll/rewrite}
    \end{center}
\end{example}

\begin{example}
    The SR latch
\end{example}



\begin{remark}
    One minor issue that arises when performing the circuit reduction arises
    when considering the fork rewrite rule.
    \begin{gather*}
        \iltikzfig{circuits/axioms/fork-lhs}
        \reduction
        \iltikzfig{circuits/axioms/fork-rhs}
        \qquad\qquad
        \raisebox{-2em}{\includestandalone{figures/graphs/circuits/fork-rewrite/rule}}
    \end{gather*}
    Now consider the term \(
    \iltikzfig{graphs/circuits/fork-rewrite/term-g}
    \), which does not immediately appear to have any instances of the fork
    rule.
    However, if we consider the hypergraph interpretation, we can see that
    modulo comonoid structure it \emph{is} possible to apply it.
    \begin{gather*}
        \includestandalone{figures/graphs/circuits/fork-rewrite/fork-rewrite}
    \end{gather*}
    This reduction has arisen due to the counitality of the comonoid.
    \begin{gather*}
        \iltikzfig{graphs/circuits/fork-rewrite/term-g}
        =
        \iltikzfig{graphs/circuits/fork-rewrite/term-g-1}
        \reduction
        \iltikzfig{graphs/circuits/fork-rewrite/term-h}
    \end{gather*}
    This means that care must be taken when performing the reduction procedure;
    a fork rewrite is only productive if the vertex in the image of \(f\) has
    out-degree strictly greater than \(1\).
    If we arbitrarily any rewrite we find, we end up stuck in a loop of forking
    and stubbing the same subcircuit.
\end{remark}