\section{Sequential circuits}

When rewriting circuits with hypergraphs there is no distinguishing between a
wire going `backwards' or a regular wire, so we must explicitly keep track of
traces if we need to rewrite them later.
To represent this in our diagrams, we will colour any trace source tentacle red.

\begin{example}\label{ex:sr-latch-interpretation}
    Recall the SR NOR latch circuit from \cref{ex:sr-latch}.
    This is interpreted as a partial left-monogamous cospan of hypergraphs as
    follows, where the \(\wedge\), \(\neg\) and \(\delta\) edges respectively
    represent the \(\orgate\) gate, \(\notgate\) gate, and delay.
    \begin{center}
        \begin{tikzcd}[bezier bounding box=true,column sep=tiny]
            \iltikzfig{graphs/circuits/sr-latch/inputs}
            \arrow{r}
            &
            \iltikzfig{graphs/circuits/sr-latch/latch}
            &
            \arrow{l}
            \iltikzfig{graphs/circuits/sr-latch/outputs}
        \end{tikzcd}
    \end{center}
\end{example}

Since the transformation into global trace-delay form is through axioms
of STMCs, as hypergraphs a circuit and its trace-delay form are isomorphic.
The first rewrite that needs to be applied is the global Mealy reduction
\((\mealyeqn)\).

\begin{example}\label{ex:sr-latch-mealy-graph}
    Applying the Mealy rewrite to \cref{ex:sr-latch-interpretation} produces
    the following cospan of hypergraphs:
    \vspace{-1em}
    \begin{center}
        \begin{tikzcd}[bezier bounding box=true,column sep=tiny]
            \iltikzfig{graphs/circuits/sr-latch/inputs}
            \arrow{r}
            &
            \iltikzfig{graphs/circuits/sr-latch/mealy}
            &
            \arrow{l}
            \iltikzfig{graphs/circuits/sr-latch/outputs}
        \end{tikzcd}
    \end{center}
\end{example}

The instant feedback reduction can produce a complicated term with many forks;
in the hypergraph representation, these forks are all absorbed into one.

\begin{example}\label{ex:instant-feedback-rewrite}
    Below is an example showing how the instant feedback rewrite is applied to
    a circuit in Mealy form containing one generator \(e\).
    \begin{center}
        \includestandalone{figures/graphs/circuits/unroll/rewrite}
    \end{center}
\end{example}

As the instant feedback rule eliminates non-delay-guarded feedback loops, there
are no red tentacles in the right hand side of the rule or the rewritten graph.

\begin{example}\label{ex:sr-latch-instant-feedback-graph}
    The interpretation of the SR latch from \cref{ex:sr-latch-mealy-graph}
    after being rewritten by the instant feedback rewrite is shown below.
    \vspace{-1em}
    \begin{center}
        \begin{tikzcd}[bezier bounding box=true, column sep=tiny]
            \scalebox{0.7}{\iltikzfig{graphs/circuits/sr-latch/inputs}}
            \arrow{r}
            &
            \scalebox{0.7}{\iltikzfig{graphs/circuits/sr-latch/instant-feedback}}
            &
            \arrow{l}
            \scalebox{0.7}{\iltikzfig{graphs/circuits/sr-latch/outputs}}
        \end{tikzcd}
    \end{center}
\end{example}

After performing the instant feedback rewrite, the graph is ready to receive
inputs.

\begin{example}\label{ex:sr-latch-streaming-graph}
    We apply the inputs \(\belnaptrue\belnapfalse\) to the prepared SR latch
    hypergraph from \cref{ex:sr-latch-instant-feedback-graph} by precomposing
    some value registers.
    \vspace{-1em}
    \begin{center}
        \begin{tikzcd}[bezier bounding box=true, column sep=tiny]
            \scalebox{0.7}{\iltikzfig{graphs/circuits/sr-latch/inputs}}
            \arrow{r}
            &
            \scalebox{0.7}{\iltikzfig{graphs/circuits/sr-latch/applied}}
            &
            \arrow{l}
            \scalebox{0.7}{\iltikzfig{graphs/circuits/sr-latch/outputs}}
        \end{tikzcd}
    \end{center}
    This graph is then rewritten by the streaming rule.
    \vspace{-2em}
    \begin{center}
        \begin{tikzcd}[bezier bounding box=true, column sep=tiny]
            \scalebox{0.7}{\iltikzfig{graphs/circuits/sr-latch/inputs}}
            \arrow{r}
            &
            \scalebox{0.7}{\iltikzfig{graphs/circuits/sr-latch/streamed}}
            &
            \arrow{l}
            \scalebox{0.7}{\iltikzfig{graphs/circuits/sr-latch/outputs}}
        \end{tikzcd}
    \end{center}
\end{example}

The final step is to propagate the values across the `top' subcircuit using the
value rules, which have straightforward hypergraph interpretations illustrated
in \cref{fig:graph-values}.

\begin{figure}
    \centering
    \scalebox{0.8}{\includestandalone{figures/graphs/circuits/gate/rule}}
    \quad
    \raisebox{0.5em}{\scalebox{0.8}{\includestandalone{figures/graphs/circuits/fork-rewrite/rule}}}
    \\
    \scalebox{0.8}{\includestandalone{figures/graphs/circuits/join/rule}}
    \quad
    \raisebox{1em}{\scalebox{0.8}{\includestandalone{figures/graphs/circuits/stub/rule}}}
    \caption{Hypergraph interpretations of the value rules}
    \label{fig:graph-values}
\end{figure}

\begin{example}
    When applying the value rules to the streamed circuit from
    \cref{ex:sr-latch-streaming-graph}, we apply the fork rules as much as
    possible to propagate the values:
    \begin{center}
        \vspace{-1em}
        \begin{tikzcd}[bezier bounding box=true, column sep=tiny]
            \scalebox{0.7}{\iltikzfig{graphs/circuits/sr-latch/inputs}}
            \arrow{r}
            &
            \scalebox{0.7}{\iltikzfig{graphs/circuits/sr-latch/values-1}}
            &
            \arrow{l}
            \scalebox{0.7}{\iltikzfig{graphs/circuits/sr-latch/outputs}}
        \end{tikzcd}
    \end{center}
    We can then repeatedly apply the gate and eliminate rule to obtain
    the outputs and next state, which can be seen below.
    \begin{center}
        \begin{tikzcd}[bezier bounding box=true, column sep=tiny]
            \scalebox{0.7}{\iltikzfig{graphs/circuits/sr-latch/inputs}}
            \arrow{r}
            &
            \scalebox{0.7}{\iltikzfig{graphs/circuits/sr-latch/values-2}}
            &
            \arrow{l}
            \scalebox{0.7}{\iltikzfig{graphs/circuits/sr-latch/outputs}}
        \end{tikzcd}
    \end{center}
    To identify the outputs, one simply needs to traverse from the outputs of
    the graph.
\end{example}

Once the circuit has been translated into Mealy form, the productive rewrite
strategy preserves this so the initial global transformations do not have to be
performed for each input value; one only needs to use the streaming rule
followed by the value rules.

\begin{remark}
    Note that the fork rule is not left-linear as it uses the comonoid
    structure.
    Consider the term \(
    \iltikzfig{graphs/circuits/fork-rewrite/term-g}
    \); in the hypergraph interpretation it is possible to apply the fork rule
    to this term.
    \begin{gather*}
        \includestandalone{figures/graphs/circuits/fork-rewrite/fork-rewrite}
    \end{gather*}
    This reduction has arisen due to the counitality of the comonoid.
    \begin{gather*}
        \iltikzfig{graphs/circuits/fork-rewrite/term-g}
        =
        \iltikzfig{graphs/circuits/fork-rewrite/term-g-1}
        \reduction
        \iltikzfig{graphs/circuits/fork-rewrite/term-h}
    \end{gather*}
    This means that a fork rewrite is only productive if the vertex in the image
    of \(f\) has out-degree greater than \(1\).
\end{remark}