\section{Hardware description language}

We motivated graph rewriting for digital circuits as an avenue for
\emph{automating} their reasoning.
To this end, the operational semantics for sequential circuits from
\cref{sec:operational} has been implemented into a
\emph{hardware description language} in Cangjie, a new programming
language developed by Huawei.

Much like the theoretical background, circuits created using the tool are
parameterised over a particular signature of components.
Rather than designing circuits using the categorical style of juxtaposing
tiles in sequence and parallel, the tool uses a more conventional approach
where the user manipulates wires and provides them as arguments to other
components.

\begin{example}
    We will first demonstrate how to define the combinational half adder circuit
    from \cref{ex:half-adder}.
    For pedagogical purposes, we begin by defining an \(\xorgate\) gate.
    \begin{lstlisting}
      let xorA = sig.UseWire(1)
      let xorB = sig.UseWire(1)
      let xorOr = UseOr(xorA, xorB)
      let xorNand = UseNot(UseAnd(xorA, xorB))
      let xorAnd = UseAnd(xorOr, xorNand)
      let xor = MakeSubcircuit(
        [InterfaceWire(xorA, "A"), InterfaceWire(xorB, "B")],
        [InterfaceWire(xorAnd, "Z")],
        "XOR"
      )
    \end{lstlisting}

    Once a subcircuit has been defined, a specification in Dot can be generated
    and rendered using Graphviz.

    \begin{center}
        \includesvg[scale=0.3]{figures/circuits/hdl/xor}
    \end{center}

    Using the \(\xorgate\) as a subcomponent, we can define the half adder.

    \begin{lstlisting}
    let addA = sig.UseWire(1)
    let addB = sig.UseWire(1)
    let sum = UseSubcircuit(xor, [addA, addB])[0]
    let carry = UseAnd(addA, addB)
    let halfAdder = MakeSubcircuit(
      [InterfaceWire(addA, "A"), InterfaceWire(addB, "B")],
      [InterfaceWire(sum, "S"), InterfaceWire(carry, "C")],
      "half adder"
    )
    \end{lstlisting}

    The generated graphs have a hierarchical structure: because we defined the
    \(\xorgate\) gate as a subcircuit, we can view it as a black box or
    expand it.

    \begin{center}
        \includesvg[scale=0.3]{figures/circuits/hdl/add-0}

        \vspace{1em}

        \includesvg[scale=0.3]{figures/circuits/hdl/add-1}
    \end{center}
\end{example}

As usual, it is the \emph{sequential} circuits which are the most interesting.
The tool can be used to insert delays and feedback loops to circuits.

\begin{example}
    The SR NOR latch from \cref{ex:sr-latch} can be created
    \begin{lstlisting}
      let r = sig.UseWire(1)
      let s = sig.UseWire(1)
      let fb = sig.UseWire(1)
      let or1 = UseOr(r, fb)
      let not1 = UseNot(or1, delay: 1)
      let or2 = UseOr(delay, s)
      let not2 = UseNot(or2)
      Feedback(not2, fb)
      let circ = MakeSubcircuit(
        [InterfaceWire(r, "R"), InterfaceWire(s, "S")],
        [InterfaceWire(delay, "Q"), InterfaceWire(not2, "Q'")],
        "SR NOR Latch"
      )
    \end{lstlisting}
    \begin{center}
        \scalebox{0.3}{\includesvg{figures/circuits/hdl/latch}}
    \end{center}
\end{example}
