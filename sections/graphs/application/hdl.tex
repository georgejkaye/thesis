\section{Hardware description language}

We motivated graph rewriting for digital circuits as an avenue for
\emph{automating} their reasoning.
To this end, the operational semantics for sequential circuits from
\cref{sec:operational} has been implemented into a
\emph{hardware description language} in Cangjie, a new programming
language developed by Huawei.

\todo[inline]{VHDL, Verilog, Hardcaml, Bluespec, Clash, Chisel}

\subsection{Design}

Much like the theoretical background, circuits created using the tool are
parameterised over a particular signature of components.
Rather than designing circuits using the categorical style of juxtaposing
tiles in sequence and parallel, the tool uses a more conventional approach
where the user manipulates wires and provides them as arguments to other
components.

\begin{example}
  We will first demonstrate how to define the combinational half adder circuit
  from \cref{ex:half-adder}.
  For pedagogical purposes, we begin by defining an \(\xorgate\) gate.
  \begin{lstlisting}
      let xorA = sig.UseWire(1)
      let xorB = sig.UseWire(1)
      let xorOr = UseOr(xorA, xorB)
      let xorNand = UseNot(UseAnd(xorA, xorB))
      let xorAnd = UseAnd(xorOr, xorNand)
      let xor = MakeSubcircuit(
        [InterfaceWire(xorA, "A"), InterfaceWire(xorB, "B")],
        [InterfaceWire(xorAnd, "Z")],
        "XOR"
      )
    \end{lstlisting}

  Once a subcircuit has been defined, a specification in Dot can be generated
  and rendered using Graphviz.

  \begin{center}
    \includesvg[scale=0.3]{figures/circuits/hdl/xor}
  \end{center}

  Using the \(\xorgate\) as a subcomponent, we can define the half adder.

  \begin{lstlisting}
    let addA = sig.UseWire(1)
    let addB = sig.UseWire(1)
    let sum = UseSubcircuit(xor, [addA, addB])[0]
    let carry = UseAnd(addA, addB)
    let halfAdder = MakeSubcircuit(
      [InterfaceWire(addA, "A"), InterfaceWire(addB, "B")],
      [InterfaceWire(sum, "S"), InterfaceWire(carry, "C")],
      "half adder"
    )
    \end{lstlisting}

  The generated graphs have a hierarchical structure: because we defined the
  \(\xorgate\) gate as a subcircuit, we can view it as a black box or
  expand it.

  \begin{center}
    \includesvg[scale=0.3]{figures/circuits/hdl/add-0}

    \vspace{1em}

    \includesvg[scale=0.3]{figures/circuits/hdl/add-1}
  \end{center}
\end{example}

As usual, it is the \emph{sequential} circuits which are the most interesting.
The tool can be used to insert delays and feedback loops to circuits.

\begin{example}\label{ex:sr-latch-hdl}
  The SR NOR latch from \cref{ex:sr-latch} can be created
  \begin{lstlisting}
      let r = sig.UseWire(1)
      let s = sig.UseWire(1)
      let fb = sig.UseWire(1)
      let or1 = UseOr(r, fb)
      let not1 = UseNot(or1, delay: 1)
      let or2 = UseOr(not1, s)
      let not2 = UseNot(or2)
      Feedback(not2, fb)
      let latch = MakeSubcircuit(
        [InterfaceWire(r, "R"), InterfaceWire(s, "S")],
        [InterfaceWire(not1, "Q"), InterfaceWire(not2, "Q'")],
        "SR NOR Latch"
      )
    \end{lstlisting}
  \begin{center}
    \scalebox{0.3}{\includesvg{figures/circuits/hdl/latch}}
  \end{center}
\end{example}

\subsection{Evaluation}

The real power of the tool comes from how it can automatically \emph{evaluate}
circuits using a reduction procedure based on the operational semantics.
As with the theory, it does this by first assembling circuits into Mealy form
and eliminating non-delay-guarded feedback.

\begin{example}
  Before evaluation can be performed, the SR NOR latch defined in
  \cref{ex:sr-latch} is automatically translated into Mealy form.
  \begin{lstlisting}
      let eval = Evaluator(latch)
    \end{lstlisting}
  \begin{center}
    \scalebox{0.25}{\includesvg{figures/circuits/hdl/latch-mealy}}
  \end{center}
  Note that this is a simpler circuit than the corresponding string diagram
  version in \cref{ex:sr-latch-unrolled} because the tool automatically
  applies combinational reductions (in particular, the elimination rule) to
  tidy up the resulting circuit.
\end{example}

Once the circuit is translated into the correct form, inputs can be provided
and the circuit evaluated step-by-step.

\begin{example}
  We now provide inputs to the evaluator created in
  \cref{ex:sr-latch-evaluator}; recall that the first input is the
  \textsf{R}eset input and the second is the \textsf{S}et input.
  \begin{lstlisting}
    let eval = Evaluator(latch)
    eval.PerformCycle([FALSE, TRUE])
    eval.PerformCycle([FALSE, FALSE])
    eval.PerformCycle([TRUE, FALSE])
    eval.PerformCycle([FALSE, FALSE])
  \end{lstlisting}
  This automatically applies the streaming and value rules in order to
  determine the output values over time.
  The inputs detailed above produce the output stream \(
  \bot\mathsf{f} \streamcons \mathsf{t}\mathsf{f} \streamcons
  \mathsf{t}\mathsf{f} \streamcons \mathsf{f}\mathsf{t}
  \).
  This is the expected output as the delay causes the outputs to be shifted back
  one time-step, and subsequently the first output is underdefined.
  \begin{center}
    \includesvg[scale=0.225]{figures/circuits/hdl/latch-outputs}
  \end{center}
\end{example}

\subsubsection{Cyclic combinational circuits}

The use of the instant feedback rule means that the tool can also handle
circuits with non-delay-guarded feedback that exhibit combinational behaviour.

\begin{example}
  Recall the cyclic combinational circuit from \cref{ex:cyclic-combinational}
  containing two blackbox circuits \(f\) and \(g\), in which the control signal
  dictates the order the two circuits are applied.
  \begin{lstlisting}
    // Input wires
    let x = sig.UseWire(1)
    let c = sig.UseWire(1)
    // Wire from feedback
    let feedback = sig.UseWire(1)
    // Top half of the circuit
    let muxa = UseMux2(s0: c, i0: x, i1: feedback)
    let fbb = belnapSignature.AddBlackbox("F", [Port(1, "A")], [Port(1, "Z")])
    let f = UseBlackbox(fbb, [muxa])[0]
    // Bottom half of the circuit
    let muxb = UseMux2(s0: c, i0: f, i1: x)
    let gbb = belnapSignature.AddBlackbox("G", [Port(1, "A")], [Port(1, "Z")])
    let g = UseBlackbox(gbb, [muxb])[0]
    Feedback(g, feedback)
    // Final multiplexer
    let muxc = UseMux2(s0: c, i0: g, i1: f)
    let cyclic = MakeSubcircuit(
      [InterfaceWire(x, "x"), InterfaceWire(c, "c")],
      [InterfaceWire(muxc, "z")],
      "cyclic_combinational"
    )
  \end{lstlisting}
  \begin{center}
    \includesvg[scale=0.25]{figures/circuits/hdl/cyclic-combinational/circuit}
  \end{center}
  By providing input values we can verify that this circuit truly does have
  combinational behaviour.
  \begin{lstlisting}
    let eval = Evaluator(sig, cyclic)
    eval.PerformCycle([TRUE, TRUE])
    eval.PerformCycle([TRUE, FALSE])
  \end{lstlisting}
  \begin{center}
    \includesvg[scale=0.3]{figures/circuits/hdl/cyclic-combinational/reduced}
  \end{center}
\end{example}

\subsection{Partial evaluation}

One of the benefits of the graph-rewrite-based evaluation style is that it
allows for \emph{partial evaluation}.
The tool implements some of the strategies discussed in \cref{sec:partial}, such
as tidying rules, shortcut rules, and uncertain values for reasoning with
protocols.


\begin{example}
  Recall the circuit from \cref{ex:protocols}, which reduces to the identity
  when the first input is fixed as either true or false.
  This circuit can be designed as follows:
  \begin{lstlisting}
    let v = sig.UseWire(1)
    let w = sig.UseWire(1)
    let not = UseNot(v)
    let or1 = UseOr(not, v)
    let fb = sig.UseWire(1)
    let or2 = UseOr(fb, or1)
    let and = UseAnd(or2, w)
    DelayGuardedFeedback(and, fb)
    let circ = MakeSubcircuit(
      [InterfaceWire(v, "A"), InterfaceWire(w, "B")],
      [InterfaceWire(and, "Z")],
      "circuit"
    )
  \end{lstlisting}
  \begin{center}
    \includesvg[scale=0.3]{figures/circuits/hdl/protocol/circuit}
  \end{center}
  We apply some uncertain values and partially evaluate.
  \begin{lstlisting}
    let eval = PartiallyEvaluate(
      sig, [GetVariable([TRUE, FALSE]), GetUnspecified(sig)], circ
    )
  \end{lstlisting}
  \begin{center}
    \includesvg[scale=0.3]{figures/circuits/hdl/protocol/rewrite-1}
    \(\grewrite\)

    \vspace{1em}

    \includesvg[scale=0.3]{figures/circuits/hdl/protocol/rewrite-2}
    \(\grewrite\)

    \vspace{1em}

    \includesvg[scale=0.3]{figures/circuits/hdl/protocol/rewrite-3}
    \(\grewrite\)

    \vspace{1em}

    \includesvg[scale=0.3]{figures/circuits/hdl/protocol/rewrite-4}
    \(\grewrite\)


    \vspace{1em}

    \includesvg[scale=0.3]{figures/circuits/hdl/protocol/rewrite-5}
    \qquad
    \(\grewrite\)
    \qquad
    \raisebox{1em}{\includesvg[scale=0.3]{figures/circuits/hdl/protocol/rewrite-6}}
  \end{center}
\end{example}
