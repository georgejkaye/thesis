\section{Hardware description language}\label{sec:hdl}

We motivated graph rewriting for digital circuits as an avenue for
\emph{automating} their reasoning.
To this end, the operational semantics for sequential circuits has been
implemented into a \emph{hardware description language} (HDL) in Cangjie, a new
programming language developed by Huawei.
The source code can be found at \url{https://github.com/georgejkaye/circuit-cj}.

\subsection{Types of HDLs}

Designing the intricate circuits of today using pen and paper would be
incredibly tedious and complicated.
For this reason, \emph{hardware description languages}, programming languages
specialised for designing hardware, are often employed to design circuits on a
computer.

A hardware description may be a completely bespoke language; this is the case
with the two most commonly used HDLs: VHDL~\cite{ieeecomputersociety1988ieee}
and Verilog~\cite{ieeecomputersociety1996ieee}.
These two languages are quite different to traditional programming languages.
As well as being able to construct circuits \emph{structurally} by piecing
components together, behavioural descriptions
can be specified in a \emph{dataflow} manner, in which constants are immediately
propagated across the program upon updating, much like in a spreadsheet.
This differs from the \emph{control flow} style of execution in ordinary
programming languages, in which each line of code runs sequentially.

Because they are so different to traditional programming languages, this can
make VHDL and Verilog inpenetrable to outsiders.
An alternative is to \emph{embed} a HDL into an already existing language.
Choices of parent language include Haskell (Lava~\cite{bjesse1998lava},
Bluespec~\cite{nikhil2004bluespec},
Clash~\cite{kooijman2009haskell,baaij2010clash}), OCaml
(Hardcaml~\cite{ray2023hardcaml}), and Scala (Chisel~\cite{bachrach2012chisel}).
Because these are primarily \emph{functional}, these HDLs are more structural in
nature; circuits are created by composing functions together.

We opted to follow the latter approach when implementing our theoretical
framework into a hardware description language, avoiding the need to design a
completely new language from scratch.

\subsection{Design}

Much like the theoretical background, circuits created using the tool are
parameterised over a particular signature of components.
Rather than designing circuits using the categorical style of juxtaposing
tiles in sequence and parallel, the tool uses a more conventional approach
where the user manipulates wires and provides them as arguments to other
components.

\begin{example}
  We will first demonstrate how to define the combinational half adder circuit
  from \cref{ex:half-adder}.
  We begin by defining an \(\xorgate\) gate.
  \begin{lstlisting}
      let xorA = sig.UseWire(1)
      let xorB = sig.UseWire(1)
      let xorOr = UseOr(xorA, xorB)
      let xorNand = UseNot(UseAnd(xorA, xorB))
      let xorAnd = UseAnd(xorOr, xorNand)
      let xor = MakeSubcircuit(
        [InterfaceWire(xorA, "A"), InterfaceWire(xorB, "B")],
        [InterfaceWire(xorAnd, "Z")],
        "XOR"
      )
    \end{lstlisting}

  Once a subcircuit has been defined, a specification in Dot can be generated
  and rendered using Graphviz.

  \begin{center}
    \includesvg[scale=0.3]{figures/circuits/hdl/xor}
  \end{center}

  Using the \(\xorgate\) as a subcomponent, we can define the half adder.

  \begin{lstlisting}
    let addA = sig.UseWire(1)
    let addB = sig.UseWire(1)
    let sum = UseSubcircuit(xor, [addA, addB])[0]
    let carry = UseAnd(addA, addB)
    let halfAdder = MakeSubcircuit(
      [InterfaceWire(addA, "A"), InterfaceWire(addB, "B")],
      [InterfaceWire(sum, "S"), InterfaceWire(carry, "C")],
      "half adder"
    )
    \end{lstlisting}

  The generated graphs have a hierarchical structure: because we defined the
  \(\xorgate\) gate as a subcircuit, we can view it as a black box or
  expand it.

  \begin{center}
    \includesvg[scale=0.3]{figures/circuits/hdl/add-0}

    \vspace{1em}

    \includesvg[scale=0.3]{figures/circuits/hdl/add-1}
  \end{center}
\end{example}

As usual, it is the \emph{sequential} circuits which are the most interesting.
The tool can be used to insert delays and feedback loops to circuits.

\begin{example}\label{ex:sr-latch-hdl}
  The SR NOR latch from \cref{ex:sr-latch} can be created
  \begin{lstlisting}
      let r = sig.UseWire(1)
      let s = sig.UseWire(1)
      let fb = sig.UseWire(1)
      let or1 = UseOr(r, fb)
      let not1 = UseNot(or1, delay: 1)
      let or2 = UseOr(not1, s)
      let not2 = UseNot(or2)
      Feedback(not2, fb)
      let latch = MakeSubcircuit(
        [InterfaceWire(r, "R"), InterfaceWire(s, "S")],
        [InterfaceWire(not1, "Q"), InterfaceWire(not2, "Q'")],
        "SR NOR Latch"
      )
    \end{lstlisting}
  \begin{center}
    \scalebox{0.3}{\includesvg{figures/circuits/hdl/latch}}
  \end{center}
\end{example}

\subsection{Evaluation}

Once circuits are designed, the tool can automatically \emph{reduce} them
using a procedure based on the operational semantics from
\cref{chap:operational}.
As with the theory, it does this by first assembling circuits into Mealy form
and eliminating non-delay-guarded feedback.

\begin{example}\label{ex:sr-latch-evaluator}
  Before evaluation can be performed, the SR NOR latch defined in
  \cref{ex:sr-latch} is automatically translated into Mealy form.
  \begin{center}
    \scalebox{0.25}{\includesvg{figures/circuits/hdl/latch-mealy}}
  \end{center}
  Note that this is a simpler circuit than the corresponding string diagram
  version in \cref{ex:sr-latch-unrolled} because the tool automatically
  applies combinational reductions (in particular, the elimination rule) to
  tidy up the resulting circuit.
\end{example}

Once the circuit is translated into the correct form, inputs can be provided
and the circuit evaluated step-by-step.

\begin{example}
  We now provide inputs to the evaluator created in
  \cref{ex:sr-latch-evaluator}; recall that the first input is the
  \textsf{R}eset input and the second is the \textsf{S}et input.
  \begin{lstlisting}
    let eval = Evaluator(latch)
    eval.PerformCycle([FALSE, TRUE])
    eval.PerformCycle([FALSE, FALSE])
    eval.PerformCycle([TRUE, FALSE])
    eval.PerformCycle([FALSE, FALSE])
  \end{lstlisting}
  This automatically applies the reduction rules to determine the output values
  over time.
  The inputs detailed above produce the output stream \(
  \bot\mathsf{f} \streamcons \mathsf{t}\mathsf{f} \streamcons
  \mathsf{t}\mathsf{f} \streamcons \mathsf{f}\mathsf{t}
  \), illustrated below in the small boxes.
  This is the expected output as the delay causes the first tick of outputs to
  be underdefined.
  \begin{center}
    \includesvg[scale=0.225]{figures/circuits/hdl/latch-outputs}
  \end{center}
\end{example}

\subsection{Cyclic combinational circuits}

The use of the instant feedback rule means that the tool can also handle
circuits with non-delay-guarded feedback that exhibit combinational behaviour.

\begin{example}
  Recall the circuit from \cref{ex:cyclic-combinational}
  containing blackboxes \(f\) and \(g\), in which the control signal
  dictates the order the circuits are applied.
  \begin{lstlisting}
    // Input wires
    let x = sig.UseWire(1)
    let c = sig.UseWire(1)
    // Wire from feedback
    let feedback = sig.UseWire(1)
    // Top half of the circuit
    let muxa = UseMux2(s0: c, i0: x, i1: feedback)
    let fbb = belnapSignature.AddBlackbox("f", [Port(1, "A")], [Port(1, "Z")])
    let f = UseBlackbox(fbb, [muxa])[0]
    // Bottom half of the circuit
    let muxb = UseMux2(s0: c, i0: f, i1: x)
    let gbb = belnapSignature.AddBlackbox("g", [Port(1, "A")], [Port(1, "Z")])
    let g = UseBlackbox(gbb, [muxb])[0]
    Feedback(g, feedback)
    // Final multiplexer
    let muxc = UseMux2(s0: c, i0: g, i1: f)
    let cyclic = MakeSubcircuit(
      [InterfaceWire(X, "x"), InterfaceWire(C, "c")],
      [InterfaceWire(muxc, "Z")],
      "cyclic_combinational"
    )
  \end{lstlisting}
  This generates the following circuit:
  \begin{center}
    \includesvg[scale=0.275]{figures/circuits/hdl/cyclic-combinational/circuit}
  \end{center}
  By providing input values we can verify that this circuit truly does have
  combinational behaviour.
  \begin{lstlisting}
    let eval = Evaluator(sig, cyclic)
    eval.PerformCycle([TRUE, TRUE])
    eval.PerformCycle([TRUE, FALSE])
  \end{lstlisting}
  Due to the blackboxes, this circuit cannot be be reduced to a stream of output
  values, but it can be reduced to an expression in terms of \(f\) and \(g\).
  \begin{center}
    \includesvg[scale=0.3]{figures/circuits/hdl/cyclic-combinational/reduced}
  \end{center}
\end{example}

\subsection{Partial evaluation}

One of the benefits of the graph-rewrite-based evaluation style is that it
allows for \emph{partial evaluation}.
The tool implements some of the strategies discussed in \cref{sec:partial}, such
as tidying rules, shortcut rules, and uncertain values for reasoning with
protocols.


\begin{example}
  Recall the circuit from \cref{ex:protocols}, which reduces to the identity
  when the first input is fixed as either true or false.
  The can be designed as follows:
  \begin{lstlisting}
    let v = sig.UseWire(1)
    let w = sig.UseWire(1)
    let not = UseNot(v)
    let or1 = UseOr(not, v)
    let fb = sig.UseWire(1)
    let or2 = UseOr(fb, or1)
    let and = UseAnd(or2, w)
    DelayGuardedFeedback(and, fb)
    let circ = MakeSubcircuit(
      [InterfaceWire(v, "A"), InterfaceWire(w, "B")],
      [InterfaceWire(and, "Z")],
      "circuit"
    )
  \end{lstlisting}
  This produces the following circuit:
  \begin{center}
    \includesvg[scale=0.3]{figures/circuits/hdl/protocol/circuit}
  \end{center}
  We apply some uncertain values and partially evaluate.
  \begin{lstlisting}
    let eval = PartiallyEvaluate(
      sig, [GetVariable([TRUE, FALSE]), GetUnspecified(sig)], circ
    )
  \end{lstlisting}
  For each input provided, the tool prepends infinite waveforms containing the1
  (potentially uncertain) waveforms.
  \begin{center}
    \includesvg[scale=0.275]{figures/circuits/hdl/protocol/rewrite-1}
  \end{center}
  We can then visualise the reduction procedure step by step. First the
  waveform is propagated over the \(\notgate\) gate:
  \begin{center}
    \includesvg[scale=0.3]{figures/circuits/hdl/protocol/rewrite-2}
  \end{center}
  As the inputs to the \(\orgate\) gate can only ever produce true, this can be
  rewritten:
  \begin{center}
    \includesvg[scale=0.3]{figures/circuits/hdl/protocol/rewrite-3}
  \end{center}
  An \(\orgate\) gate with one input as true will always produce true, so this
  can be replaced and its other input eliminated:
  \begin{center}
    \includesvg[scale=0.3]{figures/circuits/hdl/protocol/rewrite-4}
  \end{center}
  The delay can also be eliminated:
  \begin{center}
    \includesvg[scale=0.3]{figures/circuits/hdl/protocol/rewrite-5}
  \end{center}
  Finally, since an \(\andgate\) with a true input acts as the identity on the
  other input, this can also be replaced:
  \begin{center}
    \includesvg[scale=0.3]{figures/circuits/hdl/protocol/rewrite-6}
  \end{center}
  This shows how the entire circuit can be automatically reduced to the
  identity.
\end{example}

While this is just a toy example, the procedure can easily be applied to more
complex circuits and potentially find optimisations.