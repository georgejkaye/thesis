\chapter{String diagrams as hypergraphs}\label{chap:hypergraphs}

String diagrams are an appealing way of reasoning with pen and paper: they bring
intuition to one-dimensional text strings and can often shed light on the
next step of a proof.
Unfortunately, they do have drawbacks: they take up a lot of time and space, and
if not drawn with care can end up being messy, removing any benefit of using
them in the first place.

Instead it is desirable to perform reasoning with string diagrams
\emph{computationally}.
This presents some questions: how can we encode circuits constructed
categorically in such a way that a computer can understand them?
Perhaps we could input terms using traditional one-dimensional text
representations?
Text is something that computers are very good at processing, but as we have
already established it is verbose, unintuitive, and, most importantly,
means we have to apply the axioms of STMCs explicitly.
Sticking to string diagrams would be ideal, but the representation
needs some thought.
Although computers do not deal well with topological objects like string
diagrams, they are very well acquainted with combinatorial objects; for
a computer to reason effectively with string diagrams, they must first be
interpreted as \emph{graphs}.

\section{String diagram rewriting}

Graph rewriting specifalised specifically for rewriting sting diagrams is a
relatively new field, first appearing at the turn of the 2010s using
\emph{string graphs}~\cite{%
    dixon2010open,dixon2013opengraphs,kissinger2012pictures%
}.
String graphs have two classes of nodes for \emph{boxes} and \emph{wires}; the
former nodes represent generators in string diagrams and the latter nodes
represent the wires between them.
Crucially, one wire in a string diagram can be represented by arbitrarily
many wire nodes connected together; all of these different depictions are
identified by a notion of \emph{wire homeomorphism}, in which adjacent wire
nodes can be collapsed into one.

\begin{center}
    \iltikzfig{graphs/examples/string-graph}
\end{center}

String graphs modulo wire homeomorphism is a suitable setting for modelling
traced or compact closed categories.
The main drawback with this presentation is that a given term may correspond to
many different graphs thanks to wire homeomorphism.

More recently, there has recently been a flurry of work on string
diagram rewriting modulo \emph{Frobenius structure} using
\emph{hypergraphs}~\cite{%
    bonchi2016rewriting,zanasi2017rewriting,bonchi2017confluence,%
    bonchi2018rewriting,bonchi2022string,bonchi2022stringa,bonchi2022stringb%
}.
Hypergraphs are a generalisation of graphs in which edges can have arbitrarily
many sources and targets, rather than just one each.
With hypergraphs, generators are represented as hyperedges, and connections
between generators indicated by if their sources and targets overlap.
The beauty of the hypergraph formalism is that there is no restriction for
nodes to only be incident on a single source and target, so one can easily
model structures such as monoids or comonoids.

\begin{center}
    \iltikzfig{graphs/examples/hypergraph}
\end{center}

It turns out when modelling string diagrams as hypergraphs, the equations of
a \emph{special commutative Frobenius algebra} are `absorbed': string diagrams
equal by Frobenius are interpreted as isomorphic hypergraphs.
This means that, in some sense, rewriting using hypergraphs can be even more
advantageous than using string diagrams.

Naturally, there have also been variations on this work where the complete
Frobenius structure is not present.
Suitable restrictions on hypergraphs and the graph rewriting process are also
identified in~\cite{bonchi2016rewriting} for rewriting
\emph{symmetric monoidal structure}.
Research followed on rewriting modulo
\emph{(co)monoid structure}~\cite{milosavljevic2023string} (`half a Frobenius')
and our work~\cite{ghica2023rewriting} on rewriting modulo
\emph{traced comonoid structure}.
The latter is the basis for this part of the thesis.
\section{Hypergraphs}

We will begin by defining the categories of hypergraphs required, following the
pattern detailed in \cite{bonchi2022string}.
Hypergraphs are formally defined as a functor category.

\begin{definition}[Hypergraph~\cite{bonchi2016rewriting}]
    Let \(\mathbf{X}\) be the category with object set
    \((\nat \times \nat) + \star\) and morphisms
    \(\morph{\sources{i}}{(k,l)}{\star}\) for each \(i < k\)
    and \(\morph{\targets{j}}{(k,l)}{\star}\) for each \(j < l\).
    The category of hypergraphs \(\hyp\) is the functor category
    \([\mathbf{X}, \set]\).
\end{definition}

One can think of the category \(\mathbf{X}\) as a `template' for the structure
of a hypergraph: the object \(\star\) represents the nodes and each object
\((k, l)\) represents hyperedges with \(k\) sources and \(l\) targets; each such
edge must pick \(k\) sources and \(l\) targets from \(\star\).

Objects in \(\hyp\) are functors that instantiates each object in \(\mathbf{X}\)
to a concrete set.
Subsequently, for a hypergraph \(F \in \hyp\) we write \(\vertices{F}\) for its
set of nodes and \(\edges{F}{k}{l}\) for the set of edges with \(k\) sources and
\(l\) targets.
Since it is a functor category, the morphisms in \(\hyp\) are natural
transformations: structure-preserving maps between hypergraphs.

\begin{definition}[Hypergraph homomorphism]
    Given two hypergraphs \(F, G \in \hyp\), \emph{hypergraph homomorphism}
    \(F \to G\) consists of functions
    \(\morph{\vertices{f}}{\vertices{F}}{\vertices{G}}\) and
    \(\morph{\edges{f}{k}{l}}{\edges{F}{k}{l}}{\edges{G}{k}{l}}\) such that the
    following diagrams commute:
    \begin{center}
    \includestandalone{figures/graphs/category/sources-preserved}
    \qquad
    \includestandalone{figures/graphs/category/targets-preserved}
\end{center}
\end{definition}

Much like with regular graphs, it is much more intuitive to draw out hypergraphs
rather than look at their combinatorial representation.
We draw nodes as black dots, and hyperedges as `bubbles' with ordered tentacles
on the left and right that connect to source and target nodes respectively.

\begin{example}
    \todo[inline]{Do hypergraph example}
\end{example}

\subsection{Labelled hypergraphs}

From the example drawn above, it should be clear to see how hypergraphs are a
suitable representation of string diagrams: generators correspond to hyperedges
and wires to the nodes between them.
However, the hyperedges are currently not \emph{labelled} with generator
symbols.
To do this, we must first translate the notion of signature to hypergraphs.

\begin{definition}[Hypergraph signature~\cite{bonchi2016rewriting}]
    For a set of generators \(\signature\) and sorts \(\mcs\) as defined
    in \cref{def:generators}, the \emph{hypergraph signature}
    \(\hypsignature{(\Sigma, S)} \in \hyp\) is defined as follows:
    \begin{gather*}
        \vertices{\hypsignature{\Sigma}} := \{ v_n \,|\, s \in \mcs\}
        \quad
        \edges{\hypsignature{\Sigma}}{k}{l} := \{ e_g \,|\, g \in \signature\}
        \\
        \sources{i}(e_g) := v_{\dom[e_g](i)}
        \quad
        \targets{j}(e_g) := v_{\cod[e_g](j)}
    \end{gather*}
\end{definition}

\begin{example}
    \todo[inline]{Do hypergraph signature example}
\end{example}

The vertices and edges of a hypergraph \(F\) can then be assigned sorts from
\(\mcs\) and symbols from \(\Sigma\) using a homomorphism
\(F \to \hypsignature{\Sigma,\mcs}\).
To do this to \emph{all} the hypergraphs in \(\hyp\) and create a category of
\emph{labelled} hypergraphs, we make use of some more categorical machinery.

\begin{definition}[Slice category~\cite{lawvere1963functorial}]
    For a category \(\mathbf{C}\) and an object \(C \in \mathbf{C}\), the
    \emph{slice category} \(\mathbf{C} \slice C\) has objects the morphisms of
    \(\mathbf{C}\) with target \(C\) and morphisms
    \((\morph{f}{X}{C}) \to (\morph{g}{X^\prime}{C})\) the morphisms
    \(\morph{g}{X}{X^\prime} \in \mathbf{C}\) such that \(f^\prime\circ g = f\).
\end{definition}

\begin{definition}[Labelled hypergraphs~\cite{bonchi2016rewriting}]
    Let \(\hypsigmac\) be the category of hypergraphs labelled over a set of
    generators \(\Sigma\) and sorts \(\mcs\), defined as the slice category
    \(\hyp \slice \hypsignature{(\Sigma, \mcs)}\).
\end{definition}

\begin{example}
    \todo[inline]{Do labelled hypergraph example}
\end{example}
\subsection{Cospans of hypergraphs}

String diagrams also have \emph{input} and \emph{output} interfaces.
(Labelled) hypergraphs may have suggestively dangling nodes in the pictures,
but this is not actually encoded in the definition, and moreover we may wish to
set a non-dangling node as an input or output.
To set the interfaces of a hypergraph, hypergraph homomorphisms are used
to `pick' the appropriate nodes.

\begin{definition}[Cospan]
    A \emph{cospan} in a category \(\mcc\) is a pair of morphisms \(X \to A\)
    and \(X \to B\) in \(\mcc\), written \(\cospan{X}{A}{Y}\).
    A \emph{cospan morphism} \(
        (\cospan{X}[f]{A}[g]{Y}) \to (\cospan{X}[h]{B}[k]{Y})
    \) is a morphism \(\morph{\alpha}{A}{B}\) in \(\mcc\)
    such that the following diagram commutes:
    %
    \begin{center}
        \includestandalone{figures/category/diagrams/cospan-morphism}
    \end{center}
%
    Two cospans \(\cospan{X}{A}{Y}\) and \(\cospan{X}{B}{Y}\) are
    \emph{isomorphic} if there exists a morphism of cospans as above where
    \(\alpha\) is an isomorphism.
\end{definition}

As with all the constructions so far, cospans must be assembled into a category
to be useful for our purpose.
This means a notion of \emph{composition} of cospans is required.

\begin{definition}[Composition of cospans]
    \label{def:cospan-composition}
    In a category \(\mcc\) with pushouts, the composition of cospans
    \(\cospan{X}[f]{A}[g]{Y}\) and \(\cospan{Y}[h]{B}[k]{Z}\) is by pushout:
    \begin{center}
        \includestandalone{figures/category/diagrams/cospan-composition}
    \end{center}
\end{definition}

\begin{definition}[Categories of cospans]
    Let \(\mcc\) be a category with pushouts and an initial object.
    The category of cospans over \(\mathbf{C}\), denoted \(\csp{\mathbf{C}}\),
    has as objects the objects of \(\mathbf{C}\) and as morphisms \(A \to B\)
    the isomorphism classes of cospans \(\cospan{A}{X}{B}\) for some
    \(X \in \mcc\).
    Composition is by pushout as detailed in \cref{def:cospan-composition} and
    the identity is \(X \xrightarrow{\id[X]} X \xleftarrow{\id[X]} X\).

    This category is symmetric monoidal with tensor given by the coproduct in
    \(\mathbf{C}\), unit the initial object \(0 \in \mathbf{C}\), and symmetry
    by \(\cospan{A+B}{A+B}{B+A}\).
\end{definition}

Interfaces are assigned to a hypergraph \(F\) by having it occupy the `apex' of
a cospan and having the `legs' on either side pick inputs and outputs
respectively.

\begin{definition}[Discrete hypergraph]
    A \emph{discrete hypergraph} is a hypergraph in which each edge set is
    empty.
\end{definition}

\subsection{Ordering the interfaces}

Even though our hypergraphs still have input and output interfaces specified
using cospans, there is still information missing: what is the \emph{ordering}
of the vertices in these interfaces?
Without this data \(\csp{\hypsigmac}\) is not even a PROP.

A cospan \(\cospan{M}{F}{N}\) is a morphism
\(M \to N \in \csp{\hypsigmac}\).
Therefore, to restrict this category to a coloured PROP, we need to ensure that
the legs of each cospan of hypergraphs can be viewed as a word in
\(\freemon{C}\) for some countable set of colours \(C\).
This is formally performed by another functor.

\begin{theorem}[\cite{bonchi2022string}, Thm. 3.6]
    Let \(\mathbb{X}\) be a coloured PROP whose monoidal product is a coproduct,
    \(\mathbf{C}\) a category with pushouts and an initial object, and \(
        \morph{F}{\mathbb{X}}{\mathbf{C}}
    \) a coproduct-preserving functor.
    Then there exists a coloured PROP \(\csp[F]{\mathbf{C}}\) whose arrows
    \(\listvar{m} \to \listvar{n}\) are isomorphism classes of \(\mathbf{C}\)
    cospans \(\cospan{F\listvar{m}}{C}{F\listvar{n}}\).
\end{theorem}

\(F\) is the functor that imbues the objects in the legs of the cospan with the
structure of words in some coloured PROP \(\mathbb{X}\).
In our case, this PROP will be the PROP of coloured finite sets.

\begin{definition}
    Let \(\finsetprop\) be the PROP with morphisms \(m \to n\) the functions
    between finite sets \([m] \to [n]\).
\end{definition}

\(\finsetprop\) is a (monochromatic) PROP; as with hypergraphs colours can be
assigned using a slice category.
Since we are working with potentially countably infinite sets of colours, the
definition of \(\finsetprop\) must first be tweaked.

\begin{definition}
    Let \(\finsetpropwithnat\) be the category \(\finsetprop\) augmented with the
    set of natural numbers and the functions \([m] \to \nat\) for each finite
    set \([m]\).
\end{definition}

Then the PROP of finite sets \emph{coloured} over some countable set \(C\) is
the slice \(\finsetpropwithnat \slice C\).
Objects of this category are pairs \(([m], \morph{w}{[m]}{C})\); this pair can
be viewed as a word in \(\freemon{C}\) of length \(m\), with the \(i\)th letter
as \(w(i)\).

\begin{remark}
    Note that we do not include the morphisms \(\nat \to [m]\) in
    \(\finsetpropwithnat\); this is because when we view objects of
    \(\finsetpropwithnat \slice C\) as words in \(\freemon{C}\), we still only
    want to consider finite words despite there being potentially countably
    infinite colours.
\end{remark}

All that remains is to verify that \(\finsetpropwithnat \slice C\) is indeed a
coloured PROP.

\begin{lemma}
    \label{lem:slice-coproducts}
    For a category \(\mcc\) with coproducts, \(\mcc \slice X\) has coproducts.
\end{lemma}
\begin{proof}
    Let \(A,B,X\) be objects in \(\mcc\); as \(\mcc\) has coproducts \(A + B\)
    is also an object in \(\mcc\).
    Then the coproduct of \((A, A \to X)\) and \((B, B \to X)\) in
    \(\mcc \slice X\) is \(A + B \to X\); the universal morphism is \([f, g]\).
\end{proof}

\begin{proposition}
    \label{prop:hatfinsetprop-slice-is-coloured-prop}
    For a countable set \(C\), \(\finsetpropwithnat \slice C\) is a
    coloured PROP.
\end{proposition}
\begin{proof}
    This follows the same strategy as \cite[Prop. 2.23]{bonchi2022string}.
    As established, the objects of \(\finsetpropwithnat \slice C\) can be viewed
    as words in \(\freemon{C}\).
    As slice categories preserve coproducts by \cref{lem:slice-coproducts},
    \(\finsetpropwithnat \slice C\) is strict symmetric monoidal, and the
    coproduct acts as concatenation of words.
\end{proof}

We can now state the functor used to assemble interfaces of hypergraphs into
words.

\begin{definition}[\cite{bonchi2022string}, Rem. 3.12]
    Let \(\morph{\pickinterfacesc{C}}{\finsetpropwithnat \slice C}{\hypsigmac}\)
    be defined as functor sending a word \(\overline{n}\) to the corresponding
    discrete coloured hypergraph containing vertices coloured as in
    \(\overline{n}\), and sending a function \(\overline{m} \to \overline{n}\)
    to the induced homomorphism of discrete hypergraphs.
\end{definition}

Subsequently we obtain a coloured PROP \(
    \csp[\pickinterfacesc{C}]{\hypsigmac}
\), which will serve as the domain in which we interpret string diagrams
combinatorially.
\section{Frobenius terms as hypergraphs}

It is now time to establish the correspondences between cospans of hypergraphs
and string diagrams.
We will first recount the constructions used by Bonchi et al in
\cite{bonchi2022string} for settings with a Frobenius structure before showing
how we can use their ingredients for a \emph{traced} setting either with or
without a comonoid structure.

\subsection{Frobenius structure}

Monoids and comonoids are structures that crop up all over mathematics.
However, things become really interesting when multiple such structures
\emph{interact}.
One particular combination of a monoid and comonoid structure is known as a
\emph{Frobenius structure}; such structures are particularly relevant to us
because symmetric monoidal terms equipped with a Frobenius structure correspond
precisely to the cospans of hypergraphs defined in the previous section.

\begin{definition}
    \label{def:frob}
    The monoidal theory of \emph{special commutative Frobenius algebras} is
    defined as \((\generators[\frob], \equations[\frob])\), where \(
        \generators[\frob] \coloneqq \{
            \iltikzfig{strings/structure/monoid/merge}[colour=white],
            \iltikzfig{strings/structure/monoid/init}[colour=white],
            \iltikzfig{strings/structure/comonoid/copy}[colour=white],
            \iltikzfig{strings/structure/comonoid/discard}[colour=white]
        \}
    \) and the equations of \(\equations[\frob]\) are listed in
    \cref{fig:frobenius-equations}.
    We write \(\frob \coloneqq \smc{\generators[\frob], \equations[\frob]}\).
\end{definition}

\begin{figure}[t]
    \centering
    \iltikzfig{strings/structure/frobenius/frobenius-l}
    \(=\)
    \iltikzfig{strings/structure/bialgebra/merge-copy-lhs}
    \quad
    \iltikzfig{strings/structure/frobenius/frobenius-r}
    \(=\)
    \iltikzfig{strings/structure/bialgebra/merge-copy-lhs}
    \quad
    \iltikzfig{strings/structure/frobenius/copy-merge-lhs}
    \(=\)
    \iltikzfig{strings/structure/frobenius/copy-merge-rhs}
    \caption{
        Equations \(\equations[\frob]\) of a
        \emph{special commutative Frobenius algebra}, in addition to those in
        \cref{fig:monoid-equations,fig:comonoid-equations}.
    }
    \label{fig:frobenius-equations}
\end{figure}

The equations of special Frobenius algebras are those of commutative monoids and
commutative comonoids along with the `Frobenius' and the 'special' equation.

\begin{example}
    The following are all terms in \(\frob\):
    \[
        \iltikzfig{strings/structure/frobenius/example-1}
        \quad
        \iltikzfig{strings/structure/frobenius/example-2}
        \quad
        \iltikzfig{strings/structure/frobenius/example-3}
    \]
    Using the equations of \(\equations[\frob]\), it can be shown that the
    latter two terms are equal:
    \begin{gather*}
        \iltikzfig{strings/structure/frobenius/example-2}
        \eqaxioms[\monoidunitleqn]
        \iltikzfig{strings/structure/frobenius/example-equational/step-1}
        \eqaxioms[\frobleqn]
        \iltikzfig{strings/structure/frobenius/example-equational/step-2}
        \eqaxioms[\monoidassoceqn]
        \iltikzfig{strings/structure/frobenius/example-equational/step-3}
        \eqaxioms[\frobreqn]
        \iltikzfig{strings/structure/frobenius/example-3}
    \end{gather*}
\end{example}

\(\frob\) is a monochromatic PROP.
To define a \emph{coloured} version of \(\frob\) we simply use a different copy
of \(\frob\) to represent each colour, using a fact about \(\propcat\) and
\(\cprop\).

\begin{theorem}[\cite{baez2018props}, Corollary 5.3]
    \(\propcat\) has coproducts.
\end{theorem}

This easily generalises to \(\cprop\) by replacing natural numbers with words.
This means that given coloured PROPs \(\mcc\) and \(\mcd\) with objects the
words in \(\freemon{C}\) and \(\freemon{D}\) respectively, there is also a
coloured PROP \(\mcc + \mcd\) with objects the words in \(\freemon{(C + D)}\)
and morphisms defined in the obvious way.
We can use this to define a multi-coloured version of \(\frob\) as
a coproduct of copies of \(\frob\).

\begin{definition}[\cite{bonchi2022string}]
    \label{def:frobc}
    For a countable set of colours \(C\), let \(\frobc \in \cprop\) be
    defined as \(\frobc \coloneqq \sum_{c \in C}\frob\).
\end{definition}

\subsection{Hypergraph categories}

Frobenius structures have turned out to be very useful in studying compositional
processes such as quantum processes~\cite{coecke2008interacting} and signal flow
graphs~\cite{bonchi2014categorical,bonchi2015full}.
As with other structures, it is useful to talk about the categorical setting in
which \emph{every} object in a category is equipped with such a structure.

\begin{definition}[Hypergraph category~\cite{fong2019hypergraph}]
    \label{def:hypergraph-category}
    A \emph{hypergraph category} is a category in which every object is equipped
    with a special commutative Frobenius algebra subject to the
    \emph{coherence equations} in \cref{fig:hypergraph-coherence}.
\end{definition}

\begin{figure}
    \centering
    \begin{gather*}
        \iltikzfig{strings/structure/monoid/coherence-monoid-lhs}[obj1=A,obj2=B]
        =
        \iltikzfig{strings/structure/monoid/coherence-monoid-rhs}[obj1=A,obj2=B]
        \qquad
        \iltikzfig{strings/structure/monoid/coherence-unit-lhs}[obj1=A,obj2=B]
        =
        \iltikzfig{strings/structure/monoid/coherence-unit-rhs}[obj1=A,obj2=B]
        \qquad
        \iltikzfig{strings/structure/comonoid/coherence-comonoid-lhs}[obj1=A,obj2=B]
        =
        \iltikzfig{strings/structure/comonoid/coherence-comonoid-rhs}[obj1=A,obj2=B]
        \qquad
        \iltikzfig{strings/structure/comonoid/coherence-counit-lhs}[obj1=A,obj2=B]
        =
        \iltikzfig{strings/structure/comonoid/coherence-counit-rhs}[obj1=A,obj2=B]
    \end{gather*}
    \caption{Coherence equations of a hypergraph category}
    \label{fig:hypergraph-coherence}
\end{figure}

\begin{remark}
    The notion of a hypergraph category is so natural that it has been
    rediscovered numerous times over the years!
    They were originally called \emph{well-supported compact closed categories}
    by Carboni and Walters~\cite{carboni1987cartesian}, and have subsequently
    appeared as
    \emph{dgs-monoidal categories}~\cite{katis1997bicategories,gadducci1998inductive,gadducci1999bicategorical,bruni2002normal}
    and \emph{dungeon categories}~\cite{morton2014belief}.
    The term \emph{hypergraph categories} was coined more recently but has
    become the standard in the compositional processes
    community~\cite{kissinger2015finite,fong2015decorated,baez2016compositional,baez2018compositional}.
\end{remark}

As the Frobenius structure is their entire raison d'\^{e}tre, it is unsurprising
that two examples of hypergraph categories are the PROPs defined in
\cref{def:frob} and \cref{def:frobc}.

\begin{lemma}
    \(\frob\) and \(\frobc\) are hypergraph categories.
\end{lemma}
\begin{proof}
    The generators in \(\frob\) and \(\frobc\) provide the Frobenius
    structure for the object \(1\) in the former and the base colours
    \(c \in C\) in the latter.
    The Frobenius structure for the other objects constructed through
    \(\tensor\) can be derived by parallel composition, following the recipes
    given in the coherence equations in \cref{fig:hypergraph-coherence}.
\end{proof}

It is now possible to formally define what we mean when we say `Frobenius
terms'.

\begin{definition}
    For a set of colours \(C\) and set of generators over \(C\) \(\generators\),
    let \(\hypsigmac\) be the coloured PROP freely generated over the
    generators in \(\generators\) and those in \(\frobc\).
\end{definition}

To conclude this section we make a simple observation regarding \(\hypcsigmac\).

\begin{lemma}
    \(\hypcsigmac \cong \smcsigmac + \frobc\)
\end{lemma}
\begin{proof}
    Every term in \(\hypcsigmac\) can be expressed as a combination of
    generators either in \(\smcsigmac\) or \(\frobc\).
\end{proof}

Viewing \(\hypcsigmac\) as a coproduct in \(\cprop\) will prove to be beneficial
when establishing a correspondence between \(\hypcsigmac\) and \(\cspdchyp\) in
the upcoming sections.

\subsection{Hypergraph categories and hypergraphs}

Perhaps confusingly, the category of \emph{hypergraphs} \(\hypsigmac\) is
\emph{not} a hypergraph category, but the category of \emph{cospans} of
hypergraphs is!
This can be shown by exploiting a correspondence between \(\frob\) and the PROPs
of finite sets we encountered earlier.

\begin{proposition}[\cite{lack2004composing}, Ex. 5.4]
    \label{prop:frob-finset}
    There is an isomorphism of PROPs \(\frob \cong \csp{\finsetprop}\).
\end{proposition}

We omit the formal proof and sketch the correspondence.
Terms in \(\frob\) are formed of all the ways of combining \(
    \iltikzfig{strings/structure/monoid/merge}[colour=white],
    \iltikzfig{strings/structure/monoid/init}[colour=white],
    \iltikzfig{strings/structure/comonoid/copy}[colour=white],
    \iltikzfig{strings/structure/comonoid/discard}[colour=white],
    \iltikzfig{strings/category/identity}[colour=white],
\) and \(
    \iltikzfig{strings/symmetric/symmetry}[colour=white]
\) in sequence and parallel, so a string diagram for a term \(\morph{f}{m}{n}\)
is depicted as \(x\) connected components drawing paths from \(m\) inputs to
\(n\) outputs, such as in the example below.

\begin{center}
    \iltikzfig{strings/structure/frobenius/example}
\end{center}

Note there is no requirement for each component to connect to one or both
interfaces as the \(
    \iltikzfig{strings/structure/monoid/init}[colour=white]
\) and \(
    \iltikzfig{strings/structure/comonoid/discard}[colour=white]
\) generators can introduce and stub wires.
A term \(\morph{f}{m}{n}\) with \(x\) connected components corresponds to
a cospan of finite sets \(\cospan{[m]}[i]{[x]}[j]{[n]}\), where the functions
\(i\) and \(j\) map the inputs and outputs to the components they connect to.

\begin{example}
    Consider the term \(\morph{f}{5}{4}\) drawn on the left below.
    This corresponds to a cospan \(\cospan{[5]}{[3]}{[4]}\) as shown on the
    right below.
    \begin{center}
        \iltikzfig{strings/structure/frobenius/example}
        \(\Leftrightarrow\)
        \scalebox{0.75}{\tikzfig{strings/structure/frobenius/example-cospan}}
    \end{center}
\end{example}

The cospan representation shows how all connected Frobenius components can be
`squished' into a single blob.

This result shows how the correspondence works for the monochromatic case; what
about for terms coloured by elements of some countable set?
Here, we replace \(\finsetprop\) with the coloured version
\(\finsetpropwithnat \slice C\) seen in the previous section.
A coloured version of \cref{prop:frob-finset} was shown for a \emph{finite} set
of colours in \cite{bonchi2022string}; we recall its proof before extending this
to the \emph{countable} setting we work in.

\begin{lemma}
    \label{lem:slice-iso-terminal}
    In a category \(\mcc\) with a terminal object \(1\),
    \(\mcc \cong \mcc \slice 1\).
\end{lemma}
\begin{proof}
    Since \(1\) is terminal, there is a unique morphism \(A \to 1\) for each
    object \(A\) in \(\mcc\), so there is an object \((A, \morph{!_A}{A}{1})\)
    in \(\mcc \slice 1\) for each object \(A \in \mcc\).
    In \(\mcc \slice 1\) there is a morphism
    \((A, \morph{!_A}{A}{1}) \to (B, \morph{!_B}{B}{1})\) in
    for every morphism \(\morph{f}{A}{B} \in \mcc\) such
    that \(f \seq !_B = !_A\); since both \(f \seq !_B\) and \(!_A\) are
    morphisms \(A \to 1\) they must be the same unique morphism.
    Therefore \(\mcc \cong \mcc \slice 1\).
\end{proof}

\begin{theorem}[\cite{bonchi2022string}, Theorem 2.24]
    \label{thm:frobc-iso-finset-slice-c}
    For a finite set of colours \(C \in \finsetprop\), there is an isomorphism
    of coloured PROPs \(\frobc \cong \csp{\finsetprop \slice C}\).
\end{theorem}
\begin{proof}
    By definition of \(\frobc\),
    \cref{def:frobc,prop:frob-finset,lem:slice-iso-terminal}
    we have that \[
        \frobc
        \coloneqq
        \sum_{c \in C}\frob
        \cong
        \sum_{c \in C}\csp{\finsetprop}
        \cong
        \sum_{c \in C}\csp{\finsetprop \slice 1}
    \]
    In the other direction we have that \(
        \csp{\finsetprop \slice C}
        \cong
        \csp{\finsetprop \slice \sum_{c \in C} 1}
    \) as \(C\) is countable.
    So we need to show that \(
        \sum_{c \in C}\csp{\finsetprop \slice 1}
        \cong
        \csp{\finsetprop \slice \sum_{c \in C} 1}
    \).
    The objects of the former are coproducts of objects in
    \(\finsetprop \slice C\); as this is a coloured prop the coproduct is
    concatenation and subsequently the objects can be viewed as words in
    \(\freemon{C}\).
    Similarly, the objects of the latter are objects of
    \(\finsetprop \slice \sum_{c \in C} 1\), which can clearly also be
    seen as words in \(\freemon{C}\).

    The morphisms of the former are coproducts of cospans, which can
    equivalently be viewed as a single cospan with coproducts in the legs and
    apex; using the reasoning above this means it is a cospan of words in
    \(\freemon{C}\); it is easy to see that this is also the case for morphisms
    in the latter.
\end{proof}

We need to show a version of this for the case where \(C\) may be
\emph{countably infinite}.
The strategy is much the same as, but relies on one small observation.

\begin{lemma}
    \label{lem:finsetprop-finite}
    Let \(C \in \finsetprop\) be a finite cardinal.
    Then \(\finsetpropwithnat \slice C \cong \finsetprop \slice C\).
\end{lemma}
\begin{proof}
    The morphisms in \(\finsetpropwithnat \slice C\) are the morphisms
    \([m] \to C\) for finite \(C\), which are precisely the morphisms of
    \(\finsetprop \slice C\).
\end{proof}

As \(1\) is a finite cardinal, this slips in to the proof above to extend
it to \emph{countable} sums.

\begin{theorem}
    \label{thm:frobc-iso-hatfinset-slice-c}
    For a countable set of colours \(C\), there is an isomorphism of coloured
    PROPs \(\frobc \cong \csp{\finsetpropwithnat \slice C}\).
\end{theorem}
\begin{proof}
    The proof is almost the same as \cref{thm:frobc-iso-finset-slice-c} but with
    the addition of \cref{lem:finsetprop-finite}.
    We have that \[
        \frobc
        \coloneqq
        \sum_{c \in C}\frob
        \cong
        \sum_{c \in C}\csp{\finsetprop}
        \cong
        \sum_{c \in C}\csp{\finsetprop \slice 1}
        \cong
        \sum_{c \in C}\csp{\finsetpropwithnat \slice 1}.
    \]
    In the other direction we still have that \(
        \csp{\finsetpropwithnat \slice C}
        \cong
        \csp{\finsetpropwithnat \slice \sum_{c \in C} 1}
    \) as \(C\) is still countable.
    As before we need to show that \(
        \sum_{c \in C}\csp{\finsetpropwithnat \slice 1}
        \cong
        \csp{\finsetpropwithnat \slice \sum_{c \in C} 1}
    \), which follows by the same reasoning as in the prequel.
\end{proof}

We have now ascertained the relationship between \(\frobc\) and
\(\csp{\finsetpropwithnat \slice C}\).
The missing link is the relationship between the latter and \(\cspdchyp\);
this arises as a special case of the following theorem.

\begin{theorem}[\cite{bonchi2022string}, Thm. 3.8]
    \label{thm:cospan-homomorphism}
    Let \(\mathbb{X}\) be a PROP whose monoidal product is a coproduct,
    \(\mathbf{C}\) a category with pushouts and an initial object, and
    \(\morph{F}{\mathbb{X}}{\mathbf{C}}\) a coproduct-preserving functor.
    Then there is a homomorphism of PROPs \(
        \morph{\tilde{F}}{\csp{\mathbb{X}}}{\csp[F]{\mathbf{C}}}
    \) that sends \(\cospan{\listvar{m}}[f]{X}[g]{\listvar{{n}}}\) to
    \(\cospan{F\listvar{m}}[Ff]{FX}[Fg]{F\listvar{n}}\).
    If \(F\) is full and faithful, then \(\tilde{F}\) is faithful.
\end{theorem}
\begin{proof}
    Since \(F\) preserves finite colimits, it preserves composition (pushout)
    and monoidal product (coproduct); symmetries are clearly preserved.
    To show that \(\tilde{F}\) is faithful when \(F\) is full and faithful,
    suppose that \(
        \tilde{F}(\cospan{m}[f]{X}[g]{n})
        =
        \tilde{F}(\cospan{m}[f^\prime]{X}[g^\prime]{n})
    \).
    This gives us the following commutative diagram in \(\mathbf{C}\):
    \begin{center}
        \begin{tikzcd}
            & FX \arrow{dd}{\phi} & \\
            Fm \arrow{ur}{Ff} \arrow{dr}{Ff^\prime} & &
            Fn \arrow{ul}{Fg} \arrow{dl}{Fg^\prime} \\
            & FY &
        \end{tikzcd}
    \end{center}
    where \(\phi\) is an isomorphism as objects in \(\csp[F]{\mathbf{C}}\) are
    isomorphism classes of cospans.
    As \(F\) is full, there exists \(\morph{\psi}{X}{Y}\) such that
    \(F\psi = \phi\).
    As \(F\) is faithful, \(\psi\) is an isomorphism; this means
    \(\cospan{m}[f]{X}[g]{n}\) and \(\cospan{m}[f^\prime]{X}[g^\prime]{n}\) are
    equal in \(\csp{\mathbb{X}}\), so \(\tilde{F}\) is faithful.
\end{proof}

\begin{corollary}[\cite{bonchi2022string}, Cor. 3.9]
    \label{cor:finset-to-hyp}
    There is a faithful PROP homomorphism
    \(\morph{\tilde{\pickinterfacesc{C}}}{\csp{\finsetpropwithnat}}{\cspdchyp}\)
\end{corollary}

With these results we can derive a map from Frobenius terms to cospans of
hypergraphs.

\begin{definition}
    Let \(\morph{\frobtohypsigmac}{\frobc}{\cspdchyp}\) be the PROP morphism
    defined by using \cref{thm:frobc-iso-hatfinset-slice-c} followed by
    \cref{cor:finset-to-hyp}.
\end{definition}

\begin{example}
    If we let the set of colours \(C \coloneqq \{\bullet\}\), then the action of
    \(\frobtohypsigmac\) on the Frobenius generators is as follows:
    \begin{gather*}
        \frobtohyp[
            \iltikzfig{strings/structure/monoid/merge}[colour=white]
        ]{C,\generators}
        =
        \iltikzfig{graphs/frobenius/monoid}
        \quad
        \frobtohyp[
            \iltikzfig{strings/structure/monoid/init}[colour=white]
        ]{C,\generators}
        =
        \iltikzfig{graphs/frobenius/unit}
        \\
        \frobtohyp[
            \iltikzfig{strings/structure/comonoid/copy}[colour=white]
        ]{C,\generators}
        =
        \iltikzfig{graphs/frobenius/comonoid}
        \quad
        \frobtohyp[
            \iltikzfig{strings/structure/comonoid/discard}[colour=white]
        ]{C,\generators}
        =
        \iltikzfig{graphs/frobenius/counit}
    \end{gather*}
\end{example}

As there is a faithful embedding of \(\frobc\) into \(\cspdchyp\) and both
categories share the same objects, we also get the result we alluded to at the
start of this section.

\begin{corollary}
    \label{cor:csphypsigmac-hypergraph}
    \(\cspdchyp\) is a hypergraph category.
\end{corollary}

There is another useful observation to make which follows from the isomorphism
in \cref{thm:frobc-iso-hatfinset-slice-c}.

\begin{corollary}
    \label{cor:discrete-hypergraph-frob}
    Given a discrete hypergraph \(X \in \hypsigmac\), any cospan
    \(\cospan{\overline{m}}{X}{\overline{n}}\) in \(\cspdchyp\) is in the
    image of \(\frobtohypsigmac\).
\end{corollary}

\subsection{From terms to graphs}

Our goal is to map from terms in \(\hypcsigmac\) into cospans in \(\cspdchyp\).
As we know that \(\hypcsigmac\) can be viewed as the coproduct
\(\smcsigmac + \frobc\), it suffices to define this map separately from each
category.
The results of the previous section gives us the map \(\frobc \to \cspdchyp\);
all that remains is the map from \(\smcsigmac\).

\begin{definition}[\cite{bonchi2022string}, Sec. 4.1]\label{def:hyp-morphisms}
    Let \(\morph{\termtohypsigmac}{\smcsigmac}{\cspdchyp}\) be a coloured PROP
    morphism with the action on generators defined as
    \begin{gather*}
        \termtohypsigmac[\iltikzfig{%
                strings/category/generator%
            }[box=\phi,colour=white,dom=\listvar{m},cod=\listvar{n}]
        ]
        \coloneqq
        \cospan{\listvar{m}}{\iltikzfig{graphs/terms/generator}}{\listvar{n}}
    \end{gather*}
    where the source and target nodes are coloured appropriately.
\end{definition}

To map from terms in a hypergraph category to cospans of hypergraphs, we simply
put the two maps together.

\begin{definition}
    Let \(
        \morph{\termandfrobtohypsigmac}{\smcsigmac + \frobc}{\cspdchyp}
    \) be the copairing of \(\termtohypsigmac\) and
    \(\frobtohypsigmac\).
\end{definition}

Already we have all we need to state one of the key results of
\cite{bonchi2022string}: the correspondence between hypergraph terms and
cospans of hypergraphs.

\begin{theorem}[\cite{bonchi2022string}, Theorem 4.4]
    \label{thm:isomorphism-smcfrob-cospans}
    There is an isomorphism of \(C\)-coloured PROPs
    \(\smcsigmac + \frobc \cong \cspdchyp\).
\end{theorem}
\begin{proof}
    Since \(\smcsigmac + \frobc\) is a coproduct in \(\cprop\), this can be
    shown by proving that \(\cspdchyp\) satisfies the universal property of the
    coproduct: given a coloured PROP \(\mathbb{A}\) and PROP morphisms
    \(\smcsigmac \to \mathbb{A}\) and \(\frobc \to \mathbb{A}\), there exists
    a unique morphism \(\morph{u}{\cspdchyp}{\mathbb{A}}\) as below:
    %
    \begin{center}
        \includestandalone{figures/graphs/isomorphism/coproduct-iso}
    \end{center}
    %
    All the PROP morphisms involved are identity-on-objects, so all that is
    required to show the existence of \(u\) is to show that any morphism in
    \(\cspdchyp\) can be expressed as a composition of components either in the
    image of \(\termtohypsigmac\) or \(\frobtohypsigmac\).

    Consider a cospan \(\cospan{m}[f]{G}[g]{n}\) in \(\cspdchyp\); let \(N\) be
    the set of nodes, let \(E\) be the set of hyperedges, and let
    \(\morph{\chi}{E}{\generators}\) be the induced labelling function.
    Pick an order \(e_0, e_1, e_{j-1}\) on the edges; then define \(
        \cospan{\tilde{m}}[s]{\tilde{E}}[t]{\tilde{n}}
    \) as the cospan \(
        \bigtensor_{0 \leq i < j}
        \termtohypsigmac[\chi(e_i)]
    \).
    This cospan `stacks up' the edges in \(G\) without connecting them together;
    the legs of the cospan are the sources and targets of these
    edges concatenated in the order specified.
    It is easy to define functions \(\morph{f^\prime}{m}{N}\),
    \(\morph{g^\prime}{n}{N}\), \(\morph{h}{\tilde{m}}{N}\) and
    \(\morph{k}{\tilde{n}}{N}\) that send nodes to the corresponding node in the
    set of all nodes in the graph.

    With this data, the original cospan \(\cospan{m}[f]{G}[g]{n}\) can be
    viewed as the following composition of cospans: \[
        (
            \cospan{m}[f^\prime]{N}[(\id,h)]{N \tensor \tilde{n}}
        ) \seq (
            \cospan{
                N \tensor \tilde{n}
            }[
                \id \tensor s
            ]{
                N \tensor \tilde{E}
            }[
                \id \tensor t
            ]{
                N \tensor \tilde{m}
            }
        ) \seq (
            \cospan{N \tensor \tilde{m}}[(\id, k)]{N}[g^\prime]{n}
        ).
    \]
    This is all well-defined because \(\tensor\) is the coproduct in
    \(\hypsigmac\).
    By computing the composition by pushout, it can be shown that the composite
    above is isomorphic to the original cospan \(\cospan{m}[f]{G}[g]{n}\).

    Now it must be verified that each cospan in the composite is in the image
    of either \(\termtohypsigmac\) or \(\frobtohypsigmac\).
    The outer cospans are discrete so they are in the image of
    \(\frobtohypsigmac\) by \cref{cor:discrete-hypergraph-frob}.
    The centre cospan is constructed from the identity and cospans in the image
    of \(\termtohypsigmac\), so the entire cospan is in the
    image of \(\termtohypsigmac\).

    The morphism \(u\) can therefore be defined based on the actions of
    \(\termtohypsigmac\) and \(\frobtohypsigmac\).
    This morphism is unique: although different orders can be assigned on the
    edges, all the categories are symmetric monoidal so this is not an issue.
\end{proof}

At first glance, the composite cospan described above might look confusing.
As mentioned, the central cospan \(\cospan{\tilde{m}}{\tilde{E}}{\tilde{n}}\)
serves to `stack up' the edges in some order, all detached from each other.
To make the entire cospan isomorphic to the original, the connections of the
sources and targets must be the same: the job of the two outer cospans is to
`join them up' appropriately by connecting targets on the right to sources on
the left by going `over the top' of the edges via the identity cospan
\(\cospan{N}{N}{N}\).

By the same reasoning, this result can be extended to the coloured setting.

\begin{corollary}[\cite{bonchi2022string}, Cor. 4.2]
    There is an isomorphism of \(C\)-coloured PROPs
    \(\hypsigmac \cong \cspdchyp\).
\end{corollary}
\section{Symmetric monoidal terms}

The results of the previous section show that cospans of hypergraphs are an
excellent fit for reasoning about terms in a freely generated hypergraph
category.
However, there are times we might not have so much structure in our terms;
indeed for our case of digital circuits we only operate in a setting with a
trace.
This means that not every cospan of hypergraphs will correspond to a valid term.
Fortunately, Bonchi et al also characterised the cospans of hypergraphs that
correspond to \emph{symmetric monoidal} terms without any additional structure.
We will use some of this machinery when it comes to tackling the traced case.

\subsection{Monogamy}

There are two features that distinguish vanilla symmetric monoidal terms from
Frobenius terms; wires cannot arbitrarily fork or join, and cycles may not be
created.
The former is tackled by a condition on the connectivity of vertices.

\begin{definition}[Degree (\cite{bonchi2022stringa}, Def. 12)]
    For a hypergraph \(F \in \hyp\), the \emph{degree} of a vertex
    \(v \in \vertices{F}\) is a tuple \((i,o)\) where \(i\) is the number of
    pairs \((e,i)\) where \(e\) is a hyperedge with \(v\) as its \(i\)th target,
    and \(o\) is similarly the number of pairs \((e,j)\) where \(e\) is a
    hyperedge with \(v\) as its \(j\)th target.
\end{definition}

\begin{definition}[Monogamy (\cite{bonchi2022stringa}, Def. 13)]
    A cospan \(\cospan{\listvar{m}}[f]{F}[g]{\listvar{n}}\) in \(\cspdchyp\) is
    \emph{monogamous} if \(f\) and \(g\) are mono and, for all nodes
    \(v \in \vertices{F}\), the degree of \(v\) is
    \begin{center}
        \begin{tabular}{rlcrl}
            \((0,0)\)
            &
            if \(v \in f \wedge v, v \in g_\star\)
            &
            \quad
            &
            \((0,1)\)
            &
            if \(v \in f_\star\)
            \\
            \((1,0)\)
            &
            if \(v \in g_\star\)
            &
            \quad
            &
            \((1,1)\)
            &
            otherwise
        \end{tabular}
    \end{center}
\end{definition}

\begin{lemma}[\cite{bonchi2022stringa}, Lems. 15-17]
    \label{lem:monogamicity-preserved}
    The following statements hold:
    \begin{enumerate}
        \item identities and symmetries are monogamous;
        \item monogamicity is preserved by composition; and
        \item monogamicity is preserved by tesnor.
    \end{enumerate}
\end{lemma}
\begin{proof}
    For (1), the cospans involved are discrete and all vertices are in both
    interfacs.
    For (2), assume we compose two monogamous acyclic cospans \(
        \cospan{\listvar{m}}[f]{F}[g]{\listvar{n}}
    \) and \(
        \cospan{\listvar{n}}[h]{G}[k]{\listvar{p}}
    \).

    The interfaces remain mono as pushouts along monos are monos in
    \(\hypsigmac\). The only vertices that are altered are those in the image of
    \(g\) and \(h\), which are merged pointwise; since vertices in the image of
    \(g\) have out-degree \(0\) and those in the image of \(h\) have in-degree
    \(0\), the merged vertices will have at most degree \((1, 1)\).

    For (3), the degrees of nodes are unaffected as tensor is by coproduct and
    only vertices in the original interfaces will be in the new interfaces.
\end{proof}

\subsection{Acyclicity}

Preventing cycles is a much more natural condition on hypergraphs.

\begin{definition}[Predecessor (\cite{bonchi2022stringa}, Def. 18)]
    A hyperedge \(e\) is a \emph{predecessor} of another hyperedge \(e^\prime\)
    if there exists a node \(v\) in the sources of \(e\) and the targets of
    \(e^\prime\).
\end{definition}

\begin{definition}[Path (\cite{bonchi2022stringa}, Def. 19)]
    A \emph{path} between two hyperedges \(e\) and \(e^\prime\) is a sequence of
    hyperedges \(e_0, \dots, e_{n-1}\) such that \(e = e_0\),
    \(e^\prime = e_{n-1}\), and for each \(i < n-1\), \(e_i\) is a predecessor
    of \(e_{i+1}\).
    A subgraph \(H\) is the \emph{start} or \emph{end} of a path if it contains
    a node in the sources of \(e\) or the targets of \(e^\prime\) respectively.
\end{definition}

Since nodes are single-element subgraphs, one can also talk about paths from
nodes.

\begin{definition}[Acyclicity (\cite{bonchi2022stringa}, Def. 20)]
    A hypergraph \(F\) is acyclic if there is no path from a node to itself.
    A cospan \(\morph{\listvar{m}}{F}{\listvar{n}}\) is acyclic if \(F\) is.
\end{definition}

\begin{lemma}[\cite{bonchi2022stringa}, Lems. 15-17, Prop. 21]
    \label{lem:monogamous-acyclic-preserved}
    The following statements hold:
    \begin{itemize}
        \item identities and symmetries are monogamous acyclic;
        \item monogamous acyclicity is preserved by composition; and
        \item monogamous acyclicity is preserved by tensor.
    \end{itemize}
\end{lemma}
\begin{proof}
    (1) and (3) follow by \cref{lem:monogamicity-preserved}, as cycles clearly
    cannot be created by the coproduct.

    For (2), once again assume we compose two monogamous acyclic cospans \(
        \cospan{\listvar{m}}[f]{F}[g]{\listvar{n}}
    \) and \(
        \cospan{\listvar{n}}[h]{G}[k]{\listvar{p}}
    \).
    A cycle cannot be created by composition because there cannot be a path in
    \(F\) that starts in the image of \(g\) or a path in \(G\) that ends in the
    image of \(h\), because these vertices have out-degree and in-degree \(0\)
    respectively.
\end{proof}

This shows that monogamous acyclic cospans of hypergraphs form a category.

\begin{definition}
    Let \(\macspdchyp\) be the sub-PROP of \(\cspdchyp\) containing only the
    monogamous acyclic cospans of hypergraphs.
\end{definition}

\todo[inline]{Do we talk about coproducts as pushouts earlier?}

\begin{definition}
    \todo[inline]{3-for-2 condition}
\end{definition}

\begin{lemma}
    \todo[inline]{Faithful if 3-for-2 condition}
\end{lemma}

\begin{corollary}
    \tood[inline]{Map from S sigma is faithful}
\end{corollary}

\begin{corollary}[\cite{bonchi2022stringa}, Cor. 26]
    There is an isomorphism of PROPs \(\smcsigmac \cong \macspdchyp\).
\end{corollary}
\section{Traced terms}

We have now seen the classes of cospans of hypergraphs that correspond to
terms in a hypergraph category and terms in a symmetric monoidal category.
Terms in a symmetric traced monoidal category sit somewhere in the middle of
these two: cycles are permitted but wires cannot fork or join arbitrarily.
This means to characterise the cospans of hypergraphs that correspond to traced
terms we need to weaken the conditions of monogamy and acyclicity described in
the previous section.

First we will establish the morphism \(\stmcsigmac \to \cspdchyp\) we will use
to map traced terms to cospans of hypergraphs.
We could do this manually fairly easily, but this would mean we would have to
redo all the proofs in the previous section from scratch.
Instead we will reuse the previous results, by exploiting the correspondence
between traced and compact closed categories.

\begin{lemma}[\cite{rosebrugh2005generic}, Prop. 2.8]
    Every hypergraph category is self-dual compact closed.
\end{lemma}
\begin{proof}
    The cup on a given object is defined as \(
        \iltikzfig{strings/compact-closed/cup-self-dual}[colour=white,obj=A]
        \coloneqq
        \iltikzfig{strings/structure/frobenius/cup}[obj=A]
    \) and the cap as \(
        \iltikzfig{strings/compact-closed/cap-self-dual}[colour=white,obj=A]
        \coloneqq
        \iltikzfig{strings/structure/frobenius/cap}[obj=A]
    \).
    The snake equations follow by applying the Frobenius equation and unitality:
    \begin{gather*}
        \iltikzfig{strings/structure/frobenius/snake-1-0}
        =
        \iltikzfig{strings/structure/frobenius/snake-1-1}
        =
        \iltikzfig{strings/structure/frobenius/snake-1-2}
        \qquad
        \iltikzfig{strings/structure/frobenius/snake-2-0}
        =
        \iltikzfig{strings/structure/frobenius/snake-2-1}
        =
        \iltikzfig{strings/structure/frobenius/snake-2-2}
        \qedhere
    \end{gather*}
\end{proof}

\begin{lemma}
    \label{lem:stmc-subcat-hypc}
    \(\stmcsigmac\) is a subcategory of \(\hypcsigmac\).
\end{lemma}
\begin{proof}
    Since \(\hypsigmac\) is compact closed, it has a (canonical) trace.
    For \(\stmcsigmac\) to be a subcategory of \(\hypcsigmac\), every morphism
    of the former must also be a morphism on the latter.
    Since the two categories are freely generated (with the trace constructed
    through the Frobenius generators in the latter), all that remains is to
    check that every morphism in \(\stmcsigmac\) is a unique morphism in
    \(\hypcsigmac\), i.e.\ the equations of \(\frobc\) do not merge any together.
    This is trivial since the equations do not apply to the construction of the
    canonical trace.
\end{proof}

\begin{definition}
    Let \(\morph{\tracedtosymandfrobsigmac}{\stmcsigmac}{\hypcsigmac}\) be the
    inclusion functor induced by \cref{lem:stmc-subcat-hypc}.
\end{definition}

\begin{corollary}
    \(\tracedtosymandfrobsigmac\) is faithful.
\end{corollary}

To translate a term in \(\stmcsigmac\) into a cospan of hypergraphs, one uses
the inclusion functor \(\tracedtosymandfrobsigmac\) to elevate to the
Frobenius realm, before applying \(\termandfrobtohypsigmac\) from the previous
section to obtain a cospan of hypergraphs.

\begin{corollary}
    \(\termandfrobtohypsigmac \circ \tracedtosymandfrobsigmac\) is faithful.
\end{corollary}

\subsection{Partial monogamy}

Since \(\termandfrobtohypsigmac \circ \tracedtosymandfrobsigmac\) is faithful,
every distinct traced term in \(\stmcsigmac\) has a unique cospan of
hypergraphs up to isomorphism.
However, this functor is not clearly not full: as we have previously
illustrated, there are more terms in \(\hypcsigmac\) than there are in
\(\stmcsigmac\).
The next step is to characterise the image of
\(\termandfrobtohypsigmac \circ \tracedtosymandfrobsigmac\).

Since monogamous acyclic cospans correspond exactly to symmetric monoidal terms,
this property is too restrictive to be used as a setting for modelling traced
terms.
Clearly, we can drop the acyclicity condition, as the trace can introduce
cycles.
However, there is also a foible regarding the monogamicity condition which must
also be tackled.
Although wires are also not permitted to arbitrarily fork or join in a traced
category, it is possible to have a case where wires do not connect to
any generators while also remaining disconnected from the interfaces.
This special case is the trace of the identity, which in string diagrams is
depicted as a closed
loop \(
    \trace{1}{\iltikzfig{strings/category/identity}[colour=white]}
    =
    \iltikzfig{strings/traced/trace-id}
\).

\begin{remark}
    One might think a closed loop can be discarded, in that \(
        \iltikzfig{strings/traced/trace-id}
        =
        \iltikzfig{strings/monoidal/empty}
    \), but this is \emph{not} always the case.
    For example, it is not true in
    \(\finvectk{k}\)~\cite[Sec. 6.1]{hasegawa1997recursion}.
\end{remark}

These closed loops can be modelled as lone vertices (`bones') disconnected from
either interface.

\begin{definition}[Partial monogamy]
    For a cospan \(\cospan{\listvar{m}}[f]{F}[g]{\listvar{n}}\) in
    \(\cspdchyp\), let \(\mathsf{in}(F)\) be the image of \(f\) and let
    \(\mathsf{out}(F)\) be the image of \(g\).
    A cospan \(\cospan{m}[f]{F}[g]{n} \in \cspdhyp\) is
    \emph{partial monogamous} if \(f\) and \(g\) are mono and, for all nodes
    \(v\), the degree of \(v\) is
    \begin{center}
        \begin{tabular}{rlcrl}
            \((0,0)\)
            &
            if \(v \in \mathsf{in}(F) \wedge v \in \mathsf{out}(F)\)
            &
            \quad
            &
            \((0,1)\)
            &
            if \(v \in \mathsf{in}(F)\)
            \\
            \((1,0)\)
            &
            if \(v \in \mathsf{out}(F)\)
            &
            \quad
            &
            \((0,0)\)
            or \((1,1)\)
            &
            otherwise
        \end{tabular}
    \end{center}
\end{definition}

\begin{figure}
    \centering
    \[
        \underbrace{
            \iltikzfig{graphs/monogamy/yes-0}
            \iltikzfig{graphs/monogamy/yes-1}
        }_{\text{partial monogamous}}
        \qquad
        \underbrace{
            \iltikzfig{graphs/monogamy/no-0}
            \iltikzfig{graphs/monogamy/no-1}
        }_{\text{not partial monogamous}}
    \]
    \caption{Examples of cospans that are and are not partial monogamous.}
    \label{fig:partial-monogamous-examples}
\end{figure}

\begin{example}
    Examples of cospans that are and are not partial monogamous are shown
    in \cref{fig:partial-monogamous-examples}.
\end{example}

As with the monogamous acyclic cospans, a sub-PROP of \(\cspdchyp\) containing
the partial monogamous cospans must be constructed.

\begin{lemma}\label{lem:identities-symmetries-partial-monogamous}
    Identities and symmetries are partial monogamous.
\end{lemma}
\begin{proof}
    Identities and symmetries are monogamous by
    \cref{lem:identities-symmetries-monogamous} so they must also be partial
    monogamous.
\end{proof}

\begin{lemma}\label{lem:partial-monogamicity-preserved-composition}
    Partial monogamy is preserved by composition.
\end{lemma}
\begin{proof}
    By \cref{lem:monogamicity-preserved-composition}, composition preserves
    monogamicity.
    The only difference between partial monogamous cospans and monogamous ones
    is that the former may have cycles and nodes of degree \((0,0)\) not in the
    interfaces.
    However, since neither of these can be interfaces they cannot be altered by
    composition, so partial monogamy must also be preserved.
\end{proof}

\begin{lemma}\label{lem:partial-monogamicity-preserved-tensor}
    Partial monogamy is preserved by tensor.
\end{lemma}
\begin{proof}
    As with composition, tensor preserves monogamicity by
    \cref{lem:monogamicity-preserved-tensor}, and as tensor does not affect the
    degree of nodes then it preserves partial monogamy as well.
\end{proof}

As partial monogamicity is preserved by both forms of composition, the
partial monogamous cospans themselves form a PROP.

\begin{definition}
    Let \(\pmcspdchyp\) be the sub-PROP of \(\cspdchyp\) containing only the
    partial monogamous cospans of hypergraphs.
\end{definition}

While for the symmetric monoidal case we could stop here, we have no such luck
now: we must show that \(\pmcspdchyp\) is also traced.
Although \(\cspdhyp\) already has a trace in the form of the canonical trace, we
must make sure that this does not degenerate for cospans of partial monogamous
hypergraphs.

\begin{theorem}\label{thm:partial-monogamous-trace}
    The canonical trace is a trace on \(\pmcspdhyp\).
\end{theorem}
\begin{proof}
    Consider a partial monogamous cospan \(
        \cospan{x + m}[f + h]{F}[g + k]{x + n}
    \); we must show that its trace \(
        \cospan{m}[h]{F^\prime}[k]{n}
    \) is also partial monogamous.
    Only the vertices in the image of \(f\) and \(g\) are affected by the trace:
    for each element \(a \in x\), \(f(a)\) and \(g(a)\) will be merged
    together in the traced graph, summing their degrees.
    If these vertices are in the image of \(h\) or \(k\) then this will be
    preserved in the traced cospan.
    We now consider the various cases:
    \begin{itemize}
        \item if \(f(a) = g(a)\), then this vertex must have degree \((0, 0)\);
                the traced vertex will still have degree \((0, 0)\) and will no
                longer be in the interface;
        \item if \(f(a)\) is also in the image of \(n \to F\) and \(g(i)\) is
                also in the image of \(m \to F\), then both \(f(a)\) and
                \(g(a)\) have degree \((0, 0)\); the traced vertex will still
                have degree \((0, 0)\) and be in both interfaces of the traced
                cospan;
        \item if \(f(a)\) is also in the image of \(n \to F\), then \(f(i)\)
                has \((0, 0)\) and \(g(a)\) will have degree
                \((1,0)\), so the traced vertex will have degree \((1, 0)\) and
                be in the image of \(n \to F^\prime\); and
        \item if \(g(i)\) is in the image of \(m \to F\), then the above
                argument applies in reverse.
    \end{itemize}
    In all these cases, partial monogamy is preserved, so the canonical trace is
    still suitable for working with cospans of partial monogamous hypergraphs.
\end{proof}

Crucially, while we leave \(\pmcspdhyp\) in order to construct the trace using
the cup and cap, the resulting cospan \emph{is} in \(\pmcspdhyp\).

\subsection{The traced correspondence}

Now that we have a traced sub-PROP of cospans of hypergraphs, it is time to show
that this particular sub-PROP is the one that corresponds to traced terms.

\begin{theorem}\label{thm:termtohyp-image}
    A cospan \(\cospan{m}{F}{n}\) is in the image of \(
        \termandfrobtohypsigmac \circ \tracedtosymandfrobsigmac\) if
    and only if it is partial monogamous.
\end{theorem}
\begin{proof}
    Since the generators of \(\stmcsigma\) are mapped to monogamous cospans
    by \(\termandfrobtohypsigmac \circ \tracedtosymandfrobsigmac\) and partial
    monogamy is preserved by composition
    (\cref{lem:partial-monogamicity-preserved-composition}),
    tensor (\cref{lem:partial-monogamicity-preserved-tensor}),
    and trace
    (\cref{thm:partial-monogamous-trace}),
    every cospan in the image of
    \(\termandfrobtohypsigma \circ \tracedtosymandfrobsigma\) is partial
    monogamous.

    Now we show that any partial monogamous cospan \(
        \cospan{\listvar{m}}[f]{F}[g]{\listvar{n}}
    \) must be in the image of \(
        \termandfrobtohypsigmac \circ \tracedtosymandfrobsigmac
    \) by constructing an isomorphic cospan from a trace of cospans, in which
    each component under the trace is in the image of \(\termtohypsigmac\);
    subsequently this means that the entire cospan must be in the image of
    \(\termandfrobtohypsigmac \circ \tracedtosymandfrobsigmac\).
    The components of the new cospan are as follows:
    \begin{itemize}
        \item let \(L\) be the discrete hypergraph containing vertices with
                degree
                \((0,0)\) that are not in the image of \(f\) or \(g\);
        \item let \(E\) be the hypergraph containing hyperedges of \(F\)
                disconnected from each other along with their source and target
                vertices;
        \item let \(V\) be the discrete hypergraph containing all the
                vertices of \(F\); and
        \item let \(S\) and \(T\) be the discrete hypergraphs containing
                the source and target vertices of hyperedges in \(F\)
                respectively, with the ordering determined by some order
                \(e_1,e_2,\cdots,e_n\) on the edges in \(F\).
    \end{itemize}
    These parts can be composed to form the following composite:
    \begin{gather*}
        \cospan{L + T + m}[\id + \id + f]{L + V}[\id + \id + g]{L + S + n}
        \,\seq\,
        \cospan{L + S + n}[\id]{L + E + n}[\id]{L + T + n}
    \end{gather*}
    Finally we take the trace of \(L + T\) over this composite to obtain a
    cospan which can be checked to be isomorphic to the original cospan
    \(\cospan{m}[f]{F}[g]{n}\) by applying the pushouts.
    The components of the composite under the trace are all monogamous acyclic
    so must be in the image of \(\termtohypsigmac\) by
    \cref{thm:monogamous-acyclic-full}; this means there is a term
    \(f \in \smcsigmac\) such that \(\termtohypsigmac[f]\) is isomorphic to the
    original composite.
    Clearly the trace of \(f\) is in \(\stmcsigmac\), and so the trace of the
    composite is in the image of
    \(\termandfrobtohypsigmac \circ \tracedtosymandfrobsigmac\).
\end{proof}

This shows that \(
    \termandfrobtohypsigmac \circ \tracedtosymandfrobsigmac{\Sigma}
\) is a \emph{full} mapping from \(\stmcsigmac\) to \(\pmcspdchyp\).
As it is both full and faithful, we can then conclude the final result.

\begin{corollary}\label{cor:stmc-graph-iso}
    \(\stmc{\Sigma} \cong \pmcspdhyp\).
\end{corollary}

\todo[inline]{Do some examples}
\chapter{Hypergraphs for traced comonoid terms}

\todo[inline]{Adapt from FSCD paper section 3-4}
