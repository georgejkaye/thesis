\chapter{String diagrams as hypergraphs}\label{chap:hypergraphs}

String diagrams are an appealing way of reasoning with pen and paper: they bring
intuition to one-dimensional text strings and can often shed light on the
next step of a proof.
Unfortunately, they do have drawbacks: they take up a lot of time and space, and
if not drawn with care can end up being messy, removing any benefit of using
them in the first place.

Instead it is desirable to perform reasoning with string diagrams
\emph{computationally}.
This presents some questions: how can we encode circuits constructed
categorically in such a way that a computer can understand them?
Perhaps we could input terms using traditional one-dimensional text
representations?
Text is something that computers are very good at processing, but as we have
already established it is verbose, unintuitive, and, most importantly,
means we have to apply the axioms of STMCs explicitly.
Sticking to string diagrams would be ideal, but the representation
needs some thought.
Although computers do not deal well with topological objects like string
diagrams, they are very well acquainted with combinatorial objects; for
a computer to reason effectively with string diagrams, they must first be
interpreted as \emph{graphs}.

\section{String diagram rewriting}

Graph rewriting specialised for rewriting sting diagrams is a
relatively new field.
One of the first approaches was developed at the turn of the 2010s using
\emph{string graphs}~\cite{%
    dixon2010open,dixon2013opengraphs,kissinger2012pictures%
}, an example of which is illustrated in \cref{fig:string-graph}.
String graphs have two classes of nodes for \emph{boxes} and \emph{wires}; the
former nodes represent generators in string diagrams and the latter nodes
represent the wires between them.
One nuance of string graphs is that a wire in a string diagram can be
represented by arbitrarily many wire nodes connected together; all of these
different depictions are identified by a notion of \emph{wire homeomorphism}, in
which adjacent wire nodes can be collapsed into one.

\begin{figure}
    \centering
    \iltikzfig{graphs/examples/string-graph}
    \caption{Example of an interfaced string graph}
    \label{fig:string-graph}
\end{figure}

String graphs modulo wire homeomorphism are a suitable setting for modelling
traced or compact closed categories, but their main drawback is that a given
term may correspond to many different graphs thanks to wire homeomorphism.

More recently, there has been a flurry of work on string
diagram rewriting modulo \emph{Frobenius structure} using
\emph{hypergraphs}~\cite{%
    bonchi2016rewriting,zanasi2017rewriting,bonchi2017confluence,%
    bonchi2018rewriting,bonchi2022string,bonchi2022stringa,bonchi2022stringb%
}, such as that in \cref{fig:hypergraph-intro}.
Hypergraphs are a generalisation of graphs in which edges can have arbitrarily
many sources and targets, rather than just one each.
When interpreting string diagrams as hypergraphs, generators are represented as
hyperedges and connections between generators indicated by shared source or
target vertices.
As there is no restriction for nodes to only be incident on a single source and
target, one can model structures such as monoids or comonoids.

\begin{figure}
    \centering
    \iltikzfig{graphs/examples/hypergraph}
    \caption{Example of an interfaced hypergraph}
    \label{fig:hypergraph-intro}
\end{figure}

While string diagrams `absorb' the equations of SMCs, hypergraphs go one further
and absorb the equations of
a \emph{special commutative Frobenius algebra}: string diagrams
equal by Frobenius equations are interpreted as isomorphic hypergraphs.
This means rewriting using hypergraphs can be even more
advantageous than using string diagrams.

Naturally, there have also been variations on this work where the complete
Frobenius structure is not present.
Suitable restrictions on hypergraphs and the graph rewriting process are also
identified in~\cite{bonchi2016rewriting} for rewriting
\emph{symmetric monoidal structure}.
Research followed on rewriting modulo
\emph{(co)monoid structure}~\cite{milosavljevic2023string} (`half a Frobenius')
and our work~\cite{ghica2023rewriting} on rewriting modulo
\emph{traced comonoid structure}.
The latter is the basis for this part of the thesis.
\section{Hypergraphs}

We will begin by defining the categories of hypergraphs required, following the
pattern detailed in \cite{bonchi2022string}.
Hypergraphs are formally defined as a functor category.

\begin{definition}[Hypergraph~\cite{bonchi2016rewriting}]
    Let \(\mathbf{X}\) be the category with object set
    \((\nat \times \nat) + \star\) and morphisms
    \(\morph{\sources{i}}{(k,l)}{\star}\) for each \(i < k\)
    and \(\morph{\targets{j}}{(k,l)}{\star}\) for each \(j < l\).
    The category of hypergraphs \(\hyp\) is the functor category
    \([\mathbf{X}, \set]\).
\end{definition}

One can think of the category \(\mathbf{X}\) as a `template' for the structure
of a hypergraph: the object \(\star\) represents the nodes and each object
\((k, l)\) represents hyperedges with \(k\) sources and \(l\) targets; each such
edge must pick \(k\) sources and \(l\) targets from \(\star\).

Objects in \(\hyp\) are functors that instantiates each object in \(\mathbf{X}\)
to a concrete set.
Subsequently, for a hypergraph \(F \in \hyp\) we write \(\vertices{F}\) for its
set of nodes and \(\edges{F}{k}{l}\) for the set of edges with \(k\) sources and
\(l\) targets.
Since it is a functor category, the morphisms in \(\hyp\) are natural
transformations: structure-preserving maps between hypergraphs.

\begin{definition}[Hypergraph homomorphism]
    Given two hypergraphs \(F, G \in \hyp\), \emph{hypergraph homomorphism}
    \(F \to G\) consists of functions
    \(\morph{\vertices{f}}{\vertices{F}}{\vertices{G}}\) and
    \(\morph{\edges{f}{k}{l}}{\edges{F}{k}{l}}{\edges{G}{k}{l}}\) such that the
    following diagrams commute:
    \begin{center}
    \begin{tikzcd}
        \edges{F}{k}{l}
        \arrow{r}{\edges{f}{k}{l}}
        \arrow{d}{\sources{i}}
        &
        \edges{G}{k}{l}
        \arrow{d}{\sources{i}}
        \\
        \vertices{F}
        \arrow{r}{\vertices{f}}
        &
        \vertices{G}
    \end{tikzcd}
    \qquad
    \begin{tikzcd}
        \edges{F}{k}{l}
        \arrow{r}{\edges{f}{k}{l}}
        \arrow{d}{\targets{i}}
        &
        \edges{G}{k}{l}
        \arrow{d}{\targets{i}}
        \\
        \vertices{F}
        \arrow{r}{\vertices{f}}
        &
        \vertices{G}
    \end{tikzcd}
\end{center}
\end{definition}

Much like with regular graphs, it is much more intuitive to draw out hypergraphs
rather than look at their combinatorial representation.
We draw nodes as black dots, and hyperedges as `bubbles' with ordered tentacles
on the left and right that connect to source and target nodes respectively.

\begin{example}
    \todo[inline]{Do hypergraph example}
\end{example}

\subsection{Labelled hypergraphs}

From the example drawn above, it should be clear to see how hypergraphs are a
suitable representation of string diagrams: generators correspond to hyperedges
and wires to the nodes between them.
However, the hyperedges are currently not \emph{labelled} with generator
symbols.
To do this, we must first translate the notion of signature to hypergraphs.

\begin{definition}[Hypergraph signature~\cite{bonchi2016rewriting}]
    For a set of generators \(\signature\) and sorts \(\mcs\) as defined
    in \cref{def:generators}, the \emph{hypergraph signature}
    \(\hypsignature{(\Sigma, S)} \in \hyp\) is defined as follows:
    \begin{gather*}
        \vertices{\hypsignature{\Sigma}} := \{ v_n \,|\, s \in \mcs\}
        \quad
        \edges{\hypsignature{\Sigma}}{k}{l} := \{ e_g \,|\, g \in \signature\}
        \\
        \sources{i}(e_g) := v_{\dom[e_g](i)}
        \quad
        \targets{j}(e_g) := v_{\cod[e_g](j)}
    \end{gather*}
\end{definition}

\begin{example}
    \todo[inline]{Do hypergraph signature example}
\end{example}

The vertices and edges of a hypergraph \(F\) can then be assigned sorts from
\(\mcs\) and symbols from \(\Sigma\) using a homomorphism
\(F \to \hypsignature{\Sigma,\mcs}\).
To do this to \emph{all} the hypergraphs in \(\hyp\) and create a category of
\emph{labelled} hypergraphs, we make use of some more categorical machinery.

\begin{definition}[Slice category~\cite{lawvere1963functorial}]
    For a category \(\mathbf{C}\) and an object \(C \in \mathbf{C}\), the
    \emph{slice category} \(\mathbf{C} \slice C\) has objects the morphisms of
    \(\mathbf{C}\) with target \(C\) and morphisms
    \((\morph{f}{X}{C}) \to (\morph{g}{X^\prime}{C})\) the morphisms
    \(\morph{g}{X}{X^\prime} \in \mathbf{C}\) such that \(f^\prime\circ g = f\).
\end{definition}

\begin{definition}[Labelled hypergraphs~\cite{bonchi2016rewriting}]
    Let \(\hypsigmac\) be the category of hypergraphs labelled over a set of
    generators \(\Sigma\) and sorts \(\mcs\), defined as the slice category
    \(\hyp \slice \hypsignature{(\Sigma, \mcs)}\).
\end{definition}

\begin{example}
    \todo[inline]{Do labelled hypergraph example}
\end{example}
\subsection{Cospans of hypergraphs}

String diagrams also have \emph{input} and \emph{output} interfaces.
(Labelled) hypergraphs may have suggestively dangling nodes in the pictures,
but this is not actually encoded in the definition, and moreover we may wish to
set a non-dangling node as an input or output.
To set the interfaces of a hypergraph, hypergraph homomorphisms are used
to `pick' the appropriate nodes.

\begin{definition}[Cospan]
    A \emph{cospan} in a category \(\mcc\) is a pair of morphisms \(X \to A\)
    and \(X \to B\) in \(\mcc\), written \(\cospan{X}{A}{Y}\).
    A \emph{cospan morphism} \(
        (\cospan{X}[f]{A}[g]{Y}) \to (\cospan{X}[h]{B}[k]{Y})
    \) is a morphism \(\morph{\alpha}{A}{B}\) in \(\mcc\)
    such that the following diagram commutes:
    %
    \begin{center}
        \includestandalone{figures/category/diagrams/cospan-morphism}
    \end{center}
%
    Two cospans \(\cospan{X}{A}{Y}\) and \(\cospan{X}{B}{Y}\) are
    \emph{isomorphic} if there exists a morphism of cospans as above where
    \(\alpha\) is an isomorphism.
\end{definition}

As with all the constructions so far, cospans must be assembled into a category
to be useful for our purpose.
This means a notion of \emph{composition} of cospans is required.

\begin{definition}[Composition of cospans]
    \label{def:cospan-composition}
    In a category \(\mcc\) with pushouts, the composition of cospans
    \(\cospan{X}[f]{A}[g]{Y}\) and \(\cospan{Y}[h]{B}[k]{Z}\) is by pushout:
    \begin{center}
        \includestandalone{figures/category/diagrams/cospan-composition}
    \end{center}
\end{definition}

\begin{definition}[Categories of cospans]
    Let \(\mcc\) be a category with pushouts and an initial object.
    The category of cospans over \(\mathbf{C}\), denoted \(\csp{\mathbf{C}}\),
    has as objects the objects of \(\mathbf{C}\) and as morphisms \(A \to B\)
    the isomorphism classes of cospans \(\cospan{A}{X}{B}\) for some
    \(X \in \mcc\).
    Composition is by pushout as detailed in \cref{def:cospan-composition} and
    the identity is \(X \xrightarrow{\id[X]} X \xleftarrow{\id[X]} X\).

    This category is symmetric monoidal with tensor given by the coproduct in
    \(\mathbf{C}\), unit the initial object \(0 \in \mathbf{C}\), and symmetry
    by \(\cospan{A+B}{A+B}{B+A}\).
\end{definition}

Interfaces are assigned to a hypergraph \(F\) by having it occupy the `apex' of
a cospan and having the `legs' on either side pick inputs and outputs
respectively.

\begin{definition}[Discrete hypergraph]
    A \emph{discrete hypergraph} is a hypergraph in which each edge set is
    empty.
\end{definition}

\subsection{Ordering the interfaces}

Even though our hypergraphs still have input and output interfaces specified
using cospans, there is still information missing: what is the \emph{ordering}
of the vertices in these interfaces?
Without this data \(\csp{\hypsigmac}\) is not even a PROP.

A cospan \(\cospan{M}{F}{N}\) is a morphism
\(M \to N \in \csp{\hypsigmac}\).
Therefore, to restrict this category to a coloured PROP, we need to ensure that
the legs of each cospan of hypergraphs can be viewed as a word in
\(\freemon{C}\) for some countable set of colours \(C\).
This is formally performed by another functor.

\begin{theorem}[\cite{bonchi2022string}, Thm. 3.6]
    Let \(\mathbb{X}\) be a coloured PROP whose monoidal product is a coproduct,
    \(\mathbf{C}\) a category with pushouts and an initial object, and \(
        \morph{F}{\mathbb{X}}{\mathbf{C}}
    \) a coproduct-preserving functor.
    Then there exists a coloured PROP \(\csp[F]{\mathbf{C}}\) whose arrows
    \(\listvar{m} \to \listvar{n}\) are isomorphism classes of \(\mathbf{C}\)
    cospans \(\cospan{F\listvar{m}}{C}{F\listvar{n}}\).
\end{theorem}

\(F\) is the functor that imbues the objects in the legs of the cospan with the
structure of words in some coloured PROP \(\mathbb{X}\).
In our case, this PROP will be the PROP of coloured finite sets.

\begin{definition}
    Let \(\finsetprop\) be the PROP with morphisms \(m \to n\) the functions
    between finite sets \([m] \to [n]\).
\end{definition}

\(\finsetprop\) is a (monochromatic) PROP; as with hypergraphs colours can be
assigned using a slice category.
Since we are working with potentially countably infinite sets of colours, the
definition of \(\finsetprop\) must first be tweaked.

\begin{definition}
    Let \(\finsetpropwithnat\) be the category \(\finsetprop\) augmented with the
    set of natural numbers and the functions \([m] \to \nat\) for each finite
    set \([m]\).
\end{definition}

Then the PROP of finite sets \emph{coloured} over some countable set \(C\) is
the slice \(\finsetpropwithnat \slice C\).
Objects of this category are pairs \(([m], \morph{w}{[m]}{C})\); this pair can
be viewed as a word in \(\freemon{C}\) of length \(m\), with the \(i\)th letter
as \(w(i)\).

\begin{remark}
    Note that we do not include the morphisms \(\nat \to [m]\) in
    \(\finsetpropwithnat\); this is because when we view objects of
    \(\finsetpropwithnat \slice C\) as words in \(\freemon{C}\), we still only
    want to consider finite words despite there being potentially countably
    infinite colours.
\end{remark}

All that remains is to verify that \(\finsetpropwithnat \slice C\) is indeed a
coloured PROP.

\begin{lemma}
    \label{lem:slice-coproducts}
    For a category \(\mcc\) with coproducts, \(\mcc \slice X\) has coproducts.
\end{lemma}
\begin{proof}
    Let \(A,B,X\) be objects in \(\mcc\); as \(\mcc\) has coproducts \(A + B\)
    is also an object in \(\mcc\).
    Then the coproduct of \((A, A \to X)\) and \((B, B \to X)\) in
    \(\mcc \slice X\) is \(A + B \to X\); the universal morphism is \([f, g]\).
\end{proof}

\begin{proposition}
    \label{prop:hatfinsetprop-slice-is-coloured-prop}
    For a countable set \(C\), \(\finsetpropwithnat \slice C\) is a
    coloured PROP.
\end{proposition}
\begin{proof}
    This follows the same strategy as \cite[Prop. 2.23]{bonchi2022string}.
    As established, the objects of \(\finsetpropwithnat \slice C\) can be viewed
    as words in \(\freemon{C}\).
    As slice categories preserve coproducts by \cref{lem:slice-coproducts},
    \(\finsetpropwithnat \slice C\) is strict symmetric monoidal, and the
    coproduct acts as concatenation of words.
\end{proof}

We can now state the functor used to assemble interfaces of hypergraphs into
words.

\begin{definition}[\cite{bonchi2022string}, Rem. 3.12]
    Let \(\morph{\pickinterfacesc{C}}{\finsetpropwithnat \slice C}{\hypsigmac}\)
    be defined as functor sending a word \(\overline{n}\) to the corresponding
    discrete coloured hypergraph containing vertices coloured as in
    \(\overline{n}\), and sending a function \(\overline{m} \to \overline{n}\)
    to the induced homomorphism of discrete hypergraphs.
\end{definition}

Subsequently we obtain a coloured PROP \(
    \csp[\pickinterfacesc{C}]{\hypsigmac}
\), which will serve as the domain in which we interpret string diagrams
combinatorially.
\section{Frobenius terms as hypergraphs}

In order to perform graph rewriting on string diagrams, we will interpret the
latter as cospans of hypergraphs.
We will first recount the constructions used by Bonchi et al in
\cite{bonchi2022string} for a broader class of terms before showing their
recipe can be adapted for a \emph{traced} setting either with or without a
comonoid structure.

\subsection{Frobenius structure}

When reasoning with monoidal theories and string diagrams, two structures that
often appear are a \emph{commutative monoid} (for joining and introducing wires) and a
\emph{cocommutative comonoid} (for forking and eliminating wires).
\[
    \iltikzfig{strings/structure/monoid/merge}[colour=white]
    \quad
    \iltikzfig{strings/structure/monoid/init}[colour=white]
    \qquad
    \iltikzfig{strings/structure/comonoid/copy}[colour=white]
    \quad
    \iltikzfig{strings/structure/comonoid/discard}[colour=white]
\]
The monoid and comonoid are subject to the usual equations of (co)unitality,
(co)associativity and (co)commutativity.
\begin{gather*}
    \iltikzfig{strings/structure/monoid/unitality-l-lhs}
    =
    \iltikzfig{strings/structure/monoid/unitality-l-rhs}
    \quad
    \iltikzfig{strings/structure/monoid/associativity-lhs}
    =
    \iltikzfig{strings/structure/monoid/associativity-rhs}
    \quad
    \iltikzfig{strings/structure/monoid/commutativity-lhs}
    =
    \iltikzfig{strings/structure/monoid/commutativity-rhs}
    \quad
    \iltikzfig{strings/structure/comonoid/unitality-l-lhs}
    =
    \iltikzfig{strings/structure/comonoid/unitality-l-rhs}
    \quad
    \iltikzfig{strings/structure/comonoid/associativity-lhs}
    =
    \iltikzfig{strings/structure/comonoid/associativity-rhs}
    \quad
    \iltikzfig{strings/structure/comonoid/commutativity-lhs}
    =
    \iltikzfig{strings/structure/comonoid/commutativity-rhs}
\end{gather*}
When monoids and comonoids appear together, there are multiple ways they can
interact.
One way is by using the equations of a \emph{Frobenius algebra}; this is
particularly relevant to us because symmetric monoidal terms equipped
with a Frobenius structure correspond precisely to the cospans of hypergraphs
defined in the previous section.

\begin{definition}
    \label{def:frob}
    The monoidal theory of \emph{special commutative Frobenius algebras} is
    defined as \((\generators[\frob], \equations[\frob])\), where \(
    \generators[\frob] \coloneqq \{
    \iltikzfig{strings/structure/monoid/merge}[colour=white],
    \iltikzfig{strings/structure/monoid/init}[colour=white],
    \iltikzfig{strings/structure/comonoid/copy}[colour=white],
    \iltikzfig{strings/structure/comonoid/discard}[colour=white]
    \}
    \) and the equations of \(\equations[\frob]\) are listed in
    \cref{fig:frobenius-equations}.
    We write \(\frob \coloneqq \smc{\generators[\frob], \equations[\frob]}\).
\end{definition}

\begin{figure}[t]
    \centering

    \begin{alignat*}{3}
        \iltikzfig{strings/structure/monoid/unitality-l-lhs}
        &=
        \iltikzfig{strings/structure/monoid/unitality-l-rhs}
        \qquad
        \iltikzfig{strings/structure/monoid/associativity-lhs}
        &=
        \iltikzfig{strings/structure/monoid/associativity-rhs}
        \qquad
        \iltikzfig{strings/structure/monoid/commutativity-lhs}
        &=
        \iltikzfig{strings/structure/monoid/commutativity-rhs}
        \qquad
        \iltikzfig{strings/structure/frobenius/frobenius-l}
        =
        \iltikzfig{strings/structure/bialgebra/merge-copy-lhs}
        \\
        \iltikzfig{strings/structure/comonoid/unitality-l-lhs}
        &=
        \iltikzfig{strings/structure/comonoid/unitality-l-rhs}
        \qquad
        \iltikzfig{strings/structure/comonoid/associativity-lhs}
        &=
        \iltikzfig{strings/structure/comonoid/associativity-rhs}
        \qquad
        \iltikzfig{strings/structure/comonoid/commutativity-lhs}
        &=
        \iltikzfig{strings/structure/comonoid/commutativity-rhs}
        \qquad
        \iltikzfig{strings/structure/frobenius/copy-merge-lhs}
        =
        \iltikzfig{strings/structure/frobenius/copy-merge-rhs}
    \end{alignat*}
    \caption{
        Equations \(\equations[\frob]\) of a
        \emph{special commutative Frobenius algebra}.
    }
    \label{fig:frobenius-equations}
\end{figure}

The equations of special Frobenius algebras are those of commutative monoids and
cocommutative comonoids along with the `Frobenius' and `special' equations.

\begin{example}\label{ex:frobenius}
    The following are all terms in \(\frob\):
    \[
        \iltikzfig{strings/structure/frobenius/example-1}
        \quad
        \iltikzfig{strings/structure/frobenius/example-2}
        \quad
        \iltikzfig{strings/structure/frobenius/example-3}
    \]
    Using the equations of \(\equations[\frob]\), it can be shown that the
    latter two terms are equal:
    \begin{gather*}
        \iltikzfig{strings/structure/frobenius/example-2}
        \eqaxioms[(\monoidunitleqn)]
        \iltikzfig{strings/structure/frobenius/example-equational/step-1}
        \eqaxioms[(\frobleqn)]
        \iltikzfig{strings/structure/frobenius/example-equational/step-2}
        \eqaxioms[(\monoidassoceqn)]
        \iltikzfig{strings/structure/frobenius/example-equational/step-3}
        \eqaxioms[(\frobreqn)]
        \iltikzfig{strings/structure/frobenius/example-3}
    \end{gather*}
\end{example}

Effectively, any terms in \(\frob\) with the same input-output connectivity
are equal.

\subsection{Coloured Frobenius}

\(\frob\) is a monochromatic PROP.
To define a \emph{coloured} version of \(\frob\) we simply use a different copy
of \(\frob\) to represent each colour, using a fact about \(\propcat\) and
\(\cprop\).

\begin{theorem}[\cite{baez2018props}, Corollary 5.3]
    \(\propcat\) has coproducts.
\end{theorem}

This generalises to \(\cprop\) by replacing natural numbers with words.
This means that given coloured PROPs \(\mcc\) and \(\mcd\) with objects the
words in \(\freemon{C}\) and \(\freemon{D}\) respectively, there is also a
coloured PROP \(\mcc + \mcd\) with objects the words in \(\freemon{(C + D)}\)
and morphisms defined in the obvious way.
We can use this to define a multi-coloured version of \(\frob\) as
a coproduct of copies of \(\frob\).

\begin{definition}[\cite{bonchi2022string}]
    \label{def:frobc}
    For a countable set \(C\), let \(\frobc \in \cprop\) be
    defined as \(\frobc \coloneqq \sum_{c \in C}\frob\).
\end{definition}

\begin{example}
    In \(\frobc\), there is a copy of the Frobenius structure for each colour
    in \(C\).
    For example, when \(\mcc \coloneqq \{\bullet,\redbullet\}\), the following
    are terms in \(\frobc\).
    \[
        \iltikzfig{strings/structure/frobenius/example-1-coloured}
        \quad
        \iltikzfig{strings/structure/frobenius/example-2-coloured}
    \]
    Although there are two different colours of wires, these wires cannot
    interact without the addition of other generators to map between them.
\end{example}


\subsection{Hypergraph categories}

Frobenius structures have turned out to be very useful in studying compositional
processes such as quantum processes~\cite{coecke2008interacting} and signal flow
graphs~\cite{bonchi2014categorical,bonchi2015full}.
It is useful to talk about the setting in which \emph{every} object has such a
structure.

\begin{definition}[Hypergraph category~\cite{fong2019hypergraph}]
    \label{def:hypergraph-category}
    A \emph{hypergraph category} is a category in which every object is equipped
    with a special commutative Frobenius algebra subject to the
    \emph{coherence equations} in \cref{fig:hypergraph-coherence}.
\end{definition}
%
\begin{figure}
    \centering
    \begin{gather*}
        \iltikzfig{strings/structure/monoid/coherence-monoid-lhs}[obj1=A,obj2=B]
        =
        \iltikzfig{strings/structure/monoid/coherence-monoid-rhs}[obj1=A,obj2=B]
        \qquad
        \iltikzfig{strings/structure/monoid/coherence-unit-lhs}[obj1=A,obj2=B]
        =
        \iltikzfig{strings/structure/monoid/coherence-unit-rhs}[obj1=A,obj2=B]
        \qquad
        \iltikzfig{strings/structure/comonoid/coherence-comonoid-lhs}[obj1=A,obj2=B]
        =
        \iltikzfig{strings/structure/comonoid/coherence-comonoid-rhs}[obj1=A,obj2=B]
        \qquad
        \iltikzfig{strings/structure/comonoid/coherence-counit-lhs}[obj1=A,obj2=B]
        =
        \iltikzfig{strings/structure/comonoid/coherence-counit-rhs}[obj1=A,obj2=B]
    \end{gather*}
    \caption{Coherence equations of a hypergraph category}
    \label{fig:hypergraph-coherence}
\end{figure}%
%
\begin{remark}
    The notion of a hypergraph category has been
    rediscovered numerous times over the years.
    They were originally called \emph{well-supported compact closed categories}
    by Carboni and Walters~\cite{carboni1987cartesian}, and have subsequently
    appeared as
    \emph{dgs-monoidal categories}~\cite{katis1997bicategories,gadducci1998inductive,gadducci1999bicategorical,bruni2002normal}
    and \emph{dungeon categories}~\cite{morton2014belief}.
    The term \emph{hypergraph categories} was coined more recently but has
    become the standard in the compositional processes
    community~\cite{kissinger2015finite,fong2015decorated,baez2016compositional,baez2018compositional}.
\end{remark}

As the Frobenius structure is their entire \emph{raison d'\^{e}tre}, it is
unsurprising that the categories of Frobenius terms we encountered earlier are
hypergraph categories.

\begin{lemma}\label{lem:frob-hypergraph}
    \(\frob\) is a hypergraph category.
\end{lemma}
\begin{proof}
    The generators in \(\frob\) provide the Frobenius structure for the object
    \(1\); for the other objects the structure is derived by following the
    recipes given by the coherence equations in \cref{fig:hypergraph-coherence}.
\end{proof}

It is now possible to formally define what we mean when we say `Frobenius
terms'.

\begin{definition}
    For a set of generators \(\generators\), let \(\hypcsigma\) be the PROP
    freely generated over \(\generators + \generators[\frob]\).
\end{definition}

While we can view \(\hypcsigma\) as just `the category containing all the
Frobenius terms', it can be advantageous to view it as a coproduct.

\begin{lemma}
    \(\hypcsigma \cong \smcsigma + \frob\).
\end{lemma}
\begin{proof}
    Every term in \(\hypcsigma\) can be expressed as a combination of
    generators either in \(\smcsigma\) or \(\frob\).
\end{proof}

We can also proceed similarly for the multi-coloured case.

\begin{lemma}\label{lem:frobc-hypergraph}
    \(\frobc\) is a hypergraph category.
\end{lemma}
\begin{proof}
    As \cref{lem:frob-hypergraph}, but there is now a `base' Frobenius
    structure for each colour \(c \in C\).
\end{proof}

\begin{definition}
    For a set of \(C\)-coloured generators over \(\generators\), let
    \(\hypcsigmac\) be the \(C\)-coloured PROP freely generated over
    \(\generators + \generators[\frobc]\).
\end{definition}

\begin{lemma}
    \(\hypcsigmac \cong \smcsigmac + \frobc\).
\end{lemma}

Viewing \(\hypcsigma\) and \(\hypcsigmac\) as coproducts will prove to be
beneficial when establishing a correspondence between terms and graphs in the
the next section, as it allows us to consider the symmetric monoidal and the
Frobenius components separately.

\subsection{Hypergraph categories and hypergraphs}

Perhaps confusingly, the category of \emph{hypergraphs} \(\hypsigma\) is
\emph{not} a hypergraph category, but the category of \emph{cospans} of
hypergraphs is.
This can be shown by exploiting a correspondence between \(\frob\) and the PROPs
of finite sets we encountered earlier.

\begin{proposition}[\cite{lack2004composing}, Ex. 5.4]\label{prop:frob-finset}
    \(\frob \cong \csp{\finsetprop}\).
\end{proposition}

We omit the formal proof and sketch the correspondence.
Terms in \(\frob\) are formed of all the ways of combining \(
\iltikzfig{strings/structure/monoid/merge}[colour=white],
\iltikzfig{strings/structure/monoid/init}[colour=white],
\iltikzfig{strings/structure/comonoid/copy}[colour=white],
\iltikzfig{strings/structure/comonoid/discard}[colour=white],
\iltikzfig{strings/category/identity}[colour=white],
\) and \(
\iltikzfig{strings/symmetric/symmetry}[colour=white]
\) in sequence and parallel, so a string diagram for a term \(\morph{f}{m}{n}\)
is depicted as \(x\) connected components drawing paths from \(m\) inputs to
\(n\) outputs, such as in the example below.

\begin{center}
    \iltikzfig{strings/structure/frobenius/example}
\end{center}

Note there is no requirement for each component to connect to one or both
interfaces as the \(
\iltikzfig{strings/structure/monoid/init}[colour=white]
\) and \(
\iltikzfig{strings/structure/comonoid/discard}[colour=white]
\) generators can introduce and stub wires.
A term \(\morph{f}{m}{n}\) with \(x\) connected components corresponds to
a cospan of finite sets \(\cospan{[m]}[i]{[x]}[j]{[n]}\), where the functions
\(i\) and \(j\) map the inputs and outputs to the components they connect to.

\begin{example}
    Consider the term \(\morph{f}{5}{4}\) drawn on the left below.
    This corresponds to a cospan \(\cospan{[5]}{[3]}{[4]}\) as shown on the
    right below.
    \begin{center}
        \iltikzfig{strings/structure/frobenius/example}
        \(\Leftrightarrow\)
        \scalebox{0.75}{\tikzfig{strings/structure/frobenius/example-cospan}}
    \end{center}
\end{example}

The cospan representation shows how all connected Frobenius components can be
`squished' into a single blob.

We have now ascertained the relationship between \(\frob\) and
\(\csp{\finsetprop}\).
The missing link is the relationship between the latter and \(\cspdhyp\);
this arises as a special case of the following theorem.

\begin{theorem}[\cite{bonchi2022string}, Thm. 3.8]
    \label{thm:cospan-homomorphism}
    Let \(\mathbb{X}\) be a PROP in which the monoidal product is a coproduct,
    let \(\mcc\) be a category such that \(\mathbb{X}\) and
    \(\mcc\) have finite limits, and let
    \(\morph{F}{\mathbb{X}}{\mcc}\) be a colimit-preserving functor.
    Then there is a homomorphism of PROPs \(
    \morph{\tilde{F}}{\csp{\mathbb{X}}}{\csp[F]{\mcc}}
    \) that sends \(\cospan{m}[f]{X}[g]{n}\) to
    \(\cospan{Fm}[Ff]{FX}[Fg]{Fn}\).
    If \(F\) is full and faithful, then \(\tilde{F}\) is faithful.
\end{theorem}
\begin{proof}
    Since \(F\) preserves finite colimits, it preserves composition (pushout)
    and monoidal product (coproduct); symmetries are clearly preserved.
    To show that \(\tilde{F}\) is faithful when \(F\) is full and faithful,
    suppose that \(
    \tilde{F}(\cospan{m}[f]{X}[g]{n})
    =
    \tilde{F}(\cospan{m}[f^\prime]{X}[g^\prime]{n})
    \).
    This gives us the following commutative diagram in \(\mcc\):

    \begin{center}
        \begin{tikzcd}
            & FX \arrow{dd}{\phi} & \\
            Fm \arrow{ur}{Ff} \arrow{dr}{Ff^\prime} & &
            Fn \arrow{ul}{Fg} \arrow{dl}{Fg^\prime} \\
            & FY &
        \end{tikzcd}
    \end{center}
    %
    where \(\phi\) is an isomorphism because morphisms in
    \(\csp[F]{\mathbf{C}}\) are isomorphism classes of cospans.
    As \(F\) is full, there exists \(\morph{\psi}{X}{Y}\) such that
    \(F\psi = \phi\).
    As \(F\) is faithful, \(\psi\) is an isomorphism; this means
    \(\cospan{m}[f]{X}[g]{n}\) and \(\cospan{m}[f^\prime]{X}[g^\prime]{n}\) are
    equal in \(\csp{\mathbb{X}}\), so \(\tilde{F}\) is faithful.
\end{proof}

\begin{corollary}[\cite{bonchi2022string}, Cor. 3.9]
    \label{cor:finset-to-hyp}
    There is a faithful PROP homomorphism
    \(\morph{\tilde{\pickinterfaces}}{\csp{\finset}}{\cspdhyp}\)
\end{corollary}

With this we can derive a map from Frobenius terms to cospans of hypergraphs.

\begin{definition}
    Let \(\morph{\frobtohypsigma}{\frob}{\cspdhyp}\) be the PROP morphism
    defined by using \cref{prop:frob-finset} followed by
    \cref{cor:finset-to-hyp}.
\end{definition}

\begin{example}
    The action of \(\frobtohypsigma\) on the Frobenius generators is as follows:
    \begin{gather*}
        \frobtohyp[
            \iltikzfig{strings/structure/monoid/merge}[colour=white]
        ]{\generators}
        =
        \iltikzfig{graphs/frobenius/monoid}
        \quad
        \frobtohyp[
            \iltikzfig{strings/structure/monoid/init}[colour=white]
        ]{\generators}
        =
        \iltikzfig{graphs/frobenius/unit}
        \\
        \frobtohyp[
            \iltikzfig{strings/structure/comonoid/copy}[colour=white]
        ]{\generators}
        =
        \iltikzfig{graphs/frobenius/comonoid}
        \quad
        \frobtohyp[
            \iltikzfig{strings/structure/comonoid/discard}[colour=white]
        ]{\generators}
        =
        \iltikzfig{graphs/frobenius/counit}
    \end{gather*}
\end{example}

As there is a faithful embedding of \(\frob\) into \(\cspdhyp\) and both
categories share the same objects, we also get the result alluded to at the
start of this section.

\begin{corollary}
    \label{cor:csphypsigma-hypergraph}
    \(\cspdhyp\) is a hypergraph category.
\end{corollary}

\subsection{From coloured terms to coloured graphs}

This result shows how the correspondence works for the monochromatic case; what
about for coloured terms?
Here, we replace \(\finsetprop\) with the coloured version
\(\finsetpropwithnat \slice C\) seen in the previous section.
A coloured version of \cref{prop:frob-finset} was shown for a \emph{finite} set
of colours in \cite{bonchi2022string}; we recall its proof before extending this
to the \emph{countable} setting we work in.

\begin{lemma}
    \label{lem:slice-iso-terminal}
    In a category \(\mcc\) with a terminal object \(1\),
    \(\mcc \cong \mcc \slice 1\).
\end{lemma}
\begin{proof}
    Since \(1\) is terminal, there is a unique morphism \(A \to 1\) for each
    object \(A\) in \(\mcc\), so there is an object \((A, \morph{!_A}{A}{1})\)
    in \(\mcc \slice 1\) for each object \(A \in \mcc\).
    In \(\mcc \slice 1\) there is a morphism
    \((A, \morph{!_A}{A}{1}) \to (B, \morph{!_B}{B}{1})\) in
    for every morphism \(\morph{f}{A}{B} \in \mcc\) such
    that \(f \seq !_B = !_A\); since both \(f \seq !_B\) and \(!_A\) are
    morphisms \(A \to 1\) they must be the same unique morphism.
    Therefore \(\mcc \cong \mcc \slice 1\).
\end{proof}

\begin{theorem}[\cite{bonchi2022string}, Theorem 2.24]
    \label{thm:frobc-iso-finset-slice-c}
    For a finite set of colours \(C \in \finsetprop\), there is an isomorphism
    of coloured PROPs \(\frobc \cong \csp{\finsetprop \slice C}\).
\end{theorem}
\begin{proof}
    By definition of \(\frobc\),
    \cref{def:frobc,prop:frob-finset,lem:slice-iso-terminal}
    we have that \[
        \frobc
        \coloneqq
        \sum_{c \in C}\frob
        \cong
        \sum_{c \in C}\csp{\finsetprop}
        \cong
        \sum_{c \in C}\csp{\finsetprop \slice 1}
    \]
    In the other direction we have that \(
    \csp{\finsetprop \slice C}
    \cong
    \csp{\finsetprop \slice \sum_{c \in C} 1}
    \) as \(C\) is countable.
    So we need to show that \(
    \sum_{c \in C}\csp{\finsetprop \slice 1}
    \cong
    \csp{\finsetprop \slice \sum_{c \in C} 1}
    \).
    The objects of the former are coproducts of objects in
    \(\finsetprop \slice C\); as this is a coloured prop the coproduct is
    concatenation and subsequently the objects can be viewed as words in
    \(\freemon{C}\).
    Similarly, the objects of the latter are objects of
    \(\finsetprop \slice \sum_{c \in C} 1\), which can clearly also be
    seen as words in \(\freemon{C}\).

    The morphisms of the former are coproducts of cospans, which can
    equivalently be viewed as a single cospan with coproducts in the legs and
    apex; using the reasoning above this means it is a cospan of words in
    \(\freemon{C}\); it is easy to see that this is also the case for morphisms
    in the latter.
\end{proof}

We need to show a version of this for the case where \(C\) may be
\emph{countably infinite}.
The strategy is much the same as, but relies on one small observation.

\begin{lemma}
    \label{lem:finsetprop-finite}
    Let \(C \in \finsetprop\) be a finite cardinal.
    Then \(\finsetpropwithnat \slice C \cong \finsetprop \slice C\).
\end{lemma}
\begin{proof}
    The morphisms in \(\finsetpropwithnat \slice C\) are the morphisms
    \([m] \to C\) for finite \(C\), which are precisely the morphisms of
    \(\finsetprop \slice C\).
\end{proof}

This slips in to the proof above to extend it to \emph{countable} sums.

\begin{theorem}
    \label{thm:frobc-iso-hatfinset-slice-c}
    For a countable set \(C\), there is an isomorphism of coloured
    PROPs \(\frobc \cong \csp{\finsetpropwithnat \slice C}\).
\end{theorem}
\begin{proof}
    The proof is almost the same as \cref{thm:frobc-iso-finset-slice-c} but with
    the addition of \cref{lem:finsetprop-finite}.
    We have that \[
        \frobc
        \coloneqq
        \sum_{c \in C}\frob
        \cong
        \sum_{c \in C}\csp{\finsetprop}
        \cong
        \sum_{c \in C}\csp{\finsetprop \slice 1}
        \cong
        \sum_{c \in C}\csp{\finsetpropwithnat \slice 1}.
    \]
    In the other direction we still have that \(
    \csp{\finsetpropwithnat \slice C}
    \cong
    \csp{\finsetpropwithnat \slice \sum_{c \in C} 1}
    \) as \(C\) is still countable.
    As before we need to show that \(
    \sum_{c \in C}\csp{\finsetpropwithnat \slice 1}
    \cong
    \csp{\finsetpropwithnat \slice \sum_{c \in C} 1}
    \), which follows by the same reasoning as in the prequel.
\end{proof}

As with the monochromatic case, we must now define a map from finite sets to
discrete hypergraphs.

\begin{definition}
    For a countable set \(C\), let
    \(\morph{D_C}{\finsetpropwithnat \slice C}{\hypcsigmac}\) be
    defined as the functor that maps a coloured word \(\listvar{w}\) to the
    discrete coloured hypergraph containing an appropriately coloured vertex for
    each element of \(\listvar{w}\).
\end{definition}

\begin{corollary}[\cite{bonchi2022string}, Rem. 3.12]\label{cor:finsets-c-to-hyp}
    There is a faithful PROP homomorphism \(
    \morph{
        \widetilde{\pickinterfacesc{C}}
    }{
        \csp{\finsetpropwithnat \slice C}
    }{
        \cspdchyp
    }
    \).
\end{corollary}

It is now possible to map from coloured Frobenius terms to cospans of
hypergraphs.

\begin{definition}
    Let \(\morph{\frobtohypsigmac}{\frobc}{\cspdchyp}\) be the homomorphism
    obtained by composing the isomorphism of \cref{thm:frobc-iso-hatfinset-slice-c}
    with \cref{cor:finsets-c-to-hyp}.
\end{definition}

\begin{corollary}
    \(\cspdchyp\) is a hypergraph category.
\end{corollary}

\subsection{From terms to graphs}

Our goal is to map from terms in \(\hypcsigma\) into cospans in \(\cspdhyp\).
As we know that \(\hypcsigma\) can be viewed as the coproduct
\(\smcsigma + \frob\), it suffices to define this map in terms of a map from
\(\smcsigma\) and a map from \(\frob\).
The results of the previous section gives us the latter, so all that remains is
the former.

\begin{definition}[\cite{bonchi2022string}, Sec. 4.1]\label{def:hyp-morphisms}
    Let \(\morph{\termtohypsigma}{\smcsigma}{\cspdhyp}\) be a PROP
    morphism with the action on generators defined as \[
        \termtohypsigma[\iltikzfig{%
                strings/category/generator%
            }[box=\phi,colour=white,dom=m,cod=n]
        ]
        \coloneqq
        \cospan{m}{\iltikzfig{graphs/terms/generator}}{n}
    \]
\end{definition}

To map from terms in a hypergraph category to cospans of hypergraphs, we simply
put the two maps together.

\begin{definition}
    Let \(
    \morph{\termandfrobtohypsigma}{\smcsigma + \frob}{\cspdhyp}
    \) be the PROP morphism defined as the copairing of \(\termtohypsigma\) and
    \(\frobtohypsigma\).
\end{definition}

Already we have all we need to state one of the key results of
\cite{bonchi2022string}: the correspondence between terms with a Frobenius
structure and cospans of hypergraphs.
We will state one corollary concerning cospans of discrete hypergraphs before
proceeding to the main result.

\begin{corollary}
    \label{cor:discrete-hypergraph-frob}
    Given a discrete hypergraph \(k \in \hypsigma\), any cospan
    \(\cospan{m}{k}{n}\) in \(\cspdhyp\) is in the
    image of \(\frobtohypsigma\).
\end{corollary}
\begin{proof}
    By \cref{prop:frob-finset}.
\end{proof}

Cospans of this form will play a part in the main theorem, in which we show the
isomorphism between Frobenius terms and cospans of hypergraphs by decomposing a
given cospan into a particular form.

\begin{theorem}[\cite{bonchi2022string}, Theorem 4.1]\label{thm:isomorphism-smcfrob-cospans}
    There is an isomorphism of PROPs \(\smcsigma + \frob \cong \cspdchyp\).
\end{theorem}
\begin{proof}
    Since \(\smcsigma + \frob\) is a coproduct in \(\propcat\), this can be
    shown by proving that \(\cspdhyp\) satisfies the universal property of the
    coproduct: given a coloured PROP \(\mathbb{A}\) and PROP morphisms
    \(\smcsigma \to \mathbb{A}\) and \(\frob \to \mathbb{A}\), there exists
    a unique morphism \(\morph{u}{\cspdhyp}{\mathbb{A}}\) as below:

    \begin{center}
        \includestandalone{figures/graphs/isomorphism/coproduct-iso}
    \end{center}
    %
    All the PROP morphisms involved are identity-on-objects, so all that is
    required to show the existence of \(u\) is to show that any morphism in
    \(\cspdchyp\) can be expressed as a composition of components either in the
    image of \(\termtohypsigma\) or \(\frobtohypsigma\).

    Consider a cospan \(\cospan{m}[f]{G}[g]{n}\) in \(\cspdchyp\); let \(N\) be
    the set of nodes, let \(E\) be the set of hyperedges, and let
    \(\morph{\chi}{E}{\generators}\) be the induced labelling function.
    Pick an order \(e_0, e_1, e_{j-1}\) on the edges; then define \(
    \cospan{\tilde{m}}[s]{\tilde{E}}[t]{\tilde{n}}
    \) as the cospan \(
    \bigtensor_{0 \leq i < j}
    \termtohypsigma[\chi(e_i)]
    \).
    This cospan `stacks up' the edges in \(G\) without connecting them together;
    the legs of the cospan are the sources and targets of these
    edges concatenated in the order specified.
    It is easy to define functions \(\morph{f^\prime}{m}{N}\),
    \(\morph{g^\prime}{n}{N}\), \(\morph{h}{\tilde{m}}{N}\) and
    \(\morph{k}{\tilde{n}}{N}\) that send nodes to the corresponding node in the
    set of all nodes in the graph.

    With this data, the original cospan \(\cospan{m}[f]{G}[g]{n}\) can be
    viewed as the following composition of cospans: \[
        (
        \cospan{m}[f^\prime]{N}[(\id,h)]{N \tensor \tilde{n}}
        ) \seq (
        \cospan{
            N \tensor \tilde{n}
        }[
            \id \tensor s
        ]{
            N \tensor \tilde{E}
        }[
            \id \tensor t
        ]{
            N \tensor \tilde{m}
        }
        ) \seq (
        \cospan{N \tensor \tilde{m}}[(\id, k)]{N}[g^\prime]{n}
        ).
    \]
    This is all well-defined because \(\tensor\) is the coproduct in
    \(\hypsigmac\).
    By computing the composition by pushout, it can be shown that the composite
    above is isomorphic to the original cospan \(\cospan{m}[f]{G}[g]{n}\).

    Now it must be verified that each cospan in the composite is in the image
    of either \(\termtohypsigma\) or \(\frobtohypsigma\).
    The outer cospans are discrete so they are in the image of
    \(\frobtohypsigma\) by \cref{cor:discrete-hypergraph-frob}.
    The centre cospan is constructed from the identity and cospans in the image
    of \(\termtohypsigma\), so the entire cospan is in the
    image of \(\termtohypsigma\).

    The morphism \(u\) can therefore be defined based on the actions of
    \(\termtohypsigma\) and \(\frobtohypsigma\).
    This morphism is unique: although different orders can be assigned on the
    edges, all the categories are symmetric monoidal so this is not an issue.
\end{proof}

At first glance, the composite cospan described above might look confusing.
As mentioned, the central cospan \(\cospan{\tilde{m}}{\tilde{E}}{\tilde{n}}\)
serves to `stack up' the edges in some order, all detached from each other.
To make the entire cospan isomorphic to the original, the connections of the
sources and targets must be the same: the job of the two outer cospans is to
`join them up' appropriately by connecting targets on the right to sources on
the left by going `over the top' of the edges via the identity cospan
\(\cospan{N}{N}{N}\).

\begin{example}
    Consider the following term and its interpretation as a cospan of
    hypergraphs.
    \begin{center}
        \iltikzfig{graphs/frobenius/correspondence/original-term}
        \qquad
        \begin{tikzcd}
            \iltikzfig{graphs/frobenius/correspondence/inputs}
            \arrow{r}
            &
            \iltikzfig{graphs/frobenius/correspondence/example}
            &
            \arrow{l}
            \iltikzfig{graphs/frobenius/correspondence/outputs}
        \end{tikzcd}
    \end{center}
    This cospan can be assembled into the form detailed in the above proof as
    follows:
    \begin{center}
        \begin{tikzcd}[column sep=tiny]
            \iltikzfig{graphs/frobenius/correspondence/l-inputs}
            \arrow{r}
            &
            \iltikzfig{graphs/frobenius/correspondence/l}
            &
            \arrow{l}
            \iltikzfig{graphs/frobenius/correspondence/l-outputs}
        \end{tikzcd}
        \hspace{-1.1em}
        \(\seq\)
        \hspace{-1.1em}
        \begin{tikzcd}[column sep=tiny, row sep=0.1cm]
            \iltikzfig{graphs/frobenius/correspondence/e-inputs}
            \arrow{r}
            &
            \iltikzfig{graphs/frobenius/correspondence/n}
            &
            \arrow{l}
            \iltikzfig{graphs/frobenius/correspondence/e-outputs}
            \\
            \iltikzfig{graphs/frobenius/correspondence/e-inputs}
            \arrow{r}
            &
            \iltikzfig{graphs/frobenius/correspondence/e}
            &
            \arrow{l}
            \iltikzfig{graphs/frobenius/correspondence/e-outputs}
        \end{tikzcd}
        \hspace{-1.1em}
        \(\seq\)
        \hspace{-1.1em}
        \begin{tikzcd}[column sep=tiny]
            \iltikzfig{graphs/frobenius/correspondence/r-inputs}
            \arrow{r}
            &
            \iltikzfig{graphs/frobenius/correspondence/r}
            &
            \arrow{l}
            \iltikzfig{graphs/frobenius/correspondence/r-outputs}
        \end{tikzcd}
    \end{center}
    By following the vertex maps, one can verify that this is indeed isomorphic
    to the original cospan.
    The outermost components correspond to terms in \(\frob\) and the
    innermost to a term in \(\smcsigma\).
    \[
        \iltikzfig{graphs/frobenius/correspondence/term}
    \]
    This term is equal to the original term by the Frobenius equations.
\end{example}

This result means that any two terms in \(\smcsigma + \frob\) which are equal by
the Frobenius equations can be mapped to isomorphic cospans of hypergraphs.

\begin{example}
    Recall the following terms in \(\frob\) from \cref{ex:frobenius}, which
    we showed were equal by the Frobenius equations.
    \[
        \iltikzfig{strings/structure/frobenius/example-2}
        =
        \iltikzfig{strings/structure/frobenius/example-3}
    \]
    By the isomorphism of \cref{thm:isomorphism-smcfrob-cospans}, these two
    terms should map to the same cospan of hypergraphs and, indeed, they both
    map to the following:
    \[
        \iltikzfig{graphs/example-frobenius-collapse}
    \]
    All of the Frobenius structure collapses into one vertex, much like when we
    considered the correspondence between Frobenius terms and finite sets.
\end{example}

\subsection{The coloured correspondence}

The results for the monochromatic case also follow for the coloured case, so we
can restate the correspondence for coloured terms and hypergraphs.

\begin{definition}
    Let \(\morph{\termandfrobtohypsigmac}{\smcsigmac}{\cspdchyp}\) be defined
    as \(\termandfrobtohypsigma\) but assigning appropriate colours to the
    danging vertices.
\end{definition}

\begin{definition}
    Let \(
    \morph{\termandfrobtohypsigmac}{\smcsigmac + \frobc}{\cspdchyp}
    \) be the copairing of \(\termtohypsigmac\) and
    \(\frobtohypsigmac\).
\end{definition}

\begin{theorem}[\cite{bonchi2022string}, Prop. 4.4]
    There is an isomorphism of \(C\)-coloured PROPs
    \(\hypcsigmac \cong \cspdchyp\).
\end{theorem}
\begin{proof}
    In the same manner as \cref{thm:isomorphism-smcfrob-cospans}, but the
    components of the composite cospan now have appropriately coloured vertices.
\end{proof}
\section{Symmetric monoidal terms}

We have now seen that that cospans of hypergraphs are an excellent fit for
reasoning about terms in a freely generated hypergraph category.
However, there are times we might not have so much structure in our terms;
indeed for our case of digital circuits we only operate in a setting with a
trace.
This means that not every cospan of hypergraphs will correspond to a valid term.
Fortunately, Bonchi et al also characterised the cospans of hypergraphs that
correspond to \emph{symmetric monoidal} terms without any additional structure.
We will use some of this machinery when it comes to tackling the traced case.

\subsection{Monogamous acyclic cospans}

There are two features that distinguish vanilla symmetric monoidal terms from
Frobenius terms; wires cannot arbitrarily fork or join, and cycles may not be
created.
The former is tackled by a condition on the connectivity of vertices.

\begin{definition}[Degree {\cite[Def.\ 12]{bonchi2022stringa}}]
    \index{degree}
    For a hypergraph \(F \in \hyp\), the \emph{degree} of a vertex
    \(v \in \vertices{F}\) is a tuple \((i,o)\) where \(i\) is the number of
    hyperedges with with \(v\) as a target, and \(o\) is the number of
    hyperedges with \(v\) as a source.
\end{definition}

\begin{definition}[Monogamy {\cite[Def.\ 13]{bonchi2022stringa}}]
    \index{monogamy}
    \index{cospan!monogamous}
    For a cospan \(\cospan{m}[f]{F}[g]{n}\) in
    \(\cspdhyp\), let \(\mathsf{in}(F)\) and \(\mathsf{out}(F)\) be the image of
    \(f\) and \(g\) respectively.
    We call the cospan \(\cospan{m}[f]{F}[g]{n}\) \emph{monogamous} if \(f\) and
    \(g\) are mono and, for all vertices \(v\), the degree of \(v\) is
    \begin{center}
        \begin{tabular}{rlcrl}
            \((0,0)\)
             &
            if \(v \in \mathsf{in}(F) \wedge v \in \mathsf{out}(F)\)
             &
            \quad
             &
            \((0,1)\)
             &
            if \(v \in \mathsf{in}(F)\)
            \\
            \((1,0)\)
             &
            if \(v \in \mathsf{out}(F)\)
             &
            \quad
             &
            \((1,1)\)
             &
            otherwise
        \end{tabular}
    \end{center}
\end{definition}

\begin{example}\label{ex:monogamous}
    The following cospans of hypergraphs are monogamous:
    \[
        \iltikzfig{graphs/monogamy/yes-0}
        \quad
        \iltikzfig{graphs/monogamy/yes-1}
    \]
    The following cospans of hypergraphs are not monogamous:
    \[
        \iltikzfig{graphs/monogamy/no-0}
        \quad
        \iltikzfig{graphs/monogamy/no-1}
    \]
\end{example}

Since our goal is to assemble monogamous cospans into a category, it is
necessary to check that the property is preserved by the various categorical
operations.

\begin{lemma}[\cite{bonchi2022stringa}, Lem.\ 15]\label{lem:identities-symmetries-monogamous}
    Identities and symmetries are monogamous.
\end{lemma}
\begin{proof}
    The cospans involved are discrete and all vertices are in both
    interfaces, so the cospans are monogamous.
\end{proof}

\begin{lemma}[\cite{bonchi2022stringa}, Lem.\ 16]\label{lem:monogamicity-preserved-composition}
    Monogamicity is preserved by composition.
\end{lemma}
\begin{proof}
    Assume we compose two monogamous acyclic cospans \(
    \cospan{m}[f]{F}[g]{n}
    \) and \(
    \cospan{n}[h]{G}[k]{p}
    \).
    The interfaces remain mono as pushouts along monos are monos in
    \(\hypsigmac\).
    The only altered vertices are those in the image of
    \(g\) and \(h\), which are merged pointwise; vertices in the image of
    \(g\) have out-degree \(0\) and those in the image of \(h\) have in-degree
    \(0\) so the merged vertices have at most degree \((1, 1)\).
\end{proof}

\begin{lemma}[\cite{bonchi2022stringa}, Lem.\ 17]\label{lem:monogamicity-preserved-tensor}
    Monogamicity is preserved by tensor.
\end{lemma}
\begin{proof}
    The degrees of vertices are unaffected as tensor is by coproduct and
    only vertices in the original interfaces will be in the new interfaces.
\end{proof}

As seen in \cref{ex:monogamous}, monogamy makes no guarantees about cycles.
Since symmetric monoidal terms cannot have cycles, a notion of \emph{acyclicity}
must also be enforced.

\begin{definition}[Predecessor {\cite[Def.\ 18]{bonchi2022stringa}}]
    \index{predecessor}
    A hyperedge \(e\) is a \emph{predecessor} of hyperedge \(e^\prime\)
    if there exists a vertex \(v\) in the sources of \(e\) and the targets of
    \(e^\prime\).
\end{definition}

\begin{definition}[Path {\cite[Def.\ 19]{bonchi2022stringa}}]
    \index{path}
    A \emph{path} between two hyperedges \(e\) and \(e^\prime\) is a sequence of
    hyperedges \(e_0, \dots, e_{n-1}\) such that \(e = e_0\),
    \(e^\prime = e_{n-1}\), and for each \(i < n-1\), \(e_i\) is a predecessor
    of \(e_{i+1}\).
    A subgraph \(H\) is the \emph{start} or \emph{end} of a path if it contains
    a vertex in the sources of \(e\) or the targets of \(e^\prime\) respectively.
\end{definition}

Since vertices are single-element subgraphs, one can also talk about paths from
vertices.

\begin{definition}[Acyclicity {\cite[Def.\ 20]{bonchi2022stringa}}]
    \index{acyclicity}
    \index{cospan!acyclic}
    A hypergraph \(F\) is acyclic if there is no path from a vertex to itself.
    A cospan \(\cospan{m}{F}{n}\) is acyclic if \(F\) is.
\end{definition}

\begin{example}\label{ex:acyclic}
    The following cospans of hypergraphs are acyclic:
    \[
        \iltikzfig{graphs/acyclic/yes-0}
        \quad
        \iltikzfig{graphs/acyclic/yes-1}
    \]
    The following cospans of hypergraphs are not acyclic:
    \[
        \iltikzfig{graphs/acyclic/no-0}
        \quad
        \iltikzfig{graphs/acyclic/no-1}
    \]
\end{example}

Once again, for acyclicity to be a suitable condition on a category of cospans,
it needs to be preserved by categorical operations.

\begin{lemma}[{\cite[Prop.\ 21]{bonchi2022stringa}}]\label{lem:identities-symmetries-monogamous-acyclic}
    Identities and symmetries are acyclic.
\end{lemma}
\begin{proof}
    By \cref{lem:identities-symmetries-monogamous} as discrete
    hypergraphs cannot contain cycles.
\end{proof}

\begin{lemma}[{\cite[Prop.\ 21]{bonchi2022stringa}}]\label{lem:monogamous-acyclicity-preserved-tensor}
    Acyclicity is preserved by tensor.
\end{lemma}
\begin{proof}
    The original graphs are not altered.
\end{proof}

When turning to composition, we run into a problem; composition of arbitrary
cospans may not preserve acyclicity.
It is only when acyclicity is combined with monogamy that composition can be
safely performed.

\begin{lemma}[{\cite[Prop.\ 21]{bonchi2022stringa}}]\label{lem:monogamous-acyclicity-preserved-composition}
    Monogamous acyclicity is preserved by composition.
\end{lemma}
\begin{proof}
    Assume we compose two monogamous acyclic cospans \(
    \cospan{m}[f]{F}[g]{n}
    \) and \(
    \cospan{n}[h]{G}[k]{p}
    \).
    A cycle cannot be created by composition because there cannot be a path in
    \(F\) that starts in the image of \(g\) or a path in \(G\) that ends in the
    image of \(h\), because these vertices have out-degree and in-degree \(0\)
    respectively.
\end{proof}

This shows that monogamous acyclic cospans of hypergraphs form a category.

\begin{definition}[\cite{bonchi2022stringa}]
    \index{macsphyp@\(\macspdhyp\)}
    \nomenclature{\(\macspdhyp\)}{PROP of monoamous acyclic cospans of hypergraphs}
    Let \(\macspdhyp\) be defined as the sub-PROP of \(\cspdhyp\) containing the
    monogamous acyclic cospans of hypergraphs.
\end{definition}

\begin{example}\label{ex:monogamous-acyclic}
    The following cospans of hypergraphs are monogamous acyclic:
    \[
        \iltikzfig{graphs/monogamous-acyclic/yes-0}
        \quad
        \iltikzfig{graphs/monogamous-acyclic/yes-1}
    \]
    The following cospans of hypergraphs are not monogamous acyclic:
    \[
        \iltikzfig{graphs/monogamous-acyclic/no-0}
        \quad
        \iltikzfig{graphs/monogamous-acyclic/no-1}
    \]
\end{example}

Just like how cospans of hypergraphs correspond to string diagrams of Frobenius
terms, monogamous acyclic terms correspond to string diagrams of symmetric
monoidal terms.
Bonchi et al showed this by proving that \(\smcsigma\) is isomorphic to
\(\macspdhyp\); to do this they needed a few more ingredients.
The first is a lemma showing that a special class of subgraphs can always be
`extracted' from a parent graph.

\begin{definition}[Convex subgraph {\cite[Def.\ 23]{bonchi2022stringa}}]
    \index{convex subgraph}
    A subgraph \(G \subseteq F\) is convex if for any vertices \(v, v^\prime\) in
    \(G\) and any path \(p\) from \(v\) to \(v^\prime\), every edge \(e\) in
    \(p\) is also in \(G\).
\end{definition}

\begin{lemma}[Decomposition {\cite[Lem.\ 24]{bonchi2022stringa}}]
    \label{lem:decomposition}
    For a monogamous acyclic cospan \(\cospan{m}{F}{n}\) and
    and convex subgraph \(L\) of \(G\), there exist
    \(k \in \nat\) and a unique cospan
    \(\cospan{i}{L}{j}\) such that \(G\) can be factored as
    the following composite of monogamous acyclic cospans:
    \[
        (\cospan{m}{C_1}{k + i})
        \seq
        (\cospan{
            k + i
        }{
            k + L
        }{
            k + i
        })
        \seq
        (\cospan{k + j}{C_2}{n})
    \]
\end{lemma}

Essentially, we can always `pull out' a convex subgraph of a monogamous acyclic
cospan in such a way that the remaining cospans are still monogamous acyclic.
This is an important part of characterising the image of \(\termtohypsigma\).

\begin{theorem}[\cite{bonchi2022stringa}, Thm.\ 25]\label{thm:monogamous-acyclic-full}
    A cospan \(\cospan{m}{F}{n}\) is in the image of
    \(\termtohypsigma\) if and only if \(\cospan{m}{F}{n}\)
    is monogamous acyclic.
\end{theorem}
\begin{proof}
    The \(\onlyifdir\) direction is by induction on the structure of terms in
    \(\smcsigma\): the interpretation of generators is monogamous acyclic and
    the inductive cases are by
    \cref{lem:identities-symmetries-monogamous,lem:identities-symmetries-monogamous-acyclic,lem:monogamous-acyclicity-preserved-composition,lem:monogamicity-preserved-tensor,lem:monogamous-acyclicity-preserved-tensor}.

    The \(\ifdir\) direction is by induction on the number of edges in \(F\).
    If there are none, then \(m \to F\) and \(n \to F\) are
    bijections by monogamy so the term is in the image of
    identities or symmetries in \(\smcsigma\).
    For the inductive step, pick a single edge \(e\).
    This is a convex subgraph of \(F\), so
    \(\cospan{m}{F}{n}\) can be factored as in
    \cref{lem:decomposition}.
    The edge \(e\) has a label \(\chi(e) \in \generators\), so the subgraph
    \(\cospan{i}{e}{j}\) is the result of
    \(\termtohypsigma[\chi(e)]\).
    Since the remaining cospans are monogamous acyclic by
    \cref{lem:decomposition}, they are in the image of \(\termtohypsigma\) by
    inductive hypothesis, so the original cospan
    \(\cospan{m}{F}{n}\) is also in the image of
    \(\termtohypsigma\).
\end{proof}

This shows that \(\termtohypsigma\) is full; to conclude the isomorphism we
need to show that it is also faithful.
We know that the copairing \(\termtohypsigma + \frobtohypsigma\) is faithful
by \cref{thm:isomorphism-smcfrob-cospans}, so we just need to show the same is
true for its components, using a result about pushouts in \(\propcat\).

\begin{definition}[\cite{macdonald2009amalgamations}, Defs. 3.1, 3.2]
    \index{3-for-2 property}
    A functor \(\morph{F}{\mcc}{\mcd}\) satisfies the \emph{3-for-2 property}
    if, for each triple of morphisms \(f,g,h \in \mcd\) such that
    \(h = g \circ f\), if any two of \(f\), \(g\) and \(h\) are in the image of
    \(F\), then the third is also in the image of \(F\).
\end{definition}

\begin{theorem}[\cite{macdonald2009amalgamations}, Thm.\ 3.3]
    \label{thm:faithful-pushout}
    Let \(\morph{F_A}{\mcc}{\mca}\) and \(\morph{F_B}{\mcc}{\mcb}\) be faithful
    functors such that the following diagram is a pushout.
    \begin{center}
        \includestandalone{figures/category/amalgamation}
    \end{center}
    Then, if \(F_A\) and \(F_B\) both satisfy the 3-for-2 property, then the
    functors \(G_A\) and \(G_B\) are also faithful.
\end{theorem}

To apply this result, we need to show that \(\hypsigma\) is a pushout.

\begin{definition}
    \index{perm@\(\permsprop\)}
    \nomenclature{\(\permsprop\)}{PROP of finite sets and bijective functions}
    Let \(\permsprop\) be the sub-PROP of \(\finsetprop\)
    containing the bijective functions.
\end{definition}

A morphism in \(\permsprop\) is a permutation of wires.
As all the functions are bijections, there can only be morphisms \(m \to m\).

\begin{lemma}\label{lem:perms-initial}
    \(\permsprop\) is the initial object in \(\propcat\).
\end{lemma}
\begin{proof}
    All the morphisms in \(\permsprop\) are identities and symmetries; the
    coloured PROP morphism to any other PROP maps these to the corresponding
    constructs.
\end{proof}

Subsequently, \(\smcsigma + \frob\) can be expressed as a pushout and the
`3-for-2' condition applied to show the faithfulness of \(\termtohypsigma\),
using another well-known categorical lemma.

\begin{lemma}[\cite{borceux1994handbook}, Prop.\ 2.8.2]\label{lem:coproducts-pushout}
    If a category \(\mcc\) has pushouts and an initial object, then \(\mcc\)
    also has coproducts.
\end{lemma}
\begin{proof}
    Given objects \(A,B \in \mcc\), the coproduct \(A + B\) is constructed as
    follows:
    \begin{gather*}
        \includestandalone{figures/category/diagrams/coproduct-pushout}%
    \end{gather*}
    This is a coproduct due to the universal property of pushouts.
\end{proof}

\begin{proposition}
    \(\morph{\termtohypsigma}{\smcsigma}{\cspdhyp}\) is faithful.
\end{proposition}
\begin{proof}
    From \cref{thm:isomorphism-smcfrob-cospans}, we know that
    \(\cspdhyp \cong \smcsigma + \frob\).
    Both \(\smcsigma\) and \(\frob\) are objects of \(\propcat\), which has
    \(\permsprop\) as its initial object by \cref{lem:perms-initial}.
    As coproducts are pushouts from the initial object
    (\cref{lem:coproducts-pushout}), we can construct the following diagram in
    \(\propcat\):
    \begin{center}
        \includestandalone{figures/graphs/coproduct-pushout-graphs}
    \end{center}
    where \(!_1\) and \(!_2\) are the unique morphisms from
    \(\permsprop\) induced by initiality: these are both faithful.
    \(!_1\) and \(!_2\) clearly satisfy the 3-for-2 condition as every morphism
    in \(\permsprop\) is an isomorphism, so \(\termtohypsigma\) must also
    be faithful by \cref{thm:faithful-pushout}.
\end{proof}

Since \(\termandfrobtohypsigma\) is full and faithful, we have reached our
final destination.

\begin{corollary}[\cite{bonchi2022stringa}, Cor.\ 26]
    There is an isomorphism of PROPs \(\smcsigma \cong \macspdhyp\).
\end{corollary}

\subsection{Coloured symmetric monoidal terms}

To generalise the above results to the countably coloured case, the only
modification is to apply the 3-for-2 condition in the category of \(C\)-coloured
PROPs.

\begin{lemma}
    \index{permhat@\(\permspropwithnat\)}
    \nomenclature{\(\permspropwithnat\)}{PROP of finite sets augmented with natural numbers and bijective functions}
    Let \(\permspropwithnat\) be the sub-PROP of \(\finsetpropwithnat\)
    containing only the bijective functions.
\end{lemma}

\begin{lemma}
    For a countable set of colours \(C\), \(\permspropwithnat\) is the initial
    object in \(\cpropc\).
\end{lemma}

This means that we obtain correspondence results in the coloured case.

\begin{proposition}
    \(\morph{\termandfrobtohypsigmac}{\smcsigmac}{\cspdchyp}\) is faithful.
\end{proposition}

\begin{corollary}
    For a countable set of colours \(C\), there is an isomorphism of
    \(C\)-coloured PROPs \(\smcsigmac \cong \macspdchyp\).
\end{corollary}
\section{Traced terms}

We now turn our attention to \emph{traced} terms.
First, we establish the link between terms in a hypergraph category and a
those in a symmetric traced monoidal category.

\begin{lemma}[\cite{rosebrugh2005generic}, Prop. 2.8]
    Every hypergraph category is self-dual compact closed.
\end{lemma}
\begin{proof}
    The cup on a given object is defined as \(
        \iltikzfig{strings/compact-closed/cup-self-dual}[colour=white,obj=A]
        :=
        \iltikzfig{strings/structure/frobenius/cup}[obj=A]
    \) and the cap as \(
        \iltikzfig{strings/compact-closed/cap-self-dual}[colour=white,obj=A]
        :=
        \iltikzfig{strings/structure/frobenius/cap}[obj=A]
    \).
    The snake equations follow by applying the Frobenius equation and unitality:
    \begin{gather*}
        \iltikzfig{strings/structure/frobenius/snake-1-0}
        =
        \iltikzfig{strings/structure/frobenius/snake-1-1}
        =
        \iltikzfig{strings/structure/frobenius/snake-1-2}
        \qquad
        \iltikzfig{strings/structure/frobenius/snake-2-0}
        =
        \iltikzfig{strings/structure/frobenius/snake-2-1}
        =
        \iltikzfig{strings/structure/frobenius/snake-2-2}
        \qedhere
    \end{gather*}
\end{proof}

There are more terms in a freely generated compact closed category
than there are in a freely generated traced category, as in the former outputs
are not always required to connect to inputs.
Since cospans of hypergraphs in \(\cspdchyp\) are in correspondence with terms
in \(\hypcsigmac\), this means that there will be cospans in the former that do
not have a corresponding traced term.
However, this does not mean we need to discard all the work of the previous
section.
Instead, we can reuse the components and adapt them for our traced setting.

Since we have a map from \(\smcsigmac + \frobc\) to cospans of hypergraphs, we
need to interpret every term in \(\stmcsigmac\) in terms of components from
either \(\smcsigmac\) or \(\frobc\).

\begin{lemma}
    \label{lem:stmc-subcat-hypc}
    \(\stmcsigmac\) is a subcategory of \(\hypcsigmac\).
\end{lemma}
\begin{proof}
    Since \(\hypsigmac\) is compact closed, it has a (canonical) trace.
    For \(\stmcsigmac\) to be a subcategory of \(\hypcsigmac\), every morphism
    of the former must also be a morphism on the latter.
    Since the two categories are freely generated (with the trace constructed
    through the Frobenius generators in the latter), all that remains is to
    check that every morphism in \(\stmcsigmac\) is a unique morphism in
    \(\hypcsigmac\), i.e.\ the equations of \(\frobc\) do not merge any together.
    This is trivial since the equations do not apply to the construction of the
    canonical trace.
\end{proof}

\begin{definition}
    Let \(\morph{\tracedtosymandfrobsigmac}{\stmcsigmac}{\hypcsigmac}\) be the
    inclusion functor induced by \cref{lem:stmc-subcat-hypc}.
\end{definition}

\begin{corollary}
    \(\tracedtosymandfrobsigmac\) is faithful.
\end{corollary}

To translate a term in \(\stmcsigmac\) into a cospan of hypergraphs, one simply
applies \(\tracedtosymandfrobsigmac\) followed by \(\termandfrobtohypsigmac\).

\begin{corollary}
    \(\termandfrobtohypsigmac \circ \tracedtosymandfrobsigmac\) is faithful.
\end{corollary}

Since \(\termandfrobtohypsigmac \circ \tracedtosymandfrobsigmac\) is faithful,
every distinct traced term in \(\stmcsigmac\) has a \emph{unique} cospan of
hypergraphs up to isomorphism.
However, this functor is not clearly not \emph{full}: as we have previously
illustrated, there are more terms in \(\hypcsigmac\) than there are in
\(\stmcsigmac\).

The next step is to characterise the \emph{image} of
\(\termandfrobtohypsigmac \circ \tracedtosymandfrobsigmac\).
A similar task was performed in \cite{bonchi2022stringa} for terms in
\(\smcsigmac\); we will recall their definitions and then adapt them for our
case.

\subsection{Partial monogamy}

Since monogamous acyclic cospans correspond exactly to symmetric monoidal terms,
it is too restrictive to be used as a setting for modelling traced terms.
Clearly, we can drop the acyclicity condition, as the trace can introduce
cycles.
However, there is one foible regarding the monogamicity condition which must
also be tackled.
Although wires are also not permitted to arbitrarily fork or join in a traced
category, it is also possible to have a case where wires do not connect to
any generators while also remaining disconnected from the interfaces.
This special case is the trace of the identity, which in string diagrams is
depicted as a closed
loop \(
    \trace{1}{\iltikzfig{strings/category/identity}[colour=white]}
    =
    \iltikzfig{strings/traced/trace-id}
\).
One might think this can be discarded, in that \(
    \iltikzfig{strings/traced/trace-id}
    =
    \iltikzfig{strings/monoidal/empty}
\), but this is \emph{not} always the case; for example, it is not true
in \(\finvectk{k}\)~\cite[Sec. 6.1]{hasegawa1997recursion}.
Instead, we can model these closed loops as lone vertices (`bones') disconnected
from either interface.

\begin{definition}[Partial monogamy]
    A cospan \(\cospan{m}[f]{F}[g]{n} \in \cspdhyp\) is
    \emph{partial monogamous} if \(f\) and \(g\) are mono and, for all nodes
    \(v \in \vertices{F}\), the degree of \(v\) is
    \begin{center}
        \begin{tabular}{rlcrl}
            \((0,0)\)
            &
            if \(v \in f_\star \wedge v \in g_\star\)
            &
            \quad
            &
            \((0,1)\)
            &
            if \(v \in f_\star\)
            \\
            \((1,0)\)
            &
            if \(v \in g_\star\)
            &
            \quad
            &
            \((0,0)\)
            or \((1,1)\)
            &
            otherwise
        \end{tabular}
    \end{center}
\end{definition}

\begin{figure}
    \centering
    \[
        \underbrace{
            \iltikzfig{graphs/monogamy/yes-0}
            \iltikzfig{graphs/monogamy/yes-1}
        }_{\text{partial monogamous}}
        \qquad
        \underbrace{
            \iltikzfig{graphs/monogamy/no-0}
            \iltikzfig{graphs/monogamy/no-1}
        }_{\text{not partial monogamous}}
    \]
    \caption{Examples of cospans that are and are not partial monogamous.}
    \label{fig:partial-monogamous-examples}
\end{figure}

\begin{example}
    Examples of cospans that are and are not partial monogamous are shown
    in \cref{fig:partial-monogamous-examples}.
\end{example}

As with the monogamous acyclic cospans, a sub-PROP of \(\cspdchyp\) containing
the partial monogamous cospans must be constructed.

\begin{lemma}
    Identities and symmetries are partial monogamous.
\end{lemma}
\begin{proof}
    Identities and symmetries are monogamous by (1) in
    \cref{lem:monogamous-acyclic-preserved} so they must also be partial
    monogamous.
\end{proof}

\begin{lemma}
    Given partial monogamous cospans \(\cospan{\listvar{m}}{F}{\listvar{n}}\)
    and \(\cospan{\listvar{n}}{G}{\listvar{p}}\), \(
        (\cospan{\listvar{m}}{F}{\listvar{n}})
        \seq
        (\cospan{\listvar{n}}{G}{\listvar{p}})
    \) is partial monogamous.
\end{lemma}
\begin{proof}
    By (2), composition preserves monogamicity.
    The only difference between partial monogamous cospans and monogamous ones
    is that the former may have cycles and nodes of degree \((0,0)\) not in the
    interfaces.
    However, since neither of these can be interfaces they cannot be altered by
    composition, so partial monogamy must also be preserved.
\end{proof}

\begin{lemma}
    Given partial monogamous cospans \(\cospan{\listvar{m}}{F}{\listvar{n}}\)
    and \(\cospan{\listvar{p}}{G}{\listvar{q}}\), \(
        (\cospan{\listvar{m}}{F}{\listvar{n}})
        \tensor
        (\cospan{\listvar{n}}{G}{\listvar{p}})
    \) is partial monogamous.
\end{lemma}
\begin{proof}
    As with composition, tensor preserves monogamicity by
    \cite[Lem. 17]{bonchi2022stringa}, and as it does not affect the degree of
    nodes then it preserves partial monogam as well.
\end{proof}

As partial monogamicity is preserved by both forms of composition, the
partial monogamous cospans themselves form a PROP.

\begin{definition}
    Let \(\pmcspdchyp\) be the sub-PROP of \(\cspdchyp\) containing only the
    partial monogamous cospans of hypergraphs.
\end{definition}

Of course, we are not done yet, as we are concerned with \emph{traced} terms:
we must show that \(\pmcspdchyp\) is also traced.

\begin{theorem}
    \label{thm:partial-monogamous-ops}
    \(\pmcspdchyp\) has a trace defined as the trace in \(\cspdchyp\).
\end{theorem}
\begin{proof}
    Consider a partial monogamous cospan \(
        \cospan{\listvar{x} + \listvar{m}}[f + h]{F}[g + k]{\listvar{x} + \listvar{n}}
    \); we must show that its trace \(
        \cospan{\listvar{m}}[h]{F^\prime}[k]{\listvar{n}}
    \) is also partial monogamous.
    Only the vertices in the image of \(f\) and \(g\) are affected by the trace:
    for each element \(a \in \listvar{x}\), \(f(a)\) and \(g(a)\) will be merged
    together in the traced graph, summing their degrees.
    If these vertices are in the image of \(h\) or \(k\) then this will be
    preserved in the traced cospan.

    We now consider the various cases:
    \begin{itemize}
        \item if \(f(a) = g(a)\), then this vertex must have degree \((0, 0)\);
                the traced vertex will still have degree \((0, 0)\) and will no
                longer be in the interface, so will still be partial monogamous;
        \item if \(f(a)\) is also in the image of \(k\) and \(g(i)\) is also in
                the image of \(h\), then both \(f(a)\) and \(g(a)\) have degree
                \((0, 0)\); the traced vertex will still have degree
                \((0, 0)\) and be in both interfaces of the traced cospan, so
                will still be partial monogamous;
        \item if \(f(a)\) is also in the image of \(k\), then \(f(i)\) will have
                degree \((0, 0)\) and \(g(a)\) will have degree \((1,0)\), so
                the traced vertex will have degree \((1, 0)\) and be in the
                image of \(k\); and
        \item if \(g(i)\) is in the image of \(h\), then the above argument
                applies in reverse. \qedhere
    \end{itemize}
\end{proof}

Crucially, while we leave \(\pmcspdhyp\) in order to construct the trace using
the cup and cap, the resulting cospan \emph{is} in \(\pmcspdhyp\).

\subsection{The traced correspondence}

To show that cospans of partial monogamous hypergraphs are a suitable
interpretation of terms in \(\stmcsigmac\), we must show two things.
First we must show that the image of
\(\termandfrobtohypsigma \circ \tracedtosymandfrob{\Sigma}\) is always in
\(\pmcspdchyp\), and then we must show that this mapping is \emph{full} when
restricted to \(\pmcspdhyp\).
For the former, we will use a similar strategy to that used to show the
correspondence between \(\smcsigmac + \frobc\) and \(\cspdchyp\) in
\cref{thm:isomorphism-smcfrob-cospans}; for a given partial monogamous cospan of
hypergraphs we will define an isomorphic composite of components formed from
taking the trace over elements in the image of \(\termtohypsigmac\).

\begin{corollary}
    For a discrete hypergraph \(X \in \hypsigmac\), any monogamous cospan
    \(\cospan{\overline{m}}{X}{\overline{m}}\) is in the image of the unique
    morphism \(\morph{!}{\hat{\permsprop} \slice C}{\cspdchyp}\).
\end{corollary}

Cospans of this for

\begin{theorem}\label{thm:termtohyp-image}
    A cospan \(\cospan{m}{F}{n}\) is in the image of \(
        \termandfrobtohypsigmac \circ \tracedtosymandfrobsigmac\) if
    and only if it is partial monogamous.
\end{theorem}
\begin{proof}
    Since the generators \(\stmcsigmac\) are mapped to monogamous cospans
    by \(\termandfrobtohypsigmac \circ \tracedtosymandfrobsigmac\) and partial
    monogamy is preserved by composition and trace
    (\cref{thm:partial-monogamous-ops}), every cospan in the image of
    \(\termandfrobtohypsigmac \circ \tracedtosymandfrobsigmac\) is partial
    monogamous.

    Now we show that any partial monogamous cospan \(
        \cospan{m}[f]{F}[g]{n}
    \) must be in the image of \(
        \termtohypsigmac \circ \tracedtosymandfrobsigmac
    \) by constructing an isomorphic cospan from a trace of cospans
    The components of the new cospan are as follows:
    \begin{itemize}
        \item let \(L\) be the hypergraph containing vertices with degree
                \((0,0)\) that are not in the image of \(f\) or \(g\);
        \item let \(E\) be the hypergraph containing hyperedges of \(F\) and
                their source and target vertices, but disconnected;
        \item let \(V\) be the discrete hypergraph containing all the
                vertices of \(F\); and
        \item let \(S\) and \(T\) be the discrete hypergraphs containing
                the source and target vertices of hyperedges in \(F\)
                respectively, with the ordering determined by some order
                \(e_1,e_2,\cdots,e_n\) on the edges in \(F\).
    \end{itemize}

    These parts can be composed and a trace applied to obtain the follow
    cospan:
    \begin{gather}
        \trace{T}{
            \cospan{T + m}[\id + f]{V}[\id + g]{S + n}
            \,\seq\,
            \cospan{\emptyset + S + n}[\id]{L + E + n}[\id]{\emptyset + T + n}
        }
        \label{gat:cospan}
    \end{gather}

    This can be checked to be isomorphic to the original cospan
    \(\cospan{m}[f]{F}[g]{n}\) by applying the pushouts.
    From this we can read off a term in \(\stmc{\Sigma}\):
    Since the first cospan is monogamous, it corresponds to a term \(
        \iltikzfig{strings/category/f-2-2}[box=f,colour=white,dom1={|\vertices{T}|},dom2=m,cod1={|\vertices{S}|},cod2=n]
    \) by \cref{lem:monog-discrete-cospan}.
    The second cospan corresponds to \(
        \iltikzfig{strings/category/f}[box=g,colour=white,dom={|\vertices{S}|},cod={|\vertices{T}|}]
        :=
        \bigtensor_{v \in \vertices{L}}
        \iltikzfig{strings/traced/trace-id}
        \tensor
        \bigtensor_{e \in 0 \leq i \leq n}
        \iltikzfig{graphs/isomorphism/label-e}
        \tensor
        \iltikzfig{strings/category/identity}[colour=white,obj=n]
    \), where \(\elabel{}(e)\) is the generator in \(\generators\) that \(e\) is
    labelled with.
    Putting this all together yields \(
        h := \termtohypsigma[\iltikzfig{graphs/isomorphism/construction}]
    \).
    While there may be multiple orderings on the edges, the possible terms
    are equal by sliding and the naturality of symmetry, so there is one
    unique term \(
        \iltikzfig{strings/category/f}[box=h,colour=white]
    \) that corresponds to cospan (\ref{gat:cospan}).
    It is clear by definition that \(
        \termtohypsigma[\iltikzfig{strings/category/f}[box=h,colour=white]]
    \) produces (\ref{gat:cospan}), which is isomorphic to the original
    cospan \(\cospan{m}[f]{F}[g]{n}\), so it is in the image of
    \(\termtohypsigma \circ \tracedtosymandfrob{\Sigma}\).
\end{proof}

This shows that \(
    \termandfrobtohypsigma \circ \tracedtosymandfrob{\Sigma}
\) is a \emph{full} mapping from \(\stmcsigma\) to \(\pmcspdhyp\).
As \(\tracedtosymandfrob{\Sigma}\) is faithful by definition and
\(\termandfrobtohypsigma\) are faithful by \cref{prop:tohyp-faithful}, the
entire mapping is also faithful: \(\stmcsigma\) is mapped to a \emph{unique}
cospan of hypergraphs up to isomorphism.

\begin{corollary}\label{cor:stmc-graph-iso}
    \(\stmc{\Sigma} \cong \pmcspfihyp\).
\end{corollary}
\section{Hypergraphs for traced comonoid terms}

By characterising the cospans of hypergraphs that correspond to \emph{traced}
terms, we already have a setting in which we can model sequential circuit
morphisms combinatorially.
But we can go further.
When modelling \emph{Frobenius} terms, we were modelling them modulo the
Frobenius equations; when interpreted as cospans of hypergraphs the
comonoid and monoid structures merged together into single nodes so we did
not need to consider the equations of associativity, commutativity or unitality.

In the realm of sequential circuits we also have monoid and comonoid structures,
but instead of forming a Frobenius structure they only form a \emph{bialgebra}.
The equations of a bialgebra are different to those of a Frobenius algebra in
how the monoid and comonoid interact.
Compare the two Frobenius equations with the bialgebra equation shown below:
\[
    \iltikzfig{strings/structure/bialgebra/merge-copy-lhs}
    =
    \iltikzfig{strings/structure/frobenius/frobenius-l}
    \qquad
    \iltikzfig{strings/structure/bialgebra/merge-copy-lhs}
    =
    \iltikzfig{strings/structure/frobenius/frobenius-r}
    \qquad\qquad
    \iltikzfig{strings/structure/bialgebra/merge-copy-lhs}
    =
    \iltikzfig{strings/structure/bialgebra/merge-copy-rhs}
\]
In a Frobenius setting, it is possible to derive the bialgebra
equation \(\bialgcommoneqn\) from the Frobenius equations combined with the
equations of monoids and comonoids.

\begin{lemma}
    In \(\hypcsigma\), \(
    \iltikzfig{strings/structure/bialgebra/merge-copy-rhs}
    =
    \iltikzfig{strings/structure/bialgebra/merge-copy-lhs}
    \).
\end{lemma}
\begin{proof}
    \begin{gather*}
        \iltikzfig{strings/structure/bialgebra/merge-copy-rhs}
        \eqaxioms
        \iltikzfig{strings/structure/bialgebra/frobenius-bialgebra/step-1}
        \eqaxioms[(\comonoidcommeqn)]
        \iltikzfig{strings/structure/bialgebra/frobenius-bialgebra/step-2}
        \eqaxioms[(\frobreqn)]
        \iltikzfig{strings/structure/bialgebra/frobenius-bialgebra/step-3}
        \eqaxioms[(\frobleqn)]
        \\[0.5em]
        \iltikzfig{strings/structure/bialgebra/frobenius-bialgebra/step-4}
        \eqaxioms[(\monoidassoceqn)]
        \iltikzfig{strings/structure/bialgebra/frobenius-bialgebra/step-5}
        \eqaxioms[(\monoidcommeqn)]
        \iltikzfig{strings/structure/bialgebra/frobenius-bialgebra/step-6}
        \eqaxioms[(\frobleqn)]
        \\[0.5em]
        \iltikzfig{strings/structure/bialgebra/frobenius-bialgebra/step-7}
        \eqaxioms[(\frobspeceqn)]
        \iltikzfig{strings/structure/bialgebra/frobenius-bialgebra/step-8}
    \end{gather*}
\end{proof}

This is clear from the hypergraph interpretation, as all four terms
involved map to the same (discrete) cospan of hypergraphs.
%
\begin{gather*}
    \frobtohypsigma[
        \iltikzfig{strings/structure/frobenius/frobenius-l}
    ]
    =
    \frobtohypsigma[
        \iltikzfig{strings/structure/frobenius/frobenius-r}
    ]
    =
    \frobtohypsigma[
        \iltikzfig{strings/structure/bialgebra/merge-copy-lhs}
    ]
    =
    \frobtohypsigma[
        \iltikzfig{strings/structure/bialgebra/merge-copy-rhs}
    ]
    =
    \iltikzfig{graphs/example-frobenius-collapse}
\end{gather*}

However, the converse does not hold: it is not possible to derive the Frobenius
equations from the bialgebra equation without having one of the Frobenius
equations to begin with.
This poses a problem: we want to use \(\cspdchyp\) as a setting for rewriting
digital circuits, but as it by default contains a Frobenius structure,
\emph{too many} equations would hold.
Since the issue only arises with the interactions between the monoid and the
comonoid, we can use cospans of hypergraphs to reason modulo the equations of
just one of the two structures.

\begin{remark}
    Alas, we cannot claim to have pioneered the idea of interpreting terms with
    just a monoid or comonoid structure as cospans of
    hypergraphs~\cite{fritz2023free,milosavljevic2023string}.
    What we bring to the table is studying how such terms interact with the
    \emph{trace}: does removing acyclicity lead to any degeneracies?
\end{remark}

In the case of sequential circuits, it makes sense to focus on the comonoid
structure, as forking wires is far more of a natural concept than joining them.
To characterise categories of terms with a comonoid structure, we must first
define the monoidal theory of cocommutative comonoids.

\begin{definition}
    Let \((\generators[\ccomon], \equations[\ccomon])\) be the symmetric
    monoidal theory of \emph{cocommutative comonoids}, with \(
    \Sigma_{\ccomon} \coloneqq \{
    \iltikzfig{strings/structure/comonoid/copy}[colour=white],
    \iltikzfig{strings/structure/comonoid/discard}[colour=white]
    \}
    \) and \(\mce_{\ccomon}\) defined as in \cref{fig:comonoid-equations}.
    We write \(
    \ccomon \coloneqq \smc{\generators[\ccomon], \equations[\ccomon]}
    \).
\end{definition}

\begin{figure}
    \begin{gather*}
        \equationdisplay{
            \iltikzfig{strings/structure/comonoid/unitality-l-lhs}
        }{
            \iltikzfig{strings/structure/comonoid/unitality-l-rhs}
        }{
            \comonoiduniteqnletter
        }
        \qquad
        \equationdisplay{
            \iltikzfig{strings/structure/comonoid/associativity-lhs}
        }{
            \iltikzfig{strings/structure/comonoid/associativity-rhs}
        }{
            \comonoidassoceqnletter
        }
        \qquad
        \equationdisplay{
            \iltikzfig{strings/structure/comonoid/associativity-lhs}
        }{
            \iltikzfig{strings/structure/comonoid/associativity-rhs}
        }{
            \comonoidcommeqnletter
        }
    \end{gather*}
    \caption{
        Equations \(\equations[\ccomon]\) of a \emph{commutative comonoid}
    }
    \label{fig:comonoid-equations}
\end{figure}

From now on, we write `comonoid' to mean `cocommutative comonoid'.

When identifying the cospans of hypergraphs that correspond to terms with traced
comonoid structure, the notion of monogamy will once again need to be modified.
Partial monogamy is now too strong, as this means wires cannot fork.
Weakening to no monogamy at all is too much, as we do not want wires to join as
well as fork.
Effectively, nodes need to be `monogamous on one side'.

\begin{definition}[Partial left-monogamy]
    For a cospan \(\cospan{m}[f]{H}[g]{n}\), we say it is
    \emph{partial left-monogamous} if \(f\) is mono and, for all nodes
    \(v \in H_\star\), the degree of \(v\) is \((0,m)\) if \(v \in f_\star\) and
    \((0,m)\) or \((1,m)\) otherwise, for some \(m \in \nat\).
\end{definition}

Partial left-monogamy is a weakening of partial monogamy that allows nodes
to have multiple `out' connections, which represent the use of the comonoid
structure to fork wires; nodes must still only have one `in' connection.

\begin{example}\label{ex:partial-left-monogamous}
    The following cospans are partial left-monogamous:
    \[
        \iltikzfig{graphs/partial-monogamy/yes-comonoid-0}
        \quad
        \iltikzfig{graphs/partial-monogamy/yes-comonoid-1}
    \]
    The following cospans are not partial left-monogamous:
    \[
        \iltikzfig{graphs/partial-monogamy/no-comonoid-0}
        \quad
        \iltikzfig{graphs/partial-monogamy/no-comonoid-1}
    \]
\end{example}

\begin{remark}
    As with the nodes not in the interfaces with degree \((0, 0)\) in the
    vanilla traced case, the nodes not in the interface with degree
    \((0, m)\) allow for the interpretation of terms such as \(
    \trace{}{\iltikzfig{strings/structure/comonoid/copy}[colour=white]}
    \).
    \begin{gather*}
        \frobtohypsigma[\trace{}{\iltikzfig{strings/structure/comonoid/copy}}]
        =
        \trace{}{\iltikzfig{graphs/fork-cospan}}
        =
        \iltikzfig{graphs/trace-fork-cospan}
    \end{gather*}
\end{remark}

We must ensure that partial left-monogamy is preserved by the categorical
operations, so that partial left-monogamous cospans form another PROP.

\begin{lemma}\label{lem:partial-monogamous-id-sym}
    Identities and symmetries are partial left-monogamous.
\end{lemma}
\begin{proof}
    Again by \cref{lem:identities-symmetries-monogamous}, identities and
    symmetries are monogamous so they are also partial left-monogamous.
\end{proof}

\begin{lemma}
    Given two partial left-monogamous cospans
    \(\cospan{m}{F}{n}\) and
    \(\cospan{n}{G}{p}\), the composition \(
    (\cospan{m}{F}{n})
    \seq
    (\cospan{n}{G}{p})
    \) is partial left-monogamous.
\end{lemma}
\begin{proof}
    Only the nodes in the image of \(n \to G\) have their in-degree modified;
    they gain the in-tentacles of the corresponding nodes in the image of
    \(n \to F\).
    Initially the nodes in \(n \to G\) have in-degree \(0\) by partial
    monogamy; they will gain at most one in-tentacle from nodes in
    \(n \to F\) as each of these nodes has in-degree \(0\) or \(1\) and
    \(n \to G\) is mono.
    So the composite graph is partial left-monogamous.
\end{proof}

\begin{lemma}
    Given two partial left-monogamous cospans \(\cospan{m}{F}{n}\)
    and \(\cospan{p}{G}{q}\), the tensor \(
    (\cospan{m}{F}{n})
    \tensor
    (\cospan{n}{G}{p})
    \) is partial left-monogamous.
\end{lemma}
\begin{proof}
    The elements of the original graphs are unaffected.
\end{proof}

This means we can assemble the partial left-monogamous cospans of hypergraphs
into the desired sub-PROP.

\begin{definition}
    Let \(\plmcspdhyp\) be the sub-PROP of \(\cspdhyp\) containing only the
    partial left-monogamous cospans of hypergraphs.
\end{definition}

As this PROP is not restricted to acyclic cospans like those used for just
terms with a (co)monoid structure, it has the additional structure of a trace.

\begin{proposition}
    The canonical trace is a trace on \(\plmcspdhyp\).
\end{proposition}
\begin{proof}
    We must show that for any nodes in the image
    of \(x + n \to K\) merged by the canonical trace, at most one of them can
    have in-degree \(1\).
    This follows because anything in the image of
    \(x + m \to K\) must have in-degree \(0\), and \(x + m \to K\) is
    mono so it cannot merge nodes in the image of \(x + n \to K\).
\end{proof}

We now have the setting in which we will model terms with a comonoid structure.
To actually define the mapping from \(\stmcsigma + \ccomon\) we will reuse
some ingredients from the previous sections.

\begin{definition}
    Let \(
    \morph{
        \comonoidtofrob
    }{
        \ccomon
    }{
        \frob
    }
    \) be the embedding of \(\ccomon\) into \(\frob\), and let \(
    \morph{
        \tracedandcomonoidtofrobsigma
    }{
        \stmcsigma + \ccomon
    }{
        \smcsigma + \frob
    }
    \) be the copairing \(\tracedtosymandfrobsigma + \comonoidtofrob\).
\end{definition}

\begin{corollary}
    \(\comonoidtofrob\) and \(\tracedtosymandfrobsigma\) are faithful.
\end{corollary}

After translating from \(\stmcsigma + \ccomon\) to \(\smcsigma + \frob\),
we can then use the previously defined PROP morphism \(\termandfrobtohypsigma\)
to obtain a cospan of hypergraphs; as before.
To show that partial left-monogamy is the correct notion to characterise terms
in a traced comonoid setting, it is necessary to ensure that the image of these
PROP morphisms actually lands in \(\plmcspdhyp\).
First we verify that this

\begin{lemma}
    The image of \(\frobtohypsigma \circ \comonoidtofrob\) is in
    \(\plmcspdhyp\).
\end{lemma}
\begin{proof}
    This is straightforward by inspecting the cases.
\end{proof}

To show the correspondence between \(\stmcsigma + \ccomon\) and
\(\plmcspdhyp\), we use a similar strategy to \cref{thm:termtohyp-image}.

\begin{lemma}\label{lem:discrete-mono}
    Given a discrete hypergraph \(X \in \hypsigma\), any cospan
    \(\cospan{m}[f]{X}{n}\) with \(f\) mono is in the image of
    \(\frobtohypsigma \circ \comonoidtofrob\).
\end{lemma}
\begin{proof}
    By definition of \(\frobtohypsigma \circ \comonoidtofrob\).
\end{proof}

\begin{theorem}\label{thm:comonoid-fully-complete}
    A cospan of hypergraphs is in the image of
    \(\stmcsigma + \ccomon \cong \plmcspdhyp\) if and only if it is partial
    left-monogamous.
\end{theorem}
\begin{proof}
    It suffices to show that a cospan \(\cospan{m}{F}{n}\) in
    \(\plmcspdhyp\) can be decomposed into a traced cospan in which every
    component under the trace is in the image of either
    \(\termandfrobtohypsigma\) or \(\frobtohypsigma \circ \comonoidtofrob\).
    This is achieved by taking the construction of \cref{thm:termtohyp-image}
    and allowing the first component to be partial left-monogamous; by
    \cref{lem:discrete-mono} this is in the image of
    \(\frobtohypsigma \circ \comonoidtofrob\).
    The remaining components remain in the image of \(\termtohypsigma\).
    Subsequently, the entire traced cospan must be in the image of \(
    \termandfrobtohypsigma \circ \tracedandcomonoidtofrobsigma
    \).
\end{proof}

The composite cospan for the comonoid correspondence is broadly the same as that
of the traced correspondence, but now the term derived from the discrete
component may additionally contain the comonoid and the counit.

As \(\termandfrobtohypsigma\) and \(\tracedandcomonoidtofrobsigma\) are
faithful, we immediately find the following.

\begin{corollary}
    There is an isomorphism of coloured PROPs
    \(\stmcsigma + \ccomon \cong \plmcspdhyp\).
\end{corollary}

This means the PROP \(\plmcspdhyp\) of partial left-monogamous cospans of
hypergraphs is suitable for modelling terms in \(\stmcsigma + \ccomon\):
traced terms with a cocommutative comonoid structure.

\begin{example}
    The partial monogamous cospans from \cref{ex:partial-left-monogamous} are
    shown below with their corresponding terms in \(\stmcsigma + \ccomon\).
    \[
        \iltikzfig{graphs/partial-monogamy/yes-comonoid-0}
        \quad
        \iltikzfig{graphs/terms/comonoid-term-0}
        \qquad
        \iltikzfig{graphs/partial-monogamy/yes-comonoid-1}
        \quad
        \iltikzfig{graphs/terms/comonoid-term-1}
    \]
\end{example}

\subsection{Traced coloured comonoids}

As usual, the results of the monochromatic setting generalise in a
straightforward manner to the multicoloured setting.
As with \(\frob\), a multicoloured version of \(\ccomon\) is defined as a
coproduct in \(\cprop\).

\begin{definition}
    For a countable set \(C\), let
    \(\ccomonc \coloneqq \Sigma_{c \in C} \ccomon\).
\end{definition}

Partial left-monogamy follows as before, so we have a traced coloured PROP.

\begin{definition}
    Let \(\plmcspdchyp\) be the sub-PROP of \(\cspdchyp\) containing only the
    partial left-monogamous cospans of hypergraphs.
\end{definition}

\begin{proposition}
    The canonical trace is a trace on \(\plmcspdchyp\).
\end{proposition}

As we embedded \(\ccomon\) into \(\frob\), we embed \(\ccomonc\) into
\(\frobc\).

\begin{definition}
    Let \(
    \morph{
        \comonoidtofrobc
    }{
        \ccomonc
    }{
        \frobc
    }
    \) be the embedding of \(\ccomonc\) into \(\frobc\), and let \(
    \morph{
        \tracedandcomonoidtofrobsigmac
    }{
        \stmcsigmac + \ccomonc
    }{
        \smcsigmac + \frobc
    }
    \) be the copairing of \(\tracedtosymandfrobsigmac\) and
    \(\comonoidtofrobc\).
\end{definition}

\begin{corollary}
    \(\comonoidtofrobc\) and \(\tracedtosymandfrobsigmac\) are faithful.
\end{corollary}

The monochromatic results can then be lifted to the coloured case in the same
way.

\begin{theorem}\label{thm:comonoidc-fully-complete}
    A cospan of hypergraphs is in the image of
    \(\stmcsigmac + \ccomonc \cong \plmcspdchyp\) if and only if it is partial
    left-monogamous.
\end{theorem}

\begin{corollary}
    There is an isomorphism of coloured PROPs
    \(\stmcsigmac + \ccomonc \cong \plmcspdchyp\).
\end{corollary}
