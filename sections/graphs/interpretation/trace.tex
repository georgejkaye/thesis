\section{Traced terms}

We have now seen the classes of cospans of hypergraphs that correspond to
terms in a hypergraph category and terms in a symmetric monoidal category.
Terms in a symmetric traced monoidal category sit somewhere in the middle of
these two: cycles are permitted but wires cannot fork or join arbitrarily.
This means to characterise the cospans of hypergraphs that correspond to traced
terms we need to weaken the conditions of monogamy and acyclicity described in
the previous section.

First we will establish the morphism \(\stmcsigmac \to \cspdchyp\) we will use
to map traced terms to cospans of hypergraphs.
We could do this manually fairly easily, but this would mean we would have to
redo all the proofs in the previous section from scratch.
Instead we will reuse the previous results, by exploiting the correspondence
between traced and compact closed categories.

\begin{lemma}[\cite{rosebrugh2005generic}, Prop. 2.8]
    Every hypergraph category is self-dual compact closed.
\end{lemma}
\begin{proof}
    The cup on a given object is defined as \(
        \iltikzfig{strings/compact-closed/cup-self-dual}[colour=white,obj=A]
        \coloneqq
        \iltikzfig{strings/structure/frobenius/cup}[obj=A]
    \) and the cap as \(
        \iltikzfig{strings/compact-closed/cap-self-dual}[colour=white,obj=A]
        \coloneqq
        \iltikzfig{strings/structure/frobenius/cap}[obj=A]
    \).
    The snake equations follow by applying the Frobenius equation and unitality:
    \begin{gather*}
        \iltikzfig{strings/structure/frobenius/snake-1-0}
        =
        \iltikzfig{strings/structure/frobenius/snake-1-1}
        =
        \iltikzfig{strings/structure/frobenius/snake-1-2}
        \qquad
        \iltikzfig{strings/structure/frobenius/snake-2-0}
        =
        \iltikzfig{strings/structure/frobenius/snake-2-1}
        =
        \iltikzfig{strings/structure/frobenius/snake-2-2}
        \qedhere
    \end{gather*}
\end{proof}

\begin{lemma}
    \label{lem:stmc-subcat-hypc}
    \(\stmcsigmac\) is a subcategory of \(\hypcsigmac\).
\end{lemma}
\begin{proof}
    Since \(\hypsigmac\) is compact closed, it has a (canonical) trace.
    For \(\stmcsigmac\) to be a subcategory of \(\hypcsigmac\), every morphism
    of the former must also be a morphism on the latter.
    Since the two categories are freely generated (with the trace constructed
    through the Frobenius generators in the latter), all that remains is to
    check that every morphism in \(\stmcsigmac\) is a unique morphism in
    \(\hypcsigmac\), i.e.\ the equations of \(\frobc\) do not merge any together.
    This is trivial since the equations do not apply to the construction of the
    canonical trace.
\end{proof}

\begin{definition}
    Let \(\morph{\tracedtosymandfrobsigmac}{\stmcsigmac}{\hypcsigmac}\) be the
    inclusion functor induced by \cref{lem:stmc-subcat-hypc}.
\end{definition}

\begin{corollary}
    \(\tracedtosymandfrobsigmac\) is faithful.
\end{corollary}

To translate a term in \(\stmcsigmac\) into a cospan of hypergraphs, one uses
the inclusion functor \(\tracedtosymandfrobsigmac\) to elevate to the
Frobenius realm, before applying \(\termandfrobtohypsigmac\) from the previous
section to obtain a cospan of hypergraphs.

\begin{corollary}
    \(\termandfrobtohypsigmac \circ \tracedtosymandfrobsigmac\) is faithful.
\end{corollary}

\subsection{Partial monogamy}

Since \(\termandfrobtohypsigmac \circ \tracedtosymandfrobsigmac\) is faithful,
every distinct traced term in \(\stmcsigmac\) has a unique cospan of
hypergraphs up to isomorphism.
However, this functor is not clearly not full: as we have previously
illustrated, there are more terms in \(\hypcsigmac\) than there are in
\(\stmcsigmac\).
The next step is to characterise the image of
\(\termandfrobtohypsigmac \circ \tracedtosymandfrobsigmac\).

Since monogamous acyclic cospans correspond exactly to symmetric monoidal terms,
this property is too restrictive to be used as a setting for modelling traced
terms.
Clearly, we can drop the acyclicity condition, as the trace can introduce
cycles.
However, there is also a foible regarding the monogamicity condition which must
also be tackled.
Although wires are also not permitted to arbitrarily fork or join in a traced
category, it is possible to have a case where wires do not connect to
any generators while also remaining disconnected from the interfaces.
This special case is the trace of the identity, which in string diagrams is
depicted as a closed
loop \(
    \trace{1}{\iltikzfig{strings/category/identity}[colour=white]}
    =
    \iltikzfig{strings/traced/trace-id}
\).

\begin{remark}
    One might think a closed loop can be discarded, in that \(
        \iltikzfig{strings/traced/trace-id}
        =
        \iltikzfig{strings/monoidal/empty}
    \), but this is \emph{not} always the case.
    For example, it is not true in
    \(\finvectk{k}\)~\cite[Sec. 6.1]{hasegawa1997recursion}.
\end{remark}

These closed loops can be modelled as lone vertices (`bones') disconnected from
either interface.

\begin{definition}[Partial monogamy]
    For a cospan \(\cospan{\listvar{m}}[f]{F}[g]{\listvar{n}}\) in
    \(\cspdchyp\), let \(\mathsf{in}(F)\) be the image of \(f\) and let
    \(\mathsf{out}(F)\) be the image of \(g\).
    A cospan \(\cospan{m}[f]{F}[g]{n} \in \cspdhyp\) is
    \emph{partial monogamous} if \(f\) and \(g\) are mono and, for all nodes
    \(v\), the degree of \(v\) is
    \begin{center}
        \begin{tabular}{rlcrl}
            \((0,0)\)
            &
            if \(v \in \mathsf{in}(F) \wedge v \in \mathsf{out}(F)\)
            &
            \quad
            &
            \((0,1)\)
            &
            if \(v \in \mathsf{in}(F)\)
            \\
            \((1,0)\)
            &
            if \(v \in \mathsf{out}(F)\)
            &
            \quad
            &
            \((0,0)\)
            or \((1,1)\)
            &
            otherwise
        \end{tabular}
    \end{center}
\end{definition}

\begin{figure}
    \centering
    \[
        \underbrace{
            \iltikzfig{graphs/monogamy/yes-0}
            \iltikzfig{graphs/monogamy/yes-1}
        }_{\text{partial monogamous}}
        \qquad
        \underbrace{
            \iltikzfig{graphs/monogamy/no-0}
            \iltikzfig{graphs/monogamy/no-1}
        }_{\text{not partial monogamous}}
    \]
    \caption{Examples of cospans that are and are not partial monogamous.}
    \label{fig:partial-monogamous-examples}
\end{figure}

\begin{example}
    Examples of cospans that are and are not partial monogamous are shown
    in \cref{fig:partial-monogamous-examples}.
\end{example}

As with the monogamous acyclic cospans, a sub-PROP of \(\cspdchyp\) containing
the partial monogamous cospans must be constructed.

\begin{lemma}\label{lem:identities-symmetries-partial-monogamous}
    Identities and symmetries are partial monogamous.
\end{lemma}
\begin{proof}
    Identities and symmetries are monogamous by
    \cref{lem:identities-symmetries-monogamous} so they must also be partial
    monogamous.
\end{proof}

\begin{lemma}\label{lem:partial-monogamicity-preserved-composition}
    Partial monogamy is preserved by composition.
\end{lemma}
\begin{proof}
    By \cref{lem:monogamicity-preserved-composition}, composition preserves
    monogamicity.
    The only difference between partial monogamous cospans and monogamous ones
    is that the former may have cycles and nodes of degree \((0,0)\) not in the
    interfaces.
    However, since neither of these can be interfaces they cannot be altered by
    composition, so partial monogamy must also be preserved.
\end{proof}

\begin{lemma}\label{lem:partial-monogamicity-preserved-tensor}
    Partial monogamy is preserved by tensor.
\end{lemma}
\begin{proof}
    As with composition, tensor preserves monogamicity by
    \cref{lem:monogamicity-preserved-tensor}, and as tensor does not affect the
    degree of nodes then it preserves partial monogamy as well.
\end{proof}

As partial monogamicity is preserved by both forms of composition, the
partial monogamous cospans themselves form a PROP.

\begin{definition}
    Let \(\pmcspdchyp\) be the sub-PROP of \(\cspdchyp\) containing only the
    partial monogamous cospans of hypergraphs.
\end{definition}

While for the symmetric monoidal case we could stop here, we have no such luck
now: we must show that \(\pmcspdchyp\) is also traced.
Although \(\cspdhyp\) already has a trace in the form of the canonical trace, we
must make sure that this does not degenerate for cospans of partial monogamous
hypergraphs.

\begin{theorem}\label{thm:partial-monogamous-trace}
    The canonical trace is a trace on \(\pmcspdhyp\).
\end{theorem}
\begin{proof}
    Consider a partial monogamous cospan \(
        \cospan{x + m}[f + h]{F}[g + k]{x + n}
    \); we must show that its trace \(
        \cospan{m}[h]{F^\prime}[k]{n}
    \) is also partial monogamous.
    Only the vertices in the image of \(f\) and \(g\) are affected by the trace:
    for each element \(a \in x\), \(f(a)\) and \(g(a)\) will be merged
    together in the traced graph, summing their degrees.
    If these vertices are in the image of \(h\) or \(k\) then this will be
    preserved in the traced cospan.
    We now consider the various cases:
    \begin{itemize}
        \item if \(f(a) = g(a)\), then this vertex must have degree \((0, 0)\);
                the traced vertex will still have degree \((0, 0)\) and will no
                longer be in the interface;
        \item if \(f(a)\) is also in the image of \(n \to F\) and \(g(i)\) is
                also in the image of \(m \to F\), then both \(f(a)\) and
                \(g(a)\) have degree \((0, 0)\); the traced vertex will still
                have degree \((0, 0)\) and be in both interfaces of the traced
                cospan;
        \item if \(f(a)\) is also in the image of \(n \to F\), then \(f(i)\)
                has \((0, 0)\) and \(g(a)\) will have degree
                \((1,0)\), so the traced vertex will have degree \((1, 0)\) and
                be in the image of \(n \to F^\prime\); and
        \item if \(g(i)\) is in the image of \(m \to F\), then the above
                argument applies in reverse.
    \end{itemize}
    In all these cases, partial monogamy is preserved, so the canonical trace is
    still suitable for working with cospans of partial monogamous hypergraphs.
\end{proof}

Crucially, while we leave \(\pmcspdhyp\) in order to construct the trace using
the cup and cap, the resulting cospan \emph{is} in \(\pmcspdhyp\).

\subsection{The traced correspondence}

Now that we have a traced sub-PROP of cospans of hypergraphs, it is time to show
that this particular sub-PROP is the one that corresponds to traced terms.

\begin{theorem}\label{thm:termtohyp-image}
    A cospan \(\cospan{m}{F}{n}\) is in the image of \(
        \termandfrobtohypsigmac \circ \tracedtosymandfrobsigmac\) if
    and only if it is partial monogamous.
\end{theorem}
\begin{proof}
    Since the generators of \(\stmcsigma\) are mapped to monogamous cospans
    by \(\termandfrobtohypsigmac \circ \tracedtosymandfrobsigmac\) and partial
    monogamy is preserved by composition
    (\cref{lem:partial-monogamicity-preserved-composition}),
    tensor (\cref{lem:partial-monogamicity-preserved-tensor}),
    and trace
    (\cref{thm:partial-monogamous-trace}),
    every cospan in the image of
    \(\termandfrobtohypsigma \circ \tracedtosymandfrobsigma\) is partial
    monogamous.

    Now we show that any partial monogamous cospan \(
        \cospan{\listvar{m}}[f]{F}[g]{\listvar{n}}
    \) must be in the image of \(
        \termandfrobtohypsigmac \circ \tracedtosymandfrobsigmac
    \) by constructing an isomorphic cospan from a trace of cospans, in which
    each component under the trace is in the image of \(\termtohypsigmac\);
    subsequently this means that the entire cospan must be in the image of
    \(\termandfrobtohypsigmac \circ \tracedtosymandfrobsigmac\).
    The components of the new cospan are as follows:
    \begin{itemize}
        \item let \(L\) be the discrete hypergraph containing vertices with
                degree
                \((0,0)\) that are not in the image of \(f\) or \(g\);
        \item let \(E\) be the hypergraph containing hyperedges of \(F\)
                disconnected from each other along with their source and target
                vertices;
        \item let \(V\) be the discrete hypergraph containing all the
                vertices of \(F\); and
        \item let \(S\) and \(T\) be the discrete hypergraphs containing
                the source and target vertices of hyperedges in \(F\)
                respectively, with the ordering determined by some order
                \(e_1,e_2,\cdots,e_n\) on the edges in \(F\).
    \end{itemize}
    These parts can be composed to form the following composite:
    \begin{gather*}
        \cospan{L + T + m}[\id + \id + f]{L + V}[\id + \id + g]{L + S + n}
        \,\seq\,
        \cospan{L + S + n}[\id]{L + E + n}[\id]{L + T + n}
    \end{gather*}
    Finally we take the trace of \(L + T\) over this composite to obtain a
    cospan which can be checked to be isomorphic to the original cospan
    \(\cospan{m}[f]{F}[g]{n}\) by applying the pushouts.
    The components of the composite under the trace are all monogamous acyclic
    so must be in the image of \(\termtohypsigmac\) by
    \cref{thm:monogamous-acyclic-full}; this means there is a term
    \(f \in \smcsigmac\) such that \(\termtohypsigmac[f]\) is isomorphic to the
    original composite.
    Clearly the trace of \(f\) is in \(\stmcsigmac\), and so the trace of the
    composite is in the image of
    \(\termandfrobtohypsigmac \circ \tracedtosymandfrobsigmac\).
\end{proof}

This shows that \(
    \termandfrobtohypsigmac \circ \tracedtosymandfrobsigmac{\Sigma}
\) is a \emph{full} mapping from \(\stmcsigmac\) to \(\pmcspdchyp\).
As it is both full and faithful, we can then conclude the final result.

\begin{corollary}\label{cor:stmc-graph-iso}
    \(\stmc{\Sigma} \cong \pmcspdhyp\).
\end{corollary}

\todo[inline]{Do some examples}