\section{Traced terms}

We now turn our attention to \emph{traced} terms.
First, we establish the link between terms in a hypergraph category and a
those in a symmetric traced monoidal category.

\begin{lemma}[\cite{rosebrugh2005generic}, Prop. 2.8]
    Every hypergraph category is self-dual compact closed.
\end{lemma}
\begin{proof}
    The cup on a given object is defined as \(
        \iltikzfig{strings/compact-closed/cup-self-dual}[colour=white,obj=A]
        :=
        \iltikzfig{strings/structure/frobenius/cup}[obj=A]
    \) and the cap as \(
        \iltikzfig{strings/compact-closed/cap-self-dual}[colour=white,obj=A]
        :=
        \iltikzfig{strings/structure/frobenius/cap}[obj=A]
    \).
    The snake equations follow by applying the Frobenius equation and unitality:
    \begin{gather*}
        \iltikzfig{strings/structure/frobenius/snake-1-0}
        =
        \iltikzfig{strings/structure/frobenius/snake-1-1}
        =
        \iltikzfig{strings/structure/frobenius/snake-1-2}
        \qquad
        \iltikzfig{strings/structure/frobenius/snake-2-0}
        =
        \iltikzfig{strings/structure/frobenius/snake-2-1}
        =
        \iltikzfig{strings/structure/frobenius/snake-2-2}
        \qedhere
    \end{gather*}
\end{proof}

Recall that there are more terms in a freely generated compact closed category
than there are in a freely generated traced catgory, as in the former outputs
are not always required to connect to inputs.
Since cospans of hypergraphs in \(\cspdchyp\) are in correspondence with terms
in \(\hypcsigmac\), this means that there will be cospans in the former that do
not have a corresponding traced term.
However, this does not mean we need to discard all the work of the previous
section.
Instead, we can reuse the components and adapt them for our traced setting.

Since we have a map from \(\smcsigmac + \frobc\) to cospans of hypergraphs, we
need to interpret every term in \(\stmcsigmac\) in terms of components from
either \(\smcsigmac\) or \(\frobc\).

\begin{lemma}
    \(\stmcsigmac\) is a subcategory of \(\hypcsigmac\).
\end{lemma}
\begin{proof}
    Since \(\hypsigmac\) is compact closed, it has a (canonical) trace.
    For \(\stmcsigmac\) to be a subcategory of \(\hypcsigmac\), every morphism
    of the former must also be a morphism on the latter.
    Since the two categories are freely generated (with the trace constructed
    through the Frobenius generators in the latter), all that remains is to
    check that every morphism in \(\stmcsigmac\) is a unique morphism in
    \(\hypcsigmac\), i.e.\ the equations of \(\frobc\) do not merge any together.
    This is trivial since the equations do not apply to the construction of the
    canonical trace.
\end{proof}

\begin{definition}
    Let \(\morph{\tracedtosymandfrobsigmas}{\stmcsigmac}{\hypcsigmac}\) be the
    induced inclusion functor.
\end{definition}

\begin{corollary}
    \(\tracedtosymandfrobsigmas\) is faithful.
\end{corollary}

This means that a term in \(\stmcsigma\) is translated into a cospan of
hypergraphs by applying
\(\termandfrobtohypsigmac \circ \tracedtosymandfrobsigmas\).

\begin{corollary}
    \(\termandfrobtohypsigmac \circ \tracedtosymandfrobsigmas\) is faithful.
\end{corollary}
\begin{proof}
    \(\tracedtosymandfrobsigmas\) is faithful by definition and
    \(\termandfrobtohypsigmac\) is faithful by \cref{prop:tohyp-faithful}.
\end{proof}

This means that every distinct traced term has a \emph{unique} cospan of
partial monogamous hypergraphs up to isomorphism.
All that remains to show is that every

\begin{definition}
    Let \(\permsprop\) be the sub-PROP of \(\finsetprop\) containing only the
    bijective functions.
\end{definition}

\begin{lemma}\label{lem:symmetries-prop}
    \(\smc{} \cong \permsprop\).
\end{lemma}
\begin{proof}
    The morphism \(\morph{\phi}{\smc{}}{\permsprop}\) is defined over
    generators in \(\smc{}\) as \[
        \phi(\iltikzfig{strings/monoidal/empty}) = \{\}
        \quad
        \phi(\iltikzfig{strings/category/identity}[colour=white])
        =
        \{0 \mapsto 0\}
        \quad
        \phi(\iltikzfig{strings/symmetric/symmetry}[colour=white])
        =
        \{0 \mapsto 1, 1 \mapsto 0\}
    \]
    Since any term in \(\smc{}\) can be expressed using these generators,
    this defines the complete transformation.

    The reverse morphism \(\morph{\psi}{\finsetprop}{\smc{}}\) is inductively
    over the size of \(m\).
    For the base case \(\morph{f}{[0]}{[0]}\), let \(
        \phi(f) := \iltikzfig{strings/monoidal/empty}
    \).
    For \(
        \morph{f}{[k+1]}{[k+1]}
    \), let \(i\) such that \(f(i) = k+1\), and define the function \(
        \morph{f^\prime}{\nat_{k}}{\nat_{k}}
    \) as the function such that \(
        f^\prime(j) = f(j)
    \) if \(j < i\), and \(f(j+1)\) otherwise.
    Then \[
        \psi(f) := \iltikzfig{strings/symmetric/f-construction}.
    \]

    These are shown to be inverses by a simple induction in both directions.
\end{proof}

\begin{lemma}\label{lem:monog-discrete-cospan}
    Given a monogamous cospan \(\cospan{m}[f]{m}[g]{m}\), there exists a unique
    term \(
        \iltikzfig{strings/category/f}[box=h,colour=white,cod=m,dom=m]
        \in \smc{}
    \) up to the axioms of SMCs such that \(
        \termtohyp[\iltikzfig{strings/category/f}[box=h,colour=white]]{\Sigma}
        =
        \cospan{m}[f]{m}[g]{m}
    \).
\end{lemma}
\begin{proof}
    Since the cospan is monogamous, \(f\) and \(g\) are mono.
    As the cospan is also discrete, there exists a (unique) bijective
    function \(\morph{h^\prime}{[m]}{[m]}\) such that \(h^\prime(i) = j\) if
    \(f(i) = g(j)\).
    By \cref{lem:symmetries-prop}, there is a corresponding term \(
        \iltikzfig{strings/category/f}[box=h,colour=white,cod=m,dom=m]
        \in \smc{}
    \) that is unique up to SMC axioms: a simple induction shows that \(
        \termtohyp[\iltikzfig{strings/category/f}[box=h,colour=white]]{\Sigma}
        =
        \cospan{m}[f]{m}[g]{m}
    \).
\end{proof}

These cospans are used to construct a term in \(\stmcsigma\) for a given cospan
of partial monogamous hypergraphs, establishing partial monogamy as the
defining characteristic of cospans in the image of \(
    \termandfrobtohypsigma \circ \tracedtosymandfrob{\Sigma}
\).

\begin{theorem}\label{thm:termtohyp-image}
    A cospan \(\cospan{m}{F}{n}\) is in the image of \(
        \termandfrobtohypsigma \circ \tracedtosymandfrob{\Sigma}\) if
    and only if it is partial monogamous.
\end{theorem}
\begin{proof}
    To show that \(\tracedtosymandfrob[\termtohyp[f]{\Sigma}]{\Sigma}\) is partial
    monogamous for any \(f \in \stmc{\Sigma}\) we use induction on the structure of
    \(f\).
    Generators, identities and symmetries are partial monogamous, as
    semi-monogamicity is preserved by composition, tensor and trace.
    So \(\termtohyp[f]{\Sigma}\) is partial monogamous.

    Now we show that any partial monogamous cospan \(
        \cospan{m}[f]{F}[g]{n}
    \)
    must be in the image of \(
        \termtohyp{\Sigma} \circ \tracedtosymandfrob{\Sigma}
    \).
    To do this, we will now construct a cospan that is isomorphic to
    \(\cospan{m}[f]{F}[g]{n}\), but from which it is possible to read off a
    unique term in \(\stmc{\Sigma}\).
    The components of the new cospan are as follows:
    \begin{itemize}
        \item let \(L\) be the hypergraph containing vertices with degree
                \((0,0)\) that are not in the image of \(f\) or \(g\);
        \item let \(E\) be the hypergraph containing hyperedges of \(F\) and
                their source and target vertices, but disconnected;
        \item let \(V\) be the discrete hypergraph containing all the
                vertices of \(F\); and
        \item let \(S\) and \(T\) be the discrete hypergraphs containing
                the source and target vertices of hyperedges in \(F\)
                respectively, with the ordering determined by some order
                \(e_1,e_2,\cdots,e_n\) on the edges in \(F\).
    \end{itemize}

    These parts can be composed and a trace applied to obtain the follow
    cospan:
    \begin{gather}
        \trace{T}{
            \cospan{T + m}[\id + f]{V}[\id + g]{S + n}
            \,\seq\,
            \cospan{\emptyset + S + n}[\id]{L + E + n}[\id]{\emptyset + T + n}
        }
        \label{gat:cospan}
    \end{gather}

    This can be checked to be isomorphic to the original cospan
    \(\cospan{m}[f]{F}[g]{n}\) by applying the pushouts.
    From this we can read off a term in \(\stmc{\Sigma}\):
    Since the first cospan is monogamous, it corresponds to a term \(
        \iltikzfig{strings/category/f-2-2}[box=f,colour=white,dom1={|\vertices{T}|},dom2=m,cod1={|\vertices{S}|},cod2=n]
    \) by \cref{lem:monog-discrete-cospan}.
    The second cospan corresponds to \(
        \iltikzfig{strings/category/f}[box=g,colour=white,dom={|\vertices{S}|},cod={|\vertices{T}|}]
        :=
        \bigtensor_{v \in \vertices{L}}
        \iltikzfig{strings/traced/trace-id}
        \tensor
        \bigtensor_{e \in 0 \leq i \leq n}
        \iltikzfig{graphs/isomorphism/label-e}
        \tensor
        \iltikzfig{strings/category/identity}[colour=white,obj=n]
    \), where \(\elabel{}(e)\) is the generator in \(\generators\) that \(e\) is
    labelled with.
    Putting this all together yields \(
        h := \termtohypsigma[\iltikzfig{graphs/isomorphism/construction}]
    \).
    While there may be multiple orderings on the edges, the possible terms
    are equal by sliding and the naturality of symmetry, so there is one
    unique term \(
        \iltikzfig{strings/category/f}[box=h,colour=white]
    \) that corresponds to cospan (\ref{gat:cospan}).
    It is clear by definition that \(
        \termtohypsigma[\iltikzfig{strings/category/f}[box=h,colour=white]]
    \) produces (\ref{gat:cospan}), which is isomorphic to the original
    cospan \(\cospan{m}[f]{F}[g]{n}\), so it is in the image of
    \(\termtohypsigma \circ \tracedtosymandfrob{\Sigma}\).
\end{proof}

This shows that \(
    \termandfrobtohypsigma \circ \tracedtosymandfrob{\Sigma}
\) is a \emph{full} mapping from \(\stmcsigma\) to \(\pmcspdhyp\).
As \(\tracedtosymandfrob{\Sigma}\) is faithful by definition and
\(\termandfrobtohypsigma\) are faithful by \cref{prop:tohyp-faithful}, the
entire mapping is also faithful: \(\stmcsigma\) is mapped to a \emph{unique}
cospan of hypergraphs up to isomorphism.

\begin{corollary}\label{cor:stmc-graph-iso}
    \(\stmc{\Sigma} \cong \pmcspfihyp\).
\end{corollary}