\section{Symmetric monoidal terms}

The results of the previous section show that cospans of hypergraphs are an
excellent fit for reasoning about terms in a freely generated hypergraph
category.
However, there are times we might not have so much structure in our terms;
indeed for our case of digital circuits we only operate in a setting with a
trace.
This means that not every cospan of hypergraphs will correspond to a valid term.
Fortunately, Bonchi et al also characterised the cospans of hypergraphs that
correspond to \emph{symmetric monoidal} terms without any additional structure.
We will use some of this machinery when it comes to tackling the traced case.

\subsection{Monogamy}

There are two features that distinguish vanilla symmetric monoidal terms from
Frobenius terms; wires cannot arbitrarily fork or join, and cycles may not be
created.
The former is tackled by a condition on the connectivity of vertices.

\begin{definition}[Degree (\cite{bonchi2022stringa}, Def. 12)]
    For a hypergraph \(F \in \hyp\), the \emph{degree} of a vertex
    \(v \in \vertices{F}\) is a tuple \((i,o)\) where \(i\) is the number of
    pairs \((e,i)\) where \(e\) is a hyperedge with \(v\) as its \(i\)th target,
    and \(o\) is similarly the number of pairs \((e,j)\) where \(e\) is a
    hyperedge with \(v\) as its \(j\)th target.
\end{definition}

\begin{definition}[Monogamy (\cite{bonchi2022stringa}, Def. 13)]
    For a cospan \(\cospan{\listvar{m}}[f]{F}[g]{\listvar{n}}\) in
    \(\cspdchyp\), let \(\mathsf{in}(F)\) be the image of \(f\) and let
    \(\mathsf{out}(F)\) be the image of \(g\).
    \(\cospan{\listvar{m}}[f]{F}[g]{\listvar{n}}\) is \emph{monogamous} if
    \(f\) and \(g\) are mono and, for all nodes \(v\), the degree of \(v\) is
    \begin{center}
        \begin{tabular}{rlcrl}
            \((0,0)\)
            &
            if \(v \in \mathsf{in}(F) \wedge v \in \mathsf{out}(F)\)
            &
            \quad
            &
            \((0,1)\)
            &
            if \(v \in \mathsf{in}(F)\)
            \\
            \((1,0)\)
            &
            if \(v \in \mathsf{out}(F)\)
            &
            \quad
            &
            \((1,1)\)
            &
            otherwise
        \end{tabular}
    \end{center}
\end{definition}

\begin{lemma}[\cite{bonchi2022stringa}, Lem. 15]\label{lem:identities-symmetries-monogamous}
    Identities and symmetries are monogamous.
\end{lemma}
\begin{proof}
    The cospans involved are discrete and all vertices are in both
    interfaces, so the cospans are monogamous.
\end{proof}

\begin{lemma}[\cite{bonchi2022stringa}, Lem. 16]\label{lem:monogamicity-preserved-composition}
    Monogamicity is preserved by composition.
\end{lemma}
\begin{proof}
    Assume we compose two monogamous acyclic cospans \(
        \cospan{\listvar{m}}[f]{F}[g]{\listvar{n}}
    \) and \(
        \cospan{\listvar{n}}[h]{G}[k]{\listvar{p}}
    \).
    The interfaces remain mono as pushouts along monos are monos in
    \(\hypsigmac\). The only vertices that are altered are those in the image of
    \(g\) and \(h\), which are merged pointwise; since vertices in the image of
    \(g\) have out-degree \(0\) and those in the image of \(h\) have in-degree
    \(0\), the merged vertices will have at most degree \((1, 1)\).
\end{proof}

\begin{lemma}[\cite{bonchi2022stringa}, Lem. 17]\label{lem:monogamicity-preserved-tensor}
    The degrees of nodes are unaffected as tensor is by coproduct and
    only vertices in the original interfaces will be in the new interfaces.
\end{lemma}

\subsection{Acyclicity}

Identifying hypergraphs without cycles is a much more common task, but it still
requires a few additional definitions.

\begin{definition}[Predecessor (\cite{bonchi2022stringa}, Def. 18)]
    A hyperedge \(e\) is a \emph{predecessor} of another hyperedge \(e^\prime\)
    if there exists a node \(v\) in the sources of \(e\) and the targets of
    \(e^\prime\).
\end{definition}

\begin{definition}[Path (\cite{bonchi2022stringa}, Def. 19)]
    A \emph{path} between two hyperedges \(e\) and \(e^\prime\) is a sequence of
    hyperedges \(e_0, \dots, e_{n-1}\) such that \(e = e_0\),
    \(e^\prime = e_{n-1}\), and for each \(i < n-1\), \(e_i\) is a predecessor
    of \(e_{i+1}\).
    A subgraph \(H\) is the \emph{start} or \emph{end} of a path if it contains
    a node in the sources of \(e\) or the targets of \(e^\prime\) respectively.
\end{definition}

Since nodes are single-element subgraphs, one can also talk about paths from
nodes.

\begin{definition}[Acyclicity (\cite{bonchi2022stringa}, Def. 20)]
    A hypergraph \(F\) is acyclic if there is no path from a node to itself.
    A cospan \(\morph{\listvar{m}}{F}{\listvar{n}}\) is acyclic if \(F\) is.
\end{definition}

Once again, for acyclicity to be a suitable condition on a category of cospans,
it needs to be preserved by the various categorical operations.

\begin{lemma}[\cite{bonchi2022stringa}, Prop. 21]\label{lem:identities-symmetries-monogamous-acyclic}
    Identities and symmetries are acyclic.
\end{lemma}
\begin{proof}
    This follows by \cref{lem:identities-symmetries-monogamous} as discrete
    hypergraphs cannot contain cycles.
\end{proof}

\begin{lemma}[\cite{bonchi2022stringa}, Prop. 21]\label{lem:monogamous-acyclicity-preserved-tensor}
    Acyclicity is preserved by tensor.
\end{lemma}
\begin{proof}
    Cycles cannot be created by the coproduct as the original graphs are not
    altered.
\end{proof}

When we turn to composition, we run into a problem; composition of arbitrary
cospans may not preserve acyclicity!
It is only when acyclicity is combined with monogamy that we are safe.

\begin{lemma}[\cite{bonchi2022stringa}, Prop. 21]\label{lem:monogamous-acyclicity-preserved-composition}
    Monogamous acyclicity is preserved by composition.
\end{lemma}
\begin{proof}
    Once again assume we compose two monogamous acyclic cospans \(
        \cospan{\listvar{m}}[f]{F}[g]{\listvar{n}}
    \) and \(
        \cospan{\listvar{n}}[h]{G}[k]{\listvar{p}}
    \).
    A cycle cannot be created by composition because there cannot be a path in
    \(F\) that starts in the image of \(g\) or a path in \(G\) that ends in the
    image of \(h\), because these vertices have out-degree and in-degree \(0\)
    respectively.
\end{proof}


This shows that monogamous acyclic cospans of hypergraphs form a category.

\begin{definition}[\cite{bonchi2022stringa}]
    Let \(\macspdchyp\) be the sub-PROP of \(\cspdchyp\) containing only the
    monogamous acyclic cospans of hypergraphs.
\end{definition}

Bonchi et al then proceeded to show that \(\smcsigmac\) is isomorphic to
\(\macspdchyp\); to do this they needed a few more ingredients.
The first is a lemma showing that a special class of subgraphs can always be
`extracted' from a parent graph.

\begin{definition}[Convex subgraph (\cite{bonchi2022stringa}, Def. 23)]
    A subgraph \(G \subseteq F\) is convex if for any nodes \(v, v^\prime\) in
    \(G\) and any path \(p\) from \(v\) to \(v^\prime\), every edge \(e\) in
    \(p\) is also in \(G\).
\end{definition}

\begin{lemma}[Decomposition (\cite{bonchi2022stringa}, Lem. 24)]
    \label{lem:decomposition}
    For a monogamous acyclic cospan \(\cospan{\listvar{m}}{F}{\listvar{n}}\) and
    and convex subgraph \(L\) of \(G\), there exist
    \(\listvar{k} \in \freemon{C}\) and a unique cospan
    \(\cospan{\listvar{i}}{L}{\listvar{j}}\) such that \(G\) can be factored as
    the following composite of monogamous acyclic cospans:
    \[
        (\cospan{\listvar{m}}{C_1}{\listvar{k} + \listvar{i}})
        \seq
        (\cospan{
            \listvar{k} + \listvar{i}
        }{
            \listvar{k} + L
        }{
            \listvar{k} + \listvar{i}
        })
        \seq
        (\cospan{\listvar{k} + \listvar{j}}{C_2}{\listvar{n}})
    \]
\end{lemma}

Essentially, we can always `pull out' a convex subgraph of a monogamous acyclic
cospan in such a way that the remaining cospans are still monogamous acyclic.
This is an important part of characterising the image of \(\termtohypsigmac\).

\begin{theorem}[\cite{bonchi2022stringa}, Thm. 25]\label{thm:monogamous-acyclic-full}
    A cospan \(\cospan{\listvar{m}}{F}{\listvar{n}}\) is in the image of
    \(\termtohypsigmac\) if and only if \(\cospan{\listvar{m}}{F}{\listvar{n}}\)
    is monogamous acyclic.
\end{theorem}
\begin{proof}
    The only if direction is by induction on the structure of terms in
    \(\smcsigmac\): the interpretation of generators is monogamous acyclic and
    the inductive cases are by
    \cref{lem:identities-symmetries-monogamous,lem:identities-symmetries-monogamous-acyclic,lem:monogamous-acyclicity-preserved-composition,lem:monogamicity-preserved-tensor,lem:monogamous-acyclicity-preserved-tensor}.

    The if direction is by induction on the number of edges in \(F\).
    If there are none, then \(\listvar{m} \to F\) and \(\listvar{n}\) are
    bijections by monogamy and acyclicity so the term is in the image of
    identities or symmetries in \(\smcsigmac\).
    For the inductive step, pick a single edge \(e\).
    This is a convex subgraph of \(F\), so
    \(\cospan{\listvar{m}}{F}{\listvar{n}}\) can be factored as in
    \cref{lem:decomposition}.
    The edge \(e\) has a label \(\chi(e) \in \generators\), so the subgraph
    \(\cospan{\listvar{i}}{e}{\listvar{j}}\) is the result of
    \(\termtohypsigmac[\chi(e)]\).
    Since the remaining cospans are monogamous acyclic by
    \cref{lem:decomposition}, they are in the image of \(\termtohypsigmac\) by
    inductive hypothesis, so the original cospan
    \(\cospan{\listvar{m}}{F}{\listvar{n}}\) is also in the image of
    \(\termtohypsigmac\).
\end{proof}

This shows that \(\termtohypsigmac\) is full; to conclude the isomorphism we
need to show that it is also faithful.
We know that the copairing \(\termtohypsigmac + \frobtohypsigmac\) is faithful
by \cref{thm:termtohyp-image}, so we just need to show the same is true for its
components.

\begin{definition}[\cite{macdonald2009amalgamations}, Defs. 3.1, 3.2]
    A functor \(\morph{F}{\mcc}{\mcd}\) satisfies the \emph{3-for-2 property}
    if, for each triple of morphisms \(f,g,h \in \mcd\) such that
    \(h = g \circ f\), if any two of \(f\), \(g\) and \(h\) are in the image of
    \(F\), then the third is also in the image of \(F\).
\end{definition}

\begin{theorem}[\cite{macdonald2009amalgamations}, Thm. 3.3]
    \label{thm:faithful-pushout}
    Let \(\morph{F_A}{\mcc}{\mca}\) and \(\morph{F_B}{\mcc}{\mcb}\) be faithful
    functors such that the following diagram is a pushout.
    \begin{center}
        \includestandalone{figures/category/amalgamation}
    \end{center}
    Then, if \(F_A\) and \(F_B\) both satisfy the 3-for-2 property, then the
    functors \(G_A\) and \(G_B\) are also faithful.
\end{theorem}

To apply this result, we need to show that \(\hypcsigmac\) is a pushout.
Recall that in a category with an initial object, every coproduct can be
expressed as a pushout by \cref{lem:coproduct-is-pushout}.

\begin{definition}
    Let \(\permspropwithnat\) be the sub-PROP of \(\finsetpropwithnat\)
    containing only the bijective functions.
\end{definition}

A morphism \(\listvar{m} \to \listvar{m}\) in \(\permspropwithnat\) is a
permutation of \(C\)-coloured wires.

\begin{definition}
    Let \(\cpropc\) be the full subcategory of \(\cprop\) containing only the
    \(C\)-coloured PROPs.
\end{definition}

\begin{lemma}\label{lem:cpropc-initial}
    \(\hat{\permsprop} \slice C\) is the initial object in \(\cpropc\).
\end{lemma}
\begin{proof}
    All the morphisms in \(\hat{\permsprop} \slice C\) are just (coloured)
    identities and symmetries; the unique coloured PROP morphism to any other
    \(C\)-coloured PROP maps these to the corresponding constructs.
\end{proof}

Subsequently, \(\smcsigmac + \frobc\) can be expressed as a pushout and the
`3-for-2' condition applied to show the faithfulness of \(\termtohypsigmac\).

\begin{proposition}
    \(\morph{\termtohypsigmac}{\smcsigmac}{\cspdchyp}\) is faithful.
\end{proposition}
\begin{proof}
    From \cref{thm:isomorphism-smcfrob-cospans}, we know that
    \(\cspdchyp \cong \smcsigmac + \frobc\).
    Both \(\smcsigmac\) and \(\frobc\) are objects of \(\cpropc\), which has
    \(\permspropwithnat \slice C\) as its initial object by
    \cref{lem:cpropc-initial}.
    As coproducts are pushouts from the initial object
    (\cref{lem:coproducts-pushout}), we can construct the following diagram in
    \(\cpropc\)
    \begin{center}
        \includestandalone{figures/graphs/coproduct-pushout-graphs}
    \end{center}
    where \(!_1\) and \(!_2\) are the unique morphisms from
    \(\permspropwithnat\) induced by initiality: these are both faithful.
    Now we can apply the result of \cref{thm:faithful-pushout}: \(!_1\) and
    \(!_2\) clearly satisfy the 3-for-2 condition as every morphism in
    \(\permspropwithnat\) is an isomorphism, so \(\termtohypsigmac\) must also
    be faithful.
\end{proof}

Since \(\termandfrobtohypsigmac\) is full and faithful, we have reached our
final destination.

\begin{corollary}[\cite{bonchi2022stringa}, Cor. 26]
    There is an isomorphism of PROPs \(\smcsigmac \cong \macspdchyp\).
\end{corollary}