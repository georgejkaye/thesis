\section{Symmetric monoidal terms}

We have now seen that that cospans of hypergraphs are an excellent fit for
reasoning about terms in a freely generated hypergraph category.
However, there are times we might not have so much structure in our terms;
indeed for our case of digital circuits we only operate in a setting with a
trace.
This means that not every cospan of hypergraphs will correspond to a valid term.
Fortunately, Bonchi et al also characterised the cospans of hypergraphs that
correspond to \emph{symmetric monoidal} terms without any additional structure.
We will use some of this machinery when it comes to tackling the traced case.

\subsection{Monogamous acyclic cospans}

There are two features that distinguish vanilla symmetric monoidal terms from
Frobenius terms; wires cannot arbitrarily fork or join, and cycles may not be
created.
The former is tackled by a condition on the connectivity of vertices.

\begin{definition}[Degree {\cite[Def.\ 12]{bonchi2022stringa}}]
    For a hypergraph \(F \in \hyp\), the \emph{degree} of a vertex
    \(v \in \vertices{F}\) is a tuple \((i,o)\) where \(i\) is the number of
    pairs \((e,i)\) where \(e\) is a hyperedge with \(v\) as its \(i\)th target,
    and \(o\) is similarly the number of pairs \((e,j)\) where \(e\) is a
    hyperedge with \(v\) as its \(j\)th target.
\end{definition}

\begin{definition}[Monogamy {\cite[Def.\ 13]{bonchi2022stringa}}]
    For a cospan \(\cospan{m}[f]{F}[g]{n}\) in
    \(\cspdhyp\), let \(\mathsf{in}(F)\) and \(\mathsf{out}(F)\) be the image of
    \(f\) and \(g\) respectively.
    We call the cospan \(\cospan{m}[f]{F}[g]{n}\) \emph{monogamous} if \(f\) and
    \(g\) are mono and, for all nodes \(v\), the degree of \(v\) is
    \begin{center}
        \begin{tabular}{rlcrl}
            \((0,0)\)
             &
            if \(v \in \mathsf{in}(F) \wedge v \in \mathsf{out}(F)\)
             &
            \quad
             &
            \((0,1)\)
             &
            if \(v \in \mathsf{in}(F)\)
            \\
            \((1,0)\)
             &
            if \(v \in \mathsf{out}(F)\)
             &
            \quad
             &
            \((1,1)\)
             &
            otherwise
        \end{tabular}
    \end{center}
\end{definition}

\begin{example}\label{ex:monogamous}
    The following cospans of hypergraphs are monogamous:
    \[
        \iltikzfig{graphs/monogamy/yes-0}
        \quad
        \iltikzfig{graphs/monogamy/yes-1}
    \]
    The following cospans of hypergraphs are not monogamous:
    \[
        \iltikzfig{graphs/monogamy/no-0}
        \quad
        \iltikzfig{graphs/monogamy/no-1}
    \]
\end{example}

Since our goal is to assemble monogamous cospans into a category, it is
necessary to check that the property is preserved by the various categorical
operations.

\begin{lemma}[\cite{bonchi2022stringa}, Lem.\ 15]\label{lem:identities-symmetries-monogamous}
    Identities and symmetries are monogamous.
\end{lemma}
\begin{proof}
    The cospans involved are discrete and all vertices are in both
    interfaces, so the cospans are monogamous.
\end{proof}

\begin{lemma}[\cite{bonchi2022stringa}, Lem.\ 16]\label{lem:monogamicity-preserved-composition}
    Monogamicity is preserved by composition.
\end{lemma}
\begin{proof}
    Assume we compose two monogamous acyclic cospans \(
    \cospan{m}[f]{F}[g]{n}
    \) and \(
    \cospan{n}[h]{G}[k]{p}
    \).
    The interfaces remain mono as pushouts along monos are monos in
    \(\hypsigmac\).
    The only altered vertices are those in the image of
    \(g\) and \(h\), which are merged pointwise; vertices in the image of
    \(g\) have out-degree \(0\) and those in the image of \(h\) have in-degree
    \(0\) so the merged vertices have at most degree \((1, 1)\).
\end{proof}

\begin{lemma}[\cite{bonchi2022stringa}, Lem.\ 17]\label{lem:monogamicity-preserved-tensor}
    Monogamicity is preserved by tensor.
\end{lemma}
\begin{proof}
    The degrees of nodes are unaffected as tensor is by coproduct and
    only vertices in the original interfaces will be in the new interfaces.
\end{proof}

As seen in \cref{ex:monogamous}, monogamy makes no guarantees about cycles.
Since symmetric monoidal terms cannot have cycles, a notion of \emph{acyclicity}
must also be enforced.

\begin{definition}[Predecessor {\cite[Def.\ 18]{bonchi2022stringa}}]
    A hyperedge \(e\) is a \emph{predecessor} of hyperedge \(e^\prime\)
    if there exists a node \(v\) in the sources of \(e\) and the targets of
    \(e^\prime\).
\end{definition}

\begin{definition}[Path {\cite[Def.\ 19]{bonchi2022stringa}}]
    A \emph{path} between two hyperedges \(e\) and \(e^\prime\) is a sequence of
    hyperedges \(e_0, \dots, e_{n-1}\) such that \(e = e_0\),
    \(e^\prime = e_{n-1}\), and for each \(i < n-1\), \(e_i\) is a predecessor
    of \(e_{i+1}\).
    A subgraph \(H\) is the \emph{start} or \emph{end} of a path if it contains
    a node in the sources of \(e\) or the targets of \(e^\prime\) respectively.
\end{definition}

Since nodes are single-element subgraphs, one can also talk about paths from
nodes.

\begin{definition}[Acyclicity {\cite[Def.\ 20]{bonchi2022stringa}}]
    A hypergraph \(F\) is acyclic if there is no path from a node to itself.
    A cospan \(\cospan{m}{F}{n}\) is acyclic if \(F\) is.
\end{definition}

\begin{example}\label{ex:acyclic}
    The following cospans of hypergraphs are acyclic:
    \[
        \iltikzfig{graphs/acyclic/yes-0}
        \quad
        \iltikzfig{graphs/acyclic/yes-1}
    \]
    The following cospans of hypergraphs are not acyclic:
    \[
        \iltikzfig{graphs/acyclic/no-0}
        \quad
        \iltikzfig{graphs/acyclic/no-1}
    \]
\end{example}

Once again, for acyclicity to be a suitable condition on a category of cospans,
it needs to be preserved by categorical operations.

\begin{lemma}[{\cite[Prop.\ 21]{bonchi2022stringa}}]\label{lem:identities-symmetries-monogamous-acyclic}
    Identities and symmetries are acyclic.
\end{lemma}
\begin{proof}
    By \cref{lem:identities-symmetries-monogamous} as discrete
    hypergraphs cannot contain cycles.
\end{proof}

\begin{lemma}[{\cite[Prop.\ 21]{bonchi2022stringa}}]\label{lem:monogamous-acyclicity-preserved-tensor}
    Acyclicity is preserved by tensor.
\end{lemma}
\begin{proof}
    The original graphs are not altered.
\end{proof}

When turning to composition, we run into a problem; composition of arbitrary
cospans may not preserve acyclicity.
It is only when acyclicity is combined with monogamy that composition can be
safely performed.

\begin{lemma}[{\cite[Prop.\ 21]{bonchi2022stringa}}]\label{lem:monogamous-acyclicity-preserved-composition}
    Monogamous acyclicity is preserved by composition.
\end{lemma}
\begin{proof}
    Assume we compose two monogamous acyclic cospans \(
    \cospan{m}[f]{F}[g]{n}
    \) and \(
    \cospan{n}[h]{G}[k]{p}
    \).
    A cycle cannot be created by composition because there cannot be a path in
    \(F\) that starts in the image of \(g\) or a path in \(G\) that ends in the
    image of \(h\), because these vertices have out-degree and in-degree \(0\)
    respectively.
\end{proof}

This shows that monogamous acyclic cospans of hypergraphs form a category.

\begin{definition}[\cite{bonchi2022stringa}]
    Let \(\macspdhyp\) be defined as the sub-PROP of \(\cspdhyp\) containing the
    monogamous acyclic cospans of hypergraphs.
\end{definition}

\begin{example}\label{ex:monogamous-acyclic}
    The following cospans of hypergraphs are monogamous acyclic:
    \[
        \iltikzfig{graphs/monogamous-acyclic/yes-0}
        \quad
        \iltikzfig{graphs/monogamous-acyclic/yes-1}
    \]
    The following cospans of hypergraphs are not monogamous acyclic:
    \[
        \iltikzfig{graphs/monogamous-acyclic/no-0}
        \quad
        \iltikzfig{graphs/monogamous-acyclic/no-1}
    \]
\end{example}

Just like how cospans of hypergraphs correspond to string diagrams of Frobenius
terms, monogamous acyclic terms correspond to string diagrams of symmetric
monoidal terms.
Bonchi et al showed this by proving that \(\smcsigma\) is isomorphic to
\(\macspdhyp\); to do this they needed a few more ingredients.
The first is a lemma showing that a special class of subgraphs can always be
`extracted' from a parent graph.

\begin{definition}[Convex subgraph {\cite[Def.\ 23]{bonchi2022stringa}}]
    A subgraph \(G \subseteq F\) is convex if for any nodes \(v, v^\prime\) in
    \(G\) and any path \(p\) from \(v\) to \(v^\prime\), every edge \(e\) in
    \(p\) is also in \(G\).
\end{definition}

\begin{lemma}[Decomposition {\cite[Lem.\ 24]{bonchi2022stringa}}]
    \label{lem:decomposition}
    For a monogamous acyclic cospan \(\cospan{m}{F}{n}\) and
    and convex subgraph \(L\) of \(G\), there exist
    \(k \in \nat\) and a unique cospan
    \(\cospan{i}{L}{j}\) such that \(G\) can be factored as
    the following composite of monogamous acyclic cospans:
    \[
        (\cospan{m}{C_1}{k + i})
        \seq
        (\cospan{
            k + i
        }{
            k + L
        }{
            k + i
        })
        \seq
        (\cospan{k + j}{C_2}{n})
    \]
\end{lemma}

Essentially, we can always `pull out' a convex subgraph of a monogamous acyclic
cospan in such a way that the remaining cospans are still monogamous acyclic.
This is an important part of characterising the image of \(\termtohypsigma\).

\begin{theorem}[\cite{bonchi2022stringa}, Thm.\ 25]\label{thm:monogamous-acyclic-full}
    A cospan \(\cospan{m}{F}{n}\) is in the image of
    \(\termtohypsigma\) if and only if \(\cospan{m}{F}{n}\)
    is monogamous acyclic.
\end{theorem}
\begin{proof}
    The \(\onlyifdir\) direction is by induction on the structure of terms in
    \(\smcsigma\): the interpretation of generators is monogamous acyclic and
    the inductive cases are by
    \cref{lem:identities-symmetries-monogamous,lem:identities-symmetries-monogamous-acyclic,lem:monogamous-acyclicity-preserved-composition,lem:monogamicity-preserved-tensor,lem:monogamous-acyclicity-preserved-tensor}.

    The \(\ifdir\) direction is by induction on the number of edges in \(F\).
    If there are none, then \(m \to F\) and \(n \to F\) are
    bijections by monogamy so the term is in the image of
    identities or symmetries in \(\smcsigma\).
    For the inductive step, pick a single edge \(e\).
    This is a convex subgraph of \(F\), so
    \(\cospan{m}{F}{n}\) can be factored as in
    \cref{lem:decomposition}.
    The edge \(e\) has a label \(\chi(e) \in \generators\), so the subgraph
    \(\cospan{i}{e}{j}\) is the result of
    \(\termtohypsigma[\chi(e)]\).
    Since the remaining cospans are monogamous acyclic by
    \cref{lem:decomposition}, they are in the image of \(\termtohypsigma\) by
    inductive hypothesis, so the original cospan
    \(\cospan{m}{F}{n}\) is also in the image of
    \(\termtohypsigma\).
\end{proof}

This shows that \(\termtohypsigma\) is full; to conclude the isomorphism we
need to show that it is also faithful.
We know that the copairing \(\termtohypsigma + \frobtohypsigma\) is faithful
by \cref{thm:isomorphism-smcfrob-cospans}, so we just need to show the same is
true for its components, using a result about pushouts in \(\propcat\).

\begin{definition}[\cite{macdonald2009amalgamations}, Defs. 3.1, 3.2]
    A functor \(\morph{F}{\mcc}{\mcd}\) satisfies the \emph{3-for-2 property}
    if, for each triple of morphisms \(f,g,h \in \mcd\) such that
    \(h = g \circ f\), if any two of \(f\), \(g\) and \(h\) are in the image of
    \(F\), then the third is also in the image of \(F\).
\end{definition}

\begin{theorem}[\cite{macdonald2009amalgamations}, Thm.\ 3.3]
    \label{thm:faithful-pushout}
    Let \(\morph{F_A}{\mcc}{\mca}\) and \(\morph{F_B}{\mcc}{\mcb}\) be faithful
    functors such that the following diagram is a pushout.
    \begin{center}
        \includestandalone{figures/category/amalgamation}
    \end{center}
    Then, if \(F_A\) and \(F_B\) both satisfy the 3-for-2 property, then the
    functors \(G_A\) and \(G_B\) are also faithful.
\end{theorem}

To apply this result, we need to show that \(\hypsigma\) is a pushout.

\begin{definition}
    Let \(\permsprop\) be the sub-PROP of \(\finsetprop\)
    containing the bijective functions.
\end{definition}

A morphism \(m \to m\) in \(\permsprop\) is a
permutation of wires.
As all the functions are bijections, there can be morphisms of no other form.

\begin{lemma}\label{lem:perms-initial}
    \(\permsprop\) is the initial object in \(\propcat\).
\end{lemma}
\begin{proof}
    All the morphisms in \(\permsprop\) are identities and symmetries; the
    coloured PROP morphism to any other PROP maps these to the corresponding
    constructs.
\end{proof}

Subsequently, \(\smcsigma + \frob\) can be expressed as a pushout and the
`3-for-2' condition applied to show the faithfulness of \(\termtohypsigma\),
using another well-known categorical lemma.

\begin{lemma}[\cite{borceux1994handbook}, Prop.\ 2.8.2]\label{lem:coproducts-pushout}
    If a category \(\mcc\) has pushouts and an initial object, then \(\mcc\)
    also has coproducts.
\end{lemma}
\begin{proof}
    Given objects \(A,B \in \mcc\), the coproduct \(A + B\) is constructed as
    follows:
    \begin{gather*}
        \includestandalone{figures/category/diagrams/coproduct-pushout}%
    \end{gather*}
    This is a coproduct due to the universal property of pushouts.
\end{proof}

\begin{proposition}
    \(\morph{\termtohypsigma}{\smcsigma}{\cspdhyp}\) is faithful.
\end{proposition}
\begin{proof}
    From \cref{thm:isomorphism-smcfrob-cospans}, we know that
    \(\cspdhyp \cong \smcsigma + \frob\).
    Both \(\smcsigma\) and \(\frob\) are objects of \(\propcat\), which has
    \(\permsprop\) as its initial object by \cref{lem:perms-initial}.
    As coproducts are pushouts from the initial object
    (\cref{lem:coproducts-pushout}), we can construct the following diagram in
    \(\propcat\):
    \begin{center}
        \includestandalone{figures/graphs/coproduct-pushout-graphs}
    \end{center}
    where \(!_1\) and \(!_2\) are the unique morphisms from
    \(\permsprop\) induced by initiality: these are both faithful.
    \(!_1\) and \(!_2\) clearly satisfy the 3-for-2 condition as every morphism
    in \(\permsprop\) is an isomorphism, so \(\termtohypsigma\) must also
    be faithful by \cref{thm:faithful-pushout}.
\end{proof}

Since \(\termandfrobtohypsigma\) is full and faithful, we have reached our
final destination.

\begin{corollary}[\cite{bonchi2022stringa}, Cor.\ 26]
    There is an isomorphism of PROPs \(\smcsigma \cong \macspdhyp\).
\end{corollary}

\subsection{Coloured symmetric monoidal terms}

As with the Frobenius case, it is a simple exercise to generalise the above
results to the countably coloured case.
Monogamy and acyclicity translate simply to coloured cospans, so the only
modification is we need to apply the 3-for-2 condition in the category of
\(C\)-coloured PROPs.

\begin{definition}
    For a countable set of colours \(C\), let \(\cpropc\) be the category of
    \(C\)-coloured PROPs and morphisms between them.
\end{definition}

\begin{lemma}
    Let \(\permspropwithnat\) be the sub-PROP of \(\finsetpropwithnat\)
    containing only the bijective functions.
\end{lemma}

\begin{lemma}
    For a countable set of colours \(C\), \(\permspropwithnat\) is the initial
    object in \(\cpropc\).
\end{lemma}

This means that we obtain correspondence results in the coloured case.

\begin{proposition}
    \(\morph{\termandfrobtohypsigmac}{\smcsigmac}{\cspdchyp}\) is faithful.
\end{proposition}

\begin{corollary}
    For a countable set of colours \(C\), there is an isomorphism of
    \(C\)-coloured PROPs \(\smcsigmac \cong \macspdchyp\).
\end{corollary}