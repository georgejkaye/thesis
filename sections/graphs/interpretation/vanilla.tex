\section{Symmetric monoidal terms}

The results of the previous section show that cospans of hypergraphs are an
excellent fit for reasoning about terms in a freely generated hypergraph
category.
However, there are times we might not have so much structure in our terms;
indeed for our case of digital circuits we only operate in a setting with a
trace.
This means that not every cospan of hypergraphs will correspond to a valid term.
Fortunately, Bonchi et al also characterised the cospans of hypergraphs that
correspond to \emph{symmetric monoidal} terms without any additional structure.
We will use some of this machinery when it comes to tackling the traced case.

\subsection{Monogamy}

There are two features that distinguish vanilla symmetric monoidal terms from
Frobenius terms; wires cannot arbitrarily fork or join, and cycles may not be
created.
The former is tackled by a condition on the connectivity of vertices.

\begin{definition}[Degree (\cite{bonchi2022stringa}, Def. 12)]
    For a hypergraph \(F \in \hyp\), the \emph{degree} of a vertex
    \(v \in \vertices{F}\) is a tuple \((i,o)\) where \(i\) is the number of
    pairs \((e,i)\) where \(e\) is a hyperedge with \(v\) as its \(i\)th target,
    and \(o\) is similarly the number of pairs \((e,j)\) where \(e\) is a
    hyperedge with \(v\) as its \(j\)th target.
\end{definition}

\begin{definition}[Monogamy (\cite{bonchi2022stringa}, Def. 13)]
    A cospan \(\cospan{\listvar{m}}[f]{F}[g]{\listvar{n}}\) in \(\cspdchyp\) is
    \emph{monogamous} if \(f\) and \(g\) are mono and, for all nodes
    \(v \in \vertices{F}\), the degree of \(v\) is
    \begin{center}
        \begin{tabular}{rlcrl}
            \((0,0)\)
            &
            if \(v \in f \wedge v, v \in g_\star\)
            &
            \quad
            &
            \((0,1)\)
            &
            if \(v \in f_\star\)
            \\
            \((1,0)\)
            &
            if \(v \in g_\star\)
            &
            \quad
            &
            \((1,1)\)
            &
            otherwise
        \end{tabular}
    \end{center}
\end{definition}

\begin{lemma}[\cite{bonchi2022stringa}, Lems. 15-17]
    \label{lem:monogamicity-preserved}
    The following statements hold:
    \begin{enumerate}
        \item identities and symmetries are monogamous;
        \item monogamicity is preserved by composition; and
        \item monogamicity is preserved by tesnor.
    \end{enumerate}
\end{lemma}
\begin{proof}
    For (1), the cospans involved are discrete and all vertices are in both
    interfacs.
    For (2), assume we compose two monogamous acyclic cospans \(
        \cospan{\listvar{m}}[f]{F}[g]{\listvar{n}}
    \) and \(
        \cospan{\listvar{n}}[h]{G}[k]{\listvar{p}}
    \).

    The interfaces remain mono as pushouts along monos are monos in
    \(\hypsigmac\). The only vertices that are altered are those in the image of
    \(g\) and \(h\), which are merged pointwise; since vertices in the image of
    \(g\) have out-degree \(0\) and those in the image of \(h\) have in-degree
    \(0\), the merged vertices will have at most degree \((1, 1)\).

    For (3), the degrees of nodes are unaffected as tensor is by coproduct and
    only vertices in the original interfaces will be in the new interfaces.
\end{proof}

\subsection{Acyclicity}

Preventing cycles is a much more natural condition on hypergraphs.

\begin{definition}[Predecessor (\cite{bonchi2022stringa}, Def. 18)]
    A hyperedge \(e\) is a \emph{predecessor} of another hyperedge \(e^\prime\)
    if there exists a node \(v\) in the sources of \(e\) and the targets of
    \(e^\prime\).
\end{definition}

\begin{definition}[Path (\cite{bonchi2022stringa}, Def. 19)]
    A \emph{path} between two hyperedges \(e\) and \(e^\prime\) is a sequence of
    hyperedges \(e_0, \dots, e_{n-1}\) such that \(e = e_0\),
    \(e^\prime = e_{n-1}\), and for each \(i < n-1\), \(e_i\) is a predecessor
    of \(e_{i+1}\).
    A subgraph \(H\) is the \emph{start} or \emph{end} of a path if it contains
    a node in the sources of \(e\) or the targets of \(e^\prime\) respectively.
\end{definition}

Since nodes are single-element subgraphs, one can also talk about paths from
nodes.

\begin{definition}[Acyclicity (\cite{bonchi2022stringa}, Def. 20)]
    A hypergraph \(F\) is acyclic if there is no path from a node to itself.
    A cospan \(\morph{\listvar{m}}{F}{\listvar{n}}\) is acyclic if \(F\) is.
\end{definition}

\begin{lemma}[\cite{bonchi2022stringa}, Lems. 15-17, Prop. 21]
    \label{lem:monogamous-acyclic-preserved}
    The following statements hold:
    \begin{itemize}
        \item identities and symmetries are monogamous acyclic;
        \item monogamous acyclicity is preserved by composition; and
        \item monogamous acyclicity is preserved by tensor.
    \end{itemize}
\end{lemma}
\begin{proof}
    (1) and (3) follow by \cref{lem:monogamicity-preserved}, as cycles clearly
    cannot be created by the coproduct.

    For (2), once again assume we compose two monogamous acyclic cospans \(
        \cospan{\listvar{m}}[f]{F}[g]{\listvar{n}}
    \) and \(
        \cospan{\listvar{n}}[h]{G}[k]{\listvar{p}}
    \).
    A cycle cannot be created by composition because there cannot be a path in
    \(F\) that starts in the image of \(g\) or a path in \(G\) that ends in the
    image of \(h\), because these vertices have out-degree and in-degree \(0\)
    respectively.
\end{proof}

This shows that monogamous acyclic cospans of hypergraphs form a category.

\begin{definition}
    Let \(\macspdchyp\) be the sub-PROP of \(\cspdchyp\) containing only the
    monogamous acyclic cospans of hypergraphs.
\end{definition}

\todo[inline]{Do we talk about coproducts as pushouts earlier?}

\begin{definition}
    \todo[inline]{3-for-2 condition}
\end{definition}

\begin{lemma}
    \todo[inline]{Faithful if 3-for-2 condition}
\end{lemma}

\begin{corollary}
    \tood[inline]{Map from S sigma is faithful}
\end{corollary}

\begin{corollary}[\cite{bonchi2022stringa}, Cor. 26]
    There is an isomorphism of PROPs \(\smcsigmac \cong \macspdchyp\).
\end{corollary}