\section{String diagram rewriting}

Graph rewriting specialised for rewriting sting diagrams is a
relatively new field.
One of the first approaches was developed at the turn of the 2010s using
\emph{string graphs}~\cite{%
    dixon2010open,dixon2013opengraphs,kissinger2012pictures%
},\index{string graphs}
an example of which is illustrated in \cref{fig:string-graph}.
String graphs have two classes of vertices for \emph{boxes} and \emph{wires}; the
former vertices represent generators in string diagrams and the latter vertices
represent the wires between them.
One nuance of string graphs is that a wire in a string diagram can be
represented by arbitrarily many wire vertices connected together; all of these
different depictions are identified by a notion of \emph{wire homeomorphism}
\index{wire homeomorphsm}, in which adjacent wire vertices can be collapsed into
one.

\begin{figure}
    \centering
    \iltikzfig{graphs/examples/string-graph}
    \caption{Example of an interfaced string graph}
    \label{fig:string-graph}
\end{figure}

String graphs modulo wire homeomorphism are a suitable setting for modelling
traced or compact closed categories, but their main drawback is that a given
term may correspond to many different graphs thanks to wire homeomorphism.

More recently, there has been a flurry of work on string
diagram rewriting modulo \emph{Frobenius structure}\index{Frobenius} using
\emph{hypergraphs}~\cite{%
    bonchi2016rewriting,zanasi2017rewriting,bonchi2017confluence,%
    bonchi2018rewriting,bonchi2022string,bonchi2022stringa,bonchi2022stringb%
}, such as that in \cref{fig:hypergraph-intro}.\index{hypergraph}
Hypergraphs are a generalisation of graphs in which edges can have arbitrarily
many sources and targets, rather than just one each.
When interpreting string diagrams as hypergraphs, generators are represented as
hyperedges and connections between generators indicated by shared source or
target vertices.
As there is no restriction for vertices to only be incident on a single source and
target, one can model structures such as monoids or comonoids.

\begin{figure}
    \centering
    \iltikzfig{graphs/examples/hypergraph}
    \caption{Example of an interfaced hypergraph}
    \label{fig:hypergraph-intro}
\end{figure}

While string diagrams `absorb' the equations of SMCs, hypergraphs go one further
and absorb the equations of
a \emph{special commutative Frobenius algebra}: string diagrams
equal by Frobenius equations are interpreted as isomorphic hypergraphs.
This means rewriting using hypergraphs can be even more
advantageous than using string diagrams.

Naturally, there have also been variations on this work where the complete
Frobenius structure is not present.
Suitable restrictions on hypergraphs and the graph rewriting process are also
identified in~\cite{bonchi2016rewriting} for rewriting
\emph{symmetric monoidal structure}.
Research followed on rewriting modulo
\emph{(co)monoid structure}~\cite{milosavljevic2023string} (`half a Frobenius')
and our work~\cite{ghica2023rewriting} on rewriting modulo
\emph{traced comonoid structure}.
The latter is the basis for this part of the thesis.