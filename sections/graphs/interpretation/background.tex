\section{String diagram rewriting}

Graph rewriting specialised specifically for rewriting sting diagrams is a
relatively new field, first appearing at the turn of the 2010s using
\emph{string graphs}~\cite{%
    dixon2010open,dixon2013opengraphs,kissinger2012pictures%
}.
String graphs have two classes of nodes for \emph{boxes} and \emph{wires}; the
former nodes represent generators in string diagrams and the latter nodes
represent the wires between them.
Crucially, one wire in a string diagram can be represented by arbitrarily
many wire nodes connected together; all of these different depictions are
identified by a notion of \emph{wire homeomorphism}, in which adjacent wire
nodes can be collapsed into one.

\begin{center}
    \iltikzfig{graphs/examples/string-graph}
\end{center}

String graphs modulo wire homeomorphism is a suitable setting for modelling
traced or compact closed categories.
The main drawback with this presentation is that a given term may correspond to
many different graphs thanks to wire homeomorphism.

More recently, there has recently been a flurry of work on string
diagram rewriting modulo \emph{Frobenius structure} using
\emph{hypergraphs}~\cite{%
    bonchi2016rewriting,zanasi2017rewriting,bonchi2017confluence,%
    bonchi2018rewriting,bonchi2022string,bonchi2022stringa,bonchi2022stringb%
}.
Hypergraphs are a generalisation of graphs in which edges can have arbitrarily
many sources and targets, rather than just one each.
With hypergraphs, generators are represented as hyperedges, and connections
between generators indicated by if their sources and targets overlap.
The beauty of the hypergraph formalism is that there is no restriction for
nodes to only be incident on a single source and target, so one can easily
model structures such as monoids or comonoids.

\begin{center}
    \iltikzfig{graphs/examples/hypergraph}
\end{center}

It turns out when modelling string diagrams as hypergraphs, the equations of
a \emph{special commutative Frobenius algebra} are `absorbed': string diagrams
equal by Frobenius are interpreted as isomorphic hypergraphs.
This means that, in some sense, rewriting using hypergraphs can be even more
advantageous than using string diagrams.

Naturally, there have also been variations on this work where the complete
Frobenius structure is not present.
Suitable restrictions on hypergraphs and the graph rewriting process are also
identified in~\cite{bonchi2016rewriting} for rewriting
\emph{symmetric monoidal structure}.
Research followed on rewriting modulo
\emph{(co)monoid structure}~\cite{milosavljevic2023string} (`half a Frobenius')
and our work~\cite{ghica2023rewriting} on rewriting modulo
\emph{traced comonoid structure}.
The latter is the basis for this part of the thesis.