\section{Hypergraphs}

We will begin by defining categories of hypergraphs, following the
pattern outlined in \cite{bonchi2022string}.
Hypergraphs are formally defined as a functor category.

\begin{definition}[Hypergraph~\cite{bonchi2016rewriting}]\label{def:hypergraph}
    Let \(\mathbf{X}\) be the category with object set
    \((\nat \times \nat) + \star\) and morphisms
    \(\morph{\sources{i}}{(k,l)}{\star}\) for each \(i < k\)
    and \(\morph{\targets{j}}{(k,l)}{\star}\) for each \(j < l\).
    The category of hypergraphs \(\hyp\) is the functor category
    \([\mathbf{X}, \set]\).
\end{definition}

One can think of the category \(\mathbf{X}\) as a `template' for the structure
of a hypergraph: the object \(\star\) represents the nodes and each object
\((k, l)\) represents hyperedges with \(k\) sources and \(l\) targets; each such
edge must pick \(k\) sources and \(l\) targets from \(\star\).

Objects in \(\hyp\) are functors that instantiate each object in \(\mathbf{X}\)
to a concrete set.
For a hypergraph \(F \in \hyp\) we write \(\vertices{F}\) for its
set of nodes and \(\edges{F}{k}{l}\) for the set of edges with \(k\) sources and
\(l\) targets.

\begin{example}\label{ex:hypergraph}
    Let a hypergraph \(F\) be defined as follows:
    \begin{gather*}
        \vertices{F} \coloneqq \{v_0,v_1,v_2,v_3,v_4,v_5\}
        \quad
        \edges{F}{2}{1} \coloneqq \{e_0\}
        \quad
        \edges{F}{1}{2} \coloneqq \{e_1\}
        \\
        \sources{0}(e_0) \coloneqq v_1
        \quad
        \sources{1}(e_0) \coloneqq v_0
        \quad
        \sources{0}(e_1) \coloneqq v_1
        \\
        \targets{0}(e_0) \coloneqq v_3
        \quad
        \targets{0}(e_1) \coloneqq v_4
        \quad
        \targets{1}(e_1) \coloneqq v_5
    \end{gather*}
    Much like with regular graphs, it is more intuitive to draw out hypergraphs
    rather than look at their combinatorial representation.
    We draw nodes as black dots, and hyperedges as `bubbles' with ordered tentacles
    on the left and right that connect to source and target nodes respectively.
    \[
        \iltikzfig{graphs/blank-example-2}
    \]
    Note that the vertices do not have any notion of ordering or directionality.
\end{example}

Since it is a functor category, the morphisms in \(\hyp\) are natural
transformations: structure-preserving maps between hypergraphs.

\begin{definition}[Hypergraph homomorphism]\label{def:hypergraph-homomorphism}
    For two hypergraphs \(F, G \in \hyp\), a \emph{hypergraph homomorphism}
    \(F \to G\) consists of functions
    \(\morph{\vertices{f}}{\vertices{F}}{\vertices{G}}\) and
    \(\morph{\edges{f}{k}{l}}{\edges{F}{k}{l}}{\edges{G}{k}{l}}\) such that the
    following diagrams commute:
    \begin{center}
    \begin{tikzcd}
        \edges{F}{k}{l}
        \arrow{r}{\edges{f}{k}{l}}
        \arrow{d}{\sources{i}}
        &
        \edges{G}{k}{l}
        \arrow{d}{\sources{i}}
        \\
        \vertices{F}
        \arrow{r}{\vertices{f}}
        &
        \vertices{G}
    \end{tikzcd}
    \qquad
    \begin{tikzcd}
        \edges{F}{k}{l}
        \arrow{r}{\edges{f}{k}{l}}
        \arrow{d}{\targets{i}}
        &
        \edges{G}{k}{l}
        \arrow{d}{\targets{i}}
        \\
        \vertices{F}
        \arrow{r}{\vertices{f}}
        &
        \vertices{G}
    \end{tikzcd}
\end{center}
\end{definition}

\begin{example}\label{ex:hypergraph-homomorphism}
    Consider the following hypergraph \(G\):
    \begin{gather*}
        \vertices{G} \coloneqq \{v_6,v_7,v_8\}
        \quad
        \edges{G}{2}{1} \coloneqq \{e_2\}
        \\
        \sources{0}(e_3) \coloneqq v_6
        \quad
        \sources{1}(e_3) \coloneqq v_7
        \quad
        \targets{0}(e_3) \coloneqq v_8
        \\
        \iltikzfig{graphs/blank-smaller}
    \end{gather*}
    Recall the hypergraph \(F\) from \cref{ex:hypergraph}.
    A homomorphism \(\morph{h}{G}{F}\) is a map from the vertices and edges
    of the former to those of the the latter preserving sources and targets;
    one possible homomorphism could be
    \begin{gather*}
        \vertices{h}(v_6) \coloneqq v_1
        \quad
        \vertices{h}(v_7) \coloneqq v_0
        \quad
        \vertices{h}(v_8) \coloneqq v_3
        \quad
        \edges{h}{2}{1}(e_2) \coloneqq e_0
    \end{gather*}
    The image of the homomorphism is illustrated below:
    \[
        \iltikzfig{graphs/blank-hom-2}
    \]
    Injective hypergraph homomorphisms are often known as \emph{embeddings}.
    However, there is no requirement for hypergraph homomorphisms to be
    injective.
    Conisder another hypergraph \(H\) defined as
    \begin{gather*}
        \vertices{H} \coloneqq \{v_9\}
        \quad
        \edges{H}{2}{1} \coloneqq \{e_4\}
        \\
        \sources{0}(e_4) = v_9
        \quad
        \sources{1}(e_4) = v_9
        \quad
        \targets{0}(e_4) = v_9
        \\
        \iltikzfig{graphs/blank-merge}
    \end{gather*}
    There is a non-injective homomorphism \(\morph{k}{G}{H}\) defined as
    follows:
    \begin{gather*}
        \vertices{h}(v_6) \coloneqq v_9
        \quad
        \vertices{h}(v_7) \coloneqq v_9
        \quad
        \vertices{h}(v_8) \coloneqq v_9
        \quad
        \edges{h}{2}{1}(e_3) \coloneqq e_4
    \end{gather*}
    Although the vertices of \(G\) are merged by \(h\), the sources and
    targets are preserved.
\end{example}

\subsection{Labelled hypergraphs}

The graphical notation for hypergraphs is particularly evocative of string
diagrams: generators
correspond to hyperedges and wires to the nodes between them.
However, the hyperedges are currently not \emph{labelled} with generator
symbols.
To do this, we must first translate the notion of signature to hypergraphs.

\begin{definition}[Hypergraph signature~\cite{bonchi2016rewriting}]
    For a set of generators \(\generators\) with arities and coarities as defined
    in \cref{def:generators}, the corresponding \emph{hypergraph signature}
    \(\hypsignature{\generators}\) is an object of \(\hyp\) defined as follows:
    \begin{gather*}
        \vertices{\hypsignature{\generators}}
        \coloneqq
        \{ v \}
        \quad
        \edges{\hypsignature{\Sigma}}{k}{l}
        \coloneqq
        \{ e_g \,|\, g \in \signature\}
        \\
        \sources{i}(e_g) \coloneqq v
        \quad
        \targets{j}(e_g) \coloneqq v
    \end{gather*}
\end{definition}

\begin{example}\label{ex:labelled-hypergraph-signature}
    Let \(\generators[m] \coloneqq \{\morph{\phi}{2}{1}, \morph{\psi}{1}{2}\}\)
    be a monoidal signature.
    The corresponding hypergraph signature \(\hypsignature{\generators}\) is
    \begin{gather*}
        \vertices{\hypsignature{\generators}} \coloneqq \{v\}
        \quad
        \edges{\hypsignature{\generators}}{2}{1} \coloneqq \{e_\phi\}
        \quad
        \edges{\hypsignature{\generators}}{1}{2} \coloneqq \{e_\psi\}
        \\
        \sources{0}(e_\phi) \coloneqq v
        \quad
        \sources{1}(e_\phi) \coloneqq v
        \quad
        \sources{0}(e_\psi) \coloneqq v
        \quad
        \targets{0}(e_\phi) \coloneqq v
        \quad
        \targets{0}(e_\psi) \coloneqq v
        \quad
        \targets{1}(e_\psi) \coloneqq v
    \end{gather*}
    and is drawn as follows, where we notate the vertices and edges with the
    appropriate colour and label to aid in clarity.
    \[
        \iltikzfig{graphs/signature-example}
    \]
\end{example}

A hypergraph \(F\) labelled with generators from \(\generators\) is defined
as a hypergraph homomorphism \(\morph{\Gamma}{F}{\hypsignature{\generators}}\);
an edge \(e \in \edges{F}{k}{l}\) is labelled with generator
\(\phi\) if \(\edges{\Gamma}{k}{l} = e_\phi\).
This means a category of hypergraphs \emph{labelled} over a set of generators
\(\generators\) is a category in which the objects are all hypergraph
homomorphisms to \(\hypsignature{\generators}\).
This is a well-studied categorical template, and it has a special name.

\begin{definition}[Slice category~\cite{lawvere1963functorial}]\label{def:slice-category}
    For a category \(\mcc\) and an object \(C \in \mcc\), the
    \emph{slice category} \(\mcc \slice C\) has objects the morphisms of
    \(\mcc\) with target \(C\) and morphisms
    \((\morph{f}{X}{C}) \to (\morph{g}{X^\prime}{C})\) the morphisms
    \(\morph{g}{X}{X^\prime} \in \mcc\) such that \(f^\prime\circ g = f\).
\end{definition}

Each object in a slice category \(\mcc \slice C\) is a morphism with \(C\) as
its target.
When we set \(C\) to some hypergraph signature, this is a perfect setting for
labelled hypergraphs.

\begin{definition}[Labelled hypergraphs~\cite{bonchi2016rewriting}]
    For a set of generators \(\generators\), the category of
    \emph{hypergraphs labelled over \(\generators\)} is defined as the slice
    category
    \(\hypsigma \coloneqq \hyp \slice \hypsignature{\Sigma}\).
\end{definition}

\begin{example}\label{ex:labelled-hypergraph}
    Recall the hypergraph \(F\) from \cref{ex:hypergraph}; one labelling
    \(\morph{\Gamma}{F}{\hypsignature{\generators[m]}}\) could be defined as
    \begin{gather*}
        \vertices{\Gamma}(-) \coloneqq v
        \quad
        \edges{\Gamma}{2}{1}(e_0) \coloneqq e_\phi
        \quad
        \edges{\Gamma}{1}{2}(e_1) \coloneqq e_\psi
    \end{gather*}
    We draw a labelled hypergraph as a regular hypergraph but with labelled
    edges.
    \[
        \iltikzfig{graphs/blank-example-labelled-2}
    \]
    Note that if there are multiple generators with the same arity and coarity
    in a signature, there may well be multiple valid labellings of a hypergraph.
\end{example}

\subsection{Coloured hypergraphs}

We may not always deal with monochromatic PROPs, but additionally have a
countably infinite set of \emph{colours} \(\mcc\).
Accordingly, hypergraph signatured can also be generalised to \emph{coloured}
hypergraph signatures.

\begin{definition}[Coloured hypergraph signature~\cite{bonchi2016rewriting}]\label{def:coloured-hypergraph-signature}
    For a set of generators \(\generators\) and colours \(\mcc\) as defined
    in \cref{def:generators}, the corresponding
    \emph{coloured hypergraph signature}
    \(\hypsignature{(\mcc,\Sigma)}\) in an object of \(\hyp\) defined as
    follows:
    \begin{gather*}
        \vertices{\hypsignature{(\mcc,\Sigma)}} \coloneqq \{ v_c \,|\, c \in \mcc\}
        \quad
        \edges{\hypsignature{(\mcc,\Sigma)}}{k}{l} \coloneqq \{ e_g \,|\, g \in \signature\}
        \\
        \sources{i}(e_g) \coloneqq v_{\dom[e_g](i)}
        \quad
        \targets{j}(e_g) \coloneqq v_{\cod[e_g](j)}
    \end{gather*}
\end{definition}

\begin{example}\label{ex:coloured-hypergraph-signature}
    Let \(\mcc \coloneqq \{\bullet,\redbullet\}\) be a set of colours and \(
    \generators[c]
    \coloneqq \{
    \morph{\phi}{\redbullet\bullet}{\redbullet},
    \morph{\psi}{\redbullet}{\bullet\bullet}
    \}
    \) be a monoidal signature; the coloured hypergraph signature
    \(\hypsignature{(\mcc,\generators[c])}\) is
    \begin{gather*}
        \vertices{\hypsignature{(\mcc,\generators[c])}} \coloneqq \{v_{\bullet},v_{\redbullet}\}
        \quad
        \edges{\hypsignature{(\mcc,\generators[c])}}{2}{1} \coloneqq \{e_\phi\}
        \quad
        \edges{\hypsignature{(\mcc,\generators[c])}}{1}{2} \coloneqq \{e_\psi\}
        \\
        \sources{0}(e_\phi) \coloneqq v_{\redbullet}
        \quad
        \sources{1}(e_\phi) \coloneqq v_{\bullet}
        \quad
        \sources{0}(e_\psi) \coloneqq v_{\redbullet}
        \quad
        \targets{0}(e_\phi) \coloneqq v_{\redbullet}
        \quad
        \targets{0}(e_\psi) \coloneqq v_{\bullet}
        \quad
        \targets{1}(e_\psi) \coloneqq v_{\redbullet}
    \end{gather*}
    and is drawn by labelling edges appropriately:
    \[
        \iltikzfig{graphs/signature-example-coloured}
    \]
\end{example}

\begin{definition}[Coloured hypergraphs~\cite{bonchi2016rewriting}]
    Let \(\hypsigmac\) be the category of \emph{coloured hypergraphs} over a set
    of colours \(\mcc\) and generators \(\generators\), defined as the slice
    category \(\hyp \slice \hypsignature{(\mcc,\generators)}\).
\end{definition}

\begin{example}\label{ex:coloured-hypergraph}
    Returning again to the hypergraph \(F\) in \cref{ex:hypergraph}, we can
    label it with colours and generators from \((\colours,\generators[c])\) with
    the hypergraph homomorphism
    \begin{gather*}
        \vertices{\Gamma}(v_0) \coloneqq v_{\redbullet}
        \quad
        \vertices{\Gamma}(v_1) \coloneqq v_{\bullet}
        \quad
        \vertices{\Gamma}(v_2) \coloneqq v_{\bullet}
        \\
        \vertices{\Gamma}(v_3) \coloneqq v_{\redbullet}
        \quad
        \vertices{\Gamma}(v_4) \coloneqq v_{\redbullet}
        \quad
        \vertices{\Gamma}(v_5) \coloneqq v_{\bullet}
        \\
        \edges{\Gamma}{2}{1}(e_0) \coloneqq e_{\phi}
        \quad
        \edges{\Gamma}{1}{2}(e_1) \coloneqq e_{\psi}
    \end{gather*}
    Coloured hypergraphs are drawn as labelled hypergraphs, but their vertices
    are additionally coloured.
    \[
        \iltikzfig{graphs/blank-example-labelled-coloured-2}
    \]
\end{example}