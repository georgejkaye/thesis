\section{Hypergraphs}

We will begin by defining categories of hypergraphs, following the
pattern outlined in \cite{bonchi2022string}.
Hypergraphs are formally defined as a functor category.

\begin{definition}[Hypergraph~\cite{bonchi2016rewriting}]\label{def:hypergraph}
    \index{hypergraph|textbf}
    \index{h@\(\hyp\)}
    \nomenclature{\(\hyp\)}{category of hypergraphs}
    Let \(\mathbf{X}\) be the category with object set
    \((\nat \times \nat) + \star\) and morphisms
    \(\morph{\sources{i}}{(k,l)}{\star}\) for each \(i < k\)
    and \(\morph{\targets{j}}{(k,l)}{\star}\) for each \(j < l\).
    The category of hypergraphs \(\hyp\) is the functor category
    \([\mathbf{X}, \set]\).
\end{definition}

One can think of the category \(\mathbf{X}\) as a `template' for the structure
of a hypergraph: the object \(\star\) represents the vertices and each object
\((k, l)\) represents hyperedges with \(k\) sources and \(l\) targets; each such
edge must pick \(k\) sources and \(l\) targets from \(\star\).
Objects in \(\hyp\) are functors that instantiate each object in \(\mathbf{X}\)
to a concrete set.
For a hypergraph \(F \in \hyp\) we write \(\vertices{F}\)
\index{vertex}\nomenclature{\(\vertices{F}\)}{vertices of a hypergraph \(F\)}
for its vertices and \(\edges{F}{k}{l}\)
\index{edge}\nomenclature{\(\edges{F}{k}{l}\)}{edges of a hypergraph \(F\) with \(k\) sources and \(l\) targets}
for its edges with \(k\) sources and \(l\) targets.

\begin{example}\label{ex:hypergraph}
    Let a hypergraph \(F\) be defined as follows:
    \begin{gather*}
        \vertices{F} \coloneqq \{v_0,v_1,v_2,v_3,v_4,v_5\}
        \quad
        \edges{F}{2}{1} \coloneqq \{e_0\}
        \quad
        \edges{F}{1}{2} \coloneqq \{e_1\}
        \\
        \sources{0}(e_0) \coloneqq v_1
        \quad
        \sources{1}(e_0) \coloneqq v_0
        \quad
        \sources{0}(e_1) \coloneqq v_1
        \\
        \targets{0}(e_0) \coloneqq v_3
        \quad
        \targets{0}(e_1) \coloneqq v_4
        \quad
        \targets{1}(e_1) \coloneqq v_5
    \end{gather*}
    Much like with regular graphs, it is more intuitive to draw out hypergraphs
    rather than look at their combinatorial representation.
    We draw vertices as black dots and hyperedges as `bubbles' with ordered tentacles
    on the left and right that connect to source and target vertices respectively,
    as illustrated in \cref{fig:hypergraph-ex}.
    Note that the vertices do not have any notion of ordering or directionality.
\end{example}

Since it is a functor category, the morphisms in \(\hyp\) are natural
transformations: structure-preserving maps between hypergraphs.

\begin{definition}[Hypergraph homomorphism]\label{def:hypergraph-homomorphism}
    \index{hypergraph!homomorphism}
    For two hypergraphs \(F, G \in \hyp\), a \emph{hypergraph homomorphism}
    \(\morph{f}{F}{G}\) is a pair of functions
    \(\morph{\vertices{f}}{\vertices{F}}{\vertices{G}}\) and
    \(\morph{\edges{f}{k}{l}}{\edges{F}{k}{l}}{\edges{G}{k}{l}}\) such that the
    following diagrams commute:
    \begin{center}
    \begin{tikzcd}
        \edges{F}{k}{l}
        \arrow{r}{\edges{f}{k}{l}}
        \arrow{d}{\sources{i}}
        &
        \edges{G}{k}{l}
        \arrow{d}{\sources{i}}
        \\
        \vertices{F}
        \arrow{r}{\vertices{f}}
        &
        \vertices{G}
    \end{tikzcd}
    \qquad
    \begin{tikzcd}
        \edges{F}{k}{l}
        \arrow{r}{\edges{f}{k}{l}}
        \arrow{d}{\targets{i}}
        &
        \edges{G}{k}{l}
        \arrow{d}{\targets{i}}
        \\
        \vertices{F}
        \arrow{r}{\vertices{f}}
        &
        \vertices{G}
    \end{tikzcd}
\end{center}
\end{definition}
%
\begin{example}\label{ex:hypergraph-homomorphism}
    Consider the following hypergraph \(G\):
    \begin{center}
        \vspace{-\parskip}
        \begin{minipage}{0.75\textwidth}
            \begin{gather*}
                \vertices{G} \coloneqq \{v_6,v_7,v_8\}
                \quad
                \edges{G}{2}{1} \coloneqq \{e_2\}
                \\
                \sources{0}(e_2) \coloneqq v_6
                \quad
                \sources{1}(e_2) \coloneqq v_7
                \quad
                \targets{0}(e_2) \coloneqq v_8
            \end{gather*}
        \end{minipage}
        \begin{minipage}{0.2\textwidth}
            \centering

            \vspace{1.5em}

            \iltikzfig{graphs/blank-smaller}
        \end{minipage}
    \end{center}

    Recall the hypergraph \(F\) from \cref{ex:hypergraph}.
    A homomorphism \(\morph{h}{G}{F}\) is a map from the vertices and edges
    of the former to those of the the latter preserving sources and targets;
    one possible homomorphism could be
    \begin{center}
        \begin{minipage}{0.75\textwidth}
            \begin{gather*}
                \vertices{h}(v_6) \coloneqq v_1
                \quad
                \vertices{h}(v_7) \coloneqq v_0
                \quad
                \vertices{h}(v_8) \coloneqq v_3
                \\
                \edges{h}{2}{1}(e_2) \coloneqq e_0
            \end{gather*}
        \end{minipage}
        \begin{minipage}{0.2\textwidth}
            \centering

            \vspace{1.5em}

            \iltikzfig{graphs/blank-hom-2}
        \end{minipage}
    \end{center}

    Injective hypergraph homomorphisms are often known as \emph{embeddings}.
    \index{hypergraph!embedding}
    However, there is no requirement for hypergraph homomorphisms to be
    injective.
    Consider another hypergraph \(H\) defined as
    \begin{center}
        \begin{minipage}{0.75\textwidth}
            \begin{gather*}
                \vertices{H} \coloneqq \{v_9\}
                \quad
                \edges{H}{2}{1} \coloneqq \{e_3\}
                \\
                \sources{0}(e_3) = v_9
                \quad
                \sources{1}(e_3) = v_9
                \quad
                \targets{0}(e_3) = v_9
            \end{gather*}
        \end{minipage}
        \begin{minipage}{0.2\textwidth}
            \centering

            \vspace{1.5em}

            \iltikzfig{graphs/blank-merge}
        \end{minipage}
    \end{center}
    There is a non-injective homomorphism \(\morph{k}{G}{H}\) defined as
    follows:
    \begin{gather*}
        \vertices{h}(v_6) \coloneqq v_9
        \quad
        \vertices{h}(v_7) \coloneqq v_9
        \quad
        \vertices{h}(v_8) \coloneqq v_9
        \quad
        \edges{h}{2}{1}(e_2) \coloneqq e_3
    \end{gather*}
    Although the vertices of \(G\) are merged by \(h\), the sources and
    targets are preserved.
\end{example}

\subsection{Labelled hypergraphs}

\begin{figure}
    \centering
    \iltikzfig{graphs/blank-example-2}
    \qquad
    \iltikzfig{graphs/signature-example}
    \qquad
    \iltikzfig{graphs/blank-example-labelled-2}
    \caption{Illustration of the hypergraph from \cref{ex:hypergraph}, the
        hypergraph signature from \cref{ex:labelled-hypergraph-signature}, and the
        labelling of the former with the latter from
        \cref{ex:labelled-hypergraph}
    }
    \label{fig:hypergraph-ex}
\end{figure}

The graphical notation for hypergraphs is particularly evocative of string
diagrams: generators
correspond to hyperedges and wires to the vertices between them.
The only thing missing is that the hyperedges are not \emph{labelled} with
generator symbols.

\begin{definition}[Hypergraph signature~\cite{bonchi2016rewriting}]
    \index{hypergraph!signature}
    \nomenclature{\(\Sigma\)}{set of generators}
    \nomenclature{\(\hypsignature{\generators}\)}{hypergraph signature for a set of generators \(\Sigma\)}
    For a set of generators \(\generators\) with arities and coarities as defined
    in \cref{def:generators}, the corresponding \emph{hypergraph signature}
    \(\hypsignature{\generators}\) is an object of \(\hyp\) defined as follows:
    \begin{gather*}
        \vertices{\hypsignature{\generators}}
        \coloneqq
        \{ v \}
        \quad
        \edges{\hypsignature{\Sigma}}{k}{l}
        \coloneqq
        \{ e_g \,|\, g \in \signature\}
        \quad
        \sources{i}(e_g) \coloneqq v
        \quad
        \targets{j}(e_g) \coloneqq v
    \end{gather*}
\end{definition}

\begin{example}\label{ex:labelled-hypergraph-signature}
    Let \(\generators[m] \coloneqq \{\morph{\phi}{2}{1}, \morph{\psi}{1}{2}\}\)
    be a monoidal signature.
    The corresponding hypergraph signature \(\hypsignature{\generators}\) is
    \begin{gather*}
        \vertices{\hypsignature{\generators}} \coloneqq \{v\}
        \quad
        \edges{\hypsignature{\generators}}{2}{1} \coloneqq \{e_\phi\}
        \quad
        \edges{\hypsignature{\generators}}{1}{2} \coloneqq \{e_\psi\}
        \\
        \sources{0}(e_\phi) \coloneqq v
        \quad
        \sources{1}(e_\phi) \coloneqq v
        \quad
        \sources{0}(e_\psi) \coloneqq v
        \quad
        \targets{0}(e_\phi) \coloneqq v
        \quad
        \targets{0}(e_\psi) \coloneqq v
        \quad
        \targets{1}(e_\psi) \coloneqq v
    \end{gather*}
    and is drawn as in the middle of \cref{fig:hypergraph-ex}, where the edges
    are annotated with the appropriate label for clarity.
\end{example}

A hypergraph \(F\) labelled over \(\generators\) is a hypergraph homomorphism
\(\morph{\Gamma}{F}{\hypsignature{\generators}}\);
an edge \(e \in \edges{F}{k}{l}\) is labelled with generator
\(\phi\) if \(\edges{\Gamma}{k}{l}(e) = e_\phi\).
This means a category of \emph{labelled} hypergraphs is a category in which the
objects are hypergraph homomorphisms to \(\hypsignature{\generators}\).
This is a well-studied categorical template, and it has a special name.

\begin{definition}[Slice category~\cite{lawvere1963functorial}]\label{def:slice-category}
    \index{category!slice}
    \index{slice category}
    For a category \(\mcc\) and an object \(C \in \mcc\), the
    \emph{slice category} \(\mcc \slice C\) has as objects the morphisms of
    \(\mcc\) with target \(C\) and as morphisms
    \((\morph{f}{X}{C}) \to (\morph{g}{X^\prime}{C})\) the morphisms
    \(\morph{g}{X}{X^\prime} \in \mcc\) such that \(f^\prime\circ g = f\).
\end{definition}

Each object in a slice category \(\mcc \slice C\) is a morphism with \(C\) as
its target.
When we set \(C\) to some hypergraph signature, this is a perfect setting for
labelled hypergraphs.

\begin{definition}[Labelled hypergraphs~\cite{bonchi2016rewriting}]
    \index{hypergraph!labelled}
    \index{h@\(\hypsigma\)}
    \nomenclature{\(\hypsigma\)}{category of \(\generators\)-labelled hypergraphs}
    For a set of generators \(\generators\), the category of
    \emph{\(\Sigma\)-labelled hypergraphs} is the slice category
    \(\hypsigma \coloneqq \hyp \slice \hypsignature{\Sigma}\).
\end{definition}

\begin{example}\label{ex:labelled-hypergraph}
    One labelling of the hypergraph \(F\) from \cref{ex:hypergraph} could be
    defined using the homomorphism
    \(\morph{\Gamma}{F}{\hypsignature{\generators[m]}}\) with action
    \begin{gather*}
        \vertices{\Gamma}(-) \coloneqq v
        \quad
        \edges{\Gamma}{2}{1}(e_0) \coloneqq e_\phi
        \quad
        \edges{\Gamma}{1}{2}(e_1) \coloneqq e_\psi
    \end{gather*}
    We draw a labelled hypergraph as a regular hypergraph but with labelled
    edges, as shown in \cref{fig:hypergraph-ex}.
    If there are multiple generators with the same arity and coarity
    in a signature, there may well be multiple valid labellings of a hypergraph.
\end{example}

\subsection{Coloured hypergraphs}

We may additionally work over a countable set of \emph{colours}.
Accordingly, hypergraph signatures can be generalised to \emph{coloured}
hypergraph signatures.

\begin{definition}[Coloured hypergraph signature~\cite{bonchi2016rewriting}]\label{def:coloured-hypergraph-signature}
    \index{hypergraph!coloured signature}
    \index{coloured!hypergraph signature}
    For a countable set set \(C\) and a set of \(C\)-coloured generators
    \(\generators\) as defined in \cref{def:generators}, the
    \emph{coloured hypergraph signature}
    \(\hypsignature{(C,\Sigma)}\) is an object of \(\hyp\) defined as
    follows:
    \begin{gather*}
        \vertices{\hypsignature{(C,\Sigma)}} \coloneqq \{ v_c \,|\, c \in C\}
        \quad
        \edges{\hypsignature{(C,\Sigma)}}{k}{l} \coloneqq \{ e_g \,|\, g \in \signature\}
        \\
        \sources{i}(e_g) \coloneqq v_{\dom[e_g](i)}
        \quad
        \targets{j}(e_g) \coloneqq v_{\cod[e_g](j)}
    \end{gather*}
\end{definition}

\begin{example}\label{ex:coloured-hypergraph-signature}
    Let \(C \coloneqq \{\bullet,\redbullet\}\) be let \(
    \generators[c]
    \coloneqq \{
    \morph{\phi}{\redbullet\bullet}{\redbullet},
    \morph{\psi}{\redbullet}{\bullet\redbullet}
    \}
    \) be a monoidal signature; the coloured hypergraph signature
    \(\hypsignature{(\mcc,\generators[c])}\) is
    \begin{gather*}
        \vertices{\hypsignature{(C,\generators[c])}} \coloneqq \{v_{\bullet},v_{\redbullet}\}
        \quad
        \edges{\hypsignature{(C,\generators[c])}}{2}{1} \coloneqq \{e_\phi\}
        \quad
        \edges{\hypsignature{(C,\generators[c])}}{1}{2} \coloneqq \{e_\psi\}
        \\
        \sources{0}(e_\phi) \coloneqq v_{\redbullet}
        \quad
        \sources{1}(e_\phi) \coloneqq v_{\bullet}
        \quad
        \sources{0}(e_\psi) \coloneqq v_{\redbullet}
        \quad
        \targets{0}(e_\phi) \coloneqq v_{\redbullet}
        \quad
        \targets{0}(e_\psi) \coloneqq v_{\bullet}
        \quad
        \targets{1}(e_\psi) \coloneqq v_{\redbullet}
    \end{gather*}
    and is drawn by labelling edges appropriately as in
    \cref{fig:coloured-hypergraph-ex}.
\end{example}
%
\begin{figure}
    \centering
    \iltikzfig{graphs/signature-example-coloured}
    \qquad
    \iltikzfig{graphs/blank-example-labelled-coloured-2}
    \caption{
        Illustration of the coloured hypergraph signature from
        \cref{ex:coloured-hypergraph-signature} and the labelling of the hypergraph
        in \cref{ex:hypergraph} from \cref{ex:coloured-hypergraph}
    }
    \label{fig:coloured-hypergraph-ex}
\end{figure}
%
\begin{definition}[Coloured hypergraphs~\cite{bonchi2016rewriting}]
    \index{hypergraph!coloured}
    \index{coloured!hypergraph}
    \index{h@\(\hypsigmac\)}
    \nomenclature{\(\hypsigmac\)}{category of \((C,\generators)\)-labelled hypergraphs}
    For a countable set \(C\) and set of \(C\)-coloured generators \(\Sigma\),
    let \(\hypsigmac\) be the category
    of \emph{\((C,\generators)\)-labelled hypergraphs}, defined as the slice
    category \(\hyp \slice \hypsignature{(\mcc,\generators)}\).
\end{definition}
%
\begin{example}\label{ex:coloured-hypergraph}
    Returning again to the hypergraph \(F\) in \cref{ex:hypergraph}, we can
    label it with colours and generators from \((C,\generators[c])\) with
    the hypergraph homomorphism
    \begin{gather*}
        \vertices{\Gamma}(v_0) \coloneqq v_{\redbullet}
        \quad
        \vertices{\Gamma}(v_1) \coloneqq v_{\bullet}
        \quad
        \vertices{\Gamma}(v_2) \coloneqq v_{\redbullet}
        \\
        \vertices{\Gamma}(v_3) \coloneqq v_{\redbullet}
        \quad
        \vertices{\Gamma}(v_4) \coloneqq v_{\redbullet}
        \quad
        \vertices{\Gamma}(v_5) \coloneqq v_{\bullet}
        \\
        \edges{\Gamma}{2}{1}(e_0) \coloneqq e_{\phi}
        \quad
        \edges{\Gamma}{1}{2}(e_1) \coloneqq e_{\psi}
    \end{gather*}
    Coloured hypergraphs are drawn as labelled hypergraphs, but their vertices
    are additionally coloured, as shown in \cref{fig:coloured-hypergraph-ex}.
\end{example}