\section{Hypergraphs for traced comonoid terms}

\begin{figure}
    \centering
    \includestandalone{figures/graphs/roadmap}
    \caption{Interactions between categories of terms and hypergraphs}
    \label{fig:graph-roadmap}
\end{figure}

We are interested in another element of structure in addition to the trace: the
ability to \emph{fork} and \emph{eliminate} wires.
This is known as a \emph{(commutative) comonoid structure}: categories equipped
with such a structure are also known as \emph{gs-monoidal}
(\emph{garbage-sharing}) categories~\cite{fritz2023free}.

\begin{definition}
    Let \((\generators[\ccomon], \equations[\ccomon])\) be the symmetric
    monoidal theory of \emph{commutative comonoids}, with \(
        \Sigma_\ccomon \coloneqq \{
            \iltikzfig{strings/structure/comonoid/copy}[colour=white],
            \iltikzfig{strings/structure/comonoid/discard}[colour=white]
        \}
    \) and \(\mce_\ccomon\) defined as in \cref{fig:comonoid-equations}.
    We write \(
        \ccomon \coloneqq \smc{\generators[\ccomon], \equations[\ccomon]}
    \).
\end{definition}

\begin{figure}
    \begin{gather*}
        \equationdisplay{
            \iltikzfig{strings/structure/comonoid/unitality-l-lhs}
        }{
            \iltikzfig{strings/structure/comonoid/unitality-l-rhs}
        }{
            \comonoiduniteqnletter
        }
        \qquad
        \equationdisplay{
            \iltikzfig{strings/structure/comonoid/associativity-lhs}
        }{
            \iltikzfig{strings/structure/comonoid/associativity-rhs}
        }{
            \comonoidassoceqnletter
        }
        \qquad
        \equationdisplay{
            \iltikzfig{strings/structure/comonoid/associativity-lhs}
        }{
            \iltikzfig{strings/structure/comonoid/associativity-rhs}
        }{
            \comonoidcommeqnletter
        }
    \end{gather*}
    \caption{
        Equations \(\equations[\ccomon]\) of a \emph{commutative comonoid}
    }
    \label{fig:comonoid-equations}
\end{figure}

From now on, we write `comonoid' to mean `commutative comonoid'.
There has already been work using hypergraphs for PROPs with a (co)monoid
structure~\cite{fritz2023free,milosavljevic2023string} but these consider
\emph{acyclic} hypergraphs: we must ensure that removing the acyclicity
condition does not lead to any degeneracies.

\begin{definition}[Partial left-monogamy]
    For a cospan \(\cospan{m}[f]{H}[g]{n}\), we say it is
    \emph{partial left-monogamous} if \(f\) is mono and, for all nodes
    \(v \in H_\star\), the degree of \(v\) is \((0,m)\) if \(v \in f_\star\) and
    \((0,m)\) or \((1,m)\) otherwise, for some \(m \in \nat\).
\end{definition}

Partial left-monogamy is a weakening of partial monogamy that allows vertices
to have multiple `out' connections, which represent the use of the comonoid
structure to fork wires.

\begin{figure}
    \centering
    \[
        \underbrace{
            \iltikzfig{graphs/monogamy/yes-comonoid-0}
            \iltikzfig{graphs/monogamy/yes-comonoid-1}
        }_{\text{partial left-monogamous}}
        \qquad
        \underbrace{
            \iltikzfig{graphs/monogamy/no-comonoid-0}
            \iltikzfig{graphs/monogamy/no-comonoid-1}
        }_{\text{not partial left-monogamous}}
    \]
    \caption{Examples of cospans that are and are not partial left-monogamous}
    \label{fig:partial-left-monogamous-examples}
\end{figure}

\begin{example}
    Examples of cospans that are and are not partial left-monogamous are shown
    in \cref{fig:partial-left-monogamous-examples}.
\end{example}

\begin{remark}
    As with the vertices not in the interfaces with degree \((0, 0)\) in the
    vanilla traced case, the vertices not in the interface with degree
    \((0, m)\) allow for the interpretation of terms such as \(
        \trace{}{\iltikzfig{strings/structure/comonoid/copy}[colour=white]}
    \).
\end{remark}

\begin{lemma}\label{lem:partial-monogamous-id-sym}
    Identities and symmetries are partial left-monogamous.
\end{lemma}
\begin{proof}
    Again by \cref{lem:identities-symmetries-monogamous}, identities and
    symmetries are monogamous so they are also partial left-monogamous.
\end{proof}

\begin{lemma}
    Given partial left-monogamous cospans
    \(\cospan{m}{F}{n}\) and
    \(\cospan{n}{G}{p}\), the composition \(
        (\cospan{m}{F}{n})
        \seq
        (\cospan{n}{G}{p})
    \) is partial left-monogamous.
\end{lemma}
\begin{proof}
    We only need to check the in-degree of vertices, as the out-degree can be
    arbitrary.
    Only the vertices in the image of \(n \to G\) have their in-degree modified;
    they will gain the in-tentacles of the corresponding vertices in the image
    of \(n \to F\).
    Initially the vertices in \(n \to G\) must have in-degree \(0\) by partial
    monogamy.
    They can only gain at most one in-tentacle from vertices in \(n \to F\)
    as each of these vertices has in-degree \(0\) or \(1\) and \(n \to G\) is
    mono.
    So the composite graph is partial left-monogamous.
\end{proof}

\begin{lemma}
    Given partial left-monogamous cospans \(\cospan{m}{F}{n}\)
    and \(\cospan{p}{G}{q}\), the tensor \(
        (\cospan{m}{F}{n})
        \tensor
        (\cospan{n}{G}{p})
    \) is partial left-monogamous.
\end{lemma}
\begin{proof}
    The elements of the original graphs are unaffected so
    the degrees are unchanged.
\end{proof}

\begin{definition}
    Let \(\plmcspdchyp\) be the sub-PROP of \(\cspdchyp\) containing only the
    partial left-monogamous cospans of hypergraphs.
\end{definition}

\begin{proposition}
    The canonical trace is a trace on \(\plmcspdchyp\).
\end{proposition}
\begin{proof}
    We must show that for any set of vertices in the image
    of \(x + n \to K\) merged by the canonical trace, at most one of them can
    have in-degree \(1\).
    But this must be the case because any image in the image of
    \(x + m \to K\) must have in-degree \(0\), and \(x + m \to K\) is
    moreover mono so it cannot merge vertices in the image of
    \(x + n \to K\).
\end{proof}

This category can be equipped with a comonoid structure.

\begin{definition}
    Let \(
        \morph{
            \comonoidtofrob
        }{
            \ccomon
        }{
            \frob
        }
    \) be the obvious embedding of \(\ccomon\) into \(\frob\), and let \(
        \morph{
            \tracedandcomonoidtofrob{\Sigma}
        }{
            \stmc{\Sigma} + \comon
        }{
            \smc{\Sigma} + \frob
        }
    \) be the copairing of \(\tracedtosymandfrob{\Sigma}\) and
    \(\comonoidtofrob\).
\end{definition}

As before, these PROP morphisms are summarised in \cref{fig:roadmap}.
To show that partial left-monogamy is the correct notion to characterise terms
in a traced comonoid setting, it is necessary to ensure that the image of these
PROP morphisms lands in \(\plmcspdhyp\).

\begin{lemma}
    The image of \(\frobtohyp{\Sigma} \circ \comonoidtofrob\) is in
    \(\plmcspdhyp\).
\end{lemma}
\begin{proof}
    By definition.
\end{proof}

\begin{corollary}
    The image of \(
        \termandfrobtohypsigma \circ \tracedandcomonoidtofrob{\Sigma}
    \) is in \(\plmcspdhyp\).
\end{corollary}

To show the correspondence between \(\stmcsigma + \ccomon\) and
\(\plmcspdhyp\), we use a similar strategy to \cref{thm:termtohyp-image}.

\begin{lemma}\label{lem:discrete-mono}
    Given a discrete hypergraph \(X \in \hypsigma\), any cospan
    \(\cospan{m}[f]{X}{n}\) with \(f\) mono is in the image of
    \(\frobtohypsigma \circ \comonoidtofrob\).
\end{lemma}

\begin{theorem}\label{thm:comonoid-fully-complete}
    \(\stmcsigma + \ccomon \cong \plmcspdhyp\).
\end{theorem}
\begin{proof}
    Since \(\termandfrobtohypsigma\) and \(\comonoidtohypsigma\) are faithful,
    it suffices to show that a cospan \(\cospan{m}{F}{n}\) in
    \(\plmcspdhyp\) can be decomposed into a traced cospan in which every
    component under the trace is in the image of either
    \(\termandfrobtohypsigmac\) or \(\frobtohypsigmac \circ \comonoidtofrobc\).
    This is achieved by taking the construction of \cref{thm:termtohyp-image}
    and allowing the first component to be partial left-monogamous; by
    \cref{lem:discrete-mono} this is in the image of
    \(\frobtohypsigma \circ \comonoidtofrob\).
    The remaining components remain in the image of \(\termtohypsigma\).
    Subsequently, the entire traced cospan must be in the image of \(
        \termandfrobtohypsigma \circ [\tracedtosymandfrob, \comonoidtofrob].
    \).
\end{proof}
