\chapter{String diagrams as hypergraphs}

It is now time to establish the correspondence between cospans of hypergraphs
and string diagrams.
We will first recount the constructions used by Bonchi et al for the
Frobenius~\cite{bonchi2022string} and symmetric
monoidal~\cite{bonchi2022stringa} cases, before showing how we can use their
ingredients for a \emph{traced} setting either with or without a comonoid
structure.

\section{Frobenius terms}

As we shall soon see, it is actually simpler to start with the more complicated
Frobenius terms.
This is because \emph{every} Frobenius term corresponds to a cospan of
hypergraphs and vice versa.

\begin{definition}[\cite{bonchi2022string}]\label{def:hyp-morphisms}
    Let \(\morph{\termtohypsigmac}{\smcsigmas}{\cspdchyp}\) be a PROP
    morphism defined as
    \begin{gather*}
        \termtohyp[\iltikzfig{strings/category/generator}[box=\phi,colour=white,dom=m,cod=n]]{\Sigma}
        :=
        \cospan{m}{\iltikzfig{graphs/terms/generator}
        }{n}
        \\
        \termtohyp[\iltikzfig{strings/category/identity}[colour=white,obj=n]]{\Sigma}
        :=
        \cospan{n}[\id]{n}[\id]{n}
        \qquad
        \termtohyp[\iltikzfig{strings/symmetric/symmetry}[colour=white,obj1=m,obj2=n]]{\Sigma}
        :=
        \cospan{m+n}[[\id,\id]]{m+n}[[\id,\id]]{n+m}
    \end{gather*}
\end{definition}

\begin{proposition}[\cite{bonchi2022string}]
    \(\termtohypsigmac\) is faithful.
\end{proposition}

\subsection{Frobenius structure}

\begin{theorem}[\cite{bonchi2022string}, Thm. 3.8]
    \label{thm:cospan-homomorphism}
    Let \(\mathbb{X}\) be a PROP whose monoidal product is a coproduct,
    \(\mathbf{C}\) a category with pushouts and an initial object, and
    \(\morph{F}{\mathbb{X}}{\mathbf{C}}\) a coproduct-preserving functor.
    Then there is a homomorphism of PROPs \(
        \morph{\tilde{F}}{\csp{\mathbb{X}}}{\csp[F]{\mathbf{C}}}
    \) that sends \(\cospan{\listvar{m}}[f]{X}[g]{\listvar{{n}}}\) to
    \(\cospan{F\listvar{m}}[Ff]{FX}[Fg]{F\listvar{n}}\).
    If \(F\) is full and faithful, then \(\tilde{F}\) is faithful.
\end{theorem}

All that remains is to identify the exact functor to be used for
interpreting string diagrams as cospans of hypergraphs.

\begin{proposition}[\cite{lack2004composing}, Ex. 5.4]
    \label{prop:frob-finset}
    There is an isomorphism of PROPs \(\frob \cong \csp{\finsetprop}\).
\end{proposition}

We omit the formal proof and sketch the correspondence between the two
categories.
Terms in \(\frob\) are formed of all the ways of combining \(
    \iltikzfig{strings/structure/monoid/merge}[colour=white],
    \iltikzfig{strings/structure/monoid/init}[colour=white],
    \iltikzfig{strings/structure/comonoid/copy}[colour=white],
    \iltikzfig{strings/structure/comonoid/discard}[colour=white],
    \iltikzfig{strings/category/identity}[colour=white],
\) and \(
    \iltikzfig{strings/symmetric/symmetry}[colour=white]
\) in sequence and parallel, so a string diagram for a term \(\morph{f}{m}{n}\)
is depicted as \(x\) connected components drawing paths from \(m\) inputs to
\(n\) outputs, as illustrated with the example below.

\begin{center}
    \iltikzfig{strings/structure/frobenius/example}
\end{center}

Note there is no requirement for each component to connect to one or both
interfaces as the \(
    \iltikzfig{strings/structure/monoid/init}[colour=white]
\) and \(
    \iltikzfig{strings/structure/comonoid/discard}[colour=white]
\) generators can introduce and stub wires.
A term \(\morph{f}{m}{n}\) with \(x\) connected components corresponds to
a cospan of finite sets \(\cospan{[m]}[i]{[x]}[j]{[n]}\), where the functions
\(i\) and \(j\) map the inputs and outputs to the components they connect to.

\begin{example}
    Consider the term \(\morph{f}{5}{4}\) drawn on the left below.
    This corresponds to a cospan \(\cospan{[5]}{[3]}{[4]}\) as shown on the
    right below.
    \begin{center}
        \iltikzfig{strings/structure/frobenius/example}
        \(\Leftrightarrow\)
        \scalebox{0.75}{\tikzfig{strings/structure/frobenius/example-cospan}}
    \end{center}
\end{example}

The cospan representation shows how all connected Frobenius structures can be
`squished' into a single blob.

However, we are operating in a \emph{coloured} setting, in which wires may
be different \emph{colours}.
This is easily remedied by having a copy of \(\frob\) for each colour.

\begin{theorem}[\cite{baez2018props}, Corollary 5.3]
    \(\prop\) has coproducts.
\end{theorem}

This easily generalises to \(\cprop\) by replacing natural numbers with words.
This means that given coloured PROPs \(\mcc\) and \(\mcd\) with objects the
words in \(\freemon{C}\) and \(\freemon{D}\) respectively, there is also a
coloured PROP \(\mcc + \mcd\) with objects the words in \(\freemon{(C + D)}\)
and morphisms defined in the obvious way.
We can use this to define a multi-coloured version of \(\frob\) as
a coproduct of copies of \(\frob\) each representing a separate colour.

\begin{definition}[\cite{bonchi2022string}]
    \label{def:coloured-frob}
    For a countable set of colours \(C\), let \(\cfrob{C} \in \cprop\) be
    defined as \(\cfrob{C} := \sum_{c \in C}\frob\).
\end{definition}

As shown by \cref{prop:frob-finset}, to model terms in \(\frob\) combinatorially
we use the cospan category \(\csp{\finsetprop}\).
In \cite{bonchi2022string} this is extended for a finite set of colours \(C\) by
working in the slice category \(\finsetprop \slice C\).
Objects of this category are pairs \(([m], \morph{w}{[m]}{C})\); this pair can
be viewed as a word in \(\freemon{C}\) of length \(m\), with the \(i\)th letter
as \(w(i)\).

\begin{theorem}[\cite{bonchi2022string}, Theorem 2.24]
    \label{thm:frobc-iso-finset-slice-c}
    For a finite set of colours \(C \in \finsetprop\), there is an isomorphism
    of coloured PROPs \(\cfrob{C} \cong \csp{\finsetprop \slice C}\).
\end{theorem}
\begin{proof}
    By definition of \(\cfrob{C}\),
    \cref{def:coloured-frob,prop:frob-finset,lem:slice-iso-terminal}
    we have that \[
        \cfrob{C}
        :=
        \sum_{c \in C}\frob
        \cong
        \sum_{c \in C}\csp{\finsetprop}
        \cong
        \sum_{c \in C}\csp{\finsetprop \slice 1}
    \]
    In the other direction we have that \(
        \csp{\finsetprop \slice C}
        \cong
        \csp{\finsetprop \slice \sum_{c \in C} 1}
    \) as \(C\) is countable.
    So we need to show that \(
        \sum_{c \in C}\csp{\finsetprop \slice 1}
        \cong
        \csp{\finsetprop \slice \sum_{c \in C} 1}
    \).
    The objects of the former are coproducts of objects in
    \(\finsetprop \slice C\); as this is a coloured prop the coproduct is
    concatenation and subsequently the objects can be viewed as words in
    \(\freemon{C}\).
    Similarly, the objects of the latter are objects of
    \(\finsetprop \slice \sum_{c \in C} 1\), which can clearly also be
    seen as words in \(\freemon{C}\).

    The morphisms of the former are coproducts of cospans, which can
    equivalently be viewed as a single cospan with coproducts in the legs and
    apex; using the reasoning above this means it is a cospan of words in
    \(\freemon{C}\); it is easy to see that this is also the case for morphisms
    in the latter.
\end{proof}

We need to show a version of this for the case where \(C\) may be
\emph{countably infinite}.

\begin{lemma}
    \label{lem:slice-iso-terminal}
    In a category \(\mcc\) with a terminal object \(1\),
    \(\mcc \cong \mcc \slice 1\).
\end{lemma}
\begin{proof}
    Since \(1\) is terminal, there is a unique morphism \(A \to 1\) for each
    object \(A\) in \(\mcc\), so there is an object \((A, A \to 1)\) in
    \(\mcc \slice 1\) for each object in \(\mcc\).
    There is a morphism \((A, \morph{!_A}{A}{1}) \to (B, \morph{!_B}{B}{1})\) in
    \(\mcc \slice 1\) for every morphism \(\morph{f}{A}{B}\) in \(\mcc\) such
    that \(f \seq !_B = !_A\); since both \(f \seq !_B\) and \(!_A\) are
    morphisms \(A \to 1\) this condition holds for any morphism \(f\) in
    \(\mcc\).
    Therefore \(\mcc \cong \mcc \slice 1\).
\end{proof}

The strategy for this proof is much the same as that used in
\cref{thm:frobc-iso-finset-slice-c}, but relies on one small observation.

\begin{lemma}
    \label{lem:finsetprop-finite}
    Let \(C \in \finsetprop\) be a finite cardinal.
    Then \(\finsetpropwithnat \slice C \cong \finsetprop \slice C\).
\end{lemma}
\begin{proof}
    The morphisms in \(\finsetpropwithnat \slice C\) are the morphisms
    \([m] \to C\) for finite \(C\), which are precisely the morphisms of
    \(\finsetprop \slice C\).
\end{proof}

\begin{theorem}
    \label{thm:frobc-iso-hatfinset-slice-c}
    For a countable set of sorts \(C\), there is an isomorphism of coloured
    PROPs \(\cfrob{C} \cong \csp{\finsetpropwithnat \slice C}\).
\end{theorem}
\begin{proof}
    The proof is almost the same as \cref{thm:frobc-iso-finset-slice-c} but with
    the addition of \cref{lem:finsetprop-finite}

    By definition of \(\cfrob{C}\),
    \cref{def:coloured-frob,prop:frob-finset,lem:slice-iso-terminal,lem:finsetprop-finite}
    we have that \[
        \cfrob{C}
        :=
        \sum_{c \in C}\frob
        \cong
        \sum_{c \in C}\csp{\finsetprop}
        \cong
        \sum_{c \in C}\csp{\finsetprop \slice 1}
        \cong
        \sum_{c \in C}\csp{\finsetpropwithnat \slice 1}.
    \]
    In the other direction we still have that \(
        \csp{\finsetpropwithnat \slice C}
        \cong
        \csp{\finsetpropwithnat \slice \sum_{c \in C} 1}
    \) even though \(C\) is not finite as it is still countable.
    So as before we need to show that \(
        \sum_{c \in C}\csp{\finsetpropwithnat \slice 1}
        \cong
        \csp{\finsetpropwithnat \slice \sum_{c \in C} 1}
    \), which follows by the same reasoning as in the prequel.
\end{proof}

\begin{corollary}[\cite{bonchi2022string}, Cor. 3.9]
    There is a faithful PROP homomorphism
    \(\morph{\tilde{D_\mcs}}{\csp{\finsetprop}}{\cspdhyp}\)
\end{corollary}



Recall the definition of a \emph{hypergraph category} from
\cref{def:hypergraph-category}; perhaps confusingly, the category of hypergraphs
\(\hyp\) is \emph{not} a hypergraph category but the category of \emph{cospans}
of hypergraphs is!

\begin{proposition}[\cite{carboni1987cartesian,bonchi2022string}]
    \label{prop:frobenius-map}
    \(\cspdchyp\) is a hypergraph category.
\end{proposition}
\begin{proof}
    A Frobenius structure can be defined on \(\cspdhyp\) for each \(n \in \nat\)
    as follows:
    \begin{gather*}
        \iltikzfig{strings/structure/monoid/merge}[colour=white, obj=n]
        :=
        \cospan{n + n}{n}{n}
        \quad
        \iltikzfig{strings/structure/monoid/init}[colour=white, obj=n]
        :=
        \cospan{0}{n}{n}
        \\
        \iltikzfig{strings/structure/comonoid/copy}[colour=white, obj=n]
        :=
        \cospan{n}{n}{n+n}
        \quad
        \iltikzfig{strings/structure/comonoid/discard}[colour=white, obj=n]
        :=
        \cospan{n}{n}{0}
        \qedhere
    \end{gather*}
\end{proof}

\begin{definition}
    Let \(\morph{\frobtohypsigmac}{\frobs}{\cspdchyp}\) be a PROP morphism
    defined as in \cref{prop:frobenius-map}.
\end{definition}

This means we can map from both generators of symmetric monoidal terms and the
Frobenius structure to cospans of hypergraphs.
Therefore to map from Frobenius terms, we simply put them together.

\begin{definition}
    Let \(
        \morph{\termandfrobtohypsigmac}{\smcsigmas + \frobs}{\cspdchyp}
    \) be the copairing of \(\termtohypsigmac\) and
    \(\frobtohypsigmac\).
\end{definition}

\begin{proposition}[\cite{bonchi2022string}, ]

\end{proposition}

\begin{proposition}[\cite{bonchi2022string}]\label{prop:tohyp-faithful}
    \(\termtohypsigmac\) and \(\frobtohypsigmac\) are faithful
\end{proposition}
\begin{proof}

\end{proof}

\todo[inline]{Adapt from FSCD paper section 2}