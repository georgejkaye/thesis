\chapter{Abstract}

This thesis details the culmination of a project to define a
\emph{fully compositional} theory of synchronous sequential circuits built from
primitive components, motivated by applying techniques successfully used in
programming languages to hardware.

The first part of the thesis defines the syntactic foundations required to
create sequential circuit morphisms, and then builds three different semantic
theories on top of this: denotational, operational and algebraic.
We characterise the denotational semantics of sequential semantics as certain
\emph{causal stream functions}, as well as providing a link to existing circuit
methodologies by mapping between circuit morphisms, stream functions and
\emph{Mealy machines}.
The operational semantics is defined as a strategy for applying some global
transformations following by local reductions in order to demonstrate how a
circuit processes a value, leading to a notion of
\emph{observational equivalence}.
The algebraic semantics consists of equations for bringing circuits into
a pseudo-normal form, and then \emph{encoding} between different state sets.
This part of the thesis is concluded with a discussion of some potentially
novel applications, such as those for using \emph{partial evaluation} for
digital circuits.

While mathematically rigorous, the categorical string diagram formalism is not
suited for reasoning computationally.
The second part of this thesis details an extension of existing work on string
diagram rewriting with \emph{hypergraphs} so that it is compatible with the
\emph{traced comonoid} structure present in the category of digital circuits.
We identify the properties that characterise cospans of hypergraphs
corresponding to traced comonoid terms, and demonstrate how to identify
rewriting contexts valid for rewriting modulo traced comonoid structure.
As an example we apply the graph rewriting framework to
\emph{fixpoint operators} as well as the operational semantics from the first
part, and present a new hardware description language based on these theoretical
developments.