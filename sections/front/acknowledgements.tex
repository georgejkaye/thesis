\chapter{Acknowledgements}\label{chap:acknowledgements}
\markboth{Acknowledgements}{Acknowledgements}

The hardest part of writing a thesis is determining exactly who to put in the
acknowledgements.
A good place to start is the people who had the most direct input into my
development as an academic, starting of course with the inimitable Dan Ghica,
who dragged me kicking and screaming into the world of category theory.
Miriam Backens played an instrumental role, by tirelessly reading any paper
draft I sent their way and by providing useful observations on problems I was
having.
The influence of David Sprunger cannot be underestimated, as it was his
careful and methodical approach to mathematics that I attempted to emulate in
the latter half of my studies, the success of which will be determined by the
contents of this thesis.
Although the RSMG process is often lambasted by my peers, I found it an
excellent way to check how things were going, so thanks must go to Achim Jung
and Martín Escardó for providing sage advice during thesis group meetings.
And finally, thank you to Timothy Bourke, Eric Finster, and Paul Levy for
agreeing to attend the ultimate thesis group meeting: my viva.

The Theory Group at Birmingham is perhaps one of the most vibrant in the field,
so I am proud to have been a member over the past five years.
I have had some great conversations with the group's PhD students, which have
included Nicolas Blanco, Tom de Jong, Alex Rice, Calin Tataru,
Paaras Padhiar, and Ayberk Tosun.
The students have been supplemented by an army of postdocs,
including Chris Barrett, Gianluca Curzi, Abhishek De, Iris van der Giessen, and
Sam Speight; and last but not least there have been the permanent staff keeping
things (not too) serious at the top: among those already mentioned, Anupam Das
and Sonia Marin have done their bit to make the group a friendly and sociable
bunch.

Immersing oneself wholly in theory would have an extremely detrimental effect to
one's health, so fortunately I have had the pleasure of associating with
numerous non-theory PhD students along the way.
I started my journey alongside Andrea Basso and Charlotte Weitkämper, and with
Alex and Calin we managed to eventually managed to find our feet in the often
chaotic world of postgraduate research.
Thanks to a minor global event in 2020 this group was sadly splintered, but when
there is a void something always springs up to fill the gap.
To say that a Mastodon server self-hosted by Jon Freer has had a large impact on
me would be an understatement, and thanks to this tight-knit community I have
enjoyed countless entertaining shenanigans with Matthew Bowden, Dhurim Cakiqi,
Anna Clee, Matthew Hammond, Tobias Schmude, Charlie Street, Yánrong Wang, and
Jacob Thomas Whitnell Wilson.

A department is only as strong as its administration: Computer Science is lucky
to have had several diligent and competent members of Professional Services
staff, including Angeliki Bompetsi, Sarah Brookes, Kate Campbell, Jason
Fenemore, Tom Holly, Jamie Hough, Mohammed Idrees, Wenxuan Pan, Eden Whitehead,
and Mohammed Zafar.
Special mention must go to Lydia Eason for the ease at which she operated as
SRSCC secretary while juggling student experience, and also for taking my
request for better wine seriously.

If you've been following closely, you might notice that some names are missing.
I thrive in a double act and have been fortunate enough to be part of several
in my time at Birmingham.
For better or for worse, Todd Waugh Ambridge has been there since the beginning,
pointing out my most egregious errors and sharing countless balcony chats with
tea and biscuits; with him I have engaged in many adventures in academia,
including one particularly memorable trip to Edinburgh.
Bruno da Rocha Paiva has been a near-constant office companion for the past two
years; perhaps his presence has made progress on this thesis slower than it
could have been, but the time would not have been nearly as interesting.
Jacqueline Henes has acted as an incredibly capable SRSCC chair and provided a
much-needed second perspective to the issues PGRs face; together we have battled
operational issues galore and without her determination the PGR community would
be in a much graver state.

Charlotte was nice to have around too.
