\section{Algebraic semantics for Belnap logic}\label{sec:algebraic-belnap}

\renewcommand\theContinuedFloat{\alph{ContinuedFloat}}
\begin{figure}
    \ContinuedFloat*
    \centering
    \scalebox{0.85}{\(\equationdisplay{
            \iltikzfig{circuits/axioms/belnap/translation/explosion-lhs}
        }{
            \iltikzfig{circuits/axioms/belnap/translation/explosion-rhs}
        }{
            \belnapexpeqn
        }\)}
    \quad
    \scalebox{0.85}{\(\equationdisplay{
            \iltikzfig{circuits/algebraic/infinite-register-fork-lhs}
        }{
            \iltikzfig{circuits/algebraic/infinite-register-fork-rhs}
        }{
            \infregforkeqn
        }\)}
    \\[0.4em]
    \rule{\textwidth}{0.1mm}
    \\[0.5em]
    \scalebox{0.85}{\(\equationdisplay{
            \iltikzfig{circuits/algebraic/infinite-register-and}
        }{
            \iltikzfig{circuits/components/waveforms/infinite-register-wide}[val={\bot \land v}]
        }{
            \infregandeqn
        }\)}
    \,\,
    \scalebox{0.85}{\(\equationdisplay{
            \iltikzfig{circuits/algebraic/infinite-register-or}
        }{
            \iltikzfig{circuits/components/waveforms/infinite-register-wide}[val={\bot \lor v}]
        }{
            \infregoreqn
        }\)}
    \\[0.25em]
    \rule{\textwidth}{0.1mm}
    \\[0.5em]
    \scalebox{0.85}{\(\equationdisplay{
            \iltikzfig{circuits/algebraic/infinite-register-not}
        }{
            \iltikzfig{circuits/components/waveforms/infinite-register-slightly-wide}[val={\neg v}]
        }{
            \infregnoteqn
        }\)}
    \quad
    \scalebox{0.85}{\(\equationdisplay{
            \iltikzfig{circuits/axioms/fork-lhs}
        }{
            \iltikzfig{circuits/axioms/fork-rhs}
        }{
            \forkeqn
        }\)}
    \quad
    \scalebox{0.85}{\(\equationdisplay{
            \iltikzfig{strings/structure/bialgebra/init-copy-lhs}
        }{
            \iltikzfig{strings/structure/bialgebra/init-copy-rhs}
        }{
            \botforkeqn
        }\)}
    \\[0.25em]
    \rule{\textwidth}{0.1mm}
    \\[0.5em]
    \scalebox{0.85}{\(\equationdisplay{
            \iltikzfig{circuits/axioms/belnap/translation/join-and-lhs}
        }{
            \iltikzfig{circuits/axioms/belnap/translation/join-and-rhs}
        }{
            \joinandeqn
        }\)}
    \,\,
    \scalebox{0.85}{\(\equationdisplay{
            \iltikzfig{circuits/axioms/belnap/translation/join-or-lhs}
        }{
            \iltikzfig{circuits/axioms/belnap/translation/join-or-rhs}
        }{
            \joinoreqn
        }\)}
    \,\,
    \scalebox{0.85}{\(\equationdisplay{
            \iltikzfig{circuits/axioms/belnap/translation/bot-and-lhs}
        }{
            \iltikzfig{circuits/axioms/belnap/translation/bot-x-rhs}
        }{
            \botandeqn
        }\)}
    \\[0.25em]
    \rule{\textwidth}{0.1mm}
    \\[0.5em]
    \scalebox{0.85}{\(\equationdisplay{
            \iltikzfig{circuits/axioms/belnap/translation/de-morgan-and-lhs}
        }{
            \iltikzfig{circuits/axioms/belnap/translation/de-morgan-and-rhs}
        }{
            \demorganand
        }\)}
    \,\,
    \scalebox{0.85}{\(\equationdisplay{
            \iltikzfig{circuits/axioms/belnap/translation/de-morgan-or-lhs}
        }{
            \iltikzfig{circuits/axioms/belnap/translation/de-morgan-or-rhs}
        }{
            \demorganor
        }\)}
    \,\,
    \scalebox{0.85}{\(\equationdisplay{
            \iltikzfig{circuits/axioms/belnap/translation/bot-or-lhs}
        }{
            \iltikzfig{circuits/axioms/belnap/translation/bot-x-rhs}
        }{
            \botoreqn
        }\)}
    \\[0.25em]
    \rule{\textwidth}{0.1mm}
    \\[0.5em]
    \caption{First part of the set of \emph{explosion equations} \(\mathcal{X}\)}
    \label{fig:explosion-equations-1}
\end{figure}

For a sound and complete equational theory, equations are required to bring any
essentially combinational circuit into a canonical form.
We will now demonstrate this for the Belnap interpretation; recall from
\cref{sec:denotational-belnap} that the canonical form for Belnap circuits is a
circuit that `explodes' its inputs into circuits for the `falsy' and `truthy'
components of the output, before joining these together.
The first equations we will define translate any essentially combinational
circuit into such an exploded circuit.

\begin{definition}[Explosion equations]
    Let the set of \emph{explosion equations} \(\mathcal{X}\) be defined as the
    equations listed
    in \cref{fig:explosion-equations-1,fig:explosion-equations-2}.
\end{definition}

\begin{lemma}
    The explosion equations are sound.
\end{lemma}
\begin{proof}
    By checking all the inputs.
\end{proof}

Most of the equations in \(\mathcal{X}\) are well-known; the only interesting
one is \((\belnapexpeqn)\), which says that we can always `explode' a wire and
join it back together.
To translate a circuit into exploded form, we use this equation to introduce an
`empty explosion', then propagate the original components across with the
other equations.
The first obstacle to this is the forks at the left of the explosion.

\begin{lemma}\label{lem:explode-copy}
    For any essentially combinational Belnap circuit \(
    \iltikzfig{strings/category/f}[box=f,dom=m,cod=n,colour=seq]
    \), the equation \(
    \iltikzfig{strings/structure/cartesian/naturality-copy-lhs}[colour=comb,box=f]
    =
    \iltikzfig{strings/structure/cartesian/naturality-copy-rhs}[colour=comb,box=f]
    \) in \(\scirc{\belnapsignature} / \mathcal{X}\).
\end{lemma}
\begin{proof}
    This follows for the combinational generators by applying
    \((\joinforkeqn)\), \((\botforkeqn)\), \((\andforkeqn)\), \((\orforkeqn)\),
    \((\notforkeqn)\), and is immediate for the fork.
    The infinite register \(
    \iltikzfig{circuits/components/waveforms/infinite-register}[val=v]
    \) is also a base case and is covered by \((\infregforkeqn)\).
    The inductive cases are trivial.
\end{proof}

Once the circuit is past the opening forks, the remaining equations are used to
push them across the translators.

\begin{proposition}\label{prop:exploded-belnap}
    Given an essentially combinational Belnap circuit \(
    \iltikzfig{strings/category/f}[box=f,dom=m,cod=1,colour=seq]
    \), there exists combinational Belnap circuits \(
    \iltikzfig{strings/category/f-3-1}[box=f_0,dom1=1,dom2=m,dom3=m,cod=1,colour=comb]
    \) and \(
    \iltikzfig{strings/category/f-3-1}[box=f_1,dom1=1,dom2=m,dom3=m,cod=1,colour=comb]
    \) containing no \(
    \iltikzfig{circuits/components/gates/not}
    \) or \(
    \iltikzfig{strings/structure/monoid/merge}[colour=comb]
    \) generators, such that \[
        \iltikzfig{strings/category/f}[box=f,dom=m,cod=1,colour=seq]
        =
        \iltikzfig{circuits/axioms/belnap/translation/exploded-form}[box=f,dom=m,cod=1].
    \]
\end{proposition}
\begin{proof}
    First we consider the base cases.
    If \(
    \iltikzfig{strings/category/f}[box=f,colour=seq]
    \) is the identity, then it can be transformed into the desired form
    with
    \((\belnapexpeqn)\).
    Since \(\iltikzfig{strings/category/f}[box=f,colour=seq]\) has codomain
    \(1\) it cannot be a symmetry.
    For the other generators and the infinite register,
    \((\belnapexpeqn)\) can be applied to the output wire to create the
    exploded `skeleton', followed by using \cref{lem:explode-copy} to copy the
    components into four.
    The four copies can be pushed through the translators
    using \((\joinforkeqn)\), \((\joinandeqn)\), \((\joinoreqn)\),
    \((\botforkeqn)\), \((\botoreqn)\) \((\andcomm)\), \((\orcomm)\),
    \((\andorddisteqn)\), \((\oranddisteqn)\), \((\infregandeqn)\),
    \((\infregoreqn)\), \((\infregnoteqn)\), \((\demorganand)\),
    \((\demorganor)\), \((\botnoteqn)\), \((\botandeqn)\), and \((\botoreqn)\).
    As propagating the \(
    \iltikzfig{circuits/components/gates/not}
    \) flips the translators by using \((\demorganand)\) and \((\demorganor)\),
    \((\forkcommeqn)\) must be used to restore the correct order, and
    \((\dneeqn)\) is used to eliminate additional
    \(\iltikzfig{circuits/components/gates/not}\) gates.
    Any infinite registers containing \(\bot\) can be converted to
    \(\iltikzfig{strings/structure/monoid/init}[colour=comb]\) components using
    \((\botregeqn)\), and other registers can be combined using
    \((\infregforkeqn)\).

    For the composition inductive case, we have two exploded circuits and we
    need to push the first inside the second.
    Using \cref{lem:explode-copy}, the first circuit can be propagated across
    the forks at the start of the second, so each of the four translators has as
    input a copy of the first circuit.
    Using the same strategy as for the base case the components of the circuit
    can then be propagated across the translators.
    For tensor, the circuits can be interleaved using axioms of STMCs.
\end{proof}
%
\begin{figure}\ContinuedFloat
    \centering
    \scalebox{0.85}{\(\equationdisplay{
            \iltikzfig{circuits/axioms/belnap/translation/not-fork-lhs}
        }{
            \iltikzfig{circuits/axioms/belnap/translation/not-fork-rhs}
        }{
            \notforkeqn
        }\)}
    \scalebox{0.85}{\(\equationdisplay{
            \iltikzfig{circuits/axioms/belnap/translation/and-fork-lhs}
        }{
            \iltikzfig{circuits/axioms/belnap/translation/and-fork-rhs}
        }{
            \andforkeqn
        }\)}
    \scalebox{0.85}{\(\equationdisplay{
            \iltikzfig{circuits/axioms/belnap/translation/or-fork-lhs}
        }{
            \iltikzfig{circuits/axioms/belnap/translation/or-fork-rhs}
        }{
            \orforkeqn
        }\)}
    \\[0.25em]
    \rule{\textwidth}{0.1mm}
    \\[0.5em]
    \scalebox{0.85}{\(\equationdisplay{
            \iltikzfig{circuits/axioms/belnap/double-negation-lhs}
        }{
            \iltikzfig{circuits/axioms/belnap/double-negation-rhs}
        }{
            \dneeqn
        }\)}
    \,
    \scalebox{0.85}{\(\equationdisplay{
            \iltikzfig{strings/structure/comonoid/associativity-lhs}
        }{
            \iltikzfig{strings/structure/comonoid/associativity-rhs}
        }{
            \forkassoceqn
        }\)}
    \,
    \scalebox{0.85}{\(\equationdisplay{
            \iltikzfig{strings/structure/bialgebra/merge-copy-lhs}
        }{
            \iltikzfig{strings/structure/bialgebra/merge-copy-rhs}
        }{
            \joinforkeqn
        }\)}
    \,
    \scalebox{0.85}{\(\equationdisplay{
            \iltikzfig{strings/structure/comonoid/commutativity-lhs}
        }{
            \iltikzfig{strings/structure/comonoid/commutativity-rhs}
        }{
            \forkcommeqn
        }\)}
    \\[0.25em]
    \rule{\textwidth}{0.1mm}
    \\[0.5em]
    \scalebox{0.85}{\(\equationdisplay{
            \iltikzfig{circuits/axioms/belnap/and-idempotent-lhs}
        }{
            \iltikzfig{circuits/axioms/belnap/and-idempotent-rhs}
        }{
            \andidemeqn
        }\)}
    \quad
    \scalebox{0.85}{\(\equationdisplay{
            \iltikzfig{circuits/axioms/belnap/or-idempotent-lhs}
        }{
            \iltikzfig{circuits/axioms/belnap/or-idempotent-rhs}
        }{
            \oridemeqn
        }\)}
    \quad
    \scalebox{0.85}{\(\equationdisplay{
            \iltikzfig{circuits/axioms/delay-fork-lhs}
        }{
            \iltikzfig{circuits/axioms/delay-fork-rhs}
        }{
            \delayforkeqn
        }\)}
    \quad
    \scalebox{0.85}{\(\equationdisplay{
            \iltikzfig{circuits/axioms/belnap/translation/bot-not-lhs}
        }{
            \iltikzfig{circuits/axioms/belnap/translation/bot-x-rhs}
        }{
            \botnoteqn
        }\)}
    \\[0.25em]
    \rule{\textwidth}{0.1mm}
    \\[0.5em]
    \scalebox{0.85}{\(\equationdisplay{
            \iltikzfig{circuits/axioms/belnap/and-or-distributivity-lhs}
        }{
            \iltikzfig{circuits/axioms/belnap/and-or-distributivity-rhs}
        }{
            \andorddisteqn
        }\)}
    \,\,
    \scalebox{0.85}{\(\equationdisplay{
            \iltikzfig{circuits/axioms/belnap/or-and-distributivity-lhs}
        }{
            \iltikzfig{circuits/axioms/belnap/or-and-distributivity-rhs}
        }{
            \oranddisteqn
        }\)}
    \\[0.25em]
    \rule{\textwidth}{0.1mm}
    \\[0.5em]
    \scalebox{0.85}{\(\equationdisplay{
            \iltikzfig{circuits/axioms/belnap/and-commutativity-lhs}
        }{
            \iltikzfig{circuits/axioms/belnap/and-commutativity-rhs}
        }{
            \andcomm
        }\)}
    \,\,
    \scalebox{0.85}{\(\equationdisplay{
            \iltikzfig{circuits/axioms/belnap/or-commutativity-lhs}
        }{
            \iltikzfig{circuits/axioms/belnap/or-commutativity-rhs}
        }{
            \orcomm
        }\)}
    \\[0.25em]
    \rule{\textwidth}{0.1mm}
    \caption{Second part of the set of \emph{explosion equations} \(\mathcal{X}\)}
    \label{fig:explosion-equations-2}
\end{figure}

This already looks similar to a circuit in the image of
\(\mealytofunc_\belnap\), but the two subcircuits must also be translated into
truthy or falsy disjunctive normal form.

\begin{definition}[Normal form equations]
    Let the set of \emph{normal form equations} \(\mathcal{F}\) be defined as
    the equations listed in \cref{fig:normal-form-equations}.
\end{definition}

\begin{lemma}
    The equations in \(\mathcal{F}\) are sound.
\end{lemma}
\begin{proof}
    By checking all the inputs.
\end{proof}

\begin{figure*}
    \centering
    \scalebox{0.85}{\(\equationdisplay{
            \iltikzfig{circuits/axioms/belnap/and-associativity-lhs}
        }{
            \iltikzfig{circuits/axioms/belnap/and-associativity-rhs}
        }{\andassoc}\)}
    \quad
    \scalebox{0.85}{\(\equationdisplay{
            \iltikzfig{circuits/axioms/belnap/or-associativity-lhs}
        }{
            \iltikzfig{circuits/axioms/belnap/or-associativity-rhs}
        }{\orassoc}\)}
    \\[0.25em]
    \rule{\textwidth}{0.1mm}
    \\[0.7em]
    \scalebox{0.85}{\(\equationdisplay{
            \iltikzfig{circuits/axioms/belnap/and-or-distributivity-lhs}
        }{
            \iltikzfig{circuits/axioms/belnap/and-or-distributivity-rhs}
        }{\andorddisteqn}\)}
    \quad
    \scalebox{0.85}{\(\equationdisplay{
            \iltikzfig{circuits/axioms/belnap/or-and-distributivity-lhs}
        }{
            \iltikzfig{circuits/axioms/belnap/or-and-distributivity-rhs}
        }{\oranddisteqn}\)}
    \\[0.25em]
    \rule{\textwidth}{0.1mm}
    \\[0.7em]
    \scalebox{0.85}{\(\equationdisplay{
            \iltikzfig{circuits/axioms/belnap/and-commutativity-lhs}
        }{
            \iltikzfig{circuits/axioms/belnap/and-commutativity-rhs}
        }{\andcomm}\)}
    \quad
    \scalebox{0.85}{\(\equationdisplay{
            \iltikzfig{circuits/axioms/belnap/or-commutativity-lhs}
        }{
            \iltikzfig{circuits/axioms/belnap/or-commutativity-rhs}
        }{\orcomm}\)}
    \quad
    \scalebox{0.85}{\(\equationdisplay{
            \iltikzfig{circuits/axioms/belnap/and-idempotent-lhs}
        }{
            \iltikzfig{circuits/axioms/belnap/and-idempotent-rhs}
        }{\andidemeqn}\)}
    \\[0.25em]
    \rule{\textwidth}{0.1mm}
    \\[0.7em]
    \scalebox{0.85}{\(\equationdisplay{
            \iltikzfig{circuits/axioms/belnap/or-idempotent-lhs}
        }{
            \iltikzfig{circuits/axioms/belnap/or-idempotent-rhs}
        }{\oridemeqn}\)}
    \quad
    \scalebox{0.85}{\(\equationdisplay{
            \iltikzfig{circuits/axioms/belnap/translation/and-fork-lhs}
        }{
            \iltikzfig{circuits/axioms/belnap/translation/and-fork-rhs}
        }{\andforkeqn}\)}
    \quad
    \scalebox{0.85}{\(\equationdisplay{
            \iltikzfig{circuits/axioms/belnap/translation/or-fork-lhs}
        }{
            \iltikzfig{circuits/axioms/belnap/translation/or-fork-rhs}
        }{\andforkeqn}\)}
    \\[0.25em]
    \rule{\textwidth}{0.1mm}
    \\[0.7em]
    \scalebox{0.85}{\(\equationdisplay{
            \iltikzfig{strings/structure/comonoid/commutativity-rhs}
        }{
            \iltikzfig{strings/structure/comonoid/commutativity-lhs}
        }{\forkcommeqn}\)}
    \quad
    \scalebox{0.85}{\(\equationdisplay{
            \iltikzfig{strings/structure/comonoid/associativity-lhs}
        }{
            \iltikzfig{strings/structure/comonoid/associativity-rhs}
        }{\forkassoceqn}\)}
    \quad
    \scalebox{0.85}{\(\equationdisplay{
            \iltikzfig{strings/structure/comonoid/unitality-l-lhs}
        }{
            \iltikzfig{strings/structure/comonoid/unitality-l-rhs}
        }{\forkuniteqn}\)}
    \quad
    \scalebox{0.85}{\(\equationdisplay{
            \iltikzfig{strings/structure/bialgebra/init-copy-lhs}
        }{
            \iltikzfig{strings/structure/bialgebra/init-copy-rhs}
        }{
            \botforkeqn
        }\)}
    \\[0.25em]
    \rule{\textwidth}{0.1mm}
    \\[0.5em]
    \scalebox{0.85}{\(\equationdisplay{
            \iltikzfig{circuits/axioms/belnap/and-identity-lhs}
        }{
            \iltikzfig{circuits/axioms/belnap/and-identity-rhs}
        }{\andideqn}\)}
    \quad
    \scalebox{0.85}{\(\equationdisplay{
            \iltikzfig{circuits/axioms/belnap/or-identity-lhs}
        }{
            \iltikzfig{circuits/axioms/belnap/or-identity-rhs}
        }{\orideqn}\)}
    \quad
    \scalebox{0.85}{\(\equationdisplay{
            \iltikzfig{circuits/components/waveforms/infinite-register}[val=\bot]
        }{
            \iltikzfig{strings/structure/monoid/init}[colour=comb]
        }{\botregeqn}\)}
    \quad
    \\[0.25em]
    \rule{\textwidth}{0.1mm}
    \\[0.5em]
    \scalebox{0.85}{\(\equationdisplay{
            \iltikzfig{circuits/axioms/belnap/translation/and-bot-unit-lhs}
        }{
            \iltikzfig{circuits/axioms/belnap/translation/and-bot-unit-rhs}
        }{\botanduniteqn}\)}
    \quad
    \scalebox{0.85}{\(\equationdisplay{
            \iltikzfig{circuits/axioms/belnap/translation/or-bot-unit-lhs}
        }{
            \iltikzfig{circuits/axioms/belnap/translation/or-bot-unit-rhs}
        }{\botoruniteqn}\)}
    \\[0.25em]
    \rule{\textwidth}{0.1mm}
    \\[0.5em]
    \caption{
        Set of \emph{normal form equations} \(\mathcal{F}\).
    }
    \label{fig:normal-form-equations}
\end{figure*}


We will now show that these equations suffice to translate the subcircuits in
the exploded circuit into falsy or truthy disjunctive normal form.

\begin{lemma}\label{lem:truthy-conjunction-normalising}
    Given a Belnap circuit \(
    \iltikzfig{circuits/synthesis/normalised-function}[box=f,dom=m,cod=n,values=\belnaptrue]
    \) containing no \(
    \iltikzfig{strings/structure/monoid/merge}[colour=comb]
    \), \(
    \iltikzfig{circuits/components/gates/or}
    \) or \(
    \iltikzfig{circuits/components/gates/not}
    \) components, there exists a circuit \(
    \iltikzfig{strings/category/f}[box=g,colour=comb,dom=mn,cod=p]
    \) containing only identity and elimination constructs, and a circuit \(
    \iltikzfig{strings/category/f}[box=h,colour=seq,dom=p,cod=n]
    \) defined as the tensor of \(n\) truthy conjunctions, such that \(
    \iltikzfig{circuits/synthesis/normalised-function}[box=f,dom=m,cod=n,values=\belnaptrue]
    =
    \iltikzfig{circuits/algebraic/unction-circuit}
    \) in \(\scirc{\belnapsignature} / \mathcal{F}\).
\end{lemma}
\begin{proof}
    Repeatedly applying \((\andforkeqn)\) to \(
    \iltikzfig{strings/category/f}[box=f,colour=comb]
    \) propagates the \(
    \iltikzfig{circuits/components/gates/and}
    \) components in the circuit as far to the right as possible, so all the
    fork and eliminate constructs are all in the left half of the term.
    Using \((\forkcommeqn)\), \((\forkassoceqn)\) to rearrange the forks, and
    \((\forkuniteqn)\) to introduce forks where necessary,
    we can manipulate these forks such that there is a wire for each of the \(m\)
    inputs feeds into each of the \(n\) outputs.
    Similarly, we can use \((\andideqn)\) to introduce infinite registers where
    appropriate, so we have a circuit of the form below. \[
        \iltikzfig{circuits/algebraic/belnap-rearrange-forks}
    \]
    The subcircuits may not be truthy conjunctions yet, as the input wires may
    be used more than once.
    Using the \((\andidemeqn)\), \((\forkcommeqn)\), \((\forkassoceqn)\),
    \((\andassoc)\) and \((\andcomm)\), each subcircuit can be translated into
    a truthy conjunction.
\end{proof}

% \begin{example}
%     Consider the circuit \(
%     \iltikzfig{circuits/examples/conjunction/step-0}
%     \); we transform it into a truthy conjunction as follows.
%     \begin{gather*}
%         \iltikzfig{circuits/examples/conjunction/step-0}
%         \eqaxioms[(\forkuniteqn)]
%         \iltikzfig{circuits/examples/conjunction/step-1}
%         \eqaxioms[(\forkuniteqn)]
%         \iltikzfig{circuits/examples/conjunction/step-2}
%         \eqaxioms[(\forkcommeqn)]
%         \\
%         \iltikzfig{circuits/examples/conjunction/step-3}
%         \eqaxioms[(\forkassoceqn)]
%         \iltikzfig{circuits/examples/conjunction/step-4}
%         \eqaxioms[(\andcomm)]
%         \iltikzfig{circuits/examples/conjunction/step-5}
%         \eqaxioms[(\andforkeqn)]
%         \\
%         \iltikzfig{circuits/examples/conjunction/step-6}
%         \eqaxioms[(\andassoc)]
%         \iltikzfig{circuits/examples/conjunction/step-7}
%         \eqaxioms[(\forkassoceqn)]
%         \\
%         \iltikzfig{circuits/examples/conjunction/step-8}
%         \eqaxioms[(\andidemeqn)]
%         \iltikzfig{circuits/examples/conjunction/step-9}
%         \eqaxioms[(\andassoc)]
%         \\
%         \iltikzfig{circuits/examples/conjunction/step-10}
%         \eqaxioms[(\andcomm)]
%         \iltikzfig{circuits/examples/conjunction/step-11}
%         \eqaxioms[(\andassoc)]
%         \\
%         \iltikzfig{circuits/examples/conjunction/step-12}
%         \eqaxioms[(\andidemeqn)]
%         \iltikzfig{circuits/examples/conjunction/step-13}
%         \eqaxioms[(\andassoc)]
%         \\
%         \iltikzfig{circuits/examples/conjunction/step-14}
%         \eqaxioms[(\andidemeqn)]
%         \iltikzfig{circuits/examples/conjunction/step-15}
%         \eqaxioms
%         \iltikzfig{circuits/examples/conjunction/step-16}
%         \eqaxioms[(\andideqn)]
%         \iltikzfig{circuits/examples/conjunction/step-17}
%     \end{gather*}
% \end{example}

The proof for the falsy circuit is almost exactly the same.

\begin{lemma}\label{lem:falsy-conjunction-normalising}
    Given a Belnap circuit \(
    \iltikzfig{circuits/synthesis/normalised-function}[box=f,dom=m,cod=n,values=\belnaptrue]
    \) containing no \(
    \iltikzfig{strings/structure/monoid/merge}[colour=comb]
    \), \(
    \iltikzfig{circuits/components/gates/and}
    \) or \(
    \iltikzfig{circuits/components/gates/not}
    \) components, there exists a circuit \(
    \iltikzfig{strings/category/f}[box=g,colour=comb,dom=mn,cod=p]
    \) containing only identity and elimination constructs, and a circuit \(
    \iltikzfig{strings/category/f}[box=h,colour=comb,dom=p,cod=n]
    \) defined as the tensor of \(n\) falsy conjunctions, such that \(
    \iltikzfig{strings/category/f}[box=f,colour=comb]
    =
    \iltikzfig{circuits/algebraic/unction-circuit}
    \) in \(\scirc{\belnapsignature} / \mathcal{F}\).
\end{lemma}
\begin{proof}
    As \cref{lem:truthy-conjunction-normalising}, but with the equations
    on \(\iltikzfig{circuits/components/gates/or}\) components.
\end{proof}

Truthy and falsy conjunctions can then be used to create truthy and falsy
conjunctive normal forms.

\begin{proposition}\label{prop:disjunctive-normal-form}
    Given a Belnap circuit \[
        \iltikzfig{circuits/axioms/belnap/translation/exploded-form}[box=f,dom=m,cod=1]
    \] in which \(
    \iltikzfig{strings/category/f-3-1}[box=f_0,colour=comb]
    \) and \(
    \iltikzfig{strings/category/f-3-1}[box=f_1,colour=comb]
    \) contain no \(
    \iltikzfig{circuits/components/gates/not}
    \) or \(
    \iltikzfig{strings/structure/monoid/merge}[colour=comb]
    \) generators, there exists \[
        \iltikzfig{circuits/axioms/belnap/translation/exploded-form-cnf}[box=f,dom=m,cod=1]
    \] in which \(
    \iltikzfig{strings/category/f}[box=g_0,colour=comb]
    \) and \(
    \iltikzfig{strings/category/f}[box=g_1,colour=comb]
    \) contain only identity and elimination components,
    \(
    \iltikzfig{strings/category/f}[box=h_0,colour=seq]
    \) is in falsy conjunctive normal form and\(
    \iltikzfig{strings/category/f}[box=h_1,colour=seq]
    \) is in truthy conjunctive normal form, such that \[
        \iltikzfig{circuits/axioms/belnap/translation/exploded-form}[box=f]
        =
        \iltikzfig{circuits/axioms/belnap/translation/exploded-form-cnf}
    \] in \(
    \scirc{\belnapsignature} / \mathcal{F}
    \).
\end{proposition}
\begin{proof}
    First, all the \(
    \iltikzfig{circuits/components/gates/and}
    \) components need to be propagated to the far right of the \(
    \iltikzfig{strings/category/f-3-1-compressed}[box=f_0,colour=comb]
    \) circuit using \((\oranddisteqn)\), and all the \(
    \iltikzfig{circuits/components/gates/or}
    \) components need to be propagated to the far right of the \(
    \iltikzfig{strings/category/f-3-1-compressed}[box=f_1,colour=comb]
    \) circuit using \((\andorddisteqn)\).
    This means that the circuits are split into two halves, each containing
    one type of gate.

    For these circuits to be in truthy or falsy disjunctive normal form, they
    need to contain exactly one \(
    \iltikzfig{strings/structure/monoid/init}[colour=comb]
    \) component.
    If there is not already such a component inside the \(
    \iltikzfig{strings/category/f-3-1-compressed}[box=f_0,colour=comb]
    \) or \(
    \iltikzfig{strings/category/f-3-1-compressed}[box=f_1,colour=comb]
    \) subcircuits, one can be inserted using the \((\botanduniteqn)\) equation
    for the former and the \((\botoruniteqn)\) equation for the latter.
    If there are multiple unit components, these can be propagated through the
    circuit using \((\orassoc)\), \((\oranddisteqn)\), \((\andassoc)\), and
    \((\andorddisteqn)\), and combined into one by using \((\botandeqn)\) or
    \((\botoreqn)\).

    Now we have circuits that have the `root' of disjunctive normal forms, but
    the `leaves' are not conjunctions.
    This is remedied by applying \cref{lem:truthy-conjunction-normalising} and
    \cref{lem:falsy-conjunction-normalising} to the left half of each circuit.
\end{proof}

Putting this all together gives us the desired canonical form theorem.

\begin{theorem}
    Given an essentially combinational Belnap circuit \(
    \iltikzfig{strings/category/f}[box=f,colour=seq]
    \), there exists a circuit \(
    \iltikzfig{strings/category/f}[box=g,colour=seq]
    \) in the image of \(\mealytofunc_\belnap\) such that \(
    \iltikzfig{strings/category/f}[box=f,colour=seq]
    =
    \iltikzfig{strings/category/f}[box=g,colour=seq]
    \) in \(\ccirc{\Sigma_\belnap} / \mathcal{X} + \mathcal{F}\).
\end{theorem}
\begin{proof}
    This follows by applying \cref{prop:exploded-belnap} followed by
    \cref{prop:disjunctive-normal-form}.
\end{proof}

This shows that the equations detailed in this section can translate any
essentially combinational Belnap circuit into a circuit in the image of the
functional completeness map.