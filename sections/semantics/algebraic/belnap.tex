\section{Algebraic semantics for Belnap logic}\label{sec:algebraic-belnap}

For a sound and complete equational theory, equations are required to bring any
essentially combinational circuit into a canonical form.
Recall that the canonical form for circuits using components in the Belnap
signature \(\belnapsignature\) is a circuit that `explodes' its inputs into
two circuits computing the `falsy' and the `truthy' components of the output,
before joining these together.

Since this form is a variation of the standard disjunctive normal form for
Boolean circuits, most of the equations required here should be familiar.
The only unusual equation is one to translate an identity into an `exploded'
identity, so that the rest of the circuit can be divided into the two
components.

\begin{definition}
    Let the set \(\mathcal{X}\) of \emph{explosion equations} be defined as
    in \cref{fig:explosion-equations}
\end{definition}

\begin{figure*}
    \centering
    \(\equationdisplay{
        \iltikzfig{circuits/axioms/belnap/translation/explosion-lhs}
    }{
        \iltikzfig{circuits/axioms/belnap/translation/explosion-rhs}
    }{
        \belnapexpeqn
    }\)
    \quad
    \(\equationdisplay{
        \iltikzfig{circuits/algebraic/infinite-register-fork-lhs}
    }{
        \iltikzfig{circuits/algebraic/infinite-register-fork-rhs}
    }{
        \regforkeqn
    }\)
    \\[0.25em]
    \rule{\textwidth}{0.1mm}
    \\[0.5em]
    \(\equationdisplay{
        \iltikzfig{circuits/axioms/belnap/translation/join-and-lhs}
    }{
        \iltikzfig{circuits/axioms/belnap/translation/join-and-rhs}
    }{
        \joinandeqn
    }\)
    \quad
    \(\equationdisplay{
        \iltikzfig{circuits/axioms/belnap/translation/join-or-lhs}
    }{
        \iltikzfig{circuits/axioms/belnap/translation/join-or-rhs}
    }{
        \joinoreqn
    }\)
    \\[0.25em]
    \rule{\textwidth}{0.1mm}
    \\[0.5em]
    \(\equationdisplay{
        \iltikzfig{circuits/axioms/belnap/translation/de-morgan-and-lhs}
    }{
        \iltikzfig{circuits/axioms/belnap/translation/de-morgan-and-rhs}
    }{
        \demorganand
    }\)
    \(\equationdisplay{
        \iltikzfig{circuits/axioms/belnap/translation/de-morgan-or-lhs}
    }{
        \iltikzfig{circuits/axioms/belnap/translation/de-morgan-or-rhs}
    }{
        \demorganor
    }\)
    \\[0.25em]
    \rule{\textwidth}{0.1mm}
    \\[0.5em]
    \(\equationdisplay{
        \iltikzfig{circuits/axioms/belnap/translation/not-fork-lhs}
    }{
        \iltikzfig{circuits/axioms/belnap/translation/not-fork-rhs}
    }{
        \notforkeqn
    }\)
    \quad
    \(\equationdisplay{
        \iltikzfig{circuits/axioms/belnap/translation/and-fork-lhs}
    }{
        \iltikzfig{circuits/axioms/belnap/translation/and-fork-rhs}
    }{
        \andforkeqn
    }\)
    \\[0.25em]
    \rule{\textwidth}{0.1mm}
    \\[0.5em]
    \(\equationdisplay{
        \iltikzfig{circuits/axioms/belnap/translation/or-fork-lhs}
    }{
        \iltikzfig{circuits/axioms/belnap/translation/or-fork-rhs}
    }{
        \orforkeqn
    }\)
    \quad
    \(\equationdisplay{
        \iltikzfig{strings/structure/bialgebra/merge-copy-lhs}
    }{
        \iltikzfig{strings/structure/bialgebra/merge-copy-rhs}
    }{
        \joinforkeqn
    }\)
    \quad
    \(\equationdisplay{
        \iltikzfig{circuits/axioms/belnap/double-negation-lhs}
    }{
        \iltikzfig{circuits/axioms/belnap/double-negation-rhs}
    }{
        \dneeqn
    }\)
    \\[0.25em]
    \rule{\textwidth}{0.1mm}
    \\[0.5em]
    \(\equationdisplay{
        \iltikzfig{strings/structure/bialgebra/init-copy-lhs}
    }{
        \iltikzfig{strings/structure/bialgebra/init-copy-rhs}
    }{
        \botforkeqn
    }\)
    \quad
    \(\equationdisplay{
        \iltikzfig{strings/structure/comonoid/commutativity-lhs}
    }{
        \iltikzfig{strings/structure/comonoid/commutativity-rhs}
    }{
        \forkcommeqn
    }\)
    \quad
    \(\equationdisplay{
        \iltikzfig{strings/structure/comonoid/associativity-lhs}
    }{
        \iltikzfig{strings/structure/comonoid/associativity-rhs}
    }{
        \forkassoceqn
    }\)
    \\[0.25em]
    \rule{\textwidth}{0.1mm}
    \\[0.5em]
    \(\equationdisplay{
        \iltikzfig{circuits/axioms/belnap/and-idempotent-lhs}
    }{
        \iltikzfig{circuits/axioms/belnap/and-idempotent-rhs}
    }{
        \andidemeqn
    }\)
    \quad
    \(\equationdisplay{
        \iltikzfig{circuits/axioms/belnap/or-idempotent-lhs}
    }{
        \iltikzfig{circuits/axioms/belnap/or-idempotent-rhs}
    }{
        \oridemeqn
    }\)
    \\[0.25em]
    \rule{\textwidth}{0.1mm}
    \\[0.5em]
    \(\equationdisplay{
        \iltikzfig{circuits/axioms/belnap/and-or-distributivity-lhs}
    }{
        \iltikzfig{circuits/axioms/belnap/and-or-distributivity-rhs}
    }{
        \andorddisteqn
    }\)
    \quad
    \(\equationdisplay{
        \iltikzfig{circuits/axioms/belnap/or-and-distributivity-lhs}
    }{
        \iltikzfig{circuits/axioms/belnap/or-and-distributivity-rhs}
    }{
        \oranddisteqn
    }\)
    \\[0.25em]
    \rule{\textwidth}{0.1mm}
    \\[0.5em]
    \(\equationdisplay{
        \iltikzfig{circuits/axioms/belnap/translation/bot-and-lhs}
    }{
        \iltikzfig{circuits/axioms/belnap/translation/bot-x-rhs}
    }{
        \botandeqn
    }\)
    \quad
    \(\equationdisplay{
        \iltikzfig{circuits/axioms/belnap/translation/bot-or-lhs}
    }{
        \iltikzfig{circuits/axioms/belnap/translation/bot-x-rhs}
    }{
        \botoreqn
    }\)
    \quad
    \(\equationdisplay{
        \iltikzfig{circuits/axioms/belnap/translation/bot-not-lhs}
    }{
        \iltikzfig{circuits/axioms/belnap/translation/bot-x-rhs}
    }{
        \botnoteqn
    }\)
    \\[0.25em]
    \rule{\textwidth}{0.1mm}
    \\[0.5em]
    \(\equationdisplay{
        \iltikzfig{circuits/axioms/fork-lhs}
    }{
        \iltikzfig{circuits/axioms/fork-rhs}
    }{
        \forkeqn
    }\)
    \quad
    \(\equationdisplay{
        \iltikzfig{circuits/axioms/delay-fork-lhs}
    }{
        \iltikzfig{circuits/axioms/delay-fork-rhs}
    }{
        \delayforkeqn
    }\)
    \quad
    \(\equationdisplay{
        \iltikzfig{strings/structure/bialgebra/copy-merge-lhs}
    }{
        \iltikzfig{strings/structure/bialgebra/copy-merge-rhs}
    }{
        \forkjoineqn
    }\)
    \\[0.5em]
    \caption{Set \(\mathcal{X}\) of \emph{explosion equations}}
    \label{fig:explosion-equations}
\end{figure*}

Most of the equations in \(\mathcal{X}\) are well-known; the only interesting
one is \((\belnapexpeqn)\).
This says that we can always `explode' a wire into (trivial) falsy and truthy
subcircuits before joining them back together.
While on its own this may not seem very useful, when combined with the other
equations it sets the stage for translating an entire circuit into an exploded
form.

\begin{lemma}
    The equations in \cref{fig:explosion-equations} are sound.
\end{lemma}
\begin{proof}
    By checking all the inputs.
\end{proof}

\begin{lemma}\label{lem:explode-copy}
    For any combinational Belnap circuit \(
    \iltikzfig{strings/category/f}[box=f,dom=m,cod=n,colour=comb]
    \), the equation \(
    \iltikzfig{strings/structure/cartesian/naturality-copy-lhs}[colour=comb,box=f]
    =
    \iltikzfig{strings/structure/cartesian/naturality-copy-rhs}[colour=comb,box=f]
    \) in \(\ccirc{\belnapsignature} / \mathcal{X}\).
\end{lemma}
\begin{proof}
    This follows for the base cases by applying
    \(\andforkeqn\), \(\orforkeqn\), \(\notforkeqn\) and \(\joinforkeqn\); the
    inductive cases are then trivial.
\end{proof}

\begin{proposition}\label{prop:exploded-belnap}
    Given an essentially combinational Belnap circuit \(
    \iltikzfig{strings/category/f}[box=f,dom=m,cod=1,colour=comb]
    \), there exists essentially combinational Belnap circuits \(
    \iltikzfig{strings/category/f}[box=f_0,dom=4m,cod=1,colour=seq]
    \) and \(
    \iltikzfig{strings/category/f}[box=f_1,dom=4m,cod=1,colour=seq]
    \) containing no \(
    \iltikzfig{circuits/components/gates/not}
    \) or \(
    \iltikzfig{strings/structure/monoid/merge}[colour=comb]
    \) generators, such that \[
        \iltikzfig{strings/category/f}[box=f,dom=m,cod=1,colour=seq]
        =
        \iltikzfig{circuits/axioms/belnap/translation/exploded-form}[box=f,dom=m,cod=1].
    \]
\end{proposition}
\begin{proof}
    If \(\iltikzfig{strings/category/f}[box=f,colour=comb]\) is just
    the identity, then it can be transformed into the desired form with
    \((\belnapexpeqn)\).
    Since \(\iltikzfig{strings/category/f}[box=f,colour=comb]\) has codomain
    \(1\) it cannot be a symmetry.
    For the other generators, \((\belnapexpeqn)\) can first be applied
    to the output wire to create the exploded `skeleton'.
    The \(
    \iltikzfig{circuits/components/gates/and}
    \), \(
    \iltikzfig{circuits/components/gates/or}
    \), and \(
    \iltikzfig{strings/structure/monoid/merge}[colour=comb]
    \) generators can then be pushed inside using the other equations in
    \cref{fig:explosion-equations}.
    A similar principle holds for \(
    \iltikzfig{circuits/components/gates/not}
    \), although as the negation `flips' the translators using
    \((\demorganand)\) and \((\demorganor)\), \((\forkcommeqn)\) must be used to
    restore the correct order of gates, and \((\dneeqn)\) used to eliminate the
    additional \(\iltikzfig{circuits/components/gates/not}\) gates.

    For composition, we assume that the two subcircuits are in the desired form,
    so we have \[
        \iltikzfig{circuits/axioms/belnap/translation/exploded-composition}.
    \]
    By \cref{lem:explode-copy}, the first circuit can be propagated across
    the forks at the start of the second circuit, each of the four
    `translators' has as input a copy of the first circuit.
    Using the same strategy as for the base case the components of the circuit
    can then be propagated across the translators.
    To complete the proof we need to ensure there is exactly one of each
    translator.
    Duplicates are handled by \cref{lem:explode-copy}.
    Translators are flipped between using \(
    \iltikzfig{circuits/components/gates/and}
    \) and \(
    \iltikzfig{circuits/components/gates/or}
    \) when a \(\iltikzfig{circuits/components/gates/not}\) is propagated across
    them, but since the same circuit is applied to each translator the same
    flip will happen in reverse elsewhere.
    Therefore the eventual circuit will be in the correct form.

    For tensor, the circuits can be interleaved using axioms of STMCs.
\end{proof}

This form already looks very similar to a circuit in the image of
\(\mealytofunc\) in that it is the join of two circuits prefixed by
`translators'.
However, these circuits are not in conjunctive or disjunctive normal form.
Fortunately, there are standard equations for translating `ordinary' Boolean
expressions (i.e.\ those without joins) into disjunctive and conjunctive normal
forms.

\begin{definition}
    Let \(\mathcal{F}\) be defined as the set of \emph{normal form equations}
    listed in \cref{fig:normal-form-equations}.
\end{definition}

\begin{figure*}
    \centering
    \(\equationdisplay{
        \iltikzfig{circuits/axioms/belnap/and-associativity-lhs}
    }{
        \iltikzfig{circuits/axioms/belnap/and-associativity-rhs}
    }{\andassoc}\)
    \quad
    \(\equationdisplay{
        \iltikzfig{circuits/axioms/belnap/or-associativity-lhs}
    }{
        \iltikzfig{circuits/axioms/belnap/or-associativity-rhs}
    }{\orassoc}\)
    \\[0.25em]
    \rule{\textwidth}{0.1mm}
    \\[0.7em]
    \(\equationdisplay{
        \iltikzfig{circuits/axioms/belnap/and-or-distributivity-lhs}
    }{
        \iltikzfig{circuits/axioms/belnap/and-or-distributivity-rhs}
    }{\andorddisteqn}\)
    \quad
    \(\equationdisplay{
        \iltikzfig{circuits/axioms/belnap/or-and-distributivity-lhs}
    }{
        \iltikzfig{circuits/axioms/belnap/or-and-distributivity-rhs}
    }{\oranddisteqn}\)
    \\[0.25em]
    \rule{\textwidth}{0.1mm}
    \\[0.7em]
    \(\equationdisplay{
        \iltikzfig{circuits/axioms/belnap/and-commutativity-lhs}
    }{
        \iltikzfig{circuits/axioms/belnap/and-commutativity-rhs}
    }{\andcomm}\)
    \quad
    \(\equationdisplay{
        \iltikzfig{circuits/axioms/belnap/or-commutativity-lhs}
    }{
        \iltikzfig{circuits/axioms/belnap/or-commutativity-rhs}
    }{\orcomm}\)
    \quad
    \(\equationdisplay{
        \iltikzfig{circuits/axioms/belnap/and-idempotent-lhs}
    }{
        \iltikzfig{circuits/axioms/belnap/and-idempotent-rhs}
    }{\andidemeqn}\)
    \\[0.25em]
    \rule{\textwidth}{0.1mm}
    \\[0.7em]
    \(\equationdisplay{
        \iltikzfig{circuits/axioms/belnap/or-idempotent-lhs}
    }{
        \iltikzfig{circuits/axioms/belnap/or-idempotent-rhs}
    }{\oridemeqn}\)
    \quad
    \(\equationdisplay{
        \iltikzfig{circuits/axioms/belnap/translation/and-fork-lhs}
    }{
        \iltikzfig{circuits/axioms/belnap/translation/and-fork-rhs}
    }{\andforkeqn}\)
    \quad
    \(\equationdisplay{
        \iltikzfig{circuits/axioms/belnap/translation/or-fork-lhs}
    }{
        \iltikzfig{circuits/axioms/belnap/translation/or-fork-rhs}
    }{\andforkeqn}\)
    \\[0.25em]
    \rule{\textwidth}{0.1mm}
    \\[0.7em]
    \(\equationdisplay{
        \iltikzfig{strings/structure/comonoid/commutativity-rhs}
    }{
        \iltikzfig{strings/structure/comonoid/commutativity-lhs}
    }{\forkcommeqn}\)
    \quad
    \(\equationdisplay{
        \iltikzfig{strings/structure/comonoid/associativity-lhs}
    }{
        \iltikzfig{strings/structure/comonoid/associativity-rhs}
    }{\forkassoceqn}\)
    \quad
    \(\equationdisplay{
        \iltikzfig{strings/structure/comonoid/unitality-l-lhs}
    }{
        \iltikzfig{strings/structure/comonoid/unitality-l-rhs}
    }{\forkuniteqn}\)
    \\[0.25em]
    \rule{\textwidth}{0.1mm}
    \\[0.7em]
    \(\equationdisplay{
        \iltikzfig{circuits/axioms/belnap/and-annihilator-lhs}
    }{
        \iltikzfig{circuits/axioms/belnap/and-annihilator-rhs}
    }{\andanneqn}\)
    \quad
    \(\equationdisplay{
        \iltikzfig{circuits/axioms/belnap/or-annihilator-lhs}
    }{
        \iltikzfig{circuits/axioms/belnap/or-annihilator-rhs}
    }{\oranneqn}\)
    \\[0.25em]
    \rule{\textwidth}{0.1mm}
    \\[0.5em]
    \(\equationdisplay{
        \iltikzfig{circuits/axioms/belnap/and-identity-lhs}
    }{
        \iltikzfig{circuits/axioms/belnap/and-identity-rhs}
    }{\andideqn}\)
    \quad
    \(\equationdisplay{
        \iltikzfig{circuits/axioms/belnap/or-identity-lhs}
    }{
        \iltikzfig{circuits/axioms/belnap/or-identity-rhs}
    }{\orideqn}\)
    \caption{
        Set \(\mathcal{F}\) of \emph{normal form equations}.
    }
    \label{fig:normal-form-equations}
\end{figure*}


\begin{definition}
    A circuit \(
    \iltikzfig{strings/category/f}[box=f,colour=comb]
    \) is a \emph{conjunction} if it only contains \(
    \iltikzfig{circuits/components/gates/and}
    \) gates and no symmetries, and a \emph{disjunction} if it only contains \(
    \iltikzfig{circuits/components/gates/or}
    \) gates and no symmetries.
    A circuit \(
    \iltikzfig{strings/category/f}[box=g,colour=comb]
    \) is in \emph{disjunctive normal form} if it can be expressed as a
    tensor of conjunctions composed with a disjunction, and is in
    \emph{conjunctive normal form} if it can be expressed as a tensor of
    disjunctions composed with a conjunction.
\end{definition}

\begin{proposition}\label{prop:normal-form}
    Let \(
    \iltikzfig{strings/category/f}[box=f,colour=seq]
    \) be an essentially combinational Belnap circuit containing no \(
    \iltikzfig{circuits/components/gates/not}
    \) or \(
    \iltikzfig{strings/structure/monoid/merge}[colour=comb]
    \) generators.
    Then there exists a circuit \(
    \iltikzfig{strings/category/f}[box=g,colour=comb]
    \) containing only forks and symmetries, a Belnap circuit \(
    \iltikzfig{strings/category/f}[box=h,colour=comb]
    \) in conjunctive normal form such that \(
    \iltikzfig{strings/category/f}[box=f,colour=comb]
    =
    \iltikzfig{circuits/algebraic/belnap-normal}[box1=g,box2=h,val=\listvar{v}]
    \) in \(
    \scirc{\belnapsignature} / \normalisingequations
    \).
    Simlarly, there exists a circuit \(
    \iltikzfig{strings/category/f}[box=g^\prime,colour=comb]
    \) containing only forks and symmetries and a circuit \(
    \iltikzfig{strings/category/f}[box=h^\prime,colour=comb]
    \) in disjunctive normal form such that \(
    \iltikzfig{strings/category/f}[box=f,colour=comb]
    =
    \iltikzfig{circuits/algebraic/belnap-normal}[box1=g^\prime,box2=h^\prime]
    \) in \(
    \scirc{\belnapsignature} / \normalisingequations
    \).
\end{proposition}
\begin{proof}
    As \(
    \iltikzfig{strings/category/f}[box=h,colour=comb]
    \) is essentially combinational, it can be written in the form \(
    \iltikzfig{circuits/synthesis/normalised-function}[box=f^\prime]
    \).
    The combinational circuit \(
    \iltikzfig{strings/category/f}[box=f,colour=comb]
    \) can then be translated into conjunctive normal form by applying the
    well-known procedure for Booleans: i.e.\ propagating any \(\orgate\) gates
    through the \(\andgate\) gates using \(\oranddisteqn\)
    the `leaf' gates using distributivity, and using the other equations to `tidy up' the
    term until it is in the desired normal form.
    The terms of forks and symmetries is then the result of propagating the
    other gates to the left.
\end{proof}

Putting these two results together gives us the desired canonical form theorem.

\begin{theorem}
    Let \(
    \morph{\mealytofunc_\belnap}{\func{\interpretation_{\belnap}}}{\ccirc{\Sigma_\belnap}}
    \) be a map induced by functional completeness of the Belnap interpretation
    in the previous section.
    Given an essentially combinational Belnap circuit \(
    \iltikzfig{strings/category/f}[box=f,colour=seq]
    \), there exists a circuit \(
    \iltikzfig{strings/category/f}[box=g,colour=seq]
    \) in the image of \(\mealytofunc_\belnap\) such that \(
    \iltikzfig{strings/category/f}[box=f,colour=seq]
    =
    \iltikzfig{strings/category/f}[box=g,colour=seq]
    \) in \(\ccirc{\Sigma_\belnap} / \mathcal{X} + \mathcal{F}\).
\end{theorem}
\begin{proof}
    \cref{prop:exploded-belnap} and \cref{prop:normal-form} can be applied in
    sequence to create a circuit of the correct form.
    By using \((\forkuniteqn\), \(\forkassoceqn\) and \(\forkcommeqn)\) the
    initial construct of forks and symmetries can be adjusted until it in the
    image of \(\mealytofunc_\belnap\), however it was initially defined.
\end{proof}
