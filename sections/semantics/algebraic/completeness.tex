\section{Completeness of the algebraic semantics}\label{sec:algebraic-completeness}

It is now possible to collect all the equations together and define a sound and
complete algebraic theory of sequential digital circuits.

\begin{definition}
    For a complete interpretation \(\interpretation\), let
    \(\mce_{\interpretation}\) be \(
    \mealyequations +
    \normalisingequations +
    \encodingequations +
    (\restrictionequation)
    \), and let \(\scircsigmae\) be defined as
    \(\scircsigma / \mce_{\interpretation}\).
\end{definition}

For this to be a \emph{complete} set, we must be able to translate
a circuit \(
\iltikzfig{strings/category/f}[box=f,colour=seq,dom=m,cod=n]
\) into another circuit \(
\iltikzfig{strings/category/f}[box=g,colour=seq,dom=m,cod=n]
\) with the same behaviour by only using these equations.

\begin{theorem}
    For a complete interpretation \(\interpretation\), \(
    \iltikzfig{strings/category/f}[box=f,colour=seq,dom=m,cod=n]
    =
    \iltikzfig{strings/category/f}[box=g,colour=seq,dom=m,cod=n]
    \) in \(\scircsigmae\) if and only if \(
    \circuittostreami[
        \iltikzfig{strings/category/f}[box=f,colour=seq]
    ]
    =
    \circuittostreami[
        \iltikzfig{strings/category/f}[box=g,colour=seq]
    ]
    \).
\end{theorem}
\begin{proof}
    All the equations are sound, so we only need to consider the \(\ifdir\)
    direction.
    By \cref{thm:normalising}, the two circuits can be brought to
    normalised Mealy form using
    \(
    \mealyequations +
    \instantfeedbackeqn +
    \normalisingequations
    \).
    By \cref{def:mealy-to-circuit} and
    \cref{thm:circuit-stream-correspondence} there must exist a circuit \[
        \iltikzfig{strings/category/f}[box=h,colour=seq]
        \coloneqq
        \iltikzfig{circuits/algebraic/encoding}[transition={h_0},output={h_1},state={\listvar{s}}]
    \] such that \(
    \circuittostream[
        \iltikzfig{strings/category/f}[box=f,colour=seq]
    ]{\interpretation}
    =
    \circuittostream[
        \iltikzfig{strings/category/f}[box=h,colour=seq]
    ]{\interpretation}
    =
    \circuittostream[
        \iltikzfig{strings/category/f}[box=g,colour=seq]
    ]{\interpretation}
    \).
    The state words of \(
    \iltikzfig{strings/category/f}[box=h,colour=seq]
    \) are in the image of a Mealy encoding \(\gamma_\leq\); applying the
    encoding equation with \(\gamma_\leq\) to the normalised Mealy forms
    obtained above therefore produce circuits with the same state sets as \(
    \iltikzfig{strings/category/f}[box=h,colour=seq]
    \).
    To finish the proof, restriction equations can be used to bring the circuit
    to the desired form.
\end{proof}

As always, the soundness and completeness of the algebraic semantics means we
can establish another isomorphism of PROPs.

\begin{corollary}
    \(\scircsigmai \cong \scircsigmae\).
\end{corollary}

One might wonder how this improves on the operational approach, as the
normal form is quite complicated.
The beauty of the \emph{algebraic} semantics is that equations can be proven
as lemmas and used in the future as shortcuts; in time, the algebraicist will
build up a repertoire of equations and use them to bend circuits to
their will.