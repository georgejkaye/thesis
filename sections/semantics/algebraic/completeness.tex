\section{Completeness of the algebraic semantics}\label{sec:algebraic-completeness}

It is now possible to collect all the equations together and define a sound and
complete algebraic theory of sequential digital circuits.

\begin{definition}
    \index{scircsigmag@\(\scircsigmage\)}
    \nomenclature{\(\mce_\interpretation\)}{complete set of equations on circuits}
    \nomenclature{\(\scircsigmae\)}{PROP of sequential circuits quotiented by equations}
    For a complete interpretation \(\interpretation\), let
    \(\mce_{\interpretation}\) be \(
    \mealyequations +
    \normalisingequations +
    \encodingequations +
    (\restrictionequation)
    \), and let \(\scircsigmae\) be defined as
    \(\scircsigma / \mce_{\interpretation}\).
\end{definition}

For this to be a \emph{complete} set, we must be able to translate
a circuit \(
\iltikzfig{strings/category/f}[box=f,colour=seq,dom=m,cod=n]
\) into another circuit \(
\iltikzfig{strings/category/f}[box=g,colour=seq,dom=m,cod=n]
\) with the same behaviour by only using these equations.

\begin{theorem}
    For a complete interpretation \(\interpretation\), \(
    \iltikzfig{strings/category/f}[box=f,colour=seq,dom=m,cod=n]
    =
    \iltikzfig{strings/category/f}[box=g,colour=seq,dom=m,cod=n]
    \) in \(\scircsigmae\) if and only if \(
    \circuittostreami[
        \iltikzfig{strings/category/f}[box=f,colour=seq]
    ]
    =
    \circuittostreami[
        \iltikzfig{strings/category/f}[box=g,colour=seq]
    ]
    \).
\end{theorem}
\begin{proof}
    All the equations are sound, so we only need to consider the \(\ifdir\)
    direction.
    Using \cref{thm:normalising}, the circuits \(
    \iltikzfig{strings/category/f}[box=f,colour=seq]
    \) and \(
    \iltikzfig{strings/category/f}[box=g,colour=seq]
    \) can be brought to Mealy form, so we have that \(
    \circuittostreami[
        \iltikzfig{circuits/productivity/mealy-form-wide}[core=\mealytofunc[\hat{f}],state=\listvar{s}, colour=seq]
    ]
    =
    \circuittostreami[
        \iltikzfig{circuits/productivity/mealy-form-wide}[core=\mealytofunc[\hat{g}],state=\listvar{t}, colour=seq]
    ]
    \).
    This induces Mealy machines \((S_{\hat{f}, \listvar{s}}, \hat{f})\) and
    \((S_{\hat{g}, \listvar{t}}, \hat{g})\).
    As their stream functions are equal, there are Mealy homomorphisms
    \(\morph{\phi}{S_{\hat{f}, \listvar{s}}}{S_{\hat{g}, \listvar{t}}}\) and
    \(\morph{\psi}{S_{\hat{g}, \listvar{t}}}{S_{\hat{f}, \listvar{s}}}\) and
    subsequently the composite of these homomorphisms is also a Mealy
    homomorphism; these will act as a decoder-encoder pair.

    Using the encoding equation, we have by \cref{thm:encoding} that \[
        \iltikzfig{circuits/productivity/mealy-form-wide}[core=\mealytofunc[\hat{f}],state=\listvar{s}, colour=seq]
        =
        \iltikzfig{circuits/algebraic/state-encoded-theorem}.
    \]
    The circuit \(
    \iltikzfig{circuits/algebraic/state-encoded-theorem-core}
    \) is a composition of normalised circuits, so it is essentially
    combinational; when restricted to the set \(S_{\hat{g}, \listvar{t}}\) its
    truth table is the same as that of \(
    \iltikzfig{strings/category/f-wide}[box=\mealytofunc[\hat{g}], colour=seq]
    \), as the encoder-decoder pair were defined precisely as the Mealy
    homomorphisms that translate between the two Mealy machines.
    Using the normalisation equations again, the encoded circuit can be
    brought into normalised Mealy form.
    Finally, the restriction equations can be used to translate from \(
    \iltikzfig{circuits/algebraic/state-encoded-theorem}
    \) into \(
    \circuittostreami[
        \iltikzfig{circuits/productivity/mealy-form-wide}[core=\mealytofunc[\hat{g}],state=\listvar{t}, colour=seq]
    ]
    \).
\end{proof}

As always, the soundness and completeness of the algebraic semantics means we
can establish another isomorphism of PROPs.

\begin{corollary}
    \(\scircsigmai \cong \scircsigmae\).
\end{corollary}

One might wonder how this improves on the operational approach, as the
normal form is quite complicated.
The beauty of the \emph{algebraic} semantics is that equations can be proven
as lemmas and used in the future as shortcuts; in time, the algebraicist will
build up a repertoire of equations and use them to bend circuits to
their will.