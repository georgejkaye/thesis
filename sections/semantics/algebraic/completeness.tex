\section{Completeness of the algebraic semantics}\label{sec:algebraic-completeness}

It is now possible to collect all the equations together and define a sound and
complete algebraic theory of sequential digital circuits.
We start by defining a category of syntactic circuits quotiented by all the
equations we have seen so far.

\begin{definition}
    For an interpretation \(\interpretation\), let
    \(\mce_{\interpretation}\) be \(
    \mealyequations +
    \normalisingequations +
    \encodingequations +
    (\restrictionequation)
    \), and let \(\scircsigmae\) be defined as
    \(\scircsigma / \mce_{\interpretation}\).
\end{definition}

\begin{theorem}
    For a functionally complete interpretation \(\interpretation\), \(
    \iltikzfig{strings/category/f}[box=f,colour=seq,dom=m,cod=n]
    =
    \iltikzfig{strings/category/f}[box=g,colour=seq,dom=m,cod=n]
    \) in \(\scircsigmae\) if and only if \(
    \circuittostreami[
        \iltikzfig{strings/category/f}[box=f,colour=seq]
    ]
    =
    \circuittostreami[
        \iltikzfig{strings/category/f}[box=g,colour=seq]
    ]
    \).
\end{theorem}
\begin{proof}
    All the equations are sound, so we only need to consider the \(\ifdir\)
    direction.

    By \cref{thm:normalising}, the two circuits can be brought to
    normalised Mealy form using
    \(
    \mealyequations +
    \instantfeedbackeqn +
    \normalisingequations
    \).
    By \cref{def:mealy-to-circuit} and
    \cref{thm:circuit-stream-correspondence} there must exist a circuit \(
    \iltikzfig{strings/category/f}[box=h,colour=seq]
    \coloneqq
    \iltikzfig{circuits/algebraic/encoding}[transition={h_0},output={h_1},state={\listvar{s}}]
    \) such that \(
    \circuittostream[
        \iltikzfig{strings/category/f}[box=f,colour=seq]
    ]{\interpretation}
    =
    \circuittostream[
        \iltikzfig{strings/category/f}[box=h,colour=seq]
    ]{\interpretation}
    =
    \circuittostream[
        \iltikzfig{strings/category/f}[box=g,colour=seq]
    ]{\interpretation}
    \).
    The circuit \(
    \iltikzfig{strings/category/f}[box=h,colour=seq]
    \) is encoded such that the state words are in the image of a Mealy encoding
    \(\gamma_\leq\); applying the encoding equation with \(\gamma_\leq\) to the
    normalised Mealy forms obtained above will yield the circuit \(
    \iltikzfig{strings/category/f}[box=h,colour=seq]
    \).
\end{proof}

As always, the soundness and completeness of the algebraic semantics means we
can establish another isomorphism of PROPs.

\begin{corollary}
    \(\scircsigmai \cong \scircsigmae\).
\end{corollary}

This means that given a circuit \(
\iltikzfig{strings/category/f}[box=f,colour=seq,dom=m,cod=n]
\) we know that we can translate it into another circuit \(
\iltikzfig{strings/category/f}[box=g,colour=seq,dom=m,cod=n]
\) with the same behaviour by only using equations in \(\mce_\interpretation\).

Of course, the procedure of using the normalisation equations to translate a
circuit into normalised Mealy form before using the encoding equation (possibly
multiple times depending on how lucky one gets with their orderings) may be
tedious; one might wonder how this is beneficial to the operational approach in
the previous chapter.
The beauty of the \emph{algebraic} semantics is that we \emph{don't} need to
do this every time.
Equations can be proven as lemmas and then used repeatedly in the future as
`shortcuts', possibly saving many reasoning steps.
In time, the algebraicist will build up a powerful repertoire of equations and
wield them to bend circuits to their will.