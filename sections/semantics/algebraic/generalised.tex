\section{Algebraic semantics for generalised circuits}

When it comes to lifting the algebraic semantics to the generalised

\begin{definition}
    Let the set \(\mealyequations^+\) be defined as \(\mealyequations\) but
    with equations \((\monoidunitleqn)\), \((\monoidunitreqn)\), and
    \((\bottomdelayeqn)\) for wires of each with \(n \in \natplus\).
\end{definition}

Since the set of normalising equations \(\normalisingequations\) is determined
by the interpretation, we do not need to do anything there.
The encoding equations need to be extended to act on all wire widths, and we
need to be able to handle encoders that contain bundlers.

\begin{definition}
    Let the set \(\encodingequations^+\) be defined as \(\encodingequations\)
    but with all equations adjusted to operate on wires of all widths
    \(n \in \natplus\), and with the addition of equations \[
        \equationdisplay{
            \iltikzfig{circuits/axioms/unbundle-lhs}
        }{
            \iltikzfig{circuits/axioms/unbundle-rhs}
        }{
            \spliteqn
        }
        \quad
        \equationdisplay{
            \iltikzfig{circuits/axioms/bundle-lhs}
        }{
            \iltikzfig{circuits/axioms/bundle-rhs}
        }{
            \combineeqn
        }
    \] on bundlers.
\end{definition}

Finally the restriction schema also needs to operate on arbitrary-width wires.

\begin{definition}
    Let \((\mathsf{Res}^+)\) be defined as \((\mathsf{Res})\) but extended to
    operate on wires of widths \(n \in \natplus\).
\end{definition}

By putting all these equations together we obtain a set of equations for
generalised circuits.

\begin{definition}
    For a generalised interpretation \(\interpretation^+\), let \(
    \mathcal{E}^+_{\mathcal{I}}
    \coloneqq
    \mealyequations^+ +
    \mathcal{N}^+_{\mathcal{I}} +
    \encodingequations^+ +
    (\mathsf{Res^+})
    \), and let \(\scircsigmage\) be defined as
    \(\scircsigmag / \mathcal{E}^+_{\mathcal{I}}\).
\end{definition}

This set of equations is sound and complete.

\begin{theorem}
    For a functionally complete generalised interpretation \(\interpretation\),
    \(
    \iltikzfig{strings/category/f}[box=f,colour=seq,dom=\listvar{m},cod=\listvar{n}]
    =
    \iltikzfig{strings/category/f}[box=g,colour=seq,dom=\listvar{m},cod=\listvar{n}]
    \) in \(\scircsigmage\) if and only if \(
    \circuittostreamig[
        \iltikzfig{strings/category/f}[box=f,colour=seq]
    ]
    =
    \circuittostreamig[
        \iltikzfig{strings/category/f}[box=g,colour=seq]
    ]
    \).
\end{theorem}

Subsequently we obtain another isomorphism of categories.

\begin{corollary}
    \(\scircsigmaig \cong \scircsigmage\).
\end{corollary}