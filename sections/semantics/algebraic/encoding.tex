\section{Encoding equations}\label{sec:encoding}

A circuit in Mealy form is a syntactic representation of a Mealy machine: the
combinational core is the Mealy function, and the registers are the initial
state.
It is important to determine the states that the circuit might assume, as these
determine whether or not an equation is valid.

\begin{definition}[States]
    \index{states}
    \nomenclature{\(S_{f,\listvar{s}}\)}{set of circuit states of function \(f\) from initial state \(\listvar{s}\)}
    Let \(\morph{f}{\valuetuple{x+m}}{\valuetuple{x+n}}\) be a
    monotone function and let \(\listvar{s} \in  \valuetuple{x}\) be a state.
    Then the \emph{states of \(f\) from \(\listvar{s}\)}, denoted
    \(S_{f,\listvar{s}}\), is the smallest set containing \(\listvar{s}\) and
    closed under \(
    \listvar{r}
    \mapsto
    \proj{0}\mleft(f(\listvar{r},\listvar{v})\mright)
    \) for any \(\listvar{v} \in \valuetuple{m}\).
\end{definition}

\begin{example}\label{ex:circuit-states}
    Consider the circuit \(
    \iltikzfig{circuits/examples/bottrue/circuit}
    \).
    The semantics of the combinational core are clearly
    \((s, r) \mapsto \left(s \lor r, s, s \lor r\right)\), where the first two characters are the
    next state and the third is the output.
    The initial state is \(\bot\belnaptrue\), so the subsequent states are
    \((\bot \lor \belnaptrue, \bot) = (\belnaptrue,\bot)\) and
    \((\belnaptrue \lor \bot, \belnaptrue) = (\belnaptrue, \belnaptrue)\).
    As \((\belnaptrue \lor \belnaptrue, \belnaptrue) = (\belnaptrue, \belnaptrue)\),
    there are no more circuit states and the complete set is
    \(\{(\bot, \belnaptrue),(\belnaptrue,\bot),(\belnaptrue,\belnaptrue)\}\).

    Note that as the output of the circuit is computed as \(s \lor r\), for each
    circuit state the output is \(\belnaptrue\).
    This means that the circuit is denotationally equivalent to \(
    \iltikzfig{circuits/components/waveforms/infinite-register}[val=\belnaptrue]
    \), but this circuit only has a single state \(\belnaptrue\).
\end{example}

We need to \emph{encode} the states of one circuit as another; we have already
encountered this notion using \emph{Mealy homomorphisms}
(\cref{def:mealy-homomorphism});\index{Mealy!homomorphism}
functions between the state sets that preserve transitions and outputs.
While two `inverse' homomorphisms may not be isomorphisms, the round
trip will always map to a state with the same behaviour.

\begin{lemma}
    Given two Mealy homomorphisms \(\morph{h}{(S,f)}{(T,g)}\) and
    \(\morph{h^\prime}{(T,g)}{(S,f)}\), any state \(s \in S\) and input
    \(a \in A\), \(
    \mealyfunctiontransition{f}(s, a)
    =
    \mealyfunctiontransition{f}(h^\prime(h(s)), a)
    \).
\end{lemma}
\begin{proof}
    Immediate as Mealy homomorphisms preserve outputs.
\end{proof}

We will use Mealy homomorphisms as circuits to encode state sets; this means we
need to ensure the encoders and decoders are monotone.

\begin{lemma}
    For partial orders \(S\) and \(T\) and monotone Mealy coalgebra
    \((S,f)\) and \((T,g)\), any Mealy homomorphism \(\morph{h}{(S,f)}{(T,g)}\)
    is monotone.
\end{lemma}
\begin{proof}
    In a monotone Mealy coalgebra, the functions \(f\) and \(g\) are monotone,
    and for \(h\) to be a Mealy homomorphism, \(
    \mealyfunctionoutput{f}(s)
    =
    \mealyfunctionoutput{g}(h(s))
    \).
    For states \(s,r \in S\) we have \(
    \mealyfunctionoutput{g}(h(s), a)
    =
    \mealyfunctionoutput{f}(s, a)
    \leq
    \mealyfunctionoutput{f}(r, a)
    =
    \mealyfunctionoutput{g}(h(r), a)
    \).
    This means that the function \(
    s \mapsto \mealyfunctionoutput{g}(h(s), a)
    \) is monotone; as \(\mealyfunctionoutput{g}\) is monotone, \(h\) must
    also be monotone.
\end{proof}

For two circuits \(
\iltikzfig{circuits/productivity/mealy-form-wide}[core=\mealytofunc[f],state=\listvar{s},colour=seq,dom=m,cod=n,delay=x]
\) and \(
\iltikzfig{circuits/productivity/mealy-form-wide}[core=\mealytofunc[g],state=\listvar{t},colour=seq,dom=m,cod=n,delay=y]
\), the encoders and decoders will be homomorphisms
\((S_{f,\listvar{s}},f) \to (S_{g,\listvar{t}},g)\)
and
\((S_{g,\listvar{t}},g) \to (S_{f,\listvar{s}},f)\).
These are homomorphisms on the subset of states that a circuit
can assume, \emph{not} the entire set of words that can fit into the state.
This means that encoding and decoding circuits cannot be inserted arbitrarily
but only in certain contexts.

\begin{proposition}[Encoding equation]\label{prop:encoding-equation}
    \index{encoding equation}
    For a normalised circuit \(
    \iltikzfig{strings/category/f-2-2-wide}[box=\mealytofunc[f],colour=seq,dom1=x,dom2=m,cod1=x,cod2=y]
    \) and \(\listvar{s} \in \valuetuple{x}\), let
    \(\morph{\mathsf{enc}}{S_{f,\listvar{s}}}{\valuetuple{y}}\) and
    \(\morph{\mathsf{dec}}{\valuetuple{y}}{S_{f, \listvar{s}}}\) be functions
    such that \(
    \morph{\mathsf{dec} \circ \mathsf{enc}}{S_{f, \listvar{s}}}{S_{f, \listvar{s}}}
    \) is a Mealy homomorphism.
    Then the \emph{encoding equation} \((\encodingequation)\) in
    \cref{fig:encoding-equation} is sound, where
    \(\mathsf{enc}_\mathsf{m}\) and \(\mathsf{dec}_\mathsf{m}\) are monotone
    completions as defined in \cref{def:monotone-completion}.
\end{proposition}
\begin{proof}
    Let \(g\) be defined as the map \(\listvar{r} \mapsto
    \circuittostreami[\iltikzfig{circuits/algebraic/state-encoding}[core=\mealytofunc[f],delay=x,state=\listvar{r}]]
    \); by \cref{prop:mealy-form-image} we know that \(
    \mealyoutput{g(\listvar{t})}{\listvar{v}}
    =
    \proj{1}(f(\mathsf{dec}(\mathsf{enc}(\listvar{t})), \listvar{v}))
    \) and \(
    \mealytransition{g(\listvar{t})}{\listvar{v}}
    =
    g(\proj{0}(f(\mathsf{dec}(\mathsf{enc}(\listvar{t})), \listvar{v})))
    \).
    As \(\mathsf{dec} \circ \mathsf{enc}\) is a Mealy homomorphism, for
    \(\listvar{t} \in S_{f, \listvar{s}}\) we have that \(
    \mealyoutput{g(\listvar{t})}{\listvar{v}}
    =
    \proj{1}(f(\listvar{t}), \listvar{v})
    \) and that \(
    \mealytransition{g(\listvar{t})}{\listvar{v}}
    \) shares outputs and transitions with \(
    g(\proj{0}(f(\listvar{t})), \listvar{v})
    \).
    As \(
    \iltikzfig{circuits/algebraic/state-encoding}[core=\mealytofunc[f],delay=x,state=\listvar{s}]
    \coloneqq
    g(\listvar{s})
    \) and \(\listvar{s} \in S_{f,\listvar{s}}\),
    every subsequent stream derivative will also be of the form
    \(g(\listvar{t})\) where \(\listvar{t} \in S_{f,\listvar{s}}\), so the
    equation is sound.
\end{proof}

\begin{remark}
    The encoding equation is an equation \emph{schema}: this is required because
    the width of a circuit state can be arbitrarily large, and each extra bit
    adds a whole new set of Mealy homomorphisms to consider.
\end{remark}

\begin{figure}
    \centering
    \(
    \equationdisplay{
        \iltikzfig{circuits/productivity/mealy-form-wide}[core=\mealytofunc[f],delay=x, colour=seq]
    }{
        \iltikzfig{circuits/algebraic/state-encoding}[core=\mealytofunc[f],delay=x]
    }{
        \encodingequation
    }
    \)
    \\[0.25em]
    \rule{\textwidth}{0.1mm}
    \\[0.5em]
    \(
    \equationdisplay{
        \iltikzfig{circuits/axioms/gate-lhs}[gate=p]
    }{
        \iltikzfig{circuits/axioms/gate-rhs}[gate=p]
    }{
        \gateeqn
    }
    \quad
    \equationdisplay{
        \iltikzfig{circuits/axioms/fork-lhs}
    }{
        \iltikzfig{circuits/axioms/fork-rhs}
    }{
        \forkeqn
    }
    \quad
    \equationdisplay{
        \iltikzfig{circuits/axioms/join-lhs}
    }{
        \iltikzfig{circuits/axioms/join-rhs}
    }{
        \joineqn
    }
    \)
    \\[0.25em]
    \rule{\textwidth}{0.1mm}
    \\[0.5em]
    \(
    \equationdisplay{
        \iltikzfig{circuits/axioms/stub-lhs}
    }{
        \iltikzfig{strings/monoidal/empty}
    }{
        \stubeqn
    }
    \quad
    \equationdisplay{
        \iltikzfig{circuits/axioms/delay-fork-lhs}
    }{
        \iltikzfig{circuits/axioms/delay-fork-rhs}
    }{
        \delayforkeqn
    }
    \quad
    \equationdisplay{
        \iltikzfig{circuits/axioms/bottom-delay-lhs}
    }{
        \iltikzfig{circuits/axioms/bottom-delay-rhs}
    }{
        \bottomdelayeqn
    }
    \quad
    \equationdisplay{
        \iltikzfig{circuits/axioms/streaming-lhs}
    }{
        \iltikzfig{circuits/axioms/streaming-rhs}[gate=p]
    }{
        \streamingeqn
    }
    \)
    \\[0.25em]
    \rule{\textwidth}{0.1mm}
    \\[0.5em]
    \(
    \equationdisplay{
        \iltikzfig{strings/structure/comonoid/unitality-l-lhs}
    }{
        \iltikzfig{strings/structure/comonoid/unitality-l-rhs}
    }{
        \comonoiduniteqnletter
    }
    \quad
    \equationdisplay{
        \iltikzfig{strings/structure/monoid/associativity-lhs}
    }{
        \iltikzfig{strings/structure/monoid/associativity-rhs}
    }{
        \monoidassoceqnletter
    }
    \quad
    \equationdisplay{
        \iltikzfig{strings/structure/monoid/commutativity-lhs}
    }{
        \iltikzfig{strings/structure/monoid/commutativity-rhs}
    }{
        \monoidcommeqnletter
    }
    \quad
    \equationdisplay{
        \iltikzfig{strings/structure/bialgebra/merge-copy-lhs}
    }{
        \iltikzfig{strings/structure/bialgebra/merge-copy-rhs}
    }{
        \joinforkeqn
    }
    \)
    \caption{
        Set \(\encodingequations\) of equations for encoding circuit states
    }
    \label{fig:encoding-equation}
\end{figure}

The encoding equation only inserts encoder circuits; to actually change the
state we need some more equations.

\begin{lemma}
    The equations on the bottom four rows of \cref{fig:encoding-equation} are
    sound.
\end{lemma}
\begin{proof}
    It is a straightforward exercise to compare the stream functions.
\end{proof}

To show the final result we must prove some lemmas; first we show how we can
`pump' a value out of an infinite waveform.

\begin{lemma}\label{lem:unroll-waveform}
    For \(v \in \values\), \(
    \iltikzfig{circuits/algebraic/unroll-waveform/step-0}[value=v]
    =
    \iltikzfig{circuits/algebraic/unroll-waveform/step-6}[value=v]
    \) using the encoding equations.
\end{lemma}
\begin{proof}
    The proof is straightforward and is illustrated in
    \cref{fig:unroll-waveform}.
\end{proof}
%
\begin{figure}
    \centering
    \begin{gather*}
        \scalebox{0.9}{\iltikzfig{circuits/algebraic/unroll-waveform/step-0}[value=v]}
        \coloneqq
        \scalebox{0.9}{\iltikzfig{circuits/algebraic/unroll-waveform/step-1}[value=v]}
        \eqaxioms[(\joinforkeqn)]
        \scalebox{0.9}{\iltikzfig{circuits/algebraic/unroll-waveform/step-2}[value=v]}
        \eqaxioms[(\forkeqn)]
        \\[1em]
        \scalebox{0.9}{\iltikzfig{circuits/algebraic/unroll-waveform/step-3}[value=v]}
        \eqaxioms[(\delayforkeqn)]
        \scalebox{0.9}{\iltikzfig{circuits/algebraic/unroll-waveform/step-4}[value=v]}
        \coloneqq
        \scalebox{0.9}{\iltikzfig{circuits/algebraic/unroll-waveform/step-5}[value=v]}
        =
        \scalebox{0.9}{\iltikzfig{circuits/algebraic/unroll-waveform/step-6}[value=v]}
    \end{gather*}
    \caption{Proof of \cref{lem:unroll-waveform}}
    \label{fig:unroll-waveform}
\end{figure}

The next lemma shows how the familiar `streaming' rule from the operational
semantics can be derived equationally.

\begin{lemma}\label{lem:generalised-streaming}
    For a combinational circuit \(
    \iltikzfig{strings/category/f}[box=f,colour=comb]
    \), \(
    \iltikzfig{circuits/axioms/generalised-streaming-lhs}[box=f]
    =
    \iltikzfig{circuits/axioms/generalised-streaming-rhs}[box=f]
    \) by the encoding equations.
\end{lemma}
\begin{proof}
    This is by induction on the structure of \(
    \iltikzfig{strings/category/f}[box=f,colour=comb]
    \).
    The case for the primitive is immediate by \((\streamingeqn)\).
    For \(\iltikzfig{strings/structure/comonoid/copy}[colour=comb]\) we have
    that \[
        \iltikzfig{circuits/productivity/generalised-streaming/fork-step-0}
        \eqaxioms[(\joinforkeqn)]
        \iltikzfig{circuits/productivity/generalised-streaming/fork-step-1}
        \eqaxioms[(\delayforkeqn)]
        \iltikzfig{circuits/productivity/generalised-streaming/fork-step-2}
    \]
    The proof for \(\iltikzfig{strings/structure/monoid/merge}[colour=comb]\) is
    illustrated in \cref{fig:generalised-streaming-join}.
    The case for \(\iltikzfig{strings/structure/comonoid/discard}[colour=comb]\) is
    trivial, and the case for \(\iltikzfig{strings/structure/monoid/init}[colour=comb]\)
    follows by \((\comonoiduniteqnletter)\) and \((\bottomdelayeqn)\).
    The cases for \(\iltikzfig{strings/category/identity}[colour=comb]\) and
    \(\iltikzfig{strings/symmetric/symmetry}[colour=comb]\) follow by axioms of STMCs.
    Since the underlying circuit is combinational, for the inductive cases we just
    need to check composition and tensor, which are trivial.
\end{proof}
%
\begin{figure}
    \begin{gather*}
        \iltikzfig{circuits/productivity/generalised-streaming/join-step-2}
        \eqaxioms[(\monoidassoceqnletter)]
        \iltikzfig{circuits/productivity/generalised-streaming/join-step-3}
        \eqaxioms[(\monoidassoceqnletter)]
        \iltikzfig{circuits/productivity/generalised-streaming/join-step-4}
        \eqaxioms[(\monoidcommeqnletter)]
        \\[1em]
        \iltikzfig{circuits/productivity/generalised-streaming/join-step-5}
        =
        \iltikzfig{circuits/productivity/generalised-streaming/join-step-6}
        \eqaxioms[(\monoidassoceqnletter)]
        \iltikzfig{circuits/productivity/generalised-streaming/join-step-7}
        \eqaxioms[(\monoidassoceqnletter)]
        \\[1em]
        \iltikzfig{circuits/productivity/generalised-streaming/join-step-8}
        \eqaxioms[(\monoidassoceqnletter)]
        \iltikzfig{circuits/productivity/generalised-streaming/join-step-9}
    \end{gather*}
    \caption{Proof of \cref{lem:generalised-streaming} for the join case}
    \label{fig:generalised-streaming-join}
\end{figure}

We next show how the encoding equations can be used to translate combinational
circuits with inputs into values.

\begin{lemma}\label{lem:combinational-circuit-inputs}
    Let \(\iltikzfig{strings/category/f}[box=f,colour=comb,dom=m,cod=n]\) be a
    combinational circuit such that \(
    \circuittofunci[\iltikzfig{strings/category/f}[box=f,colour=comb]]
    =
    g
    \).
    Then \(
    \iltikzfig{circuits/components/circuits/f-applied}[box=f,colour=comb]
    =
    \iltikzfig{circuits/components/values/vs-even-longer}[val=g(\listvar{v})]
    \) by the encoding equations.
\end{lemma}
\begin{proof}
    For the same reasoning as \cref{lem:reduce-core-terminating}, the
    \((\gateeqn)\), \((\forkeqn)\), \((\joineqn)\) and \((\stubeqn)\) equations
    can be used to show that there exists \(\listvar{w} \in \valuetuple{n}\)
    such that \(
    \iltikzfig{strings/category/f}[box=f,colour=comb,dom=m,cod=n]
    =
    \iltikzfig{circuits/components/values/vs}[val=\listvar{w}]
    \).

    Now we need to show that \(
    \circuittostreami[
        \iltikzfig{circuits/components/circuits/f-applied}[box=f,colour=comb]
    ]
    =
    \circuittostreami[
        \iltikzfig{circuits/components/values/vs-even-longer}[val=g(\listvar{v})]
    ]
    \).
    By functoriality of \(\circuittostreami\), \(
    \circuittostreami[
        \iltikzfig{circuits/components/circuits/f-applied}[box=f,colour=comb]
    ]
    =
    \circuittostreami[
        \iltikzfig{circuits/components/values/vs}[val=\listvar{v}]
    ] \seq
    \circuittostreami[
        \iltikzfig{strings/category/f}[box=f,colour=comb]
    ]
    \).
    By \cref{lem:sequential-combinational-semantics} we know that \(
    \circuittostreami[
        \iltikzfig{strings/category/f}[box=f,colour=comb]
    ](\sigma)(i) = \circuittofunci[
        \iltikzfig{strings/category/f}[box=f,colour=comb]
    ] = g(\sigma)(i)\) for all \(\sigma \in \valuetuplestream{m}\) and
    \(i \in \nat\).
    Since \(\circuittostreami[
        \iltikzfig{circuits/components/values/vs}[val=\listvar{v}]
    ] = \listvar{v} \streamcons \bot \streamcons \bot \streamcons \dots\), we
    have that \(
    \circuittostreami[
        \iltikzfig{circuits/components/circuits/f-applied}[box=f,colour=comb]
    ]
    =
    g(\listvar{v}) \streamcons \bot \streamcons \bot \streamcons \dots
    \), which is the interpretation of \(
    \iltikzfig{circuits/components/values/vs-even-longer}[val=g(\listvar{v})]
    \).
    As the equations are sound they must preserve the stream semantics, so
    \(\listvar{w} = g(\listvar{v})\).
\end{proof}

Finally, we use the above lemma to show how values can be applied to
\emph{essentially} combinational circuits.

\begin{lemma}\label{lem:essentially-combinational-applied}
    Let \(\morph{f}{\valuetuple{m}}{\valuetuple{n}}\) be a monotone function
    such that \(
    \iltikzfig{strings/category/f-wide}[box=\mealytofunc[f],colour=seq]
    \coloneqq
    \iltikzfig{circuits/synthesis/normalised-function}[box=g]
    \).
    Then \(
    \iltikzfig{circuits/algebraic/normalised-inputs}[box=g,input1=\listvar{v},input2=\listvar{w}]
    =
    \iltikzfig{circuits/components/values/vs-even-longer}[val=f(\listvar{w})]
    \).
\end{lemma}
\begin{proof}
    Let \(h \coloneqq \circuittofunci[
        \iltikzfig{strings/category/f-2-1}[box=g,colour=comb]
    ]\); using \cref{lem:combinational-circuit-inputs}, we have that \(
    \iltikzfig{circuits/algebraic/normalised-inputs}[box=g,input1=\listvar{v},input2=\listvar{w}]
    =
    \iltikzfig{circuits/components/values/vs-really-long}[val={h(\listvar{v},\listvar{w})}]
    \).
    So we must show that \(f(\listvar{w}) = h(\listvar{v}, \listvar{w})\).
    \begin{align*}
        f(\listvar{w})
         & =
        \circuittostreami[
            \iltikzfig{strings/category/f-wide}[box=\mealytofunc[f],colour=seq]
        ](\listvar{w} \streamcons \bot^\omega)(0)
         &
        \text{\cref{def:functional-completeness}}
        \\
         & \coloneqq
        \circuittostreami[
            \iltikzfig{circuits/synthesis/normalised-function}[box=g]
        ](\listvar{w} \streamcons \bot^\omega)(0)
        \\
         & =
        \circuittostreami[
            \iltikzfig{strings/category/f-2-1}[box=g,colour=comb]
        ](\listvar{v}^\omega, \listvar{w} \streamcons \bot^\omega)(0)
        \\
         & =
        \circuittofunci[
            \iltikzfig{strings/category/f-2-1}[box=g,colour=comb]
        ](\listvar{v}, \listvar{w})
         &
        \text{\cref{lem:sequential-combinational-semantics}}
        \\
         & =
        h(\listvar{v}, \listvar{w})
    \end{align*}
    This completes the proof.
\end{proof}

With these lemmas in our toolkit, we can now show that the encoding equations
allow us to translate a circuit into one with an encoded state, and therefore
translate between the state sets of any two denotationally equivalent circuits.

\begin{theorem}\label{thm:encoding}
    For a circuit \(
    \iltikzfig{strings/category/f-2-2-wide}[box=\mealytofunc[f],colour=seq,dom1=x,dom2=m,cod1=x,cod2=y]
    \) and initial state \(\listvar{s} \in \valuetuple{x}\), the
    equation \(
    \iltikzfig{circuits/productivity/mealy-form-wide}[core=\mealytofunc[f], colour=seq]
    =
    \iltikzfig{circuits/algebraic/state-encoded}[core=\mealytofunc[f],state=\listvar{s}]
    \) is derivable by the equations in \(\encodingequations\).
\end{theorem}
\begin{proof}
    We have that \(
    \iltikzfig{circuits/productivity/mealy-form-wide}[core=\mealytofunc[f],delay=x, colour=seq]
    =
    \iltikzfig{circuits/algebraic/state-encoding}[core=\mealytofunc[f],delay=x,state=\listvar{s}]
    \) by the \((\encodingequation)\) equation; we need to `push' the encoder \(
    \iltikzfig{circuits/algebraic/encoder}
    \) through the state.
    Although the encoder is sequential, by the definition of \(\mealytofunc\),
    it must be of the form \(
    \iltikzfig{circuits/synthesis/normalised-function}[box=g]
    \) by definition of complete interpretations.
    This means we have
    \begin{align*}
        \iltikzfig{circuits/algebraic/encoding-state/step-0a}[box=g]
         & \coloneqq
        \iltikzfig{circuits/algebraic/encoding-state/step-0}[box=g]
        \\[1em]
         & =
        \iltikzfig{circuits/algebraic/encoding-state/step-1}[box=g]
         &
        \text{\cref{lem:unroll-waveform}}
        \\[1em]
         & =
        \iltikzfig{circuits/algebraic/encoding-state/step-2}[box=g]
         &
        \text{\cref{lem:generalised-streaming}}
        \\[1em]
         & =
        \iltikzfig{circuits/algebraic/encoding-state/step-3}[box=g]
         &
        \text{\cref{lem:essentially-combinational-applied}}
        \\[1em]&\coloneqq
        \iltikzfig{circuits/algebraic/encoding-state/step-4}[box=g]
        \\[1em]
         & \coloneqq\iltikzfig{circuits/algebraic/encoding-state/step-5}[box=g]
    \end{align*}
    The proof is completed by sliding the encoder around the trace.
\end{proof}

With the right encoders, the initial state of a circuit can be translated into
a different word, giving us a new circuit in Mealy form.
As all the involved components are essentially combinational, the circuit can
be normalised again to produce a circuit in normalised Mealy form.