\section{Restriction equations}

Encoding using Mealy homomorphisms allows us to map between circuits with the
same behaviour but different implementations.
In order to obtain a (pseudo-)normal form for circuits there needs to be a
canonical such encoding for any circuit behaviour.
We already have a suitable candidate for this: the image of
\(\mealytocircuiti\), the map from Mealy machines to circuits.
Circuits in this image are already in normalised Mealy form, so this is a good
start.
However, they are more specialised than that: their internal states are made up
only of \(\bot\) and \(\top\) elements, and the width of the state is equal to
the number of stream derivatives the original stream function had including
itself.
We will establish this as our pseudo-normal form for circuits; depending on the
ordering of the states picked, there may be multiple such circuits.

\begin{corollary}
    Given a circuit \(
    \iltikzfig{strings/category/f}[box=f,colour=seq]
    \), there exists at least one word \(\listvar{s} \in \{\bot,\top\}^x\) and
    normalised circuit \(
    \iltikzfig{strings/category/f}[box=g,colour=comb]
    \) such that \(
    \circuittostreami[\iltikzfig{strings/category/f}[box=f,colour=seq]]
    =
    \circuittostreami[\iltikzfig{circuits/productivity/mealy-form}[core=|g|,colour=seq]]
    \) by equations in \(
    \mealyequations + \normalisingequations + \encodingequations
    \).
\end{corollary}

By \cref{thm:possible-encodings}, we have that there is an encoding that maps
state sets between denotationally equivalent circuits.
If we can encode a circuit's state sets to one where the states are all elements
of \(\{\bot,\top\}^x\), does this mean we have a sound and complete equational
theory?
Unfortunately not: all it means is that the circuits agree on the set of
circuit states.

\begin{example}\label{ex:restriction-example}
    Consider the following two circuits in \(\scirc{\belnapsignature}\): \[
        \iltikzfig{circuits/examples/state-change/circuit-mealy}
        \quad
        \iltikzfig{circuits/examples/state-change/circuit-simpler-mealy}
    \]
    Both circuits have circuit states \(
    \{\belnaptrue\belnapfalse,\belnapfalse\belnaptrue\}
    \), but their combinational cores do \emph{not} have the same semantics!
    They only act the same because they receive certain inputs.
\end{example}

The final family of equations required is one for mapping between combinational
circuits that agree on the context, but may differ otherwise.

\begin{notation}
    Given sets \(A\), \(B\) and \(C\) where \(C \subseteq A\) and a function
    \(\morph{f}{A}{B}\), the \emph{restriction of \(f\) to \(C\)} is a function
    \(\morph{f|C}{C}{B}\), defined as \(f|C(c) \coloneqq f(c)\).
\end{notation}

\begin{definition}[Restriction equations]
    Let the schema of \emph{restriction equations} be defined as in
    \cref{fig:restriction-equation}.
\end{definition}

\begin{example}
    By a restriction equation, the circuits in \cref{ex:restriction-example} are
    now equal, as the cores produce equal outputs for inputs where the state is
    \(\belnaptrue\belnapfalse\) or \(\belnapfalse\belnaptrue\).
\end{example}
\begin{figure}
    \centering
    \(\equationdisplay{
        \iltikzfig{circuits/productivity/mealy-form}[core=\mealytofunc[f]]
    }{
        \iltikzfig{circuits/productivity/mealy-form}[core=|g|]
    }{\restrictionequation}\)
    \,\,
    \begin{minipage}{0.25\textwidth}
        \centering
        where \(
        f|S_{f,\listvar{s}} \times \valuetuple{\listvar{m}}
        =
        g|S_{f,\listvar{s}} \times \valuetuple{\listvar{m}}
        \)
    \end{minipage}
    \caption{The \emph{restriction} equation}
    \label{fig:restriction-equation}
\end{figure}