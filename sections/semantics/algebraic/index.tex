\chapter{Algebraic semantics}

Testing that the behaviour of two circuits is equivalent by testing that every
input produces the same output for each circuit is a perfectly reasonable
strategy.
But this is somewhat `nuclear'; rather than using what we know
about the structure of a circuit's components, we just blast away
exhaustively trying all the inputs to find a contradiction.

A more elegant method of reasoning is by defining a set \emph{equations} between
subcircuits and \emph{quotienting} \(\scircsigma\) by these equations.
A proof of equivalence between two circuits is then presented using algebraic
reasoning: applying equations to translate one circuit into the other.
This is often \emph{far more efficient} than having to test every input!
This is known as \emph{algebraic} semantics, which is our final flavour of
circuit semantics.

\begin{example}\label{ex:expressions-algebraic}
    For the last time we return to the language of arithmetical expressions from
    \cref{ex:expressions-denotational}.
    An algebraic semantics for this language can be defined using a set of
    equations: the familiar equations of associativity, commutativity,
    unitality, annihiliation and distributivity, along with equations for
    actually performing arithmetic.
    \begin{gather*}
        add\,(add\,e_1\,e_2)\,e_3 = add\,e_1\,(add\,e_2\,e_3)
        \qquad
        mul\,(mul\,e_1\,e_2)\,e_3 = mul\,e_1\,(mul\,e_2\,e_3)
        \\
        add\,e_1\,e_2 = add\,e_2\,e_1
        \qquad
        mul\,e_1\,e_2 = mul\,e_2\,e_1
        \\
        add\,e_1\,\overline{0} = e_1
        \qquad
        mul\,e_1\,\overline{1} = e_1
        \qquad
        mul\,e_1\,\overline{0} = \overline{0}
        \\
        mul\,e_1\,(add\,e_2\,e_3) = add\,(mul\,e_1\,e_2)\,(mul\,e_1\,e_3)
        \\
        add\,\overline{n_1}\,\overline{n_2} = add\,\overline{n_1+n_2}
        \qquad
        add\,\overline{n_1}\,\overline{n_2} = \overline{n_1 \cdot n_2}
    \end{gather*}
    If everything is specified concretely as values then one could easily just
    use the last two equations to compare two expressions by reducing
    two expressions to values as in the operational semantics.
    The power of the algebraic semantics comes from the fact we can reason
    abstractly with expressions containing \emph{blackboxes}.
    Take the following example, containing some arbitrary component \(e\).
    \begin{align*}
        mul\,(add\,e\,(mul\,e\,\overline{3}))\,\overline{2}
         & =
        mul\,(add\,(mul\,e\,1)\,(mul\,e\,\overline{3}))\,\overline{2}
        \\
         & =
        mul\,(mul\,e\, add(\overline{1}\,\overline{3}))\,\overline{2}
        \\
         & =
        mul\,(mul \,e\,\overline{4})\,\overline{2}
        \\
         & =
        mul\,e\,(mul \,\overline{4}\,\overline{2})
        \\
         & =
        mul\,e\,\overline{8}
        \\
         & =
        mul\,\overline{8}\,e
    \end{align*}
    Despite not specifying the structure of \(e\), we have
    shown how the expression is equal to a slightly simpler one.
    This process creates new general equations which can be used as `shortcuts'
    in future reasoning, potentially saving many steps.
\end{example}

As with the operational semantics, we are especially interested in defining a
\emph{sound and complete} algebraic semantics for sequential digital circuits
with respect to the denotational semantics.
That is to say, for each equation \(
\iltikzfig{strings/category/f}[box=f,colour=seq]
=
\iltikzfig{strings/category/f}[box=g,colour=seq]
\) then \(
\circuittostreami[\iltikzfig{strings/category/f}[box=f,colour=seq]]
=
\circuittostreami[\iltikzfig{strings/category/f}[box=g,colour=seq]]
\), and there must be enough equations such that if \(
\circuittostreami[\iltikzfig{strings/category/f}[box=f,colour=seq]]
=
\circuittostreami[\iltikzfig{strings/category/f}[box=g,colour=seq]]
\) then there exists a series of equations identifying \(
\circuittostreami[\iltikzfig{strings/category/f}[box=f,colour=seq]]
\) and \(
\circuittostreami[\iltikzfig{strings/category/f}[box=f,colour=seq]]
\).

\begin{remark}
    An `equational theory' for sequential circuits was one of the first things
    presented in the original circuits
    paper~\cite{ghica2016categorical,ghica2017diagrammatic}.
    In that paper, equations that `seemed right' were used to quotient the
    syntax, with the ultimate aim of showing that the resulting category was
    \emph{Cartesian}.
    This was done quite informally, and was made more confusing as
    the categories of circuits were subsequently quotiented by some notion of
    `extensional equivalence', an attempt to rectify the fact that the
    equations only dealt with closed circuits.
    Soundness and completeness of the equational theory was not considered
    because there was no other semantics to compare it against!

    In essence, the previous work was almost `the wrong way round': equations
    were defined and semantics drawn from them.
    In the more recent version of the work~\cite[Sec. 5]{ghica2024fully}, which
    forms the basis for this chapter, the equations are derived from the
    denotational semantics.
    Not only does this give us a formal way of verifying that these equations
    are sound, it sets the backdrop against which we can test if the algebraic
    semantics are sufficient: are any two denotationally equal circuits
    identified by equations?

    In fact, it turns out that the way the equations in the previous work had
    been defined looked a lot more like \emph{reduction rules}, so they have
    found their home in the previous section.
\end{remark}

When defining such an equational theory, choosing which set of equations can be
a controversial topic, as there may be several different sound and complete
formulations!
Ideally, we want to stick to simple \emph{local} equations that concern the
interactions of concrete generators as much as possible, but as we will see
we will sometimes have no choice but to define \emph{families} of equations
parameterised over some arbitrary subcircuit.

\section{Normalising circuits}\label{sec:normalising}

How does one start when trying to define a complete set of equations for some
framework?
Usually the strategy is to define enough equations to bring any term to some
sort of (pseudo-)\emph{normal form}; the theory is then complete if terms with
the same semantics have the same normal form.

We have already seen something that looks a bit like a normal form: the
\emph{Mealy form} from the previous section.
\index{Mealy form}
This is by no means a true normal form, as there are many different Mealy forms
that represent the same behaviour.
Nevertheless, it is a useful starting point so we will need equations to bring
circuits to Mealy form in our theory.

Instead of just turning the Mealy reduction rules into equations, we will show
how Mealy form can be derived using smaller equations.

\begin{figure}
    \centering
    \(
    \equationdisplay{
        \iltikzfig{strings/structure/monoid/unitality-l-lhs}
    }{
        \iltikzfig{strings/structure/monoid/unitality-l-rhs}
    }{
        \joinunitleqn
    }
    \qquad
    \equationdisplay{
        \iltikzfig{strings/structure/monoid/unitality-r-lhs}
    }{
        \iltikzfig{strings/structure/monoid/unitality-r-rhs}
    }{
        \joinunitreqn
    }
    \)
    \\[0.25em]
    \rule{\textwidth}{0.1mm}
    \\[0.5em]
    \(
    \equationdisplay{
        \iltikzfig{circuits/axioms/bottom-delay-lhs}
    }{
        \iltikzfig{circuits/axioms/bottom-delay-rhs}
    }{
        \bottomdelayeqn
    }
    \qquad
    \equationdisplay{
        \iltikzfig{circuits/instant-feedback/equation-lhs}[box=f]
    }{
        \iltikzfig{circuits/instant-feedback/equation-rhs}[box=f]
    }{
        \instantfeedbackeqn
    }
    \)
    \\[0.25em]
    \rule{\textwidth}{0.1mm}
    \caption{
        Set of Mealy equations
        \(\mealyequations\).
    }
    \label{fig:mealy-equations}
\end{figure}

\begin{definition}
    \index{Mealy equations}
    The set \(\mealyequations\) of \emph{Mealy equations} in
    \cref{fig:mealy-equations} are sound.
\end{definition}
\begin{proof}
    The first two rules hold as the join is a monoid in the stream semantics.
    The \((\bottomdelayeqn)\) holds because the semantics of the delay
    component are to output a \(\bot\) value first and then the (delayed)
    inputs: as the semantics of the \(
    \iltikzfig{strings/structure/monoid/init}[colour=comb]
    \) component are to \emph{always} produce \(\bot\), then it does not make a
    difference how delayed it is.
    The final equation is the instant feedback rule, which is sound by
    \cref{prop:instant-feedback}.
\end{proof}

\begin{proposition}\label{prop:mealy-equations}
    Given a sequential circuit \(
    \iltikzfig{strings/category/f}[box=f,colour=seq,dom=m,cod=n]
    \), there exists a combinational circuit \(
    \iltikzfig{strings/category/f-2-2}[box=g,colour=comb,dom1=x,dom2=m,cod1=x,cod2=n]
    \) and values \(\listvar{s} \in \valuetuple{x}\) such that \(
    \iltikzfig{strings/category/f}[box=f,colour=seq]
    =
    \iltikzfig{circuits/productivity/mealy-form}[core=g]
    \) in \(\scircsigma / \mealyequations\).
\end{proposition}
\begin{proof}
    Any circuit can be assembled into global trace-delay form solely using the
    axioms of STMCs.
    From this, a circuit in pre-Mealy form can be obtained by translating
    delays and values into registers using the following equations: \[
        \iltikzfig{circuits/axioms/delay-to-register/step-0}
        \eqaxioms[(\joinunitleqn)]
        \iltikzfig{circuits/axioms/delay-to-register/step-1}
        \qquad\quad
        \iltikzfig{circuits/axioms/value-to-register/step-0}
        \eqaxioms[(\joinunitreqn)]
        \iltikzfig{circuits/axioms/value-to-register/step-1}
        \eqaxioms[(\bottomdelayeqn)]
        \iltikzfig{circuits/axioms/value-to-register/step-2}
    \]
    Subsequently a circuit in Mealy form can be obtained by applying the
    \((\instantfeedbackeqn)\) rule.
\end{proof}

\(\scircsigma / \mealyequations\) is a category in which all circuits are equal
to at least one circuit in Mealy form.
In general, there will be many Mealy forms depending on the ordering one picks
for the delays and value; our task is to provide equations to map any two
denotationally equivalent circuits to the \emph{same} Mealy form.

Even if the combinational cores of two Mealy forms have the same behaviour, they
may not have the same structure.
To reduce the number of cores we have to consider, we will first establish
equations for translating any combinational circuit into some canonical circuit.
We already met a method for determining what this canonical circuit is: the
functional completeness map \(\mealytofunc\) from \(\funci\) to \(\scircsigma\).

\begin{definition}[Normalised circuit]
    \index{normalised circuit}
    A sequential circuit \(
    \iltikzfig{strings/category/f}[box=f,colour=seq,dom=m,cod=n]
    \) is \emph{normalised} if it is in the image of \(\mealytofunc\).
\end{definition}

As a shorthand, we will often abuse notation and write \(
\mealytofunc[f]
\coloneqq
\iltikzfig{strings/category/f-wide}[box=\mealytofunc[f],colour=seq]
\).
Recall that even though \(\mealytofunc\) maps into \(\scircsigma\), every
circuit in its image has combinational behaviour.
This is quite an important distinction to make, so we will give it a proper
name.

\begin{definition}[Essentially combinational]
    \index{essentially combinational}
    \index{combinational!essentially}
    A sequential circuit is \emph{essentially combinational} if it is in the
    form \(
    \iltikzfig{circuits/synthesis/normalised-function}[box=f]
    \).
\end{definition}

Such circuits are sequential circuits that exhibit combinational behaviour: any
value components are only used to introduce constants which do not alter over
time.

As the normalised version of a given circuit is interpretation-dependent, there
is no standard set of equations for normalising a circuit.
Instead, these must be specified on an interpretation-by-interpretation basis.

\begin{definition}[Normalising equations]
    \index{normalising equations}
    For a complete interpretation \((\interpretation,\mealytofunc)\), a set of
    equations \(\normalisingequations\) is \emph{normalising} if any
    essentially combinational circuit \(
    \iltikzfig{strings/category/f}[box=f,colour=seq,dom=m,cod=n]
    \) is equal to a circuit in the image of \(\mealytofunc\) by equations in
    \(\normalisingequations\).
\end{definition}

\begin{definition}[Normalisable interpretation]
    \index{interpretation!normalisable}
    \index{normalisable interpretation}
    A complete interpretation \(\interpretation\) is called \emph{normalisable} if there
    exists a set of normalising equations \(\normalisingequations\).
\end{definition}

The normalising equations for a given interpretation can be used to translate a
combinational core into a circuit from which it is easy to read off a truth
table.

\begin{theorem}\label{thm:normalising}
    For every sequential circuit \(
    \iltikzfig{strings/category/f}[box=f,colour=seq,dom=m,cod=n]
    \) in a normalisable complete interpretation
    \((\interpretation,\mealytofunc)\) over \(\Sigma\), there exists essentially
    combinational \(
    \iltikzfig{strings/category/f-2-2-wide}[box=\mealytofunc[g],colour=seq,dom1=x,dom2=m,cod1=x,cod2=n]
    \) and \(\listvar{s} \in \valuetuple{x}\) such that \(
    \iltikzfig{strings/category/f}[box=f,colour=seq,dom=m,cod=n]
    =
    \iltikzfig{circuits/productivity/mealy-form-wide}[core=\mealytofunc[g],colour=seq,state=\listvar{s}]
    \) in \(\scircsigma / \mealyequations + \normalisingequations\).
\end{theorem}
\begin{proof}
    By \cref{prop:mealy-equations}, \(
    \iltikzfig{strings/category/f}[box=f,colour=seq,dom=m,cod=n]
    =
    \iltikzfig{circuits/productivity/mealy-form}[core=h]
    \) and by equations in \(\normalisingequations\), \(
    \iltikzfig{strings/category/f-2-2}[box=h,colour=comb]
    =
    \iltikzfig{strings/category/f-2-2-wide}[box=\mealytofunc[g],colour=seq]
    \).
\end{proof}
\section{Encoding equations}\label{sec:encoding}

A circuit in Mealy form is a syntactic representation of a Mealy machine: the
combinational core is the Mealy function, and the registers are the initial
state.
It is important to determine the states that the circuit might assume, as these
determine whether or not an equation is valid.

\begin{definition}[States]
    Let \(\morph{f}{\valuetuple{x+m}}{\valuetuple{x+n}}\) be a
    monotone function and let \(\listvar{s} \in  \valuetuple{x}\) be a state.
    Then the \emph{states of \(f\) from \(\listvar{s}\)}, denoted
    \(S_{f,\listvar{s}}\), is the smallest set containing \(\listvar{s}\) and
    closed under \(
    \listvar{r}
    \mapsto
    \proj{x}\mleft(f_0(\listvar{r},\listvar{v})\mright)
    \) for any \(\listvar{v} \in \valuetuple{m}\).
\end{definition}

\begin{example}\label{ex:circuit-states}
    Consider the circuit \(
    \iltikzfig{circuits/examples/bottrue/circuit}
    \).
    The semantics of the combinational core are clearly
    \((s, r) \mapsto \left(s \lor r, s, s \lor r\right)\), where the first two characters are the
    next state and the third is the output.
    The initial state is \(\bot\belnaptrue\), so the subsequent states are
    \((\bot \lor \belnaptrue, \bot) = (\belnaptrue,\bot)\) and
    \((\belnaptrue \lor \bot, \belnaptrue) = (\belnaptrue, \belnaptrue)\).
    As \((\belnaptrue \lor \belnaptrue, \belnaptrue) = (\belnaptrue, \belnaptrue)\),
    there are no more circuit states and the complete set is
    \(\{(\bot, \belnaptrue),(\belnaptrue,\bot),(\belnaptrue,\belnaptrue)\}\).

    Note that as the output of the circuit is computed as \(s \lor r\), for each
    circuit state the output is \(\belnaptrue\).
    This means that the circuit is denotationally equivalent to \(
    \iltikzfig{circuits/components/waveforms/infinite-register}[val=\belnaptrue]
    \), but this circuit only has a single state \(\belnaptrue\).
\end{example}

We need to \emph{encode} the states of one circuit as another; we have already
encountered this notion using \emph{Mealy homomorphisms}
(\cref{def:mealy-homomorphism});
functions between the state sets that preserve transitions and outputs.
While two `inverse' homomorphisms may not be isomorphisms, the round
trip will always map to a state with the same behaviour.

\begin{lemma}
    Given two Mealy homomorphisms \(\morph{h}{(S,f)}{(T,g)}\) and
    \(\morph{h^\prime}{(T,g)}{(S,f)}\), any state \(s \in S\) and input
    \(a \in A\), \(
    \mealyfunctiontransition{f}(s, a)
    =
    \mealyfunctiontransition{f}(h^\prime(h(s)), a)
    \).
\end{lemma}
\begin{proof}
    Immediate as Mealy homomorphisms preserve outputs.
\end{proof}

We will use Mealy homomorphisms as circuits to encode state sets; this means we
need to ensure the encoders and decoders are monotone.

\begin{lemma}
    For partial orders \(S\) and \(T\) and monotone Mealy coalgebra
    \((S,f)\) and \((T,g)\), any Mealy homomorphism \(\morph{h}{(S,f)}{(T,g)}\)
    is monotone.
\end{lemma}
\begin{proof}
    In a monotone Mealy coalgebra, the functions \(f\) and \(g\) are monotone,
    and for \(h\) to be a Mealy homomorphism, \(
    \mealyfunctionoutput{f}(s)
    =
    \mealyfunctionoutput{g}(h(s))
    \).
    For states \(s,r \in S\) we have \(
    \mealyfunctionoutput{g}(h(s), a)
    =
    \mealyfunctionoutput{f}(s, a)
    \leq
    \mealyfunctionoutput{f}(r, a)
    =
    \mealyfunctionoutput{g}(h(r), a)
    \).
    This means that the function \(
    s \mapsto \mealyfunctionoutput{g}(h(s), a)
    \) is monotone; as \(\mealyfunctionoutput{g}\) is monotone, \(h\) must
    also be monotone.
\end{proof}

For two circuits \(
\iltikzfig{circuits/productivity/mealy-form-wide}[core=\mealytofunc[f],state=\listvar{s},colour=seq,dom=m,cod=n,delay=x]
\) and \(
\iltikzfig{circuits/productivity/mealy-form-wide}[core=\mealytofunc[g],state=\listvar{t},colour=seq,dom=m,cod=n,delay=y]
\), the encoders and decoders will be homomorphisms
\((S_{f,\listvar{s}},f) \to (S_{g,\listvar{t}},g)\)
and
\((S_{g,\listvar{t}},g) \to (S_{f,\listvar{s}},f)\).
These are homomorphisms on the subset of states that a circuit
can assume, \emph{not} the entire set of words that can fit into the state.
This means that encoding and decoding circuits cannot be inserted arbitrarily
but only in certain contexts.

\begin{proposition}[Encoding equation]\label{prop:encoding-equation}
    For a normalised circuit \(
    \iltikzfig{strings/category/f-2-2-wide}[box=\mealytofunc[f],colour=seq,dom1=x,dom2=m,cod1=x,cod2=y]
    \) and \(\listvar{s} \in \valuetuple{x}\), let
    \(\morph{\mathsf{enc}}{S_{f,\listvar{s}}}{\valuetuple{y}}\) and
    \(\morph{\mathsf{dec}}{\valuetuple{y}}{S_{f, \listvar{s}}}\) be functions
    such that \(
    \morph{\mathsf{dec} \circ \mathsf{enc}}{S_{f, \listvar{s}}}{S_{f, \listvar{s}}}
    \) is a Mealy homomorphism.
    Then the \emph{encoding equation} \((\encodingequation)\) in
    \cref{fig:encoding-equation} is sound, where
    \(\mathsf{enc}_\mathsf{m}\) and \(\mathsf{dec}_\mathsf{m}\) are monotone
    completions as defined in \cref{def:monotone-completion}.
\end{proposition}
\begin{proof}
    Let \(g\) be the map \(\listvar{r} \mapsto
    \circuittostreami[\iltikzfig{circuits/algebraic/state-encoding}[core=\mealytofunc[f],delay=x,state=\listvar{r}]]
    \); by \cref{prop:mealy-form-image} we know that \(
    \mealyoutput{g(\listvar{t})}{\listvar{v}}
    =
    \proj{1}(f(\mathsf{dec}(\mathsf{enc}(\listvar{t})), \listvar{v}))
    \) and \(
    \mealytransition{g(\listvar{t})}{\listvar{v}}
    =
    g(\proj{0}(f(\mathsf{dec}(\mathsf{enc}(\listvar{t})), \listvar{v})))
    \).
    As \(\mathsf{dec} \circ \mathsf{enc}\) is a Mealy homomorphism, for
    \(\listvar{t} \in S_{f, \listvar{s}}\) we have that \(
    \mealyoutput{g(\listvar{t})}{\listvar{v}}
    =
    \proj{1}(f(\listvar{t}), \listvar{v})
    \) and that \(
    \mealytransition{g(\listvar{t})}{\listvar{v}}
    \) shares outputs and transitions with \(
    g(\proj{0}(f(\listvar{t})), \listvar{v})
    \).
    As \(
    \iltikzfig{circuits/algebraic/state-encoding}[core=\mealytofunc[f],delay=x,state=\listvar{s}]
    \coloneqq
    g(\listvar{s})
    \) and \(\listvar{s} \in S_{f,\listvar{s}}\),
    every subsequent stream derivative will also be of the form
    \(g(\listvar{t})\) where \(\listvar{t} \in S_{f,\listvar{s}}\), so the
    equation is sound.
\end{proof}

\begin{remark}
    The encoding equation is an equation \emph{schema}: this is required because
    the width of a circuit state can be arbitrarily large, and each extra bit
    adds a whole new set of Mealy homomorphisms to consider.
\end{remark}

\begin{figure}
    \centering
    \(
    \equationdisplay{
        \iltikzfig{circuits/productivity/mealy-form-wide}[core=\mealytofunc[f],delay=x, colour=seq]
    }{
        \iltikzfig{circuits/algebraic/state-encoding}[core=\mealytofunc[f],delay=x]
    }{
        \encodingequation
    }
    \)
    \\[0.25em]
    \rule{\textwidth}{0.1mm}
    \\[0.5em]
    \(
    \equationdisplay{
        \iltikzfig{circuits/axioms/gate-lhs}[gate=p]
    }{
        \iltikzfig{circuits/axioms/gate-rhs}[gate=p]
    }{
        \gateeqn
    }
    \quad
    \equationdisplay{
        \iltikzfig{circuits/axioms/fork-lhs}
    }{
        \iltikzfig{circuits/axioms/fork-rhs}
    }{
        \forkeqn
    }
    \quad
    \equationdisplay{
        \iltikzfig{circuits/axioms/join-lhs}
    }{
        \iltikzfig{circuits/axioms/join-rhs}
    }{
        \joineqn
    }
    \)
    \\[0.25em]
    \rule{\textwidth}{0.1mm}
    \\[0.5em]
    \(
    \equationdisplay{
        \iltikzfig{circuits/axioms/stub-lhs}
    }{
        \iltikzfig{strings/monoidal/empty}
    }{
        \stubeqn
    }
    \quad
    \equationdisplay{
        \iltikzfig{circuits/axioms/delay-fork-lhs}
    }{
        \iltikzfig{circuits/axioms/delay-fork-rhs}
    }{
        \delayforkeqn
    }
    \quad
    \equationdisplay{
        \iltikzfig{circuits/axioms/bottom-delay-lhs}
    }{
        \iltikzfig{circuits/axioms/bottom-delay-rhs}
    }{
        \bottomdelayeqn
    }
    \quad
    \equationdisplay{
        \iltikzfig{circuits/axioms/streaming-lhs}
    }{
        \iltikzfig{circuits/axioms/streaming-rhs}[gate=p]
    }{
        \streamingeqn
    }
    \)
    \\[0.25em]
    \rule{\textwidth}{0.1mm}
    \\[0.5em]
    \(
    \equationdisplay{
        \iltikzfig{strings/structure/comonoid/unitality-l-lhs}
    }{
        \iltikzfig{strings/structure/comonoid/unitality-l-rhs}
    }{
        \comonoiduniteqnletter
    }
    \quad
    \equationdisplay{
        \iltikzfig{strings/structure/monoid/associativity-lhs}
    }{
        \iltikzfig{strings/structure/monoid/associativity-rhs}
    }{
        \monoidassoceqnletter
    }
    \quad
    \equationdisplay{
        \iltikzfig{strings/structure/monoid/commutativity-lhs}
    }{
        \iltikzfig{strings/structure/monoid/commutativity-rhs}
    }{
        \monoidcommeqnletter
    }
    \quad
    \equationdisplay{
        \iltikzfig{strings/structure/bialgebra/merge-copy-lhs}
    }{
        \iltikzfig{strings/structure/bialgebra/merge-copy-rhs}
    }{
        \joinforkeqn
    }
    \)
    \caption{
        Set \(\encodingequations\) of equations for encoding circuit states
    }
    \label{fig:encoding-equation}
\end{figure}

The encoding equation only inserts encoder circuits; to actually change the
state we need some more equations.

\begin{lemma}
    The equations on the bottom four rows of \cref{fig:encoding-equation} are
    sound.
\end{lemma}
\begin{proof}
    It is a straightforward exercise to compare the stream functions.
\end{proof}

To show the final result we must prove some lemmas; first we show how we can
`pump' a value out of an infinite waveform.

\begin{lemma}\label{lem:unroll-waveform}
    For \(v \in \values\), \(
    \iltikzfig{circuits/algebraic/unroll-waveform/step-0}[value=v]
    =
    \iltikzfig{circuits/algebraic/unroll-waveform/step-6}[value=v]
    \) using the encoding equations.
\end{lemma}
\begin{proof}
    The proof is straightforward and is illustrated in
    \cref{fig:unroll-waveform}.
\end{proof}
%
\begin{figure}
    \centering
    \begin{gather*}
        \scalebox{0.9}{\iltikzfig{circuits/algebraic/unroll-waveform/step-0}[value=v]}
        \coloneqq
        \scalebox{0.9}{\iltikzfig{circuits/algebraic/unroll-waveform/step-1}[value=v]}
        \eqaxioms[(\joinforkeqn)]
        \scalebox{0.9}{\iltikzfig{circuits/algebraic/unroll-waveform/step-2}[value=v]}
        \eqaxioms[(\forkeqn)]
        \\[1em]
        \scalebox{0.9}{\iltikzfig{circuits/algebraic/unroll-waveform/step-3}[value=v]}
        \eqaxioms[(\delayforkeqn)]
        \scalebox{0.9}{\iltikzfig{circuits/algebraic/unroll-waveform/step-4}[value=v]}
        \coloneqq
        \scalebox{0.9}{\iltikzfig{circuits/algebraic/unroll-waveform/step-5}[value=v]}
        =
        \scalebox{0.9}{\iltikzfig{circuits/algebraic/unroll-waveform/step-6}[value=v]}
    \end{gather*}
    \caption{Proof of \cref{lem:unroll-waveform}}
    \label{fig:unroll-waveform}
\end{figure}

The next lemma shows how the familiar `streaming' rule from the operational
semantics can be derived equationally.

\begin{lemma}\label{lem:generalised-streaming}
    For a combinational circuit \(
    \iltikzfig{strings/category/f}[box=f,colour=comb]
    \), \(
    \iltikzfig{circuits/axioms/generalised-streaming-lhs}[box=f]
    =
    \iltikzfig{circuits/axioms/generalised-streaming-rhs}[box=f]
    \) by the encoding equations.
\end{lemma}
\begin{proof}
    This is by induction on the structure of \(
    \iltikzfig{strings/category/f}[box=f,colour=comb]
    \).
    The case for the primitive is immediate by \((\streamingeqn)\).
    For \(\iltikzfig{strings/structure/comonoid/copy}[colour=comb]\) we have
    that \[
        \iltikzfig{circuits/productivity/generalised-streaming/fork-step-0}
        \eqaxioms[(\joinforkeqn)]
        \iltikzfig{circuits/productivity/generalised-streaming/fork-step-1}
        \eqaxioms[(\delayforkeqn)]
        \iltikzfig{circuits/productivity/generalised-streaming/fork-step-2}
    \]
    The proof for \(\iltikzfig{strings/structure/monoid/merge}[colour=comb]\) is
    illustrated in \cref{fig:generalised-streaming-join}.
    The case for \(\iltikzfig{strings/structure/comonoid/discard}[colour=comb]\) is
    trivial, and the case for \(\iltikzfig{strings/structure/monoid/init}[colour=comb]\)
    follows by \((\comonoiduniteqnletter)\) and \((\bottomdelayeqn)\).
    The cases for \(\iltikzfig{strings/category/identity}[colour=comb]\) and
    \(\iltikzfig{strings/symmetric/symmetry}[colour=comb]\) follow by axioms of STMCs.
    Since the underlying circuit is combinational, for the inductive cases we just
    need to check composition and tensor, which are trivial.
\end{proof}
%
\begin{figure}
    \begin{gather*}
        \iltikzfig{circuits/productivity/generalised-streaming/join-step-2}
        \eqaxioms[(\monoidassoceqnletter)]
        \iltikzfig{circuits/productivity/generalised-streaming/join-step-3}
        \eqaxioms[(\monoidassoceqnletter)]
        \iltikzfig{circuits/productivity/generalised-streaming/join-step-4}
        \eqaxioms[(\monoidcommeqnletter)]
        \\[1em]
        \iltikzfig{circuits/productivity/generalised-streaming/join-step-5}
        =
        \iltikzfig{circuits/productivity/generalised-streaming/join-step-6}
        \eqaxioms[(\monoidassoceqnletter)]
        \iltikzfig{circuits/productivity/generalised-streaming/join-step-7}
        \eqaxioms[(\monoidassoceqnletter)]
        \\[1em]
        \iltikzfig{circuits/productivity/generalised-streaming/join-step-8}
        \eqaxioms[(\monoidassoceqnletter)]
        \iltikzfig{circuits/productivity/generalised-streaming/join-step-9}
    \end{gather*}
    \caption{Proof of \cref{lem:generalised-streaming} for the join case}
    \label{fig:generalised-streaming-join}
\end{figure}

We next show how the encoding equations can be used to translate combinational
circuits with inputs into values.

\begin{lemma}\label{lem:combinational-circuit-inputs}
    Let \(\iltikzfig{strings/category/f}[box=f,colour=comb,dom=m,cod=n]\) be a
    combinational circuit such that \(
    \circuittofunci[\iltikzfig{strings/category/f}[box=f,colour=comb]]
    =
    g
    \).
    Then \(
    \iltikzfig{circuits/components/circuits/f-applied}[box=f,colour=comb]
    =
    \iltikzfig{circuits/components/values/vs-even-longer}[val=g(\listvar{v})]
    \) by the encoding equations.
\end{lemma}
\begin{proof}
    For the same reasoning as \cref{lem:reduce-core-terminating}, the
    \((\gateeqn)\), \((\forkeqn)\), \((\joineqn)\) and \((\stubeqn)\) equations
    can be used to show that there exists \(\listvar{w} \in \valuetuple{n}\)
    such that \(
    \iltikzfig{strings/category/f}[box=f,colour=comb,dom=m,cod=n]
    =
    \iltikzfig{circuits/components/values/vs}[val=\listvar{w}]
    \).

    Now we need to show that \(
    \circuittostreami[
        \iltikzfig{circuits/components/circuits/f-applied}[box=f,colour=comb]
    ]
    =
    \circuittostreami[
        \iltikzfig{circuits/components/values/vs-even-longer}[val=g(\listvar{v})]
    ]
    \).
    By functoriality of \(\circuittostreami\), \(
    \circuittostreami[
        \iltikzfig{circuits/components/circuits/f-applied}[box=f,colour=comb]
    ]
    =
    \circuittostreami[
        \iltikzfig{circuits/components/values/vs}[val=\listvar{v}]
    ] \seq
    \circuittostreami[
        \iltikzfig{strings/category/f}[box=f,colour=comb]
    ]
    \).
    By \cref{lem:sequential-combinational-semantics} we know that \(
    \circuittostreami[
        \iltikzfig{strings/category/f}[box=f,colour=comb]
    ](\sigma)(i) = \circuittofunci[
        \iltikzfig{strings/category/f}[box=f,colour=comb]
    ] = g(\sigma)(i)\) for all \(\sigma \in \valuetuplestream{m}\) and
    \(i \in \nat\).
    Since \(\circuittostreami[
        \iltikzfig{circuits/components/values/vs}[val=\listvar{v}]
    ] = \listvar{v} \streamcons \bot \streamcons \bot \streamcons \dots\), we
    have that \(
    \circuittostreami[
        \iltikzfig{circuits/components/circuits/f-applied}[box=f,colour=comb]
    ]
    =
    g(\listvar{v}) \streamcons \bot \streamcons \bot \streamcons \dots
    \), which is the interpretation of \(
    \iltikzfig{circuits/components/values/vs-even-longer}[val=g(\listvar{v})]
    \).
    As the equations are sound they must preserve the stream semantics, so
    \(\listvar{w} = g(\listvar{v})\).
\end{proof}

Finally, we use the above lemma to show how values can be applied to
\emph{essentially} combinational circuits.

\begin{lemma}\label{lem:essentially-combinational-applied}
    Let \(\morph{f}{\valuetuple{m}}{\valuetuple{n}}\) be a monotone function
    such that \(
    \iltikzfig{strings/category/f-wide}[box=\mealytofunc[f],colour=seq]
    \coloneqq
    \iltikzfig{circuits/synthesis/normalised-function}[box=g]
    \).
    Then \(
    \iltikzfig{circuits/algebraic/normalised-inputs}[box=g,input1=\listvar{v},input2=\listvar{w}]
    =
    \iltikzfig{circuits/components/values/vs-even-longer}[val=f(\listvar{w})]
    \).
\end{lemma}
\begin{proof}
    Let \(h \coloneqq \circuittofunci[
        \iltikzfig{strings/category/f-2-1}[box=g,colour=comb]
    ]\); using \cref{lem:combinational-circuit-inputs}, we have that \(
    \iltikzfig{circuits/algebraic/normalised-inputs}[box=g,input1=\listvar{v},input2=\listvar{w}]
    =
    \iltikzfig{circuits/components/values/vs-really-long}[val={h(\listvar{v},\listvar{w})}]
    \).
    So we must show that \(f(\listvar{w}) = h(\listvar{v}, \listvar{w})\).
    \begin{align*}
        f(\listvar{w})
         & =
        \circuittostreami[
            \iltikzfig{strings/category/f-wide}[box=\mealytofunc[f],colour=seq]
        ](\listvar{w} \streamcons \bot^\omega)(0)
         &
        \text{\cref{def:functional-completeness}}
        \\
         & \coloneqq
        \circuittostreami[
            \iltikzfig{circuits/synthesis/normalised-function}[box=g]
        ](\listvar{w} \streamcons \bot^\omega)(0)
        \\
         & =
        \circuittostreami[
            \iltikzfig{strings/category/f-2-1}[box=g,colour=comb]
        ](\listvar{v}^\omega, \listvar{w} \streamcons \bot^\omega)(0)
        \\
         & =
        \circuittofunci[
            \iltikzfig{strings/category/f-2-1}[box=g,colour=comb]
        ](\listvar{v}, \listvar{w})
         &
        \text{\cref{lem:sequential-combinational-semantics}}
        \\
         & =
        h(\listvar{v}, \listvar{w})
    \end{align*}
    This completes the proof.
\end{proof}

With these lemmas in our toolkit, we can now show that the encoding equations
allow us to translate a circuit into one with an encoded state, and therefore
translate between the state sets of any two denotationally equivalent circuits.

\begin{theorem}\label{thm:encoding}
    For a circuit \(
    \iltikzfig{strings/category/f-2-2-wide}[box=\mealytofunc[f],colour=seq,dom1=x,dom2=m,cod1=x,cod2=y]
    \) and initial state \(\listvar{s} \in \valuetuple{x}\), the
    equation \(
    \iltikzfig{circuits/productivity/mealy-form-wide}[core=\mealytofunc[f], colour=seq]
    =
    \iltikzfig{circuits/algebraic/state-encoded}[core=\mealytofunc[f],state=\listvar{s}]
    \) is derivable by the equations in \(\encodingequations\).
\end{theorem}
\begin{proof}
    We have that \(
    \iltikzfig{circuits/productivity/mealy-form-wide}[core=\mealytofunc[f],delay=x, colour=seq]
    =
    \iltikzfig{circuits/algebraic/state-encoding}[core=\mealytofunc[f],delay=x,state=\listvar{s}]
    \) by the \((\encodingequation)\) equation; we need to `push' the encoder \(
    \iltikzfig{circuits/algebraic/encoder}
    \) through the state.
    Although the encoder is sequential, by the definition of \(\mealytofunc\),
    it must be of the form \(
    \iltikzfig{circuits/synthesis/normalised-function}[box=g]
    \) by definition of complete interpretations.
    This means we have
    \begin{align*}
        \iltikzfig{circuits/algebraic/encoding-state/step-0a}[box=g]
         & \coloneqq
        \iltikzfig{circuits/algebraic/encoding-state/step-0}[box=g]
        \\[1em]
         & =
        \iltikzfig{circuits/algebraic/encoding-state/step-1}[box=g]
         &
        \text{\cref{lem:unroll-waveform}}
        \\[1em]
         & =
        \iltikzfig{circuits/algebraic/encoding-state/step-2}[box=g]
         &
        \text{\cref{lem:generalised-streaming}}
        \\[1em]
         & =
        \iltikzfig{circuits/algebraic/encoding-state/step-3}[box=g]
         &
        \text{\cref{lem:essentially-combinational-applied}}
        \\[1em]&\coloneqq
        \iltikzfig{circuits/algebraic/encoding-state/step-4}[box=g]
        \\[1em]
         & \coloneqq\iltikzfig{circuits/algebraic/encoding-state/step-5}[box=g]
    \end{align*}
    The proof is completed by sliding the encoder around the trace.
\end{proof}

With the right encoders, the initial state of a circuit can be translated into
a different word, giving us a new circuit in Mealy form.
As all the involved components are essentially combinational, the circuit can
be normalised again to produce a circuit in normalised Mealy form.
\section{Restriction equations}

Encoding using Mealy homomorphisms allows us to map between circuits with the
same behaviour but different implementations.
In order to obtain a (pseudo-)normal form for circuits there needs to be a
canonical such encoding for any circuit behaviour.
We already have a suitable candidate for this: the image of
\(\mealytocircuiti\), the map from Mealy machines to circuits.
Circuits in this image are already in normalised Mealy form, so this is a good
start.
However, they are more specialised than that: their internal states are made up
only of \(\bot\) and \(\top\) elements, and the width of the state is equal to
the number of stream derivatives of the original stream function.
We will establish this as our pseudo-normal form for circuits; depending on the
ordering of the states picked, there may be multiple such circuits.

\begin{corollary}
    Given a circuit \(
    \iltikzfig{strings/category/f}[box=f,colour=seq]
    \), there exists \(x \in \nat\) and word \(\listvar{s} \in \{\bot,\top\}^x\)
    and normalised circuit \(
    \iltikzfig{strings/category/f}[box=g,colour=comb]
    \) such that \(
    \circuittostreami[\iltikzfig{strings/category/f}[box=f,colour=seq]]
    =
    \circuittostreami[\iltikzfig{circuits/productivity/mealy-form}[core=|g|,colour=seq]]
    \) by equations in \(
    \mealyequations + \normalisingequations + \encodingequations
    \).
\end{corollary}

By \cref{cor:encoding-circuits}, we have that there is an encoding that maps
state sets between denotationally equivalent circuits.
If we can encode a circuit's state sets to one where the states are all elements
of \(\{\bot,\top\}^x\), does this mean we have a sound and complete equational
theory?
Unfortunately not: all it means is that the circuits agree on the set of
circuit states.

\begin{example}\label{ex:restriction-example}
    Consider the following two circuits in \(\scirc{\belnapsignature}\): \[
        \iltikzfig{circuits/examples/state-change/circuit-mealy}
        \quad
        \iltikzfig{circuits/examples/state-change/circuit-simpler-mealy}
    \]
    Both circuits have circuit states \(
    \{\belnaptrue\belnapfalse,\belnapfalse\belnaptrue\}
    \), but their combinational cores do \emph{not} have the same semantics.
    They only act the same because they receive certain inputs.
\end{example}

The final family of equations required is one for mapping between combinational
circuits that agree on the context, but may differ otherwise.

\begin{notation}
    Given sets \(A\), \(B\) and \(C\) where \(C \subseteq A\) and a function
    \(\morph{f}{A}{B}\), the \emph{restriction of \(f\) to \(C\)} is a function
    \(\morph{f|C}{C}{B}\), defined as \(f|C(c) \coloneqq f(c)\).
\end{notation}

\begin{definition}[Restriction equations]
    Let the schema of \emph{restriction equations} be defined as in
    \cref{fig:restriction-equation}.
\end{definition}

\begin{example}
    By a restriction equation, the circuits in \cref{ex:restriction-example} are
    now equal, as the cores produce equal outputs for inputs where the state is
    \(\belnaptrue\belnapfalse\).
\end{example}
\begin{figure}
    \centering
    \(\equationdisplay{
        \iltikzfig{circuits/productivity/mealy-form}[core=\mealytofunc[f]]
    }{
        \iltikzfig{circuits/productivity/mealy-form}[core=|g|]
    }{\restrictionequation}\)
    \,\,
    \begin{minipage}{0.25\textwidth}
        \centering
        where \(
        f|S_{f,\listvar{s}} \times \valuetuple{\listvar{m}}
        =
        g|S_{f,\listvar{s}} \times \valuetuple{\listvar{m}}
        \)
    \end{minipage}
    \caption{The \emph{restriction} equation}
    \label{fig:restriction-equation}
\end{figure}
\section{Completeness of the algebraic semantics}\label{sec:algebraic-completeness}

It is now possible to collect all the equations together and define a sound and
complete algebraic theory of sequential digital circuits.

\begin{definition}
    \index{\(\scircsigmage\)}
    \nomenclature{\(\mce_\interpretation\)}{complete set of equations on circuits}
    \nomenclature{\(\scircsigmae\)}{PROP of sequential circuits quotiented by equations}
    For a complete interpretation \(\interpretation\), let
    \(\mce_{\interpretation}\) be \(
    \mealyequations +
    \normalisingequations +
    \encodingequations +
    (\restrictionequation)
    \), and let \(\scircsigmae\) be defined as
    \(\scircsigma / \mce_{\interpretation}\).
\end{definition}

For this to be a \emph{complete} set, we must be able to translate
a circuit \(
\iltikzfig{strings/category/f}[box=f,colour=seq,dom=m,cod=n]
\) into another circuit \(
\iltikzfig{strings/category/f}[box=g,colour=seq,dom=m,cod=n]
\) with the same behaviour by only using these equations.

\begin{theorem}
    For a complete interpretation \(\interpretation\), \(
    \iltikzfig{strings/category/f}[box=f,colour=seq,dom=m,cod=n]
    =
    \iltikzfig{strings/category/f}[box=g,colour=seq,dom=m,cod=n]
    \) in \(\scircsigmae\) if and only if \(
    \circuittostreami[
        \iltikzfig{strings/category/f}[box=f,colour=seq]
    ]
    =
    \circuittostreami[
        \iltikzfig{strings/category/f}[box=g,colour=seq]
    ]
    \).
\end{theorem}
\begin{proof}
    All the equations are sound, so we only need to consider the \(\ifdir\)
    direction.
    Using \cref{thm:normalising}, the circuits \(
    \iltikzfig{strings/category/f}[box=f,colour=seq]
    \) and \(
    \iltikzfig{strings/category/f}[box=g,colour=seq]
    \) can be brought to Mealy form, so we have that \(
    \circuittostreami[
        \iltikzfig{circuits/productivity/mealy-form-wide}[core=\mealytofunc[\hat{f}],state=\listvar{s}, colour=seq]
    ]
    =
    \circuittostreami[
        \iltikzfig{circuits/productivity/mealy-form-wide}[core=\mealytofunc[\hat{g}],state=\listvar{t}, colour=seq]
    ]
    \).
    This induces Mealy machines \((S_{\hat{f}, \listvar{s}}, \hat{f})\) and
    \((S_{\hat{g}, \listvar{t}}, \hat{g})\).
    As their stream functions are equal, there are Mealy homomorphisms
    \(\morph{\phi}{S_{\hat{f}, \listvar{s}}}{S_{\hat{g}, \listvar{t}}}\) and
    \(\morph{\psi}{S_{\hat{g}, \listvar{t}}}{S_{\hat{f}, \listvar{s}}}\) and
    subsequently the composite of these homomorphisms is also a Mealy
    homomorphism; these will act as a decoder-encoder pair.

    Using the encoding equation, we have by \cref{thm:encoding} that \[
        \iltikzfig{circuits/productivity/mealy-form-wide}[core=\mealytofunc[\hat{f}],state=\listvar{s}, colour=seq]
        =
        \iltikzfig{circuits/algebraic/state-encoded-theorem}.
    \]
    The circuit \(
    \iltikzfig{circuits/algebraic/state-encoded-theorem-core}
    \) is a composition of normalised circuits, so it is essentially
    combinational; when restricted to the set \(S_{\hat{g}, \listvar{t}}\) its
    truth table is the same as that of \(
    \iltikzfig{strings/category/f-wide}[box=\mealytofunc[\hat{g}], colour=seq]
    \), as the encoder-decoder pair were defined precisely as the Mealy
    homomorphisms that translate between the two Mealy machines.
    Using the normalisation equations again, the encoded circuit can be
    brought into normalised Mealy form.
    Finally, the restriction equations can be used to translate from \(
    \iltikzfig{circuits/algebraic/state-encoded-theorem}
    \) into \(
    \circuittostreami[
        \iltikzfig{circuits/productivity/mealy-form-wide}[core=\mealytofunc[\hat{g}],state=\listvar{t}, colour=seq]
    ]
    \).
\end{proof}

As always, the soundness and completeness of the algebraic semantics means we
can establish another isomorphism of PROPs.

\begin{corollary}
    \(\scircsigmai \cong \scircsigmae\).
\end{corollary}

One might wonder how this improves on the operational approach, as the
normal form is quite complicated.
The beauty of the \emph{algebraic} semantics is that equations can be proven
as lemmas and used in the future as shortcuts; in time, the algebraicist will
build up a repertoire of equations and use them to bend circuits to
their will.