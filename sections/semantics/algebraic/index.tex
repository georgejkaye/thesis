\chapter{Algebraic semantics}\label{chap:algebraic}

The operational semantics and notion of observational equivalence means that
that the behaviour of two circuits can be compared by checking
whether every input produces the same output for both circuits.
But this is somewhat of a nuclear option; rather than using what we know
about the structure of a circuit's components, we just blast away
exhaustively trying all the inputs to find a contradiction.

A more elegant method of reasoning is by defining a set of \emph{equations}
between subcircuits and \emph{quotienting} \(\scircsigma\) by these equations.
A proof of equivalence between two circuits is then presented using algebraic
reasoning: applying equations to translate one circuit into the other.
This is often far more efficient than having test every input, and is
known formally as an \emph{algebraic} semantics.

\begin{example}\label{ex:expressions-algebraic}
    For the last time we return to the language of arithmetical expressions from
    \cref{ex:expressions-denotational}.
    An algebraic semantics for this language can be defined using a set of
    equations: the familiar equations of associativity, commutativity,
    unitality, annihiliation and distributivity, along with equations for
    performing arithmetic.
    \begin{gather*}
        add\,(add\,e_1\,e_2)\,e_3 = add\,e_1\,(add\,e_2\,e_3)
        \qquad
        mul\,(mul\,e_1\,e_2)\,e_3 = mul\,e_1\,(mul\,e_2\,e_3)
        \\
        add\,e_1\,e_2 = add\,e_2\,e_1
        \qquad
        mul\,e_1\,e_2 = mul\,e_2\,e_1
        \\
        add\,e_1\,\overline{0} = e_1
        \qquad
        mul\,e_1\,\overline{1} = e_1
        \qquad
        mul\,e_1\,\overline{0} = \overline{0}
        \\
        mul\,e_1\,(add\,e_2\,e_3) = add\,(mul\,e_1\,e_2)\,(mul\,e_1\,e_3)
        \\
        add\,\overline{n_1}\,\overline{n_2} = \overline{n_1+n_2}
        \qquad
        add\,\overline{n_1}\,\overline{n_2} = \overline{n_1 \cdot n_2}
    \end{gather*}
    If everything is specified concretely as values then one could easily just
    use the last two equations to compare two expressions by reducing
    two expressions to values as in the operational semantics.
    The power of the algebraic semantics comes from the fact we can reason
    abstractly with expressions containing \emph{blackboxes}.
    Take the following example, containing some arbitrary component \(e\).
    \begin{align*}
        mul\,(add\,e\,(mul\,e\,\overline{3}))\,\overline{2}
         & =
        mul\,(add\,(mul\,e\,1)\,(mul\,e\,\overline{3}))\,\overline{2}
        \\
         & =
        mul\,(mul\,e\, add(\overline{1}\,\overline{3}))\,\overline{2}
        \\
         & =
        mul\,(mul \,e\,\overline{4})\,\overline{2}
        \\
         & =
        mul\,e\,(mul \,\overline{4}\,\overline{2})
        \\
         & =
        mul\,e\,\overline{8}
        \\
         & =
        mul\,\overline{8}\,e
    \end{align*}
    Despite not specifying the structure of \(e\), we have
    shown how the expression is equal to a slightly simpler one.
    This process creates new general equations which can be used as `shortcuts'
    in future reasoning, potentially saving many steps.
\end{example}

As with the operational semantics, we are especially interested in defining a
\emph{sound and complete} algebraic semantics with respect to the denotational
semantics.
That is to say, for each equation \(
\iltikzfig{strings/category/f}[box=f,colour=seq]
=
\iltikzfig{strings/category/f}[box=g,colour=seq]
\) then \(
\circuittostreami[\iltikzfig{strings/category/f}[box=f,colour=seq]]
=
\circuittostreami[\iltikzfig{strings/category/f}[box=g,colour=seq]]
\), and there must be enough equations such that if \(
\circuittostreami[\iltikzfig{strings/category/f}[box=f,colour=seq]]
=
\circuittostreami[\iltikzfig{strings/category/f}[box=g,colour=seq]]
\) then there exists a series of equations identifying \(
\iltikzfig{strings/category/f}[box=f,colour=seq]
\) and \(
\iltikzfig{strings/category/f}[box=f,colour=seq]
\).

\begin{remark}
    An `equational theory' for sequential circuits was one of the first things
    presented in the previous
    work~\cite{ghica2016categorical,ghica2017diagrammatic}.
    In that paper, equations that `seemed right' were used to quotient the
    syntax, with the ultimate aim of showing that the resulting category was
    \emph{Cartesian}.
    This was done quite informally, and was made more confusing as
    the categories of circuits were subsequently quotiented by some notion of
    `extensional equivalence', an attempt to rectify the fact that the
    equations only dealt with closed circuits.
    Soundness and completeness of the equational theory was not considered
    because there was nothing to compare it against.

    In essence, the previous work was almost `the wrong way round': equations
    were defined and semantics drawn from them.
    In the more recent version of the work~\cite[Sec.\ 5]{ghica2024fully}, which
    forms the basis for this chapter, the equations are derived from the
    denotational semantics.
    Not only does this give us a formal way of verifying that these equations
    are sound, it sets the backdrop against which we can test if the algebraic
    semantics are sufficient: are any two denotationally equal circuits
    identified by equations?

\end{remark}

When defining such an equational theory, there may be several different sound
and complete formulations.
Ideally, we want to stick to simple \emph{local} equations that concern the
interactions of concrete generators as much as possible, but as we will see
we will sometimes have no choice but to define \emph{families} of equations
parameterised over some arbitrary subcircuit.

\section{Normalising circuits}\label{sec:normalising}

How does one start when trying to define a complete set of equations for some
framework?
Usually the strategy is to define enough equations to bring any term to some
sort of (pseudo-)\emph{normal form}; the theory is then complete if terms with
the same semantics have the same normal form.

We have already seen something that looks a bit like a normal form: the
\emph{Mealy form} from the previous section.
This is by no means a true normal form, as there are many different Mealy forms
that represent the same behaviour.
Nevertheless, it is a useful starting point so we will need equations to bring
circuits to Mealy form in our theory.

Instead of just turning the Mealy reduction rules into equations, we will show
how Mealy form can be derived using smaller equations.

\begin{figure}
    \centering
    \(
    \equationdisplay{
        \iltikzfig{strings/structure/monoid/unitality-l-lhs}
    }{
        \iltikzfig{strings/structure/monoid/unitality-l-rhs}
    }{
        \joinunitleqn
    }
    \qquad
    \equationdisplay{
        \iltikzfig{strings/structure/monoid/unitality-r-lhs}
    }{
        \iltikzfig{strings/structure/monoid/unitality-r-rhs}
    }{
        \joinunitreqn
    }
    \)
    \\[0.25em]
    \rule{\textwidth}{0.1mm}
    \\[0.5em]
    \(
    \equationdisplay{
        \iltikzfig{circuits/axioms/bottom-delay-lhs}
    }{
        \iltikzfig{circuits/axioms/bottom-delay-rhs}
    }{
        \bottomdelayeqn
    }
    \qquad
    \equationdisplay{
        \iltikzfig{circuits/instant-feedback/equation-lhs}[box=f]
    }{
        \iltikzfig{circuits/instant-feedback/equation-rhs}[box=f]
    }{
        \instantfeedbackeqn
    }
    \)
    \\[0.25em]
    \rule{\textwidth}{0.1mm}
    \caption{
        Set of Mealy equations
        \(\mealyequations\).
    }
    \label{fig:mealy-equations}
\end{figure}

\begin{definition}
    The set \(\mealyequations\) of \emph{Mealy equations} in
    \cref{fig:mealy-equations} are sound.
\end{definition}
\begin{proof}
    The first two rules hold as the join is a monoid in the stream semantics.
    The \((\bottomdelayeqn)\) holds because the semantics of the delay
    component are to output a \(\bot\) value first and then the (delayed)
    inputs: as the semantics of the \(
    \iltikzfig{strings/structure/monoid/init}[colour=comb]
    \) component are to \emph{always} produce \(\bot\), then it does not make a
    difference how delayed it is.
    The final equation is the instant feedback rule, which is sound by
    \cref{prop:instant-feedback}.
\end{proof}

\begin{proposition}\label{prop:mealy-equations}
    Given a sequential circuit \(
    \iltikzfig{strings/category/f}[box=f,colour=seq,dom=m,cod=n]
    \), there exists a combinational circuit \(
    \iltikzfig{strings/category/f-2-2}[box=g,colour=comb,dom1=x,dom2=m,cod1=x,cod2=n]
    \) and values \(\listvar{s} \in \valuetuple{x}\) such that \(
    \iltikzfig{strings/category/f}[box=f,colour=seq]
    =
    \iltikzfig{circuits/productivity/mealy-form}[core=g]
    \) in \(\scircsigma / \mealyequations\).
\end{proposition}
\begin{proof}
    Any circuit can be assembled into global trace-delay form solely using the
    axioms of STMCs.
    From this, a circuit in pre-Mealy form can be obtained by translating
    delays and values into registers using the following equations: \[
        \iltikzfig{circuits/axioms/delay-to-register/step-0}
        \eqaxioms[(\joinunitleqn)]
        \iltikzfig{circuits/axioms/delay-to-register/step-1}
        \qquad\quad
        \iltikzfig{circuits/axioms/value-to-register/step-0}
        \eqaxioms[(\joinunitreqn)]
        \iltikzfig{circuits/axioms/value-to-register/step-1}
        \eqaxioms[(\bottomdelayeqn)]
        \iltikzfig{circuits/axioms/value-to-register/step-2}
    \]
    Subsequently a circuit in Mealy form can be obtained by applying the
    \((\instantfeedbackeqn)\) rule.
\end{proof}

\(\scircsigma / \mealyequations\) is a category in which all circuits are equal
to at least one circuit in Mealy form.
In general, there will be many Mealy forms depending on the ordering one picks
for the delays and value; our task is to provide equations to map any two
denotationally equivalent circuits to the \emph{same} Mealy form.

Even if the combinational cores of two Mealy forms have the same behaviour, they
may not have the same structure.
To reduce the number of cores we have to consider, we will first establish
equations for translating any combinational circuit into some canonical circuit.
We already met a method for determining what this canonical circuit is: the
functional completeness map \(\mealytofunc\) from \(\funci\) to \(\scircsigma\).

\begin{definition}[Normalised circuit]
    A sequential circuit \(
    \iltikzfig{strings/category/f}[box=f,colour=seq,dom=m,cod=n]
    \) is \emph{normalised} if it is in the image of \(\mealytofunc\).
\end{definition}

As a shorthand, we will often abuse notation and write \(
\mealytofunc[f]
\coloneqq
\iltikzfig{strings/category/f-wide}[box=\mealytofunc[f],colour=seq]
\).
Recall that even though \(\mealytofunc\) maps into \(\scircsigma\), every
circuit in its image has combinational behaviour.
This is quite an important distinction to make, so we will give it a proper
name.

\begin{lemma}[Essentially combinational]
    A sequential circuit is \emph{essentially combinational} if it is in the
    form \(
    \iltikzfig{circuits/synthesis/normalised-function}[box=f]
    \).
\end{lemma}

Such circuits are sequential circuits that exhibit combinational behaviour: any
value components are only used to introduce constants which do not alter over
time.

As the normalised version of a given circuit is interpretation-dependent, there
is no standard set of equations for normalising a circuit.
Instead, these must be specified on an interpretation-by-interpretation basis.

\begin{definition}[Normalising equations]
    For a complete interpretation \((\interpretation,\mealytofunc)\), a set of
    equations \(\normalisingequations\) is \emph{normalising} if any
    essentially combinational circuit \(
    \iltikzfig{strings/category/f}[box=f,colour=seq,dom=m,cod=n]
    \) is equal to a circuit in the image of \(\mealytofunc\) by equations in
    \(\normalisingequations\).
\end{definition}

\begin{definition}[Normalisable interpretation]
    A complete interpretation \(\interpretation\) is called \emph{normalisable} if there
    exists a set of normalising equations \(\normalisingequations\).
\end{definition}

The normalising equations for a given interpretation can be used to translate a
combinational core into a circuit from which it is easy to read off a truth
table.

\begin{theorem}\label{thm:normalising}
    For every sequential circuit \(
    \iltikzfig{strings/category/f}[box=f,colour=seq,dom=m,cod=n]
    \) in a normalisable complete interpretation
    \((\interpretation,\mealytofunc)\) over \(\Sigma\), there exists essentially
    combinational \(
    \iltikzfig{strings/category/f-2-2-wide}[box=\mealytofunc[g],colour=seq,dom1=x,dom2=m,cod1=x,cod2=n]
    \) and \(\listvar{s} \in \valuetuple{x}\) such that \(
    \iltikzfig{strings/category/f}[box=f,colour=seq,dom=m,cod=n]
    =
    \iltikzfig{circuits/productivity/mealy-form-wide}[core=\mealytofunc[g],colour=seq,state=\listvar{s}]
    \) in \(\scircsigma / \mealyequations + \normalisingequations\).
\end{theorem}
\begin{proof}
    By \cref{prop:mealy-equations}, \(
    \iltikzfig{strings/category/f}[box=f,colour=seq,dom=m,cod=n]
    =
    \iltikzfig{circuits/productivity/mealy-form}[core=h]
    \) and by equations in \(\normalisingequations\), \(
    \iltikzfig{strings/category/f-2-2}[box=h,colour=comb]
    =
    \iltikzfig{strings/category/f-2-2-wide}[box=\mealytofunc[g],colour=seq]
    \).
\end{proof}
\section{Encoding equations}

A circuit in Mealy form is a syntactic representation of a Mealy machine: it
has a combinational core acting as the Mealy machine. and registers containing
the initial state of the machine.
As we saw in the operational semantics, as a circuit is applied to inputs these
registers are updated with new values as the circuit transitions to new states.
When reasoning algebraically, we cannot evaluate these states as we have no
inputs to compute with.
However, the states a circuit \emph{might} assume will still be important as
they dictate whether an equation is valid.

\begin{definition}[States]
    Let \(\morph{f}{\valuetuple{x+m}}{\valuetuple{x+n}}\) be a
    monotone function and let \(\listvar{s} \in  \valuetuple{x}\) be an
    initial state.
    Then the \emph{states of \(f\) from \(\listvar{s}\)}, denoted
    \(S_{f,\listvar{s}}\), is the smallest set containing \(\listvar{s}\) and
    closed under \(
    \listvar{r}
    \mapsto
    \proj{x}\left(\tilde{f}_0(\listvar{r},\listvar{v})\right)
    \) for any \(\listvar{v} \in \valuetuple{m}\).
\end{definition}

\begin{example}\label{ex:circuit-states}
    Consider the circuit \(
    \iltikzfig{circuits/examples/bottrue/circuit}
    \).
    The semantics of the combinational core are clearly
    \(sr \mapsto (s \lor r)s(s \lor r)\), where the first two characters are the
    next state and the third is the output.
    The initial state is \(\bot\belnaptrue\), so the subsequent states are
    \((\bot \lor \belnaptrue)\bot = \belnaptrue\bot\) and
    \((\belnaptrue \lor \bot)\belnaptrue = \belnaptrue\belnaptrue\).
    As \((\belnaptrue \lor \belnaptrue)\belnaptrue = \belnaptrue\belnaptrue\),
    there are no more circuit states and the complete set is
    \(\{\bot\belnaptrue,\belnaptrue\bot,\belnaptrue\belnaptrue\}\).

    Note that as the output of the circuit is computed as \(s \lor r\), for each
    circuit state the output is \(\belnaptrue\).
    This means that the circuit is denotationally equivalent to \(
    \iltikzfig{circuits/components/waveforms/infinite-register}[val=\belnaptrue]
    \), but this circuit only has a single state \(\belnaptrue\).
\end{example}

We need to \emph{encode} the states of one circuit as another; we have already
encountered this notion using \emph{Mealy homomorphisms}
(\cref{def:mealy-homomorphism});
functions between the state sets that preserve transitions and outputs.
While two `inverse' homomorphisms may not be isomorphisms, the round
trip will always map to a state with the same behaviour.

\begin{lemma}
    Given two Mealy homomorphisms \(\morph{h}{(S,f)}{(T,g)}\) and
    \(\morph{h^\prime}{(T,g)}{(S,f)}\), any state \(s \in S\) and input
    \(a \in A\), \(
    \mealyfunctiontransition{f}(s, a)
    =
    \mealyfunctiontransition{f}(h^\prime(h(s)), a)
    \).
\end{lemma}
\begin{proof}
    Immediate as Mealy homomorphisms preserve outputs.
\end{proof}

Pairs of `inverse' Mealy homomorphisms as described above will act as state
encoders and decoders between circuits.
To create circuits representing these homomorphisms, we once again use the
functional completeness map for an interpretation, which assigns a syntactic
circuit to any monotone (combinational) function.

\begin{lemma}
    For partial orders \(S\) and \(T\) and monotone Mealy coalgebra
    \((S,f)\) and \((T,g)\), any Mealy homomorphism \(\morph{h}{(S,f)}{(T,g)}\)
    is monotone.
\end{lemma}
\begin{proof}
    In a monotone Mealy coalgebra, the functions \(f\) and \(g\) are monotone,
    and for \(h\) to be a Mealy homomorphism, \(
    \mealyfunctionoutput{f}(s)
    =
    \mealyfunctionoutput{g}(h(s))
    \).
    For states \(s,r \in S\) we have \(
    \mealyfunctionoutput{g}(h(s), a)
    =
    \mealyfunctionoutput{f}(s, a)
    \leq
    \mealyfunctionoutput{f}(r, a)
    =
    \mealyfunctionoutput{g}(h(r), a)
    \).
    This means that the function \(
    s \mapsto \mealyfunctionoutput{g}(h(s), a)
    \) is monotone; as \(\mealyfunctionoutput{g}\) is monotone, \(h\) must
    also be monotone.
\end{proof}

For two circuits \(
\iltikzfig{circuits/productivity/mealy-form-wide}[core=\mealytofunc[f],state=\listvar{s},colour=seq,dom=m,cod=n,delay=x]
\) and \(
\iltikzfig{circuits/productivity/mealy-form-wide}[core=\mealytofunc[g],state=\listvar{t},colour=seq,dom=m,cod=n,delay=y]
\), the encoders and decoders we will use for circuits will be homomorphisms
\((S_{f,\listvar{s}},f) \to (S_{g,\listvar{t}},g)\)
and
\((S_{g,\listvar{t}},g) \to (S_{f,\listvar{s}},f)\).


\begin{proposition}
    For two denotationally equivalent circuits \(
    \iltikzfig{circuits/productivity/mealy-form-wide}[core=\mealytofunc[f],state=\listvar{s},colour=seq,dom=m,cod=n,delay=x]
    \) and \(
    \iltikzfig{circuits/productivity/mealy-form-wide}[core=\mealytofunc[g],state=\listvar{t},colour=seq,dom=m,cod=n,delay=y]
    \), there exists at least one Mealy homomorphism \(
    \morph{h}{(S_{f,\listvar{s}},f)}{(S_{g,\listvar{t}},g)}
    \).
\end{proposition}
\begin{proof}
    As the two circuits are denotationally equivalent, at least one valid
    mapping of states can be obtained by sending states in the former to the
    state in the latter obtained after inputting the same word.
    However, there will be more than one valid homomorphism if the second
    circuit has more states than the first, as there will be multiple states in
    the second that act like a single state in the first.
\end{proof}

These are homomorphisms on the subset of states that a circuit
can assume, \emph{not} the entire set of words that can fit into the state!
This means that encoding and decoding circuits cannot be inserted arbitrarily
but only in certain contexts.

\begin{proposition}[Encoding equation]\label{prop:encoding-equation}
    For a normalised circuit \(
    \iltikzfig{strings/category/f-2-2-wide}[box=\mealytofunc[f],colour=seq,dom1=x,dom2=m,cod1=x,cod2=y]
    \) and \(\listvar{s} \in \valuetuple{x}\), let
    \(\morph{\mathsf{enc}}{S_{f,\listvar{s}}}{\valuetuple{y}}\) and
    \(\morph{\mathsf{dec}}{\valuetuple{y}}{S_{f, \listvar{s}}}\) be Mealy
    homomorphisms.
    Then the \emph{encoding equation} \((\encodingequation)\) in
    \cref{fig:encoding-equation} is sound, where
    \(\mathsf{enc}_\mathsf{m},\mathsf{dec}_\mathsf{m}\) are monotone completions
    as defined in \cref{def:monotone-completion}.
\end{proposition}
\begin{proof}
    Let \(g\) be the map \(\listvar{r} \mapsto
    \circuittostreami[\iltikzfig{circuits/algebraic/state-encoding}[core=\mealytofunc[f],delay=x,state=\listvar{r}]]
    \); by \cref{prop:mealy-form-image} we know that \(
    \mealyoutput{g(\listvar{t})}{\listvar{v}}
    =
    \proj{1}(f(\mathsf{dec}(\mathsf{enc}(\listvar{t})), \listvar{v}))
    \) and \(
    \mealytransition{g(\listvar{t})}{\listvar{v}}
    =
    g(\proj{0}(f(\mathsf{dec}(\mathsf{enc}(\listvar{t})), \listvar{v})))
    \).
    When \(\listvar{t} \in S_{f, \listvar{s}}\) we have that \(
    \mathsf{dec}(\mathsf{enc}(\listvar{s}))\) is an equivalent state to
    \(\listvar{s}\),
    so
    \(
    \mealyoutput{g(\listvar{t})}{\listvar{v}}
    =
    \proj{1}(f(\listvar{t}), \listvar{v})
    \) and \(
    \mealytransition{g(\listvar{t})}{\listvar{v}}
    \) is equivalent to \(
    g(\proj{0}(f(\listvar{t})), \listvar{v})
    \) by definition of encoders.
    As \(
    \iltikzfig{circuits/algebraic/state-encoding}[core=\mealytofunc[f],delay=x,state=\listvar{s}]
    \coloneqq
    g(\listvar{s})
    \) and \(\listvar{s} \in S_{f,\listvar{s}}\),
    every subsequent stream derivative will also be of the form
    \(g(\listvar{t})\) where \(\listvar{t} \in S_{f,\listvar{s}}\), so the
    equation is sound.
\end{proof}

\begin{remark}
    The encoding equation is an equation \emph{schema}: this is required because
    the width of a circuit state can be arbitrarily large, and each extra bit
    adds a whole new set of Mealy homomorphisms to consider.
\end{remark}

\begin{figure}
    \centering
    \(
    \equationdisplay{
        \iltikzfig{circuits/productivity/mealy-form-wide}[core=\mealytofunc[f],delay=x, colour=seq]
    }{
        \iltikzfig{circuits/algebraic/state-encoding}[core=\mealytofunc[f],delay=x]
    }{
        \encodingequation
    }
    \)
    \\[0.25em]
    \rule{\textwidth}{0.1mm}
    \\[0.5em]
    \(
    \equationdisplay{
        \iltikzfig{circuits/axioms/gate-lhs}[gate=p]
    }{
        \iltikzfig{circuits/axioms/gate-rhs}[gate=p]
    }{
        \gateeqn
    }
    \quad
    \equationdisplay{
        \iltikzfig{circuits/axioms/fork-lhs}
    }{
        \iltikzfig{circuits/axioms/fork-rhs}
    }{
        \forkeqn
    }
    \quad
    \equationdisplay{
        \iltikzfig{circuits/axioms/join-lhs}
    }{
        \iltikzfig{circuits/axioms/join-rhs}
    }{
        \joineqn
    }
    \)
    \\[0.25em]
    \rule{\textwidth}{0.1mm}
    \\[0.5em]
    \(
    \equationdisplay{
        \iltikzfig{circuits/axioms/stub-lhs}
    }{
        \iltikzfig{strings/monoidal/empty}
    }{
        \stubeqn
    }
    \quad
    \equationdisplay{
        \iltikzfig{circuits/axioms/delay-fork-lhs}
    }{
        \iltikzfig{circuits/axioms/delay-fork-rhs}
    }{
        \delayforkeqn
    }
    \quad
    \equationdisplay{
        \iltikzfig{circuits/axioms/bottom-delay-lhs}
    }{
        \iltikzfig{circuits/axioms/bottom-delay-rhs}
    }{
        \bottomdelayeqn
    }
    \)
    \\[0.25em]
    \rule{\textwidth}{0.1mm}
    \\[0.5em]
    \(
    \equationdisplay{
        \iltikzfig{circuits/axioms/streaming-lhs}
    }{
        \iltikzfig{circuits/axioms/streaming-rhs}[gate=p]
    }{
        \streamingeqn
    }
    \quad
    \equationdisplay{
        \iltikzfig{strings/structure/comonoid/unitality-l-lhs}
    }{
        \iltikzfig{strings/structure/comonoid/unitality-l-rhs}
    }{
        \forkuniteqn
    }
    \quad
    \equationdisplay{
        \iltikzfig{strings/structure/monoid/associativity-lhs}
    }{
        \iltikzfig{strings/structure/monoid/associativity-rhs}
    }{
        \joinassoceqn
    }
    \)
    \\[0.25em]
    \rule{\textwidth}{0.1mm}
    \\[0.5em]
    \(
    \equationdisplay{
        \iltikzfig{strings/structure/monoid/commutativity-lhs}
    }{
        \iltikzfig{strings/structure/monoid/commutativity-rhs}
    }{
        \joincommeqn
    }
    \quad
    \equationdisplay{
        \iltikzfig{strings/structure/bialgebra/merge-copy-lhs}
    }{
        \iltikzfig{strings/structure/bialgebra/merge-copy-rhs}
    }{
        \joinforkeqn
    }
    \)
    \caption{
        Set \(\encodingequations\) of equations for encoding circuit states
    }
    \label{fig:encoding-equation}
\end{figure}

After applying the encoding equation, the state has not actually changed; we
have just inserted a pair of an encoder and a decoder circuit.
To translate the state we recycle some of the rules from the operational
semantics.
As with the Mealy equations, it is desirable to express these transformations in
terms of smaller components

\begin{lemma}
    The equations on the bottom three rows of \cref{fig:encoding-equation} are
    sound.
\end{lemma}
\begin{proof}
    It is a straightforward exercise to compare the stream functions.
\end{proof}

Using the encoding equations we can derive some useful results on larger
circuits.

\begin{lemma}\label{lem:unroll-waveform}
    For a value \(v \in \values\), \(
    \iltikzfig{circuits/algebraic/unroll-waveform/step-0}[value=v]
    =
    \iltikzfig{circuits/algebraic/unroll-waveform/step-6}[value=v]
    \) by the encoding equations.
\end{lemma}
\begin{proof}
    The proof is straightforward:
    \begin{gather*}
        \iltikzfig{circuits/algebraic/unroll-waveform/step-0}[value=v]
        \coloneqq
        \iltikzfig{circuits/algebraic/unroll-waveform/step-1}[value=v]
        \eqaxioms[(\joinforkeqn)]
        \iltikzfig{circuits/algebraic/unroll-waveform/step-2}[value=v]
        \eqaxioms[(\forkeqn)]
        \iltikzfig{circuits/algebraic/unroll-waveform/step-3}[value=v]
        \eqaxioms[(\delayforkeqn)]
        \\
        \iltikzfig{circuits/algebraic/unroll-waveform/step-4}[value=v]
        \coloneqq
        \iltikzfig{circuits/algebraic/unroll-waveform/step-5}[value=v]
        =
        \iltikzfig{circuits/algebraic/unroll-waveform/step-6}[value=v]
    \end{gather*}
    This completes the proof.
\end{proof}

\begin{lemma}\label{lem:generalised-streaming}
    For a combinational circuit \(
    \iltikzfig{strings/category/f}[box=f,colour=comb]
    \), \(
    \iltikzfig{circuits/axioms/generalised-streaming-lhs}[box=f]
    =
    \iltikzfig{circuits/axioms/generalised-streaming-rhs}[box=f]
    \) by the encoding equations.
\end{lemma}
\begin{proof}
    This is by induction on the structure of \(
    \iltikzfig{strings/category/f}[box=f,colour=comb]
    \).
    First the base cases.
    The case for the gate is immediate by \((\streamingeqn)\).
    For \(\iltikzfig{strings/structure/comonoid/copy}[colour=comb]\) we have
    that \(
    \iltikzfig{circuits/productivity/generalised-streaming/fork-step-0}
    \eqaxioms[(\joinforkeqn)]
    \iltikzfig{circuits/productivity/generalised-streaming/fork-step-1}
    \eqaxioms[(\delayforkeqn)]
    \iltikzfig{circuits/productivity/generalised-streaming/fork-step-2}
    \).
    For \(\iltikzfig{strings/structure/monoid/merge}[colour=comb]\):
    \begin{gather*}
        \iltikzfig{circuits/productivity/generalised-streaming/join-step-2}
        \eqaxioms[(\monoidassoceqnletter)]
        \iltikzfig{circuits/productivity/generalised-streaming/join-step-3}
        \eqaxioms[(\monoidassoceqnletter)]
        \iltikzfig{circuits/productivity/generalised-streaming/join-step-4}
        \eqaxioms[(\monoidcommeqnletter)]
        \iltikzfig{circuits/productivity/generalised-streaming/join-step-5}
        =
        \\
        \iltikzfig{circuits/productivity/generalised-streaming/join-step-6}
        \eqaxioms[(\monoidassoceqnletter)]
        \iltikzfig{circuits/productivity/generalised-streaming/join-step-7}
        \eqaxioms[(\monoidassoceqnletter)]
        \iltikzfig{circuits/productivity/generalised-streaming/join-step-8}
        \eqaxioms[(\monoidassoceqnletter)]
        \iltikzfig{circuits/productivity/generalised-streaming/join-step-9}
    \end{gather*}
    The case for \(\iltikzfig{strings/structure/comonoid/discard}[colour=comb]\) is
    trivial, and the case for \(\iltikzfig{strings/structure/monoid/init}[colour=comb]\)
    follows by \((\comonoiduniteqnletter)\) and \((\bottomdelayeqn)\).
    The cases for \(\iltikzfig{strings/category/identity}[colour=comb]\) and
    \(\iltikzfig{strings/symmetric/symmetry}[colour=comb]\) follow by axioms of STMCs.
    Since the underlying circuit is combinational, for the inductive cases we just
    need to check composition and tensor, which are also trivial.
\end{proof}

\begin{lemma}\label{lem:interpretation-combinational}
    Let \(\iltikzfig{strings/category/f}[box=f,colour=comb]\) be a combinational
    circuit such that \(
    \circuittofunci[\iltikzfig{strings/category/f}[box=f,colour=comb]]
    =
    g\).
    Then \(\circuittostreami[
        \iltikzfig{circuits/components/circuits/f-applied}[box=f,colour=comb]
    ]
    =
    g(\listvar{v}) \streamcons \bot \streamcons \bot \streamcons \dots\).
\end{lemma}
\begin{proof}
    By definition of \(\circuittostreami\), \(
    \circuittostreami[
        \iltikzfig{circuits/components/circuits/f-applied}[box=f,colour=comb]
    ]
    =
    \mealytostreami[\circuittomealyi[
            \iltikzfig{circuits/components/values/vs}[val=\listvar{v}]
        ]] \seq \mealytostreami[
        \circuittomealyi[
            \iltikzfig{strings/category/f}[box=f,colour=comb]]
    ]
    \).
    By \cref{lem:sequential-combinational-semantics} we know that \(
    \mealytostreami[
        \circuittomealyi[
            \iltikzfig{strings/category/f}[box=f,colour=comb]]
    ](\sigma)(i) = \circuittofunci[
        \iltikzfig{strings/category/f}[box=f,colour=comb]
    ] = g(\sigma)(i)\) for all \(\sigma \in \valuetuplestream{m}\) and
    \(i \in \nat\).
    Since \(\mealytostreami[\circuittomealyi[
            \iltikzfig{circuits/components/values/vs}[val=\listvar{v}]
        ]] = \listvar{v} \streamcons \bot \streamcons \bot \streamcons \dots\), we
    have that \(
    \circuittostreami[
        \iltikzfig{circuits/components/circuits/f-applied}[box=f,colour=comb]
    ]
    =
    g(\listvar{v}) \streamcons \bot \streamcons \bot \streamcons \dots
    \) as desired.
\end{proof}

\begin{lemma}\label{lem:combinational-circuit-inputs}
    Let \(\iltikzfig{strings/category/f}[box=f,colour=comb,dom=m,cod=n]\) be a
    combinational circuit such that \(
    \circuittofunci[\iltikzfig{strings/category/f}[box=f,colour=comb]]
    =
    g
    \).
    Then \(
    \iltikzfig{circuits/components/circuits/f-applied}[box=f,colour=comb]
    =
    \iltikzfig{circuits/components/values/vs-even-longer}[val=g(\listvar{v})]
    \) by the encoding equations.
\end{lemma}
\begin{proof}
    For the same reasoning as \cref{lem:reduce-core-terminating}, the
    \((\gateeqn)\), \((\forkeqn)\), \((\joineqn)\) and \((\stubeqn)\) equations
    can be used to show that there exists \(\listvar{w} \in \valuetuple{n}\)
    such that \(
    \iltikzfig{strings/category/f}[box=f,colour=comb,dom=m,cod=n]
    =
    \iltikzfig{circuits/components/values/vs}[val=\listvar{w}]
    \).
    By \cref{lem:interpretation-combinational}, \(
    \circuittostreami[
        \iltikzfig{circuits/components/circuits/f-applied}[box=f,colour=comb]
    ]
    =
    g(\listvar{v}) \streamcons \bot \streamcons \bot
    =
    \circuittostreami[
        \iltikzfig{circuits/components/values/vs-even-longer}[val=g(\listvar{v})]
    ]
    \).
    As the equations are sound they must preserve the stream semantics, so
    \(\listvar{w} = g(\listvar{v})\).
\end{proof}

\begin{lemma}\label{lem:essentially-combinational-applied}
    Let \(\morph{f}{\valuetuple{m}}{\valuetuple{n}}\) be a monotone function
    such that \(
    \iltikzfig{strings/category/f-wide}[box=\mealytofunc[f],colour=seq]
    \coloneqq
    \iltikzfig{circuits/synthesis/normalised-function}[box=g]
    \).
    Then \(
    \iltikzfig{circuits/algebraic/normalised-inputs}[box=g,input1=\listvar{v},input2=\listvar{w}]
    =
    \iltikzfig{circuits/components/values/vs-even-longer}[val=f(\listvar{w})]
    \).
\end{lemma}
\begin{proof}
    Let \(h \coloneqq \circuittofunci[
        \iltikzfig{strings/category/f-2-1}[box=g,colour=comb]
    ]\); using \cref{lem:combinational-circuit-inputs}, we have that \(
    \iltikzfig{circuits/algebraic/normalised-inputs}[box=g,input1=\listvar{v},input2=\listvar{w}]
    =
    \iltikzfig{circuits/components/values/vs-really-long}[val={h(\listvar{v},\listvar{w})}]
    \).
    So we must show that \(f(\listvar{w}) = h(\listvar{v}, \listvar{w})\).
    \begin{align*}
        f(\listvar{w})
         & =
        \circuittostreami[
            \iltikzfig{strings/category/f-wide}[box=\mealytofunc[f],colour=seq]
        ](\listvar{w} \streamcons \bot^\omega)(0)
         &
        \text{\cref{def:functional-completeness}}
        \\
         & \coloneqq
        \circuittostreami[
            \iltikzfig{circuits/synthesis/normalised-function}[box=g]
        ](\listvar{w} \streamcons \bot^\omega)(0)
        \\
         & =
        \circuittostreami[
            \iltikzfig{strings/category/f-2-1}[box=g,colour=comb]
        ](\listvar{v}^\omega, \listvar{w} \streamcons \bot^\omega)(0)
        \\
         & =
        \circuittofunci[
            \iltikzfig{strings/category/f-2-1}[box=g,colour=comb]
        ](\listvar{v}, \listvar{w})
         &
        \text{\cref{lem:sequential-combinational-semantics}}
        \\
         & =
        h(\listvar{v}, \listvar{w})
    \end{align*}
    This completes the proof.
\end{proof}

We will now show that the encoding equations allow us to translate a circuit
into one with an encoded state.

\begin{theorem}
    For a circuit \(
    \iltikzfig{strings/category/f-2-2-wide}[box=\mealytofunc[f],colour=seq,dom1=x,dom2=m,cod1=x,cod2=y]
    \) and initial state \(\listvar{s} \in \valuetuple{x}\), the
    equation \(
    \iltikzfig{circuits/productivity/mealy-form-wide}[core=\mealytofunc[f], colour=seq]
    =
    \iltikzfig{circuits/algebraic/state-encoded}[core=\mealytofunc[f],state=\listvar{s}]
    \) is derivable by the equations in \(\encodingequations\).
\end{theorem}
\begin{proof}
    We have that \(
    \iltikzfig{circuits/productivity/mealy-form-wide}[core=\mealytofunc[f],delay=x, colour=seq]
    =
    \iltikzfig{circuits/algebraic/state-encoding}[core=\mealytofunc[f],delay=x,state=\listvar{s}]
    \) by the \((\encodingequation)\) equation; we need to `push' the encoder \(
    \iltikzfig{circuits/algebraic/encoder}
    \) through the state.
    Although the encoder is sequential, by the definition of \(\mealytofunc\),
    it must be of the form \(
    \iltikzfig{circuits/synthesis/normalised-function}[box=g]
    \) by definition of functional completeness.
    \begin{gather*}
        \iltikzfig{circuits/algebraic/encoding-state/step-0a}[box=g]
        \coloneqq
        \iltikzfig{circuits/algebraic/encoding-state/step-0}[box=g]
        \eqaxioms[\text{(\cref{lem:unroll-waveform})}]
        \iltikzfig{circuits/algebraic/encoding-state/step-1}[box=g]
        \eqaxioms[\text{(\cref{lem:generalised-streaming})}]
        \\[0.5em]
        \iltikzfig{circuits/algebraic/encoding-state/step-2}[box=g]
        \eqaxioms[\text{(\cref{lem:essentially-combinational-applied})}]
        \iltikzfig{circuits/algebraic/encoding-state/step-3}[box=g]
        \coloneqq
        \\[0.5em]
        \iltikzfig{circuits/algebraic/encoding-state/step-4}[box=g]
        \coloneqq
        \iltikzfig{circuits/algebraic/encoding-state/step-5}[box=g]
    \end{gather*}
    The proof is completed by sliding the encoder around the trace.
\end{proof}

Before proceeding to our next set of equations, we will show how the encoding
equations can be used to encode the states of any circuit as those of a
denotationally equivalent one.
We are now ready to show that the encoding equations allow us to translate the
states of a circuit into those of a denotationally equivalent circuit.

\begin{corollary}
    Given two denotationally equivalent circuits \(
    \iltikzfig{circuits/productivity/mealy-form-wide}[core=\mealytofunc[f],state=\listvar{s},colour=seq,dom=m,cod=n,delay=x]
    \) and \(
    \iltikzfig{circuits/productivity/mealy-form-wide}[core=\mealytofunc[g],state=\listvar{t},colour=seq,dom=m,cod=n,delay=y]
    \), there exists \(
    \iltikzfig{circuits/productivity/mealy-form-wide}[core=\mealytofunc[h],state=\listvar{t},colour=seq,dom=m,cod=n,delay=x]
    \) such that \(
    \iltikzfig{circuits/productivity/mealy-form-wide}[core=\mealytofunc[f],state=\listvar{s},colour=seq,dom=m,cod=n,delay=x]
    =
    \iltikzfig{circuits/productivity/mealy-form-wide}[core=\mealytofunc[h],state=\listvar{t},colour=seq,dom=m,cod=n,delay=x]
    \) in \(\scircsigma / \encodingequations\).
\end{corollary}

If we have the right encoders, we can translate the initial state of a circuit
into a different word, and end up with a new circuit in a `Mealy form' with a
sequential core.
However, since the encoders, decoders, and core are essentially combinational,
these can be normalised once again to obtain a new normalised core.
\section{Restriction equations}\label{sec:restriction}

We can now map the state set of one circuit to another using encodings.
Does this mean that the two circuits will now be structurally equal?
Unfortunately not: all it means is that the circuits agree on the set of
circuit states.

\begin{example}\label{ex:restriction-example}
    Consider the following two circuits in \(\scirc{\belnapsignature}\): \[
        \iltikzfig{circuits/examples/state-change/circuit-mealy}
        \quad
        \iltikzfig{circuits/examples/state-change/circuit-simpler-mealy}
    \]
    Both circuits have circuit states \(\{\belnaptrue\belnapfalse\}\), but their
    combinational cores do \emph{not} have the same semantics.
    They only act the same because they receive certain inputs.
\end{example}

The final family of equations required is one for mapping between combinational
circuits that agree on the subset of possible inputs they actually receive.

\begin{notation}
    Given sets \(A\), \(B\) and \(C\) where \(C \subseteq A\) and a function
    \(\morph{f}{A}{B}\), the \emph{restriction of \(f\) to \(C\)} is a function
    \(\morph{f|C}{C}{B}\), defined as \(f|C(c) \coloneqq f(c)\).
\end{notation}

\begin{definition}[Restriction equations]
    Let the schema of \emph{restriction equations} be defined as in
    \cref{fig:restriction-equation}.
\end{definition}

\begin{example}
    By a restriction equation, the circuits in \cref{ex:restriction-example} are
    now equal, as the cores produce equal outputs for inputs where the state is
    \(\belnaptrue\belnapfalse\).
\end{example}
\begin{figure}
    \centering
    \(\equationdisplay{
        \iltikzfig{circuits/productivity/mealy-form-wide}[core=\mealytofunc[f]]
    }{
        \iltikzfig{circuits/productivity/mealy-form-wide}[core=\mealytofunc[g]]
    }{\restrictionequation}\)
    \,\,
    where \(
    f|(S_{f,\listvar{s}} \times \valuetuple{\listvar{m}})
    =
    g|(S_{g,\listvar{s}} \times \valuetuple{\listvar{m}})
    \)
    \caption{The schema of \emph{restriction} equations}
    \label{fig:restriction-equation}
\end{figure}
\section{Completeness of the algebraic semantics}

It is now possible to collect all the equations together and define a sound and
complete algebraic theory of sequential digital circuits.
We start by defining a category of syntactic circuits quotiented by all the
equations we have seen so far.

\begin{definition}
    For an interpretation \(\interpretation\), let
    \(\mce_{\interpretation}\) be \(
    \mealyequations +
    \normalisingequations +
    \encodingequations +
    (\restrictionequation)
    \), and let \(\scircsigmae\) be defined as
    \(\scircsigma / \mce_{\interpretation}\).
\end{definition}

\begin{theorem}
    For a functionally complete interpretation \(\interpretation\), \(
    \iltikzfig{strings/category/f}[box=f,colour=seq,dom=m,cod=n]
    =
    \iltikzfig{strings/category/f}[box=g,colour=seq,dom=m,cod=n]
    \) in \(\scircsigmae\) if and only if \(
    \circuittostreami[
        \iltikzfig{strings/category/f}[box=f,colour=seq]
    ]
    =
    \circuittostreami[
        \iltikzfig{strings/category/f}[box=g,colour=seq]
    ]
    \).
\end{theorem}
\begin{proof}
    All the equations are sound, so we only need to consider the \(\ifdir\)
    direction.

    By \cref{thm:normalising}, the two circuits can be brought to
    normalised Mealy form using
    \(
    \mealyequations +
    \instantfeedbackeqn +
    \normalisingequations
    \).
    By \cref{def:mealy-to-circuit} and
    \cref{thm:circuit-stream-correspondence} there must exist a circuit \(
    \iltikzfig{strings/category/f}[box=h,colour=seq]
    \coloneqq
    \iltikzfig{circuits/algebraic/encoding}[transition={h_0},output={h_1},state={\listvar{s}}]
    \) such that \(
    \circuittostream[
        \iltikzfig{strings/category/f}[box=f,colour=seq]
    ]{\interpretation}
    =
    \circuittostream[
        \iltikzfig{strings/category/f}[box=h,colour=seq]
    ]{\interpretation}
    =
    \circuittostream[
        \iltikzfig{strings/category/f}[box=g,colour=seq]
    ]{\interpretation}
    \).
    The circuit \(
    \iltikzfig{strings/category/f}[box=h,colour=seq]
    \) is encoded such that the state words are in the image of a Mealy encoding
    \(\gamma_\leq\); applying the encoding equation with \(\gamma_\leq\) to the
    normalised Mealy forms obtained above will yield the circuit \(
    \iltikzfig{strings/category/f}[box=h,colour=seq]
    \).
\end{proof}

As always, the soundness and completeness of the algebraic semantics means we
can establish another isomorphism of PROPs.

\begin{corollary}
    \(\scircsigmai \cong \scircsigmae\).
\end{corollary}

This means that given a circuit \(
\iltikzfig{strings/category/f}[box=f,colour=seq,dom=m,cod=n]
\) we know that we can translate it into another circuit \(
\iltikzfig{strings/category/f}[box=g,colour=seq,dom=m,cod=n]
\) with the same behaviour by only using equations in \(\mce_\interpretation\).

Of course, the procedure of using the normalisation equations to translate a
circuit into normalised Mealy form before using the encoding equation (possibly
multiple times depending on how lucky one gets with their orderings) may be
tedious; one might wonder how this is beneficial to the operational approach in
the previous chapter.
The beauty of the \emph{algebraic} semantics is that we \emph{don't} need to
do this every time.
Equations can be proven as lemmas and then used repeatedly in the future as
`shortcuts', possibly saving many reasoning steps.
In time, the algebraicist will build up a powerful repertoire of equations and
wield them to bend circuits to their will.
\section{Algebraic semantics for Belnap logic}\label{sec:algebraic-belnap}

\renewcommand\theContinuedFloat{\alph{ContinuedFloat}}
\begin{figure}
    \ContinuedFloat*
    \centering
    \scalebox{0.85}{\(\equationdisplay{
            \iltikzfig{circuits/axioms/belnap/translation/explosion-lhs}
        }{
            \iltikzfig{circuits/axioms/belnap/translation/explosion-rhs}
        }{
            \belnapexpeqn
        }\)}
    \quad
    \scalebox{0.85}{\(\equationdisplay{
            \iltikzfig{circuits/algebraic/infinite-register-fork-lhs}
        }{
            \iltikzfig{circuits/algebraic/infinite-register-fork-rhs}
        }{
            \infregforkeqn
        }\)}
    \\[0.4em]
    \rule{\textwidth}{0.1mm}
    \\[0.5em]
    \scalebox{0.85}{\(\equationdisplay{
            \iltikzfig{circuits/algebraic/infinite-register-and}
        }{
            \iltikzfig{circuits/components/waveforms/infinite-register-wide}[val={\bot \land v}]
        }{
            \infregandeqn
        }\)}
    \,\,
    \scalebox{0.85}{\(\equationdisplay{
            \iltikzfig{circuits/algebraic/infinite-register-or}
        }{
            \iltikzfig{circuits/components/waveforms/infinite-register-wide}[val={\bot \lor v}]
        }{
            \infregoreqn
        }\)}
    \\[0.25em]
    \rule{\textwidth}{0.1mm}
    \\[0.5em]
    \scalebox{0.85}{\(\equationdisplay{
            \iltikzfig{circuits/algebraic/infinite-register-not}
        }{
            \iltikzfig{circuits/components/waveforms/infinite-register-slightly-wide}[val={\neg v}]
        }{
            \infregnoteqn
        }\)}
    \quad
    \scalebox{0.85}{\(\equationdisplay{
            \iltikzfig{circuits/axioms/fork-lhs}
        }{
            \iltikzfig{circuits/axioms/fork-rhs}
        }{
            \forkeqn
        }\)}
    \quad
    \scalebox{0.85}{\(\equationdisplay{
            \iltikzfig{strings/structure/bialgebra/init-copy-lhs}
        }{
            \iltikzfig{strings/structure/bialgebra/init-copy-rhs}
        }{
            \botforkeqn
        }\)}
    \\[0.25em]
    \rule{\textwidth}{0.1mm}
    \\[0.5em]
    \scalebox{0.85}{\(\equationdisplay{
            \iltikzfig{circuits/axioms/belnap/translation/join-and-lhs}
        }{
            \iltikzfig{circuits/axioms/belnap/translation/join-and-rhs}
        }{
            \joinandeqn
        }\)}
    \,\,
    \scalebox{0.85}{\(\equationdisplay{
            \iltikzfig{circuits/axioms/belnap/translation/join-or-lhs}
        }{
            \iltikzfig{circuits/axioms/belnap/translation/join-or-rhs}
        }{
            \joinoreqn
        }\)}
    \,\,
    \scalebox{0.85}{\(\equationdisplay{
            \iltikzfig{circuits/axioms/belnap/translation/bot-and-lhs}
        }{
            \iltikzfig{circuits/axioms/belnap/translation/bot-x-rhs}
        }{
            \botandeqn
        }\)}
    \\[0.25em]
    \rule{\textwidth}{0.1mm}
    \\[0.5em]
    \scalebox{0.85}{\(\equationdisplay{
            \iltikzfig{circuits/axioms/belnap/translation/de-morgan-and-lhs}
        }{
            \iltikzfig{circuits/axioms/belnap/translation/de-morgan-and-rhs}
        }{
            \demorganand
        }\)}
    \,\,
    \scalebox{0.85}{\(\equationdisplay{
            \iltikzfig{circuits/axioms/belnap/translation/de-morgan-or-lhs}
        }{
            \iltikzfig{circuits/axioms/belnap/translation/de-morgan-or-rhs}
        }{
            \demorganor
        }\)}
    \,\,
    \scalebox{0.85}{\(\equationdisplay{
            \iltikzfig{circuits/axioms/belnap/translation/bot-or-lhs}
        }{
            \iltikzfig{circuits/axioms/belnap/translation/bot-x-rhs}
        }{
            \botoreqn
        }\)}
    \\[0.25em]
    \rule{\textwidth}{0.1mm}
    \caption{First part of the set of \emph{explosion equations} \(\mathcal{X}\)}
    \label{fig:explosion-equations-1}
\end{figure}

For a sound and complete equational theory, equations are required to bring any
essentially combinational circuit into a canonical form.
We will now demonstrate this for the Belnap interpretation; recall from
\cref{sec:denotational-belnap} that the canonical form for Belnap circuits is a
circuit that `explodes' its inputs into circuits for the `falsy' and `truthy'
components of the output, before joining these together.
The first equations we will define translate any essentially combinational
circuit into such an exploded circuit.

\begin{definition}[Explosion equations]
    Let the set of \emph{explosion equations} \(\mathcal{X}\) be defined as the
    equations listed
    in \cref{fig:explosion-equations-1,fig:explosion-equations-2}.
\end{definition}

\begin{lemma}
    The explosion equations are sound.
\end{lemma}
\begin{proof}
    By checking all the inputs.
\end{proof}

Most of the equations in \(\mathcal{X}\) are well-known; the only interesting
one is \((\belnapexpeqn)\), which says that we can always `explode' a wire and
join it back together.
To translate a circuit into exploded form, we use this equation to introduce an
`empty explosion', then propagate the original components across with the
other equations.
The first obstacle to this is the forks at the left of the explosion.

\begin{lemma}\label{lem:explode-copy}
    For any essentially combinational Belnap circuit \(
    \iltikzfig{strings/category/f}[box=f,dom=m,cod=n,colour=seq]
    \), the equation \(
    \iltikzfig{strings/structure/cartesian/naturality-copy-lhs}[colour=comb,box=f]
    =
    \iltikzfig{strings/structure/cartesian/naturality-copy-rhs}[colour=comb,box=f]
    \) in \(\scirc{\belnapsignature} / \mathcal{X}\).
\end{lemma}
\begin{proof}
    This follows for the combinational generators by applying
    \((\joinforkeqn)\), \((\botforkeqn)\), \((\andforkeqn)\), \((\orforkeqn)\),
    \((\notforkeqn)\), and is immediate for the fork.
    The infinite register \(
    \iltikzfig{circuits/components/waveforms/infinite-register}[val=v]
    \) is also a base case and is covered by \((\infregforkeqn)\).
    The inductive cases are trivial.
\end{proof}

Once the circuit is past the opening forks, the remaining equations are used to
push them across the translators.

\begin{proposition}\label{prop:exploded-belnap}
    Given an essentially combinational Belnap circuit \(
    \iltikzfig{strings/category/f}[box=f,dom=m,cod=1,colour=seq]
    \), there exists combinational Belnap circuits \(
    \iltikzfig{strings/category/f-3-1}[box=f_0,dom1=1,dom2=m,dom3=m,cod=1,colour=comb]
    \) and \(
    \iltikzfig{strings/category/f-3-1}[box=f_1,dom1=1,dom2=m,dom3=m,cod=1,colour=comb]
    \) containing no \(
    \iltikzfig{circuits/components/gates/not}
    \) or \(
    \iltikzfig{strings/structure/monoid/merge}[colour=comb]
    \) generators, such that \[
        \iltikzfig{strings/category/f}[box=f,dom=m,cod=1,colour=seq]
        =
        \iltikzfig{circuits/axioms/belnap/translation/exploded-form}[box=f,dom=m,cod=1].
    \]
\end{proposition}
\begin{proof}
    First we consider the base cases.
    If \(
    \iltikzfig{strings/category/f}[box=f,colour=seq]
    \) is the identity, then it can be transformed into the desired form
    with
    \((\belnapexpeqn)\).
    Since \(\iltikzfig{strings/category/f}[box=f,colour=seq]\) has codomain
    \(1\) it cannot be a symmetry.
    For the other generators and the infinite register,
    \((\belnapexpeqn)\) can be applied to the output wire to create the
    exploded `skeleton', followed by using \cref{lem:explode-copy} to copy the
    components into four.
    The four copies can be pushed through the translators
    using \((\joinforkeqn)\), \((\joinandeqn)\), \((\joinoreqn)\),
    \((\botforkeqn)\), \((\botoreqn)\) \((\andcomm)\), \((\orcomm)\),
    \((\andorddisteqn)\), \((\oranddisteqn)\), \((\infregandeqn)\),
    \((\infregoreqn)\), \((\infregnoteqn)\), \((\demorganand)\),
    \((\demorganor)\), \((\botnoteqn)\), \((\botandeqn)\), and \((\botoreqn)\).
    As propagating the \(
    \iltikzfig{circuits/components/gates/not}
    \) flips the translators by using \((\demorganand)\) and \((\demorganor)\),
    \((\forkcommeqn)\) must be used to restore the correct order, and
    \((\dneeqn)\) is used to eliminate additional
    \(\iltikzfig{circuits/components/gates/not}\) gates.
    Any infinite registers containing \(\bot\) can be converted to
    \(\iltikzfig{strings/structure/monoid/init}[colour=comb]\) components using
    \((\botregeqn)\), and other registers can be combined using
    \((\infregforkeqn)\).

    For the composition inductive case, we have two exploded circuits and we
    need to push the first inside the second.
    Using \cref{lem:explode-copy}, the first circuit can be propagated across
    the forks at the start of the second, so each of the four translators has as
    input a copy of the first circuit.
    Using the same strategy as for the base case the components of the circuit
    can then be propagated across the translators.
    For tensor, the circuits can be interleaved using axioms of STMCs.
\end{proof}
%
\begin{figure}\ContinuedFloat
    \centering
    \scalebox{0.85}{\(\equationdisplay{
            \iltikzfig{circuits/axioms/belnap/translation/not-fork-lhs}
        }{
            \iltikzfig{circuits/axioms/belnap/translation/not-fork-rhs}
        }{
            \notforkeqn
        }\)}
    \scalebox{0.85}{\(\equationdisplay{
            \iltikzfig{circuits/axioms/belnap/translation/and-fork-lhs}
        }{
            \iltikzfig{circuits/axioms/belnap/translation/and-fork-rhs}
        }{
            \andforkeqn
        }\)}
    \scalebox{0.85}{\(\equationdisplay{
            \iltikzfig{circuits/axioms/belnap/translation/or-fork-lhs}
        }{
            \iltikzfig{circuits/axioms/belnap/translation/or-fork-rhs}
        }{
            \orforkeqn
        }\)}
    \\[0.25em]
    \rule{\textwidth}{0.1mm}
    \\[0.5em]
    \scalebox{0.85}{\(\equationdisplay{
            \iltikzfig{circuits/axioms/belnap/double-negation-lhs}
        }{
            \iltikzfig{circuits/axioms/belnap/double-negation-rhs}
        }{
            \dneeqn
        }\)}
    \,
    \scalebox{0.85}{\(\equationdisplay{
            \iltikzfig{strings/structure/comonoid/associativity-lhs}
        }{
            \iltikzfig{strings/structure/comonoid/associativity-rhs}
        }{
            \forkassoceqn
        }\)}
    \,
    \scalebox{0.85}{\(\equationdisplay{
            \iltikzfig{strings/structure/bialgebra/merge-copy-lhs}
        }{
            \iltikzfig{strings/structure/bialgebra/merge-copy-rhs}
        }{
            \joinforkeqn
        }\)}
    \,
    \scalebox{0.85}{\(\equationdisplay{
            \iltikzfig{strings/structure/comonoid/commutativity-lhs}
        }{
            \iltikzfig{strings/structure/comonoid/commutativity-rhs}
        }{
            \forkcommeqn
        }\)}
    \\[0.25em]
    \rule{\textwidth}{0.1mm}
    \\[0.5em]
    \scalebox{0.85}{\(\equationdisplay{
            \iltikzfig{circuits/axioms/belnap/and-idempotent-lhs}
        }{
            \iltikzfig{circuits/axioms/belnap/and-idempotent-rhs}
        }{
            \andidemeqn
        }\)}
    \quad
    \scalebox{0.85}{\(\equationdisplay{
            \iltikzfig{circuits/axioms/belnap/or-idempotent-lhs}
        }{
            \iltikzfig{circuits/axioms/belnap/or-idempotent-rhs}
        }{
            \oridemeqn
        }\)}
    \quad
    \scalebox{0.85}{\(\equationdisplay{
            \iltikzfig{circuits/axioms/delay-fork-lhs}
        }{
            \iltikzfig{circuits/axioms/delay-fork-rhs}
        }{
            \delayforkeqn
        }\)}
    \quad
    \scalebox{0.85}{\(\equationdisplay{
            \iltikzfig{circuits/axioms/belnap/translation/bot-not-lhs}
        }{
            \iltikzfig{circuits/axioms/belnap/translation/bot-x-rhs}
        }{
            \botnoteqn
        }\)}
    \\[0.25em]
    \rule{\textwidth}{0.1mm}
    \\[0.5em]
    \scalebox{0.85}{\(\equationdisplay{
            \iltikzfig{circuits/axioms/belnap/and-or-distributivity-lhs}
        }{
            \iltikzfig{circuits/axioms/belnap/and-or-distributivity-rhs}
        }{
            \andorddisteqn
        }\)}
    \,\,
    \scalebox{0.85}{\(\equationdisplay{
            \iltikzfig{circuits/axioms/belnap/or-and-distributivity-lhs}
        }{
            \iltikzfig{circuits/axioms/belnap/or-and-distributivity-rhs}
        }{
            \oranddisteqn
        }\)}
    \\[0.25em]
    \rule{\textwidth}{0.1mm}
    \\[0.5em]
    \scalebox{0.85}{\(\equationdisplay{
            \iltikzfig{circuits/axioms/belnap/and-commutativity-lhs}
        }{
            \iltikzfig{circuits/axioms/belnap/and-commutativity-rhs}
        }{
            \andcomm
        }\)}
    \,\,
    \scalebox{0.85}{\(\equationdisplay{
            \iltikzfig{circuits/axioms/belnap/or-commutativity-lhs}
        }{
            \iltikzfig{circuits/axioms/belnap/or-commutativity-rhs}
        }{
            \orcomm
        }\)}
    \\[0.25em]
    \rule{\textwidth}{0.1mm}
    \caption{Second part of the set of \emph{explosion equations} \(\mathcal{X}\)}
    \label{fig:explosion-equations-2}
\end{figure}

This already looks similar to a circuit in the image of
\(\mealytofunc_\belnap\), but the two subcircuits must also be translated into
truthy or falsy disjunctive normal form.

\begin{definition}[Normal form equations]
    Let the set of \emph{normal form equations} \(\mathcal{F}\) be defined as
    the equations listed in \cref{fig:normal-form-equations}.
\end{definition}

\begin{lemma}
    The equations in \(\mathcal{F}\) are sound.
\end{lemma}
\begin{proof}
    By checking all the inputs.
\end{proof}

\begin{figure*}
    \centering
    \(\equationdisplay{
        \iltikzfig{circuits/axioms/belnap/and-associativity-lhs}
    }{
        \iltikzfig{circuits/axioms/belnap/and-associativity-rhs}
    }{\andassoc}\)
    \quad
    \(\equationdisplay{
        \iltikzfig{circuits/axioms/belnap/or-associativity-lhs}
    }{
        \iltikzfig{circuits/axioms/belnap/or-associativity-rhs}
    }{\orassoc}\)
    \\[0.25em]
    \rule{\textwidth}{0.1mm}
    \\[0.7em]
    \(\equationdisplay{
        \iltikzfig{circuits/axioms/belnap/and-or-distributivity-lhs}
    }{
        \iltikzfig{circuits/axioms/belnap/and-or-distributivity-rhs}
    }{\andorddisteqn}\)
    \quad
    \(\equationdisplay{
        \iltikzfig{circuits/axioms/belnap/or-and-distributivity-lhs}
    }{
        \iltikzfig{circuits/axioms/belnap/or-and-distributivity-rhs}
    }{\oranddisteqn}\)
    \\[0.25em]
    \rule{\textwidth}{0.1mm}
    \\[0.7em]
    \(\equationdisplay{
        \iltikzfig{circuits/axioms/belnap/and-commutativity-lhs}
    }{
        \iltikzfig{circuits/axioms/belnap/and-commutativity-rhs}
    }{\andcomm}\)
    \quad
    \(\equationdisplay{
        \iltikzfig{circuits/axioms/belnap/or-commutativity-lhs}
    }{
        \iltikzfig{circuits/axioms/belnap/or-commutativity-rhs}
    }{\orcomm}\)
    \quad
    \(\equationdisplay{
        \iltikzfig{circuits/axioms/belnap/and-idempotent-lhs}
    }{
        \iltikzfig{circuits/axioms/belnap/and-idempotent-rhs}
    }{\andidemeqn}\)
    \\[0.25em]
    \rule{\textwidth}{0.1mm}
    \\[0.7em]
    \(\equationdisplay{
        \iltikzfig{circuits/axioms/belnap/or-idempotent-lhs}
    }{
        \iltikzfig{circuits/axioms/belnap/or-idempotent-rhs}
    }{\oridemeqn}\)
    \quad
    \(\equationdisplay{
        \iltikzfig{circuits/axioms/belnap/translation/and-fork-lhs}
    }{
        \iltikzfig{circuits/axioms/belnap/translation/and-fork-rhs}
    }{\andforkeqn}\)
    \quad
    \(\equationdisplay{
        \iltikzfig{circuits/axioms/belnap/translation/or-fork-lhs}
    }{
        \iltikzfig{circuits/axioms/belnap/translation/or-fork-rhs}
    }{\andforkeqn}\)
    \\[0.25em]
    \rule{\textwidth}{0.1mm}
    \\[0.7em]
    \(\equationdisplay{
        \iltikzfig{strings/structure/comonoid/commutativity-rhs}
    }{
        \iltikzfig{strings/structure/comonoid/commutativity-lhs}
    }{\forkcommeqn}\)
    \quad
    \(\equationdisplay{
        \iltikzfig{strings/structure/comonoid/associativity-lhs}
    }{
        \iltikzfig{strings/structure/comonoid/associativity-rhs}
    }{\forkassoceqn}\)
    \quad
    \(\equationdisplay{
        \iltikzfig{strings/structure/comonoid/unitality-l-lhs}
    }{
        \iltikzfig{strings/structure/comonoid/unitality-l-rhs}
    }{\forkuniteqn}\)
    \\[0.25em]
    \rule{\textwidth}{0.1mm}
    \\[0.7em]
    \(\equationdisplay{
        \iltikzfig{circuits/axioms/belnap/and-annihilator-lhs}
    }{
        \iltikzfig{circuits/axioms/belnap/and-annihilator-rhs}
    }{\andanneqn}\)
    \quad
    \(\equationdisplay{
        \iltikzfig{circuits/axioms/belnap/or-annihilator-lhs}
    }{
        \iltikzfig{circuits/axioms/belnap/or-annihilator-rhs}
    }{\oranneqn}\)
    \\[0.25em]
    \rule{\textwidth}{0.1mm}
    \\[0.5em]
    \(\equationdisplay{
        \iltikzfig{circuits/axioms/belnap/and-identity-lhs}
    }{
        \iltikzfig{circuits/axioms/belnap/and-identity-rhs}
    }{\andideqn}\)
    \quad
    \(\equationdisplay{
        \iltikzfig{circuits/axioms/belnap/or-identity-lhs}
    }{
        \iltikzfig{circuits/axioms/belnap/or-identity-rhs}
    }{\orideqn}\)
    \caption{
        Set \(\mathcal{F}\) of \emph{normal form equations}.
    }
    \label{fig:normal-form-equations}
\end{figure*}


We will now show that these equations suffice to translate the subcircuits in
the exploded circuit into falsy or truthy disjunctive normal form.

\begin{lemma}\label{lem:truthy-conjunction-normalising}
    Given a Belnap circuit \(
    \iltikzfig{circuits/synthesis/normalised-function}[box=f,dom=m,cod=n,values=\belnaptrue]
    \) containing no \(
    \iltikzfig{strings/structure/monoid/merge}[colour=comb]
    \), \(
    \iltikzfig{circuits/components/gates/or}
    \) or \(
    \iltikzfig{circuits/components/gates/not}
    \) components, there exists a circuit \(
    \iltikzfig{strings/category/f}[box=g,colour=comb,dom=mn,cod=p]
    \) containing only identity and elimination constructs, and a circuit \(
    \iltikzfig{strings/category/f}[box=h,colour=seq,dom=p,cod=n]
    \) defined as the tensor of \(n\) truthy conjunctions, such that \(
    \iltikzfig{circuits/synthesis/normalised-function}[box=f,dom=m,cod=n,values=\belnaptrue]
    =
    \iltikzfig{circuits/algebraic/unction-circuit}
    \) in \(\scirc{\belnapsignature} / \mathcal{F}\).
\end{lemma}
\begin{proof}
    Repeatedly applying \((\andforkeqn)\) to \(
    \iltikzfig{strings/category/f}[box=f,colour=comb]
    \) propagates the \(
    \iltikzfig{circuits/components/gates/and}
    \) components in the circuit as far to the right as possible, so all the
    fork and eliminate constructs are all in the left half of the term.
    Using \((\forkcommeqn)\), \((\forkassoceqn)\) to rearrange the forks, and
    \((\forkuniteqn)\) to introduce forks where necessary,
    we can manipulate these forks such that there is a wire for each of the \(m\)
    inputs feeds into each of the \(n\) outputs.
    Similarly, we can use \((\andideqn)\) to introduce infinite registers where
    appropriate, so we have a circuit of the form below. \[
        \iltikzfig{circuits/algebraic/belnap-rearrange-forks}
    \]
    The subcircuits may not be truthy conjunctions yet, as the input wires may
    be used more than once.
    Using the \((\andidemeqn)\), \((\forkcommeqn)\), \((\forkassoceqn)\),
    \((\andassoc)\) and \((\andcomm)\), each subcircuit can be translated into
    a truthy conjunction.
\end{proof}

% \begin{example}
%     Consider the circuit \(
%     \iltikzfig{circuits/examples/conjunction/step-0}
%     \); we transform it into a truthy conjunction as follows.
%     \begin{gather*}
%         \iltikzfig{circuits/examples/conjunction/step-0}
%         \eqaxioms[(\forkuniteqn)]
%         \iltikzfig{circuits/examples/conjunction/step-1}
%         \eqaxioms[(\forkuniteqn)]
%         \iltikzfig{circuits/examples/conjunction/step-2}
%         \eqaxioms[(\forkcommeqn)]
%         \\
%         \iltikzfig{circuits/examples/conjunction/step-3}
%         \eqaxioms[(\forkassoceqn)]
%         \iltikzfig{circuits/examples/conjunction/step-4}
%         \eqaxioms[(\andcomm)]
%         \iltikzfig{circuits/examples/conjunction/step-5}
%         \eqaxioms[(\andforkeqn)]
%         \\
%         \iltikzfig{circuits/examples/conjunction/step-6}
%         \eqaxioms[(\andassoc)]
%         \iltikzfig{circuits/examples/conjunction/step-7}
%         \eqaxioms[(\forkassoceqn)]
%         \\
%         \iltikzfig{circuits/examples/conjunction/step-8}
%         \eqaxioms[(\andidemeqn)]
%         \iltikzfig{circuits/examples/conjunction/step-9}
%         \eqaxioms[(\andassoc)]
%         \\
%         \iltikzfig{circuits/examples/conjunction/step-10}
%         \eqaxioms[(\andcomm)]
%         \iltikzfig{circuits/examples/conjunction/step-11}
%         \eqaxioms[(\andassoc)]
%         \\
%         \iltikzfig{circuits/examples/conjunction/step-12}
%         \eqaxioms[(\andidemeqn)]
%         \iltikzfig{circuits/examples/conjunction/step-13}
%         \eqaxioms[(\andassoc)]
%         \\
%         \iltikzfig{circuits/examples/conjunction/step-14}
%         \eqaxioms[(\andidemeqn)]
%         \iltikzfig{circuits/examples/conjunction/step-15}
%         \eqaxioms
%         \iltikzfig{circuits/examples/conjunction/step-16}
%         \eqaxioms[(\andideqn)]
%         \iltikzfig{circuits/examples/conjunction/step-17}
%     \end{gather*}
% \end{example}

The proof for the falsy circuit is almost exactly the same.

\begin{lemma}\label{lem:falsy-conjunction-normalising}
    Given a Belnap circuit \(
    \iltikzfig{circuits/synthesis/normalised-function}[box=f,dom=m,cod=n,values=\belnaptrue]
    \) containing no \(
    \iltikzfig{strings/structure/monoid/merge}[colour=comb]
    \), \(
    \iltikzfig{circuits/components/gates/and}
    \) or \(
    \iltikzfig{circuits/components/gates/not}
    \) components, there exists a circuit \(
    \iltikzfig{strings/category/f}[box=g,colour=comb,dom=mn,cod=p]
    \) containing only identity and elimination constructs, and a circuit \(
    \iltikzfig{strings/category/f}[box=h,colour=comb,dom=p,cod=n]
    \) defined as the tensor of \(n\) falsy conjunctions, such that \(
    \iltikzfig{strings/category/f}[box=f,colour=comb]
    =
    \iltikzfig{circuits/algebraic/unction-circuit}
    \) in \(\scirc{\belnapsignature} / \mathcal{F}\).
\end{lemma}
\begin{proof}
    As \cref{lem:truthy-conjunction-normalising}, but with the equations
    on \(\iltikzfig{circuits/components/gates/or}\) components.
\end{proof}

Truthy and falsy conjunctions can then be used to create truthy and falsy
conjunctive normal forms.

\begin{proposition}\label{prop:disjunctive-normal-form}
    Given a Belnap circuit \[
        \iltikzfig{circuits/axioms/belnap/translation/exploded-form}[box=f,dom=m,cod=1]
    \] in which \(
    \iltikzfig{strings/category/f-3-1}[box=f_0,colour=comb]
    \) and \(
    \iltikzfig{strings/category/f-3-1}[box=f_1,colour=comb]
    \) contain no \(
    \iltikzfig{circuits/components/gates/not}
    \) or \(
    \iltikzfig{strings/structure/monoid/merge}[colour=comb]
    \) generators, there exists \[
        \iltikzfig{circuits/axioms/belnap/translation/exploded-form-cnf}[box=f,dom=m,cod=1]
    \] in which \(
    \iltikzfig{strings/category/f}[box=g_0,colour=comb]
    \) and \(
    \iltikzfig{strings/category/f}[box=g_1,colour=comb]
    \) contain only identity and elimination components,
    \(
    \iltikzfig{strings/category/f}[box=h_0,colour=seq]
    \) is in falsy conjunctive normal form and\(
    \iltikzfig{strings/category/f}[box=h_1,colour=seq]
    \) is in truthy conjunctive normal form, such that \[
        \iltikzfig{circuits/axioms/belnap/translation/exploded-form}[box=f]
        =
        \iltikzfig{circuits/axioms/belnap/translation/exploded-form-cnf}
    \] in \(
    \scirc{\belnapsignature} / \mathcal{F}
    \).
\end{proposition}
\begin{proof}
    First, all the \(
    \iltikzfig{circuits/components/gates/and}
    \) components need to be propagated to the far right of the \(
    \iltikzfig{strings/category/f-3-1-compressed}[box=f_0,colour=comb]
    \) circuit using \((\oranddisteqn)\), and all the \(
    \iltikzfig{circuits/components/gates/or}
    \) components need to be propagated to the far right of the \(
    \iltikzfig{strings/category/f-3-1-compressed}[box=f_1,colour=comb]
    \) circuit using \((\andorddisteqn)\).
    This means that the circuits are split into two halves, each containing
    one type of gate.

    For these circuits to be in truthy or falsy disjunctivenormal form, they
    need to contain exactly one \(
    \iltikzfig{strings/structure/monoid/init}[colour=comb]
    \) component.
    If there is not already such a component inside the \(
    \iltikzfig{strings/category/f-3-1-compressed}[box=f_0,colour=comb]
    \) or \(
    \iltikzfig{strings/category/f-3-1-compressed}[box=f_1,colour=comb]
    \) subcircuits, one can be inserted using the \((\botanduniteqn)\) equation
    for the former and the \((\botoruniteqn)\) equation for the latter.
    If there are multiple unit components, these can be propagated through the
    circuit using \((\orassoc)\), \((\oranddisteqn)\), \((\andassoc)\), and
    \((\andorddisteqn)\), and combined into one by using \((\botandeqn)\) or
    \((\botoreqn)\).

    Now we have circuits that have the `root' of disjunctive normal forms, but
    the `leaves' are not conjunctions.
    This is remedied by applying \cref{lem:truthy-conjunction-normalising} and
    \cref{lem:falsy-conjunction-normalising} to the left half of each circuit.
\end{proof}

Putting this all together gives us the desired canonical form theorem.

\begin{theorem}
    Given an essentially combinational Belnap circuit \(
    \iltikzfig{strings/category/f}[box=f,colour=seq]
    \), there exists a circuit \(
    \iltikzfig{strings/category/f}[box=g,colour=seq]
    \) in the image of \(\mealytofunc_\belnap\) such that \(
    \iltikzfig{strings/category/f}[box=f,colour=seq]
    =
    \iltikzfig{strings/category/f}[box=g,colour=seq]
    \) in \(\ccirc{\Sigma_\belnap} / \mathcal{X} + \mathcal{F}\).
\end{theorem}
\begin{proof}
    This follows by applying \cref{prop:exploded-belnap} followed by
    \cref{prop:disjunctive-normal-form}.
\end{proof}

This shows that the equations detailed in this section can translate any
essentially combinational Belnap circuit into a circuit in the image of the
functional completeness map.
\section{Algebraic semantics for generalised circuits}