\section{Partial evaluation}\label{sec:partial}

\index{partial evaluation}
Partial evaluation~\cite{jones1996introduction} is a paradigm used in software
optimisation in which programs are `evaluated as much as possible' while only
some of the inputs are specified.
For example, it may be the case that a particular input to a program is fixed
for long periods of time; using partial evaluation, we can define a program
specialised for this input.
This program might run significantly faster than the original.

There has been work into partial evaluation for hardware, such as constant
propagation~\cite{singh1996expressing,singh1999partial} and
unfolding~\cite{thompson2006bitlevel}.
However, this has been relatively informal, and can be made rigorous using the
categorical framework.
In this section we will focus on how we could extend the reduction-based
operational semantics to define automatic procedures for applying partial
evaluation to circuits.

\subsection{Shortcut rules}\label{sec:shortcut}

\index{shortcut rule}
\index{infinite waveform}
It is often the case that we know that some of the inputs to a circuit are
fixed.
This can be modelled by precomposing the relevant input with an
\emph{infinite waveform} \(
\iltikzfig{circuits/components/waveforms/infinite-register}[val=v]
\).
We can propagate these waveforms across a circuit to see if we can reduce it to
a circuit \emph{specialised} for these inputs.

To propagate waveforms across circuits we need to derive a version of the
\((\gateeqn)\) rule for applying waveforms to primitives.
These rules are illustrated in \cref{fig:waveform-rules}.

\begin{figure}
    \centering
    \(
    \iltikzfig{circuits/examples/shortcuts/waveform-primitive-lhs}
    \reduction
    \iltikzfig{circuits/examples/shortcuts/waveform-primitive-rhs}
    \)
    \quad
    \(
    \iltikzfig{circuits/examples/shortcuts/waveform-fork-lhs}
    \reduction
    \iltikzfig{circuits/examples/shortcuts/waveform-fork-rhs}
    \)
    \\[0.25em]
    \rule{\textwidth}{0.1mm}
    \\[0.5em]
    \(
    \iltikzfig{circuits/examples/shortcuts/waveform-join-lhs}
    \reduction
    \iltikzfig{circuits/examples/shortcuts/waveform-join-rhs}
    \)
    \\[0.25em]
    \rule{\textwidth}{0.1mm}
    \\[0.5em]
    \caption{Rules for infinite waveforms}
    \label{fig:waveform-rules}
\end{figure}

This is not the only way we can partially evaluate with some inputs.
In some interpretations, it may be that we learn something about the output of
a primitive with only some of the inputs specified.

\begin{example}[Belnap shortcuts]
    \index{Belnap!shortcut}
    In the Belnap interpretation \(\belnapinterpretation\), if one applies a
    false value to an \(\andgate\) gate then it will output false regardless of
    the other input.
    Similarly, if one applies a true value to an \(\orgate\) gate it will output
    true.
    Conversely, if one applies a true value to an \(\andgate\) gate or a false
    value to an \(\orgate\) gate, it will act as the identity on the other
    input.
\end{example}

These `shortcuts' can also be implemented as rules, as illustrated in
\cref{fig:shortcut-waveform-rules}.
Note that here the value that `triggers' the shortcut must be contained within
an infinite waveform; if we applied the rule with just an instantaneous value,
this value would produce \(\bot\) on ticks after the first and the rule would
be unsound.

\begin{figure}
    \centering
    \(
    \iltikzfig{circuits/examples/shortcuts/and-f-infinite-lhs}
    \reduction
    \iltikzfig{circuits/examples/shortcuts/and-f-infinite-rhs}
    \)
    \quad
    \(
    \iltikzfig{circuits/examples/shortcuts/or-t-infinite-lhs}
    \reduction
    \iltikzfig{circuits/examples/shortcuts/or-t-infinite-rhs}
    \)
    \\[0.3em]
    \rule{\textwidth}{0.1mm}
    \\[0.5em]
    \(
    \iltikzfig{circuits/examples/shortcuts/and-f-comm-infinite-lhs}
    \reduction
    \iltikzfig{circuits/examples/shortcuts/and-f-infinite-rhs}
    \)
    \quad
    \(
    \iltikzfig{circuits/examples/shortcuts/or-t-comm-infinite-lhs}
    \reduction
    \iltikzfig{circuits/examples/shortcuts/or-t-infinite-rhs}
    \)
    \\[0.3em]
    \rule{\textwidth}{0.1mm}
    \\[0.5em]
    \(
    \iltikzfig{circuits/examples/shortcuts/and-t-infinite-lhs}
    \reduction
    \iltikzfig{circuits/examples/shortcuts/and-t-infinite-rhs}
    \)
    \quad
    \(
    \iltikzfig{circuits/examples/shortcuts/or-f-infinite-lhs}
    \reduction
    \iltikzfig{circuits/examples/shortcuts/or-f-infinite-rhs}
    \)
    \\[0.3em]
    \rule{\textwidth}{0.1mm}
    \\[0.5em]
    \(
    \iltikzfig{circuits/examples/shortcuts/and-t-comm-infinite-lhs}
    \reduction
    \iltikzfig{circuits/examples/shortcuts/and-t-infinite-rhs}
    \)
    \quad
    \(
    \iltikzfig{circuits/examples/shortcuts/or-f-comm-infinite-lhs}
    \reduction
    \iltikzfig{circuits/examples/shortcuts/or-f-infinite-rhs}
    \)
    \\[0.3em]
    \rule{\textwidth}{0.1mm}
    \\[0.5em]
    \caption{Belnap shortcut rules for waveforms}
    \label{fig:shortcut-waveform-rules}
\end{figure}

\begin{example}[Control switches]
    Recall that a \emph{multiplexer} is a circuit component constructed as \(
    \iltikzfig{circuits/components/gates/mux}
    \coloneqq
    \iltikzfig{circuits/components/gates/mux-construction}
    \).
    The first input is a \emph{control} which specifies which of the two other
    input signals is produced as the output signal.
    It is often the case that these control signals will be fixed for long
    periods of time; perhaps they specify some sort of global circuit
    configuration.

    Consider the circuit \(
    \iltikzfig{circuits/examples/control/circuit}
    \), in which the control signal to the multiplexer determines which of two
    subcircuits will become the output.
    We will assume that the control signal is held at false, and reduce
    accordingly by instantiating the rule in \cref{fig:waveform-rules} detailing
    the interaction of gates and waveforms to the \(\notgate\) case.
    \begin{gather*}
        \iltikzfig{circuits/examples/control/circuit-mux-applied}
        \coloneqq
        \iltikzfig{circuits/examples/control/circuit-applied}
        \reduction
        \\[1em]
        \iltikzfig{circuits/examples/control/circuit-applied-1}
        \reduction
        \iltikzfig{circuits/examples/control/circuit-applied-2}
        \reduction
        \\[1em]
        \iltikzfig{circuits/examples/control/circuit-applied-3}
        \reduction
        \iltikzfig{circuits/examples/control/circuit-applied-4}
        \reduction
        \iltikzfig{circuits/examples/control/circuit-applied-5}
    \end{gather*}
\end{example}

\subsection{Shortcuts after streaming}

The rules in the previous sections are intended for use on circuits before we
even apply values to them.
However, there is still potential for partial evaluation when we consider the
outputs of a circuit one step at a time.
To do this, we can apply variants of the shortcut rules \emph{after} performing
streaming for some inputs.
These variants are illustrated in \cref{fig:shortcuts}.

\begin{figure}
    \centering
    \(
    \iltikzfig{circuits/examples/shortcuts/and-f-instant-lhs}
    \reduction
    \iltikzfig{circuits/examples/shortcuts/and-f-instant-rhs}
    \)
    \quad
    \(
    \iltikzfig{circuits/examples/shortcuts/or-t-instant-lhs}
    \reduction
    \iltikzfig{circuits/examples/shortcuts/or-t-instant-rhs}
    \)
    \quad
    \(
    \iltikzfig{circuits/examples/shortcuts/and-t-instant-lhs}
    \reduction
    \iltikzfig{circuits/examples/shortcuts/and-t-instant-rhs}
    \)
    \quad
    \(
    \iltikzfig{circuits/examples/shortcuts/or-f-instant-lhs}
    \reduction
    \iltikzfig{circuits/examples/shortcuts/or-f-instant-rhs}
    \)
    \\[0.25em]
    \rule{\textwidth}{0.1mm}
    \\[0.25em]
    \(
    \iltikzfig{circuits/examples/shortcuts/and-f-comm-instant-lhs}
    \reduction
    \iltikzfig{circuits/examples/shortcuts/and-f-instant-rhs}
    \)
    \quad
    \(
    \iltikzfig{circuits/examples/shortcuts/or-t-comm-instant-lhs}
    \reduction
    \iltikzfig{circuits/examples/shortcuts/or-t-instant-rhs}
    \)
    \quad
    \(
    \iltikzfig{circuits/examples/shortcuts/and-t-comm-instant-lhs}
    \reduction
    \iltikzfig{circuits/examples/shortcuts/and-t-instant-rhs}
    \)
    \quad
    \(
    \iltikzfig{circuits/examples/shortcuts/or-f-comm-instant-lhs}
    \reduction
    \iltikzfig{circuits/examples/shortcuts/or-f-instant-rhs}
    \)
    \caption{Examples of some derivable `shortcut rules'}
    \label{fig:shortcuts}
\end{figure}

\begin{example}[Blocking boxes]\label{ex:blocking-boxes}
    Consider the circuit \(
    \iltikzfig{circuits/examples/blocking/circuit}
    \), which contains a `blackbox' combinational circuit \(
    \iltikzfig{strings/category/f}[box=f, colour=comb]
    \) with unknown behaviour.

    Even though we cannot directly reduce the blackbox, if we set the first
    input to false and use the shortcut rule above, we can still produce an
    output value.
    \[
        \iltikzfig{circuits/examples/blocking/applied-false}
        \reduction
        \iltikzfig{circuits/examples/blocking/streamed-false}
        \reduction
        \iltikzfig{circuits/examples/blocking/reduced}
    \]
\end{example}

As well as removing redundant blackboxes, judicious use of shortcut
reductions can dramatically reduce the reductions needed to get the outputs of a
circuit.

\subsection{Protocols}\label{sec:uncertain}

\index{protocol}
Sometimes we may not know the exact inputs to a circuit, but know that they make
up a fixed subset of all possible inputs, or they follow some sort of protocol.
We can implement this in our reduction framework with \emph{uncertain values}
which we either know nothing about or know can only take some specified values.

\begin{definition}
    \index{scircsigmap@\(\scircsigmap\)}
    \nomenclature{\(\scircsigmap\)}{PROP of sequential circuits with uncertain values}
    Let \(\scircsigmap\) be the result of extending \(\scircsigma\) with value
    generators for each word \(\listvar{v?} \in \freemon{\values}\).
\end{definition}

The additional value generators indicate that they could produce one of multiple
possible values.
When a circuit contains uncertain values \(
\iltikzfig{circuits/components/values/vs-longer}[val=\listvar{v_0?}],
\iltikzfig{circuits/components/values/vs-longer}[val=\listvar{v_1?}],
\dots
\iltikzfig{circuits/components/values/vs-longerer}[val=\listvar{v_{n-1}?}],
\) where the maximum length of a given \(v_i\) is \(k\), there are \(k\)
possible value assignments.
For a given assignment \(i < k\), each value will produce a concrete value
defined as \(v_i(j)\) if \(|v_i| > j\) or \(\bot\) otherwise.

To avoid confusion with our syntax sugar for arbitrary-width values, we will
always end uncertain value lists with \(?\).
When writing out specific uncertain words, we delimit the elements with vertical
bars like \(\mathsf{f}|\mathsf{t}\) to allude to the fact that this value is
either the first \emph{or} the second element.

\begin{example}
    If a circuit contains uncertain values \(
    \iltikzfig{circuits/components/values/vs-longer}[val=\belnapfalse|\belnaptrue]
    \) and \(
    \iltikzfig{circuits/components/values/vs-longer}[val=\belnaptrue|\belnapfalse]
    \) in a circuit, then there are two universes to consider, one where the
    values output \(\belnapfalse\belnaptrue\) and one where they output
    \(\belnaptrue\belnapfalse\).
    If we add in another uncertain value with three possible values, \(
    \iltikzfig{circuits/components/values/vs-even-longer}[val=\belnaptrue|\belnapfalse|\top]
    \), we now have three possible universes, in which the values output
    \(\belnapfalse\belnaptrue\belnaptrue\),
    \(\belnaptrue\belnapfalse\belnapfalse\), and
    \(\bot\bot\top\) respectively.
\end{example}

To reason with uncertain values in the reductional framework we need to add
rules for processing them.
Once again it is useful to have versions for both waveforms and values, for
reasoning before and during execution.

\begin{definition}[Uncertain rules]
    \index{uncertain rule}
    The \emph{uncertain rules} are listed in \cref{fig:uncertain-rules}.
\end{definition}

After applying uncertain values to a primitive, it may turn out that all the
possibilities are in fact the same.
This removes any uncertainty, and means the value can be treated as an
ordinary value in future reductions and outputs.

\begin{figure}
    \centering
    \(
    \iltikzfig{circuits/examples/uncertain/waveform-primitive-lhs}[val=\listvar{v?}]
    \reduction
    \iltikzfig{circuits/examples/uncertain/waveform-primitive-rhs}
    \)
    \\[0.4em]
    \rule{\textwidth}{0.1mm}
    \\[0.5em]
    \(
    \iltikzfig{circuits/examples/uncertain/waveform-fork-lhs}[val=\listvar{v?}]
    \reduction
    \iltikzfig{circuits/examples/uncertain/waveform-fork-rhs}[val=\listvar{v?}]
    \)
    \quad
    \(
    \iltikzfig{circuits/examples/uncertain/waveform-join-lhs}[val1=\listvar{v?},val2=\listvar{w?}]
    \reduction
    \iltikzfig{circuits/examples/uncertain/waveform-join-rhs}[val1=\listvar{v?},val2=\listvar{w?}]
    \)
    \\[0.4em]
    \rule{\textwidth}{0.1mm}
    \\[0.5em]
    \(
    \iltikzfig{circuits/examples/uncertain/instantaneous-primitive-lhs}[val=\listvar{v?}]
    \reduction
    \iltikzfig{circuits/examples/uncertain/instantaneous-primitive-rhs}
    \)
    \quad
    \(
    \iltikzfig{circuits/examples/uncertain/instantaneous-fork-lhs}[val=\listvar{v?}]
    \reduction
    \iltikzfig{circuits/examples/uncertain/instantaneous-fork-rhs}[val=\listvar{v?}]
    \)
    \quad
    \(
    \iltikzfig{circuits/examples/uncertain/instantaneous-join-lhs}[val1=\listvar{v?},val2=\listvar{w?}]
    \reduction
    \iltikzfig{circuits/examples/uncertain/instantaneous-join-rhs}[val1=\listvar{v?},val2=\listvar{w?}]
    \)
    \\[0.4em]
    \rule{\textwidth}{0.1mm}
    \\[0.5em]
    \(
    \iltikzfig{circuits/components/values/vs-longer}[val=\listvar{v?}]
    \reduction
    \iltikzfig{circuits/components/values/vs}[val=v]
    \)
    \,\,
    if \(\forall i,j < |\listvar{v?}|\), \(\listvar{v?}(i) = \listvar{v?}(j)\)
    \\[0.4em]
    \rule{\textwidth}{0.1mm}
    \\[0.5em]
    \(
    \iltikzfig{circuits/components/waveforms/infinite-register}[val=\listvar{v?}]
    \reduction
    \iltikzfig{circuits/components/waveforms/infinite-register}[val=v]
    \)
    \,\,
    if \(\forall i,j < |\listvar{v?}|\), \(\listvar{v?}(i) = \listvar{v?}(j)\)
    \\[0.4em]
    \rule{\textwidth}{0.1mm}
    \caption{Rules for uncertain values}
    \label{fig:uncertain-rules}
\end{figure}

\begin{example}[Protocols]\label{ex:protocols}
    \index{protocol}
    One sticking point that arises when using the categorical framework is the
    presence of the \(\bot\) and \(\top\) values, which would not normally
    be explicitly provided to a circuit.
    These values mean that some well-known Boolean identities do not always hold.
    By using uncertain values, we can specify the values that \emph{will} be
    applied to a circuit and apply reductions that are not valid in general but
    are in this context.

    In the following example, setting the first two inputs to true/false
    inverses reduces the circuit to one with combinational behaviour.
    \begin{gather*}
        \iltikzfig{circuits/examples/protocol/circuit-with-protocol}
        \reduction
        \\[0.25em]
        \iltikzfig{circuits/examples/protocol/circuit-with-protocol-1}
        \reduction
        \\[0.25em]
        \iltikzfig{circuits/examples/protocol/circuit-with-protocol-2}
        \reduction
        \\[0.25em]
        \iltikzfig{circuits/examples/protocol/circuit-with-protocol-3}
        \reduction
        \\[0.25em]
        \iltikzfig{circuits/examples/protocol/circuit-with-protocol-4}
        \reduction
        \iltikzfig{circuits/examples/protocol/circuit-with-protocol-5}
        \reduction
        \\[0.25em]
        \iltikzfig{circuits/examples/protocol/circuit-with-protocol-6}
        \reduction
        \iltikzfig{circuits/examples/protocol/circuit-with-protocol-7}
        \reduction
        \iltikzfig{circuits/examples/protocol/circuit-with-protocol-8}
    \end{gather*}
\end{example}