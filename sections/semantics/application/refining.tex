\section{Refining circuits}


\begin{example}[Retiming]
    The clock cycle of a circuit is determined by the longest paths between
    registers. Altering the paths between registers can be achieved using
    \emph{retiming}~\cite{leiserson1991retiming}: moving registers across gates.
    This is modelled by the streaming equation (\cref{prop:streaming});
    forward retiming (streaming left to right) is always possible
    but for backward retiming (streaming right to left), the value
    in the register must be in the image of the gates.
\end{example}

The behaviour of a circuit need not be preserved exactly.
Indeed, this is not to be expected as optimisations are intended to increase the
clock cycle of the circuit!
This means that the same output signals will be produced, but over a shorter
timespan.

\begin{example}[Pipelining]
    Another technique is \emph{pipelining}~\cite{parhi1999vlsi}, which inserts
    more registers to increase throughput.
    These can be emulated in the compositional framework by apply
    transformations locally to registers.
    Ordinarily, such transformations can cause obfuscation since the state space
    dramatically changes. In the compositional model, the structure of the
    circuit is left relatively untouched.
\end{example}

\begin{example}[Scan chains]
    A common way of testing circuits is by using a
    \emph{scan chain}~\cite{mourad2000principles}, a way of forcing the
    inputs to flipflops to test how specific states affect the outputs of the
    circuit.
    Adding a flipflop to a scan chain requires some extra inputs: the
    \(\mathsf{scan}_\mathsf{en}\) wire toggles if the flipflop operates in
    normal mode or if it takes \(\mathsf{scan}_\mathsf{in}\) as its value.
    \[
        \iltikzfig{circuits/examples/scan-chain/flipflop-before-chain}
        \xRightarrow{\text{scan}}
        \iltikzfig{circuits/examples/scan-chain/scan-chain}
    \]
\end{example}

One could factor in these transformations when designing the circuit, but this
can obfuscate the design of the actual logic.
Additionally, applying these transformations where the remaining part of the
circuit is \emph{not} combinational can be quite complex.
With the compositional approach the two tasks can be kept isolated by using
blackboxes, layered explanations, and graphical reasoning.