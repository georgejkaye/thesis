\chapter{Potential applications}\label{chap:semantics-applications}

So far, we have been concerned largely with \emph{theoretical} concepts; we have
shown how the categorical framework of sequential digital circuits is rigorous
enough to handle composing circuits in sequence, parallel or with the trace
without causing any of the three semantic models to become degenerate.
Unfortunately, this is not enough for people in industry, who are more concerned
with the \emph{practical} benefits of the framework: what can we do with it that
annot already be done?

Circuit design is already a very well-studied area and existing technologies are
incredibly successful.
Our framework is therefore not meant to \emph{replace} the existing
technologies, but to \emph{complement} them by highlighting different
perspectives on reasoning with sequential digital circuits.
While the ideas provided in this section are certainly not industry-grade
applications, they are intended to demonstrate the potential of what
the compositional theory can bring to the table.

\section{Tidying up}

When building a circuit, it is often desirable to reduce the number of wires
and components used; this reduces both the physical size of the circuit and its
power consumption.
We can use partial evaluation to transform a circuit into a more minimal form.

\begin{definition}[Tidying rules]
    Let the set of \emph{tidying rules} be defined as the rules in
    \cref{fig:tidy-rules}.
\end{definition}

Most of the tidying rules are self-explanatory; the final rule is necessary in
order to deal with traced circuits with no outputs.
Since all circuits with no outputs have the same behaviour, we are permitted tot
`cut' the trace and get a circuit we can apply more tidying rules to.
As non-delay-guarded feedback is already handled by the
\((\instantfeedbackeqn)\) rule, we only need to consider the delay-guarded case.

\begin{figure}
    \centering
    \(
    \iltikzfig{circuits/axioms/gate-stub-lhs}
    \reduction
    \iltikzfig{circuits/axioms/gate-stub-rhs}
    \)
    \quad
    \(
    \iltikzfig{circuits/axioms/stub-lhs}
    \reduction
    \iltikzfig{strings/monoidal/empty}
    \)
    \quad
    \(
    \iltikzfig{circuits/examples/tidying/redundant-trace-lhs}[box=f]
    \reduction
    \iltikzfig{circuits/examples/tidying/redundant-trace-rhs}[box=f]
    \)
    \\[0.25em]
    \rule{\textwidth}{0.1mm}
    \\[0.5em]
    \(
    \iltikzfig{strings/structure/monoid/unitality-l-lhs}
    \reduction
    \iltikzfig{strings/structure/monoid/unitality-l-rhs}
    \)
    \quad
    \(
    \iltikzfig{strings/structure/monoid/unitality-r-lhs}
    \reduction
    \iltikzfig{strings/structure/monoid/unitality-r-rhs}
    \)
    \quad
    \(
    \iltikzfig{strings/structure/comonoid/unitality-l-lhs}
    \reduction
    \iltikzfig{strings/structure/comonoid/unitality-l-rhs}
    \)
    \\[0.25em]
    \rule{\textwidth}{0.1mm}
    \\[0.5em]
    \(
    \iltikzfig{strings/structure/comonoid/unitality-r-lhs}
    \reduction
    \iltikzfig{strings/structure/comonoid/unitality-r-rhs}
    \)
    \quad
    \(
    \iltikzfig{strings/structure/bialgebra/merge-discard-lhs}
    \reduction
    \iltikzfig{strings/structure/bialgebra/merge-discard-rhs}
    \)
    \quad
    \(
    \iltikzfig{circuits/axioms/unobservable-lhs}
    \reduction
    \iltikzfig{circuits/axioms/unobservable-rhs}
    \)
    \\[0.25em]
    \rule{\textwidth}{0.1mm}
    \\[0.5em]
    \caption{Rules for tidying up circuits in Mealy form}
    \label{fig:tidy-rules}
\end{figure}

\begin{proposition}
    Applying the tidying rules to a circuit in Mealy form is confluent and
    terminating.
\end{proposition}
\begin{proof}
    The tidying rules always decrease the size of the circuit.
    The only choice is raised when there is a trace around a combinational
    circuit, but this does not change the internal structure of the subcircuit,
    so rule applications are prevented.
    Moreover, since all this rule does is `cut' a trace, it does not matter if
    this is performed `all in one go', or each feedback loop is cut one by one.
\end{proof}
\section{Partial evaluation}\label{sec:partial}

Partial evaluation~\cite{jones1996introduction} is a paradigm used in software
optimisation in which programs are `evaluated as much as possible' while only
some of the inputs are specified.
For example, it may be the case that a particular input to a program is fixed
for long periods of time; using partial evaluation, we can define a program
specialised for this input.
This program might run significantly faster than the original.

There has been work into partial evaluation for hardware, such as constant
propagation~\cite{singh1996expressing,singh1999partial} and
unfolding~\cite{thompson2006bitlevel}.
However, this has been relatively informal, and can be made rigorous using the
categorical framework.
In this section we will focus on how we could extend the reduction-based
operational semantics to define automatic procedures for applying partial
evaluation to circuits.

\subsection{Shortcut rules}\label{sec:shortcut}

It is often the case that we know that some of the inputs to a circuit are
fixed.
This can be modelled by precomposing the relevant input with an
\emph{infinite waveform} \(
\iltikzfig{circuits/components/waveforms/infinite-register}[val=v]
\).
We can propagate these waveforms across a circuit to see if we can reduce it to
a circuit \emph{specialised} for these inputs.

To propagate waveforms across circuits we need to derive a version of the
\((\gateeqn)\) rule for applying waveforms to primitives.
These rules are illustrated in \cref{fig:waveform-rules}.

\begin{figure}
    \centering
    \(
    \iltikzfig{circuits/examples/shortcuts/waveform-primitive-lhs}
    \reduction
    \iltikzfig{circuits/examples/shortcuts/waveform-primitive-rhs}
    \)
    \quad
    \(
    \iltikzfig{circuits/examples/shortcuts/waveform-fork-lhs}
    \reduction
    \iltikzfig{circuits/examples/shortcuts/waveform-fork-rhs}
    \)
    \\[0.25em]
    \rule{\textwidth}{0.1mm}
    \\[0.5em]
    \(
    \iltikzfig{circuits/examples/shortcuts/waveform-join-lhs}
    \reduction
    \iltikzfig{circuits/examples/shortcuts/waveform-join-rhs}
    \)
    \\[0.25em]
    \rule{\textwidth}{0.1mm}
    \\[0.5em]
    \caption{Rules for infinite waveforms}
    \label{fig:waveform-rules}
\end{figure}

This is not the only way we can partially evaluate with some inputs.
In some interpretations, it may be that we learn something about the output of
a primitive with only some of the inputs specified.

\begin{example}[Belnap shortcuts]
    In the Belnap interpretation \(\belnapinterpretation\), if one applies a
    false value to an \(\andgate\) gate then it will output false regardless of
    the other input.
    Similarly, if one applies a true value to an \(\orgate\) gate it will output
    true.
    Conversely, if one applies a true value to an \(\andgate\) gate or a false
    value to an \(\orgate\) gate, it will act as the identity on the other
    input.
\end{example}

These `shortcuts' can also be implemented as rules, as illustrated in
\cref{fig:shortcut-waveform-rules}.
Note that here the value that `triggers' the shortcut must be contained within
an infinite waveform; if we applied the rule with just an instantaneous value,
this value would produce \(\bot\) on ticks after the first and the rule would
be unsound.

\begin{figure}
    \centering
    \(
    \iltikzfig{circuits/examples/shortcuts/and-f-infinite-lhs}
    \reduction
    \iltikzfig{circuits/examples/shortcuts/and-f-infinite-rhs}
    \)
    \quad
    \(
    \iltikzfig{circuits/examples/shortcuts/or-t-infinite-lhs}
    \reduction
    \iltikzfig{circuits/examples/shortcuts/or-t-infinite-rhs}
    \)
    \\[0.3em]
    \rule{\textwidth}{0.1mm}
    \\[0.5em]
    \(
    \iltikzfig{circuits/examples/shortcuts/and-f-comm-infinite-lhs}
    \reduction
    \iltikzfig{circuits/examples/shortcuts/and-f-infinite-rhs}
    \)
    \quad
    \(
    \iltikzfig{circuits/examples/shortcuts/or-t-comm-infinite-lhs}
    \reduction
    \iltikzfig{circuits/examples/shortcuts/or-t-infinite-rhs}
    \)
    \\[0.3em]
    \rule{\textwidth}{0.1mm}
    \\[0.5em]
    \(
    \iltikzfig{circuits/examples/shortcuts/and-t-infinite-lhs}
    \reduction
    \iltikzfig{circuits/examples/shortcuts/and-t-infinite-rhs}
    \)
    \quad
    \(
    \iltikzfig{circuits/examples/shortcuts/or-f-infinite-lhs}
    \reduction
    \iltikzfig{circuits/examples/shortcuts/or-f-infinite-rhs}
    \)
    \\[0.3em]
    \rule{\textwidth}{0.1mm}
    \\[0.5em]
    \(
    \iltikzfig{circuits/examples/shortcuts/and-t-comm-infinite-lhs}
    \reduction
    \iltikzfig{circuits/examples/shortcuts/and-t-infinite-rhs}
    \)
    \quad
    \(
    \iltikzfig{circuits/examples/shortcuts/or-f-comm-infinite-lhs}
    \reduction
    \iltikzfig{circuits/examples/shortcuts/or-f-infinite-rhs}
    \)
    \\[0.3em]
    \rule{\textwidth}{0.1mm}
    \\[0.5em]
    \caption{Belnap shortcut rules for waveforms}
    \label{fig:shortcut-waveform-rules}
\end{figure}

\begin{example}[Control switches]
    Recall that a \emph{multiplexer} is a circuit component constructed as \(
    \iltikzfig{circuits/components/gates/mux}
    \coloneqq
    \iltikzfig{circuits/components/gates/mux-construction}
    \).
    The first input is a \emph{control} which specifies which of the two other
    input signals is produced as the output signal.
    It is often the case that these control signals will be fixed for long
    periods of time; perhaps they specify some sort of global circuit
    configuration.

    Consider the circuit \(
    \iltikzfig{circuits/examples/control/circuit}
    \), in which the control signal to the multiplexer determines which of two
    subcircuits will become the output.
    We will assume that the control signal is held at false, and reduce
    accordingly by instantiating the rule in \cref{fig:waveform-rules} detailing
    the interaction of gates and waveforms to the \(\notgate\) case.
    \begin{gather*}
        \iltikzfig{circuits/examples/control/circuit-mux-applied}
        \coloneqq
        \iltikzfig{circuits/examples/control/circuit-applied}
        \reduction
        \\[1em]
        \iltikzfig{circuits/examples/control/circuit-applied-1}
        \reduction
        \iltikzfig{circuits/examples/control/circuit-applied-2}
        \reduction
        \\[1em]
        \iltikzfig{circuits/examples/control/circuit-applied-3}
        \reduction
        \iltikzfig{circuits/examples/control/circuit-applied-4}
        \reduction
        \iltikzfig{circuits/examples/control/circuit-applied-5}
    \end{gather*}
\end{example}

\subsection{Shortcuts after streaming}

The rules in the previous sections are intended for use on circuits before we
even apply values to them.
However, there is still potential for partial evaluation when we consider the
outputs of a circuit one step at a time.
To do this, we can apply variants of the shortcut rules \emph{after} performing
streaming for some inputs.
These variants are illustrated in \cref{fig:shortcuts}.

\begin{figure}
    \centering
    \(
    \iltikzfig{circuits/examples/shortcuts/and-f-instant-lhs}
    \reduction
    \iltikzfig{circuits/examples/shortcuts/and-f-instant-rhs}
    \)
    \quad
    \(
    \iltikzfig{circuits/examples/shortcuts/or-t-instant-lhs}
    \reduction
    \iltikzfig{circuits/examples/shortcuts/or-t-instant-rhs}
    \)
    \quad
    \(
    \iltikzfig{circuits/examples/shortcuts/and-t-instant-lhs}
    \reduction
    \iltikzfig{circuits/examples/shortcuts/and-t-instant-rhs}
    \)
    \\[0.4em]
    \rule{\textwidth}{0.1mm}
    \\[0.5em]
    \(
    \iltikzfig{circuits/examples/shortcuts/or-f-instant-lhs}
    \reduction
    \iltikzfig{circuits/examples/shortcuts/or-f-instant-rhs}
    \)
    \quad
    \(
    \iltikzfig{circuits/examples/shortcuts/and-f-comm-instant-lhs}
    \reduction
    \iltikzfig{circuits/examples/shortcuts/and-f-instant-rhs}
    \)
    \quad
    \(
    \iltikzfig{circuits/examples/shortcuts/or-t-comm-instant-lhs}
    \reduction
    \iltikzfig{circuits/examples/shortcuts/or-t-instant-rhs}
    \)
    \\[0.4em]
    \rule{\textwidth}{0.1mm}
    \\[0.5em]
    \(
    \iltikzfig{circuits/examples/shortcuts/and-t-comm-instant-lhs}
    \reduction
    \iltikzfig{circuits/examples/shortcuts/and-t-instant-rhs}
    \)
    \quad
    \(
    \iltikzfig{circuits/examples/shortcuts/or-f-comm-instant-lhs}
    \reduction
    \iltikzfig{circuits/examples/shortcuts/or-f-instant-rhs}
    \)
    \\[0.4em]
    \rule{\textwidth}{0.1mm}
    \caption{Examples of `instantaneous' shortcut rules}
    \label{fig:shortcuts}
\end{figure}

\begin{example}[Blocking boxes]\label{ex:blocking-boxes}
    Consider the circuit \(
    \iltikzfig{circuits/examples/blocking/circuit}
    \), which contains a `blackbox' combinational circuit \(
    \iltikzfig{strings/category/f}[box=f, colour=comb]
    \) with unknown behaviour.

    Even though we cannot directly reduce the blackbox, if we set the first
    input to false and use the shortcut rule above, we can still produce an
    output value.
    \[
        \iltikzfig{circuits/examples/blocking/applied-false}
        \reduction
        \iltikzfig{circuits/examples/blocking/streamed-false}
        \reduction
        \iltikzfig{circuits/examples/blocking/reduced}
    \]
\end{example}

As well as removing redundant blackboxes, judicious use of shortcut
reductions can dramatically reduce the reductions needed to get the outputs of a
circuit.

\subsection{Protocols}\label{sec:uncertain}

Sometimes we may not know the exact inputs to a circuit, but know that they make
up a fixed subset of all possible inputs, or they follow some sort of protocol.
We can implement this in our reduction framework with \emph{uncertain values}
which we either know nothing about or know can only take some specified values.

\begin{definition}
    Let \(\scircsigmap\) be the result of extending \(\scircsigma\) with value
    generators for each word \(\listvar{v?} \in \freemon{\values}\).
\end{definition}

The additional value generators indicate that they could produce one of multiple
possible values.
When a circuit contains uncertain values \(
\iltikzfig{circuits/components/values/vs-longer}[val=\listvar{v_0?}],
\iltikzfig{circuits/components/values/vs-longer}[val=\listvar{v_1?}],
\dots
\iltikzfig{circuits/components/values/vs-longerer}[val=\listvar{v_{n-1}?}],
\) where the maximum length of a given \(v_i\) is \(k\), there are \(k\)
possible value assignments.
For a given assignment \(i < k\), each value will produce a concrete value
defined as \(v_i(j)\) if \(|v_i| > j\) or \(\bot\) otherwise.

To avoid confusion with our syntax sugar for arbitrary-width values, we will
always end uncertain value lists with \(?\).
When writing out specific uncertain words, we delimit the elements with vertical
bars like \(\mathsf{f}|\mathsf{t}\) to allude to the fact that this value is
either the first \emph{or} the second element.

\begin{example}
    If a circuit contains uncertain values \(
    \iltikzfig{circuits/components/values/vs-longer}[val=\belnapfalse|\belnaptrue]
    \) and \(
    \iltikzfig{circuits/components/values/vs-longer}[val=\belnaptrue|\belnapfalse]
    \) in a circuit, then there are two universes to consider, one where the
    values output \(\belnapfalse\belnaptrue\) and one where they output
    \(\belnaptrue\belnapfalse\).
    If we add in another uncertain value with three possible values, \(
    \iltikzfig{circuits/components/values/vs-even-longer}[val=\belnaptrue|\belnapfalse|\top]
    \), we now have three possible universes, in which the values output
    \(\belnapfalse\belnaptrue\belnaptrue\),
    \(\belnaptrue\belnapfalse\belnapfalse\), and
    \(\bot\bot\top\) respectively.
\end{example}

To reason with uncertain values in the reductional framework we need to add
rules for processing them.
Once again it is useful to have versions for both waveforms and values, for
reasoning before and during execution.

\begin{definition}[Uncertain rules]
    The \emph{uncertain rules} are listed in \cref{fig:uncertain-rules}.
\end{definition}

After applying uncertain values to a primitive, it may turn out that all the
possibilities are in fact the same.
This removes any uncertainty, and means the value can be treated as an
ordinary value in future reductions and outputs.

\begin{figure}
    \centering
    \(
    \iltikzfig{circuits/examples/uncertain/waveform-primitive-lhs}[val=\listvar{v?}]
    \reduction
    \iltikzfig{circuits/examples/uncertain/waveform-primitive-rhs}
    \)
    \\[0.4em]
    \rule{\textwidth}{0.1mm}
    \\[0.5em]
    \(
    \iltikzfig{circuits/examples/uncertain/waveform-fork-lhs}[val=\listvar{v?}]
    \reduction
    \iltikzfig{circuits/examples/uncertain/waveform-fork-rhs}[val=\listvar{v?}]
    \)
    \quad
    \(
    \iltikzfig{circuits/examples/uncertain/waveform-join-lhs}[val1=\listvar{v?},val2=\listvar{w?}]
    \reduction
    \iltikzfig{circuits/examples/uncertain/waveform-join-rhs}[val1=\listvar{v?},val2=\listvar{w?}]
    \)
    \\[0.4em]
    \rule{\textwidth}{0.1mm}
    \\[0.5em]
    \(
    \iltikzfig{circuits/examples/uncertain/instantaneous-primitive-lhs}[val=\listvar{v?}]
    \reduction
    \iltikzfig{circuits/examples/uncertain/instantaneous-primitive-rhs}
    \)
    \quad
    \(
    \iltikzfig{circuits/examples/uncertain/instantaneous-fork-lhs}[val=\listvar{v?}]
    \reduction
    \iltikzfig{circuits/examples/uncertain/instantaneous-fork-rhs}[val=\listvar{v?}]
    \)
    \quad
    \(
    \iltikzfig{circuits/examples/uncertain/instantaneous-join-lhs}[val1=\listvar{v?},val2=\listvar{w?}]
    \reduction
    \iltikzfig{circuits/examples/uncertain/instantaneous-join-rhs}[val1=\listvar{v?},val2=\listvar{w?}]
    \)
    \\[0.4em]
    \rule{\textwidth}{0.1mm}
    \\[0.5em]
    \(
    \iltikzfig{circuits/components/values/vs-longer}[val=\listvar{v?}]
    \reduction
    \iltikzfig{circuits/components/values/vs}[val=v]
    \)
    \,\,
    if \(\forall i,j < |\listvar{v?}|\), \(\listvar{v?}(i) = \listvar{v?}(j)\)
    \\[0.4em]
    \rule{\textwidth}{0.1mm}
    \\[0.5em]
    \(
    \iltikzfig{circuits/components/waveforms/infinite-register}[val=\listvar{v?}]
    \reduction
    \iltikzfig{circuits/components/waveforms/infinite-register}[val=v]
    \)
    \,\,
    if \(\forall i,j < |\listvar{v?}|\), \(\listvar{v?}(i) = \listvar{v?}(j)\)
    \\[0.4em]
    \rule{\textwidth}{0.1mm}
    \caption{Rules for uncertain values}
    \label{fig:uncertain-rules}
\end{figure}

\begin{example}[Protocols]\label{ex:protocols}
    One sticking point that arises when using the categorical framework is the
    presence of the \(\bot\) and \(\top\) values, which would not normally
    be explicitly provided to a circuit.
    These values mean that some well-known Boolean identities do not always hold.
    By using uncertain values, we can specify the values that \emph{will} be
    applied to a circuit and apply reductions that are not valid in general but
    are in this context.

    In the following example, setting the first two inputs to true/false
    inverses reduces the circuit to one with combinational behaviour.
    \begin{gather*}
        \iltikzfig{circuits/examples/protocol/circuit-with-protocol}
        \reduction
        \\[0.25em]
        \iltikzfig{circuits/examples/protocol/circuit-with-protocol-1}
        \reduction
        \\[0.25em]
        \iltikzfig{circuits/examples/protocol/circuit-with-protocol-2}
        \reduction
        \\[0.25em]
        \iltikzfig{circuits/examples/protocol/circuit-with-protocol-3}
        \reduction
        \\[0.25em]
        \iltikzfig{circuits/examples/protocol/circuit-with-protocol-4}
        \reduction
        \iltikzfig{circuits/examples/protocol/circuit-with-protocol-5}
        \reduction
        \\[0.25em]
        \iltikzfig{circuits/examples/protocol/circuit-with-protocol-6}
        \reduction
        \iltikzfig{circuits/examples/protocol/circuit-with-protocol-7}
        \reduction
        \iltikzfig{circuits/examples/protocol/circuit-with-protocol-8}
    \end{gather*}
\end{example}
\section{Layers of abstraction}\label{sec:abstraction}

Circuits can be viewed at multiple levels of abstraction.
One could drop down to the level of transistors, as illustrated in
\cite[Sec.\ 4.1]{ghica2017diagrammatic}.
Alternatively, one could become more abstract, setting the generators to be
\emph{subcircuits}, such as arithmetic operations.

The levels of abstraction need not remain isolated.
Using \emph{layered explanations}~\cite{lobski2022string}, multiple signatures
can be mixed in one diagram, with the subcircuits acting as `windows' into
different levels of abstraction, and drawn using `functorial boxes'
\cite{mellies2006functorial}.

\begin{example}[Implementation]
    Suppose one is working in a high-level signature \(\Sigma_+\) containing a
    generator \(
    \iltikzfig{circuits/components/gates/gate}[gate=?,colour=comb]
    \), representing an \emph{IP core}: a circuit that has a known behaviour but
    with an unknown implementation.
    This component can be left as a blackbox and evaluated as
    demonstrated above.

    The designer then attempts to design their own implementation using the
    gate-level signature \(\belnapsignature\).
    To synthesise the final circuit, a map is defined from generators in \(
    \Sigma_+
    \) to morphisms in \(\scirc{\belnapsignature}\), which induces a functor
    \(\scirc{\Sigma_+} \to \scirc{\belnapsignature}\).
    Different implementations can be defined as different maps, and hence
    different functors.
    These circuits can then be tested to see if they act as intended.
    \begin{gather*}
        \iltikzfig{circuits/examples/implementation/circuit}
        \xRightarrow{\text{Imp.\ 1}}
        \iltikzfig{circuits/examples/implementation/circuit-implemented}
        \quad
        \iltikzfig{circuits/examples/implementation/circuit}
        \xRightarrow{\text{Imp.\ 2}}
        \iltikzfig{circuits/examples/implementation/circuit-implemented-2}
    \end{gather*}
\end{example}
\section{Refining circuits}\label{sec:refining}

A key part of circuit design comes in \emph{optimising circuits}: making them
run as fast as possible and reduce the \emph{clock cycle}.

\begin{example}[Retiming]
    The clock cycle of a circuit is determined by the longest paths between
    registers. Altering the paths between registers can be achieved using
    \emph{retiming}~\cite{leiserson1991retiming}: moving registers across gates.
    This is modelled by the streaming rule (\cref{lem:streaming});
    forward retiming (streaming left to right) is always possible
    but for backward retiming (streaming right to left), the value
    in the register must be in the image of the gates.
\end{example}

When reasoning equationally, the behaviour of the circuits on either side of the
equation must have exactly the same behaviour.
However, when reasoning with circuits it is sometimes the case that this is too
strict an assertion; we are looking for circuits that output the same outputs
but over a shorter period of time.
This means we may wish to use transformations that only `morally' preserve the
behaviour of a circuit.

\begin{definition}
    For two finite sequences \(
    \listlistvar{v},\listlistvar{w} \in (\valuetuple{m})^k
    \), we say that \(\listlistvar{w}\) is a \emph{stretching} of
    \(\listlistvar{v}\), written \(\listlistvar{v} \ll \listlistvar{w}\), if
    \(\listlistvar{w}\) contains the characters of \(\listlistvar{v}\) but
    possibly repeated or with additional \(\bot\) characters e.g.\ \(
    \belnaptrue\belnapfalse
    \ll
    \bot\bot\belnaptrue\belnaptrue\bot\belnapfalse
    \).
\end{definition}

\begin{definition}
    Given two sequential circuits \(
    \iltikzfig{strings/category/f}[box=f,colour=seq,dom=m,cod=n]
    \) and \(
    \iltikzfig{strings/category/f}[box=g,colour=seq,dom=m,cod=n]
    \) with \(c\) and \(c^\prime\) delay components respectively, we say that \(
    \iltikzfig{strings/category/f}[box=f,colour=seq,dom=m,cod=n]
    \) is \emph{logically equvialent} to \(
    \iltikzfig{strings/category/f}[box=g,colour=seq,dom=m,cod=n]
    \), written \(
    \iltikzfig{strings/category/f}[box=f,colour=seq,dom=m,cod=n]
    \ll
    \iltikzfig{strings/category/f}[box=g,colour=seq,dom=m,cod=n]
    \), if for all sequences \(\listlistvar{v},\listlistvar{w}\) produced by the
    productive operational semantics for inputs of length
    \(\mathsf{max}(c,c^\prime)\),  \(\listlistvar{v} \ll \listlistvar{w}\)
\end{definition}

Including this notion of equivalence in algebraic reasoning allows us to reason
with \emph{inequalities} as well as equalities.
This means that given a circuit, we can use equations as normal, determine that
one component

\begin{example}
    Take the circuit \(
    \iltikzfig{circuits/examples/refinement/circuit}
    \).
    For an input stream \(\sigma\), this circuit produces output stream
    \(
    \bot \land \sigma(0) \streamcons \sigma(1)
    \streamcons \sigma(2) \streamcons \dots
    \).
    By using equalities and logical eqivalence we can obtain a much simpler
    circuit:
    \[
        \iltikzfig{circuits/examples/refinement/circuit}
        =
        \iltikzfig{circuits/examples/refinement/circuit-1}
        \ll
        \iltikzfig{circuits/examples/refinement/circuit-2}
        =
        \iltikzfig{circuits/examples/refinement/circuit-3}
    \]
\end{example}

While this is a somewhat contrived toy example, it is possible that this
technique could be applied to actual circuit optimisation procedures.

\begin{example}[Pipelining]
    \emph{Pipelining}~\cite{parhi1999vlsi} is a technique in which more
    registers are inserted into a circuit to increase throughput.
    This can be emulated in the compositional framework by applying
    transformations locally to registers.
    Ordinarily, such transformations can obfuscate a circuit's behaviour since
    the state space dramatically changes.
    In the compositional model, the structure of the circuit is left relatively
    untouched so this is less of an issue.
\end{example}

Not all circuit transformations are for the purpose of improving performance.
Sometimes additional components must be bolted onto a circuit for \emph{testing}
purposes.

\begin{example}[Scan chains]
    A common way of testing circuits is by using a
    \emph{scan chain}~\cite{mourad2000principles}, a way of forcing the
    inputs to flipflops to test how specific states affect the outputs of the
    circuit.
    Adding a flipflop to a scan chain requires some extra inputs: the
    \(\mathsf{scan}_\mathsf{en}\) wire toggles if the flipflop operates in
    normal mode or if it takes \(\mathsf{scan}_\mathsf{in}\) as its value.
    \[
        \iltikzfig{circuits/examples/scan-chain/flipflop-before-chain}
        \xRightarrow{\text{scan}}
        \iltikzfig{circuits/examples/scan-chain/scan-chain}
    \]
\end{example}

One could factor in these transformations when designing the circuit, but this
can obfuscate the design of the actual logic.
Additionally, applying these transformations where the remaining part of the
circuit is \emph{not} combinational can be quite complex.
With the compositional approach the two tasks can be kept isolated by using
blackboxes, layered explanations, and graphical reasoning.
\section{Implementation}

Throughout this section we have discussed some potential applications for the
compositional theory for digital circuits.
However, the examples have been kept to relatively small toy examples for
ease of presentation and explanation.
For readers less convinced by theoretical results, this might not be enough; how
can the framework be adapted for real-life examples?

Because examples can quickly balloon in size, it becomes impractical to develop
and work through them by hand.
Instead, we turn to our old friend the computer and ask it to do the hard work
for us.
But how do we even communicate such things with a computer?
This will be answered in great detail in the next part of this thesis.




