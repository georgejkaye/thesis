\section{Tidying up}\label{sec:tidy}

When building a circuit, it is often desirable to reduce the number of wires
and components used; this reduces both the physical size of the circuit and its
power consumption.
We can use partial evaluation to transform a circuit into a more minimal form.

\begin{definition}[Tidying rules]
    \index{tidying}
    Let the set of \emph{tidying rules} be defined as the rules in
    \cref{fig:tidy-rules}.
\end{definition}

Most of the tidying rules are self-explanatory; the final rule is necessary in
order to deal with traced circuits with no outputs.
Since all circuits with no outputs have the same behaviour, we are permitted to
cut the trace to obtain a circuit we can apply more tidying rules to.
As non-delay-guarded feedback is already handled by the
\((\instantfeedbackeqn)\) rule, we only need to consider the delay-guarded case.

\begin{figure}
    \centering
    \(
    \iltikzfig{strings/structure/comonoid/unitality-l-lhs}
    \reduction
    \iltikzfig{strings/structure/comonoid/unitality-l-rhs}
    \)
    \quad
    \(
    \iltikzfig{strings/structure/comonoid/unitality-r-lhs}
    \reduction
    \iltikzfig{strings/structure/comonoid/unitality-r-rhs}
    \)
    \quad
    \(
    \iltikzfig{strings/structure/bialgebra/merge-discard-lhs}
    \reduction
    \iltikzfig{strings/structure/bialgebra/merge-discard-rhs}
    \)
    \\[0.5em]
    \(
    \iltikzfig{circuits/axioms/gate-stub-lhs}
    \reduction
    \iltikzfig{circuits/axioms/gate-stub-rhs}
    \)
    \quad
    \(
    \iltikzfig{circuits/axioms/bundler-stub-lhs}
    \reduction
    \iltikzfig{circuits/axioms/bundler-stub-rhs}
    \)
    \quad
    \(
    \iltikzfig{circuits/axioms/stub-lhs}
    \reduction
    \iltikzfig{strings/monoidal/empty}
    \)
    \quad
    \(
    \iltikzfig{circuits/axioms/unobservable-lhs}
    \reduction
    \iltikzfig{circuits/axioms/unobservable-rhs}
    \)
    \quad
    \(
    \iltikzfig{circuits/examples/tidying/redundant-trace-lhs}[box=f]
    \reduction
    \iltikzfig{circuits/examples/tidying/redundant-trace-rhs}[box=f]
    \)
    \caption{Rules for tidying up circuits in Mealy form}
    \label{fig:tidy-rules}
\end{figure}

\begin{proposition}
    Applying the tidying rules to a circuit in Mealy form is confluent and
    terminating.
\end{proposition}
\begin{proof}
    The tidying rules always decrease the size of the circuit.
    The only choice is raised when there is a trace around a combinational
    circuit, but this does not change the internal structure of the subcircuit,
    so rule applications are prevented.
    Moreover, since all this rule does is cut a trace, it does not matter if
    this is performed all in one go, or each feedback loop is cut one by one.
\end{proof}