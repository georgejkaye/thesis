\section{Layers of abstraction}\label{sec:abstraction}

Circuits can be viewed at multiple levels of abstraction.
One could drop down to the level of transistors, as illustrated in
\cite[Sec.\ 4.1]{ghica2017diagrammatic}.
Alternatively, one could become more abstract, setting the generators to be
\emph{subcircuits}, such as arithmetic operations.

The levels of abstraction need not remain isolated.
Using \emph{layered explanations}~\cite{lobski2022string}, multiple signatures
can be mixed in one diagram, with the subcircuits acting as `windows' into
different levels of abstraction, and drawn using `functorial boxes'
\cite{mellies2006functorial}.

\begin{example}[Implementation]
    Suppose one is working in a high-level signature \(\Sigma_+\) containing a
    generator \(
    \iltikzfig{circuits/components/gates/gate}[gate=?,colour=comb]
    \), representing an \emph{IP core}: a circuit that has a known behaviour but
    with an unknown implementation.
    This component can be left as a blackbox and evaluated as
    demonstrated above.

    The designer then attempts to design their own implementation using the
    gate-level signature \(\belnapsignature\).
    To synthesise the final circuit, a map is defined from generators in \(
    \Sigma_+
    \) to morphisms in \(\scirc{\belnapsignature}\), which induces a functor
    \(\scirc{\Sigma_+} \to \scirc{\belnapsignature}\).
    Different implementations can be defined as different maps, and hence
    different functors.
    These circuits can then be tested to see if they act as intended.
    \begin{gather*}
        \iltikzfig{circuits/examples/implementation/circuit}
        \xRightarrow{\text{Imp.\ 1}}
        \iltikzfig{circuits/examples/implementation/circuit-implemented}
        \quad
        \iltikzfig{circuits/examples/implementation/circuit}
        \xRightarrow{\text{Imp.\ 2}}
        \iltikzfig{circuits/examples/implementation/circuit-implemented-2}
    \end{gather*}
\end{example}