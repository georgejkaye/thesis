\section{Implementation}

Throughout this section we have discussed some potential applications for the
compositional theory for digital circuits.
However, the examples have been kept to relatively small toy examples for
ease of presentation and explanation.
For readers less convinced by theoretical results, this might not be enough; how
can the framework be adapted for real-life examples?
Because examples can quickly balloon in size, it becomes impractical to develop
and work through them by hand.
Instead, it is necessary to pass the work along to a computer to generate and
test things \emph{automatically}.
But how do we even communicate such things with a computer?
This will be answered in great detail in the next part of this thesis.

