\section{Denotational semantics for generalised circuits}

Although we have discussed the denotation semantics in terms of monochromatic
circuit signatures, it is straightforward to extend the results to categories
generated over \emph{generalised} circuit signatures

As the input and output wires of generalised circuits are not all the same
width, the domains and
codomains of circuit stream functions will no longer be streams of tuples of
values, but streams of tuples of tuples of values!

\begin{notation}
    Given a set \(A\) and a word \(\listvar{v} \in \natplus^\star\) of
    length \(n\), we write \(
    A^{\listvar{v}}
    \coloneqq
    A^{\listvar{v}(0)}
    \times
    A^{\listvar{v}(1)}
    \times
    \dots
    \times
    A^{\listvar{v}(n-1)}
    \).
\end{notation}

When the primitive interfaces are lists containing only \(1\), the domains and
codomains of morphisms in \(\funcig\) are tuples of single-element tuples,
which can be equivalently viewed as `flat' tuples as in \(\funci\).

We generalise the category of streams in much the same way.

\begin{definition}
    Let \(\mealyig\) be the \(\natplus\)-coloured PROP in which the moprhisms
    \(\listvar{m} \to \listvar{n}\) are the monotone
    \((\valuetuple{\listvar{m}}, \valuetuple{\listvar{n}})\)-Mealy machines.
    Let \(\morph{\circuittomealyig}{\scircsigmaplus}{\mealyig}\) be defined
    in the same manner as \(\circuittomealyi\).
\end{definition}

\begin{definition}
    Let \(\streamig\) be the \(\natplus\)-coloured PROP in which the morphisms
    \(\listvar{m} \to \listvar{n}\) are the causal, monotone, and finitely
    specified stream functions \(
    \valuetuplestream{\listvar{m}} \to \valuetuplestream{\listvar{n}}
    \).
\end{definition}




\begin{corollary}

\end{corollary}
