\section{Denotational semantics for generalised circuits}

Although we have discussed the denotational semantics in terms of monochromatic
circuit signatures, it is straightforward to extend the results to categories
generated over \emph{generalised} circuit signatures

In the semantic categories \(\funci\), \(\streami\), and \(\mealyi\), the
morphisms are all variants on functions of the form
\(\valuetuple{m} \to \valuetuple{n}\) that operate on powers of elements in
\(\values\): one element for each (single-bit) input or output wire.
In the generalised setting, these input and output wires may not all be the
same width, so the input and output sets must be \emph{powers of powers} of
values.

\begin{notation}
    Given a set \(A\) and a word \(\listvar{v} \in \natplus^\star\) of
    length \(n\), we write \(
    A^{\listvar{v}}
    \coloneqq
    A^{\listvar{v}(0)}
    \times
    A^{\listvar{v}(1)}
    \times
    \dots
    \times
    A^{\listvar{v}(n-1)}
    \).
\end{notation}

Note that for a word \(\listvar{m} \coloneqq 11\dots1\) of length \(k\), the set
\(A^{\listvar{m}} = A^1 \times A^1 \times \dots A^1\) is isomorphic to \(A^k\),
much like how setting the set of colours in a coloured PROP to the singleton
recovers a monochromatic PROP.

The semantic categories can now be extended to these \emph{coloured} interfaces.

\begin{definition}
    Let \(\funcig\) be the \(\natplus\)-coloured PROP in which the morphisms
    \(\listvar{m} \to \listvar{n}\) are the monotone functions
    \(\valuetuple{\listvar{m}} \to \valuetuple{\listvar{n}}\).
    Let \(\streamig\) be the \(\natplus\)-coloured PROP in which the morphisms
    \(\listvar{m} \to \listvar{n}\) are the causal, monotone, and finitely
    specified stream functions \(
    \valuetuplestream{\listvar{m}} \to \valuetuplestream{\listvar{n}}
    \).
    Let \(\mealyig\) be the \(\natplus\)-coloured PROP in which the moprhisms
    \(\listvar{m} \to \listvar{n}\) are the monotone
    \((\valuetuple{\listvar{m}}, \valuetuple{\listvar{n}})\)-Mealy machines.
\end{definition}

The various PROP morphisms between these categories are defined in a similar way
to the monochromatic versions, but now we have to account for the structural
generators for each \(n \in \natplus\), as well as the bundlers.

\begin{definition}
    Let \(\morph{\circuittofuncig}{\ccircsigmag}{\funcig}\) be the coloured PROP
    morphism with action defined as%
    \vspace{-\abovedisplayskip}
    \begin{align*}
        \circuittofuncig[
            \iltikzfig{circuits/components/gates/gate}[gate=p,dom=\listvar{m},cod=\listvar{n}]
        ]
         & \coloneqq
        \gateinterpretation[p]
        \\
        \circuittofuncig[
            \iltikzfig{strings/structure/comonoid/copy}[colour=comb,obj=n]
        ]
         & \coloneqq
        (\listvar{v}) \mapsto (\listvar{v}, \listvar{v})
        \\
        \circuittofuncig[
            \iltikzfig{strings/structure/comonoid/discard}[colour=comb,obj=n]
        ]
         & \coloneqq
        (\listvar{v}) \mapsto ()
        \\
        \circuittofuncig[
            \iltikzfig{strings/structure/monoid/merge}[colour=comb,obj=n]
        ]
         & \coloneqq
        (\listvar{v}, \listvar{w})
        \mapsto \listvar{v} \sqcup \listvar{w}
        \\
        \circuittofuncig[
            \iltikzfig{strings/structure/monoid/init}[colour=comb,obj=n]
        ]
         & \coloneqq
        () \mapsto \bot^n
        \\
        \circuittofuncig[
            \iltikzfig{strings/strictifiers/expand}[colour=comb,obj=n]
        ]
         & \coloneqq
        (\listvar{v}) \mapsto (\listvar{v}(0), \listvar{v}(1), \dots, \listvar{v}(n-1))
        \\
        \circuittofuncig[
            \iltikzfig{strings/strictifiers/collapse}[colour=comb,obj=n]
        ]
         & \coloneqq
        (v_0, v_1, \dots, v_{n-1}) \mapsto (v_0v_1\dots v_{n-1})
    \end{align*}
\end{definition}

The map from coloured circuits to Mealy machines proceeds in a similar
manner.

\begin{definition}
    Let \(\morph{\circuittomealyig}{\scircsigmag}{\mealyig}\) be the traced PROP
    morphism defined on generators as
    \begin{align*}
        \circuittomealyig[
            \iltikzfig{circuits/components/gates/gate}[colour=comb]
        ]
         & \coloneqq
        (\{s\},
         &             & (\listvar{v}_0,\dots,\listvar{v}_{m-1}) \mapsto
        \left\langle
        s,
        \gateinterpretation[g]\left(
        \listvar{v}_0,\listvar{v}_1,\dots,\listvar{v}_{m-1}
        \right)
        \right\rangle,
         &             & s)
        \\
        \circuittomealyig[
            \iltikzfig{strings/structure/comonoid/copy}[colour=comb,obj=n]
        ]
         & \coloneqq (
        \{s\},
         &             & \listvar{v}
        \mapsto
        \left\langle s, (\listvar{v},\listvar{v})\right\rangle,
         &             & s
        )
        \\
        \circuittomealyig[
            \iltikzfig{strings/structure/monoid/merge}[colour=comb,obj=n]
        ]
         & \coloneqq (
        \{s\},
         &             & (\listvar{v}, \listvar{w}) \mapsto
        \left\langle s, \listvar{v} \ljoin \listvar{w}\right\rangle,
         &             & s
        )
        \\
        \circuittomealyig[
            \iltikzfig{strings/structure/comonoid/discard}[colour=comb,obj=n]
        ]
         & \coloneqq
        (
        \{s\},
         &             & \listvar{v} \mapsto
        \left\langle s, s\right\rangle,
         &             & s
        )
        \\
        \circuittomealyig[
            \iltikzfig{strings/strictifiers/expand}[colour=comb,obj=n]
        ]
         & \coloneqq
        (
        \{s\},
         &             & (v_0,v_1,\dots,v_{n-1}) \mapsto ((v_0), (v_1), \dots, (v_{n-1})),
         &             & s
        )
        \\
        \circuittomealyig[
            \iltikzfig{strings/strictifiers/collapse}[colour=comb,obj=n]
        ]
         & \coloneqq
        (
        \{s\},
         &             & ((v_0), (v_1), \dots, (v_{n-1})) \mapsto (v_0,v_1,\dots,v_{n-1}),
         &             & s
        )
        \\
        \circuittomealyig[
            \iltikzfig{circuits/components/values/vs}[val=\listvar{v},width=n]
        ]
         & \coloneqq
        (
        \{s_{\listvar{v}}, s_\bot\},
         &             & \{
        s_{\listvar{v}} \mapsto \left\langle{s_\bot,\listvar{v}}\right\rangle,
        s_\bot \mapsto \left\langle{s_\bot,\bot}\right\rangle
        \},
         &             & s_{\listvar{v}}
        )
        \\
        \circuittomealyig[
            \iltikzfig{circuits/components/waveforms/delay}[width=n]
        ]
         & \coloneqq
        (
        \{ s_{\listvar{v}} \,|\, \listvar{v} \in \valuetuple{n}\},
         &             & (s_{\listvar{v}}, \listvar{w}) \mapsto
        \left\langle{s_{\listvar{w}},\listvar{v}}\right\rangle,
         &             & s_{\bot^n}
        )
    \end{align*}
\end{definition}

Although morphisms in \(\mealyig\) have different interfaces to those in
\(\streamig\), they are still monotone Mealy machines so it is simple to
translate them into stream functions or coloured circuits.

\begin{definition}
    Let the coloured PROP morphisms
    \(\morph{\mealytostreamig}{\mealyig}{\streamig}\),
    \(\morph{\streamtomealyig}{\streamig}{\mealyig}\) and
    \(\morph{\mealytocircuitig}{\mealyig}{\scircsigmag}\) be defined as before,
    and let \(\morph{\circuittostreamig}{\scircsigmag}{\streamig}\)
    be defined as \(\mealytostreamig \circ \circuittomealyig\).
\end{definition}

By putting all these coloured PROP morphisms together, we can show the same
results as we did in the previous section.

\begin{theorem}
    \(\mealytostreamig = \circuittostreamig \circ \mealytostreamig\) and
    \(\circuittostreamig \circ \mealytocircuitig \circ \streamtomealyig =
    \id[\streamig]\).
\end{theorem}

As before, we derive a notion of denotational equivalence for generalised
circuits.

\begin{definition}
    Two generalised sequential circuits are \emph{denotationally equivalent}
    under \(\interpretation\), written \(
    \iltikzfig{strings/category/f}[box=f,colour=seq,dom=\listvar{m},cod=\listvar{n}]
    \approx_{\interpretation}^+
    \iltikzfig{strings/category/f}[box=g,colour=seq,dom=\listvar{m},cod=\listvar{n}]
    \) if \(
    \circuittostreamig[
        \iltikzfig{strings/category/f}[box=f,colour=seq]
    ]
    =
    \circuittostreamig[
        \iltikzfig{strings/category/f}[box=g,colour=seq]
    ]
    \).
    Let \(\scircsigmaig\) be the result of quotienting \(\scircsigma\) by \(
    \approx_{\interpretation}^+
    \).
\end{definition}

\begin{corollary}
    \(\scircsigmaig \cong \streamig\).
\end{corollary}
