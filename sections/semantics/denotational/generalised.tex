\section{Denotational semantics for generalised circuits}

Although we have discussed the denotation semantics in terms of monochromatic
circuit signatures, it is straightforward to extend the results to categories
generated over \emph{generalised} circuit signatures

In the semantic categories \(\funci\), \(\streami\), and \(\mealyi\), the
morphisms are all variants on functions of the form
\(\valuetuple{m} \to \valuetuple{n}\) that operate on powers of elements in
\(\values\): one element for each (single-bit) input or output wire.
In the generalised setting, these input and output wires may not all be the
same width, so the input and output sets must be \emph{powers of powers} of
values!

\begin{notation}
    Given a set \(A\) and a word \(\listvar{v} \in \natplus^\star\) of
    length \(n\), we write \(
    A^{\listvar{v}}
    \coloneqq
    A^{\listvar{v}(0)}
    \times
    A^{\listvar{v}(1)}
    \times
    \dots
    \times
    A^{\listvar{v}(n-1)}
    \).
\end{notation}

Note that for a word \(\listvar{m} \coloneqq 11\dots1\) of length \(k\), the set
\(A^{\listvar{m}} = A^1 \times A^1 \times \dots A^1\) is isomorphic to \(A^k\),
much like how setting the set of colours in a coloured PROP to the singleton
recovers a monochromatic PROP.

The semantic categories can now be extended to these \emph{coloured} interfaces.

\begin{definition}
    Let \(\funcig\) be the \(\natplus\)-coloured PROP in which the morphisms
    \(\listvar{m} \to \listvar{n}\) are the monotone functions
    \(\valuetuple{\listvar{m}} \to \valuetuple{\listvar{n}}\).
    Let \(\streamig\) be the \(\natplus\)-coloured PROP in which the morphisms
    \(\listvar{m} \to \listvar{n}\) are the causal, monotone, and finitely
    specified stream functions \(
    \valuetuplestream{\listvar{m}} \to \valuetuplestream{\listvar{n}}
    \).
    Let \(\mealyig\) be the \(\natplus\)-coloured PROP in which the moprhisms
    \(\listvar{m} \to \listvar{n}\) are the monotone
    \((\valuetuple{\listvar{m}}, \valuetuple{\listvar{n}})\)-Mealy machines.
\end{definition}


\begin{definition}
    Let the coloured PROP morphisms
    \[\arraycolsep=2em\def\arraystretch{1.2}\begin{array}{cc}
            \morph{\circuittofuncig}{\ccircsigmag}{\funcig}
             &
            \morph{\circuittomealyig}{\scircsigmag}{\mealyig}
            \\
            \morph{\mealytostreamig}{\mealyig}{\streamig}
             &
            \morph{\streamtomealyig}{\streamig}{\mealyig}
            \\
            \morph{\mealytocircuitig}{\mealyig}{\scircsigmag}
             &
            \morph{\circuittostreamig}{\scircsigmag}{\streamig}
        \end{array}\]
    be defined in the same manner as the monochromatic versions.
\end{definition}

\begin{definition}

\end{definition}

\begin{definition}
    Two generalised sequential circuits are \emph{denotationally equivalent}
    under \(\interpretation\), written \(
    \iltikzfig{strings/category/f}[box=f,colour=seq,dom=\listvar{m},cod=\listvar{n}]
    \approx_{\interpretation}^+
    \iltikzfig{strings/category/f}[box=g,colour=seq,dom=\listvar{m},cod=\listvar{n}]
    \) if \(
    \circuittostreamig[
        \iltikzfig{strings/category/f}[box=f,colour=seq]
    ]
    =
    \circuittostreamig[
        \iltikzfig{strings/category/f}[box=g,colour=seq]
    ]
    \).
    Let \(\scircsigmaig\) be the result of quotienting \(\scircsigma\) by \(
    \approx_{\interpretation}^+
    \).
\end{definition}

\begin{corollary}
    \(\scircsigmaig \cong \streamig\).
\end{corollary}
