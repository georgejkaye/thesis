\chapter{Denotational semantics}\label{chap:denotational}

Circuits in \(\scircsigma\) are just \emph{syntax}; they have no
\emph{behaviour} (or \emph{semantics}), assigned to them.
A semantics for digital circuits relates circuits which have `the
same behaviour given some interpretation'.
But there is not just one way to construct such a relation.
In this thesis we will examine three such ways: a \emph{denotational} semantics,
an \emph{operational} semantics and an \emph{algebraic} semantics.
Each approach comes with advantages and drawbacks; skillful use of all three
will lead to a powerful, fully compositional, perspective on sequential
circuits.

Although each semantics relation is constructed differently, they must
relate the \emph{same} circuits; the behaviour of a circuit should not be
different depending on which lens we are viewing it through.
Formally, this means that each semantics is \emph{sound and complete} with
respect to the others; two circuits \(
\iltikzfig{strings/category/f}[box=f,colour=seq,dom=m,cod=n]
\) and \(
\iltikzfig{strings/category/f}[box=g,colour=seq,dom=m,cod=n]
\) are related by one semantics if and only if they are related by the others.

\index{denotational semantics}
First of all we will define the \emph{denotational semantics} of digital
circuits, which will act as the gold standard against which the other
semantics will be compared.
Denotational semantics is the notion of assigning meaning to structures using
values in some \emph{semantic domain}: a partially ordered set with some
extra structure.
The idea is an old one in computer science, going back to the work of Scott and
Strachey~\cite{scott1970outline,scott1971mathematical}.

\begin{example}\label{ex:expressions-denotational}
    Consider a language of mathematical expressions defined as
    follows:
    \[
        n \in N :\coloneqq \overline{0} \,|\, \overline{1} \,|\, \overline{2} \,|\,
        \cdots
        \qquad
        e \in E :\coloneqq add \, e \, e \,|\, mul \, e \, e \,|\,  n
    \]
    To define a denotational semantics for terms \(E\) in this language, we need
    to pick a \emph{semantic domain} for the denotations of terms to belong to.
    An obvious one here is the natural numbers \(\nat\); given a term
    \(e \in E\), we write \(\llbracket{e}\rrbracket\) for its denotation in
    \(\nat\).
    For each \(\overline{n} \in N\), \(\llbracket{\overline{n}}\rrbracket\) is
    the corresponding natural number, and the operations are interpreted as \(
    \llbracket{add \, e_1 \, e_2}\rrbracket
    = \llbracket{e_1}\rrbracket + \llbracket{e_2}\rrbracket
    \) and \(
    \llbracket{mul \, e_1 \, e_2}\rrbracket
    = \llbracket{e_1}\rrbracket \cdot \llbracket{e_2}\rrbracket
    \) respectively.
\end{example}

The above example illustrates quite nicely how a denotational semantics should
be \emph{compositional}; the denotations of a composite term should be
constructed by combining the denotations of its components.

\begin{remark}
    The content of this section is a refinement
    of~\cite[Section 3]{ghica2024fully}.
\end{remark}

\section{Denotational semantics of digital circuits}\label{sec:circuits}

Now that we comprehend what exactly denotational semantics is, we turn to our
goal of defining a denotational semantics for digital circuits.
We will interpret digital circuits as a certain class of
\emph{stream functions}, functions that operate on infinite sequences of values.
This represents how the output of a circuit may not just operate on the current
input, but all of the previous ones as well.

\begin{remark}
    In \cite{mendler2012constructive}, the semantics of digital circuits with
    delays and cycles are presented using \emph{timed ternary simulation}, an
    algorithm to compute how a sequence of circuit outputs stabilises over time
    given the inputs and value of the current state.
    This differs from our approach as we assign each circuit a concrete stream
    function describing its behaviour, rather than having to solve a system of
    equations in terms of the nodes inside a circuit in order to determine its
    behaviour.
\end{remark}

\subsection{Interpreting circuit components}\label{sec:interpreting-components}

Before assigning stream functions to a given circuit in \(\scircsigma\), we will
first decide how to interpret the individual \emph{components} of a given
circuit signature.
This is crucial because a denotational semantics is defined
\emph{compositionally}; eventually we will need to refer to the interpretations
of particular components.

First we consider the interpretation of the \emph{values} that flow through the
wires in our circuits.
In the denotational semantics the set of values will need to have a bit more
structure, as it must be ordered by how much \emph{information} each value
carries.

\begin{definition}[Partially ordered set]
    A \emph{partially ordered set}, or \emph{poset} for short, is a set \(A\)
    equipped with a reflexive, antisymmetric, and transitive relation \(\leq\),
    i.e.\
    \begin{itemize}
        \item for all \(a \in A\), \(a \leq a\);
        \item for all \(a, b \in A\), if \(a \leq b\) and \(b \leq a\) then
              \(a = b\); and
        \item for all \(a, b, c \in A\), if \(a \leq b\) and \(b \leq c\) then
              \(a \leq c\).
    \end{itemize}
    A poset \((A, leq)\) is called a \emph{finite poset} if \(A\) is finite.
\end{definition}

\begin{definition}[Least and greatest elements]
    In a poset \((A, \leq)\), a \emph{least element} is an element \(b \in A\)
    such that for all elements \(y \in A\), \(b \leq y\).
    Similarly, a \emph{greatest element} is an element \(a \in A\)
    such that for all elements \(x \in A\), \(x \leq a\).
\end{definition}

\begin{example}
    The natural numbers \(\nat\) are a poset ordered in the usual way; they have
    a least element \(0\) but not a greatest element.
    However, any finite subset of the natural numbers \emph{does} have a
    maximal element.
\end{example}

Even in a finite set there is no reason that least and greatest elements should
be \emph{unique}.
However, in our context of digital circuits we would like this to be the case:
the unique least element will represent a complete \emph{lack} of information,
and the unique greatest element will represent \emph{every} piece of information
at once.
To enforce the uniqueness of these elements we must add even more structure to
a poset of values.

\begin{definition}[Lower and upper bounds]
    Given a poset \((A, \leq)\), a \emph{lower bound} of a subset
    \(B \subseteq A\) is an element \(x \in B\) such that for all \(b \in B\),
    \(x \leq b\).
    Similarly, an \emph{upper bound} of \(C \subseteq A\) is an element
    \(y \in C\) such that for all \(b \in B\), \(b \leq y\).
\end{definition}

\begin{definition}[Meet and join]
    Given a poset \((A, \leq)\), a lower bound \(x\) of \(B \subseteq A\) is
    called an \emph{meet} (or infimum, or greatest lower bound) if for all lower
    bounds \(b \in B\), \(b \leq x\).
    Similarly, an upper bound \(y\) of \(B\) is called a \emph{join} (or
    supremum, or least upper bound) if for all upper bounds \(c \in B\),
    \(y \leq c\).
\end{definition}

We usually draw the meet and the join operations as rotated versions of the
order operation.
For example, in the above definitions we have used \(\wedge\) and \(\vee\) for
the order \(\leq\); in subsequent sections we will use \(\lmeet\) and
\(\ljoin\) for the order \(\sqsubseteq\).
In general, the join and meet of a pair of elements in a poset need not exist.
We are interested in the structures in which they \emph{always} exist, which are
known as \emph{lattices}.

\begin{definition}[Lattice]
    A \emph{lattice} is a poset \((A, \leq)\) in which each pair of elements
    \(a,b \in A\) has a meet and join.
    A lattice is called a \emph{finite lattice} if \(A\) is finite, and
    \emph{bounded} if it has a minimum element and a maximum element.
\end{definition}

Much like how every non-empty finite poset has at least one maximimal and
minimal element, every non-empty finite lattice has a join and meet.

\begin{notation}
    For a poset \((A, \leq)\), we write \(\bigwedge A\) for the meet of
    all the elements in \(A\) (if it exists) and \(\bigvee A\) for the join of
    all the elements in \(A\) (again, if it exists).
\end{notation}

\begin{lemma}\label{cref:finite-bounded}
    A non-empty finite lattice \((A, \leq)\) is bounded.
\end{lemma}
\begin{proof}
    As each pair of elements in \(A\) has a meet, and as \(A\) is finite, we
    can define the element \(x\) as \(\bigwedge A\).
    This element is the greatest element in \(A\) by definition of the meet:
    \((a_0 \wedge a_1) \leq a_0\) and \((a_0 \wedge a_1) \leq a_1\),
    \(((a_0 \wedge a_1) \wedge a_2) \leq (a_0 \wedge a_1)\) and
    \(((a_0 \wedge a_1) \wedge a_2) \leq a_2\), and so on.
    The same proof holds in reverse for the join and the greatest element by
    using the join \(\bigvee A\).
\end{proof}

\begin{example}\label{ex:powerset-lattice}
    Let \(A = \{0,1,2\}\), and let \((\mcp A, \subseteq)\) be the poset defined
    as the powerset of \(A\) ordered by subset inclusion.
    This poset can be illustrated by the following \emph{Hasse diagram}, in
    which a line going up from \(a\) to \(b\) indicates that \(a \leq b\).

    \begin{center}
        \begin{tikzcd}
            & \{0,1,2\} & \\
            \{0,1\} \arrow[dash]{ur} &
            \{0,2\} \arrow[dash]{u} &
            \{1,2\} \arrow[dash]{ul} \\
            \{0\} \arrow[dash]{u} \arrow[dash]{ur} &
            \{1\} \arrow[dash]{ul} \arrow[dash]{ur} &
            \{2\} \arrow[dash]{ul} \arrow[dash]{u} \\
            & \{\} \arrow[dash]{ul} \arrow[dash]{u} \arrow[dash]{ur} &
        \end{tikzcd}
    \end{center}

    The diagram clearly illustrates the lattice structure on this poset: the
    join is union and the meet is intersection.
    Subsequently the greatest element \(\top\) is the set \(A\) and the
    least element \(\bot\) is the empty set \(\{\}\).
\end{example}

\begin{remark}
    If \(A\) is a lattice, then for any \(n \in \nat\), \(A^n\) is also a
    lattice by comparing elements pointwise.
    The \(\bot\) is then the word containing \(n\) copies of the \(\bot\)
    element in \(A\), and similarly for the \(\top\) element.

    Recall the set \(A \coloneqq \{0,1,2\}\) from
    \cref{ex:powerset-lattice} and the lattice structure on its powerset
    \(\mathcal{P}A\).
    This induces an ordering on \((\mathcal{P}A)^2\):
    \(\{0,1\}\{1\} \leq \{0,1,2\}\{1,2\}\) because \(\{0,1\} \leq \{0,1,2\}\)
    and \(\{1\} \leq \{1,2\}\), and conversely
    \(\{0,1\}\{1\} \not\leq \{0\}\{1,2\}\) because \(\{0,1\} \not\leq \{0\}\).
    The join and meet are computed by taking the join and meet of each
    component: \(
    \{0,1\}\{1\} \vee \{0,1,2\}\{1,2\} = \{0,1,2\}\{1,2\}
    \) and \(
    \{0,1\}\{1\} \vee \{0\}\{1,2\} = \{0,1\}\{1,2\}
    \).
\end{remark}

Values in a circuit signature are interpreted as elements of a finite
lattice; now the concepts of no information and all information are encoded
as the \(\bot\) and \(\top\) element.
Now the primitives in the signature must be interpreted.
Clearly they should be interpreted as functions between the values, but these
functions must respect the order on the value lattice; one should not be able to
lose information by performing a computation.

\begin{definition}
    Let \((A, \leq_A)\) and \((B, \leq_B)\) be partial orders.
    A function \(\morph{f}{A}{B}\) is \emph{monotone} if, for every \(x, y \in A\),
    \(x \leq_A y\) if and only if \(f(x) \leq_B f(y)\).
\end{definition}

Another condition on the primitives is that when all the inputs to a primitive
are \(\bot\), then it should produce \(\bot\); we cannot produce information
from nothing.

\begin{definition}
    Let \(A,B\) be finite lattices, where \(\bot_A\) is the least element of
    \(A\) and \(\bot_B\) is the least element of \(B\).
    A function \(\morph{f}{A}{B}\) is \emph{\(\bot\)-preserving} if
    \(f(\bot_A) = \bot_B\).
\end{definition}

Assigning interpretations to the combinational components of a circuit sets the
stage for the entire denotational semantics.

\begin{definition}[Interpretation]
    For a signature \(
    \signature = (
    \values, \bullet, \circuitsignaturegates, \circuitsignaturearity,
    \circuitsignaturecoarity
    )\), an \emph{interpretation} of
    \(\signature\) is a tuple \((\sqsubseteq, \gateinterpretation)\) where
    \((\values, \sqsubseteq)\) is a lattice with \(\bullet\) as the least
    element, and \(\gateinterpretation\) maps each
    \(p \in \circuitsignaturegates\) to a \(\bot\)-preserving monotone function
    \(
    \valuetuple{\circuitsignaturearity(p)}
    \to
    \valuetuple{\circuitsignaturecoarity(p)}
    \).
\end{definition}

\begin{example}\label{ex:belnap-interpretation}
    Recall the Belnap signature \(
    \belnapsignature = (
    \belnapvalues, \bot, \belnapgates, \belnaparity, \belnapcoarity
    )
    \) from \cref{ex:belnap-signature}.
    We assign a partial order \(\leq_\mathsf{B}\) to values in
    \(\belnapvalues\) as follows:

    \begin{center}
        \begin{tikzcd}
            & \top & \\
            \belnapfalse \arrow[dash]{ur} & & \belnaptrue \arrow[dash]{ul} \\
            & \bot \arrow[dash]{ul} \arrow[dash]{ur} &
        \end{tikzcd}
    \end{center}

    The gate interpretation \(\belnapgateinterpretation\) has action \(
    \andgate \mapsto \land, \orgate \mapsto \lor, \notgate \mapsto \neg
    \) where \(\land\), \(\lor\) and \(\neg\) are defined by the following
    truth tables~\cite{belnap1977useful}:

    \begin{center}
        \begin{tabular}{|c|cccc|}
            \hline
            \(\land\)        & \(\bot\)         & \(\belnapfalse\) & \(\belnaptrue\)  & \(\top\)         \\
            \hline
            \(\bot\)         & \(\bot\)         & \(\belnapfalse\) & \(\bot\)         & \(\belnapfalse\) \\
            \(\belnapfalse\) & \(\belnapfalse\) & \(\belnapfalse\) & \(\belnapfalse\) & \(\belnapfalse\) \\
            \(\belnaptrue\)  & \(\bot\)         & \(\belnapfalse\) & \(\belnaptrue\)  & \(\top\)         \\
            \(\top\)         & \(\belnapfalse\) & \(\belnapfalse\) & \(\top\)         & \(\top\)         \\
            \hline
        \end{tabular}
        \quad
        \begin{tabular}{|c|cccc|}
            \hline
            \(\lor\)         & \(\bot\)        & \(\belnapfalse\) & \(\belnaptrue\) & \(\top\)        \\
            \hline
            \(\bot\)         & \(\bot\)        & \(\bot\)         & \(\belnaptrue\) & \(\belnaptrue\) \\
            \(\belnapfalse\) & \(\bot\)        & \(\belnapfalse\) & \(\belnaptrue\) & \(\top\)        \\
            \(\belnaptrue\)  & \(\belnaptrue\) & \(\belnaptrue\)  & \(\belnaptrue\) & \(\belnaptrue\) \\
            \(\top\)         & \(\belnaptrue\) & \(\top\)         & \(\belnaptrue\) & \(\top\)        \\
            \hline
        \end{tabular}
        \quad
        \begin{tabular}{|c|c|}
            \hline
            \(\neg\)         &                  \\
            \hline
            \(\bot\)         & \(\bot\)         \\
            \(\belnaptrue\)  & \(\belnapfalse\) \\
            \(\belnapfalse\) & \(\belnaptrue\)  \\
            \(\top\)         & \(\top\)         \\
            \hline
        \end{tabular}
    \end{center}

    The Belnap interpretation is then \(
    (\leq_\mathsf{B}, \belnapgateinterpretation)
    \).
    An online tool for experimenting with the Belnap interpretation can be found
    at \url{https://belnap.georgejkaye.com}.
\end{example}
\subsection{Denotational semantics of combinational circuits}

The semantic domain for \emph{combinational} circuits is straightforward: each
circuit maps to a monotone function.

\begin{definition}
    Let \(\funci\) be the PROP in which the morphisms
    \(m \to n\) are the monotone \(\bot\)-preserving
    functions \(\valuetuple{m} \to \valuetuple{n}\).
\end{definition}

The natural way to map from a PROP of syntax into a PROP of semantics is to
use a functor.

\begin{definition}
    A \emph{PROP morphism} is a strict symmetric monoidal functor between two
    PROPs i.e.\ a functor that preserves the strict symmetric monoidal
    structure.
\end{definition}

\begin{definition}
    Let \(\morph{\circuittofunci}{\ccircsigma}{\funci}\) be the PROP morphism
    with action defined as%
    \vspace{-\abovedisplayskip}
    % \vspace{-\parskip}
    \begin{center}
        \begin{minipage}{0.32\textwidth}
            \centering
            \begin{align*}
                \circuittofunci[
                    \iltikzfig{strings/structure/comonoid/copy}[colour=comb]
                ]
                 & \coloneqq
                (v) \mapsto (v, v)
                \\
                \circuittofunci[
                    \iltikzfig{strings/structure/monoid/merge}[colour=comb]
                ]
                 & \coloneqq
                (v, w) \mapsto v \sqcup w
            \end{align*}
        \end{minipage}
        \quad
        \begin{minipage}{0.25\textwidth}
            \centering
            \begin{align*}
                \circuittofunci[
                    \iltikzfig{strings/structure/comonoid/discard}[colour=comb]
                ]
                 & \coloneqq
                (v) \mapsto ()
                \\
                \circuittofunci[
                    \iltikzfig{strings/structure/monoid/init}[colour=comb]
                ]
                 & \coloneqq
                () \mapsto \bot
            \end{align*}
        \end{minipage}
        \quad
        \begin{minipage}{0.25\textwidth}
            \centering
            \vspace{1.5em}
            \(\circuittofunci[
                \iltikzfig{circuits/components/gates/gate}[gate=p,dom=m,cod=n]
            ]
            \coloneqq
            \gateinterpretation[p]
            \)
        \end{minipage}
    \end{center}
\end{definition}

\begin{remark}
    One might wonder why the fork and the join have different semantics, as they
    would be physically realised by the same wiring.
    This is because digital circuits have a notion of \emph{static causality}:
    outputs can only connect to inputs.
    This is why the semantics of combinational circuits is \emph{functions} and
    not \emph{relations}.

    In real life one could force together two digital devices, but this might
    lead to undefined behaviour in the digital realm.
    This is reflected in the semantics by the use of the join; for example, in
    the Belnap interpretation if one tries to join together \(\belnaptrue\) and
    \(\belnapfalse\) then the overspecified \(\top\) value is produced.
\end{remark}

\subsection{Denotational semantics of sequential circuits}

As one might expect, sequential circuits are slightly trickier to deal with.
In a combinational circuit, the output only depends on the inputs at the current
cycle, but for sequential circuits inputs can affect outputs many cycles after
they occur.

We therefore have to reason with \emph{infinite sequences} of inputs rather than
individual values; these are known as \emph{streams}.

\begin{notation}
    Given a set \(A\), we denote the set of streams (infinite sequences) of
    \(A\) by \(\stream{A}\).
    As a stream can equivalently be viewed as a function \(\nat \to A\), we
    write \(\sigma(i) \in A\) for the \(i\)-th element of stream
    \(\sigma \in \stream{A}\).
    Individual streams are written as \(
    \sigma \in \stream{A}
    \coloneqq
    \sigma(0) \streamcons \sigma(1) \streamcons
    \sigma(2) \streamcons \cdots
    \).
\end{notation}

Streams can be viewed a bit like lists, in that they have a head component and
an (infinite) tail component.

\begin{definition}[Operations on streams]\label{def:stream-operations}
    Given a stream \(\sigma \in \stream{A}\), its \emph{initial value}
    \(\streaminit(\sigma) \in A\) is defined as \(\sigma \mapsto \sigma(0)\)
    and its \emph{stream derivative} \(\streamderv(\sigma) \in \stream{A}\) is
    defined as \(\sigma \mapsto (i \mapsto \sigma(i+1))\).
\end{definition}

\begin{notation}
    For a stream \(\sigma\) with initial value \(a\) and stream derivative
    \(\tau\) we write it as \(\sigma \coloneqq a \streamcons \tau\).
\end{notation}

Streams will serve as the inputs and outputs to circuits, so the denotations of
sequential circuits will be \emph{stream functions}, which consume and produce
streams.
Just like with functions, we cannot claim that all streams are the
denotations of sequential circuits.

\begin{definition}[Causal stream function~\cite{rutten2006algebraic}]
    A stream function \(\morph{f}{\stream{A}}{\stream{B}}\) is \emph{causal} if,
    for all \(i \in \nat\) and \(\sigma,\tau \in \stream{A}\) we have that
    \(\sigma(j) = \tau(j)\) for all \(j \leq i\), then
    \(f(\sigma)(i) = f(\tau)(i)\).
\end{definition}

Causality is a form of continuity; a causal stream function is a stream function
in which the \(i\)-th element of its output stream only depends on elements
\(0\) through \(i\) inclusive of the input stream; it cannot look into the
future.
A neat consequence of causality is that it enables us to lift the stream
operations of initial value and stream derivative to stream \emph{functions}.

\begin{definition}[Initial output~\cite{rutten2006algebraic}]
    For a causal stream function \(\morph{f}{\stream{A}}{\stream{B}}\) and
    \(a \in A\), the \emph{initial output of \(f\) on input \(a\)} is an element
    \(\initialoutput{f}{a} \in A\) defined as
    \(\initialoutput{f}{a} \coloneqq \streaminit(f(a \streamcons \sigma))\) for
    an arbitrary \(\sigma \in \stream{A}\).
\end{definition}

Since \(f\) is causal, the stream \(\sigma\) in the definition of initial
output truly is arbitrary; the \(\streaminit\) function only depends on the
first element of the stream.

\begin{definition}[Functional stream derivative~\cite{rutten2006algebraic}]
    For a stream function \(\morph{f}{\stream{A}}{\stream{B}}\) and
    \(a \in A\), the
    \emph{functional stream derivative of \(f\) on input \(a\)} is a stream
    function \(\streamderivative{f}{a}\) defined as \(
    \streamderivative{f}{a}
    \coloneqq
    \sigma \mapsto \streamderv(f(a \streamcons \sigma))
    \).
\end{definition}

The functional stream derivative \(\streamderivative{f}{a}\) is a new stream
function which acts as \(f\) would `had it seen \(a\) first'.

\begin{remark}
    One intuitive way to view stream functions is to think of them as the states
    of some Mealy machine; the initial output is the output given some input,
    and the functional stream derivative is the transition to a new state.
    As with most things in mathematics, this is no coincidence; there is a
    homomorphism from any Mealy machine to a stream function.
    We will exploit this fact in the next section.
\end{remark}

This leads us to the next property of denotations of sequential circuits.
Although they may have infinitely many inputs and outputs, circuits themselves
are built from a finite number of components.
This means they cannot specify infinite \emph{behaviour}.

\begin{notation}
    Given a finite word \(\listvar{a} \in \freemon{A}\), we abuse notation
    and write \(\streamderivative{f}{\listvar{a}}\) for the repeated
    application of the functional stream derivative for the elements of
    \(\listvar{a}\), i.e.\ \(
    \streamderivative{f}{\varepsilon} \coloneqq f
    \) and \(
    \streamderivative{f}{a \streamcons \listvar{b}} \coloneqq
    \streamderivative{(\streamderivative{f}{a})}{\listvar{b}}
    \).
\end{notation}

\begin{definition}
    Given a stream function \(\morph{f}{\stream{A}}{\stream{B}}\), we say it is
    \emph{finitely specified} if the set \(\{
    \streamderivative{f}{\listvar{a}} \,|\, \listvar{a} \in \freemon{A}
    \}\) is finite.
\end{definition}

As the components of our circuits are monotone and \(\bot\)-preserving, a
denotational semantics for circuits must also be monotone and
\(\bot\)-preserving.
This means we need to lift the order on values to an order on streams.

\begin{notation}
    For a poset \((A, \leq_A)\) and streams \(\sigma,\tau \in \stream{A}\), we
    say that \(\sigma \leq_{\stream{A}} \tau\) if \(\sigma(i) \leq_A \tau(i)\)
    for all \(i \in \nat\).
\end{notation}

For these three properties to be suitable as a denotational semantics for
sequential circuits, we must show that stream functions with these
properties form a category we can map into from \(\scircsigma\).
We will first show that these categories form a symmetric monoidal category, so
we need a suitable candidate for composition and tensor.
There are fairly obvious choices here: for the former we use regular function
composition and for the latter we use the Cartesian product.

\begin{lemma}\label{lem:causality-preserved}
    Causality is preserved by composition and Cartesian product.
\end{lemma}
\begin{proof}
    If the \(i\)-th element of two stream functions \(f\) and \(g\) only depends
    on the first \(i+1\) elements of the input, then so will their composition
    and product.
\end{proof}

\begin{lemma}\label{lem:finitely-specified-preserved}
    Finite specification is preserved by composition and Cartesian
    product.
\end{lemma}
\begin{proof}
    For both the composition and product of two stream functions \(f\) and
    \(g\), the largest the set of stream derivatives could be is the product of
    stream derivatives of \(f\) and \(g\), so this will also be finite.
\end{proof}

\begin{lemma}\label{lem:monotonicity-preserved}
    \(\bot\)-preserving monotonicity is preserved by composition and Cartesian
    product.
\end{lemma}
\begin{proof}
    The composition and product of any monotone function is monotone, and must
    preserve the \(\bot\) element.
\end{proof}

As the categorical operations preserve the desired properties, these stream
functions form a PROP.

\begin{proposition}
    There is a PROP \(\streami\) in which the morphisms \(m \to n\) are the
    causal, finitely specified and \(\bot\)-preserving monotone stream
    functions \(\valuetuplestream{m} \to \valuetuplestream{n}\).
\end{proposition}
\begin{proof}
    Identity is the identity function, the symmetry swaps streams, composition
    is composition of functions, and tensor product on morphisms
    \(\morph{f}{\valuetuplestream{m}}{\valuetuplestream{n}}\) and
    \(\morph{g}{\valuetuplestream{p}}{\valuetuplestream{q}}\) is the Cartesian
    product of functions composed with the components of the isomorphism
    \(\valuetuplestream{m} \times \valuetuplestream{n}
    \cong \valuetuplestream{m+n}\).

    As these constructs satisfy the categorical axioms, and as function
    composition and Cartesian product preserve causality
    (\cref{lem:causality-preserved}),
    finite specification (\cref{lem:finitely-specified-preserved}),
    and monotonicity (\cref{lem:monotonicity-preserved}), this data defines a
    symmetric monoidal category.
\end{proof}

Modelling sequential circuits as stream functions deals with temporal
aspects, but what about feedback?
As the assignment of denotations needs to be compositional, we need
to map the trace on \(\scircsigma\) to a trace on \(\streami\).
A usual candidate for the trace when considering partially ordered settings is
the \emph{least fixed point}.

\begin{definition}[Least fixed point]
    For a poset \((A, \leq)\) and function \(\morph{f}{A}{A}\), the least
    fixed point of \(f\) is a value \(\mu_f\) such that \(f(\mu_f) = f\) and,
    for all values \(v\) such that \(f(v) = v\), \(\mu_f \leq v\).
\end{definition}

Least fixed points are ubiquitious in program semantics, where they are often
used to model \emph{recursion}; since feedback is a form of recursion it seems
apt that we should also follow this route.
As fixed points are so important, they are the subject of many theorems; one
that will come in very useful for us is the \emph{Kleene fixed-point theorem},
which is concerned with a special class of monotone functions.

\begin{definition}[Directed subset]
    For a poset \((A, \leq)\), a non-empty subset \(B \subseteq A\) is called a
    \emph{directed subset} if every pair of elements in \(B\) has an upper bound
    in \(B\).
    If this set has a join \(\bigvee B\) then this element is called a
    \emph{directed join}.
\end{definition}

\begin{notation}[Image]
    For a function \(\morph{f}{A}{B}\) and subset \(C \subseteq A\), we write
    \(f[C] \subseteq B\) for the \emph{image} of \(C\) under \(f\).
\end{notation}

\begin{definition}[Scott continuity]
    Given two posets \((A, \leq_A)\) and \((B, \leq_B)\), a function
    \(\morph{f}{A}{B}\) is \emph{Scott-continuous} if for every directed
    subset \(C \subseteq A\) it holds that \(f(\bigvee_A C) = \bigvee_B(f[C])\)
    i.e.\ \(f\) preserves directed joins.
\end{definition}

\begin{theorem}[Kleene fixed-point theorem~\cite{tarski1955latticetheoretical}]
    Let \((A, \leq)\) be a poset in which each of its directed subsets has a
    join, and let \(\morph{f}{L}{L}\) be a Scott-continuous function.
    Then \(f\) has a least fixed point, defined as \(
    \bot \vee f(\bot) \vee f(f(\bot)) \vee \cdots
    \).
\end{theorem}

\begin{remark}
    The Kleene fixed-point theorem was not actually proved by Kleene, but is
    only named after him!
    The result is often instead attributed to Tarski.
\end{remark}

As we have so far only considered monotone functions, it is useful to get some
intuition for what Scott-continuity brings to the table.

\begin{example}\label{ex:directed-subsets}
    An example of a directed subset of \(\belnapvalues^\omega\) is the set \(T\)
    defined as \(\{\belnaptrue^n\bot \,|\, n \in \nat\}\); the join of this set
    is \(\belnaptrue^\omega\). One Scott-continuous function \(
    \belnapvalues^\omega \to \belnapvalues^\omega
    \) is defined as \(f(\sigma)(0) = \bot\) and
    \(f(\sigma)(i+1) = f(sigma)(i)\); this is Scott-continuous because finding
    the join of a set and then prepending it with \(\bot\) is the same as
    prepending \(\bot\) to each stream in the set and then finding their join.

    An example of a stream function that is monotone but \emph{not}
    Scott-continuous is the function defined as \(
    g(\belnaptrue^n\bot^\omega) \coloneqq \belnapfalse^\omega (n )
    \) and \(g(\belnaptrue^\omega) \coloneqq \top^\omega\) (the other inputs
    do not matter for this example).
    We can show this is not Scott-continuous by considering the set \(T\) above,
    as \(f(\bigsqcup T) = f(\belnaptrue^\omega) = \top^\omega \neq
    \belnapfalse^\omega = \bigsqcup f[T]
    \).
    However, note that this function is \emph{not} causal: as
    \(\belnaptrue^\omega\) has the same prefix as every element in \(T\),
    \(f(\belnaptrue^\omega)\) must also share output prefixes.
\end{example}

So far we have not explicitly enforced Scott-continuity on stream functions; it
turns that it is implied by causality and monotonicity.

\begin{proposition}\label{prop:monotone-causal-scott}
    Let \((A, \leq_A)\) and \((B, \leq_B)\) be finite lattices, and let
    \((\stream{A}, \leq_{\stream{A}})\) and \((\stream{B}, \leq_{\stream{B}})\)
    be the induced lattices on streams.
    Then a causal and monotone function \(\stream{A} \to \stream{B}\) must also
    be Scott-continuous.
\end{proposition}
\begin{proof}
    Consider a directed subset \(D \subseteq \stream{A}\); we need to show that
    for an arbitrary causal, monotone and finitely specified function \(f\) we
    have that \(f\left(\bigvee D\right) = \bigvee f[D]\).

    First consider the case when there is a greatest element in \(D\).
    In this case, \(\bigvee D\) must be the greatest element, and as such
    \(\bigvee D \in D\).
    As \(f\) is monotone then \(f(\bigvee D)\) must be the greatest element in
    \(f[D]\); subsequently, \(f\left(\bigvee D\right) = \bigvee f[D]\).

    Now consider the case where there is no greatest element in \(D\) and
    subsequently \(\bigvee D \not\in D\); if there is no greatest element,
    \(D\) must be infinite.
    Even though it is infinite, \(D\) is a directed subset so each pair of
    elements must have an upper bound, and as \(\leq_{\stream{A}}\) is computed
    pointwise by using \(\leq_A\) we can consider what the upper bounds are
    pointwise too.
    Because \(A\) is finite, there cannot be an infinite chain of upper bounds
    for each element \(i\); there must exist an element \(a_i \in A\) such that
    \(D\) contains infinitely many streams \(\sigma\) such that
    \(\sigma(i) = a_i\).
    This means that \(\left(\bigvee D\right)(i) = a\), so every prefix of
    \(\bigvee D\) must exist as a prefix of a stream in \(D\).
    As \(f\) is causal, for each prefix
    \(f\left(\bigvee D\right)\) there must also exist a \(d \in D\) such that
    \(f(d)\) has that prefix, and as such
    \(\bigvee f[D] = f\left(\bigvee D\right)\).
\end{proof}

This means we can use the Kleene fixed point theorem as a tool to show that the
least fixed point is a trace on \(\streami\).

\begin{lemma}\label{lem:lfp-stream-function}
    Given a morphism \(
    \morph{f}{\valuetuplestream{x+m}}{\valuetuplestream{x+n}}
    \in \streami
    \), and stream \(\sigma \in \valuetuplestream{m}\), the function \(
    \tau \mapsto \proj{0}\mleft(f(\tau,\sigma)\mright)
    \) has a least fixed point.
\end{lemma}
\begin{proof}
    The function \(\tau \mapsto \proj{0}\mleft(f(\tau,\sigma)\mright)\) is
    causal and monotone because \(f\) and the projection function are
    causal and monotone, so it is Scott-continuous by
    \cref{prop:monotone-causal-scott}.
    By the Kleene fixed point theorem, this function has a least fixed point,
    defined as \(
    \proj{0}\mleft(f(\bot^\omega, \sigma)\mright) \ljoin
    \proj{0}\mleft(f(\proj{0}\mleft(f(\bot^\omega, \sigma)\mright), \sigma)\mright) \ljoin
    \cdots
    \).
\end{proof}

We must show that this notion of least fixed point is a trace on \(\streami\).
The first step is to show that taking the least fixed point of a stream function
in \(\streami\) produces another causal, finitely specified,
\(\bot\)-preserving, and monotone stream function.
The original proof idea for this is due to David Sprunger, and relies on an
ordering on stream functions themselves.

\begin{definition}\label{def:state-order}
    Let \(A\) and \(B\) be posets and let
    \(\morph{f, g}{\stream{A}}{\stream{B}}\) be stream functions.
    We say \(f \stateorder g\) if \(f(\sigma) \leq_{\stream{B}} g(\sigma)\)
    for all \(\sigma \in \stream{A}\).
\end{definition}

\begin{theorem}\label{thm:trace-well-defined}
    For a function \(\morph{f}{\valuetuplestream{x+m}}{\valuetuplestream{x+n}}\),
    let \(\mu_f(\sigma)\) be the least fixed point of the function \(
    \tau \mapsto \proj{0}\mleft(f(\tau,\sigma)\mright)
    \).
    Then, the stream function \(
    \sigma \mapsto \proj{1}\mleft(f(\mu_f(\sigma), \sigma)\mright)
    \) is in \(\streami\).
\end{theorem}
\begin{proof}
    To show this, we need to prove that
    \(\sigma \mapsto \proj{1}\mleft(f(\mu_f(\sigma)\sigma)\mright)\) is in
    \(\streami\): it is causal, finitely specified, \(\bot\)-preserving and
    monotone.

    Since \(
    \morph{f}{\valuetuplestream{x+m}}{\valuetuplestream{x+n}}
    \) is a morphism of \(\streami\), it has finitely many stream derivatives.
    For each stream derivative \(\streamderivative{f}{\,\listvar{w}}\), let the
    function \(
    \morph{
        \widehat{\streamderivative{f}{\listvar{w}}}
    }{
        \valuetuplestream{x+m}
    }{
        \valuetuplestream{x}
    }
    \) be defined as \(
    \tau\sigma
    \mapsto
    \proj{0}(\streamderivative{f}{\listvar{w}}(\tau\sigma))
    \).
    Note that each of these functions are causal, \(\bot\)-preserving, and
    monotone, because they are constructed from pieces that are causal
    \(\bot\)-preserving and monotone.

    In particular, \(\mu_f(\sigma)\) is the least fixed point of
    \(\widehat{f_\varepsilon}\left((-)\sigma\right)\).
    Using the Kleene fixed point theorem, the least fixed point of
    \(\widehat{f}((-)\sigma)\) can be obtained by composing
    \(\widehat{f}((-)\sigma)\) repeatedly with itself.
    This means that \(
    \mu_f(\sigma)
    =
    \bigsqcup_{k \in \nat} \widehat{f^k}(\bot^\omega,\sigma)
    \) where \(\widehat{f^k}\) is the \(k\)-fold composition of \(f(-,\sigma)\)
    with itself, i.e.\ \(\widehat{f^0}(\tau\sigma) = \tau\) and \(
    \widehat{f^{k+1}}(\tau\sigma)
    =
    \widehat{f}\left(\left(\widehat{f^{k}}(\sigma, \tau)\right)\sigma\right)
    \).
    That the mapping \(\mu_f\) is causal and monotone is
    straightforward: each of the functions in the join is causal and monotone,
    and join preserves these properties.
    It remains to show that this mapping has finitely many stream derivatives.

    When equipped with \(\stateorder\), the set of functions
    \(\valuetuplestream{x+m} \to \valuetuplestream{x}\)
    is a poset, of which
    \(\{\widehat{f_w} \,|\, w \in (\valuetuple{x+m})^\star\}\)
    is a finite subset.
    Restricting the ordering \(\stateorder\) to this set yields a finite poset.
    Since this poset is finite, the set of strictly increasing sequences in this
    poset is also finite.
    We will now demonstrate a relationship between these sequences and stream
    derivatives of \(\mu_f\).

    Suppose \(
    S = \widehat{f_{\,\listvar{w_0}}} \prec \widehat{f_{\,\listvar{w_1}}} \prec
    \cdots \prec \widehat{f_{\,\listvar{w_{\ell-1}}}}
    \) is a strictly increasing sequence of length \(\ell\) in the set of stream
    functions \(
    \{\widehat{f_w} \,|\, w \in (\valuetuple{x+m})^\star\}
    \).
    We define a function \(
    \morph{g_S}{
        \valuetuplestream{m}
    }{
        \valuetuplestream{x}
    }
    \) as \(
    (\sigma) \mapsto \bigsqcup_{k \in \nat} g_k(\sigma)
    \) where \[
        g_k(\sigma) =
        \begin{cases}
            \bot^\omega                                                              & \text{ if } k = 0              \\
            \widehat{f_{\,\listvar{w_k}}}(\left(g_{k-1}(\sigma)\right)\sigma)        & \text{ if } 1 \leq k \leq \ell \\
            \widehat{f_{\,\listvar{w_{\ell-1}}}}(\left(g_{k-1}(\sigma)\right)\sigma) & \text{ if } \ell < k
        \end{cases}.
    \]
    Let the set \(G \coloneqq \{
    g_S \,|\, S \text{ is a strictly increasing sequence}
    \}\).
    When \(S\) is set to the one-item sequence \(\widehat{f}\), \(g_S\) is
    \(\mu_f\), so \(\mu_f \in G\).
    As \(G\) is finite, this means that if \(G\) is closed under stream
    derivative, \(\mu_f\) has finitely many stream derivatives.
    Any element of \(G\) is either \(\bot^\omega\) or has the form \(
    \sigma
    \mapsto
    \widehat{\streamderivative{f}{\,\listvar{w}}}(g_k(\sigma)\sigma)
    \) for some \(\sigma \in \valuetuplestream{m}\) and
    \(k > 0\).
    As \(\bot^\omega\) is its own stream derivative, we need to show that
    applying stream derivative to the latter produces another element of \(G\).
    \begin{align*}
        \streamderivative{\sigma \mapsto \left(\widehat{\streamderivative{f}{\, \listvar{w}}}(\left(g_{k-1}(\sigma)\right)\sigma)\right)}{ab}
         & = \sigma \mapsto \streamderv\left(\widehat{\streamderivative{f}{\, \listvar{w}}}(ab \streamcons \left(g_{k-1}(\sigma)\right)\sigma)\right)              \\
         & = \sigma \mapsto \streamderv\left(\proj{0}\mleft(\streamderivative{f}{\, \listvar{w}}(ab \streamcons \left(g_{k-1}(\sigma)\right)\sigma)\mright)\right) \\
         & = \sigma \mapsto \proj{0}\mleft(\streamderv\left(\streamderivative{f}{\, \listvar{w}}(ab \streamcons \left(g_{k-1}(\sigma)\right)\sigma)\right)\mright) \\
         & = \sigma \mapsto \proj{0}\mleft(\streamderivative{\left(\streamderivative{f}{\, \listvar{w}}(\left(g_{k-1}(\sigma)\right)\sigma)\right)}{ab}\mright)    \\
         & = \sigma \mapsto \proj{0}\mleft(\streamderivative{f}{\, ab \streamcons \listvar{w}}(\left(g_{k-1}(\sigma)\right)\sigma)\mright)                         \\
         & = \sigma \mapsto \widehat{\streamderivative{f}{\, ab \streamcons \listvar{w}}}(\left(g_{k-1}(\sigma)\right)\sigma)
    \end{align*}
    As \(\proj{0}\mleft(\streamderivative{f}{ab \streamcons \listvar{w}}\mright)\)
    is in \(G\), the latter  is closed under stream derivative.
    Subsequently, \(\mu_f\) has finitely many stream derivatives.

    This means that all the components of
    \(\sigma \mapsto \proj{1}(f(\mu_f(\sigma)\sigma))\) are causal, monotone and
    finitely specified, and as these properties are preserved by composition,
    the composite must also have them, so
    \(\sigma \mapsto \proj{1}(f(\mu_f(\sigma)\sigma))\) is in \(\streami\).
\end{proof}

Even if \(\streami\) is closed under least fixed point, this does not mean that
it is a valid trace.
To verify this we must establish that the categorical axioms of the trace hold.

\begin{theorem}
    A trace on \(\streami\) is given for a function \(
    \morph{f}{\valuetuplestream{x+m}}{\valuetuplestream{x+n}}
    \) by the stream function \(
    \sigma \mapsto \proj{1}(f(\mu_f(\sigma), \sigma))
    \), where \(\mu_f(\sigma)\) is the least fixed point of the function \(
    \tau \mapsto \proj{0}\mleft(f(\tau,\sigma)\mright)
    \) for fixed \(\sigma\).
\end{theorem}
\begin{proof}
    By \cref{thm:trace-well-defined}, \(\streami\) is closed under taking the
    least fixed point, so we just need to show that the axioms of STMCs hold.
    Most of these follow in a straightforward way; the only interesting one is
    the sliding axiom.
    We need to show that for stream functions \(
    \morph{f}{\valuetuplestream{x+m}}{\valuetuplestream{y+n}}
    \) and \(
    \morph{g}{\valuetuplestream{y}}{\valuetuplestream{x}}
    \), we have that \(
    \trace{y}{(\tau, \sigma) \mapsto f(g(\tau), \sigma)}
    =
    \trace{x}{
        (\tau, \sigma)
        \mapsto
        g(\proj{0}\mleft(f(\tau, \sigma)\mright),
        \proj{1}\mleft(f(\tau, \sigma)\mright))
    }
    \).

    Let \(l \coloneqq (\tau, \sigma) \mapsto f(g(\tau), \sigma)\) and
    \(r \coloneqq (\tau, \sigma)
    \mapsto
    g(\proj{0}\mleft(f(\tau, \sigma)\mright),
    \proj{1}\mleft(f(\tau, \sigma)\mright))\); we must apply the candidate
    trace construction to both of these and check they are equal.
    For \(l\), the least fixed point of \(
    \tau \mapsto \proj{0}\mleft(f(g(\tau), \sigma)\mright)
    \) is \[
        \mu_l(\sigma) =
        \proj{0}\mleft(f(g(\bot^\omega), \sigma)\mright) \ljoin
        \proj{0}\mleft(f(g(\proj{0}\mleft(f(g(\bot^\omega), \sigma))\mright), \sigma)\mright) \ljoin
        \cdots.\]
    Plugging this into the candidate trace construction we have that \begin{align*}
         &
        \sigma \mapsto \proj{1}\mleft(f(g(\mu_l^l(\sigma)), \sigma)\mright)
        \\
         & \qquad=
        \sigma \mapsto \proj{1}\mleft(f(g(\proj{0}\mleft(f(g(\dots f(g(\proj{0}\mleft(f(g(\bot^\omega), \sigma)\mright)), \sigma)))\mright)), \sigma)\mright)
    \end{align*}
    For the right-hand side, the least fixed point of \(
    \tau \mapsto g(\proj{0}\mleft(f(\tau, \sigma)\mright))
    \) is \[
        \mu_r(\sigma) =
        g(\proj{0}\mleft(f(\bot^\omega, \sigma)\mright)) \ljoin
        g(\proj{0}\mleft(f(g(\proj{0}\mleft(f(g(\bot^\omega), \sigma)\mright)), \sigma)\mright)) \ljoin
        \cdots
    \]
    When plugged into the candidate trace construction this produces \begin{align*}
         & \sigma \mapsto \proj{1}\mleft(g(\proj{0}(f(\mu_r(\sigma), \sigma))), \proj{1}(f(\mu_r(\sigma), \sigma))\mright)
        \\
         & \qquad =
        \sigma \mapsto \proj{1}(f(\mu_\sigma^r, \sigma))
        \\
         & \qquad=
        \sigma \mapsto \proj{1}(f(
        g(\proj{0}\mleft(f(g(\dots f(g(\proj{0}\mleft(f(g(\bot^\omega), \sigma)\mright)), \sigma)\mright)), \sigma))))
        \\
         & \qquad=
        \sigma \mapsto \proj{1}(f(
        g(\proj{0}\mleft(f(g(\dots f(g(\proj{0}\mleft(f(\bot^\omega, \sigma)\mright)), \sigma)\mright)), \sigma))))
    \end{align*}
    Both the left-hand and the right-hand sides of the sliding equation are
    equal, so the construction is indeed a trace.
\end{proof}

We now have two traced PROPs: a \emph{syntactic} PROP of sequential circuit
terms \(\scircsigma\) and a \emph{semantic} PROP of causal, finitely
specified, monotone stream functions \(\streami\).
It would be straightforward to now define a map from circuits to these stream
functions; indeed, this is the strategy used in~\cite{ghica2024fully}.
Instead, we will first examine another structure with close links to both
circuits and stream functions; that of \emph{Mealy machines}.
The structure of Mealy machines will come in useful when considering the
\emph{completeness} of the denotational semantics.
\section{Monotone Mealy machines}\label{sec:mealy}

It is not immediately obvious how to translate back from stream functions in
\(\streami\) to circuits in \(\scircsigma\).
Even though these stream functions have finitely many stream derivatives, how
does one encapsulate this behaviour into a circuit made from finitely many
components while taking into account inputs and state?
Fortunately, we have a secret weapon: the
\emph{Mealy machine}~\cite{mealy1955method}.

Mealy machines are used in circuit design to specify the behaviour of a circuit
without having to use concrete components.  They also have
a very useful \emph{coalgebraic} viewpoint which we will wield in order to
build a bridge from circuits into stream functions.
In particular, there is a unique homomorphism from a Mealy machine to a causal,
finitely specified stream function.

Our strategy is to assemble a special class of Mealy machines which we dub
\emph{monotone Mealy machines} into another traced PROP.
As well as helping us on our way to our end goal of a sound and complete
denotational semantics, having a PROP of Mealy machines from which we can
translate into sequential circuit morphisms is nice to have in its own right.

\begin{definition}[Mealy machine~\cite{mealy1955method}]\label{def:mealy}
    Let \(A\) and \(B\) be finite sets.
    A (finite) \((A,B)\)-\emph{Mealy machine} is a tuple \((S, f)\) where
    \(S\) is a finite set called the \emph{state space},
    \(\morph{f}{S}{(S \times B)^A}\) is the \emph{Mealy function}.
\end{definition}

An \((A,B)\)-Mealy machine is parameterised over a set \(A\) of \emph{inputs} and
a set \(B\) of \emph{outputs}.
Each Mealy machine then has a set \(S\) of \emph{states}; the Mealy function
takes in a pair \((s, a)\) of a current state and an input, and
produces a pair \(\langle{t,b}\rangle\) of a transition state and an output.

\begin{notation}
    We will use the shorthand \(
    \mealyfunctiontransition{f} \coloneqq (s, a) \mapsto \proj{0}(f(s)(a))
    \) and \(
    \mealyfunctionoutput{f} \coloneqq (s, a) \mapsto \proj{1}(f(s)(a))
    \) for the transition and output component of the Mealy function respectively.
\end{notation}

\begin{example}\label{ex:mealy}
    Let the set of Booleans be defined as
    \(\booleans \coloneqq \{\mathsf{f},\mathsf{t}\}\).
    We define a \((\booleans,\booleans)\)-Mealy machine \((S, f)\) as follows:
    \begin{gather*}
        S \coloneqq \{s_0, s_1\}
        \\
        f(s_0, \mathsf{f}) \mapsto \langle{s_0, \mathsf{f}}\rangle
        \qquad
        f(s_0, \mathsf{t}) \mapsto \langle{s_1, \mathsf{t}}\rangle
        \\
        f(s_1, \mathsf{f}) \mapsto \langle{s_1, \mathsf{t}}\rangle
        \qquad
        f(s_1, \mathsf{t}) \mapsto \langle{s_0, \mathsf{f}}\rangle
    \end{gather*}
    This is a Mealy machine with two states; at state \(s_0\) the output is the
    input, and at state \(s_1\) the output is the negation.
    If the input is true then the state switches.

    It is often more intuitive to draw out a Mealy machine.
    We draw states as circles; if there is a transition from one state to
    another on input \(v\) that produces output \(w\), we draw an arrow between
    them labelled with \(v|w\).
    \begin{center}
        \includestandalone{figures/mealy/example}
    \end{center}
\end{example}

\subsection{The coalgebraic perspective}

The definition of Mealy machine above is timeless and forms the basis for most
of modern electronics.
The natural question for the categorist to ask is
\emph{can we make it more categorical?}
And as is often the case, we can, using the notion of a \emph{coalgebra}.

\begin{definition}[Coalgebra]
    For a category \(\mcc\), let \(\morph{F}{\mcc}{\mcc}\) be an endofunctor.
    A \emph{coalgebra} for \(F\), or \(F\)-coalgebra, is an object
    \(A \in \mcc\) along with a morphism \(\morph{\alpha}{A}{FA} \in \mcc\),
    usually written \((A,\alpha)\).
\end{definition}

A Mealy machine is a pair of a set
and a function, so this is a coalgebra in \(\set\).

\begin{definition}
    For sets \(A\) and \(B\), an \emph{\((A,B)\)-Mealy coalgebra} is a coalgebra
    of the functor \(\morph{Y}{\set}{\set}\), defined as
    \(S \mapsto (S \times B)^A\).
\end{definition}

\begin{example}
    In \cite{bonsangue2008coalgebraic}, the notation used for the transition
    and the output Mealy machines coincides with the notation used for
    the initial output and stream derivative of causal stream functions.
    Given sets \(A\) and \(B\), let \(\Gamma\) be the set of causal stream
    functions \(\stream{A} \to \stream{B}\), and let
    \(\morph{\nu}{\Gamma}{(\Gamma \times B)^A}\) be the function defined as \(
    (f, a) \mapsto \langle{\streamderivative{f}{a},\initialoutput{f}{a},}\rangle
    \).
    Then \((\Gamma,\nu)\) is a \((A,B)\)-Mealy coalgebra.
\end{example}

The above example lays the groundwork to establish connections between circuits,
stream functions and Mealy machines.
If we inspect it a little closer, we find that stream functions are even more
special than just being `an' \((A,B)\)-Mealy coalgebra.

\begin{definition}[Mealy homomorphism]\label{def:mealy-homomorphism}
    For sets \(A\) and \(B\), a \emph{Mealy homomorphism} between two
    \((A,B)\)-Mealy coalgebra \((S,f)\) and \((T,g)\) is a function
    \(\morph{h}{S}{T}\) preserving transitions and
    outputs, i.e.\ if \(f(s,a) = (r,b)\), then \(g(h(s),a) = (h(r),b)\).
\end{definition}

The \emph{final} \((A,B)\)-Mealy coalgebra is a \((A,B)\)-Mealy coalgebra to
which every other \((A,B)\)-Mealy coalgebra has a unique homomorphism.

\begin{proposition}[\cite{rutten2006algebraic}, Prop.\ 2.2]
    \label{prop:final-coalgebra}
    For every \((A,B)\)-Mealy coalgebra \((S,f)\), there exists a
    unique \((A,B)\)-Mealy homomorphism \(\morph{{!}}{(S,f)}{(\Gamma,\nu)}\).
\end{proposition}
\begin{proof}
    An \((A,B)\)-Mealy homomorphism \(\morph{g}{(S, f)}{(\Gamma, \nu)}\) is a
    function \(S \to \Gamma\), so for a state \(s_0 \in S\), \(g(s)\) will be a
    stream function \(\stream{A} \to \stream{B}\).
    Let \(\sigma \in \stream{A}\) be an input stream; there is a (unique) series
    of transitions \[
        s_0
        \mealyarrow{\sigma(0)}{b_0}
        s_1
        \mealyarrow{\sigma(1)}{b_1}
        s_2
        \mealyarrow{\sigma(2)}{b_2}
        s_3
        \mealyarrow{\sigma(3)}{b_3}
        \cdots
    \]
    Then \(!(s)\) is defined for input \(\sigma\) and
    index \(i \in \nat\) as \(!(s)(\sigma)(i) \coloneqq b_i\).
\end{proof}

For a Mealy coalgebra \((S, f)\) and a start state \(s_0\),
\(!(s_0)(\sigma)\) maps to the stream of outputs that \((S, f)\) would produce
by applying \(f\) to each element of \(\sigma\), starting from \(s_0\).

\subsection{Mealy machines on posets}

In order to use Mealy machines as a bridge between \(\scircsigma\) and
\(\streami\), they must be assembled into another traced PROP, and morphisms
defined between all three categories in play.
But not all Mealy machines defined thus far correspond to circuits in
\(\scircsigma\); as with stream functions, we must refine our notion of Mealy
machine in order to find those that do.

A Mealy machine \((S, f)\) with start state \(s_0\) specifies the behaviour of a
circuit in \(\scircsigma\) if the stream function \(!(s_0)\) is in \(\streami\).
As established, all stream functions in the image of \(!\) are causal, and since
we only work with \emph{finite} Mealy machines we can also conclude the
following:

\begin{lemma}
    For a Mealy machine \((S, f)\) and state \(s_0 \in S\), \(!(s_0)\)
    is finitely specified.
\end{lemma}
\begin{proof}
    \(S\) is finite, and \(\mealytostream\) must preserve transitions.
\end{proof}

\todo[inline]{Fix the preorder-poset disaster}

\begin{definition}[Monotone Mealy machine]
    Let \(A\), \(B\) and \(S\) be posets; an \((A,B)\)-Mealy machine \((S, f)\)
    is called a \emph{monotone} Mealy machine if \(f\) is \(\bot\)-preserving
    and monotone with respect to the appropriate orders.
\end{definition}

To map to Mealy machines from circuits we need to assemble the former into
another PROP, in which the morphisms \(m \to n\) are
\((\valuetuple{m},\valuetuple{n})\)-Mealy machines.
As circuits in \(\scircsigma\) have a `hardcoded' initial state in the form of
the value generators, this information will also need to be specified for Mealy
machines.

\begin{definition}[Initialised Mealy machine]
    An \emph{initialised} Mealy machine is a tuple \((S, f, s_0)\), where
    \((S, f)\) is a Mealy machine, and \(s_0 \in S\) is an \emph{initial state}.
\end{definition}

\begin{example}\label{ex:mealy-init}
    We can initialise the \((\booleans,\booleans)\)-Mealy machine
    \((\{s_0,s_1\},f)\) from \cref{ex:mealy} in two ways; here we will choose to
    initialise it as \((S,f,s_0)\).
    In the diagrams, we label the initial state with an arrow.
    \begin{center}
        \includestandalone{figures/mealy/example-init}
    \end{center}
\end{example}

All that remains to define is the composition of Mealy machines, which is
standard.

\begin{definition}[Cascade product of Mealy machines~\cite{ginzburg2014algebraic}]
    Given an initialised \((A,B)\)-Mealy machine \((S,f,s_0)\) and an
    initialised \((B,C)\)-Mealy machine \((T,g,t_0)\), their
    \emph{cascade product} is an initialised \((A,C)\)-Mealy machine defined as
    \[
        \left(S \times T, ((s, t), a) \mapsto \left\langle
        \left(
        \mealyfunctiontransition{f}(s,a),
        \mealyfunctiontransition{g}(t, \mealyfunctionoutput{f}(s, a))
        \right),
        \mealyfunctionoutput{g}(t, \mealyfunctionoutput{f}(s, a))
        \right\rangle,
        (s_0, t_0)\right).
    \]
\end{definition}

The cascade product of two Mealy machines effectively executes the first on the
inputs, then executes the second on the outputs of the first; the inputs are
`cascaded' through the two Mealy machines.

\begin{example}\label{ex:mealy-cascade}
    Recall the initialised \((\booleans,\booleans)\)-Mealy machine
    \((S, f, s_0)\) from \cref{ex:mealy-init}; we will now compose this with
    another initialised \((\booleans,\booleans)\)-Mealy machine
    \((\{t_0,t_1\},g,t_0)\) where \(g\) is defined as follows:
    \begin{gather*}
        g(t_0, \mathsf{f}) \coloneqq \langle{t_0,\mathsf{f}}\rangle
        \qquad
        g(t_0, \mathsf{t}) \coloneqq \langle{t_1,\mathsf{t}}\rangle
        \qquad
        g(t_1, \mathsf{f}) \coloneqq \langle{t_1,\mathsf{t}}\rangle
        \qquad
        g(t_1, \mathsf{t}) \coloneqq \langle{t_1,\mathsf{t}}\rangle
        \\
        \includestandalone{figures/mealy/example2-init}
    \end{gather*}
    The cascade product of these two machines is another initialised
    \((\booleans,\booleans)\)-Mealy machine \((R,h,r_0)\) defined as follows:
    \begin{gather*}
        R \coloneqq \{(s_0,t_0), (s_1,t_0), (s_0,t_1), (s_1,t_1)\}
        \qquad
        r_0 \coloneqq (s_0,t_0)
        \\
        h((s_0, t_0), \mathsf{f})
        \coloneqq
        \left\langle(s_0, t_0), \mathsf{f}\right\rangle
        \qquad
        h((s_0, t_0), \mathsf{t})
        \coloneqq
        \left\langle(s_0, t_1), \mathsf{t}\right\rangle
        \\
        h((s_1, t_0), \mathsf{f})
        \coloneqq
        \left\langle(s_1, t_1), \mathsf{t}\right\rangle
        \qquad
        h((s_1, t_0), \mathsf{t})
        \coloneqq
        \left\langle(s_0, t_0), \mathsf{f}\right\rangle
        \\
        h((s_0, t_1), \mathsf{f})
        \coloneqq
        \left\langle(s_0, t_1), \mathsf{t}\right\rangle
        \qquad
        h((s_0, t_1), \mathsf{t})
        \coloneqq
        \left\langle(s_1, t_1), \mathsf{t}\right\rangle
        \\
        h((s_1, t_1), \mathsf{f})
        \coloneqq
        \left\langle(s_1, t_1), \mathsf{t}\right\rangle
        \qquad
        h((s_1, t_1), \mathsf{t})
        \coloneqq
        \left\langle(s_0, t_1), \mathsf{t}\right\rangle
        \\
        \includestandalone{figures/mealy/cascade}
    \end{gather*}
\end{example}

Tensor product is far more straightforward.

\begin{definition}[Direct product of Mealy machines]
    Given an initialised \((A,B)\)-Mealy machine \((S,f,s_0)\) and an
    initialised \((C,D)\)-Mealy machine \((T,g,t_0)\), their
    \emph{direct product} is an initialised \((A \times C,B \times D)\)-Mealy
    machine defined as \[
        (S \times T, \left((s, t), (a, c)\right) \mapsto \left\langle
        \left(
        \mealyfunctiontransition{f}\mleft(s, a\mright),
        \mealyfunctiontransition{g}\mleft(s, a\mright)
        \right),
        \left(
        \mealyfunctionoutput{f}\mleft(s, a\mright),
        \mealyfunctionoutput{g}\mleft(s, a\mright)
        \right)\right\rangle,
        (s_0, t_0)
        ).
    \]
\end{definition}

\begin{example}\label{ex:mealy-direct}
    The direct product of the two initialised \((\booleans,\booleans)\)-Mealy
    machines introduced in \cref{ex:mealy-init} and \cref{ex:mealy-cascade} is
    a \((\booleans^2,\booleans^2)\)-Mealy machine \((Q,k,q_0)\) defined as
    follows:
    \begin{gather*}
        Q \coloneqq \{(s_0,t_0), (s_1,t_0), (s_0,t_1), (s_1,t_1)\}
        \qquad
        q_0 \coloneqq (s_0,t_0)
        \\
        h((s_0, t_0), \mathsf{ff})
        \coloneqq
        \left\langle(s_0, t_0), \mathsf{ff}\right\rangle
        \qquad
        h((s_0, t_0), \mathsf{tf})
        \coloneqq
        \left\langle(s_1, t_0), \mathsf{tf}\right\rangle
        \\
        h((s_0, t_0), \mathsf{ft})
        \coloneqq
        \left\langle(s_0, t_1), \mathsf{ft}\right\rangle
        \qquad
        h((s_0, t_0), \mathsf{tt})
        \coloneqq
        \left\langle(s_1, t_1), \mathsf{tt}\right\rangle
        \\
        h((s_1, t_0), \mathsf{ff})
        \coloneqq
        \left\langle(s_1, t_0), \mathsf{tf}\right\rangle
        \qquad
        h((s_1, t_0), \mathsf{tf})
        \coloneqq
        \left\langle(s_0, t_0), \mathsf{ff}\right\rangle
        \\
        h((s_1, t_0), \mathsf{ft})
        \coloneqq
        \left\langle(s_1, t_1), \mathsf{tt}\right\rangle
        \qquad
        h((s_1, t_0), \mathsf{tt})
        \coloneqq
        \left\langle(s_0, t_1), \mathsf{ft}\right\rangle
        \\
        h((s_0, t_1), \mathsf{ff})
        \coloneqq
        \left\langle(s_0, t_1), \mathsf{ft}\right\rangle
        \qquad
        h((s_0, t_1), \mathsf{tf})
        \coloneqq
        \left\langle(s_1, t_1), \mathsf{tt}\right\rangle
        \\
        h((s_1, t_1), \mathsf{ft})
        \coloneqq
        \left\langle(s_0, t_1), \mathsf{ft}\right\rangle
        \qquad
        h((s_0, t_1), \mathsf{tt})
        \coloneqq
        \left\langle(s_1, t_1), \mathsf{tt}\right\rangle
        \\
        h((s_1, t_1), \mathsf{ff})
        \coloneqq
        \left\langle(s_1, t_1), \mathsf{tt}\right\rangle
        \qquad
        h((s_1, t_1), \mathsf{tf})
        \coloneqq
        \left\langle(s_0, t_1), \mathsf{ft}\right\rangle
        \\
        h((s_1, t_1), \mathsf{ft})
        \coloneqq
        \left\langle(s_1, t_1), \mathsf{tt}\right\rangle
        \qquad
        h((s_1, t_1), \mathsf{tt})
        \coloneqq
        \left\langle(s_0, t_1), \mathsf{ft}\right\rangle
        \\
        \includestandalone{figures/mealy/direct}
    \end{gather*}
\end{example}

With cascade product as composition and direct product as tensor, initialised
monotone Mealy machines form a PROP.

\begin{definition}
    Let \(\mealyi\) be the PROP in which the morphisms
    \(m \to n\) are the initialised monotone
    \((\valuetuple{m}, \valuetuple{n})\)-Mealy machines.
    Composition is by cascade product and tensor on morphisms is by
    direct product.
    The identity and the symmetry are the single-state machines that output the
    input and swap the inputs respectively.
\end{definition}

Once again, we must show that this category has a trace.
This can be computed in much the same way as it was for stream functions.

\begin{definition}
    Let \((S, f)\) be a monotone \(
    (\valuetuple{x+m}, \valuetuple{x+n})
    \)-Mealy machine.
    For a state \(s \in S\) and input \(\listvar{a} \in \valuetuple{m}\), let
    \(\mu_{s}(\listvar{a})\) be the least fixed point of \(
    \listvar{r} \mapsto \proj{0}\mleft(f\left(s, \listvar{ra}\right)\mright)
    \).
    The \emph{least fixed point} of an initialised Mealy machine \((S, f, s_0)\)
    is a \((\valuetuple{m}, \valuetuple{n})\)-Mealy machine \(\left(
    S, (s, \listvar{a})
    \mapsto
    f\left(\left(\mu_{s}\left(\listvar{a}\right)\right)\listvar{a}\right), s_0
    \right)
    \)
\end{definition}

\begin{example}\label{ex:trace-mealy}
    Consider the monotone \((\valuetuple{3},\valuetuple{3})\)-Mealy machine with
    state set \(\belnapvalues\), initial state \(\bot\), and Mealy function \[
        g \coloneqq (s, (x, y, z))
        \mapsto \left\langle \neg y \land \neg x,
        \left(\neg s \land \neg z, s, \neg s \land \neg z\right)
        \right\rangle
        ).\]
    To take the trace of this machine, we must first compute the least fixed
    point of \(v \mapsto \neg s \land \neg z\), which is clearly just
    \(\neg s \land \neg z\).
    Therefore the Mealy function of the traced
    \((\valuetuple{2}, \valuetuple{2})\) machine is \(
    (s, (y, z)) \mapsto g(s, (\neg s \land \neg z, y, \neg s \land \neg z))
    \).
\end{example}

We must show that this is the trace on \(\mealyi\).

\begin{proposition}
    The least fixed point is a trace on \(\mealyi\).
\end{proposition}
\begin{proof}
    Let \((S, f)\) be a monotone
    \((\valuetuple{x+m}, \valuetuple{x+n})\)-Mealy machine.
    The Mealy function \(
    \morph{f}{
        S \times \valuetuple{x+m}
    }{
        S \times \valuetuple{x+n}
    }
    \) is monotone with regards to the orders on \(S\) and
    \(\valuetuple{x+m}\) and \(S \times x+n\) is finite, so
    \(f\) has a least fixed point.
    The function \(
    f^\prime \coloneqq (s, \listvar{a})
    \mapsto
    \proj{1}\mleft(f((\mu_{s,\listvar{a}})\listvar{a})\mright)
    \) is a composition of monotone functions, so it is itself monotone.
    This means \((S, f^\prime)\) is a monotone \(
    (\valuetuple{m}, \valuetuple{n})
    \)-Mealy machine.
    This construction is a trace for the same reason as the trace of
    \(\streami\) is.
\end{proof}

With monotone Mealy machines in a PROP, we can now represent the unique
homomorphism from a Mealy machine to a set of state functions as a PROP
morphism.

\begin{proposition}\label{prop:mealy-to-stream}
    There is a traced PROP morphism
    \(\morph{\mealytostreami}{\mealyi}{\streami}\) sending a monotone Mealy
    machine \(\morph{\left(S, f, s_0\right)}{m}{n}\) to \(!(s_0)\), where \(!\)
    is the unique homomorphism \((S,f) \to (\Gamma,\nu)\).
\end{proposition}
\begin{proof}
    Since every stream function also has a Mealy coalgebra structure and Mealy
    homomorphisms preserve transitions and outputs,
    composition of Mealy machines also coincides with composition of stream
    functions.
\end{proof}

\subsection{Streams to Mealy machines}

So far we have seen how a causal, finitely specified, and \(\bot\)-preserving
monotone stream function can be retrieved from a monotone Mealy machine.
It is also possible to retrieve a Mealy machine for a given stream function in
\(\streami\) by repeatedly taking stream derivatives; since we know there are
finitely many we will be able to compute a finite set of states in a Mealy
machine.

\begin{example}
    Let \(\morph{f}{\belnapvalues}{\belnapvalues}\) be a stream function defined
    as \(f(\sigma)(0) = \sigma(0)\) and
    \(f(\sigma)(k+1) = f(\sigma)(k) \land \sigma(k+1)\).
    We will derive a Mealy machine in \(\mealyi\) from this stream function.
    The complete set of states is \(\{
    f, \streamderivative{f}{\bot}, \streamderivative{f}{\belnapfalse},
    \streamderivative{f}{\top}
    \}\):
    \begin{itemize}
        \item \(\streamderivative{f}{\belnaptrue} = f\);
        \item \(\streamderivative{(\streamderivative{f}{\bot})}{\bot}
              =
              \streamderivative{(\streamderivative{f}{\bot})}{\belnaptrue}
              =
              \streamderivative{f}{\bot}
              \);
        \item \(\streamderivative{(\streamderivative{f}{\top})}{\top}
              =
              \streamderivative{(\streamderivative{f}{\top})}{\belnaptrue}
              =
              \streamderivative{f}{\top}
              \); and
        \item \(
              \streamderivative{(\streamderivative{f}{\bot})}{\belnapfalse}
              =
              \streamderivative{(\streamderivative{f}{\bot})}{\top}
              =
              \streamderivative{(\streamderivative{f}{\top})}{\bot}
              =
              \streamderivative{(\streamderivative{f}{\top})}{\belnapfalse}
              =
              \streamderivative{f}{\belnapfalse}
              \).
    \end{itemize}
    The Mealy function is defined for each state as the initial value and
    stream derivative of the original stream function.
    The initial state of the Mealy machine is \(f\).
\end{example}

In fact, for a function \(f\) this procedure produces a \emph{minimal} Mealy
machine.

\begin{corollary}[Corollary 2.3, \cite{rutten2006algebraic}]\label{cor:minimal-mealy}
    For a causal, finitely specified, stream function \(
    \morph{f}{\stream{M}}{\stream{N}}
    \), let \(S\) be the least set of
    causal stream functions including \(f\) and closed under stream derivatives:
    i.e.\ for all \(h \in S\) and \(a \in M\), \(h_a \in S\).
    Then the initialised Mealy machine \(
    \streamtomealy[f]{\interpretation} = (S, g, f)
    \), where \(
    g(h)(a) = \langle \mealyoutput{h}{a}, \mealytransition{h}{a}\rangle
    \), has the smallest state space of Mealy machines with the property \(
    \mealytostreami[\streamtomealyi[f]] = f
    \).
\end{corollary}
\begin{proof}
    Since \(S\) is generated from the function \(f\) and is the \emph{smallest}
    possible set, there are no unreachable states in \(S\).
    Since the output and transition of states in
    \(\streamtomealyi[f]\) are the initial output and stream derivative, two
    states can only have the same `behaviour' if they are derived from the same
    original stream function.
\end{proof}

We will encode this data into a PROP morphism from \(\streami\) to \(\mealyi\);
in order to do this we must verify that the produced Mealy machine is monotone.

\begin{lemma}\label{lem:head-tail-monotone}
    The functions \(\sigma \mapsto \streaminit(\sigma)\) and
    \(\sigma \mapsto \streamderv(\sigma)\) are monotone.
\end{lemma}
\begin{proof}
    Let \(\sigma \coloneqq a \streamcons \sigma^\prime\) and
    \(\tau \coloneqq b \streamcons \tau^\prime\) be streams in \(\stream{A}\)
    such that \(\sigma \leq_{\stream{A}} \tau\); subsequently \(a \leq b\) and
    \(\sigma^\prime \leq_{\stream{A}} \tau^\prime\).
    So \(
    \streaminit(\sigma) \coloneqq
    \streaminit(a \streamcons \sigma^\prime) =
    a \leq_{A}
    b =
    \streaminit(b \streamcons \tau^\prime) \coloneqq
    \streaminit(\tau)
    \) and \(
    \streamderv(\sigma) \coloneqq
    \streamderv(a \streamcons \sigma^\prime) =
    \sigma^\prime \leq_{\stream{A}}
    \tau^\prime =
    \streamderv(b \streamcons \tau^\prime) =
    \streamderv(\tau)
    \).
\end{proof}

\begin{lemma}\label{lem:initial-derivative-monotone}
    For posets \(A\) and \(B\) and a monotone causal stream function
    \(\morph{f}{\stream{A}}{\stream{B}}\), the functions
    \(a \mapsto \initialoutput{f}{a}\) and \(a \mapsto \streamderivative{f}{a}\)
    are monotone.
\end{lemma}
\begin{proof}
    Let \(a, b \in A\) such that \(a \leq_A b\); then by monotonicity
    \(f(a \streamcons \sigma) \leq_{\stream{B}} f(b \streamcons \sigma)\) for
    all \(\sigma \in \stream{A}\).
    By \cref{lem:head-tail-monotone}, \(\streaminit \circ f\) and
    \(\streamderv \circ f\) are monotone.
    First we show that the initial output is monotone: \(
    \initialoutput{f}{a} \coloneqq
    \streaminit(f(a \streamcons \sigma)) \leq_{A}
    \streaminit(f(b \streamcons \sigma)) =
    \initialoutput{f}{b}
    \).
    For the stream derivative, \(
    \streamderivative{f}{a}(\sigma) \coloneqq
    \streamderv(f(a \streamcons \sigma)) \leq_{\stream{B}}
    \streamderv(f(b \streamcons \sigma)) \coloneqq
    \streamderivative{f}{a}(\sigma)
    \).
\end{proof}

\begin{lemma}\label{lem:stream-to-mealy-is-monotone}
    Given a stream function \(f \in \streami\), \(\streamtomealyi[f]\) is
    a monotone Mealy machine.
\end{lemma}
\begin{proof}
    Each state in the derived Mealy machine is a monotone stream function, so
    this is a poset ordered by \(\stateorder\) as defined in
    \cref{def:state-order}. and
    The Mealy function is the pairing of the initial output and stream
    derivative; by \cref{lem:initial-derivative-monotone} these are monotone so
    the Mealy function must also be monotone.
\end{proof}

\begin{corollary}
    The procedure \(\streamtomealyi\) is a PROP morphism
    \(\streami \to \mealyi\).
\end{corollary}

This means we can map between monotone Mealy machines and causal, finitely
specified, monotone stream functions in either direction.
Mealy machines are perhaps more intuitive to work with, but stream functions
are the `purest' specification of the behaviour in that they have the smallest
possible state set.
Ideally we would be able to work in whichever setting is most beneficial at a
given time, so we need to show that the translations are
\emph{behaviour-preserving}.

\begin{proposition}\label{prop:mealy-to-stream}
    \(\mealytostreami \circ \streamtomealyi = \id[\streami]\).
\end{proposition}
\begin{proof}
    Stream functions are equal if they have the same initial output and
    stream derivative.
    \(\streamtomealyi\) preserves outputs and derivatives by definition, and
    \(\mealytostreami\) preserves transitions and outputs because it is a Mealy
    homomorphism.
\end{proof}

The reverse does not hold as many Mealy machines may map to the same stream
function, but the result of
\(\mealytostreami \circ \streamtomealyi \circ \mealytostreami\) is of course
equal to \(\mealytostreami\).
\section{Between circuits and Mealy machines}\label{sec:synthesis}

The close links between \(\streami\) and \(\mealyi\) are nice to have but hardly
ground-breaking; the main contribution of this chapter is to introduce
\(\scircsigma\) to the mix by defining maps \(\scircsigma \to \mealyi\) and
\(\mealyi \to \scircsigma\).
This allows us to use monotone Mealy machines as a stepping stone in the
correspondence between sequential circuits and stream functions.
By exploiting the coalgebraic properties shared between Mealy machines and
stream functions, this can be used to show that \(\streami\) is both a
\emph{sound} and \emph{complete} denotational semantics: there is a stream
function in \(\streami\) for every circuit in \(\scircsigma\), and there is a
circuit in \(\scircsigma\) for every stream function in \(\streami\).

\subsection{Circuits to monotone Mealy machines}

Circuits have a very natural interpretation as Mealy machines, so the action
of a PROP morphism from \(\scircsigma\) to \(\mealyi\) is fairly intuitive.

\begin{definition}
    \nomenclature{\(\circuittomealyi\)}{PROP morphism \(\scircsigma \to \mealyi\)}
    Let \(\morph{\circuittomealyi}{\scircsigma}{\mealyi}\) be the traced PROP
    morphism defined on generators as
    \begin{align*}
        \circuittomealy[
            \iltikzfig{circuits/components/gates/gate}[colour=comb]
        ]{\interpretation}
         & \coloneqq (
        \{s\},
         &             & \left(s, \listvar{v}\right) \mapsto
        \left\langle{s, \gateinterpretation[g](\listvar{v})}\right\rangle,
         &             & s
        )
        \\
        \circuittomealy[
            \iltikzfig{strings/structure/comonoid/copy}[colour=comb]
        ]{\interpretation}
         & \coloneqq (
        \{s\},
         &             & (s, v) \mapsto \left\langle{s, (v, v)}\right\rangle,
         &             & s
        )
        \\
        \circuittomealy[
            \iltikzfig{strings/structure/monoid/merge}[colour=comb]
        ]{\interpretation}
         & \coloneqq (
        \{s\},
         &             & (s, (v, w)) \mapsto
        \left\langle{s, (v \ljoin w)}\right\rangle,
         &             & s
        )
        \\
        \circuittomealy[
            \iltikzfig{strings/structure/comonoid/discard}[colour=comb]
        ]{\interpretation}
         & \coloneqq (
        \{s\},
         &             & (s, v) \mapsto
        \left\langle{s, ()}\right\rangle,
         &             & s
        )
        \\
        \circuittomealy[
            \iltikzfig{circuits/components/values/vs}[val=v]
        ]{\interpretation}
         & \coloneqq
        (
        \{s_v, s_\bot\},
         &             & \{
        s_v \mapsto \left\langle{s_\bot,v}\right\rangle,
        s_\bot \mapsto \left\langle{s_\bot,\bot}\right\rangle
        \},
         &             & s_v
        )
        \\
        \circuittomealy[
            \iltikzfig{circuits/components/waveforms/delay}
        ]{\interpretation}
         & \coloneqq
        (
        \{ s_v \,|\, v \in \values\},
         &             & (s_v, a) \mapsto \left\langle{v,s_a}\right\rangle,
         &             & s_\bot
        )
    \end{align*}
\end{definition}

\begin{example}
    The action of \(\circuittomealy{\belnapinterpretation}\) on values and
    delays in \(\scirc{\belnapsignature}\) is illustrated in
    \cref{fig:belnap-machines}.
\end{example}

\begin{figure}
    \centering
    \includestandalone{figures/mealy/value}

    \vspace{1em}

    \includestandalone{figures/mealy/delay}
    \caption{
        Mealy machines for Belnap values and delays
    }
    \label{fig:belnap-machines}
\end{figure}

\begin{example}\label{ex:mealy-translation}
    \index{SR NOR latch}
    Applying \(\circuittomealy{\belnapinterpretation}\) to the SR NOR latch from
    \cref{ex:sr-latch} produces the monotone Mealy machine in
    \cref{ex:trace-mealy}, which is illustrated in \cref{fig:latch-machine}.
\end{example}

\begin{figure}
    \centering
    \includestandalone{figures/mealy/latch-example}
    \caption{
        Mealy machine from \cref{ex:trace-mealy}
    }
    \label{fig:belnap-machines}
\end{figure}

Mealy machines are a reasonable semantics for sequential circuits, but the
image of \(\circuittomealyi\) does not always lead to minimal Mealy machines,
and there are many Mealy machines that may correspond to the same behaviour.
The `purest' semantics of a sequential circuit is a stream function in
\(\streami\).

\begin{definition}
    \nomenclature{\(\circuittostreami\)}{PROP morphism \(\scircsigma \to \streami\)}
    Let \(\morph{\circuittostreami}{\scircsigma}{\streami}\) be defined as
    \(\mealytostreami \circ \circuittomealyi\).
\end{definition}

We have now finally established the \emph{denotation} of a sequential circuit \(
\iltikzfig{strings/category/f}[box=f,colour=seq,dom=m,cod=n]
\): it is the stream function \(
\morph{\circuittostreami[\iltikzfig{strings/category/f}[box=f,colour=seq]]}{\valuetuplestream{m}}{\valuetuplestream{n}}
\).
The existence of the PROP morphism \(\circuittostreami\) confirms that causal,
finitely specified and \(\bot\)-preserving monotone stream functions are a
\emph{sound} denotational semantics for sequential circuits, as every circuit in
\(\scircsigma\) has a corresponding stream function in \(\streami\).

It is useful to verify that this denotational semantics of sequential circuits
agrees with the denotational semantics we defined earlier for
\emph{combinational} circuits in \cref{sec:interpreting-components}.

\begin{lemma}\label{lem:sequential-combinational-semantics}
    Let \(\iltikzfig{strings/category/f}[box=f,colour=comb,dom=m,cod=n]\) be
    a combinational circuit; for \(\sigma \in \valuetuplestream{m}\) and
    \(i \in \nat\), \(
    \circuittostreami[\iltikzfig{strings/category/f}[box=f,colour=comb]](\sigma)(i)
    =
    \circuittofunci[\iltikzfig{strings/category/f}[box=f,colour=comb]](\sigma(i))
    \).
\end{lemma}
\begin{proof}
    Since \(\iltikzfig{strings/category/f}[box=f,colour=comb,dom=m,cod=n]\) is
    combinational, \(
    \circuittomealyi[
        \iltikzfig{strings/category/f}[box=f,colour=comb,dom=m,cod=n]
    ]
    \) is a Mealy machine with a single state \(s\), i.e.\ there is a function
    \(\morph{g}{\valuetuple{m}}{\valuetuple{n}}\) such that  \(
    \circuittomealyi[
        \iltikzfig{strings/category/f}[box=f,colour=comb,dom=m,cod=n]
    ] = (
    \{s\},
    (s, \listvar{v}) \mapsto \left\langle{s, g(\listvar{v})}\right\rangle,
    s
    )\).
    By definition of \(\mealytostreami\), we have that \(\mealytocircuiti[
        \circuittomealyi[
            \iltikzfig{strings/category/f}[box=f,colour=comb,dom=m,cod=n]
        ]
    ](\sigma)(i) = g(\sigma(i))\).
    To complete the proof, we need to show that \(
    g(\sigma)(i) =
    \circuittofunci[\iltikzfig{strings/category/f}[box=f,colour=comb]](\sigma(i))
    \); this holds because \(\circuittomealyi\) and \(\circuittofunci\) freely
    build functions using the same constructs.
\end{proof}

Using this idea, it will be convenient to have a mapping from functions to
these constant stream functions.

\begin{definition}
    \nomenclature{\(\functostreami\)}{PROP morphism \(\funci \to \streami\)}
    Let \(\morph{\functostreami}{\funci}{\streami}\) be defined as the PROP
    morphism with action \(
    \functostreami[f] \coloneqq \sigma \mapsto i \mapsto f(\sigma)(i)
    \)
\end{definition}

\subsection{Monotone Mealy machines to circuits}

We now need a way to retrieve a circuit morphism in \(\scircsigma\) from a
stream function \(f \in \streami\).
To prevent us from picking an arbitrary circuit, the denotation of the circuit
must also be \(f\).

We already know by \cref{cor:minimal-mealy} that given a stream function
\(f\) we can retrieve a monotone Mealy machine \(\streamtomealyi[f]\).
All that remains is to translate this into a circuit morphism.
For regular Mealy machines, there is a standard procedure in circuit
design~\cite{kohavi2009switching} in which each state of a Mealy machine is
\emph{encoded} as a power of values, and the Mealy function is interpreted as
a circuit using combinational logic.

\begin{example}\label{ex:boolean-to-circuit}
    \index{Boolean values}
    Consider the following Mealy machine operating on Boolean values.
    \begin{center}
        \includestandalone{figures/mealy/boolean-example}
    \end{center}
    \vspace{-\belowdisplayskip}
    To convert this machine to a circuit, we assign each state a boolean value:
    in this case \(s_0 \mapsto \belnapfalse, s_1 \mapsto \belnaptrue\).
    We can now construct a truth table to show how a state and an input map to
    a transition and an output:
    \begin{center}
        \begin{tabular}{cc|cc}
            \(\belnapfalse\) & \(\belnapfalse\) & \(\belnaptrue\)  & \(\belnaptrue\)  \\
            \(\belnapfalse\) & \(\belnaptrue\)  & \(\belnapfalse\) & \(\belnapfalse\) \\
            \(\belnaptrue\)  & \(\belnapfalse\) & \(\belnaptrue\)  & \(\belnapfalse\) \\
            \(\belnaptrue\)  & \(\belnaptrue\)  & \(\belnapfalse\) & \(\belnaptrue\)  \\
        \end{tabular}
    \end{center}
    It is possible to describe these truth tables as logical expressions: in
    this case the expression for the next state is \(
    (v_0, v_1)
    \mapsto
    (\neg v_0 \land \neg v_1) \lor (v_0 \land \neg v_1)
    \) and the expression for the output is \(
    (v_0, v_1)
    \mapsto
    (\neg v_0 \land \neg v_1) \lor (v_0 \land v_1)
    \).
    These expressions can clearly be constructed as combinational circuits using
    \(\andgate\), \(\orgate\) and \(\notgate\) gates; the entire circuit
    corresponding to the Mealy machine is constructed by combining the
    combinational logic with registers to hold the state.
    \[\iltikzfig{mealy/synthesis}\]
\end{example}

We will use a variation of this procedure to map from \(\mealyi\) to
\(\scircsigma\).
However, when considering \emph{monotone} Mealy machines, this procedure must
additionally respect monotonicity as the combinational logic is constructed
using monotone components.
This means that an arbitrary encoding cannot be used; we will now show how to
select something suitable.

\begin{definition}[Encoding]\label{def:encoding}
    \index{encoding}
    \nomenclature{\(\gamma_\leq\)}{encoding for a total order \(\leq\)}
    Let \(S\) be a set equipped with a partial order \(\stateorder\) and a total
    order \(\leq\) such that \(S\) can be represented as
    \(s_0 \leq s_1 \leq \dots s_{k-1}\).
    The \emph{\(\leq\)-encoding} for this assignment is a function
    \(\morph{\gamma_\leq}{S}{\valuetuple{k}}\) defined as
    \(\gamma_\leq(s)(i) \coloneqq \top\) if \(s_i \stateorder s\) and
    \(\gamma_\leq(s)(i) \coloneqq \bot\) otherwise.
\end{definition}

\begin{example}
    Recall the monotone Mealy machine from \cref{ex:mealy-translation}, which
    has state set \(
    \belnapvalues \coloneqq \{\bot,\belnapfalse,\belnaptrue,\top\}
    \).
    We choose the total order on \(\belnapvalues\) as
    \(\bot \leq \belnapfalse \leq \belnaptrue \leq \top\); subsequently, the
    \(\leq\)-encoding is defined as \(
    \bot \mapsto \top\bot\bot\bot, \belnapfalse \mapsto \top\top\bot\bot,
    \belnaptrue \mapsto \top\bot\top\bot, \top \mapsto \top\top\top\top
    \).
\end{example}

It is essential that a \(\leq\)-encoding respects the original ordering of the
states.

\begin{lemma}
    For an ordered state space \((S,\stateorder)\) and a \(\leq\)-encoding
    \(\gamma_\leq\), \(s \stateorder s^\prime\) if and only if
    \(\gamma_\leq(s) \sqsubseteq \gamma_\leq(s^\prime)\).
\end{lemma}
\begin{proof}
    First the \(\onlyifdir\) direction.
    Let \(s_i, s_j \in S\) such that \(s_i \stateorder s_j\); we need to show
    that for every \(l < |S|\),
    \(\gamma_\leq(s_i)(l) \sqsubseteq \gamma_\leq(s_j)(l)\).
    The only way this can be violated is if \(s_i(l) = \top\) and
    \(s_j(l) = \bot\); this can only occur if \(s_l \stateorder s_i\) and
    \(s_l \not\stateorder s_j\).
    But since \(s_i \stateorder s_j\), this is a contradiction due to
    transitivity so \(\gamma_\leq(s_l) \sqsubseteq \gamma_\leq(s_j)\) also
    holds.

    Now the \(\ifdir\) direction.
    Assume that \(\gamma_\leq(s_i) \sqsubseteq \gamma_\leq(s_j)\); we need to
    show that \(s_i \stateorder s_j\); i.e.\ that \(\gamma_\leq(s_j)(i) = \top\)
    If \(\gamma_\leq(s_i) \sqsubseteq \gamma_\leq(s_j)\), then for each
    \(l < k\) then \(\gamma_\leq(s_i)(l) \sqsubseteq \gamma_\leq(s_j)(l)\);
    in particular \(\gamma_\leq(s_i)(i) \sqsubseteq \gamma_\leq(s_j)(i)\)
    By definition of \(\gamma_\leq\), \(\gamma_\leq(s_i)(i) = \top\), so if
    \(\gamma_\leq(s_i) \sqsubseteq \gamma_\leq(s_j)\) then
    \(\gamma_\leq(s_j)(i)\) is also \(\top\).
\end{proof}

Using this encoding, we will construct a circuit morphism that,
when interpreted as a function, implements the output and transition function
of the Mealy machine.
There is no reason for such a morphism to exist for an arbitrary interpretation:
why should we expect some collection of gates to be able to model every
function?
The useful interpretations are those that \emph{can} model every function.

\begin{definition}[Functional completeness]\label{def:functional-completeness}
    \index{functional completeness}
    A \emph{complete interpretation} is a tuple
    \((\interpretation,\mealytofunc)\) in which \(\interpretation\) is an
    interpretation of a signature \(\signature\) and \(
    \morph{\mealytofunc}{\funci}{\scircsigma}
    \) is a map that sends functions \(
    \morph{f}{\valuetuple{m}}{\valuetuple{n}}
    \), to circuits of the form \(
    \iltikzfig{circuits/synthesis/normalised-function}[box=f]
    \) for some word \(\listvar{v} \in \freemon{\values}\) such that
    \(\circuittostreami[\mealytofunc[f]](\sigma)(i) = f(\sigma(i))\).
\end{definition}

For a given complete interpretation \((\interpretation,\mealytofunc)\), we refer
to a circuit \(\mealytofunc[f]\) as the \emph{normalised circuit for \(f\)}.\

\begin{remark}
    Even though \(\mealytofunc\) maps combinational functions, its codomain is
    the category of \emph{sequential} circuits \(\scircsigma\).
    This is because instantaneous values may be required to create the
    normalised circuit.
    Despite the use of sequential components, the loop enforces that the state
    is \emph{constant}: it will always produce the word \(\listvar{v}\), so the
    the circuit still has combinational behaviour.

    Sometimes this is the only way to ensure every function can be modelled.
    For example, consider the Boolean function \(\booleans \to \booleans\) that
    always produces \(\belnapfalse\).
    Using the strategy from \cref{ex:boolean-to-circuit}, no lines of the truth
    table are true, so the expression can only be defined using the unit of the
    disjunction, the constant false.

    Note also that this sequential component is by no means mandatory: the
    functional completeness map may actually map only to combinational circuits,
    in which case the width of the sequential component would be \(0\).
\end{remark}

\begin{example}
    The Belnap interpretation from \cref{ex:belnap-interpretation} is
    functionally complete; for interests of space we postpone the proof to
    \cref{sec:denotational-belnap}.
    This is due to a variation of the standard functional
    completeness method for Boolean values.
\end{example}

With the knowledge that any monotone function has a corresponding circuit
in \(\scircsigma\), we set about encoding an arbitrary Mealy function
\(S \times \valuetuple{m} \to S \times \valuetuple{n}\) into a function
\(\valuetuple{k} \times \valuetuple{n} \to \valuetuple{k} \times \valuetuple{n}\).
One point to note here is that there may be more values in \(\valuetuple{k}\)
than there are states in \(S\), so we may need to `fill in the gaps' in a way
that is compatible with monotonicity.

\begin{definition}[Monotone completion]\label{def:monotone-completion}
    \index{monotone!completion}
    \nomenclature{\(f_\mathsf{m}\)}{monotone completion of function \(m\)}
    Let \(A\) be a finite poset and let \(B\) be a finite lattice such that
    \(A \subseteq B\).
    Then for another finite lattice \(C\) and a monotone function
    \(\morph{f}{A}{C}\), let the \emph{monotone \(B\)-completion} of \(f\) be
    the function \(\morph{f_\mathsf{m}}{B}{C}\) recursively defined as \[
        f_\mathsf{m}(v) = \begin{cases}
            f(v)
             &
            \text{if}\ v \in A
            \\
            \bot_C
             &
            \text{if}\ v = \bot_B, \bot \not\in A
            \\
            \bigvee \{ f_\mathsf{m}(w) \,|\, w \leq_B v, w \neq v \}
             &
            \text{otherwise}
        \end{cases}
    \]
\end{definition}

\begin{example}
    For \(n \in \nat\), let \(\nat_{n}\) be the subset of the natural numbers
    containing the numbers \(0,1,\dots,n-1\) with the usual order.
    Let \(\morph{f}{\{2,4\}}{\nat}\) be defined as \(2 \mapsto 6\) and
    \(4 \mapsto 7\).
    The monotone \(\nat_5\)-completion of \(f\) is a function
    \(\morph{f_\mathsf{m}}{\nat_5}{\nat}\), defined as follows:
    \(f_\mathsf{m}(0) = 0\) as \(0\) is the least element of \(\nat_5\);
    \(f_\mathsf{m}(1) = 0\) as \(0 \leq 1\) and \(g_\mathsf{m}(1) = 0\);
    \(f_\mathsf{m}(2) = 6\) as \(2 \in \{2, 4\}\) and \(g(2) = 6\);
    \(f_\mathsf{m}(3) = 6\) because \(
    f_\mathsf{m}(3) =
    f_\mathsf{m}(0) \vee f_\mathsf{m}(1) \vee f_\mathsf{m}(2)
    = 0 \vee 0 \vee 6 = 6
    \); and \(f_\mathsf{m}(4) = 7\) because \(f(4) = 7\).
\end{example}

A Mealy function can now be encoded over powers of \(\values\) by using the
monotone completion of some encoding function.

\begin{definition}[Monotone Mealy encoding]\label{def:mealy-encoding}
    \index{monotone!Mealy!encoding}
    For a monotone Mealy machine \((S, f, s_0)\) with \(k\) states and an
    encoding \(\morph{\gamma_\leq}{S}{\valuetuple{k}}\), let
    \(\morph{\gamma^p_\leq}{\gamma_\leq[S] \times \valuetuple{m}}{\valuetuple{k} \times \valuetuple{n}}\)
    be defined as the function \(
    (\gamma_\leq(s), \listvar{x}) \mapsto
    (\gamma_\leq(\mealyfunctiontransition{f}(s, \listvar{x})),
    \mealyfunctionoutput{f}(s, \listvar{x}))
    \).
    The \emph{monotone Mealy encoding} of \((S, f, s_0)\) is a function
    \(
    \morph{
        \gamma_\leq(f)
    }{
        \valuetuple{k} \times \valuetuple{m}
    }{
        \valuetuple{k} \times \valuetuple{n}
    }
    \) defined as the \((\valuetuple{k} \times \valuetuple{m})\)-completion of
    \(\gamma_\leq^p\).
\end{definition}

To obtain the syntactic circuit for a monotone Mealy function encoded
in this way, it needs to be a morphism in \(\funci\).
It is monotone by definition, but we need to make sure it is also
\(\bot\)-preserving.

\begin{lemma}
    A monotone Mealy encoding is in \(\funci\).
\end{lemma}
\begin{proof}
    A Mealy encoding is monotone as it is a monotone completion.
    There cannot be a state \(\bot^k\) since at least one bit must
    be \(\top\); this means the monotone completion will send the input
    \((\bot^k, \bot^m)\) to \((\bot^k, \bot^n)\): it is
    \(\bot\)-preserving.
\end{proof}

The foundations are now set for establishing the image of a PROP morphism from
Mealy machines to circuit terms.
There is one more thing to consider: \cref{def:encoding} depends on some
arbitrary total ordering on the states in a given monotone Mealy machine.
While this may not seem much of an issue, when
defining a PROP morphism this must be \emph{fixed}, otherwise the mapping of
Mealy machines to circuits might be nondeterministic.

\begin{definition}[Chosen state order]
    \index{chosen state order}
    Let \((S, f, s_0)\) be a monotone Mealy machine with input space
    \(\valuetuple{m}\), and let \(\leq\) be a total order on \(\values\);
    \(\leq\) can be extended to a total order \(\leq_\star\) on
    \(\freemon{(\values^m)}\) using the lexicographic order.
    Let \(S^\prime\) be the set of accessible states of \(S\).
    For each state \(s \in S^\prime\), let
    \(t_{s,\leq} \in \freemon{(\values^m)}\) be the minimal element of the
    subset of words that transition from \(s_0\) to \(s\), ordered by \(\leq\).
    Then the \emph{chosen state order} \(\leq_{S^\prime}\) is a total order on
    \(S^\prime\) defined as \(s \leq_{S^\prime} s^\prime\) if
    \(t_{s,\leq} \leq_\star t_{s^\prime,\leq}\).
\end{definition}

The PROP morphism from monotone Mealy machines to circuits can then be
parameterised by some ordering on the set of values \(\values\), ensuring that
there is a canonical term in \(\scircsigma\) for each monotone Mealy machine.

\begin{definition}\label{def:mealy-to-circuit}
    \nomenclature{\(\mealytocircuiti\)}{PROP morphism \(\mealyi \to \scircsigma\)}
    For a functionally complete interpretation \(\interpretation\) and total
    order \(\leq\) on \(\values\), let \(
    \morph{
        \mealytocircuiti
    }{
        \mealyi
    }{
        \scircsigma
    }
    \) be the traced PROP morphism with action defined for a monotone Mealy
    machine \((S,f,s)\) as producing \(
    \iltikzfig{circuits/synthesis/mealy-term-spaced}
    \).
\end{definition}

Before proceeding to the result that this PROP morphism is behaviour-preserving,
we must show a lemma linking the behaviour of circuits in the image of
\(\mealytocircuiti\) to initial outputs and stream derivatives.

\begin{proposition}
    \label{prop:mealy-form-image}
    For a combinational circuit \(
    \iltikzfig{strings/category/f-2-2}[box=f,dom1=x,dom2=m,cod1=x,cod2=n,colour=comb]
    \), let \(\mathsf{mf}(f)\) be the map with action \(
    (\listvar{s}) \mapsto
    \circuittostreami[
        \iltikzfig{circuits/productivity/mealy-form}[core=f,state=\listvar{s},dom=m,cod=n,delay=x]
    ]
    \) and let \(
    g
    :=
    \circuittofunci[
        \iltikzfig{strings/category/f-2-2}[box=f,colour=comb]
    ]
    \).
    Then, \(
    \mealyoutput{\mathsf{mf}(f)(\listvar{s})}{\listvar{a}}
    =
    \proj{1}(g(\listvar{s}, \listvar{a}))
    \) and \(
    \mealytransition{\mathsf{mf}(f)(\listvar{s})}{\listvar{a}}
    =
    \mathsf{mf}(f)(\proj{0}(g(\listvar{s}, \listvar{a})))
    \).
\end{proposition}
\begin{proof}
    The machine \(\circuittomealyi[
        \iltikzfig{circuits/productivity/mealy-form}[core=f,state=\listvar{s},dom=m,cod=n,delay=x]
    ]\) is computed as the fixed point of the machine \(
    \left(\valuetuple{x},
    (\listvar{r}, \listvar{a}) \mapsto \left\langle
    \listvar{r}, g(\listvar{s}, \listvar{a})
    \right\rangle, \listvar{s}\right)
    \), which is \(
    \left(\valuetuple{x},
    \listvar{v} \mapsto \left\langle
    \proj{0}\mleft(g(\listvar{s},\listvar{a})\mright),
    \proj{1}\mleft(g(\listvar{s}, \listvar{a})\mright)
    \right\rangle, \listvar{s}\right)
    \).
    The output and derivative of \(
    \mealytocircuiti[
        \circuittomealyi[
            \iltikzfig{circuits/productivity/mealy-form}[core=f,state=\listvar{s},dom=m,cod=n,delay=x]
        ]
    ]
    \) are the output and transition of the Mealy machine, so the original
    statement holds by \cref{lem:sequential-combinational-semantics}.
\end{proof}

The goal of this section is to show that the translation from Mealy machines to
circuits and back again using \(\circuittomealyi \circ \mealytocircuiti\) is
\emph{behaviour-preserving}: while the mapping may not be the identity in
\(\mealyi\), the stream functions obtained using \(\streamtomealyi\) should
be equal.
This is an important new result, as it means that rather than showing
results about the denotational semantics of circuits in \(\scircsigma\) by
interpreting them in \(\streami\), we can view morphisms of the former as
Mealy machines instead.

\begin{theorem}\label{thm:mealy-to-circuit}
    \(
    \mealytostream = \circuittostreami \circ \mealytocircuiti
    \).
\end{theorem}
\begin{proof}
    Let \((S ,f)\) be a monotone Mealy machine and let \(s \in S\) be an arbitrary
    state.
    By \cref{prop:mealy-to-stream}, the initial output of
    \(\mealytostreami[(S, f, s)]\) is
    \(\listvar{a} \mapsto \mealyfunctionoutput{f}\mleft(s, \listvar{a}\mright)\)
    and the stream derivative of \(\mealytostreami[(S, f, s)]\) is \(
    \listvar{a}
    \mapsto
    \mealytostreami[(\mealyfunctiontransition{f}\mleft(s, \listvar{a}\mright))]
    \).

    Now we consider the composite
    \(\circuittostreami[\mealytocircuiti[(S, f, s)]]\).
    By \cref{def:mealy-to-circuit} we have that \(
    \mealytocircuiti[(S, f, s_0)]
    =
    \iltikzfig{circuits/synthesis/mealy-term-spaced}
    \); by applying \(\mealytocircuiti\) and \(\mealytofunc\),
    there exists a combinational circuit \(
    \iltikzfig{strings/category/f-3-2}[box=g,colour=comb]
    \) such that \[
        \mealytocircuiti[(S, f, s_0)]
        =
        \iltikzfig{circuits/synthesis/mealy-term-spaced}
        =
        \iltikzfig{circuits/synthesis/mealy-term-comb-core}.
    \]
    Let \(
    g^\prime
    =
    \circuittofunci[\iltikzfig{strings/category/f-3-2}[box=g,colour=comb]]
    \); note that for all \(\listvar{r} \in \valuetuple{x}\) and
    \(\listvar{a} \in \valuetuple{m}\),
    \(g^\prime(\listvar{v}, \listvar{r}, \listvar{a})
    =
    \gamma_\leq(f)(\listvar{r},\listvar{a})
    \).

    We can now use \cref{prop:mealy-form-image} to compute the initial output
    and stream derivative of \(\circuittostreami[\mealytocircuiti[(S, f, s)]]\).
    To show that \(\mealytostreami = \circuittostreami \circ \mealytostreami\),
    we need to show that these `agree' with those of
    \(\mealytostreami[(S, f, s)]\).
    For the initial output, this means we just need to show they are equal:
    \begin{align*}
        \initialoutput{\circuittostreami[\mealytocircuiti[(S, f, s)]]}{\listvar{a}}
         & =
        \proj{1}\mleft(\circuittofunci[g](\listvar{v}, \gamma_\leq(s), \listvar{a})\mright)
        \\
         & =
        \proj{1}\mleft(\circuittofunci[\mealytofunc[\gamma_\leq(f)]](\gamma_\leq(s), \listvar{a})\mright)
        \\
         & =
        \proj{1}\mleft(\gamma_\leq(f)(\gamma_\leq(s), \listvar(a))\mright)
        \\
         & =
        \proj{1}\left(
        \gamma_\leq(
            \mealyfunctiontransition{f}(s, \listvar{a})),
        \mealyfunctionoutput{f}(s, \listvar{a})\right)
        \\
         & =
        \mealyfunctionoutput{f}(s, \listvar{a})
    \end{align*}
    For the stream derivative, we need to show that as states vary over
    \(s \in S\), the stream derivative of \(
    \circuittostreami[\mealytocircuiti[(S, f, s)]]
    \) is the \(\gamma_\leq\)-encoding of \(\mealytostreami[(S, f, s)]\).
    \begin{align*}
        \streamderivative{\left(\circuittostreami[\mealytocircuiti[(S, f, s)]]\right)}{\listvar{a}}
         & =
        \proj{0}\left(
        \circuittofunci[g](\listvar{v}, \gamma_\leq(s), \listvar{a})
        \right)
        \\
         & =
        \proj{0}\mleft(\circuittofunci[\mealytofunc[\gamma_\leq(f)]](\gamma_\leq(s), \listvar{a})\mright)
        \\
         & =
        \proj{0}\mleft(\gamma_\leq(f)(\gamma_\leq(s), \listvar(a))\mright)
        \\
         & =
        \proj{0}\left(
        \gamma_\leq(
            \mealyfunctiontransition{f}(s, \listvar{a})),
        \mealyfunctionoutput{f}(s, \listvar{a})\right)
        \\
         & =
        \gamma_\leq(
        \mealyfunctiontransition{f}(s, \listvar{a}))
    \end{align*}
    The initial outputs and stream derivatives agree, so
    \(\mealytostream = \circuittostreami \circ \mealytocircuiti\).
\end{proof}

On its own, this is a nice result to have; if we only know the specification of
a circuit in terms of a (monotone) Mealy machine, we can use the PROP morphism
\(\mealytocircuiti\) to generate a circuit in \(\scircsigma\) which has the
same behaviour as a stream function.
However, this is but one ingredient in our ultimate goal: the completeness of
the denotational semantics.
\section{Completeness of the denotational semantics}

We want \(\streami\) to be a \emph{complete} denotational semantics for digital
circuits.
This means that for every stream function \(f \in \streami\), there must be at
least one one circuit in \(\scircsigma\) such that its behaviour under
\(\interpretation\) is \(f\).

\begin{corollary}\label{thm:circuit-stream-correspondence}
    \(
    \circuittostreami
    \circ
    \mealytocircuiti
    \circ
    \streamtomealyi
    =
    \id[\streami]
    \).
\end{corollary}
\begin{proof}
    This follows immediately from \cref{thm:mealy-to-circuit} and
    \cref{prop:mealy-to-stream}, as we have that \(
    \circuittostreami
    \circ
    \mealytocircuiti
    \circ
    \streamtomealyi
    =
    \mealytostreami
    \circ
    \streamtomealyi
    =
    \id[\streami]
    \).
\end{proof}

There is no isomorphism between \(\scircsigma\) and \(\streami\)
as many circuits may have the same semantics but different syntax.

\begin{definition}[Denotational equivalence]
    Two sequential circuits are \emph{denotationally equivalent}
    under \(\interpretation\), written \(
    \iltikzfig{strings/category/f}[box=f,colour=seq,dom=m,cod=n]
    \approx_{\interpretation}
    \iltikzfig{strings/category/f}[box=g,colour=seq,dom=m,cod=n]
    \) if \(
    \circuittostream[
        \iltikzfig{strings/category/f}[box=f,colour=seq]
    ]{\interpretation}
    =
    \circuittostream[
        \iltikzfig{strings/category/f}[box=g,colour=seq]
    ]{\interpretation}
    \).
    Let \(\scircsigmai\) be the result of quotienting \(\scircsigma\) by \(
    \approx_{\interpretation}
    \).
\end{definition}

Every morphism in \(\scircsigmai\) is a class of circuits which all have the
same behaviour under \(\interpretation\).
Since each of these behaviours is a stream function in \(\streami\), we can
conclude the following.

\begin{corollary}
    \(\scircsigmai \cong \streami\).
\end{corollary}
\section{Denotational semantics for Belnap logic}\label{sec:denotational-belnap}

For an interpretation to admit a sound and complete denotational semantics it
needs to be \emph{functionally complete}.
One may wonder if this is a reasonable assumption to make, as if it is not then
the denotational semantics is not particularly useful.

To this end, we will demonstrate how the functional completeness condition holds
for the Belnap interpretation introduced in \cref{ex:belnap-interpretation}.
This will make use of the well-known functional completeness of Boolean logic.

\begin{definition}
    Let \(\booleans \coloneqq \{0, 1\}\) be the set of \emph{Boolean} values,
    and let \(\band,\bor,\bnot\) be the usual operations on Booleans.
\end{definition}

\begin{lemma}\label{lem:boolean-complete}
    All functions \(\booleans^{m} \to \booleans\) can be expressed using
    the set of operations \(\{0, 1, \band,\bor,\bnot\}\).
\end{lemma}
\begin{proof}
    Let \(\morph{f}{\booleans^{m}}{\booleans}\) be a Boolean function: we need
    to create a Boolean expression using variables \(v_0, v_1, \dots, v_{m-1}\).
    For each \(\listvar{v} \in \booleans^m\), we construct a conjunction of
    all \(m\) variables, in which \(v_i\) is negated if \(\listvar{v}(i) = 0\).
    We can then define a disjunction of the conjunctions for words
    \(\listvar{v}\) such that \(f(\listvar{v}) = 1\).
    If there are no such words, then the expression is \(0\).
    It is simple to check that this expression is equivalent to the original
    function.
\end{proof}

We use of this result by `simulating' the Boolean operations in the
Belnap realm.

\begin{lemma}
    There is an isomorphism \(\values \cong \booleans^2\).
\end{lemma}
\begin{proof}
    There are several mappings one could choose, but for this
    section we will use \(\phi \coloneqq
    \bot \mapsto 00, \belnapfalse \mapsto 10,
    \belnaptrue \mapsto 01, \top \mapsto 11
    \).
\end{proof}

The Belnap values \(\belnapfalse\) and \(\top\) are \emph{falsy}; they contain
false information.
Similarly, the Belnap values \(\belnaptrue\) and \(\top\) are \emph{truthy}:
they contain true information.
The value \(\bot\) is neither falsy nor truthy.
This is reflected in the mapping shown above; \(\phi(v)(0)\) is \(1\) if and
only if \(v\) is falsy, and \(\phi(v)(1)\) is \(1\) if and only if \(v\) is
truthy.
We write \(\phi_0(v) \coloneqq \phi(v)(0)\) and \(\phi_1(v) \coloneqq \phi(v)(1)\).

This means that rather than trying to divine an expression directly
from a Belnap function, we can instead define \emph{two} functions; one for how
falsy the output is, and one for how truthy is.

\begin{definition}
    Let \(\values_0 \coloneqq \{\bot, \belnapfalse\}\) and let
    \(\values_1 \coloneqq \{\bot, \belnaptrue\}\).
\end{definition}

For \(v \in \values_0\), \(\phi(v)(1) = 0\) (all the information is contained
within the falsy bit), and for \(v^\prime \in \values_1\), \(\phi(v)(0) = 0\)
(all the information is contained in the truthy bit).
These sets are of particular interest when comparing to Boolean operations.

\begin{lemma}
    \(\values_0\) and \(\values_1\) are closed under \(\land\) and \(\lor\).
\end{lemma}
\begin{proof}
    This can be verified by inspecting the truth tables:

    \vspace{0.5em}

    \begin{center}
        \begin{tabular}{c|cc}
            \(\land\)        & \(\bot\)         & \(\belnapfalse\) \\
            \hline
            \(\bot\)         & \(\bot\)         & \(\belnapfalse\) \\
            \(\belnapfalse\) & \(\belnapfalse\) & \(\belnapfalse\)
        \end{tabular}
        \quad
        \begin{tabular}{c|cc}
            \(\lor\)         & \(\bot\) & \(\belnapfalse\) \\
            \hline
            \(\bot\)         & \(\bot\) & \(\bot\)         \\
            \(\belnapfalse\) & \(\bot\) & \(\belnapfalse\)
        \end{tabular}
        \quad
        \begin{tabular}{c|cc}
            \(\land\)       & \(\bot\) & \(\belnaptrue\) \\
            \hline
            \(\bot\)        & \(\bot\) & \(\bot\)        \\
            \(\belnaptrue\) & \(\bot\) & \(\belnaptrue\)
        \end{tabular}
        \quad
        \begin{tabular}{c|cc}
            \(\lor\)        & \(\bot\)        & \(\belnaptrue\) \\
            \hline
            \(\bot\)        & \(\bot\)        & \(\belnaptrue\) \\
            \(\belnaptrue\) & \(\belnaptrue\) & \(\belnaptrue\)
        \end{tabular}
    \end{center}

    \qedhere
\end{proof}

If one looks closer, these are just the truth tables for \(\bor\) and \(\band\)
but with different symbols.
This means that any expression we make using \(\band\) and \(\bor\) in the
Boolean realm can be `simulated' in the falsy and truthy Belnap subsets.
Formally, we have the following.

\begin{lemma}\label{lem:belnap-bool-correspondence}
    The following diagrams commute:
    \begin{center}
        \begin{tikzcd}
            (\values_0)^2
            \arrow{r}{\land}
            \arrow[swap]{d}{(\phi_0, \phi_0)}
            &
            \values_0
            \arrow{d}{\phi}
            \\
            \booleans^2
            \arrow[swap]{r}{\bor}
            &
            \booleans
        \end{tikzcd}
        \quad
        \begin{tikzcd}
            (\values_0)^2
            \arrow{r}{\lor}
            \arrow[swap]{d}{(\phi_0, \phi_0)}
            &
            \values_0
            \arrow{d}{\phi}
            \\
            \booleans^2
            \arrow[swap]{r}{\band}
            &
            \booleans
        \end{tikzcd}

        \hspace{0.3em}
        \begin{tikzcd}
            (\values_1)^2
            \arrow{r}{\land}
            \arrow[swap]{d}{(\phi_1, \phi_1)}
            &
            \values_1
            \arrow{d}{\phi_0}
            \\
            \booleans^2
            \arrow[swap]{r}{\band}
            &
            \booleans
        \end{tikzcd}
        \quad
        \begin{tikzcd}
            (\values_1)^2
            \arrow{r}{\lor}
            \arrow[swap]{d}{(\phi_1, \phi_1)}
            &
            \values_1
            \arrow{d}{\phi_1}
            \\
            \booleans^2
            \arrow[swap]{r}{\bor}
            &
            \booleans
        \end{tikzcd}
    \end{center}
\end{lemma}
\begin{proof}
    By testing the four values in each case.
\end{proof}

The eager reader might now assume we can proceed as with
\cref{lem:boolean-complete},
constructing a disjunctive normal form from truth tables, but we replace Boolean
operations with the appropriate Belnap ones.
The more cautious reader may have noticed we have not discussed how the
Boolean \(\neg_\booleans\) can be simulated using Belnap operations.
This is because it is simply not possible to do this while remaining in the
two Belnap subsets.
We must make use of a certain subset of Boolean functions that can be
constructed \emph{without} using \(\neg_\booleans\).

\begin{definition}
    Let the total order \(\boolorder\) be defined as \(0 \leq 1\).
\end{definition}

As with \(\values\), \(\booleans^m\) inherits the order on \(\booleans\)
pointwise.
Subsequently, a Boolean function \(\morph{f}{\booleans^m}{\booleans}\) is
\emph{monotone} if \(f(\listvar{v}) \boolorder f(\listvar{w})\) whenever
\(\overline{v} \boolorder \overline{w}\).
Intuitively, flipping an input bit from \(0\) to \(1\) can never flip an output
bit from \(1\) to \(0\).

\begin{lemma}
    \label{lem:boolean-complete-monotone}
    All monotone functions \(\booleans^m \to \booleans\) can be expressed with
    the set of operations \(\{\land,\lor,1\}\).
\end{lemma}
\begin{proof}
    This progresses as with \cref{lem:boolean-complete}, but if the element of
    a word \(\listvar{v}(i) = 0\), it is omitted from the conjunction
    rather than the variable being negated.

    To show that this expresses the same truth table as the original function,
    consider an omitted variable \(v_i\); there exists an assignment of the
    other variables such that if \(v_i = 0\) then \(f(\dots, v_i, \dots) = 1\).
    By monotonicity, it must be the case that if \(v_i = 1\) then
    \(f(\dots, v_i, \dots) = 1\), so no information is lost by omitting
    the negation.

    If \(f(0, 0, \dots, 0) = 1\), then the inner conjunction is empty and must
    be represented by the constant \(1\), (the unit of \(\band\)).
    This is valid due to monotonicity, as if \(f\) produces \(1\) for the
    infimum, then it must produce \(1\) for all inputs.
\end{proof}

\begin{corollary}\label{cor:func-complete-falsy-truthy}
    All monotone functions \((\values_0)^m \to \values_0\) can be expressed
    with the operations \(\{\land,\lor,\belnapfalse\}\), and all monotone
    functions \((\values_1)^m \to \values_1\) can be expressed with the
    operations \(\{\land,\lor,\belnaptrue\}\).
\end{corollary}
\begin{proof}
    As there is an order isomorphism
    \(\values_0 \cong \values_1 \cong \booleans\), any monotone function in
    the Belnap subsets can be viewed as a monotone Boolean function.
    This means the strategy of \cref{lem:boolean-complete-monotone} can
    be applied using \cref{lem:belnap-bool-correspondence} to substitute the
    appropriate Belnap operation.
\end{proof}

All the pieces are now in place to express the final functional completeness
result; we just need to `explode' a Belnap value into its falsy and truthy
components, and then unify the two at the end.

\begin{definition}
    \label{def:translation-tables}
    Let the functions \(\morph{\psi^0_0,\psi^1_0}{\values}{\values_0}\) and
    \(\morph{\psi^0_1,\psi^1_1}{\values}{\values_1}\) be defined according to
    the table below.
    \begin{center}
        \begin{tabular}{c|cccc}
                             & \(\psi^0_0\)     & \(\psi^1_0\)     & \(\psi^0_1\)    & \(\psi^1_1\)    \\
            \hline
            \(\bot\)         & \(\bot\)         & \(\bot\)         & \(\bot\)        & \(\bot\)        \\
            \(\belnaptrue\)  & \(\bot\)         & \(\belnapfalse\) & \(\bot\)        & \(\belnaptrue\) \\
            \(\belnapfalse\) & \(\belnapfalse\) & \(\bot\)         & \(\belnaptrue\) & \(\bot\)        \\
            \(\top\)         & \(\belnapfalse\) & \(\belnapfalse\) & \(\belnaptrue\) & \(\belnaptrue\) \\
        \end{tabular}
    \end{center}
\end{definition}

The functions \(\psi^0_0\) and \(\psi^0_1\) send a value \(v\) to
\(\belnapfalse\) or \(\belnaptrue\) respectively if \(v\) is falsy;
\(\psi^1_0\) and \(\psi^1_1\) send a value \(v\) to
\(\belnapfalse\) or \(\belnaptrue\) if \(v\) is truthy.
Otherwise, they produce \(\bot\).

\begin{lemma}
    \label{lem:translation-functions}
    The functions in \cref{def:translation-tables} can be expressed using the
    operations \(\{\land,\lor,\neg,\bot\}\).
\end{lemma}
\begin{proof}
    From left to right, the columns in the table above are the functions \(
    v \mapsto - \land \bot
    \), \(
    v \mapsto \neg(- \lor \bot)
    \), \(
    v \mapsto \neg(- \land \bot)
    \) and \(
    v \mapsto - \lor \bot
    \).
\end{proof}

\begin{definition}\label{def:falsy-truthy-functions}
    For a monotone function \(\morph{f}{\valuetuple{m}}{\values}\), let
    \(\morph{f_0}{((\values_0)^m)^2}{\values_0}\) be defined as
    \(
    f_0(\psi^0_0(\listvar{v}), \psi^1_0(\listvar{v}))
    \coloneqq
    \phi_0(f(\listvar{v}))
    \).
    Similarly, let \(\morph{f_1}{((\values_1)^m)^2}{\values_1}\) be defined as
    \(
    f_1(\psi^0_1(\listvar{v}), \psi^1_1(\listvar{v}))
    \coloneqq
    \phi_1(f(\listvar{v}))
    \).
\end{definition}

By putting these pieces all together we can express all monotone Belnap
functions.

\begin{theorem}
    \label{thm:belnap-complete}
    All monotone functions \(\valuetuple{m} \to \values\) can be expressed
    using the operations \(
    \{\land,\lor,\neg,\ljoin,\bot,\belnaptrue,\belnapfalse\}
    \).
\end{theorem}
\begin{proof}
    This follows by defining a function with the same behaviour as the original,
    but made up of components known to be expressible using the operations
    specified.

    Let \(\morph{f^\prime}{\valuetuple{m}}{\valuetuple{2}}\) be defined as
    \(
    f^\prime(\listvar{v})
    \coloneqq
    \left(
    f_0(\psi^0_0(\listvar{v}), \psi^1_0(\listvar{v})),
    f_1(\psi^0_1(\listvar{v}), \psi^1_1(\listvar{v}))
    \right).
    \).
    By \cref{cor:func-complete-falsy-truthy}, \(f_0\) and \(f_1\) can be defined
    using \(\{\land,\lor,\belnaptrue,\belnapfalse\}\), and by
    \cref{lem:translation-functions}, \(\psi^0_0,\psi^1_0,\psi^0_1\) and
    \(\psi^1_1\) can be defined using \(\{\land,\lor,\bot\}\).

    The output of \(f^\prime(\listvar{v})\) is \(
    \left(\phi_0(f(\listvar{v})), \phi_1(f(\listvar{v}))\right)
    \) by definition; the falsiness and the truthiness of \(f(\listvar{v})\).
    To combine the two outputs into a single output we want to
    implement the following truth table:

    \begin{center}
        \begin{tabular}{cc|c}
            \(\bot\)         & \(\bot\)        & \(\bot\)         \\
            \(\bot\)         & \(\belnaptrue\) & \(\belnaptrue\)  \\
            \(\belnapfalse\) & \(\bot\)        & \(\belnapfalse\) \\
            \(\belnapfalse\) & \(\belnaptrue\) & \(\top\)
        \end{tabular}
    \end{center}

    This is clearly just the truth table for \(\ljoin\), so the entire
    expression can be defined using the operations \(
    \{\land,\lor,\neg,\ljoin,\bot,\belnaptrue,\belnapfalse\}
    \).
\end{proof}

\begin{example}
    Consider the following truth table (in fact just the table for \(\neg\)).
    \begin{center}
        \begin{tabular}{c|c}
            \(\bot\)         & \(\bot\)         \\
            \(\belnaptrue\)  & \(\belnapfalse\) \\
            \(\belnapfalse\) & \(\belnaptrue\)  \\
            \(\top\)         & \(\top\)
        \end{tabular}
    \end{center}
    We translate these into the falsy and truthy tables as follows:
    \begin{center}
        \begin{tabular}{c|c}
            \(\bot\bot\)                 & \(\bot\)         \\
            \(\bot\belnapfalse\)         & \(\belnapfalse\) \\
            \(\belnapfalse\bot\)         & \(\bot\)         \\
            \(\belnapfalse\belnapfalse\) & \(\belnapfalse\)
        \end{tabular}
        \begin{tabular}{c|c}
            \(\bot\bot\)               & \(\bot\)        \\
            \(\bot\belnaptrue\)        & \(\bot\)        \\
            \(\belnaptrue\bot\)        & \(\belnaptrue\) \\
            \(\belnaptrue\belnaptrue\) & \(\belnaptrue\)
        \end{tabular}
    \end{center}
    Using \cref{cor:func-complete-falsy-truthy}, the corresponding Belnap
    expressions are \[
        (v_0,v_1) \mapsto v_1 \land (v_0 \lor v_1)
        \qquad
        (v_0,v_1) \mapsto v_0 \lor (v_0 \land v_1)
    \]
    To combine these expressions on two inputs into a single expression on one
    input, we need to add the appropriate translators.
    We obtain the expression \[
        v_0 \mapsto
        \neg(\bot \lor v_0)
        \land
        \left((\bot \land v_0) \lor \neg(\bot \lor v_0)\right)
        \ljoin
        \neg(\bot \land v_0)
        \lor
        (\neg(\bot \land v_0) \land (\bot \lor v_0))
    \]
    which can be verified to act the same as the original table.
\end{example}

Although this result only applies to functions with a single output, it is
easily generalised to arbitrary-output functions.

\begin{corollary}
    All monotone functions \(\valuetuple{m+1} \to \valuetuple{n}\) can be
    expressed using the operations \(\{\land,\lor,\neg,\bot,\ljoin\}\).
\end{corollary}
\begin{proof}
    By repeating the process in \cref{thm:belnap-complete} for each output.
\end{proof}

Since an expression can be  using only operations with counterparts in
the syntactic realm, the map \(
\morph{\mealytofunc_\belnap}{\func{\interpretation_\belnap}}{\scirc{\Sigma_\belnap}}
\) sends functions \(f\) to circuits of the form \[
    \iltikzfig{circuits/axioms/belnap/translation/exploded-form}[box=f]
\] in which \(
\iltikzfig{strings/category/f}[box=f_0,colour=comb]
\) and \(
\iltikzfig{strings/category/f}[box=f_1,colour=comb]
\) are `syntactic' falsy and truthy disjunctive normal forms respectively.
While the truthy circuit is just a regular disjunctive normal form, because
the falsy operations are simulated by the opposite gate, it looks a bit
different.

\begin{definition}[Conjunction]
    A Belnap circuit is a \emph{truthy conjunction} if it is the infinite
    register \(
    \iltikzfig{circuits/components/waveforms/infinite-register}[val=\belnaptrue]
    \) or of the form \(
    \iltikzfig{circuits/algebraic/conjunction}[ccolour=seq]
    \), where \(
    \iltikzfig{strings/category/f}[box=f, colour=seq]
    \) is another truthy conjunction.
    A Belnap circuit is a \emph{falsy conjunction} if it is the infinite
    register \(
    \iltikzfig{circuits/components/waveforms/infinite-register}[val=\belnapfalse]
    \) or of the form \(
    \iltikzfig{circuits/algebraic/disjunction}[dcolour=seq]
    \), where \(
    \iltikzfig{strings/category/f}[box=f, colour=seq]
    \) is another falsy conjunction.
\end{definition}

\begin{definition}[Disjunctive normal form]
    A Belnap circuit is in \emph{truthy disjunctive normal form} if it is the
    eliminator \(
    \iltikzfig{strings/structure/monoid/init}[colour=comb]
    \) or of the form \(
    \iltikzfig{circuits/algebraic/disjunctive-normal}[ccolour=seq]
    \), where \(
    \iltikzfig{strings/category/f}[box=f, colour=seq]
    \) is in truthy disjunctive normal form and \(
    \iltikzfig{strings/category/f}[box=g, colour=seq]
    \) is a truthy conjunction.
    A Belnap circuit is in \emph{falsy disjunctive normal form} if it is the
    eliminator \(
    \iltikzfig{strings/structure/monoid/init}[colour=comb]
    \) or of the form \(
    \iltikzfig{circuits/algebraic/conjunctive-normal}[dcolour=seq]
    \), where \(
    \iltikzfig{strings/category/f}[box=f, colour=seq]
    \) is in falsy disjunctive normal form and \(
    \iltikzfig{strings/category/f}[box=g, colour=seq]
    \) is a falsy conjunction.
\end{definition}

The two subcircuits are falsy and truthy disjunctive normal forms, and can be
defined syntactically by using a `composite fork' to copy the inputs for each
clause in the normal form.

\begin{definition}\label{def:mk-fork}
    For \(n \in \nat\), an \emph{\(m,k\)-fork}
    \(\iltikzfig{circuits/components/structural/mn-fork}\)
    is defined
    inductively with \(
    \iltikzfig{circuits/components/structural/m0-fork} \coloneqq
    \iltikzfig{strings/category/identity-blank}[obj=m]
    \) and \(
    \iltikzfig{circuits/components/structural/mkp1-fork} \coloneqq
    \iltikzfig{circuits/components/structural/n-fork}
    \).
\end{definition}

\begin{definition}
    Let \(\morph{\mealytofunc_\belnap}{\func{\belnapinterpretation}}{\ccirc{\belnapsignature}}\)
    be defined as the map sending a function \(\morph{f}{\valuetuple{m}}{\valuetuple{n}}\)
    to a circuit \[
        \iltikzfig{circuits/axioms/belnap/translation/exploded-form-cnf}[dom=m,cod=n,box=f]
    \] where \(
    \iltikzfig{strings/category/f}[box=g_0,colour=comb]
    \) and \(
    \iltikzfig{strings/category/f}[box=g_1,colour=comb]
    \) only contain identity and elimination constructs, and \(
    \iltikzfig{strings/category/f}[box=h_0,colour=seq]
    \) and \(
    \iltikzfig{strings/category/f}[box=h_1,colour=seq]
    \) and in falsy and truthy conjunctive normal form respectively, defined
    in the obvious way derived from the procedure detailed in this section.
\end{definition}

This means that the denotational semantics for sequential circuits can
definitely be used for the Belnap interpretation \(\belnapinterpretation\).
In particular, this means we can translate any Mealy machine in
\(\mealy{\belnapinterpretation}\) (and subsequently, any stream function in
\(\stream{\belnapinterpretation})\) into a syntactic circuit in
\(\scirc{\belnapsignature}\).

While the signature containing \(\andgate\), \(\orgate\) and \(\notgate\) gates
is common, it is by no means the only useful one.
For example, it is well known that the \(\nandgate\) gate on its own is also
functionally complete.
Since the other logic gates can be expressed in terms of \(\andgate\),
\(\orgate\) and \(\notgate\) gates, the functional completeness result for
\(\belnapinterpretation\) can be adapted for these other gate sets.

Even when applying the above techniques to small concrete examples,
the results quickly balloon in size and it is not feasible nor desirable to work
them out by hand.
A tool has been developed to generate Belnap expressions and
visualise their circuits automatically, and it can be found at
\url{https://github.com/georgejkaye/catcircs}.
\section{Denotational semantics for generalised circuits}

Although we have discussed the denotation semantics in terms of monochromatic
circuit signatures, it is straightforward to extend the results to categories
generated over \emph{generalised} circuit signatures

In the semantic categories \(\funci\), \(\streami\), and \(\mealyi\), the
morphisms are all variants on functions of the form
\(\valuetuple{m} \to \valuetuple{n}\) that operate on powers of elements in
\(\values\): one element for each (single-bit) input or output wire.
In the generalised setting, these input and output wires may not all be the
same width, so the input and output sets must be \emph{powers of powers} of
values!

\begin{notation}
    Given a set \(A\) and a word \(\listvar{v} \in \natplus^\star\) of
    length \(n\), we write \(
    A^{\listvar{v}}
    \coloneqq
    A^{\listvar{v}(0)}
    \times
    A^{\listvar{v}(1)}
    \times
    \dots
    \times
    A^{\listvar{v}(n-1)}
    \).
\end{notation}

Note that for a word \(\listvar{m} \coloneqq 11\dots1\) of length \(k\), the set
\(A^{\listvar{m}} = A^1 \times A^1 \times \dots A^1\) is isomorphic to \(A^k\),
much like how setting the set of colours in a coloured PROP to the singleton
recovers a monochromatic PROP.

The semantic categories can now be extended to these \emph{coloured} interfaces.

\begin{definition}
    Let \(\funcig\) be the \(\natplus\)-coloured PROP in which the morphisms
    \(\listvar{m} \to \listvar{n}\) are the monotone functions
    \(\valuetuple{\listvar{m}} \to \valuetuple{\listvar{n}}\).
    Let \(\streamig\) be the \(\natplus\)-coloured PROP in which the morphisms
    \(\listvar{m} \to \listvar{n}\) are the causal, monotone, and finitely
    specified stream functions \(
    \valuetuplestream{\listvar{m}} \to \valuetuplestream{\listvar{n}}
    \).
    Let \(\mealyig\) be the \(\natplus\)-coloured PROP in which the moprhisms
    \(\listvar{m} \to \listvar{n}\) are the monotone
    \((\valuetuple{\listvar{m}}, \valuetuple{\listvar{n}})\)-Mealy machines.
\end{definition}


\begin{definition}
    Let the coloured PROP morphisms
    \[\arraycolsep=2em\def\arraystretch{1.2}\begin{array}{cc}
            \morph{\circuittofuncig}{\ccircsigmag}{\funcig}
             &
            \morph{\circuittomealyig}{\scircsigmag}{\mealyig}
            \\
            \morph{\mealytostreamig}{\mealyig}{\streamig}
             &
            \morph{\streamtomealyig}{\streamig}{\mealyig}
            \\
            \morph{\mealytocircuitig}{\mealyig}{\scircsigmag}
             &
            \morph{\circuittostreamig}{\scircsigmag}{\streamig}
        \end{array}\]
    be defined in the same manner as the monochromatic versions.
\end{definition}

\begin{definition}

\end{definition}

\begin{definition}
    Two generalised sequential circuits are \emph{denotationally equivalent}
    under \(\interpretation\), written \(
    \iltikzfig{strings/category/f}[box=f,colour=seq,dom=\listvar{m},cod=\listvar{n}]
    \approx_{\interpretation}^+
    \iltikzfig{strings/category/f}[box=g,colour=seq,dom=\listvar{m},cod=\listvar{n}]
    \) if \(
    \circuittostreamig[
        \iltikzfig{strings/category/f}[box=f,colour=seq]
    ]
    =
    \circuittostreamig[
        \iltikzfig{strings/category/f}[box=g,colour=seq]
    ]
    \).
    Let \(\scircsigmaig\) be the result of quotienting \(\scircsigma\) by \(
    \approx_{\interpretation}^+
    \).
\end{definition}

\begin{corollary}
    \(\scircsigmaig \cong \streamig\).
\end{corollary}
