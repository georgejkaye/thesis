\section{Denotational semantics for Belnap logic}\label{sec:denotational-belnap}

For an interpretation to admit a sound and complete denotational semantics it
needs to be \emph{functionally complete}.
One may wonder if this is a reasonable assumption to make, as if it is not then
the denotational semantics is not particularly useful.

To this end, we will demonstrate how the functional completeness condition holds
for the Belnap interpretation introduced in \cref{ex:belnap-interpretation}.
This will make use of the well-known functional completeness of Boolean logic.

\begin{definition}
    Let \(\booleans \coloneqq \{0, 1\}\) be the set of \emph{Boolean} values,
    and let \(\band,\bor,\bnot\) be the usual operations on Booleans.
\end{definition}

\begin{lemma}\label{lem:boolean-complete}
    All functions \(\booleans^{m} \to \booleans\) can be expressed using
    the set of operations \(\{0, 1, \band,\bor,\bnot\}\).
\end{lemma}
\begin{proof}
    Let \(\morph{f}{\booleans^{m}}{\booleans}\) be a Boolean function: we need
    to create a Boolean expression using variables \(v_0, v_1, \dots, v_{m-1}\).
    For each \(\listvar{v} \in \booleans^m\), we construct a conjunction of
    all \(m\) variables, in which \(v_i\) is negated if \(\listvar{v}(i) = 0\).
    We can then define a disjunction of the conjunctions for words
    \(\listvar{v}\) such that \(f(\listvar{v}) = 1\).
    If there are no such words, then the expression is \(0\).
    It is simple to check that this expression is equivalent to the original
    function.
\end{proof}

We use of this result by `simulating' the Boolean operations in the
Belnap realm.

\begin{lemma}
    There is an isomorphism \(\values \cong \booleans^2\).
\end{lemma}
\begin{proof}
    There are several mappings one could choose, but for this
    section we will use \(\phi \coloneqq
    \bot \mapsto 00, \belnapfalse \mapsto 10,
    \belnaptrue \mapsto 01, \top \mapsto 11
    \).
\end{proof}

The Belnap values \(\belnapfalse\) and \(\top\) are \emph{falsy}; they contain
false information.
Similarly, the Belnap values \(\belnaptrue\) and \(\top\) are \emph{truthy}:
they contain true information.
The value \(\bot\) is neither falsy nor truthy.
This is reflected in the mapping shown above; \(\phi(v)(0)\) is \(1\) if and
only if \(v\) is falsy, and \(\phi(v)(1)\) is \(1\) if and only if \(v\) is
truthy.
We write \(\phi_0(v) \coloneqq \phi(v)(0)\) and \(\phi_1(v) \coloneqq \phi(v)(1)\).

This means that rather than trying to divine an expression directly
from a Belnap function, we can instead define \emph{two} functions; one for how
falsy the output is, and one for how truthy is.

\begin{definition}
    Let \(\values_0 \coloneqq \{\bot, \belnapfalse\}\) and let
    \(\values_1 \coloneqq \{\bot, \belnaptrue\}\).
\end{definition}

For \(v \in \values_0\), \(\phi(v)(1) = 0\) (all the information is contained
within the falsy bit), and for \(v^\prime \in \values_1\), \(\phi(v)(0) = 0\)
(all the information is contained in the truthy bit).
These sets are of particular interest when comparing to Boolean operations.

\begin{lemma}
    \(\values_0\) and \(\values_1\) are closed under \(\land\) and \(\lor\).
\end{lemma}
\begin{proof}
    This can be verified by inspecting the truth tables:

    \vspace{0.5em}

    \begin{center}
        \begin{tabular}{c|cc}
            \(\land\)        & \(\bot\)         & \(\belnapfalse\) \\
            \hline
            \(\bot\)         & \(\bot\)         & \(\belnapfalse\) \\
            \(\belnapfalse\) & \(\belnapfalse\) & \(\belnapfalse\)
        \end{tabular}
        \quad
        \begin{tabular}{c|cc}
            \(\lor\)         & \(\bot\) & \(\belnapfalse\) \\
            \hline
            \(\bot\)         & \(\bot\) & \(\bot\)         \\
            \(\belnapfalse\) & \(\bot\) & \(\belnapfalse\)
        \end{tabular}
        \quad
        \begin{tabular}{c|cc}
            \(\land\)       & \(\bot\) & \(\belnaptrue\) \\
            \hline
            \(\bot\)        & \(\bot\) & \(\bot\)        \\
            \(\belnaptrue\) & \(\bot\) & \(\belnaptrue\)
        \end{tabular}
        \quad
        \begin{tabular}{c|cc}
            \(\lor\)        & \(\bot\)        & \(\belnaptrue\) \\
            \hline
            \(\bot\)        & \(\bot\)        & \(\belnaptrue\) \\
            \(\belnaptrue\) & \(\belnaptrue\) & \(\belnaptrue\)
        \end{tabular}
    \end{center}

    \qedhere
\end{proof}

If one looks closer, these are just the truth tables for \(\bor\) and \(\band\)
but with different symbols.
This means that any expression we make using \(\band\) and \(\bor\) in the
Boolean realm can be `simulated' in the falsy and truthy Belnap subsets.
Formally, we have the following.

\begin{lemma}\label{lem:belnap-bool-correspondence}
    The following diagrams commute:
    \begin{center}
        \begin{tikzcd}
            (\values_0)^2
            \arrow{r}{\land}
            \arrow[swap]{d}{(\phi_0, \phi_0)}
            &
            \values_0
            \arrow{d}{\phi}
            \\
            \booleans^2
            \arrow[swap]{r}{\bor}
            &
            \booleans
        \end{tikzcd}
        \quad
        \begin{tikzcd}
            (\values_0)^2
            \arrow{r}{\lor}
            \arrow[swap]{d}{(\phi_0, \phi_0)}
            &
            \values_0
            \arrow{d}{\phi}
            \\
            \booleans^2
            \arrow[swap]{r}{\band}
            &
            \booleans
        \end{tikzcd}

        \hspace{0.3em}
        \begin{tikzcd}
            (\values_1)^2
            \arrow{r}{\land}
            \arrow[swap]{d}{(\phi_1, \phi_1)}
            &
            \values_1
            \arrow{d}{\phi_0}
            \\
            \booleans^2
            \arrow[swap]{r}{\band}
            &
            \booleans
        \end{tikzcd}
        \quad
        \begin{tikzcd}
            (\values_1)^2
            \arrow{r}{\lor}
            \arrow[swap]{d}{(\phi_1, \phi_1)}
            &
            \values_1
            \arrow{d}{\phi_1}
            \\
            \booleans^2
            \arrow[swap]{r}{\bor}
            &
            \booleans
        \end{tikzcd}
    \end{center}
\end{lemma}
\begin{proof}
    By testing the four values in each case.
\end{proof}

The eager reader might now assume we can proceed as with
\cref{lem:boolean-complete},
constructing a disjunctive normal form from truth tables, but we replace Boolean
operations with the appropriate Belnap ones.
The more cautious reader may have noticed we have not discussed how the
Boolean \(\neg_\booleans\) can be simulated using Belnap operations.
This is because it is simply not possible to do this while remaining in the
two Belnap subsets.
We must make use of a certain subset of Boolean functions that can be
constructed \emph{without} using \(\neg_\booleans\).

\begin{definition}
    Let the total order \(\boolorder\) be defined as \(0 \leq 1\).
\end{definition}

As with \(\values\), \(\booleans^m\) inherits the order on \(\booleans\)
pointwise.
Subsequently, a Boolean function \(\morph{f}{\booleans^m}{\booleans}\) is
\emph{monotone} if \(f(\listvar{v}) \boolorder f(\listvar{w})\) whenever
\(\overline{v} \boolorder \overline{w}\).
Intuitively, flipping an input bit from \(0\) to \(1\) can never flip an output
bit from \(1\) to \(0\).

\begin{lemma}
    \label{lem:boolean-complete-monotone}
    All monotone functions \(\booleans^m \to \booleans\) can be expressed with
    the set of operations \(\{\land,\lor,1\}\).
\end{lemma}
\begin{proof}
    This progresses as with \cref{lem:boolean-complete}, but if the element of
    a word \(\listvar{v}(i) = 0\), it is omitted from the conjunction
    rather than the variable being negated.

    To show that this expresses the same truth table as the original function,
    consider an omitted variable \(v_i\); there exists an assignment of the
    other variables such that if \(v_i = 0\) then \(f(\dots, v_i, \dots) = 1\).
    By monotonicity, it must be the case that if \(v_i = 1\) then
    \(f(\dots, v_i, \dots) = 1\), so no information is lost by omitting
    the negation.

    If \(f(0, 0, \dots, 0) = 1\), then the inner conjunction is empty and must
    be represented by the constant \(1\), (the unit of \(\band\)).
    This is valid due to monotonicity, as if \(f\) produces \(1\) for the
    infimum, then it must produce \(1\) for all inputs.
\end{proof}

\begin{corollary}\label{cor:func-complete-falsy-truthy}
    All monotone functions \((\values_0)^m \to \values_0\) can be expressed
    with the operations \(\{\land,\lor,\belnapfalse\}\), and all monotone
    functions \((\values_1)^m \to \values_1\) can be expressed with the
    operations \(\{\land,\lor,\belnaptrue\}\).
\end{corollary}
\begin{proof}
    As there is an order isomorphism
    \(\values_0 \cong \values_1 \cong \booleans\), any monotone function in
    the Belnap subsets can be viewed as a monotone Boolean function.
    This means the strategy of \cref{lem:boolean-complete-monotone} can
    be applied using \cref{lem:belnap-bool-correspondence} to substitute the
    appropriate Belnap operation.
\end{proof}

All the pieces are now in place to express the final functional completeness
result; we just need to `explode' a Belnap value into its falsy and truthy
components, and then unify the two at the end.

\begin{definition}
    \label{def:translation-tables}
    Let the functions \(\morph{\psi^0_0,\psi^1_0}{\values}{\values_0}\) and
    \(\morph{\psi^0_1,\psi^1_1}{\values}{\values_1}\) be defined according to
    the table below.
    \begin{center}
        \begin{tabular}{c|cccc}
                             & \(\psi^0_0\)     & \(\psi^1_0\)     & \(\psi^0_1\)    & \(\psi^1_1\)    \\
            \hline
            \(\bot\)         & \(\bot\)         & \(\bot\)         & \(\bot\)        & \(\bot\)        \\
            \(\belnaptrue\)  & \(\bot\)         & \(\belnapfalse\) & \(\bot\)        & \(\belnaptrue\) \\
            \(\belnapfalse\) & \(\belnapfalse\) & \(\bot\)         & \(\belnaptrue\) & \(\bot\)        \\
            \(\top\)         & \(\belnapfalse\) & \(\belnapfalse\) & \(\belnaptrue\) & \(\belnaptrue\) \\
        \end{tabular}
    \end{center}
\end{definition}

The functions \(\psi^0_0\) and \(\psi^0_1\) send a value \(v\) to
\(\belnapfalse\) or \(\belnaptrue\) respectively if \(v\) is falsy;
\(\psi^1_0\) and \(\psi^1_1\) send a value \(v\) to
\(\belnapfalse\) or \(\belnaptrue\) if \(v\) is truthy.
Otherwise, they produce \(\bot\).

\begin{lemma}
    \label{lem:translation-functions}
    The functions in \cref{def:translation-tables} can be expressed using the
    operations \(\{\land,\lor,\neg,\bot\}\).
\end{lemma}
\begin{proof}
    From left to right, the columns in the table above are the functions \(
    v \mapsto - \land \bot
    \), \(
    v \mapsto \neg(- \lor \bot)
    \), \(
    v \mapsto \neg(- \land \bot)
    \) and \(
    v \mapsto - \lor \bot
    \).
\end{proof}

\begin{definition}\label{def:falsy-truthy-functions}
    For a monotone function \(\morph{f}{\valuetuple{m}}{\values}\), let
    \(\morph{f_0}{((\values_0)^m)^2}{\values_0}\) be defined as
    \(
    f_0(\psi^0_0(\listvar{v}), \psi^1_0(\listvar{v}))
    \coloneqq
    \phi_0(f(\listvar{v}))
    \).
    Similarly, let \(\morph{f_1}{((\values_1)^m)^2}{\values_1}\) be defined as
    \(
    f_1(\psi^0_1(\listvar{v}), \psi^1_1(\listvar{v}))
    \coloneqq
    \phi_1(f(\listvar{v}))
    \).
\end{definition}

By putting these pieces all together we can express all monotone Belnap
functions.

\begin{theorem}
    \label{thm:belnap-complete}
    All monotone functions \(\valuetuple{m} \to \values\) can be expressed
    using the operations \(
    \{\land,\lor,\neg,\ljoin,\bot,\belnaptrue,\belnapfalse\}
    \).
\end{theorem}
\begin{proof}
    This follows by defining a function with the same behaviour as the original,
    but made up of components known to be expressible using the operations
    specified.

    Let \(\morph{f^\prime}{\valuetuple{m}}{\valuetuple{2}}\) be defined as
    \(
    f^\prime(\listvar{v})
    \coloneqq
    \left(
    f_0(\psi^0_0(\listvar{v}), \psi^1_0(\listvar{v})),
    f_1(\psi^0_1(\listvar{v}), \psi^1_1(\listvar{v}))
    \right).
    \).
    By \cref{cor:func-complete-falsy-truthy}, \(f_0\) and \(f_1\) can be defined
    using \(\{\land,\lor,\belnaptrue,\belnapfalse\}\), and by
    \cref{lem:translation-functions}, \(\psi^0_0,\psi^1_0,\psi^0_1\) and
    \(\psi^1_1\) can be defined using \(\{\land,\lor,\bot\}\).

    The output of \(f^\prime(\listvar{v})\) is \(
    \left(\phi_0(f(\listvar{v})), \phi_1(f(\listvar{v}))\right)
    \) by definition; the falsiness and the truthiness of \(f(\listvar{v})\).
    To combine the two outputs into a single output we want to
    implement the following truth table:

    \begin{center}
        \begin{tabular}{cc|c}
            \(\bot\)         & \(\bot\)        & \(\bot\)         \\
            \(\bot\)         & \(\belnaptrue\) & \(\belnaptrue\)  \\
            \(\belnapfalse\) & \(\bot\)        & \(\belnapfalse\) \\
            \(\belnapfalse\) & \(\belnaptrue\) & \(\top\)
        \end{tabular}
    \end{center}

    This is clearly just the truth table for \(\ljoin\), so the entire
    expression can be defined using the operations \(
    \{\land,\lor,\neg,\ljoin,\bot,\belnaptrue,\belnapfalse\}
    \).
\end{proof}

\begin{example}
    Consider the following truth table (in fact just the table for \(\neg\)).
    \begin{center}
        \begin{tabular}{c|c}
            \(\bot\)         & \(\bot\)         \\
            \(\belnaptrue\)  & \(\belnapfalse\) \\
            \(\belnapfalse\) & \(\belnaptrue\)  \\
            \(\top\)         & \(\top\)
        \end{tabular}
    \end{center}
    We translate these into the falsy and truthy tables as follows:
    \begin{center}
        \begin{tabular}{c|c}
            \(\bot\bot\)                 & \(\bot\)         \\
            \(\bot\belnapfalse\)         & \(\belnapfalse\) \\
            \(\belnapfalse\bot\)         & \(\bot\)         \\
            \(\belnapfalse\belnapfalse\) & \(\belnapfalse\)
        \end{tabular}
        \begin{tabular}{c|c}
            \(\bot\bot\)               & \(\bot\)        \\
            \(\bot\belnaptrue\)        & \(\bot\)        \\
            \(\belnaptrue\bot\)        & \(\belnaptrue\) \\
            \(\belnaptrue\belnaptrue\) & \(\belnaptrue\)
        \end{tabular}
    \end{center}
    Using \cref{cor:func-complete-falsy-truthy}, the corresponding Belnap
    expressions are \[
        (v_0,v_1) \mapsto v_1 \land (v_0 \lor v_1)
        \qquad
        (v_0,v_1) \mapsto v_0 \lor (v_0 \land v_1)
    \]
    To combine these expressions on two inputs into a single expression on one
    input, we need to add the appropriate translators.
    We obtain the expression \[
        v_0 \mapsto
        \neg(\bot \lor v_0)
        \land
        \left((\bot \land v_0) \lor \neg(\bot \lor v_0)\right)
        \ljoin
        \neg(\bot \land v_0)
        \lor
        (\neg(\bot \land v_0) \land (\bot \lor v_0))
    \]
    which can be verified to act the same as the original table.
\end{example}

Although this result only applies to functions with a single output, it is
easily generalised to arbitrary-output functions.

\begin{corollary}
    All monotone functions \(\valuetuple{m+1} \to \valuetuple{n}\) can be
    expressed using the operations \(\{\land,\lor,\neg,\bot,\ljoin\}\).
\end{corollary}
\begin{proof}
    By repeating the process in \cref{thm:belnap-complete} for each output.
\end{proof}

Since an expression can be  using only operations with counterparts in
the syntactic realm, the map \(
\morph{\mealytofunc_\belnap}{\func{\interpretation_\belnap}}{\scirc{\Sigma_\belnap}}
\) sends functions \(f\) to circuits of the form \[
    \iltikzfig{circuits/axioms/belnap/translation/exploded-form}[box=f]
\] in which \(
\iltikzfig{strings/category/f}[box=f_0,colour=comb]
\) and \(
\iltikzfig{strings/category/f}[box=f_1,colour=comb]
\) are `syntactic' falsy and truthy disjunctive normal forms respectively.
While the truthy circuit is just a regular disjunctive normal form, because
the falsy operations are simulated by the opposite gate, it looks a bit
different.

\begin{definition}[Conjunction]
    A Belnap circuit is a \emph{truthy conjunction} if it is the infinite
    register \(
    \iltikzfig{circuits/components/waveforms/infinite-register}[val=\belnaptrue]
    \) or of the form \(
    \iltikzfig{circuits/algebraic/conjunction}[ccolour=seq]
    \), where \(
    \iltikzfig{strings/category/f}[box=f, colour=seq]
    \) is another truthy conjunction.
    A Belnap circuit is a \emph{falsy conjunction} if it is the infinite
    register \(
    \iltikzfig{circuits/components/waveforms/infinite-register}[val=\belnapfalse]
    \) or of the form \(
    \iltikzfig{circuits/algebraic/disjunction}[dcolour=seq]
    \), where \(
    \iltikzfig{strings/category/f}[box=f, colour=seq]
    \) is another falsy conjunction.
\end{definition}

\begin{definition}[Disjunctive normal form]
    A Belnap circuit is in \emph{truthy disjunctive normal form} if it is the
    eliminator \(
    \iltikzfig{strings/structure/monoid/init}[colour=comb]
    \) or of the form \(
    \iltikzfig{circuits/algebraic/disjunctive-normal}[ccolour=seq]
    \), where \(
    \iltikzfig{strings/category/f}[box=f, colour=seq]
    \) is in truthy disjunctive normal form and \(
    \iltikzfig{strings/category/f}[box=g, colour=seq]
    \) is a truthy conjunction.
    A Belnap circuit is in \emph{falsy disjunctive normal form} if it is the
    eliminator \(
    \iltikzfig{strings/structure/monoid/init}[colour=comb]
    \) or of the form \(
    \iltikzfig{circuits/algebraic/conjunctive-normal}[dcolour=seq]
    \), where \(
    \iltikzfig{strings/category/f}[box=f, colour=seq]
    \) is in falsy disjunctive normal form and \(
    \iltikzfig{strings/category/f}[box=g, colour=seq]
    \) is a falsy conjunction.
\end{definition}

The two subcircuits are falsy and truthy disjunctive normal forms, and can be
defined syntactically by using a `composite fork' to copy the inputs for each
clause in the normal form.

\begin{definition}\label{def:mk-fork}
    For \(n \in \nat\), an \emph{\(m,k\)-fork}
    \(\iltikzfig{circuits/components/structural/mn-fork}\)
    is defined
    inductively with \(
    \iltikzfig{circuits/components/structural/m0-fork} \coloneqq
    \iltikzfig{strings/category/identity-blank}[obj=m]
    \) and \(
    \iltikzfig{circuits/components/structural/mkp1-fork} \coloneqq
    \iltikzfig{circuits/components/structural/n-fork}
    \).
\end{definition}

\begin{definition}
    Let \(\morph{\mealytofunc_\belnap}{\func{\belnapinterpretation}}{\ccirc{\belnapsignature}}\)
    be defined as the map sending a function \(\morph{f}{\valuetuple{m}}{\valuetuple{n}}\)
    to a circuit \[
        \iltikzfig{circuits/axioms/belnap/translation/exploded-form-cnf}[dom=m,cod=n,box=f]
    \] where \(
    \iltikzfig{strings/category/f}[box=g_0,colour=comb]
    \) and \(
    \iltikzfig{strings/category/f}[box=g_1,colour=comb]
    \) only contain identity and elimination constructs, and \(
    \iltikzfig{strings/category/f}[box=h_0,colour=seq]
    \) and \(
    \iltikzfig{strings/category/f}[box=h_1,colour=seq]
    \) and in falsy and truthy conjunctive normal form respectively, defined
    in the obvious way derived from the procedure detailed in this section.
\end{definition}

This means that the denotational semantics for sequential circuits can
definitely be used for the Belnap interpretation \(\belnapinterpretation\).
In particular, this means we can translate any Mealy machine in
\(\mealy{\belnapinterpretation}\) (and subsequently, any stream function in
\(\stream{\belnapinterpretation})\) into a syntactic circuit in
\(\scirc{\belnapsignature}\).

While the signature containing \(\andgate\), \(\orgate\) and \(\notgate\) gates
is common, it is by no means the only useful one.
For example, it is well known that the \(\nandgate\) gate on its own is also
functionally complete.
Since the other logic gates can be expressed in terms of \(\andgate\),
\(\orgate\) and \(\notgate\) gates, the functional completeness result for
\(\belnapinterpretation\) can be adapted for these other gate sets.

Even when applying the above techniques to small concrete examples,
the results quickly balloon in size and it is not feasible nor desirable to work
them out by hand.
A tool has been developed to generate Belnap expressions from functions and
truth tables, and it can be found at \url{https://belnap.georgejkaye.com}.