\section{Between circuits and Mealy machines}\label{sec:synthesis}

The close links between \(\streami\) and \(\mealyi\) are nice to have but hardly
ground-breaking; the main contribution of this chapter is to introduce
\(\scircsigma\) to the mix by defining maps \(\scircsigma \to \mealyi\) and
\(\mealyi \to \scircsigma\).
This allows us to use monotone Mealy machines as a stepping stone in the
correspondence between sequential circuits and stream functions.
By exploiting the coalgebraic properties shared between Mealy machines and
stream functions, this can be used to show that \(\streami\) is both a
\emph{sound} and \emph{complete} denotational semantics: there is a stream
function in \(\streami\) for every circuit in \(\scircsigma\), and there is a
circuit in \(\scircsigma\) for every stream function in \(\streami\).

\subsection{Circuits to monotone Mealy machines}

Circuits have a very natural interpretation as Mealy machines, so the action
of a PROP morphism from \(\scircsigma\) to \(\mealyi\) is fairly intuitive.

\begin{definition}
    \nomenclature{\(\circuittomealyi\)}{PROP morphism from sequential circuits to Mealy machines}
    Let \(\morph{\circuittomealyi}{\scircsigma}{\mealyi}\) be the traced PROP
    morphism defined on generators as
    \begin{align*}
        \circuittomealy[
            \iltikzfig{circuits/components/gates/gate}[colour=comb]
        ]{\interpretation}
         & \coloneqq (
        \{s\},
         &             & \left(s, \listvar{v}\right) \mapsto
        \left\langle{s, \gateinterpretation[g](\listvar{v})}\right\rangle,
         &             & s
        )
        \\
        \circuittomealy[
            \iltikzfig{strings/structure/comonoid/copy}[colour=comb]
        ]{\interpretation}
         & \coloneqq (
        \{s\},
         &             & (s, v) \mapsto \left\langle{s, (v, v)}\right\rangle,
         &             & s
        )
        \\
        \circuittomealy[
            \iltikzfig{strings/structure/monoid/merge}[colour=comb]
        ]{\interpretation}
         & \coloneqq (
        \{s\},
         &             & (s, (v, w)) \mapsto
        \left\langle{s, (v \ljoin w)}\right\rangle,
         &             & s
        )
        \\
        \circuittomealy[
            \iltikzfig{strings/structure/comonoid/discard}[colour=comb]
        ]{\interpretation}
         & \coloneqq (
        \{s\},
         &             & (s, v) \mapsto
        \left\langle{s, ()}\right\rangle,
         &             & s
        )
        \\
        \circuittomealy[
            \iltikzfig{circuits/components/values/vs}[val=v]
        ]{\interpretation}
         & \coloneqq
        (
        \{s_v, s_\bot\},
         &             & \{
        s_v \mapsto \left\langle{s_\bot,v}\right\rangle,
        s_\bot \mapsto \left\langle{s_\bot,\bot}\right\rangle
        \},
         &             & s_v
        )
        \\
        \circuittomealy[
            \iltikzfig{circuits/components/waveforms/delay}
        ]{\interpretation}
         & \coloneqq
        (
        \{ s_v \,|\, v \in \values\},
         &             & (s_v, a) \mapsto \left\langle{v,s_a}\right\rangle,
         &             & s_\bot
        )
    \end{align*}
\end{definition}

\begin{example}
    The action of \(\circuittomealy{\belnapinterpretation}\) on values and
    delays in \(\scirc{\belnapsignature}\) is illustrated in
    \cref{fig:belnap-machines}.
\end{example}

\begin{figure}
    \centering
    \includestandalone{figures/mealy/value}

    \vspace{1em}

    \includestandalone{figures/mealy/delay}
    \caption{
        Mealy machines for Belnap values and delays
    }
    \label{fig:belnap-machines}
\end{figure}

\begin{example}\label{ex:mealy-translation}
    \index{SR NOR latch}
    Applying \(\circuittomealy{\belnapinterpretation}\) to the SR NOR latch from
    \cref{ex:sr-latch} produces the monotone Mealy machine in
    \cref{ex:trace-mealy}, which is illustrated in \cref{fig:latch-machine}.
\end{example}

\begin{figure}
    \centering
    \includestandalone{figures/mealy/latch-example}
    \caption{
        Mealy machine from \cref{ex:trace-mealy}
    }
    \label{fig:belnap-machines}
\end{figure}

Mealy machines are a reasonable semantics for sequential circuits, but the
image of \(\circuittomealyi\) does not always lead to minimal Mealy machines,
and there are many Mealy machines that may correspond to the same behaviour.
The `purest' semantics of a sequential circuit is a stream function in
\(\streami\).

\begin{definition}
    \nomenclature{\(\circuittostreami\)}{PROP morphism from sequential circuits to stream functions}
    Let \(\morph{\circuittostreami}{\scircsigma}{\streami}\) be defined as
    \(\mealytostreami \circ \circuittomealyi\).
\end{definition}

We have now finally established the \emph{denotation} of a sequential circuit \(
\iltikzfig{strings/category/f}[box=f,colour=seq,dom=m,cod=n]
\): it is the stream function \(
\morph{\circuittostreami[\iltikzfig{strings/category/f}[box=f,colour=seq]]}{\valuetuplestream{m}}{\valuetuplestream{n}}
\).
The existence of the PROP morphism \(\circuittostreami\) confirms that causal,
finitely specified and monotone stream functions are a \emph{sound} denotational
semantics for sequential circuits, as every circuit in \(\scircsigma\) has a
corresponding stream function in \(\streami\).

It is useful to verify that this denotational semantics of sequential circuits
agrees with the denotational semantics we defined earlier for
\emph{combinational} circuits in \cref{sec:interpreting-components}.

\begin{lemma}\label{lem:sequential-combinational-semantics}
    Let \(\iltikzfig{strings/category/f}[box=f,colour=comb,dom=m,cod=n]\) be
    a combinational circuit; for \(\sigma \in \valuetuplestream{m}\) and
    \(i \in \nat\), \(
    \circuittostreami[\iltikzfig{strings/category/f}[box=f,colour=comb]](\sigma)(i)
    =
    \circuittofunci[\iltikzfig{strings/category/f}[box=f,colour=comb]](\sigma(i))
    \).
\end{lemma}
\begin{proof}
    Since \(\iltikzfig{strings/category/f}[box=f,colour=comb,dom=m,cod=n]\) is
    combinational, \(
    \circuittomealyi[
        \iltikzfig{strings/category/f}[box=f,colour=comb,dom=m,cod=n]
    ]
    \) is a Mealy machine with a single state \(s\), i.e.\ there is a function
    \(\morph{g}{\valuetuple{m}}{\valuetuple{n}}\) such that  \(
    \circuittomealyi[
        \iltikzfig{strings/category/f}[box=f,colour=comb,dom=m,cod=n]
    ] = (
    \{s\},
    (s, \listvar{v}) \mapsto \left\langle{s, g(\listvar{v})}\right\rangle,
    s
    )\).
    By definition of \(\mealytostreami\), we have that \(\mealytocircuiti[
        \circuittomealyi[
            \iltikzfig{strings/category/f}[box=f,colour=comb,dom=m,cod=n]
        ]
    ](\sigma)(i) = g(\sigma(i))\).
    To complete the proof, we need to show that \(
    g(\sigma)(i) =
    \circuittofunci[\iltikzfig{strings/category/f}[box=f,colour=comb]](\sigma(i))
    \); this holds because \(\circuittomealyi\) and \(\circuittofunci\) freely
    build functions using the same constructs.
\end{proof}

Using this idea, it will be convenient to have a mapping from functions to
these constant stream functions.

\begin{definition}
    \nomenclature{\(\functostreami\)}{PROP morphism from functions on values to stream functions}
    Let \(\morph{\functostreami}{\funci}{\streami}\) be defined as the PROP
    morphism with action \(
    \functostreami[f] \coloneqq \sigma \mapsto i \mapsto f(\sigma)(i)
    \)
\end{definition}

\subsection{Monotone Mealy machines to circuits}

We now need a way to retrieve a circuit morphism in \(\scircsigma\) from a
stream function \(f \in \streami\).
To prevent us from picking an arbitrary circuit, the denotation of the circuit
must also be \(f\).

We already know by \cref{cor:minimal-mealy} that given a stream function
\(f\) we can retrieve a monotone Mealy machine \(\streamtomealyi[f]\).
All that remains is to translate this into a circuit morphism.
For regular Mealy machines, there is a standard procedure in circuit
design~\cite{kohavi2009switching} in which each state of a Mealy machine is
\emph{encoded} as a power of values, and the Mealy function is interpreted as
a circuit using combinational logic.

\begin{example}\label{ex:boolean-to-circuit}
    \index{Boolean values}
    Consider the following Mealy machine operating on Boolean values.
    \begin{center}
        \includestandalone{figures/mealy/boolean-example}
    \end{center}
    \vspace{-\belowdisplayskip}
    To convert this machine to a circuit, we assign each state a boolean value:
    in this case \(s_0 \mapsto \belnapfalse, s_1 \mapsto \belnaptrue\).
    We can now construct a truth table to show how a state and an input map to
    a transition and an output:
    \begin{center}
        \begin{tabular}{cc|cc}
            \(\belnapfalse\) & \(\belnapfalse\) & \(\belnaptrue\)  & \(\belnaptrue\)  \\
            \(\belnapfalse\) & \(\belnaptrue\)  & \(\belnapfalse\) & \(\belnapfalse\) \\
            \(\belnaptrue\)  & \(\belnapfalse\) & \(\belnaptrue\)  & \(\belnapfalse\) \\
            \(\belnaptrue\)  & \(\belnaptrue\)  & \(\belnapfalse\) & \(\belnaptrue\)  \\
        \end{tabular}
    \end{center}
    It is possible to describe these truth tables as logical expressions: in
    this case the expression for the next state is \(
    (v_0, v_1)
    \mapsto
    (\neg v_0 \land \neg v_1) \lor (v_0 \land \neg v_1)
    \) and the expression for the output is \(
    (v_0, v_1)
    \mapsto
    (\neg v_0 \land \neg v_1) \lor (v_0 \land v_1)
    \).
    These expressions can clearly be constructed as combinational circuits using
    \(\andgate\), \(\orgate\) and \(\notgate\) gates; the entire circuit
    corresponding to the Mealy machine is constructed by combining the
    combinational logic with registers to hold the state.
    \[\iltikzfig{mealy/synthesis}\]
\end{example}

We will use a variation of this procedure to map from \(\mealyi\) to
\(\scircsigma\).
However, when considering \emph{monotone} Mealy machines, this procedure must
additionally respect monotonicity as the combinational logic is constructed
using monotone components.
This means that an arbitrary encoding cannot be used; we will now show how to
select something suitable.

\begin{definition}[Encoding]\label{def:encoding}
    \index{encoding}
    \nomenclature{\(\gamma_\leq\)}{encoding for a total order \(\leq\)}
    Let \(S\) be a set equipped with a partial order \(\stateorder\) and a total
    order \(\leq\) such that \(S\) can be represented as
    \(s_0 \leq s_1 \leq \dots s_{k-1}\).
    The \emph{\(\leq\)-encoding} for this assignment is a function
    \(\morph{\gamma_\leq}{S}{\valuetuple{k}}\) defined as
    \(\gamma_\leq(s)(i) \coloneqq \top\) if \(s_i \stateorder s\) and
    \(\gamma_\leq(s)(i) \coloneqq \bot\) otherwise.
\end{definition}

\begin{example}
    Recall the monotone Mealy machine from \cref{ex:mealy-translation}, which
    has state set \(
    \belnapvalues \coloneqq \{\bot,\belnapfalse,\belnaptrue,\top\}
    \).
    We choose the total order on \(\belnapvalues\) as
    \(\bot \leq \belnapfalse \leq \belnaptrue \leq \top\); subsequently, the
    \(\leq\)-encoding is defined as \(
    \bot \mapsto \top\bot\bot\bot, \belnapfalse \mapsto \top\top\bot\bot,
    \belnaptrue \mapsto \top\bot\top\bot, \top \mapsto \top\top\top\top
    \).
\end{example}

It is essential that a \(\leq\)-encoding respects the original ordering of the
states.

\begin{lemma}
    For an ordered state space \((S,\stateorder)\) and a \(\leq\)-encoding
    \(\gamma_\leq\), \(s \stateorder s^\prime\) if and only if
    \(\gamma_\leq(s) \sqsubseteq \gamma_\leq(s^\prime)\).
\end{lemma}
\begin{proof}
    First the \(\onlyifdir\) direction.
    Let \(s_i, s_j \in S\) such that \(s_i \stateorder s_j\); we need to show
    that for every \(l < |S|\),
    \(\gamma_\leq(s_i)(l) \sqsubseteq \gamma_\leq(s_j)(l)\).
    The only way this can be violated is if \(s_i(l) = \top\) and
    \(s_j(l) = \bot\); this can only occur if \(s_l \stateorder s_i\) and
    \(s_l \not\stateorder s_j\).
    But since \(s_i \stateorder s_j\), this is a contradiction due to
    transitivity so \(\gamma_\leq(s_l) \sqsubseteq \gamma_\leq(s_j)\) also
    holds.

    Now the \(\ifdir\) direction.
    Assume that \(\gamma_\leq(s_i) \sqsubseteq \gamma_\leq(s_j)\); we need to
    show that \(s_i \stateorder s_j\); i.e.\ that \(\gamma_\leq(s_j)(i) = \top\)
    If \(\gamma_\leq(s_i) \sqsubseteq \gamma_\leq(s_j)\), then for each
    \(l < k\) then \(\gamma_\leq(s_i)(l) \sqsubseteq \gamma_\leq(s_j)(l)\);
    in particular \(\gamma_\leq(s_i)(i) \sqsubseteq \gamma_\leq(s_j)(i)\)
    By definition of \(\gamma_\leq\), \(\gamma_\leq(s_i)(i) = \top\), so if
    \(\gamma_\leq(s_i) \sqsubseteq \gamma_\leq(s_j)\) then
    \(\gamma_\leq(s_j)(i)\) is also \(\top\).
\end{proof}

Using this encoding, we will construct a circuit morphism that,
when interpreted as a function, implements the output and transition function
of the Mealy machine.
There is no reason for such a morphism to exist for an arbitrary interpretation:
why should we expect some collection of gates to be able to model every
function?
The useful interpretations are those that \emph{can} model every function.

\begin{definition}[Functional completeness]\label{def:functional-completeness}
    \index{functional completeness}
    A \emph{complete interpretation} is a tuple
    \((\interpretation,\mealytofunc)\) in which \(\interpretation\) is an
    interpretation of a signature \(\signature\) and \(
    \morph{\mealytofunc}{\funci}{\scircsigma}
    \) is a map that sends functions \(
    \morph{f}{\valuetuple{m}}{\valuetuple{n}}
    \), to circuits of the form \(
    \iltikzfig{circuits/synthesis/normalised-function}[box=f]
    \) for some word \(\listvar{v} \in \freemon{\values}\) such that
    \(\circuittostreami[\mealytofunc[f]](\sigma)(i) = f(\sigma(i))\).
\end{definition}

For a given complete interpretation \((\interpretation,\mealytofunc)\), we refer
to a circuit \(\mealytofunc[f]\) as the \emph{normalised circuit for \(f\)}.\

\begin{remark}
    Even though \(\mealytofunc\) maps combinational functions, its codomain is
    the category of \emph{sequential} circuits \(\scircsigma\).
    This is because instantaneous values may be required to create the
    normalised circuit.
    Despite the use of sequential components, the loop enforces that the state
    is \emph{constant}: it will always produce the word \(\listvar{v}\), so the
    the circuit still has combinational behaviour.

    Sometimes this is the only way to ensure every function can be modelled.
    For example, consider the Boolean function \(\booleans \to \booleans\) that
    always produces \(\belnapfalse\).
    Using the strategy from \cref{ex:boolean-to-circuit}, no lines of the truth
    table are true, so the expression can only be defined using the unit of the
    disjunction, the constant false.

    Note also that this sequential component is by no means mandatory: the
    functional completeness map may actually map only to combinational circuits,
    in which case the width of the sequential component would be \(0\).
\end{remark}

\begin{example}
    The Belnap interpretation from \cref{ex:belnap-interpretation} is
    functionally complete; for interests of space we postpone the proof to
    \cref{sec:denotational-belnap}.
    This is due to a variation of the standard functional
    completeness method for Boolean values.
\end{example}

With the knowledge that any monotone function has a corresponding circuit
in \(\scircsigma\), we set about encoding an arbitrary Mealy function
\(S \times \valuetuple{m} \to S \times \valuetuple{n}\) into a function
\(\valuetuple{k} \times \valuetuple{n} \to \valuetuple{k} \times \valuetuple{n}\).
One point to note here is that there may be more values in \(\valuetuple{k}\)
than there are states in \(S\), so we may need to `fill in the gaps' in a way
that is compatible with monotonicity.

\begin{definition}[Monotone completion]\label{def:monotone-completion}
    \index{monotone!completion}
    \nomenclature{\(f_\mathsf{m}\)}{monotone completion of function \(m\)}
    Let \(A\) be a finite poset and let \(B\) be a finite lattice such that
    \(A \subseteq B\).
    Then for another finite lattice \(C\) and a monotone function
    \(\morph{f}{A}{C}\), let the \emph{monotone \(B\)-completion} of \(f\) be
    the function \(\morph{f_\mathsf{m}}{B}{C}\) recursively defined as \[
        f_\mathsf{m}(v) = \begin{cases}
            f(v)
             &
            \text{if}\ v \in A
            \\
            \bot_C
             &
            \text{if}\ v = \bot_B, \bot \not\in A
            \\
            \bigvee \{ f_\mathsf{m}(w) \,|\, w \leq_B v, w \neq v \}
             &
            \text{otherwise}
        \end{cases}
    \]
\end{definition}

\begin{example}
    For \(n \in \nat\), let \(\nat_{n}\) be the subset of the natural numbers
    containing the numbers \(0,1,\dots,n-1\) with the usual order.
    Let \(\morph{f}{\{2,4\}}{\nat}\) be defined as \(2 \mapsto 6\) and
    \(4 \mapsto 7\).
    The monotone \(\nat_5\)-completion of \(f\) is a function
    \(\morph{f_\mathsf{m}}{\nat_5}{\nat}\), defined as follows:
    \(f_\mathsf{m}(0) = 0\) as \(0\) is the least element of \(\nat_5\);
    \(f_\mathsf{m}(1) = 0\) as \(0 \leq 1\) and \(g_\mathsf{m}(1) = 0\);
    \(f_\mathsf{m}(2) = 6\) as \(2 \in \{2, 4\}\) and \(g(2) = 6\);
    \(f_\mathsf{m}(3) = 6\) because \(
    f_\mathsf{m}(3) =
    f_\mathsf{m}(0) \vee f_\mathsf{m}(1) \vee f_\mathsf{m}(2)
    = 0 \vee 0 \vee 6 = 6
    \); and \(f_\mathsf{m}(4) = 7\) because \(f(4) = 7\).
\end{example}

A Mealy function can now be encoded over powers of \(\values\) by using the
monotone completion of some encoding function.

\begin{definition}[Monotone Mealy encoding]\label{def:mealy-encoding}
    \index{monotone!Mealy!encoding}
    For a monotone Mealy machine \((S, f, s_0)\) with \(k\) states and an
    encoding \(\morph{\gamma_\leq}{S}{\valuetuple{k}}\), let
    \(\morph{\gamma^p_\leq}{\gamma_\leq[S] \times \valuetuple{m}}{\valuetuple{k} \times \valuetuple{n}}\)
    be defined as the function \(
    (\gamma_\leq(s), \listvar{x}) \mapsto
    (\gamma_\leq(\mealyfunctiontransition{f}(s, \listvar{x})),
    \mealyfunctionoutput{f}(s, \listvar{x}))
    \).
    The \emph{monotone Mealy encoding} of \((S, f, s_0)\) is a function
    \(
    \morph{
        \gamma_\leq(f)
    }{
        \valuetuple{k} \times \valuetuple{m}
    }{
        \valuetuple{k} \times \valuetuple{n}
    }
    \) defined as the \((\valuetuple{k} \times \valuetuple{m})\)-completion of
    \(\gamma_\leq^p\).
\end{definition}

To obtain the syntactic circuit for a monotone Mealy function encoded
in this way, it needs to be a morphism in \(\funci\).
It is monotone by definition, but we need to make sure it is also
\(\bot\)-preserving.

\begin{lemma}
    A monotone Mealy encoding is in \(\funci\).
\end{lemma}
\begin{proof}
    A Mealy encoding is monotone as it is a monotone completion.
    There cannot be a state encoded as \(\bot^k\), since at least one bit must
    be \(\top\); this means the monotone completion will send the input
    \((\bot^k, \bot^m)\) to \((\bot^k, \bot^n)\): it is
    \(\bot\)-preserving.
\end{proof}

The foundations are now set for establishing the image of a PROP morphism from
Mealy machines to circuit terms.
There is one more thing to consider: \cref{def:encoding} depends on some
arbitrary total ordering on the states in a given monotone Mealy machine.
While this may not seem much of an issue, when
defining a PROP morphism this must be \emph{fixed}, otherwise the mapping of
Mealy machines to circuits might be nondeterministic.

\begin{definition}[Chosen state order]
    \index{chosen state order}
    Let \((S, f, s_0)\) be a monotone Mealy machine with input space
    \(\valuetuple{m}\), and let \(\leq\) be a total order on \(\values\);
    \(\leq\) can be extended to a total order \(\leq_\star\) on
    \(\freemon{(\values^m)}\) using the lexicographic order.
    Let \(S^\prime\) be the set of accessible states of \(S\).
    For each state \(s \in S^\prime\), let
    \(t_{s,\leq} \in \freemon{(\values^m)}\) be the minimal element of the
    subset of words that transition from \(s_0\) to \(s\), ordered by \(\leq\).
    Then the \emph{chosen state order} \(\leq_{S^\prime}\) is a total order on
    \(S^\prime\) defined as \(s \leq_{S^\prime} s^\prime\) if
    \(t_{s,\leq} \leq_\star t_{s^\prime,\leq}\).
\end{definition}

The PROP morphism from monotone Mealy machines to circuits can then be
parameterised by some ordering on the set of values \(\values\), ensuring that
there is a canonical term in \(\scircsigma\) for each monotone Mealy machine.

\begin{definition}\label{def:mealy-to-circuit}
    \nomenclature{\(\mealytocircuiti\)}{PROP morphism from Mealy machines to sequential circuits}
    For a functionally complete interpretation \(\interpretation\) and total
    order \(\leq\) on \(\values\), let \(
    \morph{
        \mealytocircuiti
    }{
        \mealyi
    }{
        \scircsigma
    }
    \) be the traced PROP morphism with action defined for a monotone Mealy
    machine \((S,f,s)\) as producing \(
    \iltikzfig{circuits/synthesis/mealy-term-spaced}
    \).
\end{definition}

Before proceeding to the result that this PROP morphism is behaviour-preserving,
we must show a lemma linking the behaviour of circuits in the image of
\(\mealytocircuiti\) to initial outputs and stream derivatives.

\begin{proposition}
    \label{prop:mealy-form-image}
    For a combinational circuit \(
    \iltikzfig{strings/category/f-2-2}[box=f,dom1=x,dom2=m,cod1=x,cod2=n,colour=comb]
    \), let \(\mathsf{mf}(f)\) be the map with action \(
    (\listvar{s}) \mapsto
    \circuittostreami[
        \iltikzfig{circuits/productivity/mealy-form}[core=f,state=\listvar{s},dom=m,cod=n,delay=x]
    ]
    \) and let \(
    g
    :=
    \circuittofunci[
        \iltikzfig{strings/category/f-2-2}[box=f,colour=comb]
    ]
    \).
    Then, \(
    \mealyoutput{\mathsf{mf}(f)(\listvar{s})}{\listvar{a}}
    =
    \proj{1}(g(\listvar{s}, \listvar{a}))
    \) and \(
    \mealytransition{\mathsf{mf}(f)(\listvar{s})}{\listvar{a}}
    =
    \mathsf{mf}(f)(\proj{0}(g(\listvar{s}, \listvar{a})))
    \).
\end{proposition}
\begin{proof}
    The machine \(\circuittomealyi[
        \iltikzfig{circuits/productivity/mealy-form}[core=f,state=\listvar{s},dom=m,cod=n,delay=x]
    ]\) is computed as the fixed point of the machine \(
    \left(\valuetuple{x},
    (\listvar{r}, \listvar{a}) \mapsto \left\langle
    \listvar{r}, g(\listvar{s}, \listvar{a})
    \right\rangle, \listvar{s}\right)
    \), which is \(
    \left(\valuetuple{x},
    \listvar{v} \mapsto \left\langle
    \proj{0}\mleft(g(\listvar{s},\listvar{a})\mright),
    \proj{1}\mleft(g(\listvar{s}, \listvar{a})\mright)
    \right\rangle, \listvar{s}\right)
    \).
    The output and derivative of \(
    \mealytocircuiti[
        \circuittomealyi[
            \iltikzfig{circuits/productivity/mealy-form}[core=f,state=\listvar{s},dom=m,cod=n,delay=x]
        ]
    ]
    \) are the output and transition of the Mealy machine, so the original
    statement holds by \cref{lem:sequential-combinational-semantics}.
\end{proof}

The goal of this section is to show that the translation from Mealy machines to
circuits and back again using \(\circuittomealyi \circ \mealytocircuiti\) is
\emph{behaviour-preserving}: while the mapping may not be the identity in
\(\mealyi\), the stream functions obtained using \(\streamtomealyi\) should
be equal.
This is an important new result, as it means that rather than showing
results about the denotational semantics of circuits in \(\scircsigma\) by
interpreting them in \(\streami\), we can view morphisms of the former as
Mealy machines instead.

\begin{theorem}\label{thm:mealy-to-circuit}
    \(
    \mealytostream = \circuittostreami \circ \mealytocircuiti
    \).
\end{theorem}
\begin{proof}
    Let \((S ,f)\) be a monotone Mealy machine and let \(s \in S\) be an arbitrary
    state.
    By \cref{prop:mealy-to-stream}, the initial output of
    \(\mealytostreami[(S, f, s)]\) is
    \(\listvar{a} \mapsto \mealyfunctionoutput{f}\mleft(s, \listvar{a}\mright)\)
    and the stream derivative of \(\mealytostreami[(S, f, s)]\) is \(
    \listvar{a}
    \mapsto
    \mealytostreami[(\mealyfunctiontransition{f}\mleft(s, \listvar{a}\mright))]
    \).

    Now we consider the composite
    \(\circuittostreami[\mealytocircuiti[(S, f, s)]]\).
    By \cref{def:mealy-to-circuit} we have that \(
    \mealytocircuiti[(S, f, s_0)]
    =
    \iltikzfig{circuits/synthesis/mealy-term-spaced}
    \); by applying \(\mealytocircuiti\) and \(\mealytofunc\),
    there exists a combinational circuit \(
    \iltikzfig{strings/category/f-3-2}[box=g,colour=comb]
    \) such that \[
        \mealytocircuiti[(S, f, s_0)]
        =
        \iltikzfig{circuits/synthesis/mealy-term-spaced}
        =
        \iltikzfig{circuits/synthesis/mealy-term-comb-core}.
    \]
    Let \(
    g^\prime
    =
    \circuittofunci[\iltikzfig{strings/category/f-3-2}[box=g,colour=comb]]
    \); note that for all \(\listvar{r} \in \valuetuple{x}\) and
    \(\listvar{a} \in \valuetuple{m}\),
    \(g^\prime(\listvar{v}, \listvar{r}, \listvar{a})
    =
    \gamma_\leq(f)(\listvar{r},\listvar{a})
    \).

    We can now use \cref{prop:mealy-form-image} to compute the initial output
    and stream derivative of \(\circuittostreami[\mealytocircuiti[(S, f, s)]]\).
    To show that \(\mealytostreami = \circuittostreami \circ \mealytostreami\),
    we need to show that these `agree' with those of
    \(\mealytostreami[(S, f, s)]\).
    For the initial output, this means we just need to show they are equal:
    \begin{align*}
        \initialoutput{\circuittostreami[\mealytocircuiti[(S, f, s)]]}{\listvar{a}}
         & =
        \proj{1}\mleft(\circuittofunci[g](\listvar{v}, \gamma_\leq(s), \listvar{a})\mright)
        \\
         & =
        \proj{1}\mleft(\circuittofunci[\mealytofunc[\gamma_\leq(f)]](\gamma_\leq(s), \listvar{a})\mright)
        \\
         & =
        \proj{1}\mleft(\gamma_\leq(f)(\gamma_\leq(s), \listvar(a))\mright)
        \\
         & =
        \proj{1}\left(
        \gamma_\leq(
            \mealyfunctiontransition{f}(s, \listvar{a})),
        \mealyfunctionoutput{f}(s, \listvar{a})\right)
        \\
         & =
        \mealyfunctionoutput{f}(s, \listvar{a})
    \end{align*}
    For the stream derivative, we need to show that as states vary over
    \(s \in S\), the stream derivative of \(
    \circuittostreami[\mealytocircuiti[(S, f, s)]]
    \) is the \(\gamma_\leq\)-encoding of \(\mealytostreami[(S, f, s)]\).
    \begin{align*}
        \streamderivative{\left(\circuittostreami[\mealytocircuiti[(S, f, s)]]\right)}{\listvar{a}}
         & =
        \proj{0}\left(
        \circuittofunci[g](\listvar{v}, \gamma_\leq(s), \listvar{a})
        \right)
        \\
         & =
        \proj{0}\mleft(\circuittofunci[\mealytofunc[\gamma_\leq(f)]](\gamma_\leq(s), \listvar{a})\mright)
        \\
         & =
        \proj{0}\mleft(\gamma_\leq(f)(\gamma_\leq(s), \listvar(a))\mright)
        \\
         & =
        \proj{0}\left(
        \gamma_\leq(
            \mealyfunctiontransition{f}(s, \listvar{a})),
        \mealyfunctionoutput{f}(s, \listvar{a})\right)
        \\
         & =
        \gamma_\leq(
        \mealyfunctiontransition{f}(s, \listvar{a}))
    \end{align*}
    The initial outputs and stream derivatives agree, so
    \(\mealytostream = \circuittostreami \circ \mealytocircuiti\).
\end{proof}

On its own, this is a nice result to have; if we only know the specification of
a circuit in terms of a (monotone) Mealy machine, we can use the PROP morphism
\(\mealytocircuiti\) to generate a circuit in \(\scircsigma\) which has the
same behaviour as a stream function.
However, this is but one ingredient in our ultimate goal: the completeness of
the denotational semantics.