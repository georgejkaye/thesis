\section{Monotone Mealy machines}\label{sec:mealy}

It is not immediately obvious how to translate back from stream functions in
\(\streami\) to circuits in \(\scircsigma\).
Even though these stream functions have finitely many stream derivatives, how
does one encapsulate this behaviour into a circuit?
Fortunately, we have a secret weapon: the
\emph{Mealy machine}~\cite{mealy1955method}.

Mealy machines are used in circuit design to specify the behaviour of a circuit
without having to use concrete components.  They also have
a very useful \emph{coalgebraic} viewpoint which we will wield in order to
build a bridge from circuits into stream functions.
In particular, there is a unique homomorphism from a Mealy machine to a causal,
finitely specified stream function.
Our strategy is to assemble a special class of Mealy machines which we dub
\emph{monotone Mealy machines} into another traced PROP.

\begin{definition}[Mealy machine~\cite{mealy1955method}]\label{def:mealy}
    \index{Mealy!machine}
    Let \(A\) and \(B\) be finite sets.
    A (finite) \((A,B)\)-\emph{Mealy machine} is a tuple \((S, f)\) where
    \(S\) is a finite set called the \emph{state space},
    \(\morph{f}{S}{(S \times B)^A}\) is the \emph{Mealy function}.
\end{definition}

An \((A,B)\)-Mealy machine is parameterised over a set \(A\) of \emph{inputs}
and a set \(B\) of \emph{outputs}, and is comprised of a set \(S\) of
\emph{states} and a function transforming a pair \((s, a)\) of a current state
and an input into a pair \(\langle{t,b}\rangle\) of a transition state and an
output.

\begin{notation}
    \nomenclature{\(\mealyfunctiontransition{f}\)}{transition component of Mealy function \(f\)}
    \nomenclature{\(\mealyfunctionoutput{f}\)}{output component of Mealy function \(f\)}
    We will use the shorthand \(
    \mealyfunctiontransition{f} \coloneqq (s, a) \mapsto \proj{0}(f(s)(a))
    \) and \(
    \mealyfunctionoutput{f} \coloneqq (s, a) \mapsto \proj{1}(f(s)(a))
    \) for the transition and output component of the Mealy function respectively.
\end{notation}

\begin{example}\label{ex:mealy}
    \nomenclature{\(\booleans\)}{set of Booleans}
    Let the set of Booleans be defined as
    \(\booleans \coloneqq \{\mathsf{f},\mathsf{t}\}\).
    We define a \((\booleans,\booleans)\)-Mealy machine \((S, f)\) as follows:
    \begin{center}
        \begin{minipage}{0.33\textwidth}
            \[S \coloneqq \{s_0, s_1\}\]
        \end{minipage}
        \begin{minipage}{0.66\textwidth}
            \begin{gather*}
                f(s_0, \mathsf{f}) \mapsto \langle{s_0, \mathsf{f}}\rangle
                \qquad
                f(s_0, \mathsf{t}) \mapsto \langle{s_1, \mathsf{t}}\rangle
                \\
                f(s_1, \mathsf{f}) \mapsto \langle{s_1, \mathsf{t}}\rangle
                \qquad
                f(s_1, \mathsf{t}) \mapsto \langle{s_0, \mathsf{f}}\rangle
            \end{gather*}
        \end{minipage}
    \end{center}
    This is a Mealy machine with two states; at state \(s_0\) the output is the
    input, and at state \(s_1\) the output is the negation.
    If the input is true then the state switches.
    To illustrate Mealy machines we draw states as circles; an arrow between
    states labelled \(v|w\) represents a transition on input \(v\) producing
    output \(w\).
    \begin{center}
        \includestandalone[scale=1]{figures/mealy/example}
    \end{center}
\end{example}

\subsection{The coalgebraic perspective}

The definition of Mealy machine above is timeless and forms the basis for most
of modern electronics.
The natural question for the categorist to ask is
\emph{can we make it more categorical?}
And as is often the case, we can, using the notion of a \emph{coalgebra}.

\begin{definition}[Coalgebra]
    \index{coalgebra}
    For a category \(\mcc\), let \(\morph{F}{\mcc}{\mcc}\) be an endofunctor.
    A \emph{coalgebra} for \(F\), or \(F\)-coalgebra, is an object
    \(A \in \mcc\) along with a morphism \(\morph{\alpha}{A}{FA} \in \mcc\),
    usually written \((A,\alpha)\).
\end{definition}

A Mealy machine is a pair of a set
and a function, so this is a coalgebra in \(\set\).

\begin{definition}
    \index{Mealy!coalgebra}
    For sets \(A\) and \(B\), an \emph{\((A,B)\)-Mealy coalgebra} is a coalgebra
    of the functor \(\morph{Y}{\set}{\set}\), defined as
    \(S \mapsto (S \times B)^A\).
\end{definition}

\begin{example}[\cite{bonsangue2008coalgebraic}]
    Given sets \(A\) and \(B\), let \(\Gamma\) be the set of causal stream
    functions \(\stream{A} \to \stream{B}\), and let
    \(\morph{\nu}{\Gamma}{(\Gamma \times B)^A}\) be the function defined as \(
    f \mapsto a \mapsto \langle{
        \streamderivative{f}{a},\initialoutput{f}{a}
    }\rangle\).
    Then \((\Gamma,\nu)\) is a \((A,B)\)-Mealy coalgebra.
\end{example}

The above example lays the groundwork to establish connections between circuits,
stream functions and Mealy machines.
If we inspect it a little closer, we find that stream functions are even more
special than just being `an' \((A,B)\)-Mealy coalgebra.

\begin{definition}[Mealy homomorphism]\label{def:mealy-homomorphism}
    \index{Mealy!homomorphism}
    For sets \(A\) and \(B\), a \emph{Mealy homomorphism} between two
    \((A,B)\)-Mealy coalgebra \((S,f)\) and \((T,g)\) is a function
    \(\morph{h}{S}{T}\) preserving transitions and
    outputs, i.e.\ if \(f(s,a) = (r,b)\), then \(g(h(s),a) = (h(r),b)\).
\end{definition}

The \emph{final} \((A,B)\)-Mealy coalgebra is a \((A,B)\)-Mealy coalgebra to
which every other \((A,B)\)-Mealy coalgebra has a unique homomorphism.

\begin{proposition}[\cite{rutten2006algebraic}, Prop.\ 2.2]
    \label{prop:final-coalgebra}
    For every \((A,B)\)-Mealy coalgebra \((S,f)\), there exists a
    unique \((A,B)\)-Mealy homomorphism \(\morph{{!}}{(S,f)}{(\Gamma,\nu)}\).
\end{proposition}
\begin{proof}
    An \((A,B)\)-Mealy homomorphism \(\morph{g}{(S, f)}{(\Gamma, \nu)}\) is a
    function \(S \to \Gamma\), so for a state \(s_0 \in S\), \(g(s)\) will be a
    stream function \(\stream{A} \to \stream{B}\).
    Let \(\sigma \in \stream{A}\) be an input stream; there is a (unique) series
    of transitions \[
        s_0
        \mealyarrow{\sigma(0)}{b_0}
        s_1
        \mealyarrow{\sigma(1)}{b_1}
        s_2
        \mealyarrow{\sigma(2)}{b_2}
        s_3
        \mealyarrow{\sigma(3)}{b_3}
        \cdots
    \]
    Then \(!(s)\) is defined for input \(\sigma\) and
    index \(i \in \nat\) as \(!(s)(\sigma)(i) \coloneqq b_i\).
\end{proof}

For a Mealy coalgebra \((S, f)\) and a start state \(s_0\),
\(!(s_0)(\sigma)\) maps to the stream of outputs that \((S, f)\) would produce
by applying \(f\) to each element of \(\sigma\), starting from \(s_0\).

\subsection{Mealy machines on posets}

To use Mealy machines as a bridge between \(\scircsigma\) and \(\streami\) they
must be assembled into another traced PROP.
Not all Mealy machines defined so far correspond to circuits in
\(\scircsigma\); we must refine our notion of Mealy machine in order to find
those that do: those that map to stream functions in \(\streami\).

\begin{lemma}
    For a Mealy machine \((S, f)\) and state \(s_0 \in S\), \(!(s_0)\)
    is finitely specified.
\end{lemma}
\begin{proof}
    \(S\) is finite, and \(\mealytostream\) must preserve transitions.
\end{proof}

We must also impose a monotonicity condition.

\begin{definition}[Monotone Mealy machine]
    \index{Mealy!machine!monotone}
    \index{monotone!Mealy!machine}
    Let \(A\), \(B\) be posets; an \((A,B)\)-Mealy machine \((S, f)\)
    is called a \emph{monotone} Mealy machine if \(S\) is also a poset and
    \(f\) is \(\bot\)-preserving monotone with respect to the ordering on
    \(A\), \(B\), and \(S\).
\end{definition}

To map to Mealy machines from circuits we need to assemble the former into
another PROP, in which the morphisms \(m \to n\) are
\((\valuetuple{m},\valuetuple{n})\)-Mealy machines; we must also take into
account the `initial state' of circuits in \(\scircsigma\).

\begin{definition}[Initialised Mealy machine]
    \index{Mealy!machine!initialised}
    \index{initialised Mealy machine}
    An \emph{initialised} Mealy machine is a tuple \((S, f, s_0)\), where
    \((S, f)\) is a Mealy machine, and \(s_0 \in S\) is an \emph{initial state}.
\end{definition}

\begin{example}\label{ex:mealy-init}
    We can initialise the \((\booleans,\booleans)\)-Mealy machine
    \((\{s_0,s_1\},f)\) from \cref{ex:mealy} in two ways; here we will choose to
    initialise it as \((S,f,s_0)\).
    In the diagrams, we label the initial state with an arrow.
    \begin{center}
        \includestandalone[scale=1]{figures/mealy/example-init}
    \end{center}
\end{example}

All that remains to define is the composition of Mealy machines, which is
standard.

\begin{definition}[Cascade product of Mealy machines~\cite{ginzburg2014algebraic}]
    \index{cascade product}
    \index{Mealy!machines!cascade product}
    \index{composition!of Mealy machines}
    Given an initialised \((A,B)\)-Mealy machine \((S,f,s_0)\) and an
    initialised \((B,C)\)-Mealy machine \((T,g,t_0)\), their
    \emph{cascade product} is an initialised \((A,C)\)-Mealy machine defined as
    \[
        \left(S \times T, ((s, t), a) \mapsto \left\langle
        \left(
        \mealyfunctiontransition{f}(s,a),
        \mealyfunctiontransition{g}(t, \mealyfunctionoutput{f}(s, a))
        \right),
        \mealyfunctionoutput{g}(t, \mealyfunctionoutput{f}(s, a))
        \right\rangle,
        (s_0, t_0)\right).
    \]
\end{definition}

The cascade product of two Mealy machines effectively executes the first on the
inputs, then executes the second on the outputs of the first.

\begin{example}\label{ex:mealy-cascade}
    Recall the initialised \((\booleans,\booleans)\)-Mealy machine
    \((S, f, s_0)\) from \cref{ex:mealy-init}; we will now compose this with
    \((\{t_0,t_1\},g,t_0)\) where \(g\) is defined as follows:
    \begin{gather*}
        g(t_0, \mathsf{f}) \coloneqq \langle{t_0,\mathsf{f}}\rangle
        \qquad
        g(t_0, \mathsf{t}) \coloneqq \langle{t_1,\mathsf{t}}\rangle
        \qquad
        g(t_1, \mathsf{f}) \coloneqq \langle{t_1,\mathsf{t}}\rangle
        \qquad
        g(t_1, \mathsf{t}) \coloneqq \langle{t_1,\mathsf{t}}\rangle
    \end{gather*}
    \begin{center}
        \includestandalone[scale=1]{figures/mealy/example2-init}
    \end{center}
    The cascade product \((R, h, r_0)\) of these two machines defined as
    follows:
    \begin{gather*}
        R \coloneqq \{(s_0,t_0), (s_1,t_0), (s_0,t_1), (s_1,t_1)\}
        \qquad
        r_0 \coloneqq (s_0,t_0)
        \\
        h((s_0, t_0), \mathsf{f})
        \coloneqq
        \left\langle(s_0, t_0), \mathsf{f}\right\rangle
        \qquad
        h((s_0, t_0), \mathsf{t})
        \coloneqq
        \left\langle(s_0, t_1), \mathsf{t}\right\rangle
        \\
        h((s_1, t_0), \mathsf{f})
        \coloneqq
        \left\langle(s_1, t_1), \mathsf{t}\right\rangle
        \qquad
        h((s_1, t_0), \mathsf{t})
        \coloneqq
        \left\langle(s_0, t_0), \mathsf{f}\right\rangle
        \\
        h((s_0, t_1), \mathsf{f})
        \coloneqq
        \left\langle(s_0, t_1), \mathsf{t}\right\rangle
        \qquad
        h((s_0, t_1), \mathsf{t})
        \coloneqq
        \left\langle(s_1, t_1), \mathsf{t}\right\rangle
        \\
        h((s_1, t_1), \mathsf{f})
        \coloneqq
        \left\langle(s_1, t_1), \mathsf{t}\right\rangle
        \qquad
        h((s_1, t_1), \mathsf{t})
        \coloneqq
        \left\langle(s_0, t_1), \mathsf{t}\right\rangle
    \end{gather*}
    \begin{center}
        \includestandalone[scale=1]{figures/mealy/cascade}
    \end{center}
\end{example}

Tensor product is far more straightforward.

\begin{definition}[Direct product of Mealy machines]
    \index{Mealy!machines!direct product}
    \index{tensor product!of Mealy machines}
    \index{direct product}
    Given an initialised \((A,B)\)-Mealy machine \((S,f,s_0)\) and an
    initialised \((C,D)\)-Mealy machine \((T,g,t_0)\), their
    \emph{direct product} is an initialised \((A \times C,B \times D)\)-Mealy
    machine defined as \[
        (S \times T, \left((s, t), (a, c)\right) \mapsto \left\langle
        \left(
        \mealyfunctiontransition{f}\mleft(s, a\mright),
        \mealyfunctiontransition{g}\mleft(s, a\mright)
        \right),
        \left(
        \mealyfunctionoutput{f}\mleft(s, a\mright),
        \mealyfunctionoutput{g}\mleft(s, a\mright)
        \right)\right\rangle,
        (s_0, t_0)
        ).
    \]
\end{definition}

\begin{example}\label{ex:mealy-direct}
    The direct product of the two initialised \((\booleans,\booleans)\)-Mealy
    machines introduced in \cref{ex:mealy-init} and \cref{ex:mealy-cascade} is
    a \((\booleans^2,\booleans^2)\)-Mealy machine \((Q,k,q_0)\) defined as
    follows:
    \begin{gather*}
        Q \coloneqq \{(s_0,t_0), (s_1,t_0), (s_0,t_1), (s_1,t_1)\}
        \qquad
        q_0 \coloneqq (s_0,t_0)
        \\
        h((s_0, t_0), \mathsf{ff})
        \coloneqq
        \left\langle(s_0, t_0), \mathsf{ff}\right\rangle
        \qquad
        h((s_0, t_0), \mathsf{tf})
        \coloneqq
        \left\langle(s_1, t_0), \mathsf{tf}\right\rangle
        \\
        h((s_0, t_0), \mathsf{ft})
        \coloneqq
        \left\langle(s_0, t_1), \mathsf{ft}\right\rangle
        \qquad
        h((s_0, t_0), \mathsf{tt})
        \coloneqq
        \left\langle(s_1, t_1), \mathsf{tt}\right\rangle
        \\
        h((s_1, t_0), \mathsf{ff})
        \coloneqq
        \left\langle(s_1, t_0), \mathsf{tf}\right\rangle
        \qquad
        h((s_1, t_0), \mathsf{tf})
        \coloneqq
        \left\langle(s_0, t_0), \mathsf{ff}\right\rangle
        \\
        h((s_1, t_0), \mathsf{ft})
        \coloneqq
        \left\langle(s_1, t_1), \mathsf{tt}\right\rangle
        \qquad
        h((s_1, t_0), \mathsf{tt})
        \coloneqq
        \left\langle(s_0, t_1), \mathsf{ft}\right\rangle
        \\
        h((s_0, t_1), \mathsf{ff})
        \coloneqq
        \left\langle(s_0, t_1), \mathsf{ft}\right\rangle
        \qquad
        h((s_0, t_1), \mathsf{tf})
        \coloneqq
        \left\langle(s_1, t_1), \mathsf{tt}\right\rangle
        \\
        h((s_1, t_1), \mathsf{ft})
        \coloneqq
        \left\langle(s_0, t_1), \mathsf{ft}\right\rangle
        \qquad
        h((s_0, t_1), \mathsf{tt})
        \coloneqq
        \left\langle(s_1, t_1), \mathsf{tt}\right\rangle
        \\
        h((s_1, t_1), \mathsf{ff})
        \coloneqq
        \left\langle(s_1, t_1), \mathsf{tt}\right\rangle
        \qquad
        h((s_1, t_1), \mathsf{tf})
        \coloneqq
        \left\langle(s_0, t_1), \mathsf{ft}\right\rangle
        \\
        h((s_1, t_1), \mathsf{ft})
        \coloneqq
        \left\langle(s_1, t_1), \mathsf{tt}\right\rangle
        \qquad
        h((s_1, t_1), \mathsf{tt})
        \coloneqq
        \left\langle(s_0, t_1), \mathsf{ft}\right\rangle
    \end{gather*}
    \begin{center}
        \includestandalone[scale=1]{figures/mealy/direct}
    \end{center}
\end{example}

With cascade product as composition and direct product as tensor, initialised
monotone Mealy machines form a PROP.

\begin{definition}
    \index{mealyi@\(\mealyi\)}
    \nomenclature{\(\mealyi\)}{PROP of initialised monotone Mealy machines}
    Let \(\mealyi\) be the PROP in which the morphisms
    \(m \to n\) are the initialised monotone
    \((\valuetuple{m}, \valuetuple{n})\)-Mealy machines.
    Composition is by cascade product and tensor on morphisms is by
    direct product.
    The identity and the symmetry are the single-state machines that output the
    input and swap the inputs respectively.
\end{definition}

Once again, we must show that this category has a trace.
This can be computed in much the same way as it was for stream functions.

\begin{definition}
    \index{least fixed point!of Mealy machines}
    \index{Mealy!machine!least fixed point}
    Let \((S, f)\) be a monotone \(
    (\valuetuple{x+m}, \valuetuple{x+n})
    \)-Mealy machine.
    For a state \(s \in S\) and input \(\listvar{a} \in \valuetuple{m}\), let
    \(\mu_{s}(\listvar{a})\) be the least fixed point of \(
    \listvar{r} \mapsto \proj{0}\mleft(f\left(s, \listvar{r} \concat \listvar{a}\right)\mright)
    \).
    The \emph{least fixed point} of an initialised Mealy machine \((S, f, s_0)\)
    is a \((\valuetuple{m}, \valuetuple{n})\)-Mealy machine \(\left(
    S, (s, \listvar{a})
    \mapsto
    f\left(\mu_{s}\listvar{a} \concat \listvar{a}\right), s_0
    \right)
    \)
\end{definition}

\begin{example}\label{ex:trace-mealy}
    Consider the monotone \((\valuetuple{3},\valuetuple{3})\)-Mealy machine with
    state set \(\belnapvalues\), initial state \(\bot\), and Mealy function \[
        g \coloneqq (s, (x, y, z))
        \mapsto \left\langle \neg y \land \neg x,
        \left(\neg s \land \neg z, s, \neg s \land \neg z\right)
        \right\rangle
        ).\]
    To take the trace of this machine, we must first compute the least fixed
    point of \(v \mapsto \neg s \land \neg z\), which is clearly just
    \(\neg s \land \neg z\).
    Therefore the Mealy function of the traced
    \((\valuetuple{2}, \valuetuple{2})\) machine is \(
    (s, (y, z)) \mapsto g(s, (\neg s \land \neg z, y, \neg s \land \neg z))
    \).
\end{example}

We must show that this is the trace on \(\mealyi\).

\begin{proposition}
    The least fixed point is a trace on \(\mealyi\).
\end{proposition}
\begin{proof}
    Let \((S, f)\) be a monotone
    \((\valuetuple{x+m}, \valuetuple{x+n})\)-Mealy machine; this means that
    \(S\) must be a poset.
    The Mealy function \(
    \morph{f}{
        S \times \valuetuple{x+m}
    }{
        S \times \valuetuple{x+n}
    }
    \) is monotone with regards to the orders on \(S\) and
    \(\valuetuple{x+m}\) and \(S \times (x+n)\) is finite, so
    \(f\) has a least fixed point.
    The function \(
    f^\prime \coloneqq (s, \listvar{a})
    \mapsto
    \proj{1}\mleft(f(\mu_{s}(\listvar{a}) \concat \listvar{a})\mright)
    \) is a composition of \(\bot\)-preserving monotone functions, so it is
    itself \(\bot\)-preserving monotone; this makes \((S, f^\prime)\) a monotone
    \((\valuetuple{m}, \valuetuple{n})\)-Mealy machine.
    This construction is a trace for the same reason as the trace of
    \(\streami\) is.
\end{proof}

With monotone Mealy machines in a PROP, we can now represent the unique
homomorphism from a Mealy machine to a set of state functions as a PROP
morphism.

\begin{proposition}\label{prop:mealy-to-stream}
    \nomenclature{\(\mealytostreami\)}{PROP morphism \(\mealyi \to \streami\)}
    There is a traced PROP morphism
    \(\morph{\mealytostreami}{\mealyi}{\streami}\) sending a monotone Mealy
    machine \(\morph{\left(S, f, s_0\right)}{m}{n}\) to \(!(s_0)\), where \(!\)
    is the unique homomorphism \((S,f) \to (\Gamma,\nu)\).
\end{proposition}
\begin{proof}
    Since every stream function also has a Mealy coalgebra structure and Mealy
    homomorphisms preserve transitions and outputs,
    composition of Mealy machines also coincides with composition of stream
    functions.
\end{proof}

\subsection{Streams to Mealy machines}

So far we have seen how a causal, finitely specified, and \(\bot\)-preserving
monotone stream function can be retrieved from a monotone Mealy machine.
It is also possible to retrieve a Mealy machine for a given stream function in
\(\streami\) by repeatedly taking stream derivatives; since we know there are
finitely many we will be able to compute a finite set of states in a Mealy
machine.

\begin{example}
    Let \(\morph{f}{\belnapvalues}{\belnapvalues}\) be a stream function defined
    as \(f(\sigma)(0) = \sigma(0)\) and
    \(f(\sigma)(k+1) = f(\sigma)(k) \land \sigma(k+1)\).
    We will derive a Mealy machine in \(\mealyi\) from this stream function.
    The complete set of states is \(\{
    f, \streamderivative{f}{\bot}, \streamderivative{f}{\belnapfalse},
    \streamderivative{f}{\top}
    \}\):
    \begin{itemize}
        \item \(\streamderivative{f}{\belnaptrue} = f\);
        \item \(\streamderivative{(\streamderivative{f}{\bot})}{\bot}
              =
              \streamderivative{(\streamderivative{f}{\bot})}{\belnaptrue}
              =
              \streamderivative{f}{\bot}
              \);
        \item \(\streamderivative{(\streamderivative{f}{\top})}{\top}
              =
              \streamderivative{(\streamderivative{f}{\top})}{\belnaptrue}
              =
              \streamderivative{f}{\top}
              \); and
        \item \(
              \streamderivative{(\streamderivative{f}{\bot})}{\belnapfalse}
              =
              \streamderivative{(\streamderivative{f}{\bot})}{\top}
              =
              \streamderivative{(\streamderivative{f}{\top})}{\bot}
              =
              \streamderivative{(\streamderivative{f}{\top})}{\belnapfalse}
              =
              \streamderivative{f}{\belnapfalse}
              \).
    \end{itemize}
    The Mealy function is defined for each state as the initial value and
    stream derivative of the original stream function.
    The initial state of the Mealy machine is \(f\).
\end{example}

In fact, for a function \(f\) this procedure produces a \emph{minimal} Mealy
machine.

\begin{corollary}[Corollary 2.3, \cite{rutten2006algebraic}]\label{cor:minimal-mealy}
    For a causal, finitely specified, stream function \(
    \morph{f}{\stream{M}}{\stream{N}}
    \), let \(S\) be the least set of
    causal stream functions including \(f\) and closed under stream derivatives:
    i.e.\ for all \(h \in S\) and \(a \in M\), \(h_a \in S\).
    Then the initialised Mealy machine \(
    \streamtomealy[f]{\interpretation} = (S, g, f)
    \), where \(
    g(h)(a) = \langle \mealyoutput{h}{a}, \mealytransition{h}{a}\rangle
    \), has the smallest state space of Mealy machines with the property \(
    \mealytostreami[\streamtomealyi[f]] = f
    \).
\end{corollary}
\begin{proof}
    Since \(S\) is generated from the function \(f\) and is the \emph{smallest}
    possible set, there are no unreachable states in \(S\).
    Since the output and transition of states in
    \(\streamtomealyi[f]\) are the initial output and stream derivative, two
    states can only have the same `behaviour' if they are derived from the same
    original stream function.
\end{proof}

We will encode this data into a PROP morphism from \(\streami\) to \(\mealyi\);
in order to do this we must verify that the produced Mealy machine is monotone.

\begin{lemma}\label{lem:head-tail-monotone}
    The functions \(\sigma \mapsto \streaminit(\sigma)\) and
    \(\sigma \mapsto \streamderv(\sigma)\) are monotone.
\end{lemma}
\begin{proof}
    Let \(\sigma \coloneqq a \streamcons \sigma^\prime\) and
    \(\tau \coloneqq b \streamcons \tau^\prime\) be streams in \(\stream{A}\)
    such that \(\sigma \leq_{\stream{A}} \tau\); subsequently \(a \leq b\) and
    \(\sigma^\prime \leq_{\stream{A}} \tau^\prime\).
    So \(
    \streaminit(\sigma) \coloneqq
    \streaminit(a \streamcons \sigma^\prime) =
    a \leq_{A}
    b =
    \streaminit(b \streamcons \tau^\prime) \coloneqq
    \streaminit(\tau)
    \) and \(
    \streamderv(\sigma) \coloneqq
    \streamderv(a \streamcons \sigma^\prime) =
    \sigma^\prime \leq_{\stream{A}}
    \tau^\prime =
    \streamderv(b \streamcons \tau^\prime) =
    \streamderv(\tau)
    \).
\end{proof}

\begin{lemma}\label{lem:initial-derivative-monotone}
    For posets \(A\) and \(B\) and a monotone causal stream function
    \(\morph{f}{\stream{A}}{\stream{B}}\), the functions
    \(a \mapsto \initialoutput{f}{a}\) and \(a \mapsto \streamderivative{f}{a}\)
    are monotone.
\end{lemma}
\begin{proof}
    Let \(a, b \in A\) such that \(a \leq_A b\); then by monotonicity
    \(f(a \streamcons \sigma) \leq_{\stream{B}} f(b \streamcons \sigma)\) for
    all \(\sigma \in \stream{A}\).
    By \cref{lem:head-tail-monotone}, \(\streaminit \circ f\) and
    \(\streamderv \circ f\) are monotone.
    First we show that the initial output is monotone: \(
    \initialoutput{f}{a} \coloneqq
    \streaminit(f(a \streamcons \sigma)) \leq_{A}
    \streaminit(f(b \streamcons \sigma)) =
    \initialoutput{f}{b}
    \).
    For the stream derivative, \(
    \streamderivative{f}{a}(\sigma) \coloneqq
    \streamderv(f(a \streamcons \sigma)) \leq_{\stream{B}}
    \streamderv(f(b \streamcons \sigma)) \coloneqq
    \streamderivative{f}{a}(\sigma)
    \).
\end{proof}

\begin{lemma}\label{lem:stream-to-mealy-is-monotone}
    Given \(f \in \streami\), \(\streamtomealyi[f]\) is
    a monotone Mealy machine.
\end{lemma}
\begin{proof}
    Each state in the derived Mealy machine is a monotone stream function, so
    this is a poset ordered by \(\stateorder\) as defined in
    \cref{def:state-order}. and
    The Mealy function is the pairing of the initial output and stream
    derivative; by \cref{lem:initial-derivative-monotone} these are monotone so
    the Mealy function must also be monotone.
\end{proof}

\begin{corollary}
    The procedure \(\streamtomealyi\) is a PROP morphism
    \(\streami \to \mealyi\).
\end{corollary}

This means we can map between monotone Mealy machines and causal, finitely
specified, monotone stream functions in either direction.
Mealy machines are perhaps more intuitive to work with, but stream functions
are the `purest' specification of the behaviour in that they have the smallest
possible state set.
Ideally we would be able to work in whichever setting is most beneficial at a
given time, so we need to show that the translations are
\emph{behaviour-preserving}.

\begin{proposition}\label{prop:mealy-stream-id}
    \(\mealytostreami \circ \streamtomealyi = \id[\streami]\).
\end{proposition}
\begin{proof}
    Stream functions are equal if they have the same initial output and
    stream derivative.
    \(\streamtomealyi\) preserves outputs and derivatives by definition, and
    \(\mealytostreami\) preserves transitions and outputs because it is a Mealy
    homomorphism.
\end{proof}

The reverse does not hold as many Mealy machines may map to the same stream
function, but the result of
\(\mealytostreami \circ \streamtomealyi \circ \mealytostreami\) is of course
equal to \(\mealytostreami\).