\section{Denotational semantics of digital circuits}\label{sec:circuits}

Now that we comprehend what exactly denotational semantics is, we turn to our
goal of defining a denotational semantics for digital circuits.
We will interpret digital circuits as a certain class of
\emph{stream functions}, functions that operate on infinite sequences of values.
This represents how the output of a circuit may not just operate on the current
input, but all of the previous ones as well.

\begin{remark}
    In \cite{mendler2012constructive}, the semantics of digital circuits with
    delays and cycles are presented using \emph{timed ternary simulation}, an
    algorithm to compute how a sequence of circuit outputs stabilises over time
    given the inputs and value of the current state.
    This differs from our approach as we assign each circuit a concrete stream
    function describing its behaviour, rather than having to solve a system of
    equations in terms of the nodes inside a circuit in order to determine its
    behaviour.
\end{remark}

\subsection{Interpreting circuit components}\label{sec:interpreting-components}

Before assigning stream functions to a given circuit in \(\scircsigma\), we will
first decide how to interpret the individual \emph{components} of a given
circuit signature.
This is crucial because a denotational semantics is defined
\emph{compositionally}; eventually we will need to refer to the interpretations
of particular components.

First we consider the interpretation of the \emph{values} that flow through the
wires in our circuits.
In the denotational semantics the set of values will need to have a bit more
structure, as it must be ordered by how much \emph{information} each value
carries.

\begin{definition}[Partially ordered set]
    \index{partially ordered set}
    \index{partial order}
    \index{poset}
    \nomenclature{\(\leq_A\)}{partial order on a set \(A\)}
    A \emph{partially ordered set}, or \emph{poset} for short, is a set \(A\)
    equipped with a reflexive, antisymmetric, and transitive relation \(\leq\),
    i.e.\
    \begin{itemize}
        \item for all \(a \in A\), \(a \leq a\);
        \item for all \(a, b \in A\), if \(a \leq b\) and \(b \leq a\) then
              \(a = b\); and
        \item for all \(a, b, c \in A\), if \(a \leq b\) and \(b \leq c\) then
              \(a \leq c\).
    \end{itemize}
    A poset \((A, leq)\) is called a \emph{finite poset} if \(A\) is finite.
\end{definition}

\begin{definition}[Least and greatest elements]
    \index{least element}
    \index{greatest element}
    \nomenclature{\(\bot\)}{least element}
    \nomenclature{\(\top\)}{greatest element}
    In a poset \((A, \leq)\), a \emph{least element} is an element \(\bot \in A\)
    such that for all elements \(y \in A\), \(\bot \leq y\).
    Similarly, a \emph{greatest element} is an element \(\top \in A\)
    such that for all elements \(x \in A\), \(x \leq \top\).
\end{definition}

\begin{example}
    The natural numbers \(\nat\) are a poset ordered in the usual way; they have
    a least element \(0\) but not a greatest element.
    However, any finite subset of the natural numbers \emph{does} have a
    maximal element.
\end{example}

Even in a finite set there is no reason that least and greatest elements should
be \emph{unique}.
However, in our context of digital circuits we would like this to be the case:
the unique least element will represent a complete \emph{lack} of information,
and the unique greatest element will represent \emph{every} piece of information
at once.
To enforce the uniqueness of these elements we must add even more structure to
a poset of values.

\begin{definition}[Lower and upper bounds]
    \index{lower bound}
    \index{upper bound}
    Given a poset \((A, \leq)\), a \emph{lower bound} of a subset
    \(B \subseteq A\) is an element \(x \in B\) such that for all \(b \in B\),
    \(x \leq b\).
    Similarly, an \emph{upper bound} of \(C \subseteq A\) is an element
    \(y \in C\) such that for all \(b \in B\), \(b \leq y\).
\end{definition}

\begin{definition}[Meet and join]
    \index{meet}
    \index{join}
    \index{greatest lower bound}
    \index{least upper bound}
    \index{supremum}
    \index{infimum}
    \nomenclature{\(\vee_A\)}{join operation on a poset \(A\)}
    \nomenclature{\(\wedge_A\)}{join operation on a poset \(A\)}
    Given a poset \((A, \leq)\), a lower bound \(x\) of \(B \subseteq A\) is
    called an \emph{meet} (or infimum, or greatest lower bound) if for all lower
    bounds \(b \in B\), \(b \leq x\).
    Similarly, an upper bound \(y\) of \(B\) is called a \emph{join} (or
    supremum, or least upper bound) if for all upper bounds \(c \in B\),
    \(y \leq c\).
\end{definition}

We usually draw the meet and the join operations as rotated versions of the
order operation.
For example, in the above definitions we have used \(\wedge\) and \(\vee\) for
the order \(\leq\); in subsequent sections we will use \(\lmeet\) and
\(\ljoin\) for the order \(\sqsubseteq\).
In general, the join and meet of a pair of elements in a poset need not exist.
We are interested in the structures in which they \emph{always} exist, which are
known as \emph{lattices}.

\begin{definition}[Lattice]
    \index{lattice}
    \index{lattice!bounded}
    \index{lattice!finite}
    A \emph{lattice} is a poset \((A, \leq)\) in which each pair of elements
    \(a,b \in A\) has a meet and join.
    A lattice is called a \emph{finite lattice} if \(A\) is finite, and
    \emph{bounded} if it has a minimum element and a maximum element.
\end{definition}

Much like how every non-empty finite poset has at least one maximimal and
minimal element, every non-empty finite lattice has a join and meet.

\begin{notation}
    For a poset \((A, \leq)\), we write \(\bigwedge A\) for the meet of
    all the elements in \(A\) (if it exists) and \(\bigvee A\) for the join of
    all the elements in \(A\) (again, if it exists).
\end{notation}

\begin{lemma}\label{cref:finite-bounded}
    A non-empty finite lattice \((A, \leq)\) is bounded.
\end{lemma}
\begin{proof}
    As each pair of elements in \(A\) has a meet, and as \(A\) is finite, we
    can define the element \(x\) as \(\bigwedge A\).
    This element is the greatest element in \(A\) by definition of the meet:
    \((a_0 \wedge a_1) \leq a_0\) and \((a_0 \wedge a_1) \leq a_1\),
    \(((a_0 \wedge a_1) \wedge a_2) \leq (a_0 \wedge a_1)\) and
    \(((a_0 \wedge a_1) \wedge a_2) \leq a_2\), and so on.
    The same proof holds in reverse for the join and the greatest element by
    using the join \(\bigvee A\).
\end{proof}

\begin{example}\label{ex:powerset-lattice}
    Let \(A = \{0,1,2\}\), and let \((\mcp A, \subseteq)\) be the poset defined
    as the powerset of \(A\) ordered by subset inclusion.
    This poset can be illustrated by the following \emph{Hasse diagram}, in
    which a line going up from \(a\) to \(b\) indicates that \(a \leq b\).

    \begin{center}
        \begin{tikzcd}
            & \{0,1,2\} & \\
            \{0,1\} \arrow[dash]{ur} &
            \{0,2\} \arrow[dash]{u} &
            \{1,2\} \arrow[dash]{ul} \\
            \{0\} \arrow[dash]{u} \arrow[dash]{ur} &
            \{1\} \arrow[dash]{ul} \arrow[dash]{ur} &
            \{2\} \arrow[dash]{ul} \arrow[dash]{u} \\
            & \{\} \arrow[dash]{ul} \arrow[dash]{u} \arrow[dash]{ur} &
        \end{tikzcd}
    \end{center}

    The diagram clearly illustrates the lattice structure on this poset: the
    join is union and the meet is intersection.
    Subsequently the greatest element \(\top\) is the set \(A\) and the
    least element \(\bot\) is the empty set \(\{\}\).
\end{example}

\begin{remark}
    If \(A\) is a lattice, then for any \(n \in \nat\), \(A^n\) is also a
    lattice by comparing elements pointwise.
    The \(\bot\) is then the word containing \(n\) copies of the \(\bot\)
    element in \(A\), and similarly for the \(\top\) element.

    Recall the set \(A \coloneqq \{0,1,2\}\) from
    \cref{ex:powerset-lattice} and the lattice structure on its powerset
    \(\mathcal{P}A\).
    This induces an ordering on \((\mathcal{P}A)^2\):
    \(\{0,1\}\{1\} \leq \{0,1,2\}\{1,2\}\) because \(\{0,1\} \leq \{0,1,2\}\)
    and \(\{1\} \leq \{1,2\}\), and conversely
    \(\{0,1\}\{1\} \not\leq \{0\}\{1,2\}\) because \(\{0,1\} \not\leq \{0\}\).
    The join and meet are computed by taking the join and meet of each
    component: \(
    \{0,1\}\{1\} \vee \{0,1,2\}\{1,2\} = \{0,1,2\}\{1,2\}
    \) and \(
    \{0,1\}\{1\} \vee \{0\}\{1,2\} = \{0,1\}\{1,2\}
    \).
\end{remark}

Values in a circuit signature are interpreted as elements of a finite
lattice; now the concepts of no information and all information are encoded
as the \(\bot\) and \(\top\) element.
Now the primitives in the signature must be interpreted.
Clearly they should be interpreted as functions between the values, but these
functions must respect the order on the value lattice; one should not be able to
lose information by performing a computation.

\begin{definition}
    \index{monotone}
    Let \((A, \leq_A)\) and \((B, \leq_B)\) be partial orders.
    A function \(\morph{f}{A}{B}\) is \emph{monotone} if, for every \(x, y \in A\),
    \(x \leq_A y\) if and only if \(f(x) \leq_B f(y)\).
\end{definition}

Another condition on the primitives is that when all the inputs to a primitive
are \(\bot\), then it should produce \(\bot\); we cannot produce information
from nothing.

\begin{definition}
    \index{\(\bot\)-preserving}
    Let \(A,B\) be finite lattices, where \(\bot_A\) is the least element of
    \(A\) and \(\bot_B\) is the least element of \(B\).
    A function \(\morph{f}{A}{B}\) is \emph{\(\bot\)-preserving} if
    \(f(\bot_A) = \bot_B\).
\end{definition}

Assigning interpretations to the combinational components of a circuit sets the
stage for the entire denotational semantics.

\begin{definition}[Interpretation]
    \index{interpretation}
    \nomenclature{\(\interpretation\)}{circuit interpretation}
    For a signature \(
    \signature = (
    \values, \bullet, \circuitsignaturegates, \circuitsignaturearity,
    \circuitsignaturecoarity
    )\), an \emph{interpretation} of
    \(\signature\) is a tuple \((\sqsubseteq, \gateinterpretation)\) where
    \((\values, \sqsubseteq)\) is a lattice with \(\bullet\) as the least
    element, and \(\gateinterpretation\) maps each
    \(p \in \circuitsignaturegates\) to a \(\bot\)-preserving monotone function
    \(
    \valuetuple{\circuitsignaturearity(p)}
    \to
    \valuetuple{\circuitsignaturecoarity(p)}
    \).
\end{definition}

\begin{example}\label{ex:belnap-interpretation}
    \index{Belnap!interpretation}
    Recall the Belnap signature \(
    \belnapsignature = (
    \belnapvalues, \bot, \belnapgates, \belnaparity, \belnapcoarity
    )
    \) from \cref{ex:belnap-signature}.
    We assign a partial order \(\leq_\mathsf{B}\) to values in
    \(\belnapvalues\) as follows:

    \begin{center}
        \begin{tikzcd}
            & \top & \\
            \belnapfalse \arrow[dash]{ur} & & \belnaptrue \arrow[dash]{ul} \\
            & \bot \arrow[dash]{ul} \arrow[dash]{ur} &
        \end{tikzcd}
    \end{center}

    The gate interpretation \(\belnapgateinterpretation\) has action \(
    \andgate \mapsto \land, \orgate \mapsto \lor, \notgate \mapsto \neg
    \) where \(\land\), \(\lor\) and \(\neg\) are defined by the following
    truth tables~\cite{belnap1977useful}:

    \begin{center}
        \begin{tabular}{|c|cccc|}
            \hline
            \(\land\)        & \(\bot\)         & \(\belnapfalse\) & \(\belnaptrue\)  & \(\top\)         \\
            \hline
            \(\bot\)         & \(\bot\)         & \(\belnapfalse\) & \(\bot\)         & \(\belnapfalse\) \\
            \(\belnapfalse\) & \(\belnapfalse\) & \(\belnapfalse\) & \(\belnapfalse\) & \(\belnapfalse\) \\
            \(\belnaptrue\)  & \(\bot\)         & \(\belnapfalse\) & \(\belnaptrue\)  & \(\top\)         \\
            \(\top\)         & \(\belnapfalse\) & \(\belnapfalse\) & \(\top\)         & \(\top\)         \\
            \hline
        \end{tabular}
        \quad
        \begin{tabular}{|c|cccc|}
            \hline
            \(\lor\)         & \(\bot\)        & \(\belnapfalse\) & \(\belnaptrue\) & \(\top\)        \\
            \hline
            \(\bot\)         & \(\bot\)        & \(\bot\)         & \(\belnaptrue\) & \(\belnaptrue\) \\
            \(\belnapfalse\) & \(\bot\)        & \(\belnapfalse\) & \(\belnaptrue\) & \(\top\)        \\
            \(\belnaptrue\)  & \(\belnaptrue\) & \(\belnaptrue\)  & \(\belnaptrue\) & \(\belnaptrue\) \\
            \(\top\)         & \(\belnaptrue\) & \(\top\)         & \(\belnaptrue\) & \(\top\)        \\
            \hline
        \end{tabular}
        \quad
        \begin{tabular}{|c|c|}
            \hline
            \(\neg\)         &                  \\
            \hline
            \(\bot\)         & \(\bot\)         \\
            \(\belnaptrue\)  & \(\belnapfalse\) \\
            \(\belnapfalse\) & \(\belnaptrue\)  \\
            \(\top\)         & \(\top\)         \\
            \hline
        \end{tabular}
    \end{center}

    The Belnap interpretation is then \(
    (\leq_\mathsf{B}, \belnapgateinterpretation)
    \).
    An online tool for experimenting with the Belnap interpretation can be found
    at \url{https://belnap.georgejkaye.com}.
\end{example}
\subsection{Denotational semantics of combinational circuits}

The semantic domain for \emph{combinational} circuits is straightforward: each
circuit maps to a monotone function.

\begin{definition}
    \index{\(\funci\)}
    \nomenclature{\(\funci\)}{PROP of \(\bot\)-preserving monotone functions}
    Let \(\funci\) be the PROP in which the morphisms
    \(m \to n\) are the \(\bot\)-preserving monotone
    functions \(\valuetuple{m} \to \valuetuple{n}\).
\end{definition}

The natural way to map from a PROP of syntax into a PROP of semantics is to
use a functor.

\begin{definition}
    \index{PROP morphism}
    A \emph{PROP morphism} is a strict symmetric monoidal functor between two
    PROPs i.e.\ a functor that preserves the strict symmetric monoidal
    structure.
\end{definition}

\begin{definition}
    \nomenclature{\(\circuittofunci\)}{PROP morphism assigning semantics to circuits in \(\ccircsigma\)}
    Let \(\morph{\circuittofunci}{\ccircsigma}{\funci}\) be the PROP morphism
    with action defined as%
    \vspace{-\abovedisplayskip}
    % \vspace{-\parskip}
    \begin{center}
        \begin{minipage}{0.32\textwidth}
            \centering
            \begin{align*}
                \circuittofunci[
                    \iltikzfig{strings/structure/comonoid/copy}[colour=comb]
                ]
                 & \coloneqq
                (v) \mapsto (v, v)
                \\
                \circuittofunci[
                    \iltikzfig{strings/structure/monoid/merge}[colour=comb]
                ]
                 & \coloneqq
                (v, w) \mapsto v \sqcup w
            \end{align*}
        \end{minipage}
        \quad
        \begin{minipage}{0.25\textwidth}
            \centering
            \begin{align*}
                \circuittofunci[
                    \iltikzfig{strings/structure/comonoid/discard}[colour=comb]
                ]
                 & \coloneqq
                (v) \mapsto ()
                \\
                \circuittofunci[
                    \iltikzfig{strings/structure/monoid/init}[colour=comb]
                ]
                 & \coloneqq
                () \mapsto \bot
            \end{align*}
        \end{minipage}
        \quad
        \begin{minipage}{0.25\textwidth}
            \centering
            \vspace{1.5em}
            \(\circuittofunci[
                \iltikzfig{circuits/components/gates/gate}[gate=p,dom=m,cod=n]
            ]
            \coloneqq
            \gateinterpretation[p]
            \)
        \end{minipage}
    \end{center}
\end{definition}

\begin{remark}
    One might wonder why the fork and the join have different semantics, as they
    would be physically realised by the same wiring.
    This is because digital circuits have a notion of \emph{static causality}:
    outputs can only connect to inputs.
    This is why the semantics of combinational circuits is \emph{functions} and
    not \emph{relations}.

    In real life one could force together two digital devices, but this might
    lead to undefined behaviour in the digital realm.
    This is reflected in the semantics by the use of the join; for example, in
    the Belnap interpretation if one tries to join together \(\belnaptrue\) and
    \(\belnapfalse\) then the overspecified \(\top\) value is produced.
\end{remark}

\subsection{Denotational semantics of sequential circuits}

As one might expect, sequential circuits are slightly trickier to deal with.
In a combinational circuit, the output only depends on the inputs at the current
cycle, but for sequential circuits inputs can affect outputs many cycles after
they occur.

We therefore have to reason with \emph{infinite sequences} of inputs rather than
individual values; these are known as \emph{streams}.

\begin{notation}
    \index{streams}
    \nomenclature{\(\stream{A}\)}{set of streams of of \(A\)}
    Given a set \(A\), we denote the set of streams (infinite sequences) of
    \(A\) by \(\stream{A}\).
    As a stream can equivalently be viewed as a function \(\nat \to A\), we
    write \(\sigma(i) \in A\) for the \(i\)-th element of stream
    \(\sigma \in \stream{A}\).
    Individual streams are written as \(
    \sigma \in \stream{A}
    \coloneqq
    \sigma(0) \streamcons \sigma(1) \streamcons
    \sigma(2) \streamcons \cdots
    \).
\end{notation}

Streams can be viewed a bit like lists, in that they have a head component and
an (infinite) tail component.

\begin{definition}[Operations on streams]\label{def:stream-operations}
    \index{initial value}
    \index{stream derivative}
    Given a stream \(\sigma \in \stream{A}\), its \emph{initial value}
    \(\streaminit(\sigma) \in A\) is defined as \(\sigma \mapsto \sigma(0)\)
    and its \emph{stream derivative} \(\streamderv(\sigma) \in \stream{A}\) is
    defined as \(\sigma \mapsto (i \mapsto \sigma(i+1))\).
\end{definition}

\begin{notation}
    For a stream \(\sigma\) with initial value \(a\) and stream derivative
    \(\tau\) we write it as \(\sigma \coloneqq a \streamcons \tau\).
\end{notation}

Streams will serve as the inputs and outputs to circuits, so the denotations of
sequential circuits will be \emph{stream functions}, which consume and produce
streams.
Just like with functions, we cannot claim that all streams are the
denotations of sequential circuits.

\begin{definition}[Causal stream function~\cite{rutten2006algebraic}]
    \index{stream function!causal}
    \index{causal stream function}
    A stream function \(\morph{f}{\stream{A}}{\stream{B}}\) is \emph{causal} if,
    for all \(i \in \nat\) and \(\sigma,\tau \in \stream{A}\) we have that
    \(\sigma(j) = \tau(j)\) for all \(j \leq i\), then
    \(f(\sigma)(i) = f(\tau)(i)\).
\end{definition}

Causality is a form of continuity; a causal stream function is a stream function
in which the \(i\)-th element of its output stream only depends on elements
\(0\) through \(i\) inclusive of the input stream; it cannot look into the
future.
A neat consequence of causality is that it enables us to lift the stream
operations of initial value and stream derivative to stream \emph{functions}.

\begin{definition}[Initial output~\cite{rutten2006algebraic}]
    \index{initial output}
    \nomenclature{\(\initialoutput{f}{a}\)}{initial output of \(f\) on input \(a\)}
    For a causal stream function \(\morph{f}{\stream{A}}{\stream{B}}\) and
    \(a \in A\), the \emph{initial output of \(f\) on input \(a\)} is an element
    \(\initialoutput{f}{a} \in A\) defined as
    \(\initialoutput{f}{a} \coloneqq \streaminit(f(a \streamcons \sigma))\) for
    an arbitrary \(\sigma \in \stream{A}\).
\end{definition}

Since \(f\) is causal, the stream \(\sigma\) in the definition of initial
output truly is arbitrary; the \(\streaminit\) function only depends on the
first element of the stream.

\begin{definition}[Functional stream derivative~\cite{rutten2006algebraic}]
    \index{functional stream derivative}
    \nomenclature{\(\streamderivative{f}{a}\)}{functional stream derivative of \(f\) on input \(a\)}
    For a stream function \(\morph{f}{\stream{A}}{\stream{B}}\) and
    \(a \in A\), the
    \emph{functional stream derivative of \(f\) on input \(a\)} is a stream
    function \(\streamderivative{f}{a}\) defined as \(
    \streamderivative{f}{a}
    \coloneqq
    \sigma \mapsto \streamderv(f(a \streamcons \sigma))
    \).
\end{definition}

The functional stream derivative \(\streamderivative{f}{a}\) is a new stream
function which acts as \(f\) would `had it seen \(a\) first'.

\begin{remark}
    One intuitive way to view stream functions is to think of them as the states
    of some Mealy machine; the initial output is the output given some input,
    and the functional stream derivative is the transition to a new state.
    As with most things in mathematics, this is no coincidence; there is a
    homomorphism from any Mealy machine to a stream function.
    We will exploit this fact in the next section.
\end{remark}

This leads us to the next property of denotations of sequential circuits.
Although they may have infinitely many inputs and outputs, circuits themselves
are built from a finite number of components.
This means they cannot specify infinite \emph{behaviour}.

\begin{notation}
    \nomenclature{\(\streamderivative{f}{\listvar{a}}\)}{repeated application of stream derivative of \(f\) for a word \(\listvar{a}\) of inputs}
    Given a finite word \(\listvar{a} \in \freemon{A}\), we abuse notation
    and write \(\streamderivative{f}{\listvar{a}}\) for the repeated
    application of the functional stream derivative for the elements of
    \(\listvar{a}\), i.e.\ \(
    \streamderivative{f}{\varepsilon} \coloneqq f
    \) and \(
    \streamderivative{f}{a \streamcons \listvar{b}} \coloneqq
    \streamderivative{(\streamderivative{f}{a})}{\listvar{b}}
    \).
\end{notation}

\begin{definition}
    \index{finitely specified stream function}
    \index{stream function!finitely specified}
    Given a stream function \(\morph{f}{\stream{A}}{\stream{B}}\), we say it is
    \emph{finitely specified} if the set \(\{
    \streamderivative{f}{\listvar{a}} \,|\, \listvar{a} \in \freemon{A}
    \}\) is finite.
\end{definition}

As the components of our circuits are monotone and \(\bot\)-preserving, a
denotational semantics for circuits must also be monotone and
\(\bot\)-preserving.
This means we need to lift the order on values to an order on streams.

\begin{notation}
    For a poset \((A, \leq_A)\) and streams \(\sigma,\tau \in \stream{A}\), we
    say that \(\sigma \leq_{\stream{A}} \tau\) if \(\sigma(i) \leq_A \tau(i)\)
    for all \(i \in \nat\).
\end{notation}

For these three properties to be suitable as a denotational semantics for
sequential circuits, we must show that stream functions with these
properties form a category we can map into from \(\scircsigma\).
We will first show that these categories form a symmetric monoidal category, so
we need a suitable candidate for composition and tensor.
There are fairly obvious choices here: for the former we use regular function
composition and for the latter we use the Cartesian product.

\begin{lemma}\label{lem:causality-preserved}
    Causality is preserved by composition and Cartesian product.
\end{lemma}
\begin{proof}
    If the \(i\)-th element of two stream functions \(f\) and \(g\) only depends
    on the first \(i+1\) elements of the input, then so will their composition
    and product.
\end{proof}

\begin{lemma}\label{lem:finitely-specified-preserved}
    Finite specification is preserved by composition and Cartesian
    product.
\end{lemma}
\begin{proof}
    For both the composition and product of two stream functions \(f\) and
    \(g\), the largest the set of stream derivatives could be is the product of
    stream derivatives of \(f\) and \(g\), so this will also be finite.
\end{proof}

\begin{lemma}\label{lem:monotonicity-preserved}
    \(\bot\)-preserving monotonicity is preserved by composition and Cartesian
    product.
\end{lemma}
\begin{proof}
    The composition and product of any monotone function is monotone, and must
    preserve the \(\bot\) element.
\end{proof}

As the categorical operations preserve the desired properties, these stream
functions form a PROP.

\begin{proposition}
    \index{\(\streami\)}
    \nomenclature{\(\streami\)}{PROP of causal, finitely specified, and \(\bot\)-preserving monotone stream functions}
    There is a PROP \(\streami\) in which the morphisms \(m \to n\) are the
    causal, finitely specified and \(\bot\)-preserving monotone stream
    functions \(\valuetuplestream{m} \to \valuetuplestream{n}\).
\end{proposition}
\begin{proof}
    Identity is the identity function, the symmetry swaps streams, composition
    is composition of functions, and tensor product on morphisms
    \(\morph{f}{\valuetuplestream{m}}{\valuetuplestream{n}}\) and
    \(\morph{g}{\valuetuplestream{p}}{\valuetuplestream{q}}\) is the Cartesian
    product of functions composed with the components of the isomorphism
    \(\valuetuplestream{m} \times \valuetuplestream{n}
    \cong \valuetuplestream{m+n}\).

    As these constructs satisfy the categorical axioms, and as function
    composition and Cartesian product preserve causality
    (\cref{lem:causality-preserved}),
    finite specification (\cref{lem:finitely-specified-preserved}),
    and monotonicity (\cref{lem:monotonicity-preserved}), this data defines a
    symmetric monoidal category.
\end{proof}

Modelling sequential circuits as stream functions deals with temporal
aspects, but what about feedback?
As the assignment of denotations needs to be compositional, we need
to map the trace on \(\scircsigma\) to a trace on \(\streami\).
A usual candidate for the trace when considering partially ordered settings is
the \emph{least fixed point}.

\begin{definition}[Least fixed point]
    \index{least fixed point}
    For a poset \((A, \leq)\) and function \(\morph{f}{A}{A}\), the least
    fixed point of \(f\) is a value \(\mu_f\) such that \(f(\mu_f) = f\) and,
    for all values \(v\) such that \(f(v) = v\), \(\mu_f \leq v\).
\end{definition}

Least fixed points are ubiquitious in program semantics, where they are often
used to model \emph{recursion}; since feedback is a form of recursion it seems
apt that we should also follow this route.
As fixed points are so important, they are the subject of many theorems; one
that will come in very useful for us is the \emph{Kleene fixed-point theorem},
which is concerned with a special class of monotone functions.

\begin{definition}[Directed subset]
    \index{directed subset}
    For a poset \((A, \leq)\), a non-empty subset \(B \subseteq A\) is called a
    \emph{directed subset} if every pair of elements in \(B\) has an upper bound
    in \(B\).
    If this set has a join \(\bigvee B\) then this element is called a
    \emph{directed join}.
\end{definition}

\begin{notation}[Image]
    \index{image}
    \nomenclature{\(f[A]\)}{image of set \(A\) under function \(f\)}
    For a function \(\morph{f}{A}{B}\) and subset \(C \subseteq A\), we write
    \(f[C] \subseteq B\) for the \emph{image} of \(C\) under \(f\).
\end{notation}

\begin{definition}[Scott continuity]
    \index{Scott continuity}
    \index{Scott-continuous}
    Given two posets \((A, \leq_A)\) and \((B, \leq_B)\), a function
    \(\morph{f}{A}{B}\) is \emph{Scott-continuous} if for every directed
    subset \(C \subseteq A\) it holds that \(f(\bigvee_A C) = \bigvee_B(f[C])\)
    i.e.\ \(f\) preserves directed joins.
\end{definition}

\begin{theorem}[Kleene fixed-point theorem~\cite{tarski1955latticetheoretical}]
    \index{Kleene fixed-point theorem}
    Let \((A, \leq)\) be a poset in which each of its directed subsets has a
    join, and let \(\morph{f}{L}{L}\) be a Scott-continuous function.
    Then \(f\) has a least fixed point, defined as \(
    \bot \vee f(\bot) \vee f(f(\bot)) \vee \cdots
    \).
\end{theorem}

\begin{remark}
    The Kleene fixed-point theorem was not actually proved by Kleene, but is
    only named after him!
    The result is often instead attributed to Tarski.
\end{remark}

As we have so far only considered monotone functions, it is useful to get some
intuition for what Scott-continuity brings to the table.

\begin{example}\label{ex:directed-subsets}
    An example of a directed subset of \(\belnapvalues^\omega\) is the set \(T\)
    defined as \(\{\belnaptrue^n\bot \,|\, n \in \nat\}\); the join of this set
    is \(\belnaptrue^\omega\). One Scott-continuous function \(
    \belnapvalues^\omega \to \belnapvalues^\omega
    \) is defined as \(f(\sigma)(0) = \bot\) and
    \(f(\sigma)(i+1) = f(sigma)(i)\); this is Scott-continuous because finding
    the join of a set and then prepending it with \(\bot\) is the same as
    prepending \(\bot\) to each stream in the set and then finding their join.

    An example of a stream function that is monotone but \emph{not}
    Scott-continuous is the function defined as \(
    g(\belnaptrue^n\bot^\omega) \coloneqq \belnapfalse^\omega (n )
    \) and \(g(\belnaptrue^\omega) \coloneqq \top^\omega\) (the other inputs
    do not matter for this example).
    We can show this is not Scott-continuous by considering the set \(T\) above,
    as \(f(\bigsqcup T) = f(\belnaptrue^\omega) = \top^\omega \neq
    \belnapfalse^\omega = \bigsqcup f[T]
    \).
    However, note that this function is \emph{not} causal: as
    \(\belnaptrue^\omega\) has the same prefix as every element in \(T\),
    \(f(\belnaptrue^\omega)\) must also share output prefixes.
\end{example}

So far we have not explicitly enforced Scott-continuity on stream functions; it
turns that it is implied by causality and monotonicity.

\begin{proposition}\label{prop:monotone-causal-scott}
    Let \((A, \leq_A)\) and \((B, \leq_B)\) be finite lattices, and let
    \((\stream{A}, \leq_{\stream{A}})\) and \((\stream{B}, \leq_{\stream{B}})\)
    be the induced lattices on streams.
    Then a causal and monotone function \(\stream{A} \to \stream{B}\) must also
    be Scott-continuous.
\end{proposition}
\begin{proof}
    Consider a directed subset \(D \subseteq \stream{A}\); we need to show that
    for an arbitrary causal, monotone and finitely specified function \(f\) we
    have that \(f\left(\bigvee D\right) = \bigvee f[D]\).

    First consider the case when there is a greatest element in \(D\).
    In this case, \(\bigvee D\) must be the greatest element, and as such
    \(\bigvee D \in D\).
    As \(f\) is monotone then \(f(\bigvee D)\) must be the greatest element in
    \(f[D]\); subsequently, \(f\left(\bigvee D\right) = \bigvee f[D]\).

    Now consider the case where there is no greatest element in \(D\) and
    subsequently \(\bigvee D \not\in D\); if there is no greatest element,
    \(D\) must be infinite.
    Even though it is infinite, \(D\) is a directed subset so each pair of
    elements must have an upper bound, and as \(\leq_{\stream{A}}\) is computed
    pointwise by using \(\leq_A\) we can consider what the upper bounds are
    pointwise too.
    Because \(A\) is finite, there cannot be an infinite chain of upper bounds
    for each element \(i\); there must exist an element \(a_i \in A\) such that
    \(D\) contains infinitely many streams \(\sigma\) such that
    \(\sigma(i) = a_i\).
    This means that \(\left(\bigvee D\right)(i) = a\), so every prefix of
    \(\bigvee D\) must exist as a prefix of a stream in \(D\).
    As \(f\) is causal, for each prefix
    \(f\left(\bigvee D\right)\) there must also exist a \(d \in D\) such that
    \(f(d)\) has that prefix, and as such
    \(\bigvee f[D] = f\left(\bigvee D\right)\).
\end{proof}

This means we can use the Kleene fixed point theorem as a tool to show that the
least fixed point is a trace on \(\streami\).

\begin{notation}[Concatenation]
    \index{concatenation}
    \nomenclature{\(\concat\)}{concatenation operation for tuples}
    For a set \(A\), \(\listvar{a} \in A^m\), and \(\listvar{b} \in A^n\), we
    write \(\listvar{a} \concat \listvar{b}\) for the \emph{concatenation}
    of \(\listvar{a}\) and \(\listvar{b}\): the tuple of length \(m+n\)
    containing the elements of \(\listvar{a}\) followed by the elements of
    \(\listvar{b}\).
\end{notation}

\begin{notation}[Projection]
    \index{projection}
    \nomenclature{\(\proj{i}\)}{zero-indexed \(i\)-th projection}
    For a set \(A\), let \(\listvar{a} \in A^m\) and \(\listvar{b} \in A^n\).
    Then for their concatenation \(
    \listvar{c} \coloneqq \listvar{a} \concat \listvar{b} \in A^{m+n}\),
    we write \(\proj{0}(\listvar{c}) = \listvar{a}\) and
    \(\proj{1}(\listvar{c}) = \listvar{b}\) for the \emph{projections}.
\end{notation}

\begin{lemma}\label{lem:lfp-stream-function}
    Given a morphism \(
    \morph{f}{\valuetuplestream{x+m}}{\valuetuplestream{x+n}}
    \in \streami
    \), and stream \(\sigma \in \valuetuplestream{m}\), the function \(
    \tau \mapsto \proj{0}\mleft(f(\tau,\sigma)\mright)
    \) has a least fixed point.
\end{lemma}
\begin{proof}
    The function \(\tau \mapsto \proj{0}\mleft(f(\tau,\sigma)\mright)\) is
    causal and monotone because \(f\) and the projection function are
    causal and monotone, so it is Scott-continuous by
    \cref{prop:monotone-causal-scott}.
    By the Kleene fixed point theorem, this function has a least fixed point,
    defined as \(
    \proj{0}\mleft(f(\bot^\omega, \sigma)\mright) \ljoin
    \proj{0}\mleft(f(\proj{0}\mleft(f(\bot^\omega, \sigma)\mright), \sigma)\mright) \ljoin
    \cdots
    \).
\end{proof}

We must show that this notion of least fixed point is a trace on \(\streami\).
The first step is to show that taking the least fixed point of a stream function
in \(\streami\) produces another causal, finitely specified,
\(\bot\)-preserving, and monotone stream function.
The original proof idea for this is due to David Sprunger, and relies on an
ordering on stream functions themselves.

\begin{definition}\label{def:state-order}
    Let \(A\) and \(B\) be posets and let
    \(\morph{f, g}{\stream{A}}{\stream{B}}\) be stream functions.
    We say \(f \stateorder g\) if \(f(\sigma) \leq_{\stream{B}} g(\sigma)\)
    for all \(\sigma \in \stream{A}\).
\end{definition}

\begin{theorem}\label{thm:trace-well-defined}
    For a function \(\morph{f}{\valuetuplestream{x+m}}{\valuetuplestream{x+n}}\),
    let \(\mu_f(\sigma)\) be the least fixed point of the function \(
    \tau \mapsto \proj{0}\mleft(f(\tau,\sigma)\mright)
    \).
    Then, the stream function \(
    \sigma \mapsto \proj{1}\mleft(f(\mu_f(\sigma), \sigma)\mright)
    \) is in \(\streami\).
\end{theorem}
\begin{proof}
    To show this, we need to prove that
    \(\sigma \mapsto \proj{1}\mleft(f(\mu_f(\sigma)\sigma)\mright)\) is in
    \(\streami\): it is causal, finitely specified, \(\bot\)-preserving and
    monotone.

    Since \(
    \morph{f}{\valuetuplestream{x+m}}{\valuetuplestream{x+n}}
    \) is a morphism of \(\streami\), it has finitely many stream derivatives.
    For each stream derivative \(\streamderivative{f}{\,\listvar{w}}\), let the
    function \(
    \morph{
        \widehat{\streamderivative{f}{\listvar{w}}}
    }{
        \valuetuplestream{x+m}
    }{
        \valuetuplestream{x}
    }
    \) be defined as \(
    \tau\sigma
    \mapsto
    \proj{0}(\streamderivative{f}{\listvar{w}}(\tau\sigma))
    \).
    Note that each of these functions are causal, \(\bot\)-preserving, and
    monotone, because they are constructed from pieces that are causal
    \(\bot\)-preserving and monotone.

    In particular, \(\mu_f(\sigma)\) is the least fixed point of
    \(\widehat{f_\varepsilon}\left((-)\sigma\right)\).
    Using the Kleene fixed point theorem, the least fixed point of
    \(\widehat{f}((-)\sigma)\) can be obtained by composing
    \(\widehat{f}((-)\sigma)\) repeatedly with itself.
    This means that \(
    \mu_f(\sigma)
    =
    \bigsqcup_{k \in \nat} \widehat{f^k}(\bot^\omega,\sigma)
    \) where \(\widehat{f^k}\) is the \(k\)-fold composition of \(f(-,\sigma)\)
    with itself, i.e.\ \(\widehat{f^0}(\tau\sigma) = \tau\) and \(
    \widehat{f^{k+1}}(\tau\sigma)
    =
    \widehat{f}\left(\left(\widehat{f^{k}}(\sigma, \tau)\right)\sigma\right)
    \).
    That the mapping \(\mu_f\) is causal and monotone is
    straightforward: each of the functions in the join is causal and monotone,
    and join preserves these properties.
    It remains to show that this mapping has finitely many stream derivatives.

    When equipped with \(\stateorder\), the set of functions
    \(\valuetuplestream{x+m} \to \valuetuplestream{x}\)
    is a poset, of which
    \(\{\widehat{f_w} \,|\, w \in (\valuetuple{x+m})^\star\}\)
    is a finite subset.
    Restricting the ordering \(\stateorder\) to this set yields a finite poset.
    Since this poset is finite, the set of strictly increasing sequences in this
    poset is also finite.
    We will now demonstrate a relationship between these sequences and stream
    derivatives of \(\mu_f\).

    Suppose \(
    S = \widehat{f_{\,\listvar{w_0}}} \prec \widehat{f_{\,\listvar{w_1}}} \prec
    \cdots \prec \widehat{f_{\,\listvar{w_{\ell-1}}}}
    \) is a strictly increasing sequence of length \(\ell\) in the set of stream
    functions \(
    \{\widehat{f_w} \,|\, w \in (\valuetuple{x+m})^\star\}
    \).
    We define a function \(
    \morph{g_S}{
        \valuetuplestream{m}
    }{
        \valuetuplestream{x}
    }
    \) as \(
    (\sigma) \mapsto \bigsqcup_{k \in \nat} g_k(\sigma)
    \) where \[
        g_k(\sigma) =
        \begin{cases}
            \bot^\omega                                                              & \text{ if } k = 0              \\
            \widehat{f_{\,\listvar{w_k}}}(\left(g_{k-1}(\sigma)\right)\sigma)        & \text{ if } 1 \leq k \leq \ell \\
            \widehat{f_{\,\listvar{w_{\ell-1}}}}(\left(g_{k-1}(\sigma)\right)\sigma) & \text{ if } \ell < k
        \end{cases}.
    \]
    Let the set \(G \coloneqq \{
    g_S \,|\, S \text{ is a strictly increasing sequence}
    \}\).
    When \(S\) is set to the one-item sequence \(\widehat{f}\), \(g_S\) is
    \(\mu_f\), so \(\mu_f \in G\).
    As \(G\) is finite, this means that if \(G\) is closed under stream
    derivative, \(\mu_f\) has finitely many stream derivatives.
    Any element of \(G\) is either \(\bot^\omega\) or has the form \(
    \sigma
    \mapsto
    \widehat{\streamderivative{f}{\,\listvar{w}}}(g_k(\sigma)\sigma)
    \) for some \(\sigma \in \valuetuplestream{m}\) and
    \(k > 0\).
    As \(\bot^\omega\) is its own stream derivative, we need to show that
    applying stream derivative to the latter produces another element of \(G\).
    \begin{align*}
        \streamderivative{\sigma \mapsto \left(\widehat{\streamderivative{f}{\, \listvar{w}}}(\left(g_{k-1}(\sigma)\right)\sigma)\right)}{ab}
         & = \sigma \mapsto \streamderv\left(\widehat{\streamderivative{f}{\, \listvar{w}}}(ab \streamcons \left(g_{k-1}(\sigma)\right)\sigma)\right)              \\
         & = \sigma \mapsto \streamderv\left(\proj{0}\mleft(\streamderivative{f}{\, \listvar{w}}(ab \streamcons \left(g_{k-1}(\sigma)\right)\sigma)\mright)\right) \\
         & = \sigma \mapsto \proj{0}\mleft(\streamderv\left(\streamderivative{f}{\, \listvar{w}}(ab \streamcons \left(g_{k-1}(\sigma)\right)\sigma)\right)\mright) \\
         & = \sigma \mapsto \proj{0}\mleft(\streamderivative{\left(\streamderivative{f}{\, \listvar{w}}(\left(g_{k-1}(\sigma)\right)\sigma)\right)}{ab}\mright)    \\
         & = \sigma \mapsto \proj{0}\mleft(\streamderivative{f}{\, ab \streamcons \listvar{w}}(\left(g_{k-1}(\sigma)\right)\sigma)\mright)                         \\
         & = \sigma \mapsto \widehat{\streamderivative{f}{\, ab \streamcons \listvar{w}}}(\left(g_{k-1}(\sigma)\right)\sigma)
    \end{align*}
    As \(\proj{0}\mleft(\streamderivative{f}{ab \streamcons \listvar{w}}\mright)\)
    is in \(G\), the latter  is closed under stream derivative.
    Subsequently, \(\mu_f\) has finitely many stream derivatives.

    This means that all the components of
    \(\sigma \mapsto \proj{1}(f(\mu_f(\sigma)\sigma))\) are causal, monotone and
    finitely specified, and as these properties are preserved by composition,
    the composite must also have them, so
    \(\sigma \mapsto \proj{1}(f(\mu_f(\sigma)\sigma))\) is in \(\streami\).
\end{proof}

Even if \(\streami\) is closed under least fixed point, this does not mean that
it is a valid trace.
To verify this we must establish that the categorical axioms of the trace hold.

\begin{theorem}
    A trace on \(\streami\) is given for a function \(
    \morph{f}{\valuetuplestream{x+m}}{\valuetuplestream{x+n}}
    \) by the stream function \(
    \sigma \mapsto \proj{1}(f(\mu_f(\sigma), \sigma))
    \), where \(\mu_f(\sigma)\) is the least fixed point of the function \(
    \tau \mapsto \proj{0}\mleft(f(\tau,\sigma)\mright)
    \) for fixed \(\sigma\).
\end{theorem}
\begin{proof}
    By \cref{thm:trace-well-defined}, \(\streami\) is closed under taking the
    least fixed point, so we just need to show that the axioms of STMCs hold.
    Most of these follow in a straightforward way; the only interesting one is
    the sliding axiom.
    We need to show that for stream functions \(
    \morph{f}{\valuetuplestream{x+m}}{\valuetuplestream{y+n}}
    \) and \(
    \morph{g}{\valuetuplestream{y}}{\valuetuplestream{x}}
    \), we have that \(
    \trace{y}{(\tau, \sigma) \mapsto f(g(\tau), \sigma)}
    =
    \trace{x}{
        (\tau, \sigma)
        \mapsto
        g(\proj{0}\mleft(f(\tau, \sigma)\mright),
        \proj{1}\mleft(f(\tau, \sigma)\mright))
    }
    \).

    Let \(l \coloneqq (\tau, \sigma) \mapsto f(g(\tau), \sigma)\) and
    \(r \coloneqq (\tau, \sigma)
    \mapsto
    g(\proj{0}\mleft(f(\tau, \sigma)\mright),
    \proj{1}\mleft(f(\tau, \sigma)\mright))\); we must apply the candidate
    trace construction to both of these and check they are equal.
    For \(l\), the least fixed point of \(
    \tau \mapsto \proj{0}\mleft(f(g(\tau), \sigma)\mright)
    \) is \[
        \mu_l(\sigma) =
        \proj{0}\mleft(f(g(\bot^\omega), \sigma)\mright) \ljoin
        \proj{0}\mleft(f(g(\proj{0}\mleft(f(g(\bot^\omega), \sigma))\mright), \sigma)\mright) \ljoin
        \cdots.\]
    Plugging this into the candidate trace construction we have that \begin{align*}
         &
        \sigma \mapsto \proj{1}\mleft(f(g(\mu_l^l(\sigma)), \sigma)\mright)
        \\
         & \qquad=
        \sigma \mapsto \proj{1}\mleft(f(g(\proj{0}\mleft(f(g(\dots f(g(\proj{0}\mleft(f(g(\bot^\omega), \sigma)\mright)), \sigma)))\mright)), \sigma)\mright)
    \end{align*}
    For the right-hand side, the least fixed point of \(
    \tau \mapsto g(\proj{0}\mleft(f(\tau, \sigma)\mright))
    \) is \[
        \mu_r(\sigma) =
        g(\proj{0}\mleft(f(\bot^\omega, \sigma)\mright)) \ljoin
        g(\proj{0}\mleft(f(g(\proj{0}\mleft(f(g(\bot^\omega), \sigma)\mright)), \sigma)\mright)) \ljoin
        \cdots
    \]
    When plugged into the candidate trace construction this produces \begin{align*}
         & \sigma \mapsto \proj{1}\mleft(g(\proj{0}(f(\mu_r(\sigma), \sigma))), \proj{1}(f(\mu_r(\sigma), \sigma))\mright)
        \\
         & \qquad =
        \sigma \mapsto \proj{1}(f(\mu_\sigma^r, \sigma))
        \\
         & \qquad=
        \sigma \mapsto \proj{1}(f(
        g(\proj{0}\mleft(f(g(\dots f(g(\proj{0}\mleft(f(g(\bot^\omega), \sigma)\mright)), \sigma)\mright)), \sigma))))
        \\
         & \qquad=
        \sigma \mapsto \proj{1}(f(
        g(\proj{0}\mleft(f(g(\dots f(g(\proj{0}\mleft(f(\bot^\omega, \sigma)\mright)), \sigma)\mright)), \sigma))))
    \end{align*}
    Both the left-hand and the right-hand sides of the sliding equation are
    equal, so the construction is indeed a trace.
\end{proof}

We now have two traced PROPs: a \emph{syntactic} PROP of sequential circuit
terms \(\scircsigma\) and a \emph{semantic} PROP of causal, finitely
specified, monotone stream functions \(\streami\).
It would be straightforward to now define a map from circuits to these stream
functions; indeed, this is the strategy used in~\cite{ghica2024fully}.
Instead, we will first examine another structure with close links to both
circuits and stream functions; that of \emph{Mealy machines}.
The structure of Mealy machines will come in useful when considering the
\emph{completeness} of the denotational semantics.