\section{Completeness of the denotational semantics}\label{sec:denotational-completeness}

We want \(\streami\) to be a \emph{complete} denotational semantics for digital
circuits.
This means that for every stream function \(f \in \streami\), there must be at
least one one circuit in \(\scircsigma\) such that its behaviour under
\(\interpretation\) is \(f\).

\begin{corollary}\label{thm:circuit-stream-correspondence}
    \(
    \circuittostreami
    \circ
    \mealytocircuiti
    \circ
    \streamtomealyi
    =
    \id[\streami]
    \).
\end{corollary}
\begin{proof}
    This follows immediately from \cref{thm:mealy-to-circuit} and
    \cref{prop:mealy-to-stream}, as we have that \(
    \circuittostreami
    \circ
    \mealytocircuiti
    \circ
    \streamtomealyi
    =
    \mealytostreami
    \circ
    \streamtomealyi
    =
    \id[\streami]
    \).
\end{proof}

There is no isomorphism between \(\scircsigma\) and \(\streami\)
as many circuits may have the same semantics but different syntax.
Instead, we can work in \emph{equivalence classes} of syntactic circuits based
on their behaviour in \(\streami\).

\begin{definition}[Denotational equivalence]
    Two sequential circuits are \emph{denotationally equivalent}
    under \(\interpretation\), written \(
    \iltikzfig{strings/category/f}[box=f,colour=seq,dom=m,cod=n]
    \approx_{\interpretation}
    \iltikzfig{strings/category/f}[box=g,colour=seq,dom=m,cod=n]
    \) if \(
    \circuittostream[
        \iltikzfig{strings/category/f}[box=f,colour=seq]
    ]{\interpretation}
    =
    \circuittostream[
        \iltikzfig{strings/category/f}[box=g,colour=seq]
    ]{\interpretation}
    \).
    Let \(\scircsigmai\) be the result of quotienting \(\scircsigma\) by \(
    \approx_{\interpretation}
    \).
\end{definition}

Every morphism in \(\scircsigmai\) is a class of circuits that share the
same behaviour under \(\interpretation\).
As we have a PROP morphism \(\mealytocircuiti \circ \streamtomealyi\), we know
that for every such behaviour there must be at least one such syntactic circuit,
and subsequently exactly one equivalence class \(\scircsigmai\).
Using \cref{thm:circuit-stream-correspondence}, we know that all the circuits in
this equivalence class have the same behaviour as the original stream function,
so we can derive an isomorphism between \(\scircsigmai\) and
\(\streami\).

\begin{corollary}
    \(\scircsigmai \cong \streami\).
\end{corollary}

This gives us, for the first time, a fully compositional, sound and complete,
denotational semantics for sequential circuits with delay and (possibly
non-delay-guarded) feedback.
This formal model with serve as a backdrop against the operational and
algebraic semantics presented in the upcoming chapters.