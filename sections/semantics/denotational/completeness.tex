\section{Completeness of the denotational semantics}

We want \(\streami\) to be a \emph{complete} denotational semantics for digital
circuits.
This means that for every stream function \(f \in \streami\), there must be at
least one one circuit in \(\scircsigma\) such that its behaviour under
\(\interpretation\) is \(f\).

\begin{corollary}\label{thm:circuit-stream-correspondence}
    \(
    \circuittostreami
    \circ
    \mealytocircuiti
    \circ
    \streamtomealyi
    =
    \id[\streami]
    \).
\end{corollary}
\begin{proof}
    This follows immediately from \cref{thm:mealy-to-circuit} and
    \cref{prop:mealy-to-stream}, as we have that \(
    \circuittostreami
    \circ
    \mealytocircuiti
    \circ
    \streamtomealyi
    =
    \mealytostreami
    \circ
    \streamtomealyi
    =
    \id[\streami]
    \).
\end{proof}

There is no isomorphism between \(\scircsigma\) and \(\streami\)
as many circuits may have the same semantics but different syntax.

\begin{definition}[Denotational equivalence]
    Two sequential circuits are \emph{denotationally equivalent}
    under \(\interpretation\), written \(
    \iltikzfig{strings/category/f}[box=f,colour=seq,dom=m,cod=n]
    \approx_{\interpretation}
    \iltikzfig{strings/category/f}[box=g,colour=seq,dom=m,cod=n]
    \) if \(
    \circuittostream[
        \iltikzfig{strings/category/f}[box=f,colour=seq]
    ]{\interpretation}
    =
    \circuittostream[
        \iltikzfig{strings/category/f}[box=g,colour=seq]
    ]{\interpretation}
    \).
    Let \(\scircsigmai\) be the result of quotienting \(\scircsigma\) by \(
    \approx_{\interpretation}
    \).
\end{definition}

Every morphism in \(\scircsigmai\) is a class of circuits which all have the
same behaviour under \(\interpretation\).
Since each of these behaviours is a stream function in \(\streami\), we can
conclude the following.

\begin{corollary}
    \(\scircsigmai \cong \streami\).
\end{corollary}