\section{Productivity}\label{sec:productivity}

It is not possible to reduce an open circuit to some output values, as there will
be open wires awaiting the next inputs.
Nevertheless, if we precompose a circuit with some inputs we can provide some
rules for propagating them across the circuit.

Formally, for sequential circuit \(
\iltikzfig{strings/category/f}[box=f,colour=seq,dom=m,cod=n]
\) and values \(
\listvar{v} \in \valuetuplestream{m}
\), this corresponds to finding reductions such that \(
\iltikzfig{circuits/productivity/productive-goal-lhs}[box=f,input=\listvar{v},dom=m,cod=n]
\reductions
\iltikzfig{circuits/productivity/productive-goal-rhs}[box=g,output=\listvar{w},dom=m,cod=n]
\).
We first consider the combinational case, with our final global transformation.

\begin{lemma}[Streaming]\label{lem:streaming}
    The following \emph{streaming rule} is sound: \[
        \iltikzfig{circuits/axioms/generalised-streaming-lhs}[box=f]
        \reduction[(\streamingeqn)]
        \iltikzfig{circuits/axioms/generalised-streaming-rhs}[box=f]
    \]
\end{lemma}
\begin{proof}
    Once again this can be shown by considering the stream semantics.
    First note that by unfolding the notation, \(
    \iltikzfig{circuits/axioms/generalised-streaming-lhs}[box=f]
    \coloneqq
    \iltikzfig{circuits/axioms/generalised-streaming-lhs-verbose}[box=f]
    \).
    The streaming rule is then effectively `pushing' the combinational circuit
    \(\iltikzfig{strings/category/f}[box=f,colour=comb]\) across the join.

    The join is \emph{not} a natural transformation so this does not hold in
    general, but because one argument is an instantaneous value and the other
    is a delay, at least one of the inputs to the join will be \(\bot\) for a
    given circuit.
    As the interpretations of combinational circuits must be
    \(\bot\)-preserving, the circuit can safely be pushed across the join and
    delay.
\end{proof}

The streaming rule shows that when a combinational circuit is applied to an
input with an instantaneous and a delayed component, the circuit can be copied
so that one copy handles what is happening `now' and the other handles what is
happening `later'.

\begin{example}\label{ex:streaming}
    Pulsing the signals \(\belnapfalse\belnaptrue\) to the inputs of an SR NOR
    latch starts the procedure for `setting' the latch, causing it to output
    \(\belnaptrue\belnapfalse\).
    \cref{fig:sr-latch-streamed} shows how the \(\streamingeqn\) rule is applied
    to the unrolled SR NOR circuit from
    \cref{fig:sr-latch-unrolled} with these inputs to create a copy for what
    is happening `now' and another for what is happening `later'.
\end{example}

\begin{figure*}
    \centering
    \scalebox{0.45}{\iltikzfig{circuits/examples/sr-latch/circuit-streamed-3}}
    \caption{
        Applying \(\streamingeqn\) with inputs \(\belnaptrue\belnapfalse\) to
        the circuit from \cref{fig:sr-latch-unrolled}
    }
    \label{fig:sr-latch-streamed}
\end{figure*}

As there is a delay on the bottom argument of the join, the output of a streamed
circuit at the current tick is now contained entirely in the top argument of the
join.
The final rules we present will reduce this copy to values, as desired.

\begin{lemma}[Value rules]
    The following \emph{value rules} are sound:
    \begin{gather*}
        \iltikzfig{circuits/axioms/fork-lhs}[val=v]
        \reduction[(\forkeqn)]
        \iltikzfig{circuits/axioms/fork-rhs}[val=v]
        \qquad
        \iltikzfig{circuits/axioms/join-lhs}[val1=v,val2=w]
        \reduction[(\joineqn)]
        \iltikzfig{circuits/axioms/join-rhs}[val1=v,val2=w]
        \\
        \iltikzfig{circuits/axioms/stub-lhs}[val=v]
        \reduction[(\stubeqn)]
        \iltikzfig{strings/monoidal/empty}
        \qquad
        \iltikzfig{circuits/axioms/gate-lhs}[gate=p,input=\listvar{v}]
        \reduction[(\gateeqn)]
        \iltikzfig{circuits/axioms/gate-rhs}[gate=p,input=\listvar{v}]
    \end{gather*}
\end{lemma}
\begin{proof}
    Straightforward by considering the interpretations of values as stream
    functions.
\end{proof}

Reducing the `now' core is the only time in which exhaustive application is
required, as more is involved than just copying circuit components.

\begin{lemma}\label{lem:reduce-core-confluent}
    Applying the value rules is confluent.
\end{lemma}
\begin{proof}
    There are no overlaps between the rules.
\end{proof}

\begin{lemma}\label{lem:reduce-core-terminating}
    For a combinational circuit \(
    \iltikzfig{strings/category/f}[box=f,colour=comb,dom=m,cod=n]
    \) and \(\listvar{v} \in \valuetuple{m}\), there exists
    \(\listvar{w} \in \valuetuple{n}\) such that applying the value
    rules to \(
    \iltikzfig{circuits/components/circuits/f-applied}[box=f,colour=comb]
    \) terminates at \(
    \iltikzfig{circuits/components/values/vs}[val=\listvar{w}]
    \).
\end{lemma}
\begin{proof}
    By induction on the structure of \(
    \iltikzfig{strings/category/f}[box=f,colour=comb,dom=m,cod=n]
    \).
\end{proof}

These rules are all we need to propagate input values across a circuit.

\begin{corollary}\label{cor:mealy-form-productivity}
    For a circuit \(
    \iltikzfig{circuits/productivity/mealy-form-applied}[core=f,dom=m,cod=n,delay=x]
    \) there exist \(
    \listvar{t} \in \valuetuple{x}
    \) and \(
    \listvar{w} \in \valuetuple{n}
    \) such that \(
    \iltikzfig{circuits/productivity/mealy-form-applied}[core=f,dom=m,cod=n,delay=x]
    \reductions
    \iltikzfig{circuits/productivity/mealy-form-produced}[core=f,dom=m,cod=n,delay=x]
    \) by applying \(\streamingeqn\) once followed by the value rules
    exhaustively.
\end{corollary}

\begin{example}
    \cref{fig:sr-latch-value-rules} shows how the value rules are applied to
    the streamed circuit from \cref{fig:sr-latch-streamed}.
    After performing all the reductions exhaustively on the `now' circuit, the
    next state is \(\belnaptrue\), the first output is \(\bot\) and
    the second is \(\belnapfalse\).
    While the next state and second output make sense (if we apply Set, the
    state of the latch should turn true and the negated output false), the first
    output may raise eyebrows.
    This arises due to the presence of the delay; it will take another cycle to
    produce the expected output \(\belnaptrue\belnapfalse\).
\end{example}

\begin{figure*}
    \centering
    \scalebox{0.6}{\iltikzfig{circuits/examples/sr-latch/circuit-streamed-3}}
    \(\reductions\)
    \\[1em]
    \scalebox{0.6}{\iltikzfig{circuits/examples/sr-latch/reduction/step-1}}
    \(\reductions\)
    \\[1em]
    \scalebox{0.6}{\iltikzfig{circuits/examples/sr-latch/reduction/step-10}}
    \caption{
        Using the value rules to reduce the streamed SR NOR latch circuit from
        \cref{fig:sr-latch-streamed}.
    }
    \label{fig:sr-latch-value-rules}
\end{figure*}

By now putting together all the components in this section and the previous,
we have a productive strategy for processing inputs to \emph{any} sequential
circuit.

\begin{corollary}[Productivity]\label{cor:productivity}
    For sequential circuit \(
    \iltikzfig{strings/category/f}[box=f,colour=seq,dom=m,cod=n]
    \) and inputs \(\listvar{v} \in \valuetuple{m}\), there exists
    \(\listvar{w} \in \valuetuple{n}\) such that \(
    \iltikzfig{circuits/productivity/productive-goal-lhs}[box=f,input=\listvar{v},dom=m,cod=n]
    \reductions
    \iltikzfig{circuits/productivity/productive-goal-rhs}[box=g,output=\listvar{w},dom=m,cod=n]
    \) by applying \((\mealyeqn)\), \((\instantfeedbackeqn)\) and
    \((\streamingeqn)\) once successively followed by the value rules
    exhaustively.
\end{corollary}

\begin{remark}
    As we saw in \cref{cor:mealy-form-productivity}, applying
    \((\streamingeqn)\) followed by the value rules to a circuit in Mealy form
    produces another circuit in Mealy form.
    This means that for one circuit and a long waveform stream of inputs,
    \((\mealyeqn)\) and \((\instantfeedbackeqn)\) need only be applied
    \emph{once} at the very start before processing values.
\end{remark}

\begin{remark}
    This style of operational semantics differs from some other approaches in
    the area, such as the work on signal flow graphs~\cite{bonchi2021survey}.
    In these works, the operational semantics is specified in terms of
    the state transitions that take place in a circuit over time.
    For example, the rule that applies to the fork in signal flow graphs is
    \[
        t \triangleright \iltikzfig{strings/structure/comonoid/copy-notile}
        \xrightarrow[k\ k]{k}
        t+1 \triangleright \iltikzfig{strings/structure/comonoid/copy-notile}
    \]
    where \(t\) is the current timestep, \(k\) is the input signal and
    \(k\ k\) is the (forked) output signal.
    Note that the fork itself does not change; the `computation' occurring is
    contained entirely within the inputs and outputs.

    In our world of digital circuits we are more interested in propagating
    values to see how this affects the internal structure of a circuit; this is
    another instance of how we are working with a \emph{causal} rather than
    \emph{relational} semantics.
    This means we specify inputs as explicit components, and the reductions
    actually change the structure of the circuit.
    \[
        \iltikzfig{circuits/axioms/fork-lhs}
        \reduction
        \iltikzfig{circuits/axioms/fork-rhs}
    \]
\end{remark}