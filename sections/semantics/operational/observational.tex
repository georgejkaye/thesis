\section{Observational equivalence}\label{sec:observational}

In the denotational semantics, we defined the relation of
\emph{denotational equivalence}, in which circuits are related if their
denotations as streams are equal.
For operational semantics we have another notion of relation on circuits: that
of \emph{observational equivalence}.
This is due to Morris~\cite{morris1969lambdacalculus}, who named it
`extensional equivalence': essentially, two processes are observationally
equivalent if they cannot be distinguished by their input-output behaviour.

Testing for observational equivalence is traditionally performed by checking
that a program behaves the same in all \emph{contexts}.
For digital circuits, this means that for all possible streams of inputs, the
circuit produces the same outputs.
Of course, there are infinitely many streams of inputs, despite the set of
values being finite.
Fortunately, since circuits are constructed from a finite number of
\emph{components}, we need not check them all.

\begin{lemma}\label{lem:number-of-states}
    Let \(
    \iltikzfig{strings/category/f}[box=f,colour=seq]
    \) be a sequential circuit with \(c\) delay components.
    Then applying \cref{cor:productivity} successively to a Mealy form of this
    circuit will produce at most \(|\values|^c\) unique states.
\end{lemma}
\begin{proof}
    The only varying elements of the state word are contributed by
    the \(c\) delay components, as the values transition to \(\bot\).
\end{proof}

To test all of the possible internal states of a circuit, we must use
sequences of inputs long enough in time to guarantee that every possible state
of a circuit is triggered.

\begin{notation}[Waveform]\label{def:waveform}
    \index{waveform}
    The empty waveform is defined as \(
    \iltikzfig{circuits/components/waveforms/sigma0}
    \coloneqq
    \iltikzfig{strings/category/identity}[colour=comb,obj=n]
    \).
    Given values \(\overline{v} \in \valuetuple{n}\) and sequence \(
    \overline{\underline{w}} \in (\valuetuple{n})^\star
    \), the waveform for sequence \(
    \overline{v} \streamcons \overline{\underline{w}}
    \) is drawn as \(
    \iltikzfig{circuits/components/waveforms/sigmatp1-spaced}
    \coloneqq
    \iltikzfig{circuits/components/waveforms/sigmatp1-construction-spaced}
    \).
\end{notation}

As a circuit with \(c\) delay components has at most \(|\values|^c\) states,
to fully evaluate the behaviour of a circuit it suffices to check every
waveform of length \(|\values|^c + 1\).
This is because even if such a waveform causes all \(|\values|^c\) states to
occur, the final element must produce a previously visited state, as there are
no other states that could arise.

\begin{corollary}\label{cor:repeated-state}
    Given a circuit in Mealy form \(
    \iltikzfig{circuits/productivity/mealy-form}[dom=m,cod=n,delay=x,core=f]
    \) and input sequence \(\listlistvar{v} \in \valuetupleseq{m}\) of length
    \(|\values|^c + 1\), there exists a state \(
    \listvar{r} \in \valuetuple{x}
    \), an input sequence \(
    \listlistvar{u} \in \valuetupleseq{m}
    \) and output sequences \(
    \listlistvar{w},\listlistvar{z} \in \valuetupleseq{n}
    \) such that applying \cref{cor:productivity} yields the
    following reduction pattern: \begin{gather*}
        \iltikzfig{circuits/productivity/mealy-form-waveform}
        \reductions
        \iltikzfig{circuits/productivity/mealy-form-waveform-partial}[state=\listvar{r},output=\listlistvar{w},input=\listlistvar{u}]
        \reductions
        \iltikzfig{circuits/productivity/mealy-form-waveform-output-2}[state=\listvar{r},output1=\listlistvar{z},output2=\listlistvar{w}]
    \end{gather*}
\end{corollary}

This means that every possible behaviour of a circuit can be evaluated using
a finite number of sequences.
This can be used to define our notion of observational equivalence for digital
circuits.

\begin{definition}[Observational equivalence of circuits]
    \index{observational equivalence}
    \nomenclature{\(\sim_\interpretation\)}{observational equivalence of circuits}
    We say that two sequential circuits \(
    \iltikzfig{strings/category/f}[box=f,colour=seq,dom=m,cod=n]
    \) and \(
    \iltikzfig{strings/category/f}[box=g,colour=seq,dom=m,cod=n]
    \) with no more than \(c\) delays are said to be
    \emph{observationally equivalent under} \(\interpretation\), written \(
    \iltikzfig{strings/category/f}[box=f,colour=seq]
    \sim_{\interpretation}
    \iltikzfig{strings/category/f}[box=g,colour=seq]
    \) if applying productivity produces the same output
    waveforms for all input waveforms \(
    \listlistvar{v} \in \valuetupleseq{m}\) of length
    \(|\values^c| + 1\).
\end{definition}

Observational equivalence is our semantic relation for operational semantics,
which relates two circuits based on their \emph{execution}.
To ensure it is suitable, it must be sound and complete with respect to the
denotational semantics.

\begin{theorem}\label{thm:operational-denotational}
    Two sequential circuits \(
    \iltikzfig{strings/category/f}[box=f,colour=seq,dom=m,cod=n]
    \) and \(
    \iltikzfig{strings/category/f}[box=g,colour=seq,dom=m,cod=n]
    \) are observationally equivalent \(
    \iltikzfig{strings/category/f}[box=f,colour=seq,dom=m,cod=n]
    \sim_\interpretation
    \iltikzfig{strings/category/f}[box=g,colour=seq,dom=m,cod=n]
    \) if and only if \(
    \circuittostreami[
        \iltikzfig{strings/category/f}[box=f,colour=seq,dom=m,cod=n]
    ]
    =
    \circuittostreami[
        \iltikzfig{strings/category/f}[box=g,colour=seq,dom=m,cod=n]
    ]
    \).
\end{theorem}
\begin{proof}
    The \(\onlyifdir\) direction follows by \cref{cor:repeated-state}, as every
    possible internal configuration of the circuit will be tested.
    For \(\ifdir\), if \(
    \circuittostreami[
        \iltikzfig{strings/category/f}[box=f,colour=seq,dom=m,cod=n]
    ]
    =
    \circuittostreami[
        \iltikzfig{strings/category/f}[box=g,colour=seq,dom=m,cod=n]
    ]
    \), then this means \(
    \circuittostreami[
        \iltikzfig{strings/category/f}[box=f,colour=seq,dom=m,cod=n]
    ](\listlistvar{v} \streamcons \sigma)
    =
    \circuittostreami[
        \iltikzfig{strings/category/f}[box=g,colour=seq,dom=m,cod=n]
    ](\listlistvar{v} \streamcons \sigma)
    \) for any \(\sigma,\tau \in \valuetuplestream{m}\).
    By definition of \(\circuittostreami\), we then have that \(
    \circuittostreami[
        \iltikzfig{circuits/components/circuits/f-applied-waveform}[box=f,colour=seq,dom=m,cod=n,input=\listlistvar{v}]
    ](\sigma)
    =
    \circuittostreami[
        \iltikzfig{circuits/components/circuits/f-applied-waveform}[box=g,colour=seq,dom=m,cod=n,input=\listlistvar{v}]
    ](\sigma)
    \).
    Since this holds for \emph{all} sequences \(\listlistvar{v}\), it must hold
    for those of length \(|\values|^c + 1\), so the condition for observational
    equivalence is met.
\end{proof}

To verify that this is the `best' equivalence relation, we turn to a
definition of observational equivalence in terms of universal
properties~\cite{gordon1998operational}.
Gordon states that a relation is an \emph{adequate} observational semantics if
it only relates circuits that have the same denotational semantics;
observational equivalence is defined as the largest adequate congruence.

\begin{corollary}
    \(\sim_\interpretation\) is the largest adequate congruence on
    \(\scircsigma\).
\end{corollary}
\begin{proof}
    For \(\sim_\interpretation\) to be a congruence it must be preserved by
    composition, tensor and trace, and for it to be the largest there must be
    no denotationally equal circuit it does not relate.
    These, along with adequacy, all follow by
    \cref{thm:operational-denotational}.
\end{proof}

This makes \(\sim_\interpretation\) a suitable notion of observational
equivalence for sequential circuits.

\begin{definition}
    \index{\(\scircsigmaobs\)}
    \nomenclature{\(\scircsigmaobs\)}{PROP of sequential circuits quotiented by observational equivalence}
    Let \(\scircsigmaobs\) be defined as
    \(\scircsigma / \sim_{\interpretation}\).
\end{definition}

\begin{corollary}
    There is an isomorphism \(\scircsigmai \cong \scircsigmaobs\).
\end{corollary}

The results of the previous section give us an upper bound on the length of
waveforms required to establish observational semantics; this means that we have
a terminating strategy for comparing the behaviour of digital circuits.
Unfortunately, this is still an \emph{exponential} upper bound, so it is
infeasible to check for the equivalence of circuits with more than a few delay
components.

Nevertheless, the operational semantics gives us a straightforward way to
\emph{evaluate} circuits while respecting their internal structure, which can
unlock more insight as to \emph{why} circuits are behaving the way they
are.

Moreover, while it may be infeasible to check \emph{every single possible input}
to a circuit, it may be far easier to check equivalence for a subset of known
inputs.
Often, it is the case that one knows a particular input is fixed for long
periods of time: perhaps it is some sort of control signal.
By precomposing the circuit with appropriate infinite waveforms to represent the
fixed inputs, insight and potential optimisations may be gleaned; this is known
as \emph{partial evaluation}, which will be examined more
in \cref{chap:semantics-applications}.