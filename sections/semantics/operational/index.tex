\chapter{Operational semantics}\label{chap:operational}

With the sound and complete denotational semantics, the behaviour of circuits is
determined by observing their behaviour as stream functions.
This already gives us a perspective on digital circuits closer to that of
programming languages.
To compare the behaviour of two circuits in \(\scircsigma\), we examine their
corresponding stream functions.

Unfortunately, denotational semantics is not the be-all and end-all of circuit
semantics.
Crucially, it obscures the \emph{structure} of a circuit by compressing all the
behaviour into one function: we don't know \emph{why} the circuit is behaving
the way it does, just that something has caused it to do so.
When it comes to circuit design, the structure of the circuit is of course
important, as that is what is going to be printed onto silicon.
Space is at a premium, so knowing how each part of a circuit contributes to the
output behaviour is of critical importance.

We now turn our attention to the next course in our menu of semantics,
\emph{operational semantics}.
This is quite a different beast to denotational semantics: rather than assigning
a mathematical structure to each circuit, semantics are derived from how
something is \emph{executed}.
One can think of an operational semantics as stepping through a
program using a debugger, with rules applied in order to derive the next state.

Operational semantics is another classic concept in computer science; `steps' of
execution were used to define the semantics of ALGOL
68~\cite{vanwijngaarden1976revised}.
The name itself, as with many topics of the time, is attributed to Dana
Scott~\cite{scott1970outline}, who acknowledged that even with the abstraction
of denotational semantics, `the operational aspects cannot be completely
ignored'.

\begin{example}\label{ex:expressions-operational}
    Recall the language of expressions from \cref{ex:expressions-denotational}.
    We can define an observational semantics on this language with the following
    set of rules:
    \begin{gather*}
        \inferrule{ }{\overline{n} \Rightarrow \overline{n}} \,\, (\text{Value})
        \quad
        \inferrule{
            e_0 \Rightarrow \overline{n_0} \\
            e_1 \Rightarrow \overline{n_1}
        }{
            add \, e_0 \, e_1 \Rightarrow \overline{n_0 + n_1}
        } \,\, (\text{Add})
        \quad
        \inferrule{
            e_0 \Rightarrow \overline{n_0} \\
            e_1 \Rightarrow \overline{n_1}
        }{
            mul \, e_0 \, e_1 \Rightarrow \overline{n_0 \cdot n_1}
        } \,\, (\text{Mul})
    \end{gather*}
    We can use these rules to \emph{reduce} expressions to values.
    Two expressions have the same semantics if they reduce to the same expression.
    Formally this can be written as a proof tree:
    \begin{gather*}
        \inferrule*{
            \inferrule*{
                \inferrule{ }{\overline{4} \Rightarrow \overline{4}} \\
                \inferrule{ }{\overline{2} \Rightarrow \overline{2}}
            }{
                mul \, \overline{4} \, \overline{2} \Rightarrow \overline{8}
            }
            \\
            \inferrule*{
                \inferrule{ }{\overline{2} \Rightarrow \overline{3}} \\
                \inferrule{ }{\overline{3} \Rightarrow \overline{3}}
            }{
                add \, \overline{2} \, \overline{3} \Rightarrow \overline{5}
            }
        }{
            add \, (mul \, \overline{4} \, \overline{2}) \, (add \, \overline{2} \, \overline{3}) \Rightarrow \overline{13}
        }
    \end{gather*}
\end{example}

The operational semantics described thus far is commonly known as
\emph{structural operational semantics}~\cite{plotkin1981structural}
semantics; it shows how the behaviour of the whole is represented by the
behaviour of its parts.
However, deriving a proof tree as above can be clunky: to find the meaning of a
large term one has to `drill down' into the contexts until a value is reached
before propagating these values back up the tree.

A more intuitive way to view this style of operational semantics is using
a \emph{reduction semantics}.
First applied by Plotkin~\cite{plotkin1975callbyname} before being properly
coined and generalised in the subsequent
decade~\cite{felleisen1987reduction,felleisen1987calculi}, a reduction semantics
specifes a set of rules which can be successively applied to individual
components of some larger context. These reduction rules can be derived from
the rules in the main operational semantics by replacing subexpressions in
derivations by the primitives in the language.
These rules can then be applied to the `smallest' terms in the expression
(those higher up in the proof tree), reducing these to primitives themselves,
such that reductions may be applied to their parent expressions and so on.

\begin{example}\label{ex:expressions-reduction}
    Returning to \cref{ex:expressions-operational}, the two rules of the
    corresponding reductions semantics are \(
    add \, \overline{n_0} \, \overline{n_1}
    \,\reduction[(\text{Add})]\,
    \overline{n_0 + n_1}
    \) and \(
    mul \, \overline{n_0} \, \overline{n_1}
    \,\reduction[(\text{Add})]\,
    \overline{n_0 \cdot n_1}
    \).
    Note that as the numbers \(\overline{n}\) are the `primitives' of the
    language (their meaning is theirselves), there is no corresponding reduction
    for the \((\text{Value})\) rule.
    Using these reductions an expression can be reduced to a value.
    \begin{gather*}
        add \, (mul \, \overline{4} \, \overline{2}) \, (add \, \overline{2} \, \overline{3})
        \,\reduction[(\text{Mul})]\,
        add \, \overline{8} \, (add \, \overline{2} \, \overline{3})
        \,\reduction[(\text{Add})]\,
        add \, \overline{8} \, \overline{5}
        \,\reduction[(\text{Add})]\,
        \overline{13}
    \end{gather*}
\end{example}


This reduction sequence is not necessarily canonical.
Indeed, while there is one canonical proof tree for a given term there may be
potentially many reduction sequences, but in an ideal reduction system this
should not be an issue.

\begin{definition}
    A reduction system is \emph{confluent} if, for any term \(e\), if there
    exists distinct reductions \(e \reduction e_1\) and \(e \reduction e_2\),
    then there exists term \(e_3\) along with reduction sequences
    \(e_1 \reductions e_3\) and \(e_2 \reductions e_3\).
\end{definition}

While it is important that different reduction sequences should always converge,
it is equally important that we are not stuck performing reductions forever.

\begin{definition}
    A reduction system is \emph{terminating} if, for every term \(e\), there is
    no infinite reduction sequence starting from \(e\).
\end{definition}

If a reduction system is terminating, then repeatedly applying reductions will
eventually lead to a term with no opportunity to apply any more.

\begin{definition}
    A term is in \emph{normal form} if no reductions apply to it.
\end{definition}

If a terminating reduction system is also confluent, every term must have a
\emph{unique} such normal form.
In the setting of an operational semantics, this normal form is the behaviour
of the term.

\begin{example}
    The reduction rules in \cref{ex:expressions-reduction} are clearly
    terminating since they both reduce the number of operations in a term.
    The rules are also confluent: we could have chosen a different order of
    reductions but the result is the same.
    \begin{gather*}
        add \, (mul \, \overline{4} \, \overline{2}) \, (add \, \overline{2} \, \overline{3})
        \,\reduction[(\text{Mul})]\,
        add \, \overline{8} \, (add \, \overline{2} \, \overline{3})
        \,\reduction[(\text{Add})]\,
        add \, \overline{8} \, \overline{5}
        \,\reduction[(\text{Add})]\,
        \overline{13}
    \end{gather*}
    This means that \(\overline{13}\) is the normal form of
    \(add \, (mul \, \overline{4} \, \overline{2}) \, (add \, \overline{2} \, \overline{3})\)
    and subsequently its semantics.
\end{example}

When defining an operational semantics for digital circuits we prefer to use
the reduction semantics style of presentation, as one of our motivations is for
a computer to evaluate circuits step-by-step.

As we will come to see, it is not feasible to create a small-step reduction
semantics for digital circuits while remaining terminating and confluent.
However, what we \emph{can} do is specify some larger transformations to apply
to a circuit followed by some more traditional exhaustive reductions.

Our goal for this chapter is to develop a sound and complete notion of
\emph{observational equivalence} i.e.\ circuits are related if and only if
`executing' them using the operational semantics produces the same values.
To determine which transformations and reductions are sound, we turn to the
denotational semantics; a reduction between circuits \(
\iltikzfig{strings/category/f}[box=f,colour=seq,dom=m,cod=n]
\reduction
\iltikzfig{strings/category/f}[box=g,colour=seq,dom=m,cod=n]
\) is sound with respect to some interpretation \(\interpretation\) if and only
if \(
\circuittostreami[\iltikzfig{strings/category/f}[box=f,colour=seq,dom=m,cod=n]]
=
\circuittostreami[\iltikzfig{strings/category/f}[box=g,colour=seq,dom=m,cod=n]]
\).

\begin{remark}
    The content of this chapter is a refined version
    of~\cite[Sec. 4]{ghica2024fully}.
\end{remark}

\section{Feedback}\label{sec:feedback}

One of the major issues that comes with trying to reduce circuits in
\(\scircsigma\) is the presence of feedback.
Without proper attention, one could end up infinitely unfolding and we never
produce any output values.
The first portion of our operational semantics revolves around some
\emph{global transformations} to make a circuit suitable for reduction.

The first observation we make does not even need anything new to be defined as
it follows immediately from axioms of STMCs.

\begin{lemma}[Global trace-delay form]\label{lem:trace-delay}
    For a sequential circuit \(
    \iltikzfig{strings/category/f}[box=f,colour=seq,dom=m,cod=n]
    \) there exists a combinational circuit \(
    \iltikzfig{circuits/productivity/pre-mealy-core}
    \) and \(\overline{v} \in \valuetuple{z}\) such that \(
    \iltikzfig{strings/category/f}[box=f,colour=seq,dom=m,cod=n]
    =
    \iltikzfig{circuits/productivity/trace-delay}[core=f,state=\listvar{v},dom=m,cod=n,delay=y,trace=x,valwidth=z]
    \) by axioms of STMCs.
\end{lemma}
\begin{proof}
    By applying the axioms of traced categories; any trace can be `pulled'
    to the outside of a term by superposing and tightening.
    For the delays, a trace can be introduced using yanking and then the
    same procedure as above followed.
\end{proof}

\begin{example}
    The SR NOR latch circuit from \cref{ex:sr-latch} is assembled into global
    trace-delay form in \cref{fig:sr-latch-global-trace-delay}.
\end{example}

\begin{figure}
    \centering
    \iltikzfig{circuits/examples/sr-latch/circuit-global-trace-delay}
    \(\reduction[(\mealyeqn)]\)
    \iltikzfig{circuits/examples/sr-latch/circuit-premealy}
    \caption{
        The SR NOR latch from \cref{fig:sr-latch} assembled into global
        trace-delay form, and transformed into pre-Mealy form with
        \((\mealyeqn)\).
    }
    \label{fig:sr-latch-global-trace-delay}
\end{figure}

This form is evocative of what we saw when mapping from Mealy machines to
circuits in the previous section, but rather than the state being determined by
one word, the instantaneous values and the delays are kept separate.

\begin{definition}[Pre-Mealy form]\label{def:pre-mealy}
    A sequential circuit is in \emph{pre-Mealy form} if it is in the form \(
    \iltikzfig{circuits/productivity/pre-mealy-form}[core=f,state=\listvar{s},dom=m,cod=n,trace=x,delay=y]
    \).
\end{definition}

Our first reduction will transform any circuit in global trace-delay form into
pre-Mealy form.

\begin{lemma}\label{lem:mealy-rule}
    The rule \(
    \iltikzfig{circuits/productivity/trace-delay}[core=f,state=\listvar{v},dom=m,cod=n,delay=y,trace=x,valwidth=z]
    \reduction[(\mealyeqn)]
    \iltikzfig{circuits/productivity/mealy-rule}[core=f,trace=x,delays=y,values=z]
    \) is sound.
\end{lemma}
\begin{proof}
    It is a simple exercise to check the corresponding stream functions.
\end{proof}

By assembling a circuit into global trace-delay form and
applying the \((\mealyeqn)\) rule, we can establish a word \(\listvar{s}\)
representing the initial state of a circuit.
This word is by no means unique: it depends on how the circuit is put into
global trace-delay form.
What matters most is that we \emph{can} do it.

\begin{corollary}
    For any sequential circuit \(
    \iltikzfig{strings/category/f}[box=f, colour=seq, dom=m, cod=n]
    \), there exists at least one valid application of the Mealy rule.
\end{corollary}

\begin{example}
    The \((\mealyeqn)\) rule is applied to the global trace-delay form
    SR NOR latch in \cref{fig:sr-latch-global-trace-delay}.
    Here the initial state word is just \(\bot\).
\end{example}

The result of applying the \((\mealyeqn)\) reduction still differs from the
image of \(\mealytocircuiti\) as it may have a trace with no delay on it: an
instance of \emph{non-delay-guarded feedback}.

The mere mention of non-delay-guarded feedback may trigger alarm bells in
the minds of those well-acquainted with circuit design.
It is often common in industry to enforce that circuits have no
non-delay-guarded feedback; one might ask if we should also enforce this
tenet in order to stick to `well-behaved' circuits.

\begin{remark}
    \emph{Categories with feedback}~\cite{katis2002feedback} are a weakening of
    traced categories that remove the yanking axiom: this effectively makes all
    traces delay-guarded.
    \emph{Categories with delayed trace}~\cite{sprunger2019differentiable}
    weaken this further by removing the sliding axiom, so no components can be
    `pushed round' into the next tick of execution.
    Neither of these are suitable for us as we actually \emph{want} to allow
    non-delay-guarded trace.
\end{remark}

In fact, careful use of non-delay-guarded feedback can still result in useful
circuits as a clever way of sharing
resources~\cite{malik1994analysis,riedel2004cyclic,mendler2012constructive}.
The minimal circuit to implement a function often \emph{must} be
constructed using cycles~\cite{rivest1977necessity,riedel2003synthesis}.



\begin{example}\label{ex:cyclic-combinational}
    A particularly famous circuit~\cite{malik1994analysis} which is useful
    despite the presence of non-delay-guarded feedback is shown in
    \cref{fig:cyclic-combinational}, where \(
    \iltikzfig{strings/category/f}[box=f,colour=comb]
    \) and \(
    \iltikzfig{strings/category/f}[box=g,colour=comb]
    \) are arbitrary combinational circuits.
    The trapezoidal gate is a \emph{multiplexer}; it has a vertical
    \emph{control} input and two horizontal \emph{data} inputs.
    The multiplexer is defined as \(
    \iltikzfig{circuits/components/gates/mux}
    \coloneqq
    \iltikzfig{circuits/components/gates/mux-construction}
    \).
    The multiplexer is effectively an if statement: when the control is
    \(\belnapfalse\) the output is the first data input and when it is
    \(\belnaptrue\) the output is the second data input.

    The circuit in \cref{fig:cyclic-combinational} has no state and its trace is
    global so it is already in pre-Mealy form, and has
    non-delay-guarded feedback.
    Despite this, it produces useful output when the control signal is true or
    false:
    when the control signal is \(\belnapfalse\) then the behaviour of the
    circuit is \(
    \circuittofunc[
        \iltikzfig{circuits/examples/cyclic-combinational/reduced-false}
    ]{\interpretation_\star}
    \) and when the control is \(\belnaptrue\) then the behaviour is \(
    \circuittofunc[
        \iltikzfig{circuits/examples/cyclic-combinational/reduced-true}
    ]{\interpretation_\star}
    \).
    The feedback is just used as a clever way to share circuit components.
\end{example}

\begin{figure}
    \centering
    \iltikzfig{circuits/examples/cyclic-combinational/circuit}
    \quad
    \iltikzfig{circuits/examples/cyclic-combinational/circuit-scirc}
    \caption{
        A useful cyclic combinational circuit
        \cite[Fig. 1]{mendler2012constructive}, and a possible interpretation in
        \(\scircsigma\).
    }
    \label{fig:cyclic-combinational}
\end{figure}

A combinational circuit surrounded by non-delay-guarded feedback still
implements a function, as there are no delay components.
Nevertheless, non-delay-guarded feedback does still block our path to future
transformations, so it must be eliminated.
Using a methodology also employed by~\cite{riedel2012cyclic}, we turn to the
Kleene fixed-point theorem.

\begin{lemma}\label{lem:monotone-fixpoint}
    For a monotone function \(\morph{f}{\values^{n+m}}{\valuetuple{n}}\) and
    \(i \in \nat\), let \(\morph{f^i}{\valuetuple{m}}{\valuetuple{n}}\) be
    defined as \(f^0(x)  = f(\bot^n,x)\) and \(f^{k+1}(x) = f(f^k(x), x)\).
    Let \(c\) be the length of the longest chain in the lattice
    \(\valuetuple{n}\).
    Then, for \(j > c\), \(f^c(x) = f^{j}(x)\).
\end{lemma}
\begin{proof}
    Since \(f\) is monotone, it has a least fixed point by the Kleene
    fixed-point theorem.
    This will either be some value \(v\) or, since \(\values\) is finite, the
    \(\top\) element.
    The most iterations of \(f\) it would take to obtain this fixpoint is \(c\),
    i.e.\ the function produces a value one step up the lattice each time.
\end{proof}

\begin{definition}[Iteration]\label{def:iteration}
    Given a combinational circuit \(
    \iltikzfig{strings/category/f-2-2}[box=f,colour=comb,dom1=x,dom2=m,cod1=x,cod2=n],
    \)
    its \emph{\(n\)-th iteration} \(
    \iltikzfig{strings/category/f-1-2}[box=f^n,colour=comb,dom=m,cod1=x,cod2=n]
    \) is defined inductively over \(n\) in the following way: \[
        \iltikzfig{circuits/instant-feedback/f0-box}[dom=m,trace=x,cod=n]
        \coloneqq
        \iltikzfig{circuits/instant-feedback/f0-definition}[box=f,dom=m,trace=x,cod=n]
        \qquad
        \iltikzfig{circuits/instant-feedback/fkp1-box}[dom=m,trace=x,cod=n]
        \coloneqq
        \iltikzfig{circuits/instant-feedback/fkp1-definition}[dom=m,trace=x,cod=n]
    \]
\end{definition}

The trace in \(\streami\) is by the least fixed point, computed by repeatedly
applying \(f\) to itself starting from \(\bot\).
The above lemma gives a fixed upper bound for the number of times we need to
apply \(f\) to reach this fixed point, based on the size of the lattice.
We can use this in the syntactic setting.

\begin{definition}[Unrolling]\label{def:unrolling}
    For an interpretation with values \(\values\), the \emph{unrolling}
    of a combinational circuit \(
    \iltikzfig{strings/category/f-2-2}[box=f,colour=comb,dom1=x,dom2=m,cod1=x,cod2=n]
    \), written \(
    \iltikzfig{strings/category/f-1-2}[box=f^\dagger,colour=comb,dom=m,cod1=x,cod2=n]
    \), is defined as \(
    \iltikzfig{circuits/instant-feedback/fc-box}[dom=m,trace=x,cod=n]
    \) where \(c\) is the length of the longest chain in \(\valuetuple{x}\).
\end{definition}

Using these constructs we can replace a combinational circuit wrapped in
non-delay-guarded feedback with a behaviourally equivalent circuit with no
feedback at all.

\begin{proposition}\label{prop:instant-feedback}
    The instant feedback rule \(
    \iltikzfig{circuits/instant-feedback/equation-lhs}[box=f]
    \reduction[(\instantfeedbackeqn)]
    \iltikzfig{circuits/instant-feedback/equation-rhs}[box=f]
    \) is sound.
\end{proposition}
\begin{proof}
    By \cref{lem:monotone-fixpoint}, applying the function \(
    (\listvar{x}) \mapsto \proj{x}\left(\circuittofunci[
        \iltikzfig{strings/category/f-2-2}[box=f,colour=comb]
    ]\right)(\listvar{x}, \listvar{v})\) to itself \(c\) times reaches a
    fixpoint.
    The circuit is combinational so each element of the output
    \(\circuittostreami[
        \iltikzfig{strings/category/f-2-2}[box=f,colour=comb]
    ](\sigma)(i)\) is a function; this means that \cref{lem:monotone-fixpoint}
    can be applied to each element.
\end{proof}

\begin{example}\label{ex:sr-latch-unrolled}
    In \cref{fig:sr-latch-unrolled}, the \(\instantfeedbackeqn\) is applied to
    the SR latch circuit in pre-Mealy form from
    \cref{fig:sr-latch-global-trace-delay}.
\end{example}

\begin{example}
    In \cref{fig:cyclic-combinational-unrolled}, the \((\instantfeedbackeqn)\)
    rule is applied to the cyclic combinational circuit from
    \cref{fig:cyclic-combinational}.
\end{example}

\begin{figure}
    \centering
    \scalebox{0.85}{\iltikzfig{circuits/examples/sr-latch/circuit-instfb-2}}
    \caption{
        Applying the \((\instantfeedbackeqn)\) rule to the SR NOR latch in
        \cref{fig:sr-latch-global-trace-delay}
    }
    \label{fig:sr-latch-unrolled}
\end{figure}
\begin{figure}
    \centering
    \scalebox{0.8}{
        \iltikzfig{circuits/examples/cyclic-combinational/circuit-scirc-instfb}
    }
    \caption{
        Applying the \((\instantfeedbackeqn)\) rule to the circuit from
        \cref{fig:cyclic-combinational}
    }
    \label{fig:cyclic-combinational-unrolled}
\end{figure}

If applied locally for every feedback loop, the \((\instantfeedbackeqn)\)
rule would cause an exponential blowup, but if a circuit is in global
trace-delay form, the rule need only be applied once to the global loop.
Although the value of \(c\) increases as the number of feedback wires increases,
it only does so linearly in the height of the lattice.

With a method to eliminate non-delay-guarded feedback, we can establish the
class of circuits which will act as the keystone of both the operational
semantics in this section and the algebraic semantics of the next.

\begin{definition}[Mealy form]\label{def:delay-guarded}
    A sequential circuit
    \iltikzfig{strings/category/f}[box=f,colour=seq,dom=m,cod=n]
    is in \emph{Mealy form} if it is in the form \(
    \iltikzfig{circuits/productivity/mealy-form}[core=g,state=\listvar{s},dom=m,cod=n,delay=x]
    \).
\end{definition}

\begin{theorem}\label{thm:all-mealy-form}
    For a sequential circuit
    \(\iltikzfig{strings/category/f}[box=f,colour=seq,dom=m,cod=n]\), there exist
    at least one combinational circuit \(
    \iltikzfig{strings/category/f-2-2}[box=g,colour=comb,dom1=x,dom2=m,cod1=x,cod2=n]
    \) and values \(\listvar{s} \in \valuetuple{x}\) such that \(
    \iltikzfig{strings/category/f}[box=f,colour=seq,dom=m,cod=n]
    \reductions
    \iltikzfig{circuits/productivity/mealy-form}[core=g,state=\listvar{s},dom=m,cod=n]
    \) by applying \((\mealyeqn)\) followed by \((\instantfeedbackeqn)\).
\end{theorem}
\begin{proof}
    Any circuit can be assembled into global trace-delay form by
    \cref{lem:trace-delay} and furthermore transformed into pre-Mealy form by
    using \((\mealyeqn)\).
    Since the core of a circuit in pre-Mealy form is combinational and has a
    non-delay-guarded trace, \((\instantfeedbackeqn)\) can be applied to it to
    produce a circuit with only delay-guarded feedback: a circuit in Mealy form.
\end{proof}

Non-delay-guarded feedback can be exhaustively unrolled because the circuit
essentially models a function despite the presence of the trace: this means that
we can transform the circuit without having to `look into the future'.
This is not the case for delay-guarded feedback as the internal state of the
circuit may depend on future inputs.
Indeed, a circuit with delay-guarded feedback may never `settle' on one
internal configuration but rather oscillate between multiple states.
This is simply a facet of sequential circuits and there is nothing we can do
about that.
What we \emph{can} do is show how to \emph{process} inputs at a given tick of
the clock.

\section{Productivity}

For an open circuit it is not possible to reduce an entire circuit down to some
output values, as there will still be open wires awaiting the next inputs.
Nevertheless, if we precompose a circuit with some inputs we can provide some
rules for propagating these inputs across the circuit.

Formally, for sequential circuit \(
\iltikzfig{strings/category/f}[box=f,colour=seq,dom=m,cod=n]
\) and values \(
\listvar{v} \in \valuetuplestream{m}
\), this corresponds to finding reductions such that \(
\iltikzfig{circuits/productivity/productive-goal-lhs}[box=f,input=\listvar{v},dom=m,cod=n]
\reductions
\iltikzfig{circuits/productivity/productive-goal-rhs}[box=g,output=\listvar{w},dom=m,cod=n]
\).
We first consider the combinational case, with our final global transformation.

\begin{lemma}[Streaming]\label{lem:streaming}
    The \emph{streaming rule} \(
    \iltikzfig{circuits/axioms/generalised-streaming-lhs}[box=f]
    \reduction[(\streamingeqn)]
    \iltikzfig{circuits/axioms/generalised-streaming-rhs}[box=f]
    \) is sound.
\end{lemma}
\begin{proof}
    Once again this can be shown by considering the stream semantics.
    First note that by unfolding the notation, \(
    \iltikzfig{circuits/axioms/generalised-streaming-lhs}[box=f]
    \coloneqq
    \iltikzfig{circuits/axioms/generalised-streaming-lhs-verbose}[box=f]
    \).
    The streaming rule is then effectively `pushing' the combinational circuit
    \(\iltikzfig{strings/category/f}[box=f,colour=comb]\) across the join.

    The join is \emph{not} a natural transformation so this does not hold in
    general, but because one argument is an instantaneous value and the other
    is a delay, at least one of the inputs to the join will be \(\bot\) for a
    given circuit.
    As the interpretations of combinational circuits must be
    \(\bot\)-preserving, the circuit can safely be pushed across the join and
    delay.
\end{proof}

The streaming rule shows that when a combinational circuit is applied to an
input with an instantaneous and a delayed component, the circuit can be copied
so that one copy handles what is happening `now' and the other handles what is
happening `later'.

\begin{example}
    \cref{fig:sr-latch-streamed} shows how the SR NOR latch in Mealy form from
    \cref{fig:sr-latch-unrolled} is transformed by the \((\streamingeqn)\) rule.

\end{example}

As there is a delay on the bottom argument of the join, the output of a streamed
circuit at the current tick is now contained entirely in the top argument of the
join.
The final rules we present will reduce this copy to values, as desired.

\begin{lemma}[Value rules]
    The following \emph{value rules} are sound:
    \begin{gather*}
        \iltikzfig{circuits/axioms/fork-lhs}[val=v]
        \reduction[(\forkeqn)]
        \iltikzfig{circuits/axioms/fork-rhs}[val=v]
        \quad
        \iltikzfig{circuits/axioms/join-lhs}[val1=v,val2=w]
        \reduction[(\joineqn)]
        \iltikzfig{circuits/axioms/join-rhs}[val1=v,val2=w]
        \quad
        \iltikzfig{circuits/axioms/stub-lhs}[val=v]
        \reduction[(\stubeqn)]
        \iltikzfig{strings/monoidal/empty}
        \quad
        \iltikzfig{circuits/axioms/gate-lhs}[gate=p,input=\listvar{v}]
        \reduction[(\gateeqn)]
        \iltikzfig{circuits/axioms/gate-rhs}[gate=p,input=\listvar{v}]
    \end{gather*}
\end{lemma}
\begin{proof}
    These follow immediately from \cref{lem:combinational-streams} and the
    interpretation of combinational circuits as function.
\end{proof}

Reducing the `now' core is the only time in which exhaustive application is
required, as more is involved than than just copying circuit components.

\begin{lemma}\label{lem:reduce-core-confluent}
    Applying the value rules is confluent.
\end{lemma}
\begin{proof}
    There are no overlaps between the rules.
\end{proof}

\begin{lemma}\label{lem:reduce-core-terminating}
    For a combinational circuit \(
    \iltikzfig{strings/category/f}[box=f,colour=comb,dom=m,cod=n]
    \) and \(\listvar{v} \in \valuetuple{m}\), there exists a word
    \(\listvar{w} \in \valuetuple{n}\) such that applying the value
    rules exhaustively to \(
    \iltikzfig{circuits/components/circuits/f-applied}[box=f,colour=comb]
    \) terminates at \(
    \iltikzfig{circuits/components/values/vs}[val=\listvar{w}]
    \).
\end{lemma}

These rules are all we need to propagate input values across a circuit.

\begin{corollary}\label{cor:mealy-form-productivity}
    For circuit \(
    \iltikzfig{circuits/productivity/mealy-form-applied}[core=f,dom=m,cod=n,delay=x]
    \) there exist \(
    \listvar{t} \in \valuetuple{x}
    \) and \(
    \listvar{w} \in \valuetuple{n}
    \) such that \(
    \iltikzfig{circuits/productivity/mealy-form-applied}[core=f,dom=m,cod=n,delay=x]
    \reductions
    \iltikzfig{circuits/productivity/mealy-form-produced}[core=f,dom=m,cod=n,delay=x]
    \) by applying \(\streamingeqn\) once followed by the value rules
    exhaustively.
\end{corollary}

\begin{example}\label{ex:productivity}
    \cref{fig:sr-latch-streamed} shows how the `now' copy of the transformed SR
    latch circuit from \cref{fig:sr-latch-unrolled} for inputs
    \(\belnaptrue\belnapfalse\) (a `reset' pulse) is reduced by the
    combinational rules.
    The next state is \(\belnaptrue\), the first output is \(\belnapfalse\) and
    the second is \(\bot\).
    The first output (false) is what we would expect given a reset pulse, but
    the second may raise an eyebrow.
    This arises due to the delay; recall that this models inertial delay in the
    wires rather than an actual memory element.
    Subsequently, it will take another cycle to produce the expected output
    \(\belnapfalse\belnaptrue\).
\end{example}

\begin{figure*}
    \centering
    \scalebox{0.45}{\iltikzfig{circuits/examples/sr-latch/circuit-streamed-3}}
    \caption{
        Applying \(\streamingeqn\) with inputs \(\belnaptrue\belnapfalse\) to
        the circuit from \cref{fig:sr-latch-unrolled}
    }
    \label{fig:sr-latch-streamed}
\end{figure*}

By now putting together all the components in this section and the previous,
we have a productive strategy for processing inputs to \emph{any} sequential
circuit.

\begin{corollary}[Productivity]\label{cor:productivity}
    For sequential circuit \(
    \iltikzfig{strings/category/f}[box=f,colour=seq,dom=m,cod=n]
    \) and inputs \(\listvar{v} \in \valuetuple{m}\), there exists
    \(\listvar{w} \in \valuetuple{n}\) such that \(
    \iltikzfig{circuits/productivity/productive-goal-lhs}[box=f,input=\listvar{v},dom=m,cod=n]
    \reductions
    \iltikzfig{circuits/productivity/productive-goal-rhs}[box=g,output=\listvar{w},dom=m,cod=n]
    \) by applying \(\mealyeqn\), \(\instantfeedbackeqn\) and \(\streamingeqn\)
    once successively followed by the value rules exhaustively.
\end{corollary}

\begin{remark}
    As we saw in \cref{cor:mealy-form-productivity}, applying
    \((\streamingeqn)\) followed by the value rules to a circuit in Mealy form
    produces another circuit in Mealy form.
    This means that for one circuit and a whole stream of inputs,
    \((\mealyeqn)\) and \((\instantfeedbackeqn)\) need only be applied
    \emph{once} to get a circuit into Mealy form at the beginning before
    processing values.
    As we shall see late on this is very handy for implementing these reductions
    mechanically, as very few large circuit transformations are required.
\end{remark}
\section{Observational equivalence}

In the denotational semantics, we defined the relation of
\emph{denotational equivalence}, in which circuits are related if their
denotations as streams are equal.
For operational semantics we have another notion of relation on circuits: that
of \emph{observational equivalence}.
This is due to Morris~\cite{morris1969lambdacalculus}, who named it
`extensional equivalence': essentially, two processes are observationally if they
cannot be distinguished by their input-output behaviour.
To apply this to circuits we first define some notation for sequences of inputs
to a circuit.

\begin{notation}[Waveform]\label{def:waveform}
    The empty waveform is defined as \(
    \iltikzfig{circuits/components/waveforms/sigma0}
    \coloneqq
    \iltikzfig{strings/category/identity}[colour=comb,obj=n]
    \).
    Given values \(\overline{v} \in \valuetuple{n}\) and sequence \(
    \overline{\underline{w}} \in (\valuetuple{n})^\star
    \), the waveform for sequence \(
    \overline{v} \streamcons \overline{\underline{w}}
    \) is drawn as \(
    \iltikzfig{circuits/components/waveforms/sigmatp1-spaced}
    \coloneqq
    \iltikzfig{circuits/components/waveforms/sigmatp1-construction-spaced}
    \).
\end{notation}

Testing for observational equivalence is traditionally performed by checking
that a program behaves the same in all \emph{contexts}.
In the setting of digital circuits, this means that for all possible streams of
inputs, the circuit produces the same inputs.
Of course, there are infinitely many streams of inputs, despite the set of
values being finite.
Fortunately, since circuits are constructed from a finite number of
\emph{components}, we need not check them all.

\begin{lemma}\label{lem:number-of-states}
    Let \(
    \iltikzfig{strings/category/f}[box=F,colour=seq]
    \) be a sequential circuit with \(c\) delay components.
    Then applying \cref{cor:productivity} successively to a Mealy form of this
    circuit will produce at most \(|\values|^c\) unique states.
\end{lemma}
\begin{proof}
    The only varying elements of the state word are contributed by
    the \(c\) delay components, as the values transition to \(\bot\).
\end{proof}

\begin{corollary}\label{cor:repeated-state}
    Given a circuit in Mealy form \(
    \iltikzfig{circuits/productivity/mealy-form}[dom=m,cod=n,delay=x]
    \) and input sequence \(\listlistvar{v} \in \valuetupleseq{m}\) of length
    \(|\values|^c + 1\), there exists a state \(
    \listvar{r} \in \valuetuple{x}
    \), an input sequence \(
    \listlistvar{u} \in \valuetupleseq{m}
    \) and output sequences \(
    \listlistvar{w},\listlistvar{z} \in \valuetupleseq{n}
    \) such that applying \cref{cor:productivity} yields the
    following reduction pattern: \begin{gather*}
        \iltikzfig{circuits/productivity/mealy-form-waveform}
        \reductions
        \iltikzfig{circuits/productivity/mealy-form-waveform-partial}[state=\listvar{r},output=\listlistvar{w},input=\listlistvar{u}]
        \reductions
        \iltikzfig{circuits/productivity/mealy-form-waveform-output-2}[state=\listvar{r},output1=\listlistvar{z},output2=\listlistvar{w}]
    \end{gather*}
\end{corollary}

This means that every possible behaviour of a circuit can be evaluated using
a finite number of streams.
This can be used to define our notion of observational equivalence for digital
circuits.

\begin{definition}[Observational equivalence of circuits]
    We say that two sequential circuits \(
    \iltikzfig{strings/category/f}[box=F,colour=seq,dom=m,cod=n]
    \) and \(
    \iltikzfig{strings/category/f}[box=G,colour=seq,dom=m,cod=n]
    \) with no more than \(c\) delays are said to be
    \emph{observationally equivalent under} \(\interpretation\), written \(
    \iltikzfig{strings/category/f}[box=F,colour=seq]
    \sim_{\interpretation}
    \iltikzfig{strings/category/f}[box=G,colour=seq]
    \) if applying productivity produces the same output
    waveforms for all input waveforms \(
    \listlistvar{v} \in \valuetupleseq{m}\) of length
    \(|\values^c| + 1\).
\end{definition}

To verify that this is the `best' such relation, we turn to a
definition of observational equivalence in terms of universal
properties~\cite{gordon1998operational}.
Gordon states that a relation is an \emph{adequate} observational semantics if
it only relates circuits that have the same denotational semantics;
observational equivalence is then defined as the largest adequate congruence.

\begin{theorem}\label{thm:operational-denotational}
    Given two sequential circuits \(
    \iltikzfig{strings/category/f}[box=F,colour=seq,dom=m,cod=n]
    \) and \(
    \iltikzfig{strings/category/f}[box=G,colour=seq,dom=m,cod=n]
    \), \(
    \iltikzfig{strings/category/f}[box=F,colour=seq,dom=m,cod=n]
    \sim_\interpretation
    \iltikzfig{strings/category/f}[box=G,colour=seq,dom=m,cod=n]
    \) if and only if \(
    \circuittostreami[
        \iltikzfig{strings/category/f}[box=F,colour=seq,dom=m,cod=n]
    ]
    =
    \circuittostreami[
        \iltikzfig{strings/category/f}[box=G,colour=seq,dom=m,cod=n]
    ]
    \).
\end{theorem}
\begin{proof}
    The \(\onlyifdir\) direction follows by \cref{cor:repeated-state}, as every
    possible internal configuration of the circuit will be tested.
    For \(\ifdir\), if \(
    \circuittostreami[
        \iltikzfig{strings/category/f}[box=F,colour=seq,dom=m,cod=n]
    ]
    =
    \circuittostreami[
        \iltikzfig{strings/category/f}[box=G,colour=seq,dom=m,cod=n]
    ]
    \), then this means \(
    \circuittostreami[
        \iltikzfig{strings/category/f}[box=F,colour=seq,dom=m,cod=n]
    ](\listlistvar{v} \streamcons \sigma)
    =
    \circuittostreami[
        \iltikzfig{strings/category/f}[box=G,colour=seq,dom=m,cod=n]
    ](\listlistvar{v} \streamcons \sigma)
    \) for any \(\sigma,\tau \in \valuetuplestream{m}\).
    By definition of \(\circuittostreami\), we then have that \(
    \circuittostreami[
        \iltikzfig{circuits/components/circuits/f-applied-waveform}[box=F,colour=seq,dom=m,cod=n,input=\listlistvar{v}]
    ](\sigma)
    =
    \circuittostreami[
        \iltikzfig{circuits/components/circuits/f-applied-waveform}[box=G,colour=seq,dom=m,cod=n,input=\listlistvar{v}]
    ](\sigma)
    \).
    Since this holds for \emph{all} sequences \(\listlistvar{v}\), it must hold
    for those of length \(|\values|^c + 1\), so the condition for observational
    equivalence is met.
\end{proof}

\begin{corollary}
    \(\sim_\interpretation\) is the largest adequate congruence on
    \(\scircsigma\).
\end{corollary}
\begin{proof}
    For \(\sim_\interpretation\) to be a congruence it must be preserved by
    composition, tensor and trace, and for it to be the largest there must be
    no denotationally equal circuit it does not relate.
    These, along with adequacy, all follow by
    \cref{thm:operational-denotational}.
\end{proof}

This makes \(\sim_\interpretation\) a suitable notion of observational
equivalence for sequential circuits.

\begin{definition}
    Let \(\scircsigmaobs\) be defined as
    \(\scircsigma / \sim_{\interpretation}\).
\end{definition}

\begin{corollary}
    There is an isomorphism \(\scircsigmai \cong \scircsigmaobs\).
\end{corollary}

The results of the previous section give us an upper bound on the length of
waveforms required to establish observational semantics; this means that we have
a terminating strategy for comparing the behaviour of digital circuits.
Unfortunately, this is still an \emph{exponential} upper bound, so it is
infeasible to check for the equivalence of circuits with more than a few delay
components.

Nevertheless, the operational semantics gives us a straightforward way to
\emph{evaluate} circuits while respecting their internal structure, which can
unlock more insight as to \emph{why} circuits are behaving the way they
are.

Moreover, while it may be infeasible to check \emph{every single possible input}
to a circuit, it may be far easier to check equivalence for a subset of known
inputs.
Often, it is the case that one knows a particular input is fixed for long
periods of time: perhaps it is some sort of control signal.
By precomposing the circuit with appropriate infinite waveforms to represent the
fixed inputs, the behaviour of the circuit can be evaluated and inspected.
This is known as \emph{partial evaluation}, which will be examined more
in \cref{sec:semantics-applications}.
\section{Operational semantics for generalised circuits}

When dealing with arbitrary-width wires, the only part of the operational
semantics that does not completely generalise in the obvious way are the value
rules.

\begin{lemma}[Generalised value rules]
    \index{generalised value rules}
    The following \emph{generalised value rules} are sound:
    \begin{gather*}
        \iltikzfig{circuits/axioms/fork-lhs}[val=\listvar{v}, cod=n]
        \reduction[(\forkeqn)]
        \iltikzfig{circuits/axioms/fork-rhs}[val=\listvar{v}, cod=n]
        \qquad
        \iltikzfig{circuits/axioms/join-lhs}[val1=\listvar{v},val2=\listvar{w}, cod=n]
        \reduction[(\joineqn)]
        \iltikzfig{circuits/axioms/join-rhs}[val1=\listvar{v},val2=\listvar{w}, cod=n]
        \\[1em]
        \iltikzfig{circuits/axioms/stub-lhs}[val=\listvar{v}]
        \reduction[(\stubeqn)]
        \iltikzfig{strings/monoidal/empty}
        \qquad
        \iltikzfig{circuits/axioms/gate-lhs}[gate=p,input=\listvar{v}]
        \reduction[(\gateeqn)]
        \iltikzfig{circuits/axioms/gate-rhs}[gate=p,input=\listvar{v}]
        \\[1em]
        \iltikzfig{circuits/axioms/unbundle-lhs}
        \reduction[(\spliteqn)]
        \iltikzfig{circuits/axioms/unbundle-rhs}
        \qquad
        \iltikzfig{circuits/axioms/bundle-lhs}
        \reduction[(\combineeqn)]
        \iltikzfig{circuits/axioms/bundle-rhs}
    \end{gather*}
\end{lemma}

\begin{lemma}
    Applying the generalised value rules is confluent.
\end{lemma}
\begin{proof}
    There are no overlaps between the rules.
\end{proof}

\begin{lemma}l
    For a generalised combinational circuit \(
    \iltikzfig{strings/category/f}[box=f,colour=comb,dom=\listvar{m},cod=\listvar{n}]
    \) and \(\listvar{v} \in \valuetuple{\listvar{m}}\), there exists a word
    \(\listvar{w} \in \valuetuple{\listvar{n}}\) such that applying the value
    rules exhaustively to \(
    \iltikzfig{circuits/components/circuits/f-applied}[box=f,colour=comb]
    \) terminates at \(
    \iltikzfig{circuits/components/values/vs}[val=\listvar{w}]
    \).
\end{lemma}

With these rules, the inputs to a generalised circuit can be processed.

\begin{corollary}
    For generalised circuit \(
    \iltikzfig{circuits/productivity/mealy-form-applied}[core=f,dom=\listvar{m},cod=\listvar{n},delay=\listvar{x}]
    \) there exist \(
    \listvar{t} \in \valuetuple{\listvar{x}}
    \) and \(
    \listvar{w} \in \valuetuple{\listvar{n}}
    \) such that \(
    \iltikzfig{circuits/productivity/mealy-form-applied}[core=f,dom=\listvar{m},cod=\listvar{n},delay=\listvar{x}]
    \reductions
    \iltikzfig{circuits/productivity/mealy-form-produced}[core=f,dom=\listvar{m},cod=\listvar{n},delay=\listvar{x}]
    \) by applying \((\streamingeqn)\) once followed by the generalised value
    rules exhaustively.
\end{corollary}

\begin{corollary}[Generalised productivity]
    \index{generalised productivity}
    For sequential circuit \(
    \iltikzfig{strings/category/f}[box=f,colour=seq,dom=\listvar{m},cod=\listvar{n}]
    \) and inputs \(\listvar{v} \in \valuetuple{\listvar{m}}\), there exists
    \(\listvar{w} \in \valuetuple{\listvar{n}}\) such that \(
    \iltikzfig{circuits/productivity/productive-goal-lhs}[box=f,input=\listvar{v},dom=\listvar{m},cod=\listvar{n}]
    \reductions
    \iltikzfig{circuits/productivity/productive-goal-rhs}[box=g,output=\listvar{w},dom=\listvar{m},cod=\listvar{n}]
    \) by applying \(\mealyeqn\), \(\instantfeedbackeqn\) and \(\streamingeqn\)
    once successively followed by the value rules exhaustively.
\end{corollary}

Since register components can now hold words rather than just values, for
observational equivalence we must consider longer input waveforms.

\begin{definition}[Register width]
    \index{register width}
    For a generalised sequential circuit \(
    \iltikzfig{strings/category/f}[box=f,colour=seq,dom=\listvar{m},cod=\listvar{n}]
    \), let \(c_n\) be the number of \(n\)-width delay components
    \(\iltikzfig{circuits/components/waveforms/delay}[width=n]\).
    Then the \emph{register width} of \(
    \iltikzfig{strings/category/f}[box=f,colour=seq,dom=\listvar{m},cod=\listvar{n}]
    \) is computed as \(\Sigma_{n \in \nat}\ c_n \cdot n\).
\end{definition}

\begin{definition}
    \index{observational equivalence!of generalised circuits}
    \nomenclature{\(\sim^+_\interpretation\)}{observational equivalence of sequential circuits}
    We say that two generalised sequential circuits \(
    \iltikzfig{strings/category/f}[box=f,colour=seq,dom=\listvar{m},cod=\listvar{n}]
    \) and \(
    \iltikzfig{strings/category/f}[box=g,colour=seq,dom=\listvar{m},cod=\listvar{n}]
    \) with register width at most \(c\) are said to be
    \emph{observationally equivalent under} \(\interpretation\), written \(
    \iltikzfig{strings/category/f}[box=f,colour=seq]
    \sim^+_{\interpretation}
    \iltikzfig{strings/category/f}[box=g,colour=seq]
    \) if applying productivity produces the same output
    waveforms for all input waveforms \(
    \listlistvar{v} \in \valuetupleseq{\listvar{m}}\) of length
    \(|\values^c| + 1\).
\end{definition}

The observational equivalence results from the previous section then generalise
nicely to the multicoloured case.

\begin{theorem}
    Given two sequential circuits \(
    \iltikzfig{strings/category/f}[box=f,colour=seq,dom=\listvar{m},cod=\listvar{n}]
    \) and \(
    \iltikzfig{strings/category/f}[box=g,colour=seq,dom=\listvar{m},cod=\listvar{n}]
    \), \(
    \iltikzfig{strings/category/f}[box=f,colour=seq,dom=\listvar{m},cod=\listvar{n}]
    \sim^+_{\interpretation}
    \iltikzfig{strings/category/f}[box=g,colour=seq,dom=\listvar{m},cod=\listvar{n}]
    \) if and only if \(
    \circuittostreami[
        \iltikzfig{strings/category/f}[box=f,colour=seq,dom=\listvar{m},cod=\listvar{n}]
    ]
    =
    \circuittostreami[
        \iltikzfig{strings/category/f}[box=g,colour=seq,dom=\listvar{m},cod=\listvar{n}]
    ]
    \).
\end{theorem}

\begin{corollary}
    \(\sim^+_\interpretation\) is the largest adequate congruence on
    \(\scircsigmag\).
\end{corollary}

\begin{definition}
    \index{\(\scircsigmagobs\)}
    \nomenclature{\(\scircsigmagobs\)}{PROP of generalised sequential circuits quotiented by observational equivalence}
    Let \(\scircsigmagobs\) be defined as \(\scircsigmag / \sim^+_{\interpretation}\).
\end{definition}

\begin{corollary}
    There is an isomorphism \(\scircsigmaig \cong \scircsigmagobs\).
\end{corollary}