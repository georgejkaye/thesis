\section{Feedback}\label{sec:feedback}

One of the major issues that comes with trying to reduce circuits in
\(\scircsigma\) is the presence of feedback.
Without proper attention, one could end up infinitely unfolding and we never
produce any output values.
The first portion of our operational semantics revolves around some
\emph{global transformations} to make a circuit suitable for reduction.

The first observation we make does not even need anything new to be defined as
it follows immediately from axioms of STMCs.

\begin{lemma}[Global trace-delay form]\label{lem:trace-delay}
    \index{global trace-delay form}
    For a sequential circuit \(
    \iltikzfig{strings/category/f}[box=f,colour=seq,dom=m,cod=n]
    \) there exists a combinational circuit \(
    \iltikzfig{circuits/productivity/pre-mealy-core}
    \) and values \(\overline{v} \in \valuetuple{z}\) such that \(
    \iltikzfig{strings/category/f}[box=f,colour=seq,dom=m,cod=n]
    =
    \iltikzfig{circuits/productivity/trace-delay}[core=f,state=\listvar{v},dom=m,cod=n,delay=y,trace=x,valwidth=z]
    \) by axioms of STMCs.
\end{lemma}
\begin{proof}
    By applying the axioms of traced categories; any trace can be `pulled'
    to the outside of a term by superposing and tightening.
    For the delays, a trace can be introduced using yanking and then the
    same procedure as above followed.
\end{proof}

\begin{example}
    \index{SR NOR latch}
    The SR NOR latch circuit from \cref{ex:sr-latch} is assembled into global
    trace-delay form in \cref{fig:sr-latch-global-trace-delay}.
\end{example}

\begin{figure}
    \centering
    \iltikzfig{circuits/examples/sr-latch/circuit-global-trace-delay}
    \(\reduction[(\mealyeqn)]\)
    \iltikzfig{circuits/examples/sr-latch/circuit-premealy}
    \caption{
        The SR NOR latch from \cref{fig:sr-latch} assembled into global
        trace-delay form, and transformed into pre-Mealy form with
        \((\mealyeqn)\).
    }
    \label{fig:sr-latch-global-trace-delay}
\end{figure}

This form is evocative of what we saw when mapping from Mealy machines to
circuits in the previous section, but rather than the state being determined by
one word, the instantaneous values and the delays are kept separate.

\begin{definition}[Pre-Mealy form]\label{def:pre-mealy}
    \index{pre-Mealy form}
    A sequential circuit is in \emph{pre-Mealy form} if it is in the form \(
    \iltikzfig{circuits/productivity/pre-mealy-form}[core=f,state=\listvar{s},dom=m,cod=n,trace=x,delay=y]
    \).
\end{definition}

Our first reduction will transform any circuit in global trace-delay form into
pre-Mealy form.

\begin{lemma}\label{lem:mealy-rule}
    The following rule is sound: \[
        \iltikzfig{circuits/productivity/trace-delay}[core=f,state=\listvar{v},dom=m,cod=n,delay=y,trace=x,valwidth=z]
        \reduction[(\mealyeqn)]
        \iltikzfig{circuits/productivity/mealy-rule}[core=f,trace=x,delays=y,values=z]
    \]
\end{lemma}
\begin{proof}
    It is a simple exercise to check the corresponding stream functions.
\end{proof}

\begin{figure}
    \centering
    \iltikzfig{circuits/examples/sr-latch/circuit-premealy-2}
    \caption{
        Applying the \((\mealyeqn)\) rule to the circuit in
        \cref{fig:sr-latch-global-trace-delay}
    }
    \label{fig:sr-latch-pre-mealy}
\end{figure}

By assembling a circuit into global trace-delay form and
applying the \((\mealyeqn)\) rule, we can construct a word \(\listvar{s}\)
from the juxtaposition of \(\bot\) elements for each register combined with
the instantaneous values.
This word represents the initial state of the circuit, but it is by no means
unique: it depends on how the circuit is put into global trace-delay form.
What matters most is that we \emph{can} do it.

\begin{corollary}
    For any sequential circuit \(
    \iltikzfig{strings/category/f}[box=f, colour=seq, dom=m, cod=n]
    \), there exists at least one valid application of the Mealy rule.
\end{corollary}

\begin{example}
    \index{SR NOR latch}
    The \((\mealyeqn)\) rule is applied to the global trace-delay form
    SR NOR latch in \cref{fig:sr-latch-global-trace-delay}.
    Here the initial state word is just \(\bot\).
\end{example}

The result of applying the \((\mealyeqn)\) reduction still differs from the
image of \(\mealytocircuiti\) as it may have a trace with no delay on it: an
instance of \emph{non-delay-guarded feedback}\index{non-delay-guarded feedback}.

The mere mention of non-delay-guarded feedback may trigger alarm bells in
the minds of those well-acquainted with circuit design.
It is often common in industry to enforce that circuits have no
non-delay-guarded feedback; one might ask if we should also enforce this
tenet in order to stick to `well-behaved' circuits.

\begin{remark}
    \index{category!with feedback}
    \index{category!with delayed trace}
    \emph{Categories with feedback}~\cite{katis2002feedback} are a weakening of
    traced categories that remove the yanking axiom: this effectively makes all
    traces delay-guarded.
    \emph{Categories with delayed trace}~\cite{sprunger2019differentiable}
    weaken this further by removing the sliding axiom, so no components can be
    `pushed round' into the next tick of execution.
    Neither of these are suitable for us as we actually \emph{want} to allow
    non-delay-guarded feedback.
\end{remark}

In fact, careful use of non-delay-guarded feedback can still result in useful
circuits as a clever way of sharing
resources~\cite{malik1994analysis,riedel2004cyclic,mendler2012constructive}.
The minimal circuit to implement a function often \emph{must} be
constructed using cycles~\cite{rivest1977necessity,riedel2003synthesis}.

\begin{example}\label{ex:cyclic-combinational}
    \index{cyclic combinational}
    A particularly famous circuit~\cite{malik1994analysis} which is useful
    despite the presence of non-delay-guarded feedback is shown in
    \cref{fig:cyclic-combinational}, where \(
    \iltikzfig{strings/category/f}[box=f,colour=comb]
    \) and \(
    \iltikzfig{strings/category/f}[box=g,colour=comb]
    \) are arbitrary combinational circuits.
    The trapezoidal gate is a \emph{multiplexer}; it has a vertical
    \emph{control} input and two horizontal \emph{data} inputs.
    The multiplexer is defined as \(
    \iltikzfig{circuits/components/gates/mux}
    \coloneqq
    \iltikzfig{circuits/components/gates/mux-construction}
    \).
    The multiplexer is effectively an if statement: when the control is
    \(\belnapfalse\) the output is the first data input and when it is
    \(\belnaptrue\) the output is the second data input.

    The circuit in \cref{fig:cyclic-combinational} has no state and its trace is
    global so it is already in pre-Mealy form, and has
    non-delay-guarded feedback.
    Despite this, it produces useful output when the control signal is true or
    false:
    when the control signal is \(\belnapfalse\) then the behaviour of the
    circuit is \(
    \circuittofunc[
        \iltikzfig{circuits/examples/cyclic-combinational/reduced-false}
    ]{\interpretation_\star}
    \) and when the control is \(\belnaptrue\) then the behaviour is \(
    \circuittofunc[
        \iltikzfig{circuits/examples/cyclic-combinational/reduced-true}
    ]{\interpretation_\star}
    \).
    The feedback is just used as a clever way to share circuit components.
\end{example}

\begin{figure}
    \centering
    \iltikzfig{circuits/examples/cyclic-combinational/circuit}
    \quad
    \iltikzfig{circuits/examples/cyclic-combinational/circuit-scirc}
    \caption{
        A useful cyclic combinational circuit
        \cite[Fig. 1]{mendler2012constructive}, and a possible interpretation in
        \(\scircsigma\).
    }
    \label{fig:cyclic-combinational}
\end{figure}

A combinational circuit surrounded by non-delay-guarded feedback still
implements a function, as there are no delay components.
Nevertheless, non-delay-guarded feedback does still block our path to future
transformations, so it must be eliminated.
Using a methodology also employed by~\cite{riedel2012cyclic}, we turn to the
Kleene fixed-point theorem.

\begin{lemma}\label{lem:monotone-fixpoint}
    For a monotone function \(\morph{f}{\values^{n+m}}{\valuetuple{n}}\) and
    \(i \in \nat\), let \(\morph{f^i}{\valuetuple{m}}{\valuetuple{n}}\) be
    defined as \(f^0(x)  = f(\bot^n,x)\) and \(f^{k+1}(x) = f(f^k(x), x)\).
    Let \(c\) be the length of the longest chain in the lattice
    \(\valuetuple{n}\).
    Then, for \(j > c\), \(f^c(x) = f^{j}(x)\).
\end{lemma}
\begin{proof}
    Since \(f\) is monotone and \(\values^{n}\) is finite, the former has a
    least fixed point by the Kleene fixed-point theorem.
    This will either be some value \(v\) or the \(\top\) element.
    The most iterations of \(f\) it would take to obtain this fixed point is
    \(c\), i.e.\ the function produces a value one step up the lattice each
    time.
\end{proof}

\begin{definition}[Iteration]\label{def:iteration}
    \index{iteration}
    Given a combinational circuit \(
    \iltikzfig{strings/category/f-2-2}[box=f,colour=comb,dom1=x,dom2=m,cod1=x,cod2=n],
    \)
    its \emph{\(n\)-th iteration} \(
    \iltikzfig{strings/category/f-1-2}[box=f^n,colour=comb,dom=m,cod1=x,cod2=n]
    \) is defined inductively over \(n\) in the following way: \[
        \iltikzfig{circuits/instant-feedback/f0-box}[dom=m,trace=x,cod=n]
        \coloneqq
        \iltikzfig{circuits/instant-feedback/f0-definition}[box=f,dom=m,trace=x,cod=n]
        \qquad
        \iltikzfig{circuits/instant-feedback/fkp1-box}[dom=m,trace=x,cod=n]
        \coloneqq
        \iltikzfig{circuits/instant-feedback/fkp1-definition}[dom=m,trace=x,cod=n]
    \]
\end{definition}

The trace in \(\streami\) is by the least fixed point, computed by repeatedly
applying \(f\) to itself starting from \(\bot\).
The above lemma gives a fixed upper bound for the number of times we need to
apply \(f\) to reach this fixed point, based on the size of the lattice.
We can use this in the syntactic setting.

\begin{definition}[Unrolling]\label{def:unrolling}
    \index{unrolling}
    For an interpretation with values \(\values\), the \emph{unrolling}
    of a combinational circuit \(
    \iltikzfig{strings/category/f-2-2}[box=f,colour=comb,dom1=x,dom2=m,cod1=x,cod2=n]
    \), written \(
    \iltikzfig{strings/category/f-1-2}[box=f^\dagger,colour=comb,dom=m,cod1=x,cod2=n]
    \), is defined as \(
    \iltikzfig{circuits/instant-feedback/fc-box}[dom=m,trace=x,cod=n]
    \) where \(c\) is the length of the longest chain in \(\valuetuple{x}\).
\end{definition}

Using these constructs we can eliminate non-delay-guarded feedback around a
combinational circuit.

\begin{proposition}\label{prop:instant-feedback}
    \index{instant feedback rule}
    The instant feedback rule \(
    \iltikzfig{circuits/instant-feedback/equation-lhs}[box=f]
    \reduction[(\instantfeedbackeqn)]
    \iltikzfig{circuits/instant-feedback/equation-rhs}[box=f]
    \) is sound.
\end{proposition}
\begin{proof}
    By \cref{lem:monotone-fixpoint}, applying the function \(
    (\listvar{x}) \mapsto \proj{x}\left(\circuittofunci[
        \iltikzfig{strings/category/f-2-2}[box=f,colour=comb]
    ]\right)(\listvar{x}, \listvar{v})\) to itself \(c\) times reaches a
    fixpoint.
    The circuit is combinational so each element of the output
    \(\circuittostreami[
        \iltikzfig{strings/category/f-2-2}[box=f,colour=comb]
    ](\sigma)(i)\) is a function; this means that \cref{lem:monotone-fixpoint}
    can be applied to each element.
\end{proof}

\begin{example}\label{ex:sr-latch-unrolled}
    \index{SR NOR latch}
    In \cref{fig:sr-latch-unrolled}, the \(\instantfeedbackeqn\) is applied to
    the SR latch circuit in pre-Mealy form from
    \cref{fig:sr-latch-global-trace-delay}.
\end{example}

\begin{example}
    In \cref{fig:cyclic-combinational-unrolled}, the \((\instantfeedbackeqn)\)
    rule is applied to the cyclic combinational circuit from
    \cref{fig:cyclic-combinational}.
\end{example}

\begin{figure}
    \centering
    \scalebox{0.85}{\iltikzfig{circuits/examples/sr-latch/circuit-instfb-2}}
    \caption{
        Applying the \((\instantfeedbackeqn)\) rule to the SR NOR latch in
        \cref{fig:sr-latch-global-trace-delay}
    }
    \label{fig:sr-latch-unrolled}
\end{figure}
\begin{figure}
    \centering
    \scalebox{0.8}{
        \iltikzfig{circuits/examples/cyclic-combinational/circuit-scirc-instfb}
    }
    \caption{
        Applying the \((\instantfeedbackeqn)\) rule to the circuit from
        \cref{fig:cyclic-combinational}
    }
    \label{fig:cyclic-combinational-unrolled}
\end{figure}

If applied locally for every feedback loop, the \((\instantfeedbackeqn)\)
rule would cause an exponential blowup, but if a circuit is in global
trace-delay form, the rule need only be applied once to the global loop.
Although the value of \(c\) increases as the number of feedback wires increases,
it only does so linearly in the height of the lattice.

\begin{remark}
    \cite{mendler2012constructive} uses a ternary set of values and monotone
    functions to present a notion of \emph{constructive} circuits: the circuits
    that stabilise to unique Boolean values for all Boolean inputs.
    This definition excludes circuits that may oscillate between two values, as
    these are not considered to be monotone circuits.
    Conversely, in our model such circuits \emph{can} be monotone.
    For example a Belnap circuit may alternate between \(\belnaptrue\) and
    \(\belnapfalse\) because these occupy the same level of the lattice.
\end{remark}

With a method to eliminate non-delay-guarded feedback, we can establish the
class of circuits which will act as the keystone of both the operational
semantics in this section and the algebraic semantics of the next.

\begin{definition}[Mealy form]\label{def:delay-guarded}
    \index{Mealy!form}
    A sequential circuit
    \iltikzfig{strings/category/f}[box=f,colour=seq,dom=m,cod=n]
    is in \emph{Mealy form} if it is in the form \(
    \iltikzfig{circuits/productivity/mealy-form}[core=g,state=\listvar{s},dom=m,cod=n,delay=x]
    \).
\end{definition}

\begin{theorem}\label{thm:all-mealy-form}
    For a sequential circuit
    \(\iltikzfig{strings/category/f}[box=f,colour=seq,dom=m,cod=n]\), there
    exists at least one combinational circuit \(
    \iltikzfig{strings/category/f-2-2}[box=g,colour=comb,dom1=x,dom2=m,cod1=x,cod2=n]
    \) and values \(\listvar{s} \in \valuetuple{x}\) such that \(
    \iltikzfig{strings/category/f}[box=f,colour=seq,dom=m,cod=n]
    \reductions
    \iltikzfig{circuits/productivity/mealy-form}[core=g,state=\listvar{s},dom=m,cod=n]
    \) by applying \((\mealyeqn)\) followed by \((\instantfeedbackeqn)\).
\end{theorem}
\begin{proof}
    Any circuit can be assembled into global trace-delay form by
    \cref{lem:trace-delay} and furthermore transformed into pre-Mealy form by
    using \((\mealyeqn)\).
    Since the core of a circuit in pre-Mealy form is combinational and has a
    non-delay-guarded trace, \((\instantfeedbackeqn)\) can be applied to it to
    produce a circuit with only delay-guarded feedback: a circuit in Mealy form.
\end{proof}

Non-delay-guarded feedback can be exhaustively unrolled because the circuit
essentially models a function despite the presence of the trace: this means that
we can transform the circuit without having to `look into the future'.
This is not the case for delay-guarded feedback as the internal state of the
circuit may depend on future inputs.
Indeed, a circuit with delay-guarded feedback may never `settle' on one
internal configuration but rather oscillate between multiple states.
This is simply a facet of sequential circuits and there is nothing we can do
about that.
What we \emph{can} do is show how to \emph{process} inputs at a given tick of
the clock.