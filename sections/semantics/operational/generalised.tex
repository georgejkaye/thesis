\section{Operational semantics for generalised circuits}

When dealing with arbitrary-width wires, the only part of the operational
semantics that does not completely generalise in the obvious way are the value
rules.

\begin{lemma}[Generalised value rules]
    \index{generalised!value rules}
    The following \emph{generalised value rules} are sound:
    \begin{gather*}
        \iltikzfig{circuits/axioms/fork-lhs}[val=\listvar{v}, cod=n]
        \reduction[(\forkeqn)]
        \iltikzfig{circuits/axioms/fork-rhs}[val=\listvar{v}, cod=n]
        \qquad
        \iltikzfig{circuits/axioms/join-lhs}[val1=\listvar{v},val2=\listvar{w}, cod=n]
        \reduction[(\joineqn)]
        \iltikzfig{circuits/axioms/join-rhs}[val1=\listvar{v},val2=\listvar{w}, cod=n]
        \\[1em]
        \iltikzfig{circuits/axioms/stub-lhs}[val=\listvar{v}]
        \reduction[(\stubeqn)]
        \iltikzfig{strings/monoidal/empty}
        \qquad
        \iltikzfig{circuits/axioms/gate-lhs}[gate=p,input=\listvar{v}]
        \reduction[(\gateeqn)]
        \iltikzfig{circuits/axioms/gate-rhs}[gate=p,input=\listvar{v}]
        \\[1em]
        \iltikzfig{circuits/axioms/unbundle-lhs}
        \reduction[(\spliteqn)]
        \iltikzfig{circuits/axioms/unbundle-rhs}
        \qquad
        \iltikzfig{circuits/axioms/bundle-lhs}
        \reduction[(\combineeqn)]
        \iltikzfig{circuits/axioms/bundle-rhs}
    \end{gather*}
\end{lemma}

\begin{lemma}
    Applying the generalised value rules is confluent.
\end{lemma}
\begin{proof}
    There are no overlaps between the rules.
\end{proof}

\begin{lemma}l
    For a generalised combinational circuit \(
    \iltikzfig{strings/category/f}[box=f,colour=comb,dom=\listvar{m},cod=\listvar{n}]
    \) and \(\listvar{v} \in \valuetuple{\listvar{m}}\), there exists a word
    \(\listvar{w} \in \valuetuple{\listvar{n}}\) such that applying the value
    rules exhaustively to \(
    \iltikzfig{circuits/components/circuits/f-applied}[box=f,colour=comb]
    \) terminates at \(
    \iltikzfig{circuits/components/values/vs}[val=\listvar{w}]
    \).
\end{lemma}

With these rules, the inputs to a generalised circuit can be processed.

\begin{corollary}
    For generalised circuit \(
    \iltikzfig{circuits/productivity/mealy-form-applied}[core=f,dom=\listvar{m},cod=\listvar{n},delay=\listvar{x}]
    \) there exist \(
    \listvar{t} \in \valuetuple{\listvar{x}}
    \) and \(
    \listvar{w} \in \valuetuple{\listvar{n}}
    \) such that \(
    \iltikzfig{circuits/productivity/mealy-form-applied}[core=f,dom=\listvar{m},cod=\listvar{n},delay=\listvar{x}]
    \reductions
    \iltikzfig{circuits/productivity/mealy-form-produced}[core=f,dom=\listvar{m},cod=\listvar{n},delay=\listvar{x}]
    \) by applying \((\streamingeqn)\) once followed by the generalised value
    rules exhaustively.
\end{corollary}

\begin{corollary}[Generalised productivity]
    \index{generalised!productivity}
    For sequential circuit \(
    \iltikzfig{strings/category/f}[box=f,colour=seq,dom=\listvar{m},cod=\listvar{n}]
    \) and inputs \(\listvar{v} \in \valuetuple{\listvar{m}}\), there exists
    \(\listvar{w} \in \valuetuple{\listvar{n}}\) such that \(
    \iltikzfig{circuits/productivity/productive-goal-lhs}[box=f,input=\listvar{v},dom=\listvar{m},cod=\listvar{n}]
    \reductions
    \iltikzfig{circuits/productivity/productive-goal-rhs}[box=g,output=\listvar{w},dom=\listvar{m},cod=\listvar{n}]
    \) by applying \(\mealyeqn\), \(\instantfeedbackeqn\) and \(\streamingeqn\)
    once successively followed by the value rules exhaustively.
\end{corollary}

Since register components can now hold words rather than just values, for
observational equivalence we must consider longer input waveforms.

\begin{definition}[Register width]
    \index{register width}
    For a generalised sequential circuit \(
    \iltikzfig{strings/category/f}[box=f,colour=seq,dom=\listvar{m},cod=\listvar{n}]
    \), let \(c_n\) be the number of \(n\)-width delay components
    \(\iltikzfig{circuits/components/waveforms/delay}[width=n]\).
    Then the \emph{register width} of \(
    \iltikzfig{strings/category/f}[box=f,colour=seq,dom=\listvar{m},cod=\listvar{n}]
    \) is computed as \(\Sigma_{n \in \nat}\ c_n \cdot n\).
\end{definition}

\begin{definition}
    \index{observational equivalence!of generalised circuits}
    \nomenclature{\(\sim^+_\interpretation\)}{observational equivalence of sequential circuits}
    We say that two generalised sequential circuits \(
    \iltikzfig{strings/category/f}[box=f,colour=seq,dom=\listvar{m},cod=\listvar{n}]
    \) and \(
    \iltikzfig{strings/category/f}[box=g,colour=seq,dom=\listvar{m},cod=\listvar{n}]
    \) with register width at most \(c\) are said to be
    \emph{observationally equivalent under} \(\interpretation\), written \(
    \iltikzfig{strings/category/f}[box=f,colour=seq]
    \sim^+_{\interpretation}
    \iltikzfig{strings/category/f}[box=g,colour=seq]
    \) if applying productivity produces the same output
    waveforms for all input waveforms \(
    \listlistvar{v} \in \valuetupleseq{\listvar{m}}\) of length
    \(|\values^c| + 1\).
\end{definition}

The observational equivalence results from the previous section then generalise
nicely to the multicoloured case.

\begin{theorem}
    Given two sequential circuits \(
    \iltikzfig{strings/category/f}[box=f,colour=seq,dom=\listvar{m},cod=\listvar{n}]
    \) and \(
    \iltikzfig{strings/category/f}[box=g,colour=seq,dom=\listvar{m},cod=\listvar{n}]
    \), \(
    \iltikzfig{strings/category/f}[box=f,colour=seq,dom=\listvar{m},cod=\listvar{n}]
    \sim^+_{\interpretation}
    \iltikzfig{strings/category/f}[box=g,colour=seq,dom=\listvar{m},cod=\listvar{n}]
    \) if and only if \(
    \circuittostreami[
        \iltikzfig{strings/category/f}[box=f,colour=seq,dom=\listvar{m},cod=\listvar{n}]
    ]
    =
    \circuittostreami[
        \iltikzfig{strings/category/f}[box=g,colour=seq,dom=\listvar{m},cod=\listvar{n}]
    ]
    \).
\end{theorem}

\begin{corollary}
    \(\sim^+_\interpretation\) is the largest adequate congruence on
    \(\scircsigmag\).
\end{corollary}

\begin{definition}
    \index{scircsigmag@\(\scircsigmagobs\)}
    \nomenclature{\(\scircsigmagobs\)}{PROP of generalised sequential circuits quotiented by observational equivalence}
    Let \(\scircsigmagobs\) be defined as \(\scircsigmag / \sim^+_{\interpretation}\).
\end{definition}

\begin{corollary}
    There is an isomorphism \(\scircsigmaig \cong \scircsigmagobs\).
\end{corollary}