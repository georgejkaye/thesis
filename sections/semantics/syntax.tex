\chapter{Syntax}\label{chap:syntax}

Our \emph{soup du jour} is that of
\emph{sequential synchronous digital circuits}
constructed from primitive components such as logic gates or transistors.
These circuits are \emph{sequential} as they have a notion of \emph{state}:
outputs can be impacted by inputs in previous cycles rather than solely those in
the current cycle, and \emph{synchronous} because their state changes in time
with some global clock.

\begin{remark}
    Digital (electric) circuits are not to be confused with \emph{electronic}
    circuits of switches and resistors.
    Essentially, the difference boils down to the difference between traced
    categories and compact closed categories: digital circuits have a clear
    notion of \emph{causality} whereas electronic circuits are \emph{relational}
    in nature.
    For a study of the latter, see \cite{boisseau2022string}.
\end{remark}

\begin{remark}
    The content of this section is a refinement of~\cite[Section 2]{ghica2024fully}.
\end{remark}

\section{Circuit signatures}

To construct circuits, we define a category in which a morphism
\(m \to n\) is a circuit with \(m\) inputs and \(n\) outputs.
Rather than restricting to any particular gate set, we parameterise a given
category of circuits over a \emph{circuit signature} containing details about
the available components.

\begin{definition}[Circuit signature, value, primitive symbol]
    \index{circuit signature}
    \index{value}
    \index{gate symbol}
    \nomenclature{\(\circuitsignature\)}{circuit signature}
    \nomenclature{\(\values\)}{set of values in a circuit signature}
    \nomenclature{\(\circuitsignaturearity\)}{arity operation for primitives in a circuit signature}
    \nomenclature{\(\circuitsignaturearity\)}{coarity operation for primitives in a circuit signature}
    A \emph{circuit signature} \(\circuitsignature\) is a tuple \((
    \values,
    \disconnected,
    \circuitsignaturegates,
    \circuitsignaturearity,
    \circuitsignaturecoarity
    )\) where \(\values\) is a finite set of \emph{values}, \(
    \disconnected \in \values
    \) is a \emph{disconnected} value, \(\circuitsignaturegates\) is a (usually
    finite) set of \emph{primitive symbols}, \(
    \morph{\circuitsignaturearity}{\circuitsignaturegates}{\nat}
    \) is an \emph{arity} function and \(
    \morph{\circuitsignaturecoarity}{\circuitsignaturegates}{\nat}
    \) is a \emph{coarity} function.
\end{definition}

A particularly important signature, and one which we will turn to for the
majority of examples in this thesis, is that of gate-level circuits.

\begin{example}[Gate-level circuits]\label{ex:belnap-signature}
    \index{gate-level circuits}
    \index{Belnap!signature}
    \nomenclature{\(\belnapsignature\)}{signature for gate-level circuits}
    The circuit signature for \emph{gate-level circuits} is \(
    \belnapsignature \coloneqq (
    \belnapvalues,
    \bot,
    \belnapgates,
    \belnaparity,
    \belnapcoarity
    )\), where \(
    \belnapvalues \coloneqq \{\bot, \belnapfalse, \belnaptrue, \top\}
    \), respectively representing \emph{no} signal, a \emph{false} signal, a
    \emph{true} signal and \emph{both} signals at once, \(
    \belnapgates \coloneqq \{\andgate,\orgate,\notgate\}
    \), \(
    \belnaparity \coloneqq
    \andgate \mapsto 2,
    \orgate \mapsto 2,
    \notgate \mapsto 1
    \) and \(
    \belnapcoarity \coloneqq - \mapsto 1,
    \)
\end{example}

\begin{remark}
    Using four values may come as a surprise to those expecting the usual
    `true' and 'false' logical values.
    This logical system is an old idea going back to
    Belnap~\cite{belnap1977useful} who remarked that this is `how a computer
    should think'.
    \index{Belnap!logic}
    Rather than just thinking about how much \emph{truth content} a value
    carries, the four value system adds a notion of \emph{information content}:
    the \(\bot\) value is no information at all (a disconnected wire), whereas
    the \(\top\) value is both true and false information at once
    (a short circuit).
\end{remark}

\section{Combinational circuits}

Before diving straight into sequential circuits, we will define a category of
\emph{combinational circuits}.\index{combinational circuits}\index{combinational}
These are circuits with no state; they compute \emph{functions} of their inputs.

\begin{definition}[Combinational circuit generators]
    Given a circuit signature \(
    \circuitsignature = (
    \circuitsignaturevalues,
    \bullet,
    \circuitsignaturegates,
    \circuitsignaturearity,
    \circuitsignaturecoarity
    )
    \), let the set \(\generators[\ccirc{}]\) of
    \emph{combinational circuit generators} be defined as the set containing \(
    \iltikzfig{circuits/components/gates/gate}[gate=g,colour=comb,dom=\circuitsignaturearity(g),cod=\circuitsignaturecoarity(g)]
    \) for each \(g \in \circuitsignaturegates\),
    \(\iltikzfig{strings/structure/monoid/init}[colour=comb]\),
    \(\iltikzfig{strings/structure/comonoid/copy}[colour=comb]\),
    \(\iltikzfig{strings/structure/monoid/merge}[colour=comb]\), and
    \(\iltikzfig{strings/structure/comonoid/discard}[colour=comb]\).
    We write \(\ccircsigma\) for the freely generated PROP
    \(\smc{\generators[\ccirc{}]}\).
    \nomenclature{\(\ccircsigma\)}{PROP of combinational circuits}
\end{definition}

Each primitive symbol \(p \in \circuitsignaturegates\) has a corresponding
generator in \(\ccircsigma\).
The remaining generators are \emph{structural} generators \index{structural generators}
for manipulating
wires: these are present regardless of the signature.
In order, they are for \emph{introducing} wires, \emph{forking}
wires, \emph{joining} wires and \emph{eliminating} wires.

\begin{example}
    The gate generators of \(\ccirc{\belnapsignature}\) are \(
    \iltikzfig{circuits/components/gates/and}
    \), \(
    \iltikzfig{circuits/components/gates/or}
    \), and \(
    \iltikzfig{circuits/components/gates/not}
    \).
\end{example}

When drawing circuits, the coloured backgrounds of generators will often be
omitted in the interests of clarity.
Since the category is freely generated, morphisms are defined by
juxtaposing the generators in a given signature sequentially or in parallel with
each other, the identity and the symmetry.
Arbitrary combinational circuit morphisms defined in this way are drawn as light
boxes \iltikzfig{strings/category/f}[box=f,colour=comb,dom=m,cod=n].

\begin{notation}\label{not:arbitrary-width-structure}
    The structural generators are only defined on single bits, but it is
    straightforward to define versions for arbitrary bit wires.
    Much like we often draw multiple wires as one
    (\cref{not:arbitrary-width-wires}), we can also draw these `thicker'
    constructs in a similar way to the single-bit versions:
    \begin{gather*}
        \iltikzfig{strings/structure/monoid/init}[colour=comb,obj=n]
        \quad
        \iltikzfig{strings/structure/comonoid/copy}[colour=comb,obj=n]
        \quad
        \iltikzfig{strings/structure/monoid/merge}[colour=comb,obj=n]
        \quad
        \iltikzfig{strings/structure/comonoid/discard}[colour=comb,obj=n]
    \end{gather*}
    These composite constructs are defined inductively over the width of the
    wires.

    \begin{center}
        \begin{minipage}{0.48\textwidth}
            \centering
            \(\iltikzfig{strings/structure/comonoid/copy-unit}[obj=0,colour=comb]
            \coloneqq
            \iltikzfig{strings/monoidal/empty}
            \)
            \quad
            \(
            \iltikzfig{strings/structure/comonoid/copy}[obj=n+1,colour=comb]
            \coloneqq
            \iltikzfig{strings/structure/comonoid/copy-construction}
            \)

            \vspace{1em}

            \(
            \iltikzfig{strings/structure/monoid/init-unit}[obj=0,colour=comb]
            \coloneqq
            \iltikzfig{strings/monoidal/empty}
            \)\quad\(
            \iltikzfig{strings/structure/monoid/init}[obj=n+1,colour=comb]
            \coloneqq
            \iltikzfig{strings/structure/monoid/init-construction}
            \)
        \end{minipage}
        \quad
        \begin{minipage}{0.48\textwidth}
            \centering
            \(\iltikzfig{strings/structure/monoid/merge-unit}[obj=0,colour=comb]
            \coloneqq
            \iltikzfig{strings/monoidal/empty}
            \)
            \quad
            \(
            \iltikzfig{strings/structure/monoid/merge}[obj=n+1,colour=comb]
            \coloneqq
            \iltikzfig{strings/structure/monoid/merge-construction}
            \)

            \vspace{1em}

            \(\iltikzfig{strings/structure/comonoid/discard-unit}[obj=0,colour=comb]
            \coloneqq
            \iltikzfig{strings/monoidal/empty}
            \)\quad\(
            \iltikzfig{strings/structure/comonoid/discard}[obj=n+1,colour=comb]
            \coloneqq
            \iltikzfig{strings/structure/comonoid/discard-construction}
            \)
        \end{minipage}
    \end{center}
\end{notation}

\begin{remark}
    As mentioned during \cref{not:arbitrary-width-wires}, wires of zero width
    are usually drawn as empty space; in a similar fashion the zero width fork,
    join, and elimination constructs can be drawn as empty space or using
    `faded' wires.
\end{remark}

\begin{example}[More logic gates]
    The \(\andgate\), \(\orgate\), and \(\notgate\) gates are not the only logic
    gates used in circuit design.
    A \(\nandgate\) gate \(
    \iltikzfig{circuits/components/gates/nand}
    \) acts as the inverse of an \(\andgate\) gate: it
    outputs true if none of the inputs are true.
    Similarly, a \(\norgate\) gate \(
    \iltikzfig{circuits/components/gates/nor}
    \) is the inverse of an \(\orgate\) gate.
    These two gates can be constructed in terms of the primitive gates in
    \(\belnapsignature\):

    \[
        \iltikzfig{circuits/components/gates/nand}
        \coloneqq
        \iltikzfig{circuits/components/gates/nand-construction}
        \qquad
        \iltikzfig{circuits/components/gates/nor}
        \coloneqq
        \iltikzfig{circuits/components/gates/nor-construction}
    \]

    Another type of gate is the \(\xorgate\) gate \(
    \iltikzfig{circuits/components/gates/xor}
    \), which outputs true if and only if exactly one of the inputs is
    true.
    In \(\ccirc{\belnapsignature}\) this is constructed as \[
        \iltikzfig{circuits/components/gates/xor}
        \coloneqq
        \iltikzfig{circuits/components/gates/xor-construction-1}
        =
        \iltikzfig{circuits/components/gates/xor-construction-2}.
    \]
\end{example}

\begin{example}[Half adder]\label{ex:half-adder}
    The \(\xorgate\) gate is used in a classic combinational circuit known as a
    \emph{half adder}, the basic building block of circuit arithmetic.
    A half adder takes two inputs and computes their \emph{sum} modulo
    \(2\) and the resulting \emph{carry}.
    That is to say, \(0+0\) has sum \(0\) and carry \(0\), \(1+0\) and \(0+1\)
    have sum \(1\) and carry \(0\), and \(1+1\) has sum \(0\) and carry \(1\).

    The sum is computed using an \(\xorgate\) gate and the carry by an
    \(\andgate\) gate.
    The design of a half adder along with its construction in
    \(\ccirc{\belnapsignature}\) is shown below.
    \[
        \iltikzfig{circuits/examples/half-adder/circuit}
        \qquad
        \iltikzfig{circuits/examples/half-adder/circuit-ccirc}
        =
        \iltikzfig{circuits/examples/half-adder/circuit-ccirc-2}
    \]
\end{example}

\section{Sequential circuits}

Combinational circuits compute functions of their inputs, but have no internal
state.
This is all very well for doing simple calculations, but for all but the most
simple of circuits we need to be able to have \emph{memory}.
As we have mentioned earlier, such circuits are called
\emph{sequential circuits}.

Circuits gain state with \emph{delay} and \emph{feedback}.
The latter means we need to move into a symmetric \emph{traced} monoidal category.

\begin{definition}[Sequential circuits]\index{sequential circuits}\index{sequential}
    \nomenclature{\(\scircsigma\)}{PROP of sequential circuits}
    For a circuit signature \(\circuitsignature\) with value set \(\values\),
    let \(\scircsigma\) be the STMC freely generated over the generators of
    \(\ccircsigma\) along with new generators \(
    \iltikzfig{circuits/components/values/vs}[val=v]
    \) for each \(v \in \values \setminus \bullet\), and a generator \(
    \iltikzfig{circuits/components/waveforms/delay}
    \).\index{value}\index{delay}
\end{definition}

The first set of generators are \emph{instantaneous values} for each value in
\(\values \setminus \bullet\).
Value generators are intended to be interpreted as an \emph{initial state}:
in the first cycle of execution they will emit their value, and produce the
disconnected \(\bullet\) value after that.
This is why there is no \(\bullet\) value generator; instead it is a
\emph{combinational} generator \(
\iltikzfig{strings/structure/monoid/init}[colour=comb]
\) intended to always produce the \(\bullet\) value.

\begin{notation}
    Although \(
    \iltikzfig{strings/structure/monoid/init}[colour=comb]
    \) is itself not a sequential value, when we refer to an arbitrary value
    \(
    \iltikzfig{circuits/components/values/vs}[val=v]
    \), \(v\) can be any value in \(\values\) including \(\bullet\).
    For a word of values \(\listvar{v} \in \valuetuple{n}\) (again possibly
    including \(\bullet\)), we may draw multiple value generators collapsed into
    one as \(
    \iltikzfig{circuits/components/values/vs}[val=\listvar{v},width=n]
    \), defined inductively over \(\listvar{v}\) as
    \begin{gather*}
        \iltikzfig{circuits/components/values/vs}[val=\varepsilon,width=0]
        \coloneqq
        \iltikzfig{strings/monoidal/empty}
        \qquad
        \iltikzfig{circuits/components/values/vs-even-longer}[val=v\listvar{w},width=n+1]
        \coloneqq
        \iltikzfig{circuits/components/values/vs-construction}
    \end{gather*}
\end{notation}

\begin{example}
    The `values' of \(\scirc{\belnapsignature}\) are \(
    \iltikzfig{strings/structure/monoid/init}[colour=comb]
    \), \(
    \iltikzfig{circuits/components/values/vs}[val=\belnapfalse]
    \), \(
    \iltikzfig{circuits/components/values/vs}[val=\belnaptrue]
    \), \(
    \iltikzfig{circuits/components/values/vs}[val=\top]
    \); the first is a combinational generator and the latter three are
    sequential.
\end{example}

The delay component is the opposite of a value; in the first cycle of execution
it is intended to produce the \(\bullet\) value, but in future cycles it outputs
the signal it received in the previous cycle.

\begin{remark}
    While the mathematical interpretation of a delay is straightforward, the
    physical aspect of a digital circuit it models is less clear.
    An obvious interpretation could be that the delay models a D flipflop in
    a clocked circuit, so the delay is one clock cycle.
    A more subtle interpretation is the `minimum obervable duration'; in this
    case the delay models inertial delay on wires up to some fixed precision.
\end{remark}

\begin{notation}
    Like values, we can derive delay components for arbitrary-bit wires, drawn
    like \(
    \iltikzfig{circuits/components/waveforms/delay}[width=n]
    \).
    \begin{gather*}
        \iltikzfig{circuits/components/waveforms/unit-delay}[width=0]
        \coloneqq
        \iltikzfig{strings/monoidal/empty}
        \qquad
        \iltikzfig{circuits/components/waveforms/delay}[width=n+1]
        \coloneqq
        \iltikzfig{circuits/components/waveforms/delay-construction}
    \end{gather*}
\end{notation}

Often one may also want to think about delays with some explicit `initial
value', like a sort of register.
This is so common that we introduce special notation for it.

\begin{notation}[Register]\label{not:register}\index{register}
    For a word \(\listvar{v} \in \valuetuple{m}\), let \(
    \iltikzfig{circuits/components/waveforms/register}
    \coloneqq
    \iltikzfig{circuits/components/waveforms/register-shorthand}
    \).
\end{notation}

To distinguish them from combinational circuits, arbitrary sequential circuit
morphisms are drawn as green boxes \(
\iltikzfig{strings/category/f}[box=f,colour=seq,dom=m,cod=n]
\).

\begin{example}[SR latch]\label{ex:sr-latch}
    \index{SR NOR latch}
    A sequential circuit one might come across early on in an electronics
    textbook is the \emph{SR NOR latch}, one of the simplest registers.
    A possible design and interpretation in \(\scirc{\belnapsignature}\) are
    illustrated below.
    \begin{gather*}
        \iltikzfig{circuits/examples/sr-latch/real-circuit}
        \qquad
        \iltikzfig{circuits/examples/sr-latch/circuit}
    \end{gather*}

    SR NOR latches are used to hold state.
    They have two inputs: Reset (\(\mathsf{R}\)) and Set (\(\mathsf{S}\)), and
    two outputs \(\mathsf{Q}\) and \(\overline{\mathsf{Q}}\) which are always
    negations of each other.
    When Set receives a true signal, the \(\mathsf{Q}\) output is forced true,
    and will remain as such even if the Set input stops being pulsed true.
    It is only when the Reset input is pulsed true that the \(\mathsf{Q}\)
    output will return to false.
    (It is illegal for both Set and Reset to be pulsed high simultaneously; this
    issue is fixed in more complicated latches).

    SR latches work because of delays in how gates and wires transmit signals;
    one of the feedback loops between the two \(\norgate\) gates will `win'.
    We can model this in \(\scircsigma\) by using a different number of delay
    generators between the top and the bottom of the latch.
    We have opted for just the one because otherwise the upcoming examples
    become excessively complicated, but any number would do, so long as the top
    and bottom differ.
\end{example}

\section{Generalised circuit signatures}

In a circuit signature, gates are assigned a number of input and output wires.
This serves us well when we want to model lower level circuits in which we
really are dealing with single-bit wires.
However, when designing circuits it is often advantageous to work at a higher
level of abstraction with `thicker' wires carrying words of information.
For example, the values in the circuits could be used to represent binary
numbers.

This can still be modelled in \(\scircsigma\) `as is' by using lots of parallel
wires to connect to the various primitives, but this can get messy with wires
all over the place.
Instead, we will introduce a generalisation of circuit signatures in which these
thicker buses of wires are treated as first-class objects.

\begin{definition}[Generalised circuit signature]\index{generalised!circuit signature}
    A \emph{generalised circuit signature} \(\circuitsignature\) is a tuple \((
    \values,
    \disconnected,
    \circuitsignaturegates,
    \circuitsignaturearity,
    \circuitsignaturecoarity
    )\) where \(\values\) is a finite set of values, \(
    \disconnected \in \values
    \) is a \emph{disconnected} value, \(\circuitsignaturegates\) is a (usually
    finite) set of \emph{gate symbols}, \(
    \morph{\circuitsignaturearity}{\circuitsignaturegates}{\natplus^\star}
    \) is an \emph{arity} function and \(
    \morph{\circuitsignaturecoarity}{\circuitsignaturegates}{\natplus^\star}
    \) is a \emph{coarity} function.
\end{definition}

In a generalised circuit signature, primitives are typed with input and output
\emph{words} rather than just natural numbers.

\begin{example}
    The generalised circuit signature for \emph{simple arithmetic circuits} is
    \(
    \belnapsignature^+ \coloneqq \left(
    \belnapvalues,
    \bot,
    \belnapgates^+,
    \belnaparity^+,
    \belnapcoarity^+
    \right)
    \), where \begin{gather*}
        \belnapgates
        \coloneqq \{
        \andgate_{k,n},
        \orgate_{k,n},
        \notgate_{n},
        \mathsf{add}_n
        \,|\,
        n \in \natplus
        \}
        \\[0.5em]
        \belnaparity^+(\andgate_{k,n}) \coloneqq [n \,|\, i < k]
        \quad
        \belnaparity^+(\orgate_{k,n}) \coloneqq [n \,|\, i < k]
        \\[0.5em]
        \belnaparity^+(\notgate_{n}) \coloneqq [n]
        \quad
        \belnaparity^+(\mathsf{add}_n) \coloneqq [n,n]
        \\[0.5em]
        \belnapcoarity^+
        \coloneqq
        \andgate_{k,n} \mapsto [n],
        \orgate_{k,n} \mapsto [n],
        \notgate_n \mapsto [n],
        \mathsf{add}_n \mapsto [n]
    \end{gather*}
    The gates \(\andgate_{k,n}\) and \(\orgate_{k,n}\) are gates that operate
    on \(k\) input wires of width \(n\); similarly the \(\notgate_n\) gate
    operates on input wires of width \(n\).
    The \(\mathsf{add}_n\) component represents an adder that takes as input
    two \(n\)-bit wires and outputs their \(n\)-bit sum.
\end{example}

Just like a monochromatic circuit signature generates monochromatic PROPs, a
generalised circuit signature generates \(\natplus\)-coloured PROPs.

\begin{definition}
    \index{ccircsigmag@\(\ccircsigmag\)}
    \nomenclature{\(\ccircsigmag\)}{PROP of generalised combinational circuits}
    For a generalised circuit signature \(\circuitsignature\), let the set
    \(\generators[\ccirc{}^+]\) of
    \emph{generalised combinational circuit generators} be defined as the set
    containing
    \begin{gather*}
        \iltikzfig{circuits/components/gates/gate}[gate=g,colour=comb,dom=\circuitsignaturearity(g),cod=\circuitsignaturecoarity(g)]
        \,
        \text{for each}\ g \in \circuitsignaturegates
        \\[1em]
        \iltikzfig{strings/structure/monoid/init}[colour=comb,obj=n]
        \quad
        \iltikzfig{strings/structure/comonoid/copy}[colour=comb,obj=n]
        \quad
        \iltikzfig{strings/structure/monoid/merge}[colour=comb, obj=n]
        \quad
        \iltikzfig{strings/structure/comonoid/discard}[colour=comb,obj=n]
        \quad
        \iltikzfig{strings/strictifiers/expand}[colour=comb,obj=n]
        \quad
        \iltikzfig{strings/strictifiers/collapse}[colour=comb,obj=n]
        \quad
        \text{for each}\ n \in \natplus
    \end{gather*}
    We write \(\ccircsigmag\) for the freely generated \(\natplus\)-coloured
    PROP \(\smc{\generators[\ccirc{}^+]}\).
\end{definition}

Most of the generators  in \(\ccircsigmag\) are fairly straightforward
generalisations of the primitives in \(\ccircsigma\) to act on each
colour (width) of wires.
The only new generators are the \emph{bundlers} at the end of the bottom row;
their intended meaning is that they can be used to \emph{split} and
\emph{combine} bundles of wire buses into bundles with different widths.
These constructs were first proposed by Wilson et al in~\cite{wilson2023string}
as a notation for \emph{non-strict categories}.
We take inspiration from their observation that a similar idea could also be
applied to strict categories.

\begin{example}[ALU]
    \index{ALU}
    The computation of a CPU is performed by an \emph{arithmetic logic unit},
    or ALU for short.
    An ALU takes some input wires of a fixed width and performs an operation
    on them given some control signal.
    While ALUs can often perform many different operations, we will look at an
    example operating on four-bit wires that performs a bitwise \(\andgate\)
    operation when the control is false, and an addition when the control is
    true.
    This ALU will also produce an output indicating if the sum is zero, and
    the sign of the sum; these auxiliary outputs only produce useful output when
    the addition operation is selected.

    \begin{center}
        \iltikzfig{circuits/examples/alu}
    \end{center}

    To apply the single-bit control to the four-bit \(\andgate\) gates, the
    top bundler and forks are used to create a wire containing only the
    original bit.

    The sum is zero if all of the bits are false.
    To determine this, the \(\orgate_{1,4}\) gate folds the four-bit sum into
    a one-bit value which is true if at least one of the bits is true.
    The \(\notgate_{1,1}\) inverts the output to produce true if there are no
    true bits.

    In two's complement representation, the most significant bit indicates if
    the sum is negative.
    To model this, the lower bundler splits the four-bit sum into its
    constituent bits, discarding the least significant three.
\end{example}

Sequential circuits are generalised in the same way.

\begin{definition}
    \index{scircsigmag@\(\scircsigmag\)}
    \nomenclature{\(\scircsigmag\)}{PROP of generalised sequential circuits}
    For a generalised circuit signature \(\circuitsignature\), let the set
    \(\generators[\scirc{}^+]\) of
    \emph{generalised sequential circuit generators} be the set of
    generalised combinational circuit generators along with
    \(
    \iltikzfig{circuits/components/values/vs}[val=\listvar{v},width=n]
    \) for each \(n \in \natplus\) and \(\listvar{v} \in \valuetuple{n}\), and
    \(
    \iltikzfig{circuits/components/waveforms/delay}[width=n]
    \) for each \(n \in \natplus\).
    We write \(\scircsigmag\) for the freely generated PROP
    \(\stmc{\generators[\scirc{}^+]}\).
\end{definition}

Most of the upcoming results will be shown for the monochromatic case, as the
proofs are more elegant.
However, most of the results also generalise to the coloured case, and this will
be remarked upon throughout.