\section{Reversing the wires}\label{sec:traced-compact-closed}

When considering string diagrams for symmetric monoidal categories, there is a
strict notion of causality: it is not possible to create a cycle from the output
of a box to its input, and outputs may only be joined to inputs.
This enforces an implicit \emph{left-to-right} directionality across the page.

This may not always be desirable: one may wish to model a feedback
loop, or perhaps enforce some condition by unifying two outputs.
To do this sort of thing, we must examine symmetric monoidal categories with
some extra structure.

\subsection{Symmetric traced monoidal categories}

First we consider removing the acyclicity condition.

\begin{definition}[Symmetric traced monoidal category \cite{joyal1996traced}]\label{def:stmc}
    \index{symmetric traced monoidal category}
    \index{monoidal category!symmetric traced}
    \index{category!symmetric traced monoidal}
    \index{sliding}
    \index{superposing}
    \index{tightening}
    \index{yanking}
    \index{vanishing}
    \nomenclature{\(\tr^X_{A,B}\)}{trace of \(X\) with input \(A\) and output \(B\)}
    A \emph{symmetric traced monoidal category} (STMC) is a symmetric monoidal
    category \(\mcc\) equipped with a family of functions \(
    \morph{
        \trace{X}[A][B]{-}
    }{
        \mcc(X \tensor A, X \tensor B)
    }{
        \mcc(A, B)
    }
    \) for any three objects \(A\), \(B\) and \(X\) subject to the equations in
    \cref{fig:stmc-equations}.
\end{definition}

This means that if we have a morphism \(\morph{f}{X \tensor A}{X \tensor B}\) in
a STMC, we also have the morphism \(\morph{\trace{X}[A][B]{f}}{A}{B}\).
Traced categories were given the string diagrammatic treatment in
\cite{joyal1996traced}, in which the trace is depicted as a loop.
\index{string diagram!trace}
\begin{gather*}
    \trace{X}[A][B]{
        \iltikzfig{strings/category/f-2-2}[colour=white,box=f,dom1=X,dom2=A,cod1=X,cod2=B]
    }
    =
    \iltikzfig{strings/traced/trace-rhs}[colour=white,box=f,dom=A,cod=B,trace=X]
\end{gather*}
The equations of STMCs then have pleasant graphical interpretations, shown in
\cref{fig:stmc-equations}.
Note that as with the equations of SMCs, these equations amount to deforming the
diagram without altering connections between boxes, so do not need to be
applied explicitly when performing equational reasoning.

Usually we will omit the subscripts from \(\trace{X}[A][B]{f}\) and write it
simply as \(\trace{X}{f}\).

\begin{figure}
    \setlength{\tabcolsep}{3pt}
    \centering
    \begin{tabular}{ccc}
        \multicolumn{3}{c}{\textbf{Tightening}}
        \\[0.4em]
        \scalebox{0.9}{\(\trace{X}[A][D]{
                \id[X] \tensor f \seq g \seq \id[X] \tensor h
            }
            \)}
        \qquad
         &
        \(=\)
         &
        \scalebox{0.9}{\(f \seq \trace{X}[B][C]{g} \seq h\)}
        \\[0.6em]
        \scalebox{0.85}{\(
            \iltikzfig{strings/traced/naturality-lhs}[colour=white,box1=f,box2=g,box3=h,dom=A,cod=D]
            \)}\qquad
         &
        \(=\)
         &
        \scalebox{0.85}{\(
            \iltikzfig{strings/traced/naturality-rhs}[colour=white,box1=f,box2=g,box3=h,dom=A,cod=D]
            \)}
    \end{tabular}
    \\[0.6em]
    \rule[0.6em]{\textwidth}{0.1mm}
    \\[0.1em]
    \begin{tabular}{ccccccc}
        \multicolumn{3}{c}{\textbf{Sliding}}
        &&
        \multicolumn{3}{c}{\textbf{Yanking}}
        \\[0.4em]
        \scalebox{0.9}{\(
            \trace{X}[A][B]{f \seq g \tensor \id[B]}
            \)}
         &
        \(=\)
         &
        \scalebox{0.9}{\(
            \trace{Y}[A][B]{g \tensor \id[A] \seq f}
            \)}
         &
         &
        \scalebox{0.9}{\(
            \trace{X}[X][X]{\swap{X}{X}}
            \)}
         &
        \(=\)
         &
        \scalebox{0.9}{\(\id[X]\)}
        \\[0.6em]
        \scalebox{0.9}{\(
            {\iltikzfig{strings/traced/sliding-lhs}[colour=white,box1=f,box2=g,dom=A,cod=B]}
            \)}
         &
        \(=\)
         &
        \scalebox{0.9}{\(
            \iltikzfig{strings/traced/sliding-rhs}[colour=white,box1=f,box2=g,dom=A,cod=B]
            \)}
         &
        \qquad\qquad
         &
        \scalebox{0.9}{\(
            \iltikzfig{strings/traced/yanking-lhs}[obj=X,colour=white]
            \)}
         &
        \(=\)
         &
        \scalebox{0.9}{\(
            \iltikzfig{strings/category/identity}[obj=X,colour=white]
            \)}
    \end{tabular}
    \\[0.6em]
    \rule[0.6em]{\textwidth}{0.1mm}
    \\[0.1em]
    \begin{tabular}{ccccccc}
        \multicolumn{3}{c}{\textbf{Vanishing}}
        &&
        \multicolumn{3}{c}{\textbf{Superposing}}
        \\[0.4em]
        \scalebox{0.9}{\(
            \trace{X}[A][B]{\trace{Y}[X \tensor A][X \tensor B]{f}}
            \)
        }
         &
        \(=\)
         &
        \scalebox{0.9}{
            \(
            \trace{X \tensor Y}[A][B]{f}
            \)
        }
         &
         &
        \scalebox{0.9}{
            \( \trace{X}[A][B]{f} \tensor g\)
        }
         &
        \(=\)
         &
        \scalebox{0.9}{\(
            \trace{X}[A \tensor C][B \tensor D]{f \tensor g}
            \)}
        \\[0.6em]
        \scalebox{0.85}{
            \iltikzfig{strings/traced/vanishing-lhs}[colour=white,box=f,dom=A,cod=B,trace1=X,trace2=Y]
        }
         &
        \(=\)
         &
        \scalebox{0.85}{
            \iltikzfig{strings/traced/vanishing-rhs}[colour=white,box=f,dom=A,cod=B,trace1=X,trace2=Y]
        }
         &
         &
        \scalebox{0.85}{
            \iltikzfig{strings/traced/superposing-lhs}[colour=white,box1=f,box2=g,dom1=A,cod1=B,dom2=C,cod2=D,trace=X]
        }
         &
        \(=\)
         &
        \scalebox{0.85}{
            \iltikzfig{strings/traced/superposing-rhs}[colour=white,box1=f,box2=g,dom1=A,cod1=B,dom2=C,cod2=D,trace=X]
        }
    \end{tabular}
    \caption{Equations of STMCs in string diagram notation}
    \label{fig:stmc-equations}
\end{figure}

\begin{example}
    The classic example of a symmetric traced monoidal category is the category
    \(\finvectk{k}\), with finite dimensional vector spaces over a field \(k\)
    as objects, and linear maps as morphisms.
    The monoidal product is the tensor product of vector spaces and the
    trace is an operation known as the `partial trace'.
\end{example}

In the context of computer science, traces are often used to model
\emph{fixpoints}~\cite{hasegawa1997recursion}; we will examine this in further
detail in \cref{sec:cartesian-structure} as a possible application of our work
on traced string diagram rewriting.

\subsection{Compact closed categories}

While wires can flow backwards across the page in a traced category, regular
left-to-right flow must still be in effect at a wire's \emph{endpoints}, so
outputs must be connected to inputs.
We will now consider another setting in which this is not the case.

\begin{definition}[Compact closed category]
    \index{compact closed category}
    \index{category!compact closed}
    \index{unit}
    \index{counit}
    \index{dual}
    \nomenclature{\(\ccunit[A]\)}{unit of a compact closed category}
    \nomenclature{\(\cccounit[A]\)}{counit of a compact closed category}
    \nomenclature{\(\dual{A}\)}{dual of object \(A\)}
    A \emph{compact closed category} (CCC) is a symmetric monoidal category in
    which every object \(X\) has a \emph{dual} \(\dual{X}\) equipped with
    morphisms called the \emph{unit} \(
    \morph{\ccunit[A]}{I}{\dual{A} \tensor A}
    \) and the \emph{counit} \(
    \morph{\cccounit[A]}{A \tensor \dual{A}}{I}
    \) such that the diagrams in~\cref{fig:snake-equations} commute.
\end{definition}

\begin{figure}
    \centering
    \includestandalone{figures/category/coherences/snake-1}
    \quad
    \includestandalone{figures/category/coherences/snake-2}
    \caption{Commutative diagrams of a compact closed category}
    \label{fig:snake-equations}
\end{figure}

In string diagrams, the dual is drawn as a wire flowing from right-to-left
instead of left-to-right; when labelling wires with objects we will drop the
notation for duals and recover the information solely from directionality of the
wires.
\index{string diagram!dual}
The unit and counit `bend' wires: the unit is drawn as \(
\iltikzfig{strings/compact-closed/cup}[obj1=\dual{A},obj2=A,colour=white]
\)
\index{cup}\index{string diagram!compact closed category}\index{string diagram!cup} and the counit as \(
\iltikzfig{strings/compact-closed/cap}[obj1=A,obj2=\dual{A},colour=white]
\)\index{cap}\index{string diagram!cap}.
As a result of this units and counits are often referred to as \emph{cups} and
\emph{caps} respectively.
The two equations of compact closed categories are depicted as in
\cref{fig:ccc-equations}; this should provide some insight as to why they are
often referred to as \emph{snake equations}\index{snake equations}.

\begin{figure}
    \centering
    \iltikzfig{strings/compact-closed/snake-1}
    \(=\)
    \iltikzfig{strings/compact-closed/identity-backward}[colour=white]
    \quad
    \iltikzfig{strings/compact-closed/snake-2}
    \(=\)
    \iltikzfig{strings/compact-closed/identity-forward}[colour=white]
    \caption{Equations of CCCs string diagram notation}
    \label{fig:ccc-equations}
\end{figure}

There are some cases where the actual directionality of wires is irrelevant;
we only care about the ability to bend wires.

\begin{definition}[Self-dual compact closed category]
    \index{self-dual compact closed category}
    \index{compact closed category!self-dual}
    A compact closed category is \emph{self-dual} if for every object \(A\),
    \(\dual{A} \coloneqq A\).
\end{definition}

In a self-dual compact closed category, we do not need to label wires with
arrows.

\subsection{Traced vs compact closed}

The graphical notation is particularly suggestive of links between the
trace, the cup and the cap.
This is no coincidence, as there is a well-known result that allows one to
construct a trace in a compact closed setting.

\begin{proposition}[Canonical trace {\cite[Prop.\ 3.1]{joyal1996traced}}]
    \index{canonical trace}
    \label{prop:canonical-trace}
    Given a compact closed category \(\mcc\), a trace on \(\mcc\) can be
    constructed as follows: \[
        \trace{X}{f}
        \coloneqq
        \ccunit[X] \tensor \id[A]
        \seq
        \id[\dual{X}] \tensor f
        \seq
        (\swap{\dual{X}}{X} \seq \cccounit[X]) \tensor \id[B]
    \]
    \[
        \iltikzfig{strings/compact-closed/canonical-trace}[colour=white,box=f,dom=A,cod=B,trace=X]
    \]
    This is called the \emph{canonical trace}.
\end{proposition}

In this thesis we are primarily concerned with traced categories, but a plethora
of related work is based in the compact closed realm.
The canonical trace allows us to adapt existing results for our setting as well.

\begin{remark}
    \index{Int-construction}
    It is also possible to consider the other direction: using the
    \emph{Int-construction}~\cite{joyal1996traced}, given any STMC
    \(\mcc\) one can construct a compact closed category \(\mathsf{Int}(\mcc)\).
    However, this will not be of relevance to us.
\end{remark}

\subsection{Adding more structure}

While working in a traced or compact closed category allows one to model more
processes than working in a vanilla symmetric monoidal category, it is by no
means the only way in which categories can have structure.
However, when stacking multiple structures on top of each other, we must be
careful that we do not accidentally create something unwanted.

One common structure in a category is \emph{finite products}, which were
discussed in \cref{def:product}.

\begin{theorem}[{\cite[Thm.\ 1]{houston2008finite}}]
    If a compact closed category has finite products then it also has finite
    coproducts.
\end{theorem}

It may not be the case that we want coproducts, so in this case a compact closed
category is not suitable.
On the other hand, adding structure to a category can cause it to become
\emph{trivial}: it can only have one object and one morphism.
This can arise when enforcing more structure on the tensor product.

\begin{definition}[Cartesian category]\label{def:cartesian-category}
    \index{category!Cartesian category}
    \index{Cartesian category}
    A monoidal category is \emph{Cartesian} if its tensor product is given by
    the category-theoretic product.
\end{definition}

\begin{lemma}
    In a Cartesian category, the monoidal unit is the terminal object.
\end{lemma}
\begin{proof}
    We must show that any arrow \(\morph{f}{A}{I}\) is unique.
    Using the left unitor \(\morph{\lambda_A}{A}{I \times A}\) and the
    projection maps \(\morph{\proj{0}}{I \times A}{I}\) and
    \(\morph{\proj{1}}{I \times A}{A}\), we can construct the following diagram:
    \begin{center}
        \begin{tikzcd}
            & A \arrow[swap]{dl}{f} \arrow{d}{\lambda} \arrow{dr}{\id[A]} &
            \\
            I
            &
            I \times A \arrow{l}{\proj{0}} \arrow[swap]{r}{\proj{1}}
            &
            A
        \end{tikzcd}
        \vspace{0.5em}
    \end{center}
    By the universal property of the product, any morphism \(\morph{f}{A}{I}\)
    can be uniquely factored as \(\lambda \seq \proj{0}\).
\end{proof}

Accordingly, the unique maps \(\morph{!_A}{A}{I}\) are drawn string
diagrammatically as \(
\iltikzfig{strings/structure/comonoid/discard}
\), which will help with reasoning.

\begin{proposition}
    If a compact closed category is Cartesian, then it must be trivial i.e.\ it
    must have only one object and morphism.
\end{proposition}
\begin{proof}
    We show that any object \(A\) is isomorphic to the monoidal unit \(I\) by
    proving that the composites \(A \to I \to A\) and \(I \to A \to I\) are both
    the identity.
    For the former, \(\morph{!_A}{A}{I}\) is unique because \(I\)
    is the terminal object, and \(I \to A\) is uniquely determined to be
    \(\eta_A \seq \proj{1}\) by the universal property of the product.
    \begin{center}
        \begin{tikzcd}
            & I \arrow[swap]{dl}{f} \arrow{d}{\eta_A} \arrow{dr}{\id[A]} &
            \\
            \dual{A}
            &
            \dual{A} \times A \arrow{l}{\proj{0}} \arrow[swap]{r}{\proj{1}}
            &
            A
        \end{tikzcd}
        \vspace{0.5em}
    \end{center}

    So we must show that \(!_A \seq \eta_A \seq \proj{0} = \id[A]\).
    This is clear when using string diagrams, noting that the cap is equal to
    the unique morphism \(\morph{!_{A \times \dual{A}}}{A \times \dual{A}}{I}\).
    \[
        \iltikzfig{strings/compact-closed/cartesian-trivial/step-0}
        =
        \iltikzfig{strings/compact-closed/cartesian-trivial/step-1}
        =
        \iltikzfig{strings/compact-closed/cartesian-trivial/step-2}
        =
        \iltikzfig{strings/compact-closed/cartesian-trivial/step-3}
    \]
    The latter equality holds because there is exactly one morphism
    \(I \to I\).
    So any object \(A\) is isomorphic to \(I\).
\end{proof}

This means that if we want to reason about wires going backwards in Cartesian
categories, we must do so in a traced setting.