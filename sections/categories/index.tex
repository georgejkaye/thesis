\chapter{A crash course in category theory}

When setting about developing a mathematical foundation for something,
the first step is to decide exactly what foundation to pick.
As a basis, we will model circuits as \emph{terms} constructed by combining
primitive generators in sequence and in parallel (\cref{sec:terms}).
But terms on their own are not enough, as there may be many terms that represent
the same structure.
To use the necessary `structural' equations and identify these terms, we turn
to \emph{category theory}\index{category theory}, in which mathematical structures are expressed using
\emph{categories}: collections of \emph{objects} and composable \emph{morphisms}
between them (\cref{sec:categories}).

Category theory started out in the early 1940s, when Eilenberg and Mac Lane
were working on problems in algebraic topology.
The pair's original goal was to define what it meant for a homomorphism to be
`natural'~\cite{eilenberg1942natural}, a concept which nowadays we would call a
\emph{natural transformation}.
To properly define this required the notions of a \emph{functor}
and a \emph{category}; the latter was first presented in
1945~\cite{eilenberg1945general}.

It has since turned out that category theory can be used as something of a
`universal language' in mathematics, as many familiar mathematical structures
can be expressed solely in terms of objects and morphisms.
This makes category theory an incredibly powerful tool; rather than having to
repeatedly prove things about concrete structures, abstract results can be
instantiated to a plethora of settings.

\begin{example}
    A common structure in set theory is the \emph{Cartesian product}, in
    which two sets \(A\) and \(B\) can be combined to create a new set
    \(A \times B\) containing pairs of elements \((a,b)\).
    A product is equipped with two functions
    \(\morph{\mathsf{fst}}{A \times B}{A}\) and
    \(\morph{\mathsf{snd}}{A \times B}{B}\) to extract the first and second
    element from a pair respectively.

    Many properties of Cartesian products are not specific to sets,
    but arise from the presence of the two projections.
    Category theory has a notion of \emph{categorical product} in which
    sets are generalised to objects and the projection functions to morphisms;
    this allows results about products to be applied to a wide range of
    structures in addition to the Cartesian product, such as set intersection,
    greatest lower bounds, greatest common divisors, and more complicated
    algebraic structures.
\end{example}

Category theory is especially relevant to computer scientists, and it has shed
light upon topics such as semantics of programming
languages~\cite{oles1982categorytheoretic}, databases~\cite{spivak2012functorial},
probabilistic programming~\cite{cho2019disintegration,fritz2020synthetic}, and
machine learning~\cite{fong2019backprop,cockett2020reverse}.
Category theory also underpins functional programming languages such as
Haskell and OCaml, which use categorical concepts for computational purposes.

Using category theory, a circuit can be modelled as a morphism \(A \to B\), but
this is not expressive enough to work with complicated, two-dimensional
circuits.
By using functors (\cref{sec:functors}) and natural transformations
(\cref{sec:natural-transformations}), we can define a class of categories known
as \emph{symmetric monoidal categories}~\cite{maclane1963natural}, in which the
usual \emph{sequential} composition is joined by \emph{parallel} composition
and morphisms for \emph{swapping} inputs (\cref{sec:monoidal-categories}).
In digital circuits, we additionally create a
\emph{feedback loop} by attaching some outputs to inputs.
This too can be viewed in a categorical perspective in the form of a
\emph{symmetric traced monoidal category}~\cite{joyal1996traced}
(\cref{sec:traced-compact-closed}).

To reason with circuits we must additionally make use of a
\emph{monoidal theory}
(\cref{sec:monoidal-theories}), a set of context-specific equations between
the primitive components in a category.
Monoidal theories have been successfully applied to study quantum
protocols~\cite{coecke2008interacting}, dynamical
systems~\cite{baez2015categories,fong2016categorical}, signal flow
diagrams~\cite{bonchi2014categorical,bonchi2015full,bonchi2017refinement,bonchi2021survey},
linear algebra~\cite{bonchi2017interacting,zanasi2015interacting,bonchi2019graphical,boisseau2022graphical},
finite-state automata~\cite{piedeleu2021string,piedeleu2022finite}, electrical
circuits~\cite{boisseau2022string}, automatic
differentiation~\cite{alvarez-picallo2023functorial}, synthetic
chemistry~\cite{gale2023categorical}, and first order
logic~\cite{bonchi2024diagrammatic}, among many others.

To begin, we will take a meander through the various categorical definitions and
notation that underpin the mathematical work of this thesis.
A similar outline can be found in most category theory textbooks, but
inspiration was taken in particular from the opening of
\cite{ghica2023string}.

\section{Terms}\label{sec:terms}

We are interested in using category theory as a tool to study \emph{circuits}
built from some primitive components, or \emph{generators}.

\begin{definition}[Generators]\label{def:generators}
    A set of \emph{generators} \(\generators\) is a set equipped with two
    functions \(\morph{\dom,\cod}{\generators}{\nat}\).
\end{definition}

Generators are the primitive building blocks of terms: their domains and
codomains specify how many input and output wires they have.
Terms are defined by combining these primitives together.

\begin{definition}[Term]
    \label{def:terms}
    Let \(\generators\) be a set of generators.
    A \(\generators\)-term is written \(\morph{f}{m}{n}\)
    where \(m,n \in \nat\).
    The set of \(\generators\)-terms, denoted \(\sigmaterms\), is
    generated as follows:
    \begin{center}
        \begin{bprooftree}
            \AxiomC{\(\phi \in \generators\)}
            \UnaryInfC{\(
                \morph{\phi}{\dom[\phi]}{\cod[\phi]} \in \sigmaterms
                \)}
        \end{bprooftree}
        \begin{bprooftree}
            \AxiomC{\phantom{\(\phi\)}}
            \UnaryInfC{\(\morph{\id[1]}{1}{1} \in \sigmaterms\)}
        \end{bprooftree}

        \vspace{1em}

        \begin{bprooftree}
            \AxiomC{}
            \UnaryInfC{\(\morph{\id[0]}{0}{0} \in \sigmaterms\)}
        \end{bprooftree}
        \begin{bprooftree}
            \AxiomC{}
            \UnaryInfC{\(
                \morph{\swap{1}{1}}{2}{2} \in \sigmaterms
                \)}
        \end{bprooftree}

        \vspace{1em}

        \begin{bprooftree}
            \AxiomC{\(\morph{f}{m}{n} \in \sigmaterms\)}
            \AxiomC{\(\morph{g}{n}{p} \in \sigmaterms\)}
            \BinaryInfC{\(\morph{f \seq g}{m}{p} \in \sigmaterms\)}
        \end{bprooftree}
        \begin{bprooftree}
            \AxiomC{\(\morph{f}{m}{n} \in \sigmaterms\)}
            \AxiomC{\(\morph{g}{p}{q} \in \sigmaterms\)}
            \BinaryInfC{\(
                \morph{f \tensor g}{m+p}{n+q} \in \sigmaterms
                \)}
        \end{bprooftree}
    \end{center}
\end{definition}

\(\Sigma\)-terms are constructed recursively.
There are four base cases: a generator from \(\generators\) with appropriate
inputs and outputs; an \emph{identity} for single wires, empty space, and
a \emph{symmetry} for swapping over two wires.
The two inductive cases are called \emph{composition} and \emph{tensor}
respectively.
Intuitively, these can be thought of as generating larger terms by composing
subterms in sequence or parallel.

Although here identities and symmetries only operate on single wires, it is
a simple exercise to define them for larger numbers of wires.

\begin{notation}[Composite identities]\label{not:composite-identities}
    Composite identities \(\id[n]\) are defined inductively for \(n \in \nat\)
    as \(
    \id[0] \coloneqq \id[0]\) and \(
    \id[k+1]
    \coloneqq
    \id[k] \tensor \id[1]
    \).
\end{notation}
\begin{notation}[Composite symmetries]\label{not:composite-symmetries}
    Composite symmetries \(\swap{m}{n}\) for \(m, n \in \nat\) are defined
    inductively as \[
        \swap{0}{n}
        \coloneqq
        \id[n]
        \quad
        \swap{m}{0}
        \coloneqq \id[m]
        \quad
        \swap{k+1}{l+1}
        \coloneqq
        \id[k] \tensor \swap{1}{l} \tensor \id[1]
        \seq
        \swap{k}{l} \tensor \swap{1}{1}
        \seq
        \id[l] \tensor \swap{k}{1} \tensor \id[1]
    \]
\end{notation}

\(\Sigma\)-terms will be abbreviated to `terms' when the signature is clear from
context.

\begin{example}\label{ex:terms}
    Let \(
    \Sigma_g
    \coloneqq
    \{\morph{\andgate}{2}{1}, \morph{\orgate}{2}{1}, \morph{\notgate}{1}{1}\}
    \) be a set of \emph{logic gate generators}.
    Examples of terms in \((\Sigma_g)_\mathsf{t}\) include
    \begin{gather*}
        \morph{(\orgate \seq \id[1]) \tensor ((\id[1] \tensor \notgate) \seq \andgate)}{4}{2}
        \\
        \morph{((\id[1] \tensor \id[1]) \seq \orgate) \tensor ((\id[1] \tensor \notgate) \seq \andgate)}{4}{2}
        \\
        \morph{((\id[1] \tensor \id[1]) \tensor (\id[1] \tensor \notgate)) \seq (\orgate \tensor \andgate)}{4}{2}
    \end{gather*}
\end{example}

\section{String diagrams}

Even simple terms described using one-dimensional text strings quickly become
indecipherable.
Fortunately, terms have a graphical syntax known as
\emph{string diagrams}~\cite{joyal1991geometry} that makes reading terms far
more intuitive.
In a string diagram, a generator \(\morph{\phi}{m}{n}\) is drawn as a box with
\(m\) inputs and \(n\) outputs \(
\iltikzfig{strings/category/generator-wires-tile}[dom=m,cod=n]
\), the identity \(\id[1]\) as a wire \(
\iltikzfig{strings/category/identity-tile}
\), the empty identity \(\id[0]\) as empty space \(
\iltikzfig{strings/monoidal/empty-tile}
\), and the symmetry as two wires swapping over \(
\iltikzfig{strings/symmetric/symmetry-tile}
\).
Composite terms are drawn as boxes \(
\iltikzfig{strings/category/f-wires-tile}[box=f,dom=m,cod=n]
\); composition is then depicted as \emph{horizontal juxtaposition} and
tensor as \emph{vertical juxtaposition}; both are illustrated in
\cref{fig:strings-composition}.
\begin{figure}
    \centering
    \begin{gather*}
        \iltikzfig{strings/category/f-wires-tile}[box=f,dom=m,cod=n]
        \seq
        \iltikzfig{strings/category/f-wires-tile}[box=g,dom=n,cod=p]
        =
        \iltikzfig{strings/category/composition-tiles}[box1=f,box2=g,dom1=m,dom2=n,cod2=p]
        \\[0.5em]
        \iltikzfig{strings/category/f-wires-tile}[box=f,dom=m,cod=n]
        \tensor
        \iltikzfig{strings/category/f-wires-tile}[box=g,dom=p,cod=q]
        =
        \iltikzfig{strings/monoidal/tensor-tiles}[box1=f,box2=g,dom1=m,cod1=n,dom2=p,cod2=q]
    \end{gather*}
    \caption{Sequential and parallel composition of string diagrams}
    \label{fig:strings-composition}
\end{figure}

\begin{remark}
    The direction that the `flow' of string diagrams travels from inputs to
    outputs is a hotly-debated topic; in this thesis we adopt the left-to-right
    approach.
    If you disagree, you could rotate the document by ninety degrees or
    use a mirror.
\end{remark}

\begin{example}\label{ex:term-diagrams}
    Recall the set of generators from \cref{ex:terms}.
    We will draw each logic gate using the usual notation, i.e.\
    \[
        \andgate \coloneqq \iltikzfig{circuits/components/gates/and-tile}
        \quad
        \orgate \coloneqq \iltikzfig{circuits/components/gates/or-tile}
        \quad
        \notgate \coloneqq \iltikzfig{circuits/components/gates/not-tile}
    \]
    Now the terms in \cref{ex:terms} can be illustrated as:
    \begin{gather*}
        \iltikzfig{strings/terms/example-0}
        \qquad
        \iltikzfig{strings/terms/example-1}
        \qquad
        \iltikzfig{strings/terms/example-2}
    \end{gather*}
    Note that while each term has a very different text string, and is tiled
    differently in each diagram, the connectivity of the thick wires is actually
    the same.
\end{example}

The above example illuminates the drawback of reasoning with term strings or
even the `tile' graphical syntax; there are terms which are `morally' the same
but because of variations in how they are constructed they are \emph{not} the
same term syntactically.
We need some equations to identify terms appropriately; we will soon see how
these equations can be derived using the mathematical structure of a
\emph{category}.

\begin{notation}\label{not:arbitrary-width-wires}
    So far we have drawn multiple wires in parallel either explicitly or
    by using vertical dots.
    In the interests of clarity, we will now start to collapse multiple wires
    into a single wire, annotated with the appropriate number when not clear
    from context: so \(
    \iltikzfig{strings/category/f-tile}[box=f,dom=m,cod=n]
    \coloneqq
    \iltikzfig{strings/category/f-wires-tile}[box=f,dom=m,cod=n]
    \).

    Note that these values \(m\) or \(n\) could be zero, and as such can just be
    drawn as empty space.
    It may be useful to explicitly draw wires of zero width for applying
    reasoning steps; we will do so with a `faded' wire, like \(
    \iltikzfig{strings/category/identity-unit-tile}[obj=0,colour=white]
    \).
\end{notation}

\section{Coloured terms}

In \(\Sigma\)-terms, the wires are \emph{monochromatic}; there is no
distinguishing between them.
Sometimes it is advantageous to \emph{annotate} wires with some information: in
the realm of terms this is known as assigning the wires \emph{colours} or
\emph{sorts}.
When working with coloured terms, we need to fix the set of colours before
specifying a set of generators.

\begin{notation}
    We say that a set \(C\) is \emph{countable} if it is finite or countably
    infinite, i.e.\ there exists a set \(X \subseteq \nat\) such that there is a
    bijection \(C \cong X\).
\end{notation}

\begin{remark}
    Usually the set of colours is finite, but we will see later in this thesis
    how having a colour for every single natural number might be useful.
\end{remark}

In the monochromatic world the interface of a generator can be specified solely
by two natural numbers \(m\) and \(n\), as there are \(m\) input wires and
\(n\) output wires.
When the wires are coloured, more information is needed: the inputs and outputs
must be specified in terms of their colours and their ordering.

\begin{notation}[Words]\label{not:words}
    Given a set \(A\), the set of (finite) words of elements of \(A\) is denoted
    \(\freemon{A}\).
    Words are written \(x_0x_1x_2{\cdots}x_{n-1}\); arbitrary words are written
    with an overline \(
    \listvar{x}, \listvar{y}, \listvar{z}... \in \freemon{A}
    \).
    Given two words \(\listvar{x}, \listvar{y}\), their concatenation is
    denoted \(\listvar{xy}\).
    For an element \(a \in A\), the concatenation of this element \(n\) times is
    written \(a^n\), e.g.\ \(a^3 \coloneqq aaa\).
    Given a word \(\listvar{x}\), its \emph{length} is denoted
    \(\wordlen{\listvar{x}}\); for \(i < \wordlen{\listvar{x}}\) the
    \(i\)-\emph{th element of \(\listvar{x}\)} is denoted \(\listvar{x}(i)\),
    i.e.\ in word \(\listvar{x} \coloneqq abc\), \(\listvar{x}(0) = a\),
    \(\listvar{x}(1) = b\) and \(\listvar{x}(2) = c\).
\end{notation}

\begin{definition}[Coloured generators]
    For a countable set \(C\), a set of \emph{\(C\)-coloured generators}
    \(\Sigma\) is a set equipped with two functions
    \(\morph{\dom,\cod}{\Sigma}{\freemon{C}}\).
\end{definition}

Coloured terms are generated in the same way as before, but terms can only be
composed if the coloured wires in the interfaces match up.

\begin{definition}[Coloured terms]
    For a countable set \(C\) and a set of \(C\)-coloured generators, a
    \((C,\Sigma)\)-term is written \(\morph{f}{\listvar{m}}{\listvar{n}}\),
    where \(\listvar{m},\listvar{n} \in \freemon{C}\).
    The set of \((C,\Sigma)\)-terms, denoted \((C,\Sigma)_\mathsf{t}\), is
    generated as the monochromatic set of terms, but with an
    identity and symmetry for each \(c \in C\), composition between
    terms whose output and input words agree on colours, and addition replaced
    by word concatenation.
\end{definition}

\begin{remark}
    When the set of colours is a singleton \(C \coloneqq \{\bullet\}\), the
    \(C\)-coloured terms are just the monochromatic terms.
\end{remark}

When drawing coloured terms string diagrammatically, the wires that connect the
generators are coloured appropriately.

\begin{example}
    Recall the logic gate generators from \cref{ex:terms}; let us now say that
    the \(\orgate\) gate is rated for a different voltage, so it is no longer
    compatible with the other components.
    We can model this with coloured terms; let
    \(C_\mathsf{g} \coloneqq \{\bullet,\redbullet\}\) be a set of
    \emph{voltage colours} and let \(
    \Sigma_g^+
    \coloneqq
    \{\morph{\andgate}{\bullet\bullet}{\bullet}, \morph{\orgate}{\redbullet\redbullet}{\redbullet}, \morph{\notgate}{\bullet}{\bullet}\}
    \).
    We draw the `red' \(\orgate\) appropriately as \(
    \iltikzfig{circuits/components/gates/red-or}
    \).
    Now we can recreate the diagrams from \cref{ex:term-diagrams} in the
    coloured setting.
    Note how all wires attached to the \(\orgate\) gate must also be red
    to ensure the colours all match up.
    \begin{gather*}
        \iltikzfig{strings/terms/example-0-coloured}
        \qquad
        \iltikzfig{strings/terms/example-1-coloured}
        \qquad
        \iltikzfig{strings/terms/example-2-coloured}
    \end{gather*}
\end{example}

\section{Categories}

Terms are purely syntax, so two terms are only equal if they are constructed in
precisely the same way.
As we have already seen, this is far too strong a relation; there may be
many terms that, while constructed in different ways, look the same when drawn
out as a diagram modulo the tiling.
We need a way to discard the tile boundaries and work just with
wires and generators.

To identify the terms that describe the same process, we need to impose some
equations on terms.
There is an obvious question: what are these equations?

\begin{example}
    Consider the term \(f \seq \id[n]\), which can be read as `run \(f\) and
    then do nothing' and is drawn as \(
    \iltikzfig{strings/category/identity-l-lhs-tiles}[dom=m,cod=n]
    \).
    Clearly this is the same \(
    \iltikzfig{strings/category/f-tile}[dom=m,cod=n]
    \) but with the left wire elongated a bit.
    So one equation we need is \(
    \iltikzfig{strings/category/identity-l-lhs-tiles}[dom=m,cod=n]
    =
    \iltikzfig{strings/category/f-tile}[dom=m,cod=n]
    \).
\end{example}

We could continue like this, thinking up the `right' equations, but this is a
very ad-hoc way of working.
It turns out that all the work has actually already been done before us, in the
form of \emph{symmetric monoidal categories}.

We start by defining what a \emph{category} is.

\begin{definition}[Category]
    \label{def:category}
    A \emph{category} \(\mcc\) consists of a class of \emph{objects}
    \(\ob{\mcc}\); a class of \emph{morphisms} \(\mor{\mcc}{A}{B}\)
    for every pair of objects \(A, B \in \ob{\mcc}\); and a \emph{composition}
    operation \(
    \morph{
        {-} \circ {=}
    }{
        \mor{\mcc}{B}{C} \times \mor{\mcc}{A}{B}
    }{
        \mor{\mcc}{A}{C}
    }
    \) such that
    \begin{itemize}
        \item composition is \emph{unital}: for every \(
              A \in \ob{\mcc}
              \) there exists an \emph{identity morphism} \(
              \id[A] \in \mor{\mcc}{A}{A}
              \) satisfying \(
              f \circ \id[A] = f = \id[B] \circ f
              \) for any \(
              f \in \mor{\mcc}{A}{B}
              \); and
        \item composition is \emph{associative}: for any morphisms \(
              f \in \mor{\mcc}{A}{B}
              \), \(
              g \in \mor{\mcc}{B}{C}
              \) and \(h \in \mor{\mcc}{C}{D}\), \(
              (h \circ g) \circ f = h \circ (g \circ f).
              \)
    \end{itemize}
\end{definition}

A morphism \(f \in \mor{\mcc}{A}{B}\) is also called an \emph{arrow}, and will
often be written \(\morph{f}{A}{B}\) accordingly.
The subscripts on the object and morphism classes are also often omitted.

\begin{remark}
    The definition of a category uses \emph{classes} rather than sets as one
    might expect; this is due to size issues and Russell's paradox regarding
    the impossibility of the `set of all sets'.
    Later on we will encounter categories where the class of objects really is
    too big to be encapsulated in a set, such as the category of categories.
\end{remark}

By treating terms as morphisms, it is clear to see how they they slip into the
categorical setting.
When discussing terms in the previous section it was more intutiive to use
composition `left-to-right' rather than the `right-to-left' composition
\(\circ\).

\begin{notation}
    \emph{Diagrammatic order} composition is written as
    \(f \seq g \coloneqq g \circ f\).
\end{notation}

\begin{figure}
    \begin{center}
        \begin{tabular}{ccccc}
            \(
            \id[A] \seq f = f
            \)
             &  &
            \(f \seq \id[B] = f\)
             &  &
            \((f \seq g) \seq h = f \seq (g \seq h)\)
            \\[1em]
            \(
            \iltikzfig{strings/category/identity-l-lhs}[box=f,colour=white,dom=A,cod=B]
            =
            \iltikzfig{strings/category/f}[box=f,dom=A,cod=B,colour=white]
            \)
             &  &
            \(
            \iltikzfig{strings/category/identity-r-lhs}[box=f,dom=A,cod=B,colour=white]
            =
            \iltikzfig{strings/category/f}[box=f,dom=A,cod=B,colour=white]
            \)
             &  &
            \(
            \iltikzfig{strings/category/associativity-lhs}[box1=f,box2=g,box3=h,colour=white,dom=A,cod=C]
            =
            \iltikzfig{strings/category/associativity-rhs}[box1=f,box2=g,box3=h,colour=white,dom=A,cod=C]
            \)
        \end{tabular}
    \end{center}
    \caption{
        Equations of a category
    }
    \label{fig:c-equations}
\end{figure}

The equations of categories are illustrated with string diagram notation
in~\cref{fig:c-equations}.
With these equations in play, we can drop the tile boundaries around terms
comprised only of generators, identities and sequential composition.
However, this only helps with one-dimensional diagrams; how do we deal with
\emph{parallel} composition?
To do this we will require some more structure on our categories.

\subsection{Commutative diagrams}

Equations in category theory can be expressed using \emph{commutative diagrams}.
For example, the unitality and associativity of composition can be illustrated
as follows:

\begin{center}
    \includestandalone{figures/category/coherences/unitality}
    \quad
    \includestandalone{figures/category/coherences/associativity}
\end{center}

We say that the above two diagrams \emph{commute} precisely because \(
\id[B] \circ f = f = f \circ \id[A]
\) and \((h \circ g) \circ f = h \circ (g \circ f)\): no matter which path one
takes, the results are equal.

\subsection{Examples of categories}

Categories generalise a plethora of mathematical structures: some examples will
now be provided.

\begin{example}[Preorder]
    A \emph{preorder} is a binary relation \(\leq\) on a set \(X\) which is
    reflexive and transitive.
    Any preorder generates a category \(\mcc_\leq\): the objects are the
    elements of \(X\) and \(\mcc_\leq(x, y)\) contains exactly one morphism if
    \(x \leq y\) and none otherwise.
\end{example}

\begin{example}[Sets]
    The category \(\set\) has sets as objects and functions as morphisms.
    Composition is then just function composition.
    Compare this with the category \(\rel\), which has sets as objects and
    \emph{relations} as morphisms.
    There is also a category \(\finset\) containing only the \emph{finite} sets
    and functions between them.
\end{example}

\begin{example}[Posets]
    A \emph{partial order} on a set \(A\) is a reflexive, antisymmetric and
    transitive relation \({\leq} \subseteq A \times B\).
    A set equipped with a partial order is called a
    \emph{partially ordered set}, or \emph{poset} for short.
    For posets \((A,{\leq_A})\) and \((B,{\leq_B})\), a function
    \(\morph{f}{A}{B}\) is called \emph{monotone} if \(a \leq_A a^\prime\)
    implies that \(f(a) \leq_B f(a^\prime)\).

    Much like how sets form a category, posets form the category \(\pos\), where
    \(\ob{\pos}\) are sets and \(\mor{\pos}{X}{Y}\) are the monotone functions
    \(X \to Y\).
\end{example}

\begin{example}[Monoids]
    A \emph{monoid} is a tuple \((A, *, \bullet)\) where \(A\) is a set called
    the \emph{carrier}, \(\morph{*}{A \times A}{A}\) is a binary operation
    called the \emph{multiplication}, and \(\bullet \in A\) is an element called
    the \emph{unit}, such that \(a * \bullet = a = \bullet * a\) for any
    \(a \in A\).
    A \emph{monoid homomorphism} between two monoids \((A, *, e_A)\) and
    \((B, +, e_B)\) is a map \(\morph{h}{A}{B}\) such that
    \(h(a * a^\prime) = h(a) + h(a^\prime)\) and \(h(e_A) = e_B\).
    There is a category \(\mon\) with the objects as monoids and morphisms as
    their homomorphisms.
\end{example}

\begin{example}[Product category]
    Given two categories \(\mcc\) and \(\mcd\), their \emph{product category}
    \(\mcc \times \mcd\) is the category with objects are defined as \(
    \ob{(\mcc \times \mcd)} \coloneqq \ob{\mcc} \times \ob{\mcd}
    \) and the morphisms as \[
        \mor{(\mcc \times \mcd)}{(A, A^\prime)}{(B, B^\prime)}
        \coloneqq
        \{
        (f, f^\prime)
        \,|\,
        f \in \mor{\mcc}{A}{B},
        f^\prime \in \mor{\mcd}{A^\prime}{B^\prime}
        \}
    \]
\end{example}

\subsection{Properties of objects and morphisms}

As we have established, the whole point of category theory is to \emph{abstract}
away from concrete settings.
This means that rather than proving results about specific objects and
morphisms, we can often reuse results about particular \emph{properties} they
may have.

\begin{definition}[Initial object]
    An object \(C\) in a category \(\mcc\) is \emph{initial} if, for any other
    object \(X \in \mcc\) there exists a unique morphism \(C \to X\), the
    so-called `absurd function'.
\end{definition}

\begin{example}
    In \(\set\), the initial object is the empty set \(\emptyset\), as there is
    a unique function from \(\emptyset\) to any set \(X\).
\end{example}

Most notions in category also have a \emph{dual} variant, in which all the
arrows are flipped.

\begin{definition}[Terminal object]
    An object \(C\) in a category \(\mcc\) is \emph{terminal} if, for any other
    object \(X \in \mcd\) there exists a unique morphism \(X \to C\).
\end{definition}

\begin{example}
    In \(\set\), the terminal object is the singleton set \(\{\star\}\):
    from any set \(X\) there is a unique function \(X \to \{\star\}\); namely
    the function with action \(x \mapsto \star\).
\end{example}

There are also some properties of morphisms that we will make use of.

\begin{definition}[Monomorphism]
    A morphism \(\morph{f}{A}{B} \in \mcc\) is called a \emph{monomorphism} (or
    simply \emph{mono} for short) if for any two morphisms
    \(\morph{g_1,g_2}{C}{A}\), if \(f \circ g_1 = f \circ g_2\), then
    \(g_1 = g_2\).
\end{definition}

One can think of monomorphisms as morphisms which are \emph{left-cancellative}.
There is also a way to describe \emph{invertible} morphisms.

\begin{definition}[Isomorphism]
    A morphism \(\morph{f}{A}{B} \in \mcc\) is called an \emph{isomorphism} (or
    simply \emph{iso} for short) if there also exists a morphism \(
    \morph{\inverse{f}}{B}{A} \in \mcc
    \) such that \(
    \inverse{f} \circ f = \id[A]
    \) and \(
    f \circ \inverse{f} = \id[B]
    \).
\end{definition}

\begin{example}
    In \(\set\), the monomorphisms are the injective functions and the
    isomorphisms are the bijective functions.
\end{example}

\subsection{Universal constructions}

One of the core ideas in category theory is that of a \emph{universal property}:
a unique morphism that shows that a certain candidate construction is indeed the
`best' way to describe something.
If something satisfies a universal property, then we know that we can safely
apply the relevant results without worrying about the precise construction.

\begin{definition}[Product]
    Given a category \(\mcc\) and objects \(A,B \in \mcc\), their \emph{product}
    is an object \(A \times B\) equipped with a pair of morphisms
    \(\morph{p_0}{A \times B}{A}\) and \(\morph{p_1}{A \times B}{B}\) called
    \emph{projections} such that for every other object \(Z\) with pair of
    morphisms \(\morph{f}{Z}{A}\) and \(\morph{g}{Z}{B}\), there exists a unique
    morphism \(\morph{u}{Z}{A \times B}\) such that the following diagram
    commutes:
    \begin{center}
        \includestandalone{figures/category/diagrams/product}
    \end{center}
    A category \(\mcc\) is said to \emph{have products} if the product exists
    for all pairs of objects \(A,B \in \mcc\).
\end{definition}

\begin{example}
    The product is \(\set\) is the Cartesian product.
\end{example}

As usual, there is the dual notion to consider.

\begin{definition}[Coproduct]
    Given a category \(\mcc\) and objects \(A,B \in \mcc\), their \emph{coproduct}
    is an object \(A + B\) equipped with a pair of morphisms
    \(\morph{i_0}{A}{A + B}\) and \(\morph{i_1}{B}{A + B}\) called
    \emph{injections} such that for every other object \(Z\) with pair of
    morphisms \(\morph{f}{A}{B}\) and \(\morph{g}{B}{Z}\), there exists a unique
    morphism \(\morph{u}{A + B}{Z}\) such that the following diagram
    commutes:
    \begin{center}
        \includestandalone{figures/category/diagrams/coproduct}
    \end{center}
    A category \(\mcc\) is said to \emph{have coproducts} if the coproduct
    exists for all pairs of objects \(A,B \in \mcc\).
\end{definition}

\begin{example}
    The coproduct in \(\set\) is the disjoint union.
\end{example}

Finally, we will look at a useful construction concerning \emph{pairs} of
morphisms with a common domain.

\begin{definition}[Pushout]
    Given morphisms \(\morph{f}{A}{B}\) and \(\morph{g}{A}{C}\), a
    \emph{pushout} is an object \(D\) and a pair of morphisms
    \(\morph{h}{B}{D}\) and \(\morph{k}{C}{D}\) such that for any other pair of
    morphisms \(\morph{h^\prime}{B}{Z}\) and \(\morph{k^\prime}{C}{Z}\) there
    exists a unique morphism \(\morph{u}{D}{Z}\), i.e.\ the following diagram
    commutes:
    \begin{center}
        \includestandalone{figures/category/diagrams/pushout}
    \end{center}
    A category \(\mcc\) is said to \emph{have pushouts} if a pushout exists for
    any pair of morphisms.
\end{definition}

A pushout square is normally indicated with a \raisebox{-0.25em}{\(\ulcorner\)}
symbol as shown above.

\begin{example}
    \(\set\) has pushouts: given morphisms \(\morph{f}{X}{Y}\) and
    \(\morph{g}{X}{Z}\), the pushout is the union of \(Y\) and \(Z\) identifying
    elements with a common preimage in \(X\).
    Concretely, let \({\sim} \subseteq Y \cup Z\) be a relation defined as
    \(\{(y, z) \,|\, \exists x \in X.\, y = f(x) \wedge z = g(x)\}\).
    Then the pushout set \(U\) is defined as \(X \cup Y / \sim\) and the
    morphisms \(\morph{h}{Y}{U}\) and \(\morph{k}{Z}{U}\) to the appropriate
    elements.
\end{example}

Using universal properties means we do not need to restrict ourselves to a
concrete category when using or proving results; we can instead say, for
example, that the result holds in all categories with products.
It is also often the case that if a category satisfies some universal
properties, then others follow for free!
We demonstrate this with a well-known result.

\begin{lemma}
    \label{lem:coproducts-pushout}
    If a category \(\mcc\) has pushouts and an initial object, then \(\mcc\)
    also has coproducts.
\end{lemma}
\begin{proof}
    Given objects \(A,B \in \mcc\), the coproduct \(A + B\) is constructed as
    follows:
    \begin{gather*}
        \includestandalone{figures/category/diagrams/coproduct-pushout}%
    \end{gather*}
    This is a coproduct due to the universal property of pushouts.
\end{proof}

As

\begin{definition}[Pullback]
    Given morphisms \(\morph{f}{A}{B}\) and \(\morph{g}{C}{B}\), a
    \emph{pullback} is an object \(D\) and a pair of morphisms
    \(\morph{h}{D}{A}\) and \(\morph{k}{h}{k}\) such that for any other pair of
    morphisms \(\morph{h^\prime}{B}{Z}\) and \(\morph{k^\prime}{C}{Z}\) there
    exists a unique morphism \(\morph{u}{D}{Z}\), i.e.\ the following diagram
    commutes:
    \begin{center}
        \includestandalone{figures/category/diagrams/pullback}
    \end{center}
    A category \(\mcc\) is said to \emph{have pullbacks} if a pushout exists for
    any pair of morphisms.
\end{definition}
\section{Functors}\label{sec:functors}

It is actually quite rare that we are only making use of one category at a time.
In order to compare categories, we need a notion of \emph{mapping between} them.
Such a map \(\morph{F}{\mcc}{\mcd}\) sends an object \(A \in \mcc\) to an
object \(FA \in \mcd\), and a morphism \(\morph{f}{A}{B} \in \mcc\) to
\(\morph{Ff}{FA}{FB}\).
This is generally not a strong enough definition to be useful, so some
additional restrictions must be added to define the notion of a \emph{functor}.

\begin{definition}[Functor]\index{functor}
    Given two categories \(\mcc\) and \(\mcd\), a \emph{functor} \(
    \morph{F}{\mcc}{\mcd}
    \) maps objects and morphisms in \(\mcc\) to objects and morphisms in
    \(\mcd\) such that
    \begin{itemize}
        \item \(F(\id[A]) = \id[FA]\) for every \(A \in \ob{\mcc}\); and
        \item \(F(g \circ f) = F(g) \circ F(f)\) for every \(\morph{f}{A}{B}\)
              and \(\morph{g}{B}{C}\).
    \end{itemize}

    \begin{center}
        \begin{tikzcd}[row sep=large]
            A
            \arrow{r}{\id[A]}
            \arrow{d}{F}
            &
            A
            \arrow{d}{F}
            \\
            FA
            \arrow{r}{\id[FA]}
            &
            FA
        \end{tikzcd}
        \qquad
        \begin{tikzcd}[column sep=large]
            A
            \arrow{rr}{g \circ f}
            \arrow{dr}{f}
            \arrow{ddd}{F}
            &
            &
            C
            \arrow{ddd}{F}
            \\
            &
            B
            \arrow{d}{F}
            \arrow{ur}{g}
            &
            \\
            &
            FB
            \arrow{dr}{Fg}
            &
            \\
            FA
            \arrow{rr}{F(g \circ f)}
            \arrow{ur}{Ff}
            &
            &
            FC
        \end{tikzcd}
    \end{center}
\end{definition}

The two equations above are known as the \emph{functoriality} equations; if a
map satisfies these it is said to be \emph{functorial}.

Functors have a graphical representation as `functorial
boxes'~\cite{mellies2006functorial}\index{functor!functorial box}\index{string diagram!functor}; the notation for applying a functor
\(\morph{F}{\mcc}{\mcd}\) to a morphism \(\morph{f}{X}{Y}\) is shown in
\cref{fig:functoriality-equations}, along with the depictions of the
functoriality equations.
As always, the wire labels are optional and will be omitted if unambiguous.

\begin{figure}
    \centering
    \iltikzfig{strings/category/functors/f}[box=f,functor=F,colour=white,dom=X,cod=Y]

    \begin{gather*}
        \iltikzfig{strings/category/functors/identity-lhs}[functor=F,object=X]
        =
        \iltikzfig{strings/category/functors/identity-rhs}[functor=F,object=X]
        \quad
        \iltikzfig{strings/category/functors/composition-lhs}[box1=f,box2=g,functor=F,colour=white,dom=X,cod=Z]
        =
        \iltikzfig{strings/category/functors/composition-rhs}[box1=f,box2=g,functor=F,colour=white,dom=X,cod=Z]
    \end{gather*}
    \caption{
        Diagrammatic depiction of `functorial boxes', and graphical
        equations of functoriality
    }
    \label{fig:functoriality-equations}
\end{figure}

\subsection{Examples of functors}

Many notions in mathematics and computer science can be viewed as functors.

\begin{definition}[Endofunctor]\index{functor!endofunctor}
    An \emph{endofunctor} on category \(\mcc\) is a functor \(\mcc \to \mcc\).
\end{definition}

\begin{example}[Powerset functor]\index{functor!powerset functor}
    The notion of powerset can be interpreted as an endofunctor \(
    \morph{\powerset}{\set}{\set}
    \), mapping a set \(X\) to its powerset \(\powerset(X)\) and a morphism
    \(\morph{f}{X}{Y}\) to the function \(\powerset(X) \to \powerset(Y)\) which
    applies \(f\) pointwise.
\end{example}

\begin{example}[List functor]\label{ex:list-functor}\index{functor!list functor}
    A functor that crops up frequently in computer science is the
    \emph{list functor} \(\morph{\listf}{\set}{\set}\), which sends a set
    \(X\) to its set of lists \(\freemon{X}\), and sends a function
    \(\morph{f}{X}{Y}\) to the function
    \(\morph{\freemon{f}}{\freemon{X}}{\freemon{Y}}\): which applies \(f\)
    to each element of a list.
\end{example}

\begin{example}[Free monoid]\label{ex:free-monoid}\index{free monoid}\index{functor!free monoid}
    The \emph{free construction} of a mathematical structure can be viewed as
    its most `bare-bones' version.
    Recall the definition of monoids from \cref{ex:monoid}: the set of words
    \(\freemon{X}\) as defined in \cref{not:words} is the carrier of the
    \emph{free monoid} on \(X\).
    This means there is a functor \(\morph{F}{\set}{\mon}\) (the
    \emph{free functor})\index{functor!free functor} that acts on objects as \(
    X \mapsto (
    \freemon{X},
    (\listvar{v}, \listvar{w}) \mapsto \listvar{v}\listvar{w},
    \varepsilon
    )\) and sends morphisms \(X \to Y\) to the corresponding monoid homomorphism
    \(\freemon{X} \to \freemon{Y}\).

    There is also a \emph{forgetful} or \emph{underlying} functor\index{functor!forgetful}
    \(\morph{U}{\mon}{\set}\) which sends a monoid \((X, *, e)\) to its carrier
    set \(X\) and `forgets' the monoid structure.
    The functors \(F\) and \(U\) form an \emph{adjunction}, but this
    is beyond the scope of this thesis.
\end{example}

\subsection{Functors as morphisms}

It is possible to view functors purely as a standalone concept, but the
core tenet of category theory is to view everything in terms of morphisms;
functors are no different.

\begin{example}[Identity functor]\index{functor!identity functor}\index{identity!functor}
    A trivial endofunctor for any category \(\mcc\) is the
    \emph{identity functor} \(\morph{\idf}{\mcc}{\mcc}\) which acts as the
    identity on objects and morphisms.
\end{example}

\begin{example}[Functor composition]\index{functor!functor composition}\index{composition!of functors}
    If \(\morph{F}{\mcc}{\mcd}\) and \(\morph{G}{\mcd}{\mce}\) are functors
    then their composite \(\morph{G \circ F}{\mcc}{\mce}\) is also a functor.
\end{example}

Identity and (associative) composition are all we need to define a category.
Although we cannot define a category containing \emph{all} categories, we can
define a category containing all `small' categories.

\begin{definition}\index{category!small category}
    A category \(\mcc\) is \emph{small} if \(\ob{\mcc}\) and
    \(\mor{\mcc}{A}{B}\) are sets for all \(A, B \in \mcc\).
\end{definition}

\begin{example}[Category of categories]\index{category!of categories}
    \(\cat\) is the category in which \(\ob{\cat}\) are small categories and
    \(\mor{\cat}{\mcc}{\mcd}\) contains functors \(\mcc \to \mcd\).
    Identity is the identity functor, and composition is functor composition.
\end{example}

Using the category of categories means we can reason about categories using the
same line of reasoning as for objects and morphisms.

\subsection{Full and faithful functors}

One important way we can class functors is based on how they act on the classes
of morphisms between two objects.

\begin{notation}
    Given a functor \(\morph{F}{\mcc}{\mcd}\), let \(
    \morph{F_{A,B}}{\mor{\mcc}{A}{B}}{\mor{\mcd}{FA}{FB}}
    \) be the induced map sending classes of morphisms \(A \to B\) in \(\mcc\)
    to the classes of morphisms \(FA \to FB\) in \(\mcd\).
\end{notation}

\begin{definition}[Faithful functor~\cite{maclane1978categories}]
    \index{functor!faithful}\index{faithful functor}
    A functor \(\morph{F}{\mcc}{\mcd}\) is \emph{faithful} if \(F_{A,B}\) is
    injective for all \(A,B \in \mcc\).
\end{definition}

A faithful functor \(\morph{F}{\mcc}{\mcd}\) does not coalesce morphisms: every
morphism \(f \in \mor{\mcc}{A}{B}\) has a unique morphism
\(Ff \in \mor{\mcd}{FA}{FB}\).

\begin{definition}[Full functor~\cite{maclane1978categories}]
    \index{functor!full}
    \index{full!functor}
    A functor \(\morph{F}{\mcc}{\mcd}\) is \emph{full} if \(F_{A,B}\) is
    surjective for all \(A,B \in \mcc\).
\end{definition}

Every morphism \(FA \to FB\) is in the image of a full functor.

\begin{definition}[Fully faithful functor~\cite{maclane1978categories}]\index{functor!fully faithful}
    A functor \(\morph{F}{\mcc}{\mcd}\) is \emph{fully faithful} if \(F_{A,B}\)
    is bijective for all \(A, B \in \mcc\); i.e.\ the functor is full and
    faithful.
\end{definition}

\begin{example}
    Recall the categories \(\set\) and \(\mon\), and the functor
    \(\morph{U}{\mon}{\set}\) defined in \cref{ex:free-monoid}.
    \(U\) is faithful as monoid homomorphisms are just functions, but it is not
    full as not all functions are monoid homomorphisms.
    Note that even though \(U\) is faithful, it is not injective on objects
    because there may be many monoids with the same carrier set.
\end{example}

Functors can be used to compare categories, and the notions of fullness and
faithfulness show how exactly two categories are related.
One way a category \(\mcc\) can be related to another category \(\mcd\) is that
the former is the latter with `some of the bits taken out'.

\begin{definition}[Subcategory]
    \index{subcategory}
    Given a category \(\mcc\), a \emph{subcategory} \(\mcd\) of \(\mcc\) is a
    subclass of the objects in \(\mcc\) and a subclass of the morphisms in
    \(\mcd\) such that
    \begin{itemize}
        \item for every object \(A \in \mcd\), the identity \(\id[A]\) is in
              \(\mcd\);
        \item for every morphism \(\morph{f}{A}{B} \in \mcd\), the source and
              targets \(A\) and \(B\) are in \(\mcd\); and
        \item for morphisms \(\morph{f}{A}{B}, \morph{g}{B}{C} \in \mcd\), the
              composition \(g \circ f\) is in \(\mcd\).
    \end{itemize}
\end{definition}

These conditions on a subcategory \(\mcd\) enforce that \(\mcd\) also has the
structure of a category.
This means there is an obvious induced functor \(\morph{S}{\mcd}{\mcc}\) mapping
objects and morphisms in \(\mcd\) to the same objects in \(\mcc\); this is
called an \emph{inclusion functor}\index{functor!inclusion}, and is
often written with a hooked arrow \(\mcd \hookrightarrow \mcc\)
\nomenclature{\(\mcc \hookrightarrow \mcc\)}{inclusion functor from \(\mcc\) to \(\mcd\)}.
An inclusion functor is clearly faithful, since there cannot be two morphisms in
the subcategory that map to the same morphism in the parent category.
Inclusion functors that are also \emph{full} are of particular interest.

\begin{definition}[Full subcategory]
    \index{full!subcategory}
    \index{subcategory!full}
    A subcategory \(\mcd\) is a \emph{full subcategory} if its inclusion functor
    \(\mcd \to \mcc\) is full and faithful; i.e.\ for all objects
    \(A, B \in \mcd\), the morphisms \(\mor{\mcd}{A}{B} = \mor{\mcc}{A}{B}\).
\end{definition}

\begin{example}
    \(\finset\) is a full subcategory of \(\set\), as every function between
    finite sets is a morphism in both \(\finset\) and \(\set\).
    \(\set\) is a subcategory of \(\rel\) as every function is a relation, but
    it is not a full subcategory because there are more relations \(A \sim B\)
    than there are functions \(A \to B\).
\end{example}

Sometimes a category is not merely a subcategory of another, but the two
categories are actually (almost) the same.

\begin{definition}\index{isomorphism!of categories}
    Two categories \(\mcc\) and \(\mcd\) are \emph{isomorphic} if there exist
    functors \(\morph{F}{\mcc}{\mcd}\) and \(\morph{G}{\mcd}{\mcc}\) such that
    \(G \circ F = \idf[\mcc]\) and \(F \circ G = \idf[\mcd]\).
\end{definition}

It can be inconvenient to construct the functors in both directions; fortunately
isomorphism can be shown by constructing just the functor in one direction, as
long as it has the required properties.

\begin{lemma}[\cite{maclane1978categories}]
    Two categories \(\mcc\) and \(\mcd\) are isomorphic if and only if there
    exists a fully faithful functor \(\morph{F}{\mcc}{\mcd}\) which is also
    bijective on objects.
\end{lemma}

\begin{remark}\index{equivalence!of categories}
    Usually, isomorphism of categories is too restrictive; often the weaker
    notion \emph{equivalence of categories} is used, in which the two functors
    \(F\) and \(G\) need only be \emph{naturally isomorphic} to the identity
    functor rather than equal.
    However, in this thesis it turns out that all the results we need really are
    strong enough to be isomorphisms.
\end{remark}


\subsection{Universal properties through functors}

Previously we defined some \emph{universal properties} such as initial
objects and pullbacks.
Using functors, these universal constructions can be viewed as special cases of
even more abstract notions: \emph{limits} and \emph{colimits}.

\begin{definition}[Cone]\index{cone}
    Let \(\mathcal{J}\) and \(\mathcal{C}\) be categories, and let
    \(\morph{F}{\mathcal{J}}{\mathcal{C}}\) be a functor.
    A \emph{cone to \(F\)} is an object \(N \in \mathcal{C}\) equipped with
    a family of morphisms \(\morph{\phi_X}{N}{FX}\) for each
    \(X \in \mathcal{J}\), such that for each
    \(\morph{f}{X}{Y} \in \mathcal{J}\),
    \(Ff \circ \phi_X = \phi_Y\).
    \begin{center}
        \includestandalone{figures/category/diagrams/cone}
    \end{center}
\end{definition}

The limit is the `best possible cone'.

\begin{definition}[Limit]\index{limit}
    Let \(\mathcal{J}\) and \(\mathcal{C}\) be categories, and let
    \(\morph{F}{\mathcal{J}}{\mathcal{C}}\) be a functor.
    The \emph{limit of \(F\)} is a cone to \(F\) \((L,\phi) \) such that for
    every cone to \(F\) \((N, \psi)\), there exists a unique morphism
    \(\morph{u}{N}{L}\) such that \(\phi_X \circ u = \psi_X\) for all
    \(X \in \mathcal{J}\).
    \begin{center}
        \includestandalone{figures/category/diagrams/limit}
    \end{center}
\end{definition}

Limits generalise several of the universal constructions we have encountered so
far.

\begin{example}[Terminal object]\label{ex:terminal-limit}
    \index{limit!terminal object}\index{terminal object!as a limit}
    When the category \(\mathcal{J}\) is the empty category, the only functor
    \(\morph{F}{\mathcal{J}}{\mathcal{C}}\) is the empty functor.
    Since there are no objects in \(\mathcal{J}\), a cone of \(F\) is just an
    object \(N\), so the limit of \(F\) is an object \(L\) to which there is a
    unique morphism \(N \to L\); the terminal object.

\end{example}

\begin{example}[Products]\label{ex:product-limit}
    \index{limit!product}\index{product!as a limit}
    When the category \(\mathcal{J}\) has objects but no morphisms other than
    identities, a functor \(\morph{F}{\mathcal{J}}{\mathcal{C}}\) is one that
    indexes objects of \(\mathcal{C}\) by objects in \(\mathcal{J}\).
    A cone of \(F\) is an object \(N\) with morphisms that pick each of these
    indexed elements, so the limit of \(F\) is the product.
\end{example}

\begin{example}[Pullbacks]\label{ex:pullback-limit}
    \index{limit!pullback}\index{pullback!as a limit}
    Let \(\mathcal{J}\) be a category containing objects \(A\), \(B\),
    \(C\) with non-identity morphisms \(B \to A\) and \(C \to A\).
    A cone of \(\morph{F}{\mathcal{J}}{\mathcal{C}}\) is an object \(N\) and
    morphisms \(N \to FA\), \(N \to FB\) and \(N \to FC\), so the limit of
    \(F\) is the pullback.
\end{example}

Since limits define so many categorical structures, the ability to define
arbitrary limits in a category makes it a much more appealing setting to work
in.

\begin{definition}
    For a small category \(\mathcal{J}\), a category \(\mathcal{C}\)
    \emph{has limits of shape \(\mathcal{J}\)} if every functor
    \(\morph{F}{\mathcal{J}}{\mathcal{C}}\) has a limit in \(\mathcal{C}\).
\end{definition}

\begin{definition}[Complete category]\index{category!complete}
    A category \(\mathcal{C}\) is \emph{complete} if it has limits for all
    functors \(\morph{F}{\mathcal{J}}{\mathcal{C}}\), where \(\mathcal{J}\) is a
    small category.
\end{definition}

Limits do not generalise all the universal constructions we have seen so far;
for the rest we must flip the arrows and consider the dual version.

\begin{definition}[Cocone]\index{cocone}
    Let \(\mathcal{J}\) and \(\mathcal{C}\) be categories, and let
    \(\morph{F}{\mathcal{J}}{\mathcal{C}}\) be a functor.
    A \emph{cocone to \(F\)} is an object \(N \in \mathcal{C}\) equipped with
    a family of morphisms \(\morph{\phi_X}{FX}{N}\) for each
    \(X \in \mathcal{J}\), such that for each
    \(\morph{f}{X}{Y} \in \mathcal{J}\),
    \(\phi_Y \circ Ff = \phi_X\).
    \begin{center}
        \includestandalone{figures/category/diagrams/cocone}
    \end{center}
\end{definition}

\begin{definition}[Colimit]\index{colimit}
    Let \(\mathcal{J}\) and \(\mathcal{C}\) be categories, and let
    \(\morph{F}{\mathcal{J}}{\mathcal{C}}\) be a functor.
    The \emph{colimit of \(F\)} is a cocone to \(F\) \((L,\phi)\) such that for
    every other cocone to \(F\) \((N, \psi)\), there exists a unique morphism
    \(\morph{u}{L}{N}\) such that \(u \circ \phi_X = \psi_X\) for all
    \(X \in \mathcal{J}\).
    \begin{center}
        \includestandalone{figures/category/diagrams/colimit}
    \end{center}
\end{definition}

\begin{example}
    \index{colimit!initial object}\index{colimit!coproduct}\index{colimit!pushout}
    \index{initial object!as a colimit}\index{coproduct!as a colimit}\index{pushout!as a colimit}
    By using the reasoning in
    \cref{ex:terminal-limit,ex:product-limit,ex:pullback-limit} and reversing
    the arrows, it is straightforward to see that initial objects, coproducts,
    and pushouts are examples of colimits.
\end{example}

Once again, we often want to work in a setting in which we can construct all
colimits.

\begin{definition}
    For a small category \(\mathcal{J}\), a category \(\mathcal{C}\)
    \emph{has colimits of shape \(\mathcal{J}\)} if every functor
    \(\morph{F}{\mathcal{J}}{\mathcal{C}}\) has a colimit in \(\mathcal{C}\).
\end{definition}

\begin{definition}[Cocomplete category]\index{category!cocomplete}
    A category \(\mathcal{C}\) is \emph{cocomplete} if it has colimits for all
    functors \(\morph{F}{\mathcal{J}}{\mathcal{C}}\), where \(\mathcal{J}\) is a
    small category.
\end{definition}

It is often the case that we are interested in functors that \emph{preserve}
structure, so we can exploit it in both categories.
Since a lot of structure can be expressed in terms of limits or colimits, this
can be expressed succinctly by saying that a functor merely preserves these
limits or colimits.

\begin{definition}\index{limit!preservation}\index{colimit!preservation}
    A functor \(\morph{F}{\mcc}{\mcd}\) \emph{preserves all (co)limits} if
    \((FL,F\phi)\) is a (co)limit whenever \((L,\phi)\) is a (co)limit.
\end{definition}

Sometimes a functor may not preserve \emph{all} (co)limits but only some of
them.
For example, we may talk about \(F\) being a coproduct-preserving functor .
This is defined in the same way as above: \(F(A+B)\) must itself be a coproduct
\(FA + FB\).
\section{Natural transformations}\label{sec:natural-transformations}

As we have seen, functors are useful for mapping objects and morphisms from one
category to another in a way that respects the underlying compositional
structure.
Of course, there may be many such functors \(\mcc \to \mcd\); the logical next
step is to consider maps between functors themselves.
These maps are known as \emph{natural transformations}.

\begin{definition}[Natural transformation]
    Given two functors \(\morph{F, G}{\mcc}{\mcd}\), a
    \emph{natural transformation} \(\morph{\eta}{F}{G}\) is a family of
    morphisms \(
    \eta_A \in \mor{\mcd}{FA}{GA}
    \) for each \(A \in \ob{\mcc}\), called the \emph{components} of \(\eta\),
    such that \(
    \eta_B \circ Ff = Gf \circ \eta_A
    \), i.e.\ the following diagram commutes:
    \begin{center}
        \includestandalone{figures/category/coherences/natural}
    \end{center}
\end{definition}

The key point to take is that for functors \(\morph{F,G}{\mcc}{\mcd}\), a
natural transformation \(\morph{\eta}{F}{G}\) is a \emph{family} of morphisms
\(\morph{\eta_A}{FA}{GA} \in \mcd\) (the components) for each object
\(A \in \mcd\).
Subsequently, one can think of a natural transformation as a way of inducing
morphisms of a certain structure across an \emph{entire category}.
Graphically, the naturality equation can be seen as how a natural transformation
can be `pushed through' morphisms, as shown in \cref{fig:naturality-equations}.

\begin{figure}
    \begin{gather*}
        \iltikzfig{strings/category/functors/naturality-lhs}[][f][F][G][\eta][A][B][white]
        =
        \iltikzfig{strings/category/functors/naturality-rhs}[][f][F][G][\eta][A][B][white]
    \end{gather*}
    \caption{
        Naturality of transformations in string diagram notation
    }
    \label{fig:naturality-equations}
\end{figure}

We established that functors are morphisms between categories; in a similar
vein natural transformations are morphisms between functors.

\begin{definition}[Functor category]
    Given two categories \(\mcc\) and \(\mcd\), a \emph{functor category}
    \([\mcc,\mcd]\) has as objects the functors \(\mcc \to \mcd\) and as
    morphisms \(F \to G\) the natural transformations between these functors.
\end{definition}

Like with the category of categories, viewing functors in this way allows us to
reason with them in the same way as regular objects and morphisms.

\subsection{Examples of natural transformations}

Natural transformations also often arise across mathematics and
computer science.

\begin{example}[Singleton transformation]
    Recall the \(\listf\) functor from \cref{ex:list-functor}.
    An example of a natural transformation is the
    \emph{singleton transformation} \(\morph{[-]}{\idf}{\listf}\), which
    induces a function \(\morph{[-]}{X}{\freemon{X}}\) for each set \(X\),
    defined as \(x \mapsto [x]\).
\end{example}

\begin{example}[Reduce]
    Recall the functors \(\morph{F}{\set}{\mon}\) and \(\morph{U}{\mon}{\set}\)
    from \cref{ex:free-monoid}.
    Functors can be composed just like morphisms, so \(F \circ U\) is a
    functor \(\mon \to \mon\): such a functor has action \(
    (X, *, e) \mapsto (\freemon{X}, \concat, [])
    \).
    The component of a natural transformation \(
    F \circ U \to \idf
    \) at object \((X, *, e)\) is a monoid homomorphism \(
    (\freemon{X}, \concat, []) \to (X, *, e)
    \), i.e.\ a function \(\freemon{X} \to X\).

    One example of such a natural transformation is the \emph{reduce} or
    \emph{fold} operation, which takes a list in \(\freemon{X}\) and reduces it
    to an element of \(X\) by starting with the unit \(e\) and multiplying it
    with each element of the list in turn.
\end{example}

As natural transformations are defined in terms of families of morphisms, they
can inherit properties of the components.

\begin{definition}[Natural isomorphism]
    A natural transformation is called a \emph{natural isomorphism} if every
    component is an isomorphism.
\end{definition}
\section{Monoidal categories}

The concepts of functors and natural transformations are used to interpret
the \emph{parallel} composition \(\tensor\).
To do this, a special kind of functor known as a \emph{bifunctor} is used.

\begin{definition}[Bifunctor]
    A \emph{bifunctor} is a functor with a product category as its domain, i.e.\
    a functor of the form \(\mcc \times \mcd \to \mce\).
\end{definition}

The notation for functor boxes can be extended in order to show how bifunctors
map from two categories into one.
\[
    \iltikzfig{strings/category/functors/bif}[box1=f,box2=g,dom1=A,dom2=C,cod1=B,cod2=D,colour=white,functor=F]
\]
This notation suggests that a bifunctor is exactly what we need to model
parallel composition.

\begin{definition}[Monoidal category]
    \label{def:monoidal-category}
    A \emph{monoidal category} is a category \(\mcc\) equipped with a
    bifunctor \(\morph{{-} \tensor {=}}{\mcc \times \mcc}{\mcc}\) called the
    \emph{tensor product} and an additional object \(I\) called the
    \emph{monoidal unit},
    along with natural isomorphisms
    \begin{itemize}
        \item \(
              \associator{A}{B}{C}
              \colon
              A \tensor (B \tensor C)
              \cong
              (A \tensor B) \tensor C
              \) called the \emph{associator};
        \item \(
              \leftunitor{A}
              \colon
              I \tensor A
              \cong
              A
              \) called the \emph{left unitor}; and
        \item \(
              \rightunitor{A}
              \colon
              A \tensor I
              \cong
              A
              \) called the \emph{right unitor}
    \end{itemize}
    such that the \emph{pentagon} and the \emph{triangle} diagrams below
    commute:
    \begin{center}
        \includestandalone{figures/category/coherences/pentagon}

        \vspace{1em}

        \includestandalone{figures/category/coherences/triangle}
    \end{center}
\end{definition}

\begin{example}
    \(\set\) is a monoidal category, with the tensor product defined as the
    Cartesian product (\(A \tensor B \coloneqq A \times B\)) and the unit as the
    singleton set (\(I \coloneqq \mathbbm{1}\)).
\end{example}

\begin{notation}
    We adopt the convention that \(f \tensor g \seq h \tensor k\) should be
    bracketed as \((f \tensor g) \seq (h \tensor k)\), i.e.\ \(\tensor\) binds
    more strongly than \(\seq\).
\end{notation}

We will use the \(\tensor\) bifunctor extensively through this thesis.
For this reason, when drawing string diagrams for monoidal categories we will
not forego drawing the (bi)functor boxes and usually draw wires exclusively in
their `deconstructed' state: instead of a single wire \(A \tensor B\) we will
draw two wires \(A\) and \(B\).
\begin{gather*}
    \iltikzfig{strings/monoidal/tensor-notation-compact}[colour=white,dom1=A,dom2=C,cod1=B,cod2=D,box1=f,box2=g]
    \coloneqq
    \iltikzfig{strings/monoidal/tensor-notation-boxes}[colour=white,dom1=A,dom2=C,cod1=B,cod2=D,box1=f,box2=g]
    \coloneqq
    \iltikzfig{strings/monoidal/tensor-notation-hybrid}[colour=white,dom1=A,dom2=C,cod1=B,cod2=D,box1=f,box2=g]
    \coloneqq
    \iltikzfig{strings/monoidal/tensor-notation-noboxes}[colour=white,dom1=A,dom2=C,cod1=B,cod2=D,box1=f,box2=g]
\end{gather*}
\begin{example}
    \(\set\) is a monoidal category, with the tensor product defined as the
    Cartesian product (\(A \tensor B \coloneqq A \times B\)) and the unit as the
    singleton set (\(I \coloneqq \mathbbm{1}\)).
\end{example}

The definition of monoidal category we have presented is quite general,
particularly with regards to the natural isomorphisms for unitors and
associators.
In our setting, it is normally sufficient for these isomorphisms to hold `on the
nose'.

\begin{definition}[Strict monoidal category]
    A monoidal category is \emph{strict} if \(\lambda\), \(\rho\) and \(\alpha\)
    are identities.
\end{definition}

In a strict monoidal category, the unitality and associativity of the tensor
hold as equations, as they do for regular composition in a category.
With this in mind, it can be instructive to view the coherences of a strict
monoidal category in terms of equations: these are illustrated in
\cref{fig:mc-equations}.

\begin{figure}
    \begin{center}
        \begin{tabular}{ccccccc}
            \(f \tensor \id[I]\) & \(=\) & \(f\)
                                 &       &
            \(\id[I] \tensor f\) & \(=\) & \(f\)
            \\[1em]
            \iltikzfig{strings/monoidal/unit-l-lhs}[box=f,colour=white,dom=A,cod=B]
                                 &
            \(=\)
                                 &
            \iltikzfig{strings/category/f}[box=f,dom=A,cod=B,colour=white]
                                 &       &
            \iltikzfig{strings/monoidal/unit-r-lhs}[box=f,colour=white,dom=A,cod=B]
                                 &
            \(=\)
                                 &
            \iltikzfig{strings/category/f}[box=f,dom=A,cod=B,colour=white]
        \end{tabular}
        \\[1em]
        \rule[1em]{\textwidth}{0.1mm}
        \\[0.1em]
        \begin{tabular}{ccccccc}
            \((f \tensor g) \tensor h\) & \(=\) & \(f \tensor (g \tensor h)\)
                                        &       &
            \(\id[A] \tensor \id[B]\)   & \(=\) & \(\id[A \tensor B]\)
            \\[1em]
            \iltikzfig{strings/monoidal/associativity-lhs}[box1=f,box2=g,box3=h,colour=white,dom1=A,cod1=B,dom2=C,cod2=D,dom3=E,cod3=F,colour=white]
                                        &
            \(=\)
                                        &
            \iltikzfig{strings/monoidal/associativity-rhs}[box1=f,box2=g,box3=h,colour=white,dom1=A,cod1=B,dom2=C,cod2=D,dom3=E,cod3=F,colour=white]
                                        &       &
            \iltikzfig{strings/monoidal/identity-tensor-lhs}[colour=white,obj1=A,obj2=B]
                                        &
            \(=\)
                                        &
            \iltikzfig{strings/category/identity}[colour=white,obj=A \tensor B]
        \end{tabular}
        \\[1em]
        \rule[1em]{\textwidth}{0.1mm}
        \\[0.1em]
        \begin{tabular}{ccc}
            \((f \tensor g) \seq (h \tensor k)\) & \(=\) & \((f \seq h) \tensor (g \seq k)\)
            \\[1em]
            \iltikzfig{strings/monoidal/interchange-lhs}[box1=f,box2=g,box3=h,box4=k,colour=white,dom1=A,cod1=C,dom2=D,cod2=F]
                                                 &
            \(=\)
                                                 &
            \iltikzfig{strings/monoidal/interchange-rhs}[box1=f,box2=g,box3=h,box4=k,colour=white,dom1=A,cod1=C,dom2=D,cod2=F]
        \end{tabular}
    \end{center}
    \caption{
        Equations of a strict monoidal category
    }
    \label{fig:mc-equations}
\end{figure}


\section{Symmetric monoidal categories}

We can now construct morphisms by composing them in sequence and in parallel.
However, there is no way to cross over the wires that connect boxes together.
This can be achieved by equipping the categorical setting with another natural
isormorphism.

\begin{definition}[Symmetric monoidal category]
    \label{def:symmetric-monoidal-category}
    A \emph{symmetric monoidal category} (SMC) is a monoidal category \(\mcc\)
    equipped with a natural isomorphism \(
    \swap{A}{B} \colon A \tensor B \cong B \tensor A
    \) such that the following diagrams commute:
    \begin{center}
        \includestandalone{figures/category/coherences/symmetry-unit}
        \includestandalone{figures/category/coherences/symmetry-inverse}

        \vspace{1em}

        \includestandalone[scale=0.95]{figures/category/coherences/hexagon}
    \end{center}
\end{definition}

As \(\sigma\) is a natural isomorphism, it induces
a \emph{family} of morphisms \(
\morph{\swap{A}{B}}{A \tensor B}{B \tensor A}
\) for every pair of objects \(A\) and \(B\).
In string diagrams, each such morphism \(\swap{A}{B}\) is drawn as \(
\iltikzfig{strings/symmetric/symmetry}[obj1=A, obj2=B, colour=white]
\): the swapping of wires \(A\) and \(B\).
As with identities, the use of this morphism is so common that we usually
drop the box around it.

Once again, we are primarily concerned with \emph{strict} symmetric
monoidal categories.
The equations of strict symmetric monoidal categories are listed in
\cref{fig:smc-equations}.

\begin{figure}
    \begin{center}
        \begin{tabular}{ccc}
            \(f \tensor g \seq \swap{B}{D} = \swap{A}{C} \seq g \tensor f\)
             &  &
            \(
            \swap{A}{B} \tensor \id[C] \seq \id[B] \tensor \swap{A}{C}
            =
            \swap{A}{B \tensor C}
            \)
            \\[1em]
            \(
            \iltikzfig{strings/symmetric/naturality-lhs}[box1=f,box2=g,colour=white,dom1=A,cod1=B,dom2=C,cod2=D]
            =
            \iltikzfig{strings/symmetric/naturality-rhs}[box1=f,box2=g,colour=white,dom1=A,cod1=B,dom2=C,cod2=D]
            \)
             &  &
            \(
            \iltikzfig{strings/symmetric/hexagon-lhs}[obj1=A,obj2=B,obj3=C,colour=white]
            =
            \iltikzfig{strings/symmetric/symmetry}[obj1=A,obj2=B \tensor C,colour=white]
            \)
        \end{tabular}
        \\[1em]
        \rule[1em]{\textwidth}{0.1mm}
        \\[0.1em]
        \begin{tabular}{ccccc}
            \(\swap{I}{A} = \id[A]\)
             &  &
            \(\swap{A}{I} = \id[A]\)
             &  &
            \(\swap{A}{B} \seq \swap{B}{A} = \id[A \tensor B]\)
            \\[1em]
            \(
            \iltikzfig{strings/symmetric/unit-l-lhs}[colour=white,obj=A,unit=I]
            =
            \iltikzfig{strings/category/identity}[obj=A,colour=white]
            \)
             &  &
            \(
            \iltikzfig{strings/symmetric/unit-r-lhs}[colour=white,obj=A,unit=I]
            =
            \iltikzfig{strings/category/identity}[obj=A,colour=white]
            \)
             &  &
            \(
            \iltikzfig{strings/symmetric/self-inverse-lhs}[colour=white,obj1=A,obj2=B]
            =
            \iltikzfig{strings/category/identity}[obj=A \tensor B,colour=white]
            \)
        \end{tabular}
    \end{center}
    \caption{
        Equations of a strict symmetric monoidal category
    }
    \label{fig:smc-equations}
\end{figure}

\begin{example}
    \(\set\) is a symmetric monoidal category, with \(
    \morph{\swap{A}{B}}{A \times B}{B \times A}
    \) defined as the function that swaps elements of a pair.
\end{example}
\subsection{PROPs}

Symmetric monoidal categories are an excellent setting for reasoning modulo
`structural equations'.
We are especially interested in a subclass of SMCs called \emph{PROP}s:
categories of \emph{PRO}ducts and \emph{P}ermutations.

\begin{definition}[PROP~\cite{maclane1965categorical}]\label{def:prop}
    \index{PROP}
    A \emph{PROP} is a strict symmetric monoidal category with the
    natural numbers as objects and addition as tensor product on objects.
\end{definition}

PROPs are a good fit for reasoning with string diagrams; as any
object \(n\) is equal to \(\bigotimes_{i < n} 1\), a morphism
\(m \to n\) is a box with \(m\) incoming wires and \(n\) outgoing wires.

\begin{definition}\label{def:freely-generated-prop}
    \index{PROP!freely generated}
    \nomenclature{\(\smcsigma\)}{PROP generated over set \(\generators\)}
    Given a set of generators \(\generators\), let \(\smcsigma\) be the
    PROP where \(\smcsigma(m, n)\) is the set of \(\Sigma\)-terms of type
    \(m \to n\) quotiented by the equations of SMCs.
\end{definition}

This is known as a category \emph{freely generated over} \(\Sigma\), in that all
of the morphisms in \(\smc{\generators}\) have been `generated' by combining
elements of \(\Sigma\) in various ways using composition and tensor.
Crucially, many \(\Sigma\)-terms correspond to the \emph{same} morphism in
\(\smc{\generators}\), as the latter are subject to the equations of SMCs.

When we discussed terms we also mentioned the notion of \emph{coloured} terms
where the wires can be of different colours; appropriately, there are also
coloured PROPs.

\begin{definition}[Coloured PROP]
    \index{PROP!coloured}
    Given a set of \emph{colours} \(C\), a \emph{\(C\)-coloured PROP} is a strict
    symmetric monoidal category with the objects as lists in \(\freemon{C}\) and
    tensor product as word concatenation.
\end{definition}

\begin{remark}
    A `regular' PROP as defined in \cref{def:prop} is isomorphic to a
    coloured PROP with only one colour.
\end{remark}

Note that this means the empty list \(\varepsilon\) is the unit
object in any \(\mcc\)-coloured PROP.

\begin{definition}\label{def:freely-generated-coloured-prop}
    \index{PROP!freely generated coloured}
    \nomenclature{\(\smcsigmac\)}{coloured PROP generated over coloured set \(\generators\)}
    Given a countable set of colours \(C\) and \(C\)-coloured generators
    \(\generators\), let \(\smcsigmac\) be the \(C\)-coloured PROP where
    \(\smcsigmac(\listvar{m}, \listvar{n})\) is the set of
    \((C,\Sigma)\)-terms of type \(\listvar{m} \to \listvar{n}\) quotiented by
    the equations of SMCs.
\end{definition}

\section{Reversing the wires}

When considering string diagrams for symmetric monoidal categories, there is a
strict notion of causality: it is not possible to create a cycle from the output
of a box to its input, and outputs may only be joined to outputs.
This enforces an implicit \emph{left-to-right} directionality across the page.

This may not always be desirable: one may wish to model a feedback
loop, or perhaps enforce some condition by unifying two outputs.
To do this sort of thing, we must examine symmetric monoidal categories with
some extra structure.

\subsection{Symmetric traced monoidal categories}

First we consider removing the acyclicity condition.

\begin{definition}[Symmetric traced monoidal category \cite{joyal1996traced}]\label{def:stmc}
    A \emph{symmetric traced monoidal category} (STMC) is a symmetric monoidal
    category \(\mcc\) equipped with a family of functions \(
    \morph{
        \trace{X}[A][B]{-}
    }{
        \mcc(X \tensor A, X \tensor B)
    }{
        \mcc(A, B)
    }
    \) for any three objects \(A\), \(B\) and \(X\) subject to the following
    equations:
    \begin{itemize}
        \item \textbf{Tightening (naturality in \(A,B\))}\\\null\qquad
              \(\trace{X}[A][D]{
                  \id[X] \tensor f \seq g \seq \id[X] \seq h
              }
              =
              f \seq \trace{X}[B][C]{g} \seq h
              \)
        \item \textbf{Sliding (naturality in \(X\))}\\\null\qquad
              \(
              \trace{X}[A][B]{f \seq g \tensor \id[B]}
              =
              \trace{Y}[A][B]{g \tensor \id[A] \seq f}
              \)
        \item \textbf{Yanking}\\\null\qquad
              \(
              \trace{X}[X][X]{\swap{X}{X}} = \id[X]
              \)
        \item \textbf{Vanishing}\\\null\qquad
              \(
              \trace{X}[A][B]{\trace{X}[Y \tensor A][Y \tensor B]{f}}
              =
              \trace{X \tensor Y}[A][B]{f}
              \)
        \item \textbf{Superposing}\\\null\qquad
              \(
              \trace{X}[A \tensor C][B \tensor F]{f \tensor g}
              = \trace{X}[A][B]{f} \tensor g
              \)
    \end{itemize}
\end{definition}

What this means is that in a STMC, if we have a morphism
\(\morph{f}{X \tensor A}{X \tensor B}\) we also have a morphism \(
\morph{\trace{X}[A][B]{f}}{A}{B}
\).

Traced categories were given the string diagrammatic treatment in
\cite{joyal1996traced}, in which the trace is depicted as a loop.
\begin{gather*}
    \trace{X}[A][B]{
        \iltikzfig{strings/category/f-2-2}[colour=white,box=f,dom1=X,dom2=A,cod1=X,cod2=B]
    }
    =
    \iltikzfig{strings/traced/trace-rhs}[colour=white,box=f,dom=A,cod=B,trace=X]
\end{gather*}
The equations of STMCs then have pleasant graphical interpretations, shown in
\cref{fig:stmc-equations}.
Note that as with the equations of SMCs, these equations amount to deforming the
diagram without altering connections between boxes, so do not need to be
applied explicitly when performing equational reasoning.

Usually we will omit the subscripts from \(\trace{X}[A][B]{f}\) and write it
simply as \(\trace{X}{f}\).

\begin{figure}
    \centering
    \begin{tabular}{ccc}
        \(\trace{X}[A][D]{
            \id[X] \tensor f \seq g \seq \id[X] \seq h
        }
        \)
         &
        \(=\)
         &
        \(f \seq \trace{X}[B][C]{g} \seq h\)
        \\[1em]
        \(
        \iltikzfig{strings/traced/naturality-lhs}[colour=white,box1=f,box2=g,box3=h,dom=A,cod=D]
        \)
         &
        \(=\)
         &
        \(
        \iltikzfig{strings/traced/naturality-rhs}[colour=white,box1=f,box2=g,box3=h,dom=A,cod=D]
        \)
    \end{tabular}
    \\[1em]
    \rule[1em]{\textwidth}{0.1mm}
    \\[0.1em]
    \begin{tabular}{ccccccc}
        \(
        \trace{X}[A][B]{f \seq g \tensor \id[B]}
        \)
         &
        \(=\)
         &
        \(
        \trace{Y}[A][B]{g \tensor \id[A] \seq f}
        \)
         &   &
        \(
        \trace{X}[X][X]{\swap{X}{X}}
        \)
         &
        \(=\)
         &
        \(\id[X]\)
        \\[1em]
        \(
        \iltikzfig{strings/traced/sliding-lhs}[colour=white,box1=f,box2=g,dom=A,cod=B]
        \)
         &
        \(=\)
         &
        \(\iltikzfig{strings/traced/sliding-rhs}[colour=white,box1=f,box2=g,dom=A,cod=B]\)
         &   &
        \(
        \iltikzfig{strings/traced/yanking-lhs}[obj=X,colour=white]
        \)
         &
        \(=\)
         &
        \(
        \iltikzfig{strings/category/identity}[obj=X,colour=white]
        \)
    \end{tabular}
    \\[1em]
    \rule[1em]{\textwidth}{0.1mm}
    \\[0.1em]
    \begin{tabular}{ccccccc}
        \(
        \trace{X}[A][B]{\trace{Y}[X \tensor A][X \tensor B]{f}}
        \)
         &
        \(=\)
         &
        \(
        \trace{X \tensor Y}[A][B]{f}
        \)
         &   &
        \(
        \trace{X}[A \tensor C][B \tensor D]{f \tensor g}
        \)
         &
        \(=\)
         &
        \( \trace{X}[A][B]{f} \tensor g\)
        \\[1em]
        \iltikzfig{strings/traced/vanishing-lhs}[colour=white,box=f,dom=A,cod=B,trace1=X,trace2=Y]
         &
        \(=\)
         &
        \iltikzfig{strings/traced/vanishing-rhs}[colour=white,box=f,dom=A,cod=B,trace1=X,trace2=Y]
         &   &
        \iltikzfig{strings/traced/superposing-lhs}[colour=white,box1=f,box2=g,dom1=A,cod1=B,dom2=C,cod2=D,trace=X]
         &
        \(=\)
         &
        \iltikzfig{strings/traced/superposing-rhs}[colour=white,box1=f,box2=g,dom1=A,cod1=B,dom2=C,cod2=D,trace=X]
    \end{tabular}
    \caption{Equations of STMCs in string diagram notation}
    \label{fig:stmc-equations}
\end{figure}

\begin{example}
    The classic example of a symmetric traced monoidal category is the category
    \(\finvectk{k}\) in which the objects are finite dimensional vector spaces
    over a field \(k\) and the morphisms are linear maps.
    The monoidal product is the usual tensor product of vector spaces and the
    trace by an operation known as the `partial trace'.
\end{example}

In a computer science contexts, traces are often used to model \emph{fixpoints}.

\subsection{Compact closed categories}

In a traced category, while wires can flow backwards across the page, regular
left-to-right flow must still be in effect at a wire's \emph{endpoints}.
This means that outputs will always be connected to inputs.
We will now consider another setting in which this is not the case.

\begin{definition}[Compact closed category]
    A \emph{compact closed category} (CCC) is a symmetric monoidal category in
    which every object \(X\) has a \emph{dual} \(\dual{X}\) equipped with
    morphisms called the \emph{unit} \(
    \morph{\ccunit[A]}{I}{\dual{A} \tensor A}
    \) and the \emph{counit} \(
    \morph{\cccounit[A]}{A \tensor \dual{A}}{I}
    \) satisfying the following \emph{snake equations}:
    \begin{center}
        \includestandalone{figures/category/coherences/snake-1}
        \quad
        \includestandalone{figures/category/coherences/snake-2}
    \end{center}
\end{definition}

In string diagrams, the dual is drawn as a wire flowing from right-to-left
instead of left-to-right; when labelling wires with objects we will drop the
notation for duals and recover the information solely from directionality of the
wires.
The unit and counit `bend' wires: the unit is drawn as \(
\iltikzfig{strings/compact-closed/cup-self-dual}[obj=A,colour=white]
\) and the counit as \(
\iltikzfig{strings/compact-closed/cap-self-dual}[obj=A,colour=white]
\).
As a result of this units and counits are often referred to as \emph{cups} and
\emph{caps} respectively.
The snake equations are depicted as in \cref{fig:ccc-equations}, which
sheds some light on their name!

\begin{figure}
    \centering
    \iltikzfig{strings/compact-closed/snake-1}
    \(=\)
    \iltikzfig{strings/category/identity}[colour=white]
    \quad
    \iltikzfig{strings/compact-closed/snake-2}
    \(=\)
    \iltikzfig{strings/category/identity}[colour=white]
    \caption{Equations of CCCs string diagram notation}
    \label{fig:ccc-equations}
\end{figure}

There are some cases where the actual directionality of wires is irrelevant;
we only care about the ability to bend wires.

\begin{definition}[Self-dual compact closed category]
    A compact closed category is \emph{self-dual} if for every object \(A\),
    \(\dual{A} \coloneqq A\).
\end{definition}

In a self-dual compact closed category, we do not need to label wires with
arrows.

\subsection{Traced vs compact closed}

The graphical notation is particularly suggestive of links between the
trace, the cup and the cap.
This is no coincidence, as there is a well-known result that allows one to
construct a trace in a compact closed setting.

\begin{proposition}[Canonical trace (\cite{joyal1996traced}, Prop. 3.1)]
    \label{prop:canonical-trace}
    Any compact closed category has a trace called the \emph{canonical trace},
    defined as \[
        \trace{X}{f}
        \coloneqq
        \ccunit[X] \tensor \id[A]
        \seq
        \id[\dual{X}] \tensor f
        \seq
        (\swap{\dual{X}}{X} \seq \cccounit[X]) \tensor \id[B]
    \]
    \[
        \iltikzfig{strings/compact-closed/canonical-trace}[colour=white,box=f,dom=A,cod=B,trace=X]
    \]
\end{proposition}

In this thesis we are primarily concerned with traced categories, but a plethora
of related work is based in the compact closed realm.
The canonical trace allows us to adapt existing results for our setting as well.

\begin{remark}
    It is also possible to consider the other direction: using the
    \emph{Int-construction}~\cite{joyal1996traced}, given any STMC
    \(\mcc\) one can construct a compact closed category \(\mathsf{Int}(\mcc)\).
    However, this will not be of relevance to us.
\end{remark}
\section{Monoidal theories}\label{sec:monoidal-theories}

So far we have only concerned ourselves with \emph{structural} equations:
equations that show how the same term can be constructed using different
combinations of composition, tensor, the identity and symmetry.
However, these only serve as a form of housekeeping: the true `computational
content' of processes comes from equations that show how the generators interact
with \emph{each other}.
These equations are provided by a \emph{monoidal theory}.

\begin{definition}\index{equation}
    An \emph{equation} is a pair of terms with the same domain and
    codomain.
\end{definition}

\begin{definition}[Monoidal theory]\index{monoidal theory}
    A \emph{monoidal theory} is a tuple \((\generators, \equations)\) where
    \(\generators\) is a set of generators and \(\equations\) is a set of
    equations.
\end{definition}

An equation \(f = g\) in a monoidal theory \emph{identifies} the two morphisms
\(f\) and \(g\), so that they are actually equal.
When reasoning with a monoidal theory, we therefore need to work in a category
subject to this identification of morphisms.

\begin{definition}[Quotient category]
    \index{quotient}\index{category!quotient}
    \nomenclature{\(\mcc / \mce\)}{category \(\mcc\) quotiented by set of equations \(\mce\)}
    Given a category \(\mcc\) and a set of equations \(\mce\) between
    morphisms in \(\mcc\) with the same source and target, the
    \emph{quotient category} \(\mcc / \mce\) is the category in which
    \(\ob{\mcc / \mce} \coloneqq \ob{\mcc}\) and \(
    \mor{(\mcc / \mce)}{X}{Y}
    \coloneqq
    \mor{\mcc}{X}{Y} / \mce
    \).
\end{definition}

In a quotient category \(\mcc / \mce\) the morphisms are the
\emph{equivalence classes} of morphisms in \(\mcc\) modulo the equations in
\(\mce\).

\begin{definition}
    \nomenclature{\(\smc{\generators, \equations}\)}{\(\smcsigma\) quotiented by set of equations \(\mce\)}
    Given a monoidal theory \((\generators, \equations)\), let
    \(\smc{\generators, \equations} \coloneqq \smc{\generators} / \equations\).
\end{definition}

When the set of equations \(\equations\) is empty,
\(\smc{\generators, \emptyset} = \smc{\generators}\), and we recover an ordinary
PROP.

We can also use the same procedure to define \emph{coloured} monoidal theories.

\begin{definition}[Coloured monoidal theory]\index{monoidal theory!coloured}
    A \emph{coloured monoidal theory} is a tuple
    \((C, \generators, \equations)\) where \(C\) is a countable set of colours,
    \(\generators\) is a set of \(C\)-coloured generators, and \(\equations\)
    is a set of equations.
\end{definition}

\begin{definition}
    \nomenclature{\(\smc{C, \generators, \equations}\)}{\(\smcsigmac\) quotiented by set of equations \(\mce\)}
    Given a coloured monoidal theory \((C, \generators, \equations)\), let
    \(\smc{C, \generators, \equations} \coloneqq \smc{C, \generators} / \equations\).
\end{definition}


\subsection{Case study: commutative monoids}

Monoidal theories for PROPs can be used to reason with many structures in
mathematics.
As we have seen, viewing terms in terms of diagrams rather than text strings is
far more intuitive, so we will often forgo writing the terms at all and reason
exclusively using diagrams.
This extends to defining generators, which allows us to give them suggestive
graphical representations rather than having to stick to a symbol with some
domain and codomain.

As an example, we will explore the monoidal theory of
\emph{commutative monoids}; using the graphical notation the intended behaviour
of the two generators is much clearer.

\index{commutative monoid}
\begin{definition}[Commutative monoids]\label{def:commutative-monoid}
    The monoidal theory of
    \emph{commutative monoids} is \(
    (\generators[\cmon], \equations[\cmon])
    \), where \(
    \generators[\cmon] \coloneqq \{
    \iltikzfig{strings/structure/monoid/merge}[colour=white],
    \iltikzfig{strings/structure/monoid/init}[colour=white]
    \}
    \) called the \emph{multiplication} and the \emph{unit} respectively,
    and \(\equations[\cmon]\) comprises the equations
    \begin{gather*}
        \iltikzfig{strings/structure/monoid/unitality-l-lhs}
        =
        \iltikzfig{strings/structure/monoid/unitality-l-rhs}
        \quad
        \text{(left unitality)}
        \qquad
        \iltikzfig{strings/structure/monoid/associativity-lhs}
        =
        \iltikzfig{strings/structure/monoid/associativity-rhs}
        \quad
        \text{(associativity)}
        \\[0.5em]
        \iltikzfig{strings/structure/monoid/commutativity-lhs}
        =
        \iltikzfig{strings/structure/monoid/commutativity-rhs}
        \quad
        \text{(commutativity)}
    \end{gather*}
    We write \(\cmon \coloneqq \smc{\generators[\cmon], \equations[\cmon]}\).
\end{definition}

The equations describe the properties of the multiplication: it is unital with
respect to the unit; it is associative; and it is commutative.
These equations could be described textually, but the string diagrams provide
intuitive visual interpretations; often it is insightful to reason
\emph{diagrammatically}.

\index{unitality}
\begin{example}[Right unitality]
    \(
    \iltikzfig{strings/structure/monoid/unitality-r-lhs}
    =
    \iltikzfig{strings/structure/monoid/unitality-r-rhs}
    \) is a valid equation in \(\cmon\).
\end{example}
\begin{proof}
    \(
    \iltikzfig{strings/structure/monoid/unitality-r-lhs}
    \eqaxioms[(\dagger)]
    \iltikzfig{strings/structure/monoid/right-unitality/step-1}
    =
    \iltikzfig{strings/structure/monoid/unitality-l-lhs}
    =
    \iltikzfig{strings/structure/monoid/unitality-l-rhs}
    \)
\end{proof}

Note that the first step \((\dagger)\) of the proof is performed solely by
deforming the string diagram; this is permitted so long as connectivity is
preserved.
Deforming the string diagram corresponds to implicitly applying equations
of symmetric monoidal categories.
These explicit steps are shown below, with the unit wire in grey:
%
\begin{gather*}
    \iltikzfig{strings/structure/monoid/right-unitality/lhs-unit}
    \eqaxioms[\text{unitality of } \seq]
    \iltikzfig{strings/structure/monoid/right-unitality/step-1-1}
    \eqaxioms[\text{self-inverse}]
    \iltikzfig{strings/structure/monoid/right-unitality/step-1-2}
    \\[0.5em]
    \eqaxioms[\text{naturality of } \sigma]
    \iltikzfig{strings/structure/monoid/right-unitality/step-1-3}
    \eqaxioms[\text{unitality of } \sigma]
    \iltikzfig{strings/structure/monoid/right-unitality/step-1-4}
    \eqaxioms[\text{unitality of } \seq]
    \iltikzfig{strings/structure/monoid/right-unitality/step-1-5}
\end{gather*}
%
Already much more verbose than the simple deformation, this does not
even take into account the repeated applications of associativity of both
composition and tensor required if reasoning in the term language.
If we write \(
\iltikzfig{strings/structure/monoid/merge}[colour=white]
\) as \(\mu\) and \(
\iltikzfig{strings/structure/monoid/init}[colour=white]
\) as \(\eta\), then the proof on terms becomes:
\begin{align*}
    \id[1] \tensor \eta \seq \mu
     & =
    \id[1] \tensor (\eta \seq \id[1]) \seq \mu
     &
    \text{unitality of } \seq
    \\
     & =
    (\id[1] \seq \id[1]) \tensor (\eta \seq \id[1]) \seq \mu
     &
    \text{unitality of } \seq
    \\
     & =
    ((\id[1] \tensor \eta) \seq (\id[1] \tensor \id[1])) \seq \mu
     &
    \text{functoriality of } \tensor
    \\
     & =
    ((\id[1] \tensor \eta) \seq (\swap{n}{n} \seq \swap{n}{n})) \seq \mu
     &
    \sigma \text{ is self inverse}
    \\
     & =
    (((\id[1] \tensor \eta) \seq \swap{n}{n}) \seq \swap{n}{n}) \seq \mu
     &
    \text{associativity of } \seq
    \\
     & =
    ((\swap{\varepsilon}{n} \seq (\eta \tensor \id[1])) \seq \swap{n}{n}) \seq \mu
     &
    \text{naturality of } \sigma
    \\
     & =
    ((\id[1] \seq (\eta \tensor \id[1])) \seq \swap{n}{n}) \seq \mu
     &
    \text{unitality of } \sigma
    \\
     & =
    ((\eta \tensor \id[1]) \seq \swap{n}{n}) \seq \mu
     &
    \text{unitality of } \seq
    \\
     & =
    (\eta \tensor \id[1]) \seq (\swap{n}{n} \seq \mu)
     &
    \text{associativity of} \seq
    \\
     & =
    (\eta \tensor \id[1]) \seq \mu
     &
    \text{commutativity of } \mu
    \\
     & =
    \id[1]
     &
    \text{left unitality of } \mu
\end{align*}

This is already far more verbose than the string diagram proof, but the term
notation also blocks the insight required to make a proof step, which is often
much easier to see in the string diagrammatic representation.