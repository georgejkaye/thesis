\section{Categories}

Terms are purely syntax, so two terms are only equal if they are constructed in
precisely the same way.
As we have already seen, this is far too strong a relation; there may be
many terms that, while constructed in different ways, look the same when drawn
out as a diagram modulo the tiling.
To identify the terms that actually describe the same process, we need to impose
some equations on terms.
There is an obvious question: what are these equations?

\begin{example}
    Consider the term \(\id[m] \seq f\), which can be read as `do nothing and
    then run \(f\)`, and is drawn as \(
    \iltikzfig{strings/category/identity-l-lhs-tiles}[dom=m,cod=n]
    \).
    Clearly this is the same \(
    \iltikzfig{strings/category/f-tile}[dom=m,cod=n]
    \) but with an elongated input wire.
    So one equation we need is \(
    \iltikzfig{strings/category/identity-l-lhs-tiles}[dom=m,cod=n]
    =
    \iltikzfig{strings/category/f-tile}[dom=m,cod=n]
    \).
\end{example}

Rather than having to conjure up equations in this adh
It turns out that all the work has actually already been done before us, as the
required equations are the equations of \emph{symmetric monoidal categories}.
To grasp how these equations are derived one requires quite a
bit of technical knowledge, so we will build it up one step at a time.
We start with the basic notion of a \emph{category}.

\begin{definition}[Category]
    \label{def:category}
    A \emph{category} \(\mcc\) consists of a class of \emph{objects}
    \(\ob{\mcc}\); a class of \emph{morphisms} \(\mor{\mcc}{A}{B}\)
    for every pair of objects \(A, B \in \ob{\mcc}\); and a \emph{composition}
    operation \(
    \morph{
        {-} \circ {=}
    }{
        \mor{\mcc}{B}{C} \times \mor{\mcc}{A}{B}
    }{
        \mor{\mcc}{A}{C}
    }
    \) such that
    \begin{itemize}
        \item composition is \emph{unital}: for every \(
              A \in \ob{\mcc}
              \) there exists an \emph{identity morphism} \(
              \id[A] \in \mor{\mcc}{A}{A}
              \) satisfying \(
              f \circ \id[A] = f = \id[B] \circ f
              \) for any \(
              f \in \mor{\mcc}{A}{B}
              \); and
        \item composition is \emph{associative}: for any morphisms \(
              f \in \mor{\mcc}{A}{B}
              \), \(
              g \in \mor{\mcc}{B}{C}
              \) and \(h \in \mor{\mcc}{C}{D}\), \(
              (h \circ g) \circ f = h \circ (g \circ f).
              \)
    \end{itemize}
\end{definition}

A morphism \(f \in \mor{\mcc}{A}{B}\) is also called an \emph{arrow}, and will
often be written \(\morph{f}{A}{B}\) accordingly.
When clear from context, we will use the notation \(A \in \mcc\) or
\(f \in \mcc\) for objects and morphisms belonging to a particular category.
The subscripts on the object and morphism classes are also often omitted.

\begin{remark}
    The definition of a category uses \emph{classes} rather than sets as one
    might expect; this is due to size issues and Russell's paradox regarding
    the impossibility of the `set of all sets'.
\end{remark}

By treating terms as morphisms, it is clear to see how they they slip into the
categorical setting.
One difference is that in the previous section we used `left-to-right'
composition \(\seq\) rather than `right-to-left' composition \(\circ\).

\begin{notation}
    \emph{Diagrammatic order} composition is written as
    \(f \seq g \coloneqq g \circ f\).
\end{notation}

\begin{figure}
    \begin{center}
        \begin{tabular}{ccccc}
            \(
            \id[A] \seq f = f
            \)
             &  &
            \(f \seq \id[B] = f\)
             &  &
            \((f \seq g) \seq h = f \seq (g \seq h)\)
            \\[1em]
            \(
            \iltikzfig{strings/category/identity-l-lhs}[box=f,colour=white,dom=A,cod=B]
            =
            \iltikzfig{strings/category/f}[box=f,dom=A,cod=B,colour=white]
            \)
             &  &
            \(
            \iltikzfig{strings/category/identity-r-lhs}[box=f,dom=A,cod=B,colour=white]
            =
            \iltikzfig{strings/category/f}[box=f,dom=A,cod=B,colour=white]
            \)
             &  &
            \(
            \iltikzfig{strings/category/associativity-lhs}[box1=f,box2=g,box3=h,colour=white,dom=A,cod=C]
            =
            \iltikzfig{strings/category/associativity-rhs}[box1=f,box2=g,box3=h,colour=white,dom=A,cod=C]
            \)
        \end{tabular}
    \end{center}
    \caption{
        Equations of a category
    }
    \label{fig:c-equations}
\end{figure}

The equations of categories are illustrated with string diagram notation
in~\cref{fig:c-equations}.

\subsection{Commutative diagrams}

Equations in category theory can be expressed using \emph{commutative diagrams}.
For example, the unitality and associativity of composition can be illustrated
as follows:

\begin{center}
    \includestandalone{figures/category/coherences/unitality}
    \quad
    \includestandalone{figures/category/coherences/associativity}
\end{center}

We say that the above two diagrams \emph{commute} precisely because \(
\id[B] \circ f = f = f \circ \id[A]
\) and \((h \circ g) \circ f = h \circ (g \circ f)\): no matter which path one
takes, the results are equal.

\subsection{Examples of categories}

The definition of a category is quite abstract and might take some getting used
to: it can be helpful to consider some concrete examples.

\begin{example}[Preorder]
    A \emph{preorder} is a binary relation \(\leq\) on a set \(X\) which is
    reflexive and transitive.
    Any preorder generates a category \(\mcc_\leq\): the objects are the
    elements of \(X\) and \(\mcc_\leq(x, y)\) contains exactly one morphism if
    \(x \leq y\) and none otherwise.
\end{example}

\begin{example}[Sets]
    The category \(\set\) has sets as objects and functions as morphisms.
    Composition is then just function composition.
    Compare this with the category \(\rel\), which has the same objects as
    \(\set\) but with \emph{relations} as morphisms.
    There is also a category \(\finset\) containing only the \emph{finite} sets
    and functions between them.
\end{example}

\begin{example}[Posets]
    A \emph{partial order} on a set \(A\) is a reflexive, antisymmetric and
    transitive relation \({\leq} \subseteq A \times B\).
    A set equipped with a partial order is called a
    \emph{partially ordered set}, or \emph{poset} for short.
    For posets \((A,{\leq_A})\) and \((B,{\leq_B})\), a function
    \(\morph{f}{A}{B}\) is called \emph{monotone} if \(a \leq_A a^\prime\)
    implies that \(f(a) \leq_B f(a^\prime)\).

    Much like how sets form a category, posets form the category \(\pos\), where
    \(\ob{\pos}\) are posets and \(\mor{\pos}{X}{Y}\) are the monotone functions
    \(X \to Y\).
\end{example}

\begin{example}[Monoids]
    A \emph{monoid} is a tuple \((A, *, \bullet)\) where \(A\) is a set called
    the \emph{carrier}, \(\morph{*}{A \times A}{A}\) is a binary operation
    called the \emph{multiplication}, and \(\bullet \in A\) is an element called
    the \emph{unit}, such that \(a * \bullet = a = \bullet * a\) for any
    \(a \in A\).
    A \emph{monoid homomorphism} between two monoids \((A, *, e_A)\) and
    \((B, +, e_B)\) is a map \(\morph{h}{A}{B}\) such that
    \(h(a * a^\prime) = h(a) + h(a^\prime)\) and \(h(e_A) = e_B\).
    There is a category \(\mon\) with monoids as the objects as monoids and
    monoid homomorphisms as the morphisms.
\end{example}

\begin{example}[Product category]
    Given two categories \(\mcc\) and \(\mcd\), their \emph{product category}
    \(\mcc \times \mcd\) is the category with objects are defined as \(
    \ob{\mcc \times \mcd} \coloneqq \ob{\mcc} \times \ob{\mcd}
    \) and the morphisms as \[
        \mor{(\mcc \times \mcd)}{(A, A^\prime)}{(B, B^\prime)}
        \coloneqq
        \{
        (f, f^\prime)
        \,|\,
        f \in \mor{\mcc}{A}{B},
        f^\prime \in \mor{\mcd}{A^\prime}{B^\prime}
        \}
    \]
\end{example}

\subsection{Universal properties}

One of the reasons that category theory is an appealing foundation is because
of how it can be used to \emph{abstract} away from concrete constructions.
Rather than proving results about some particular objects and morphisms, we can
show how they are an instantiation of some more abstract concept.
This means that the important theorems only need to be shown for the abstract
case and then reused in a variety of applications.

One simple way we can abstract away from concrete definitions is by considering
the properties of a particular morphism.

\begin{definition}[Monomorphism]
    A morphism \(\morph{f}{A}{B} \in \mcc\) is called a \emph{monomorphism} (or
    simply \emph{mono} for short) if for any two morphisms
    \(\morph{g_1,g_2}{C}{A}\), if \(f \circ g_1 = f \circ g_2\), then
    \(g_1 = g_2\).
    \begin{center}
        \begin{tikzcd}
            C
            \arrow[bend left]{r}{g_1}
            \arrow[bend right, swap]{r}{g_2}
            &
            A
            \arrow{r}{f}
            &
            B
        \end{tikzcd}
    \end{center}
\end{definition}

One can think of monomorphisms as morphisms which are \emph{left-cancellative}.
There is also a way to describe \emph{invertible} morphisms.

\begin{definition}[Isomorphism]
    A morphism \(\morph{f}{A}{B} \in \mcc\) is called an \emph{isomorphism} (or
    simply \emph{iso} for short) if there also exists a morphism \(
    \morph{\inverse{f}}{B}{A} \in \mcc
    \) such that \(
    \inverse{f} \circ f = \id[A]
    \) and \(
    f \circ \inverse{f} = \id[B]
    \).
    \begin{center}
        \begin{tikzcd}
            A
            \arrow{r}{f}
            \arrow[swap, bend right]{rr}{\id[A]}
            &
            B
            \arrow{r}{\inverse{f}}
            &
            A
            &
            &
            B
            \arrow{r}{\inverse{f}}
            \arrow[swap, bend right]{rr}{\id[B]}
            &
            A
            \arrow{r}{f}
            &
            B
        \end{tikzcd}
    \end{center}
\end{definition}

\begin{example}
    In \(\set\), the monomorphisms are the injective functions and the
    isomorphisms are the bijective functions.
\end{example}

Often we want to say that a \emph{category} has some property or admits
some class of constructions.
This is specified in terms of a \emph{universal property}: a unique
morphism that shows that a certain candidate construction is the `best' way to
describe something.

The first universal properties we will consider give particular objects in a
category special significance.

\begin{definition}[Initial object]
    An object \(C\) in a category \(\mcc\) is \emph{initial} if, for any other
    object \(X \in \mcc\) there exists a unique morphism \(C \to X\).
\end{definition}

\begin{example}
    In \(\set\), the initial object is the empty set \(\emptyset\), as there is
    a unique function from \(\emptyset\) to any set \(X\), the so-called
    `absurd function'.
\end{example}

Most notions in category also have a \emph{dual}, in which all the
arrows are flipped.

\begin{definition}[Terminal object]
    An object \(C\) in a category \(\mcc\) is \emph{terminal} if, for any other
    object \(X \in \mcd\) there exists a unique morphism \(X \to C\).
\end{definition}

\begin{example}
    In \(\set\), the terminal object is the singleton set \(\{\star\}\):
    from any set \(X\) there is a unique function \(X \to \{\star\}\); namely
    the function with action \(x \mapsto \star\).
\end{example}

If a category has an initial or terminal object, then it is
\emph{unique up to unique isomorphism}: if we have two objects that
satisfies the universal property, then these objects are isomorphic in a unique
way.

The next properties we will see illustrate how a notion of objects being `built
up' from other objects can be expressed in terms of the morphisms between them.

\begin{definition}[Product]
    Given a category \(\mcc\) and objects \(A,B \in \mcc\), their \emph{product}
    is an object \(A \times B\) equipped with a pair of morphisms
    \(\morph{p_0}{A \times B}{A}\) and \(\morph{p_1}{A \times B}{B}\) called
    \emph{projections} such that for every other object \(Z\) with pair of
    morphisms \(\morph{f}{Z}{A}\) and \(\morph{g}{Z}{B}\), there exists a unique
    morphism \(\morph{u}{Z}{A \times B}\) such that the following diagram
    commutes:
    \begin{center}
        \includestandalone{figures/category/diagrams/product}
    \end{center}
    A category \(\mcc\) is said to \emph{have products} if the product exists
    for all objects \(A,B \in \mcc\).
\end{definition}

\begin{example}
    The product is \(\set\) is the Cartesian product.
\end{example}

The dual of a product is a construction with \emph{in}jections rather than
\emph{pro}jections.

\begin{definition}[Coproduct]
    Given a category \(\mcc\) and objects \(A,B \in \mcc\), their \emph{coproduct}
    is an object \(A + B\) equipped with a pair of morphisms
    \(\morph{i_0}{A}{A + B}\) and \(\morph{i_1}{B}{A + B}\) called
    \emph{injections} such that for every other object \(Z\) with pair of
    morphisms \(\morph{f}{A}{B}\) and \(\morph{g}{B}{Z}\), there exists a unique
    morphism \(\morph{u}{A + B}{Z}\) such that the following diagram
    commutes:
    \begin{center}
        \includestandalone{figures/category/diagrams/coproduct}
    \end{center}
    A category \(\mcc\) is said to \emph{have coproducts} if the coproduct
    exists for all pairs of objects \(A,B \in \mcc\).
\end{definition}

\begin{example}
    The coproduct in \(\set\) is the disjoint union.
\end{example}

Finally, we will look at properties showing how pairs of morphisms with a
common domain or codomain can be in some way `unified'.

\begin{definition}[Pushout]
    Given morphisms \(\morph{f}{A}{B}\) and \(\morph{g}{A}{C}\), a
    \emph{pushout} is an object \(D\) and a pair of morphisms
    \(\morph{h}{B}{D}\) and \(\morph{k}{C}{D}\) such that for any other pair of
    morphisms \(\morph{h^\prime}{B}{Z}\) and \(\morph{k^\prime}{C}{Z}\) there
    exists a unique morphism \(\morph{u}{D}{Z}\), i.e.\ the following diagram
    commutes:
    \begin{center}
        \includestandalone{figures/category/diagrams/pushout}
    \end{center}
    A category \(\mcc\) is said to \emph{have pushouts} if a pushout exists for
    any pair of morphisms.
\end{definition}

\begin{example}
    \(\set\) has pushouts: given morphisms \(\morph{f}{A}{B}\) and
    \(\morph{g}{A}{C}\), the pushout is the union of \(B\) and \(C\) identifying
    elements with a common preimage in \(X\).
    Concretely, let \({\sim} \subseteq B \cup C\) be a relation defined as
    \(\{(b, c) \,|\, \exists a \in A.\, b = f(a) \wedge c = g(a)\}\).
    Then the pushout set \(D\) is defined as \(B \cup C / \sim\) with the
    morphisms \(\morph{h}{B}{D}\) and \(\morph{k}{C}{D}\) sending elements in
    \(B\) and \(C\) to the appropriate element in \(D\).
\end{example}

A pushout square is normally indicated with a \raisebox{-0.25em}{\(\ulcorner\)}
symbol as shown above.
The dual of the pushout is a \emph{pullback}.

\begin{definition}[Pullback]
    Given morphisms \(\morph{f}{B}{A}\) and \(\morph{g}{C}{A}\), a
    \emph{pullback} is an object \(D\) and a pair of morphisms
    \(\morph{h}{D}{B}\) and \(\morph{k}{D}{C}\) such that for any other pair of
    morphisms \(\morph{h^\prime}{Z}{B}\) and \(\morph{k^\prime}{Z}{C}\) there
    exists a unique morphism \(\morph{u}{Z}{D}\), i.e.\ the following diagram
    commutes:
    \begin{center}
        \includestandalone{figures/category/diagrams/pullback}
    \end{center}
    A category \(\mcc\) is said to \emph{have pullbacks} if a pullback exists
    for any pair of morphisms.
\end{definition}

\begin{example}
    \(\set\) also has pullbacks: given morphisms \(\morph{f}{B}{A}\) and
    \(\morph{g}{C}{A}\), the pullback is defined as
    \(\{(b,c) \in B \times C \, f(b) = g(c)\}\) and the maps
    \(\morph{h}{D}{B}\) and \(\morph{k}{D}{C}\) defined as the first and second
    projection respectively.
\end{example}

Using universal properties means we do not need to restrict ourselves to a
concrete category when using or proving results; we can instead say, for
example, that the result holds in all categories with products.
It is also often the case that if a category satisfies some universal
properties, then others follow for free.
We demonstrate this with a well-known result.

\begin{lemma}
    \label{lem:coproducts-pushout}
    If a category \(\mcc\) has pushouts and an initial object, then \(\mcc\)
    also has coproducts.
\end{lemma}
\begin{proof}
    Given objects \(A,B \in \mcc\), the coproduct \(A + B\) is constructed as
    follows:
    \begin{gather*}
        \includestandalone{figures/category/diagrams/coproduct-pushout}%
    \end{gather*}
    This is a coproduct due to the universal property of pushouts.
\end{proof}