\section{Categories}\label{sec:categories}

Terms are purely syntax, so two terms are only equal if they are constructed in
precisely the same way.
As we have already seen, this is far too strong a relation; there may be
many terms that, while constructed in different ways, look the same when drawn
out as a diagram modulo the tiling.
We can identify terms that describe the same process using \emph{equations}, but
which ones?

\begin{example}
    Consider the term \(\id[m] \seq f\), read as `do nothing and
    then run \(f\)`, and drawn as \(
    \iltikzfig{strings/category/identity-l-lhs-tiles}[dom=m,cod=n]
    \).
    Clearly this is the same as \(
    \iltikzfig{strings/category/f-tile}[dom=m,cod=n]
    \) but with an elongated input wire.
    So one equation we need is \(
    \iltikzfig{strings/category/identity-l-lhs-tiles}[dom=m,cod=n]
    =
    \iltikzfig{strings/category/f-tile}[dom=m,cod=n]
    \).
\end{example}

It turns out that the required equations are the equations of
\emph{symmetric monoidal categories}.
To grasp how these equations are derived one requires quite a
bit of technical knowledge, so we will build it up one step at a time.
We start with a \emph{category}.

\begin{definition}[Category]
    \label{def:category}
    A \emph{category} \(\mcc\) consists of a class of \emph{objects}
    \(\ob{\mcc}\); a class of \emph{morphisms} \(\mor{\mcc}{A}{B}\)
    for every pair of objects \(A, B \in \ob{\mcc}\); and a \emph{composition}
    operation \(
    \morph{
        {-} \circ {-}
    }{
        \mor{\mcc}{B}{C} \times \mor{\mcc}{A}{B}
    }{
        \mor{\mcc}{A}{C}
    }
    \) such that
    \begin{itemize}
        \item
              for any object \(A \in \ob{\mcc}\) there exists a unique
              \emph{identity} morphism \(\id[A]\);
        \item
              for any \(f \in \mor{\mcc}{A}{B}\), it holds that \(
              f \circ \id[A] = f = \id[B] \circ f
              \); and
        \item for any morphisms \(
              f \in \mor{\mcc}{A}{B}
              \), \(
              g \in \mor{\mcc}{B}{C}
              \) and \(h \in \mor{\mcc}{C}{D}\), it holds that \(
              (h \circ g) \circ f = h \circ (g \circ f).
              \)
    \end{itemize}
\end{definition}

A morphism \(f \in \mor{\mcc}{A}{B}\) is also called an \emph{arrow}, and will
often be written \(\morph{f}{A}{B}\) accordingly.
When clear from context, we will use the notation \(A \in \mcc\) or
\(f \in \mcc\) for objects and morphisms belonging to a particular category.

\begin{remark}
    The definition of a category uses \emph{classes} rather than sets as one
    might expect; this is due to size issues and Russell's paradox regarding
    the impossibility of the `set of all sets'.
\end{remark}

We interpret terms as morphisms; one difference here is that we previously used
`left-to-right' composition \(\seq\) rather than `right-to-left' composition
\(\circ\).

\begin{notation}
    \emph{Diagrammatic order} composition is written as
    \(f \seq g \coloneqq g \circ f\).
\end{notation}

\begin{figure}
    \begin{center}
        \begin{tabular}{ccccc}
            \(
            \id[A] \seq f = f
            \)
             &  &
            \(f \seq \id[B] = f\)
             &  &
            \((f \seq g) \seq h = f \seq (g \seq h)\)
            \\[1em]
            \(
            \iltikzfig{strings/category/identity-l-lhs}[box=f,colour=white,dom=A,cod=B]
            =
            \iltikzfig{strings/category/f}[box=f,dom=A,cod=B,colour=white]
            \)
             &  &
            \(
            \iltikzfig{strings/category/identity-r-lhs}[box=f,dom=A,cod=B,colour=white]
            =
            \iltikzfig{strings/category/f}[box=f,dom=A,cod=B,colour=white]
            \)
             &  &
            \(
            \iltikzfig{strings/category/associativity-lhs}[box1=f,box2=g,box3=h,colour=white,dom=A,cod=C]
            =
            \iltikzfig{strings/category/associativity-rhs}[box1=f,box2=g,box3=h,colour=white,dom=A,cod=C]
            \)
        \end{tabular}
    \end{center}
    \caption{
        Equations of a category
    }
    \label{fig:c-equations}
\end{figure}

The equations of categories are illustrated with string diagram notation
in~\cref{fig:c-equations}.

\subsection{Commutative diagrams}

Equations in category theory can be expressed using \emph{commutative diagrams}.
For example, the unitality and associativity of composition can be illustrated
as follows:

\begin{center}
    \includestandalone{figures/category/coherences/unitality}
    \quad
    \includestandalone{figures/category/coherences/associativity}
\end{center}

We say that the above two diagrams \emph{commute} precisely because \(
\id[B] \circ f = f = f \circ \id[A]
\) and \((h \circ g) \circ f = h \circ (g \circ f)\): no matter which path one
takes, the results are equal.

\subsection{Examples of categories}

The definition of a category is quite abstract and might take some getting used
to: it can be helpful to consider some concrete examples.

\begin{example}[Preorder]
    A \emph{preorder} is a reflexive, transitive binary relation \(\leq\) on a
    set \(X\).
    Any preorder generates a category \(\mcc_\leq\): the objects are the
    elements of \(X\) and \(\mcc_\leq(x, y)\) contains exactly one morphism if
    \(x \leq y\) and none otherwise.
\end{example}

\begin{example}[Sets]
    The category \(\set\) has sets as objects and functions as morphisms, with
    composition of functions as composition.
    There are other categories with sets as objects, such as \(\rel\), which has
    \emph{relations} as morphisms and composition of relations as morphisms.
    There is also a category \(\finset\) containing the \emph{finite} sets
    and functions between them.
\end{example}

\begin{example}[Posets]
    A \emph{partial order} on a set \(A\) is a reflexive, antisymmetric and
    transitive relation \({\leq} \subseteq A \times B\).
    A set equipped with a partial order is called a
    \emph{partially ordered set}, or \emph{poset} for short.
    For posets \((A,{\leq_A})\) and \((B,{\leq_B})\), a function
    \(\morph{f}{A}{B}\) is called \emph{monotone} if \(a \leq_A a^\prime\)
    implies that \(f(a) \leq_B f(a^\prime)\).

    Much like how sets form a category, posets form the category \(\pos\), where
    \(\ob{\pos}\) are posets and \(\mor{\pos}{X}{Y}\) are the monotone functions
    \(X \to Y\).
\end{example}

\begin{example}[Monoids]\label{ex:monoid}
    A \emph{monoid} is a tuple \((A, *, e)\) where \(A\) is a set called
    the \emph{carrier}, \(\morph{*}{A \times A}{A}\) is a binary operation
    called the \emph{multiplication}, and \(e \in A\) is an element called
    the \emph{unit}, such that \(a * e = a = e * a\) for any
    \(a \in A\).
    A \emph{monoid homomorphism} between two monoids \((A, *, e_A)\) and
    \((B, +, e_B)\) is a map \(\morph{h}{A}{B}\) such that
    \(h(a * a^\prime) = h(a) + h(a^\prime)\) and \(h(e_A) = e_B\).
    There is a category \(\mon\) with monoids as the objects and
    monoid homomorphisms as the morphisms.
\end{example}

\begin{example}[Product category]
    Given two categories \(\mcc\) and \(\mcd\), their \emph{product category}
    \(\mcc \times \mcd\) is the category with objects defined as \(
    \ob{\mcc \times \mcd} \coloneqq \ob{\mcc} \times \ob{\mcd}
    \) and the morphisms as defined as \[
        \mor{(\mcc \times \mcd)}{(A, A^\prime)}{(B, B^\prime)}
        \coloneqq
        \{
        (f, f^\prime)
        \,|\,
        f \in \mor{\mcc}{A}{B},
        f^\prime \in \mor{\mcd}{A^\prime}{B^\prime}
        \}
    \]
    For morphisms \(\morph{f}{A}{B}, \morph{g}{B}{C} \in \mcc\) and \(
    \morph{f^\prime}{A^\prime}{B^\prime}, \morph{g^\prime}{B^\prime}{C^\prime}
    \in \mcd\), the composition of \(\morph{(f,g)}{(A,A^\prime)}{(B,B^\prime)}\)
    and \(\morph{(g,g^\prime)}{(B,B^\prime)}{(C,C^\prime)}\) is defined as \(
    (g, g^\prime) \circ (f, f^\prime)
    \coloneqq
    (g \circ f, g^\prime \circ f^\prime)\).
\end{example}

\subsection{Universal properties}

Category theory is an appealing foundation because it can be used to
\emph{abstract} away from concrete constructions.
Rather than proving results about particular objects and morphisms, we can
show how they are an instantiation of some more abstract concept.
One such way we can do this is by considering the properties a morphism might
have.

\begin{definition}[Monomorphism]
    A morphism \(\morph{f}{A}{B} \in \mcc\) is called a \emph{monomorphism} (or
    simply \emph{mono} for short) if for any two morphisms
    \(\morph{g_1,g_2}{C}{A}\), if \(f \circ g_1 = f \circ g_2\), then
    \(g_1 = g_2\).
    \begin{center}
        \begin{tikzcd}
            C
            \arrow[bend left]{r}{g_1}
            \arrow[bend right, swap]{r}{g_2}
            &
            A
            \arrow{r}{f}
            &
            B
        \end{tikzcd}
    \end{center}
\end{definition}

One can think of monomorphisms as morphisms which are \emph{left-cancellative}.
There is also a way to describe \emph{invertible} morphisms.

\begin{definition}[Isomorphism]
    A morphism \(\morph{f}{A}{B} \in \mcc\) is called an \emph{isomorphism} (or
    simply \emph{iso} for short) if there also exists a morphism \(
    \morph{\inverse{f}}{B}{A} \in \mcc
    \) such that \(
    \inverse{f} \circ f = \id[A]
    \) and \(
    f \circ \inverse{f} = \id[B]
    \).
    \begin{center}
        \begin{tikzcd}
            A
            \arrow{r}{f}
            \arrow[swap, bend right]{rr}{\id[A]}
            &
            B
            \arrow{r}{\inverse{f}}
            &
            A
            &
            &
            B
            \arrow{r}{\inverse{f}}
            \arrow[swap, bend right]{rr}{\id[B]}
            &
            A
            \arrow{r}{f}
            &
            B
        \end{tikzcd}
    \end{center}
\end{definition}

\begin{example}
    In \(\set\), the monomorphisms are the injective functions and the
    isomorphisms are the bijective functions.
\end{example}

Often we are concerned with particular \emph{constructions} in a category;
some interaction of objects and morphisms which has special properties.
These are specified in terms of a \emph{universal property}: a unique morphism
that indicates the `best' way to describe something.
We will first consider universal properties concerning special objects.

\begin{definition}[Initial object]
    An object \(C\) in a category \(\mcc\) is \emph{initial} if, for any other
    object \(X \in \mcc\) there exists a unique morphism \(C \to X\).
\end{definition}

\begin{example}
    In \(\set\), the initial object is the empty set \(\emptyset\), as there is
    a unique function from \(\emptyset\) to any set \(X\), the so-called
    `absurd function'.
\end{example}

Most categorical notions also have a \emph{dual}, in which all the
arrows are flipped.
This means that when defining constructions and proving results about them, we
often also get results for free about the dual case.

\begin{definition}[Terminal object]
    An object \(C\) in a category \(\mcc\) is \emph{terminal} if, for any other
    object \(X \in \mcc\) there exists a unique morphism \(X \to C\).
\end{definition}

\begin{example}
    In \(\set\), the terminal object is the set containing a single object
    \(\star\): from any set \(X\) there is a unique function
    \(X \to \{\star\}\); namely the function \(x \mapsto \star\).
\end{example}

If a category has an initial or terminal object, then it is
\emph{unique up to unique isomorphism}; this means that if we have two objects
that satisfy the universal property, then these objects are isomorphic in a
unique way.

We will now explore some universal properties that illustrate how common
structures can be expressed in terms of the morphisms between them.

\begin{definition}[Product]\label{def:product}
    Given a category \(\mcc\) and objects \(A,B \in \mcc\), their \emph{product}
    is an object \(A \times B\) equipped with a pair of morphisms
    \(\morph{p_0}{A \times B}{A}\) and \(\morph{p_1}{A \times B}{B}\) called
    \emph{projections} such that for every other object \(Z\) with pair of
    morphisms \(\morph{f}{Z}{A}\) and \(\morph{g}{Z}{B}\), there exists a unique
    morphism \(\morph{u}{Z}{A \times B}\) such that the following diagram
    commutes:
    \begin{center}
        \includestandalone{figures/category/diagrams/product}
    \end{center}
    A category \(\mcc\) is said to \emph{have products} if the product exists
    for all objects \(A,B \in \mcc\).
\end{definition}

Again, if a category has a product \(A \times B\) then it is unique up to unique
isomorphism: any other construction \(A \times^\prime B\) also satisfying the
universal property must be isomorphic to \(A \times B\).
This means that we are justified in referring to `the' product if one exists.

\begin{example}
    The product in \(\set\) is the Cartesian product.
\end{example}

The dual of a product is a construction with \emph{in}jections rather than
\emph{pro}jections.

\begin{definition}[Coproduct]
    Given a category \(\mcc\) and objects \(A,B \in \mcc\), their \emph{coproduct}
    is an object \(A + B\) equipped with a pair of morphisms
    \(\morph{i_0}{A}{A + B}\) and \(\morph{i_1}{B}{A + B}\) called
    \emph{injections} such that for every other object \(Z\) with pair of
    morphisms \(\morph{f}{A}{B}\) and \(\morph{g}{B}{Z}\), there exists a unique
    morphism \(\morph{u}{A + B}{Z}\) such that the following diagram
    commutes:
    \begin{center}
        \includestandalone{figures/category/diagrams/coproduct}
    \end{center}
    A category \(\mcc\) is said to \emph{have coproducts} if the coproduct
    exists for all pairs of objects \(A,B \in \mcc\).
\end{definition}

\begin{example}
    The coproduct in \(\set\) is the disjoint union.
\end{example}

Finally, we will look at properties showing how pairs of morphisms with a
common domain or codomain can be in some way `unified'.

\begin{definition}[Pushout]
    Given a category \(\mcc\) and morphisms
    \(\morph{f}{A}{B}\) and \(\morph{g}{A}{C}\), a
    \emph{pushout} is an object \(D \in \mcc\) and a pair of morphisms
    \(\morph{h}{B}{D}\) and \(\morph{k}{C}{D}\) such that for any other pair of
    morphisms \(\morph{h^\prime}{B}{Z}\) and \(\morph{k^\prime}{C}{Z}\) there
    exists a unique morphism \(\morph{u}{D}{Z}\), i.e.\ the following diagram
    commutes:
    \begin{center}
        \includestandalone{figures/category/diagrams/pushout}
    \end{center}
    A category \(\mcc\) is said to \emph{have pushouts} if a pushout exists for
    any pair of morphisms.
\end{definition}

\begin{example}
    \(\set\) has pushouts: given morphisms \(\morph{f}{A}{B}\) and
    \(\morph{g}{A}{C}\), the pushout is the union of \(B\) and \(C\) identifying
    elements with a common preimage in \(A\).
    Concretely, let \({\sim} \subseteq B \times C\) be a relation defined as
    \(\{(b, c) \,|\, \exists a \in A.\, b = f(a) \wedge c = g(a)\}\).
    Then the pushout set \(D\) is defined as \(B \cup C / \sim\) with the
    morphisms \(\morph{h}{B}{D}\) and \(\morph{k}{C}{D}\) sending elements in
    \(B\) and \(C\) to the appropriate element in \(D\).
\end{example}

A pushout square is normally indicated with a \raisebox{-0.25em}{\(\ulcorner\)}
symbol as shown above.
The dual of the pushout is a \emph{pullback}.

\begin{definition}[Pullback]
    Given a category \(\mcc\) and morphisms \(\morph{f}{B}{A}\) and
    \(\morph{g}{C}{A}\), a \emph{pullback} is an object \(D \in \mcc\) and a
    pair of morphisms \(\morph{h}{D}{B}\) and \(\morph{k}{D}{C}\) such that for
    any other pair of morphisms \(\morph{h^\prime}{Z}{B}\) and
    \(\morph{k^\prime}{Z}{C}\) there exists a unique morphism
    \(\morph{u}{Z}{D}\), i.e.\ the following diagram commutes:
    \begin{center}
        \includestandalone{figures/category/diagrams/pullback}
    \end{center}
    A category \(\mcc\) is said to \emph{have pullbacks} if a pullback exists
    for any pair of morphisms.
\end{definition}

\begin{example}
    \(\set\) also has pullbacks: given morphisms \(\morph{f}{B}{A}\) and
    \(\morph{g}{C}{A}\), the pullback is defined as
    \(\{(b,c) \in B \times C \,|\, f(b) = g(c)\}\) and the maps
    \(\morph{h}{D}{B}\) and \(\morph{k}{D}{C}\) are defined as the first and
    second projection respectively.
\end{example}

Universal properties mean we do not need to restrict ourselves to a
concrete category when using or proving results; we can instead say, for
example, that the result holds in all categories with products.