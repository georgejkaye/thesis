\section{Monoidal categories}

The concepts of functors and natural transformations are used to interpret
the \emph{parallel} composition \(\tensor\).
To do this, a special kind of functor known as a \emph{bifunctor} is used.

\begin{definition}[Bifunctor]
    A \emph{bifunctor} is a functor with a product category as its domain, i.e.\
    a functor of the form \(\mcc \times \mcd \to \mce\).
\end{definition}

The notation for functor boxes can be extended in order to show how bifunctors
map from two categories into one.
\[
    \iltikzfig{strings/category/functors/bif}[box1=f,box2=g,dom1=A,dom2=C,cod1=B,cod2=D,colour=white,functor=F]
\]
This suggests that a bifunctor is what we need to model
parallel composition.

\begin{definition}[Monoidal category]
    \label{def:monoidal-category}
    A \emph{monoidal category} is a category \(\mcc\) equipped with a
    bifunctor \(\morph{{-} \tensor {=}}{\mcc \times \mcc}{\mcc}\) called the
    \emph{tensor product} and an additional object \(I\) called the
    \emph{monoidal unit},
    along with natural isomorphisms
    \begin{itemize}
        \item \(
              \associator{A}{B}{C}
              \colon
              A \tensor (B \tensor C)
              \cong
              (A \tensor B) \tensor C
              \) called the \emph{associator};
        \item \(
              \leftunitor{A}
              \colon
              I \tensor A
              \cong
              A
              \) called the \emph{left unitor}; and
        \item \(
              \rightunitor{A}
              \colon
              A \tensor I
              \cong
              A
              \) called the \emph{right unitor}
    \end{itemize}
    such that the \emph{pentagon} and the \emph{triangle} diagrams in
    \cref{fig:mc-diagrams} commute.
\end{definition}

\begin{figure}
    \centering
    \includestandalone{figures/category/coherences/pentagon}

    \vspace{1em}

    \includestandalone{figures/category/coherences/triangle}
    \caption{Commutative diagrams for monoidal categories}
    \label{fig:mc-diagrams}
\end{figure}

\begin{example}
    \(\set\) is a monoidal category, with the tensor product defined as the
    Cartesian product (\(A \tensor B \coloneqq A \times B\)) and the unit as the
    singleton set (\(I \coloneqq \mathbbm{1}\)).
\end{example}

\begin{notation}
    We adopt the convention that \(\tensor\) takes precedence over \(\seq\),
    i.e.\ \(f \tensor g \seq h \tensor k\) should be
    bracketed as \((f \tensor g) \seq (h \tensor k)\).
\end{notation}

We will use the \(\tensor\) bifunctor extensively through this thesis.
For this reason, when drawing string diagrams for monoidal categories we will
forego drawing the (bi)functor boxes and usually draw wires exclusively in
their `deconstructed' state: instead of a single wire \(A \tensor B\) we will
draw two wires \(A\) and \(B\).
\begin{gather*}
    \iltikzfig{strings/monoidal/tensor-notation-compact}[colour=white,dom1=A,dom2=C,cod1=B,cod2=D,box1=f,box2=g]
    \coloneqq
    \iltikzfig{strings/monoidal/tensor-notation-boxes}[colour=white,dom1=A,dom2=C,cod1=B,cod2=D,box1=f,box2=g]
    \coloneqq
    \iltikzfig{strings/monoidal/tensor-notation-hybrid}[colour=white,dom1=A,dom2=C,cod1=B,cod2=D,box1=f,box2=g]
    \coloneqq
    \iltikzfig{strings/monoidal/tensor-notation-noboxes}[colour=white,dom1=A,dom2=C,cod1=B,cod2=D,box1=f,box2=g]
\end{gather*}

The definition of monoidal category we have presented is quite general,
particularly with regards to the natural isomorphisms for unitors and
associators.
In our setting, it is normally sufficient for these isomorphisms to hold `on the
nose'.

\begin{definition}[Strict monoidal category]
    A monoidal category is \emph{strict} if \(\lambda\), \(\rho\) and \(\alpha\)
    are identities.
\end{definition}

In a strict monoidal category, the unitality and associativity of the tensor
hold as equations, as they do for regular composition in a category.
With this in mind, it can be instructive to view the coherences of a strict
monoidal category in terms of equations: these are illustrated in
\cref{fig:mc-equations}.

\begin{figure}
    \begin{center}
        \begin{tabular}{ccccccc}
            \(f \tensor \id[I]\) & \(=\) & \(f\)
                                 &       &
            \(\id[I] \tensor f\) & \(=\) & \(f\)
            \\[1em]
            \iltikzfig{strings/monoidal/unit-l-lhs}[box=f,colour=white,dom=A,cod=B]
                                 &
            \(=\)
                                 &
            \iltikzfig{strings/category/f}[box=f,dom=A,cod=B,colour=white]
                                 &       &
            \iltikzfig{strings/monoidal/unit-r-lhs}[box=f,colour=white,dom=A,cod=B]
                                 &
            \(=\)
                                 &
            \iltikzfig{strings/category/f}[box=f,dom=A,cod=B,colour=white]
        \end{tabular}
        \\[1em]
        \rule[1em]{\textwidth}{0.1mm}
        \\[0.1em]
        \begin{tabular}{ccccccc}
            \((f \tensor g) \tensor h\) & \(=\) & \(f \tensor (g \tensor h)\)
                                        &       &
            \(\id[A] \tensor \id[B]\)   & \(=\) & \(\id[A \tensor B]\)
            \\[1em]
            \iltikzfig{strings/monoidal/associativity-lhs}[box1=f,box2=g,box3=h,colour=white,dom1=A,cod1=B,dom2=C,cod2=D,dom3=E,cod3=F,colour=white]
                                        &
            \(=\)
                                        &
            \iltikzfig{strings/monoidal/associativity-rhs}[box1=f,box2=g,box3=h,colour=white,dom1=A,cod1=B,dom2=C,cod2=D,dom3=E,cod3=F,colour=white]
                                        &       &
            \iltikzfig{strings/monoidal/identity-tensor-lhs}[colour=white,obj1=A,obj2=B]
                                        &
            \(=\)
                                        &
            \iltikzfig{strings/category/identity}[colour=white,obj=A \tensor B]
        \end{tabular}
        \\[1em]
        \rule[1em]{\textwidth}{0.1mm}
        \\[0.1em]
        \begin{tabular}{ccc}
            \((f \tensor g) \seq (h \tensor k)\) & \(=\) & \((f \seq h) \tensor (g \seq k)\)
            \\[1em]
            \iltikzfig{strings/monoidal/interchange-lhs}[box1=f,box2=g,box3=h,box4=k,colour=white,dom1=A,cod1=C,dom2=D,cod2=F]
                                                 &
            \(=\)
                                                 &
            \iltikzfig{strings/monoidal/interchange-rhs}[box1=f,box2=g,box3=h,box4=k,colour=white,dom1=A,cod1=C,dom2=D,cod2=F]
        \end{tabular}
    \end{center}
    \caption{
        Equations of a strict monoidal category
    }
    \label{fig:mc-equations}
\end{figure}

\section{Symmetric monoidal categories}

We can now construct morphisms by composing them in sequence and in parallel,
but there is no way to cross over the wires that connect boxes together.
This can be achieved by equipping the categorical setting with another natural
isormorphism.

\begin{definition}[Symmetric monoidal category]
    \label{def:symmetric-monoidal-category}
    A \emph{symmetric monoidal category} (SMC) is a monoidal category \(\mcc\)
    equipped with a natural isomorphism \(
    \swap{A}{B} \colon A \tensor B \cong B \tensor A
    \) such that the diagrams in \cref{fig:smc-diagrams} commute.
\end{definition}

\begin{figure}
    \centering
    \includestandalone{figures/category/coherences/symmetry-unit}
    \includestandalone{figures/category/coherences/symmetry-inverse}

    \vspace{1em}

    \includestandalone[scale=0.95]{figures/category/coherences/hexagon}
    \caption{Commutative diagrams for symmetric monoidal categories}
    \label{fig:smc-diagrams}
\end{figure}

As \(\sigma\) is a natural isomorphism, it induces
a \emph{family} of morphisms \(
\morph{\swap{A}{B}}{A \tensor B}{B \tensor A}
\) for every pair of objects \(A\) and \(B\).
In string diagrams, each such morphism \(\swap{A}{B}\) is drawn as \(
\iltikzfig{strings/symmetric/symmetry}[obj1=A, obj2=B, colour=white]
\): the swapping of wires \(A\) and \(B\).
As with identities, the use of this morphism is so common that we usually
drop the box around it.

Once again, we are primarily concerned with \emph{strict} symmetric
monoidal categories.
The equations of strict symmetric monoidal categories are listed in
\cref{fig:smc-equations}.

\begin{figure}
    \begin{center}
        \begin{tabular}{ccc}
            \(f \tensor g \seq \swap{B}{D} = \swap{A}{C} \seq g \tensor f\)
             &  &
            \(
            \swap{A}{B} \tensor \id[C] \seq \id[B] \tensor \swap{A}{C}
            =
            \swap{A}{B \tensor C}
            \)
            \\[1em]
            \(
            \iltikzfig{strings/symmetric/naturality-lhs}[box1=f,box2=g,colour=white,dom1=A,cod1=B,dom2=C,cod2=D]
            =
            \iltikzfig{strings/symmetric/naturality-rhs}[box1=f,box2=g,colour=white,dom1=A,cod1=B,dom2=C,cod2=D]
            \)
             &  &
            \(
            \iltikzfig{strings/symmetric/hexagon-lhs}[obj1=A,obj2=B,obj3=C,colour=white]
            =
            \iltikzfig{strings/symmetric/symmetry}[obj1=A,obj2=B \tensor C,colour=white]
            \)
        \end{tabular}
        \\[1em]
        \rule[1em]{\textwidth}{0.1mm}
        \\[0.1em]
        \begin{tabular}{ccccc}
            \(\swap{I}{A} = \id[A]\)
             &  &
            \(\swap{A}{I} = \id[A]\)
             &  &
            \(\swap{A}{B} \seq \swap{B}{A} = \id[A \tensor B]\)
            \\[1em]
            \(
            \iltikzfig{strings/symmetric/unit-l-lhs}[colour=white,obj=A,unit=I]
            =
            \iltikzfig{strings/category/identity}[obj=A,colour=white]
            \)
             &  &
            \(
            \iltikzfig{strings/symmetric/unit-r-lhs}[colour=white,obj=A,unit=I]
            =
            \iltikzfig{strings/category/identity}[obj=A,colour=white]
            \)
             &  &
            \(
            \iltikzfig{strings/symmetric/self-inverse-lhs}[colour=white,obj1=A,obj2=B]
            =
            \iltikzfig{strings/category/identity}[obj=A \tensor B,colour=white]
            \)
        \end{tabular}
    \end{center}
    \caption{
        Equations of a strict symmetric monoidal category
    }
    \label{fig:smc-equations}
\end{figure}

\begin{example}
    \(\set\) is a symmetric monoidal category, with \(
    \morph{\swap{A}{B}}{A \times B}{B \times A}
    \) defined as the function \(\swap{A,B}(a, b) = (b, a)\).
\end{example}