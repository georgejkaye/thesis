\section{Freely generated PROPs}

We now have the pieces define the categorical setting for reasoning
about terms.

The morphisms of a freely generated category are defined as

\begin{definition}\label{def:freely-generated-prop}
    Given a set of generators \(\generators\), let \(\smc{\generators}\) be the
    divisible \(\natplus\)-sorted PROP where
    \(\smc{\generators}(\listvar{m}, \listvar{n})\) is the set of
    \(\Sigma\)-terms of type \(\listvar{m} \to \listvar{n}\) quotiented by
    the equations of divisible \(\natplus\)-sorted PROPs.
\end{definition}

Such a category is known as a category \emph{freely generated over}
\(\Sigma\), in that all of the morphisms in \(\smc{\generators}\) have been
`generated' by combining elements of \(\Sigma\) in various ways using
composition and tensor.
Crucially, many \(\Sigma\)-terms correspond to the \emph{same} morphism in
\(\smc{\generators}\)
subject to the
equations of divisible \(\natplus\)-sorted PROPs.
