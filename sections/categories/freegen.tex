\section{Freely generated PROPs}

We now have the ingredients required to model terms in a categorical setting.

\begin{definition}\label{def:freely-generated-prop}
    Given a set of generators \(\generators\), let \(\smcsigma\) be the
    PROP where \(\smcsigma(m, n)\) is the set of \(\Sigma\)-terms of type
    \(m \to n\) quotiented by the equations of PROPs.
\end{definition}

Such a category is known as a category \emph{freely generated over}
\(\Sigma\), in that all of the morphisms in \(\smc{\generators}\) have been
`generated' by combining elements of \(\Sigma\) in various ways using
composition and tensor.
Crucially, many \(\Sigma\)-terms correspond to the \emph{same} morphism in
\(\smc{\generators}\), as the latter are subject to the equations of PROPs.

When we have a set of colours \(C\) and generators with inputs and outputs which
are words of \(C\), we can generate a coloured PROP.

\begin{definition}\label{def:freely-generated-coloured-prop}
    Given a countable set of colours \(C\) generators \(\generators\), let
    \(\smcsigmac\) be the
    \(\natplus\)-sorted PROP where
    \(\smc{\generators}(\listvar{m}, \listvar{n})\) is the set of
    \(\Sigma\)-terms of type \(\listvar{m} \to \listvar{n}\) quotiented by
    the equations of \(C\)-coloured PROPs.
\end{definition}
