\section{Terms}\label{sec:terms}

We are interested in using category theory as a tool to study \emph{circuits}
built from some primitive components, or \emph{generators}.

\begin{definition}[Generators]\label{def:generators}
    A set of \emph{generators} \(\generators\) is a set equipped with two
    functions \(\morph{\dom,\cod}{\generators}{\nat}\).
\end{definition}

Generators are the primitive building blocks of terms: their domains and
codomains specify how many input and output wires they have.
Terms are defined by combining these primitives together.

\begin{definition}[Term]
    \label{def:terms}
    Let \(\generators\) be a set of generators.
    A \(\generators\)-term is written \(\morph{f}{m}{n}\)
    where \(m,n \in \nat\).
    The set of \(\generators\)-terms, denoted \(\sigmaterms\), is
    generated as follows:
    \begin{center}
        \begin{bprooftree}
            \AxiomC{\(\phi \in \generators\)}
            \UnaryInfC{\(
                \morph{\phi}{\dom[\phi]}{\cod[\phi]} \in \sigmaterms
                \)}
        \end{bprooftree}
        \begin{bprooftree}
            \AxiomC{\phantom{\(\phi\)}}
            \UnaryInfC{\(\morph{\id[1]}{1}{1} \in \sigmaterms\)}
        \end{bprooftree}

        \vspace{1em}

        \begin{bprooftree}
            \AxiomC{}
            \UnaryInfC{\(\morph{\id[0]}{0}{0} \in \sigmaterms\)}
        \end{bprooftree}
        \begin{bprooftree}
            \AxiomC{}
            \UnaryInfC{\(
                \morph{\swap{1}{1}}{2}{2} \in \sigmaterms
                \)}
        \end{bprooftree}

        \vspace{1em}

        \begin{bprooftree}
            \AxiomC{\(\morph{f}{m}{n} \in \sigmaterms\)}
            \AxiomC{\(\morph{g}{n}{p} \in \sigmaterms\)}
            \BinaryInfC{\(\morph{f \seq g}{m}{p} \in \sigmaterms\)}
        \end{bprooftree}
        \begin{bprooftree}
            \AxiomC{\(\morph{f}{m}{n} \in \sigmaterms\)}
            \AxiomC{\(\morph{g}{p}{q} \in \sigmaterms\)}
            \BinaryInfC{\(
                \morph{f \tensor g}{m+p}{n+q} \in \sigmaterms
                \)}
        \end{bprooftree}
    \end{center}
\end{definition}

\(\Sigma\)-terms are constructed recursively.
There are four base cases: a generator from \(\generators\) with appropriate
inputs and outputs; an \emph{identity} for single wires, empty space, and
a \emph{symmetry} for swapping over two wires.
The two inductive cases are called \emph{composition} and \emph{tensor}
respectively.
Intuitively, these can be thought of as generating larger terms by composing
subterms in sequence or parallel.

Although here identities and symmetries only operate on single wires, it is
a simple exercise to define them for larger numbers of wires.

\begin{notation}[Composite identities]\label{not:composite-identities}
    Composite identities \(\id[n]\) are defined inductively for \(n \in \nat\)
    as \(
    \id[0] \coloneqq \id[0]\) and \(
    \id[k+1]
    \coloneqq
    \id[k] \tensor \id[1]
    \).
\end{notation}
\begin{notation}[Composite symmetries]\label{not:composite-symmetries}
    Composite symmetries \(\swap{m}{n}\) for \(m, n \in \nat\) are defined
    inductively as \[
        \swap{0}{n}
        \coloneqq
        \id[n]
        \quad
        \swap{m}{0}
        \coloneqq \id[m]
        \quad
        \swap{k+1}{l+1}
        \coloneqq
        \id[k] \tensor \swap{1}{l} \tensor \id[1]
        \seq
        \swap{k}{l} \tensor \swap{1}{1}
        \seq
        \id[l] \tensor \swap{k}{1} \tensor \id[1]
    \]
\end{notation}

\(\Sigma\)-terms will be abbreviated to `terms' when the signature is clear from
context.

\begin{example}\label{ex:terms}
    Let \(
    \Sigma_g
    \coloneqq
    \{\morph{\andgate}{2}{1}, \morph{\orgate}{2}{1}, \morph{\notgate}{1}{1}\}
    \) be a set of \emph{logic gate generators}.
    Examples of terms in \((\Sigma_g)_\mathsf{t}\) include
    \begin{gather*}
        \morph{(\orgate \seq \id[1]) \tensor ((\id[1] \tensor \notgate) \seq \andgate)}{4}{2}
        \\
        \morph{((\id[1] \tensor \id[1]) \seq \orgate) \tensor ((\id[1] \tensor \notgate) \seq \andgate)}{4}{2}
        \\
        \morph{((\id[1] \tensor \id[1]) \tensor (\id[1] \tensor \notgate)) \seq (\orgate \tensor \andgate)}{4}{2}
    \end{gather*}
\end{example}

\section{String diagrams}

Even simple terms described using one-dimensional text strings quickly become
indecipherable.
Fortunately, terms have a graphical syntax known as
\emph{string diagrams}~\cite{joyal1991geometry} that makes reading terms far
more intuitive.
In a string diagram, a generator \(\morph{\phi}{m}{n}\) is drawn as a box with
\(m\) inputs and \(n\) outputs \(
\iltikzfig{strings/category/generator-wires-tile}[dom=m,cod=n]
\), the identity \(\id[1]\) as a wire \(
\iltikzfig{strings/category/identity-tile}
\), the empty identity \(\id[0]\) as empty space \(
\iltikzfig{strings/monoidal/empty-tile}
\), and the symmetry as two wires swapping over \(
\iltikzfig{strings/symmetric/symmetry-tile}
\).
Composite terms are drawn as boxes \(
\iltikzfig{strings/category/f-wires-tile}[box=f,dom=m,cod=n]
\); composition is then depicted as \emph{horizontal juxtaposition} and
tensor as \emph{vertical juxtaposition}; both are illustrated in
\cref{fig:strings-composition}.
\begin{figure}
    \centering
    \begin{gather*}
        \iltikzfig{strings/category/f-wires-tile}[box=f,dom=m,cod=n]
        \seq
        \iltikzfig{strings/category/f-wires-tile}[box=g,dom=n,cod=p]
        =
        \iltikzfig{strings/category/composition-tiles}[box1=f,box2=g,dom1=m,dom2=n,cod2=p]
        \\[0.5em]
        \iltikzfig{strings/category/f-wires-tile}[box=f,dom=m,cod=n]
        \tensor
        \iltikzfig{strings/category/f-wires-tile}[box=g,dom=p,cod=q]
        =
        \iltikzfig{strings/monoidal/tensor-tiles}[box1=f,box2=g,dom1=m,cod1=n,dom2=p,cod2=q]
    \end{gather*}
    \caption{Sequential and parallel composition of string diagrams}
    \label{fig:strings-composition}
\end{figure}

\begin{remark}
    The direction that the `flow' of string diagrams travels from inputs to
    outputs is a hotly-debated topic; in this thesis we adopt the left-to-right
    approach.
    If you disagree, you could rotate the document by ninety degrees or
    use a mirror.
\end{remark}

\begin{example}\label{ex:term-diagrams}
    Recall the set of generators from \cref{ex:terms}.
    We will draw each logic gate using the usual notation, i.e.\
    \[
        \andgate \coloneqq \iltikzfig{circuits/components/gates/and-tile}
        \quad
        \orgate \coloneqq \iltikzfig{circuits/components/gates/or-tile}
        \quad
        \notgate \coloneqq \iltikzfig{circuits/components/gates/not-tile}
    \]
    Now the terms in \cref{ex:terms} can be illustrated as:
    \begin{gather*}
        \iltikzfig{strings/terms/example-0}
        \qquad
        \iltikzfig{strings/terms/example-1}
        \qquad
        \iltikzfig{strings/terms/example-2}
    \end{gather*}
    Note that while each term has a very different text string, and is tiled
    differently in each diagram, the connectivity of the thick wires is actually
    the same.
\end{example}

The above example illuminates the drawback of reasoning with term strings or
even the `tile' graphical syntax; there are terms which are `morally' the same
but because of variations in how they are constructed they are \emph{not} the
same term syntactically.
We need some equations to identify terms appropriately; we will soon see how
these equations can be derived using the mathematical structure of a
\emph{category}.

\begin{notation}\label{not:arbitrary-width-wires}
    So far we have drawn multiple wires in parallel either explicitly or
    by using vertical dots.
    In the interests of clarity, we will now start to collapse multiple wires
    into a single wire, annotated with the appropriate number when not clear
    from context: so \(
    \iltikzfig{strings/category/f-tile}[box=f,dom=m,cod=n]
    \coloneqq
    \iltikzfig{strings/category/f-wires-tile}[box=f,dom=m,cod=n]
    \).

    Note that these values \(m\) or \(n\) could be zero, and as such can just be
    drawn as empty space.
    It may be useful to explicitly draw wires of zero width for applying
    reasoning steps; we will do so with a `faded' wire, like \(
    \iltikzfig{strings/category/identity-unit-tile}[obj=0,colour=white]
    \).
\end{notation}

\section{Coloured terms}

In \(\Sigma\)-terms, the wires are \emph{monochromatic}; there is no
distinguishing between them.
Sometimes it is advantageous to \emph{annotate} wires with some information: in
the realm of terms this is known as assigning the wires \emph{colours} or
\emph{sorts}.
When working with coloured terms, we need to fix the set of colours before
specifying a set of generators.

\begin{notation}
    We say that a set \(C\) is \emph{countable} if it is finite or countably
    infinite, i.e.\ there exists a set \(X \subseteq \nat\) such that there is a
    bijection \(C \cong X\).
\end{notation}

\begin{remark}
    Usually the set of colours is finite, but we will see later in this thesis
    how having a colour for every single natural number might be useful.
\end{remark}

In the monochromatic world the interface of a generator can be specified solely
by two natural numbers \(m\) and \(n\), as there are \(m\) input wires and
\(n\) output wires.
When the wires are coloured, more information is needed: the inputs and outputs
must be specified in terms of their colours and their ordering.

\begin{notation}[Words]\label{not:words}
    Given a set \(A\), the set of (finite) words of elements of \(A\) is denoted
    \(\freemon{A}\).
    Words are written \(x_0x_1x_2{\cdots}x_{n-1}\); arbitrary words are written
    with an overline \(
    \listvar{x}, \listvar{y}, \listvar{z}... \in \freemon{A}
    \).
    Given two words \(\listvar{x}, \listvar{y}\), their concatenation is
    denoted \(\listvar{xy}\).
    For an element \(a \in A\), the concatenation of this element \(n\) times is
    written \(a^n\), e.g.\ \(a^3 \coloneqq aaa\).
    Given a word \(\listvar{x}\), its \emph{length} is denoted
    \(\wordlen{\listvar{x}}\); for \(i < \wordlen{\listvar{x}}\) the
    \(i\)-\emph{th element of \(\listvar{x}\)} is denoted \(\listvar{x}(i)\),
    i.e.\ in word \(\listvar{x} \coloneqq abc\), \(\listvar{x}(0) = a\),
    \(\listvar{x}(1) = b\) and \(\listvar{x}(2) = c\).
\end{notation}

\begin{definition}[Coloured generators]
    For a countable set \(C\), a set of \emph{\(C\)-coloured generators}
    \(\Sigma\) is a set equipped with two functions
    \(\morph{\dom,\cod}{\Sigma}{\freemon{C}}\).
\end{definition}

Coloured terms are generated in the same way as before, but terms can only be
composed if the coloured wires in the interfaces match up.

\begin{definition}[Coloured terms]
    For a countable set \(C\) and a set of \(C\)-coloured generators, a
    \((C,\Sigma)\)-term is written \(\morph{f}{\listvar{m}}{\listvar{n}}\),
    where \(\listvar{m},\listvar{n} \in \freemon{C}\).
    The set of \((C,\Sigma)\)-terms, denoted \((C,\Sigma)_\mathsf{t}\), is
    generated as the monochromatic set of terms, but with an
    identity and symmetry for each \(c \in C\), composition between
    terms whose output and input words agree on colours, and addition replaced
    by word concatenation.
\end{definition}

\begin{remark}
    When the set of colours is a singleton \(C \coloneqq \{\bullet\}\), the
    \(C\)-coloured terms are just the monochromatic terms.
\end{remark}

When drawing coloured terms string diagrammatically, the wires that connect the
generators are coloured appropriately.

\begin{example}
    Recall the logic gate generators from \cref{ex:terms}; let us now say that
    the \(\orgate\) gate is rated for a different voltage, so it is no longer
    compatible with the other components.
    We can model this with coloured terms; let
    \(C_\mathsf{g} \coloneqq \{\bullet,\redbullet\}\) be a set of
    \emph{voltage colours} and let \(
    \Sigma_g^+
    \coloneqq
    \{\morph{\andgate}{\bullet\bullet}{\bullet}, \morph{\orgate}{\redbullet\redbullet}{\redbullet}, \morph{\notgate}{\bullet}{\bullet}\}
    \).
    We draw the `red' \(\orgate\) appropriately as \(
    \iltikzfig{circuits/components/gates/red-or}
    \).
    Now we can recreate the diagrams from \cref{ex:term-diagrams} in the
    coloured setting.
    Note how all wires attached to the \(\orgate\) gate must also be red
    to ensure the colours all match up.
    \begin{gather*}
        \iltikzfig{strings/terms/example-0-coloured}
        \qquad
        \iltikzfig{strings/terms/example-1-coloured}
        \qquad
        \iltikzfig{strings/terms/example-2-coloured}
    \end{gather*}
\end{example}
