\section{PROPs}

By this point, it should be clear that symmetric monoidal categories provide
an excellent setting for modelling terms modulo `structural equations'.
However, we do not need the full generality of an arbitrary SMC; we can restrict
to a more specialised setting known as \emph{PROPs}.

\begin{definition}[PROP~\cite{maclane1965categorical}]
    A \emph{PROP} (short for category of \emph{PRO}ducts and
    \emph{P}ermutations) is a strict symmetric monoidal category with the
    natural numbers as objects and addition as tensor product on objects.
\end{definition}

PROPs are an especially good fit for reasoning with string diagrams; as any
object \(n\) is equal to \(\bigotimes_{i < n} 1\), we can represent a morphism
\(m \to n\) as a box with \(m\) incoming wires and \(n\) outgoing wires.

\begin{definition}\label{def:freely-generated-prop}
    Given a set of generators \(\generators\), let \(\smcsigma\) be the
    PROP where \(\smcsigma(m, n)\) is the set of \(\Sigma\)-terms of type
    \(m \to n\) quotiented by the equations of PROPs.
\end{definition}

Such a category is known as a category \emph{freely generated over}
\(\Sigma\), in that all of the morphisms in \(\smc{\generators}\) have been
`generated' by combining elements of \(\Sigma\) in various ways using
composition and tensor.
Crucially, many \(\Sigma\)-terms correspond to the \emph{same} morphism in
\(\smc{\generators}\), as the latter are subject to the equations of PROPs.

When we discussed terms we also mentioned the notion of \emph{coloured} terms
where the wires can be of different colours; appropriately, there is also a
notion of a coloured PROP.

\begin{definition}[Coloured PROP]
    Given a set of \emph{colours} \(C\), a \emph{\(C\)-coloured PROP} is a strict
    symmetric monoidal category with the objects as lists in \(\freemon{C}\) and
    tensor product as word concatenation.
\end{definition}

\begin{remark}
    If a coloured PROP only has one colour, then it is isomorphic to a regular
    monochromatic PROP.
\end{remark}

Note that this definition means that the empty list \(\varepsilon\) is the unit
object in any \(\mcc\)-sorted PROP.

\begin{definition}\label{def:freely-generated-coloured-prop}
    Given a countable set of colours \(C\) generators \(\generators\), let
    \(\smcsigmac\) be the
    \(\natplus\)-sorted PROP where
    \(\smc{\generators}(\listvar{m}, \listvar{n})\) is the set of
    \(\Sigma\)-terms of type \(\listvar{m} \to \listvar{n}\) quotiented by
    the equations of \(C\)-coloured PROPs.
\end{definition}
