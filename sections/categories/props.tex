\section{PROPs}

By this point, it should be clear that symmetric monoidal categories provide
an excellent setting for modelling terms modulo `structural equations'.
However, we do not need the full generality of an arbitrary SMC; we can restrict
to a more specialised setting known as \emph{PROPs}.

\begin{definition}[PROP~\cite{maclane1965categorical}]
    A \emph{PROP} (category of \emph{PRO}ducts and \emph{P}ermutations) is a
    strict symmetric monoidal category with the natural numbers as objects and
    addition as tensor product on objects.
\end{definition}

PROPs are an especially good fit for reasoning with string diagrams; as any
object \(n\) is equal to \(\bigotimes_{i < n} 1\), we can represent a morphism
\(m \to n\) as a box with \(m\) incoming wires and \(n\) outgoing wires.

\begin{example}

\end{example}

\begin{definition}[Coloured PROP]
    Given a set of \emph{olours} \(C\), a \emph{\(C\)-coloured PROP} is a strict
    symmetric monoidal category with the objects as lists in \(\freemon{C}\) and
    tensor product as word concatenation.
\end{definition}

\begin{remark}
    If a coloured PROP only has one colour, then it is isomorphic to a regular
    monochromatic PROP.
\end{remark}

Note that this definition means that the empty list \(\varepsilon\) is the unit
object in any \(\mcc\)-sorted PROP.

\begin{example}

\end{example}