\subsection{PROPs}

Symmetric monoidal categories are an excellent setting for reasoning modulo
`structural equations'.
We are especially interested in a subclass of SMCs called \emph{PROP}s:
categories of \emph{PRO}ducts and \emph{P}ermutations.

\begin{definition}[PROP~\cite{maclane1965categorical}]\label{def:prop}
    \index{PROP}
    A \emph{PROP} is a strict symmetric monoidal category with the
    natural numbers as objects and addition as tensor product on objects.
\end{definition}

PROPs are a good fit for reasoning with string diagrams; as any
object \(n\) is equal to \(\bigotimes_{i < n} 1\), a morphism
\(m \to n\) is a box with \(m\) incoming wires and \(n\) outgoing wires.

\begin{definition}\label{def:freely-generated-prop}
    \index{PROP!freely generated}
    \nomenclature{\(\smcsigma\)}{PROP generated over set \(\generators\)}
    Given a set of generators \(\generators\), let \(\smcsigma\) be the
    PROP where \(\smcsigma(m, n)\) is the set of \(\Sigma\)-terms of type
    \(m \to n\) quotiented by the equations of SMCs.
\end{definition}

This is known as a category \emph{freely generated over} \(\Sigma\), in that all
of the morphisms in \(\smc{\generators}\) have been `generated' by combining
elements of \(\Sigma\) in various ways using composition and tensor.
Crucially, many \(\Sigma\)-terms correspond to the \emph{same} morphism in
\(\smc{\generators}\), as the latter are subject to the equations of SMCs.

In this thesis we will use multiple PROPs to represent different processes;
often we will need to map between them.
As PROPs are just special categories the most natural way to do this is by
using functors.

\begin{definition}[PROP morphism]
    \index{PROP morphism}
    A \emph{PROP morphism} is a strict symmetric monoidal functor between two
    PROPs i.e.\ a functor that preserves the strict symmetric monoidal
    structure.
\end{definition}

As we saw earlier, functors can be composed and there is an identity functor,
so PROPs themselves form a category.

\begin{definition}
    \index{p@\(\propcat\)}
    \nomenclature{\(\propcat\)}{category of PROPs}
    Let \(\propcat\) be the category with PROPs as objects and PROP morphisms
    as morphisms.
\end{definition}

When we discussed terms we also mentioned the notion of \emph{coloured} terms
where the wires can be of different colours; appropriately, there are also
coloured PROPs.

\begin{definition}[Coloured PROP]
    \index{PROP!coloured}
    Given a set of \emph{colours} \(C\), a \emph{\(C\)-coloured PROP} is a strict
    symmetric monoidal category with the objects as lists in \(\freemon{C}\) and
    tensor product as word concatenation.
\end{definition}

\begin{remark}
    A `regular' PROP as defined in \cref{def:prop} is isomorphic to a
    coloured PROP with only one colour.
\end{remark}

Note that this means the empty list \(\varepsilon\) is the unit
object in any \(\mcc\)-coloured PROP.

\begin{definition}\label{def:freely-generated-coloured-prop}
    \index{PROP!freely generated coloured}
    \index{\(\smcsigmac\)}
    \nomenclature{\(\smcsigmac\)}{coloured PROP generated over coloured set \(\generators\)}
    Given a countable set of colours \(C\) and \(C\)-coloured generators
    \(\generators\), let \(\smcsigmac\) be the \(C\)-coloured PROP where
    \(\smcsigmac(\listvar{m}, \listvar{n})\) is the set of
    \((C,\Sigma)\)-terms of type \(\listvar{m} \to \listvar{n}\) quotiented by
    the equations of SMCs.
\end{definition}

Just like regular PROPs, there are morphisms of coloured PROPs and these form
a category.

\begin{definition}
    \index{@c\(\cprop\)}
    \index{category!of coloured PROPs}
    Let \(\cprop\) be the category with coloured PROPs as objects and coloured
    PROP morphisms as morphisms.
\end{definition}

It can also be useful to consider the category of coloured PROPs over a fixed
set of colours \(C\).

\begin{definition}
    \index{c@\(\cpropc\)}
    \index{category!of \(C\)-coloured PROPs}
    For a countable set of colours \(C\), let \(\cpropc\) be the category with
    \(C\)-coloured PROPs as objects and \(C\)-coloured PROP morphisms as
    morphisms.
\end{definition}