\section{PROPs}

By this point, it should be clear that symmetric monoidal categories provide
an excellent setting for modelling terms modulo `structural equations'.
However, we do not need the full generality of an arbitrary SMC; we can restrict
to a more specialised setting known as \emph{PROPs}.

\begin{definition}[PROP~\cite{maclane1965categorical}]
    A \emph{PROP} (category of \emph{PRO}ducts and \emph{P}ermutations) is a
    strict symmetric monoidal category with the natural numbers as objects and
    addition as tensor product on objects.
\end{definition}

PROPs are an especially good fit for reasoning with string diagrams; as any
object \(n\) is equal to \(\bigotimes_{i < n} 1\), we can represent a morphism
\(m \to n\) as a box with \(m\) incoming wires and \(n\) outgoing wires.

However, the generators of our term language do not just have natural numbers
as their (co)domains, but \emph{lists} of natural numbers.
Fortunately, there is a generalisation of PROPs

\begin{notation}
    We say that a set \(C\) is \emph{countable} if it is finite or countably
    infinite, i.e.\ there is a bijection \(C \cong \nat\).
\end{notation}

\begin{definition}[Multi-sorted PROP]
    Given a set of \emph{sorts} \(\mcc\), a \(\mcc\)-sorted \emph{PROP}
    (category of \emph{PRO}ducts and \emph{P}ermutations) is a strict symmetric
    monoidal category with the objects as lists in \(\freemon{\mcc}\) and tensor
    product as concatenation.
\end{definition}

\begin{remark}
    If the set of sorts is the singleton \(\{\bullet\}\), then the category is
    isomorphic to a `vanilla' PROP.
\end{remark}

Note that this definition means that the empty list \(\varepsilon\) is the unit
object in any \(\mcc\)-sorted PROP.
As the domain and codomains of generators in \(\generators\) are lists of
natural numbers, it is clear to see how a multi-sorted PROP can be used for our
scenario, by setting the set of sorts \(\mcc\) to the positive naturals
\(\natplus\).

\begin{remark}
    Of course, we could just model generators with `flat' (co)domains, and
    reason entirely with PROPs.
    We have chosen the list approach to better reflect systems which have
    distinct input buses of different widths: this will prove useful when
    considering equational theories and rewrite systems.
    As a bonus, diagrams get less cluttered as `thicker' buses can be drawn
    as single wires without using `syntax sugar' as is often the case in string
    diagrammatic theories.
\end{remark}