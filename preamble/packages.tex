\usepackage[a4paper,bottom=1in]{geometry}
% Fix fancyhdr warnings
\setlength{\headheight}{14.5pt}

\usepackage[english]{babel}

% Only used for text generation in the template
%
\usepackage{lipsum}
%%

\usepackage[utf8]{inputenc}
\usepackage[T1]{fontenc}
\usepackage{csquotes}

\usepackage[final]{microtype}

\usepackage[ttscale=.75]{libertine}
\usepackage[libertine]{newtxmath}
\usepackage{caption}

% Set nicer (= less bold, less vertical spacing) mathcal font
\usepackage[cal=cm]{mathalpha}

% Set up the opening page for the chapters
\usepackage{titlesec}
\titleformat{\chapter}[display]
{\Huge}
{\filleft\textsc{\chaptertitlename} \thechapter}
{4ex}
{\bfseries
  \titlerule%
  \vspace{2ex}%
  \filright}
[\vspace{2ex}%
  \titlerule]

% Set up the headers and footers
\usepackage{fancyhdr}
\usepackage{ifthen}
\pagestyle{fancy}
\fancyhf{}
% Use ifthenelse to work around the fact that we wish to have alternate headers
% but a onesided document
% (https://tex.stackexchange.com/questions/69100/distinguish-even-odd-pages-in-header-with-oneside-option)
\fancyhead[R]{\ifthenelse{\isodd{\value{page}}}{\thepage\hfill\textsc{\nouppercase\leftmark}}{}}
\fancyhead[L]{\ifthenelse{\isodd{\value{page}}}{}{\textsc{\nouppercase\rightmark}\hfill\thepage}}
\fancyfoot{}

% Remove page numbers on the first page of a chapter
% https://tex.stackexchange.com/questions/278420/remove-page-number-on-chapter-page
\fancypagestyle{plain}{%
  \renewcommand{\headrulewidth}{0pt}%
  \fancyhf{}%
}

% See the excellent biblatex documentation for more information
\usepackage[
  backend=biber,%
  style=alphabetic,%
  block=ragged,%
  backref=true,%
  useprefix=true,%
  maxnames=8,%
  minnames=7,%
  minalphanames=3,%
  maxalphanames=4,%
  backrefstyle=two]%
{biblatex}
\renewcommand{\subtitlepunct}{\addcolon\addspace}

% Enumerations and tables
\usepackage{calc}
\usepackage[shortlabels]{enumitem}
\setlist{nosep}
\setlist[description]{font={\textnormal},labelindent=\parindent}

\usepackage{booktabs}
\usepackage{longtable}
\usepackage[width=.8\textwidth]{caption}
\captionsetup[table]{skip=1em}

% Math packages
\usepackage{mathtools}

\usepackage{savesym}
\usepackage{amsmath}
\savesymbol{openbox}
\usepackage{amsthm}
\usepackage{bbm}
\usepackage{mathpartir}

\usepackage{thmtools}
\savesymbol{Bbbk}
\usepackage{amssymb}
\usepackage{stmaryrd}

\usepackage{bussproofs}
% allow for multiple proof trees on one line
\newenvironment{bprooftree}
{\leavevmode\hbox\bgroup}
{\DisplayProof\egroup}

% tocbibind allows us to have the toc in the toc
\usepackage[notbib,notindex]{tocbibind}
% Supposedly it should also allow us to have the index and the bibliography in
% the toc, but it has some bugs (e.g. displaying the right page number in the
% toc, but getting the wrong link with hyperref), so we disable those options
% here and use corresponding separate options for the index, index of symbols
% (nomenclature) and bibliography instead.
%
% The whole is rather finicky and it is somehow crucial that tocbibind is loaded
% *before* imakeidx.

\usepackage{imakeidx}
\makeindex[intoc,columns=2]
\usepackage[refpage,intoc,noprefix]{nomencl}
% Set fixed width so that descriptions in the index of symbols are aligned.
\setlength{\nomlabelwidth}{5cm}
\renewcommand{\nomname}{Index of symbols}
% Make page numbers links
\renewcommand*{\pagedeclaration}[1]{\unskip, \hyperpage{#1}}
\makenomenclature%

\usepackage{stringstrings}
\usepackage{xstring}

% Used in hyperref's setup, and must be loaded before tikz-cd.
\usepackage[dvipsnames, usenames]{xcolor}
\usepackage[most]{tcolorbox}
\usepackage{tikz}
\usetikzlibrary{bbox, automata, positioning, arrows}
\usepackage{tikz-cd}
\usepackage{circuitikz}
\usepackage{figures/tikzit}
\newcommand{\bgcolour}{white}
\usepackage{todonotes}

\usepackage{standalone}
\usepackage{svg}

\usepackage{listings}
\makeatletter
\newcommand\BeraMonottfamily{%
  \def\fvm@Scale{0.85}% scales the font down
  \fontfamily{fvm}\selectfont% selects the Bera Mono font
}
\makeatother
\definecolor{keyword}{HTML}{BA2CA3}
\definecolor{string}{HTML}{D12F1B}
\definecolor{comment}{HTML}{008400}
\definecolor{function}{HTML}{4040ff}
\lstdefinelanguage{swift}{
  morekeywords={
      open,catch,@escaping,nil,throws,func,if,then,else,for,in,while,do,switch,case,default,where,break,continue,fallthrough,return,
      typealias,struct,class,enum,protocol,var,func,let,get,set,willSet,didSet,inout,init,deinit,extension,
      subscript,prefix,operator,infix,postfix,precedence,associativity,left,right,none,convenience,dynamic,
      final,lazy,mutating,nonmutating,optional,override,required,static,unowned,safe,weak,internal,
      private,public,is,as,self,unsafe,dynamicType,true,false,nil,Type,Protocol,
    },
  keywordstyle={\color{keyword}},
  keywordstyle=[2]{\color{function}},
  otherkeywords={UseWire,UseSubcircuit,UseAnd,MakeSubcircuit,InterfaceWire,Feedback,UseDelay,UseOr},
  morekeywords=[2]{UseWire,UseSubcircuit,UseAnd,MakeSubcircuit,InterfaceWire,Feedback,UseDelay,UseOr},
  morecomment=[l]{//}, % l is for line comment
  morecomment=[s]{/*}{*/}, % s is for start and end delimiter
  morestring=[b]", % defines that strings are enclosed in double quotes
  stringstyle=\color{string},
  commentstyle=\color{comment},
  breaklines=true,
  escapeinside={\%*}{*)},
  captionpos=b,
  breakatwhitespace=true,
  basicstyle=\linespread{1.0}\ttfamily\footnotesize, % https://tex.stackexchange.com/a/102728/129441
}


\lstset{
  language=swift,
  inputencoding=utf8,
  extendedchars=\true,
  basicstyle=\BeraMonottfamily\small,
  showstringspaces=false, % lets spaces in strings appear as real spaces
  columns=fixed,
  keepspaces=true,
  belowskip=-0.5\baselineskip
}

\usepackage[
  colorlinks=true  % Remove the boxes
  , linktocpage=true % Make page numbers (not section titles) links in ToC
  , linkcolor=NavyBlue    % Colour for internal links
  , citecolor=Green  % Colour for bibliographical citations
  , urlcolor=BrickRed % Colour for (external) urls
  %
  % Metadata
  %
  , pdftitle={Foundations of Digital Circuits}%
  , pdfauthor={George Kaye}%
  , pdfsubject={PhD thesis}%
  , pdfkeywords={digital circuits, symmetric traced monoidal categories, string diagrams, category theory, graph rewriting}
]{hyperref}
\usepackage[numbered]{bookmark}

\usepackage[noabbrev,capitalize]{cleveref}
\creflabelformat{equation}{#2\textup{#1}#3} % Write Equation x.y.z instead of Equation (x.y.z)