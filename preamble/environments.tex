% Setting up the coloured environments
%
\newbool{shade-envs}
% This can be used to toggle the coloured environments on or off.
\setboolean{shade-envs}{true}
%%
\ifthenelse{\boolean{shade-envs}}{%
    % Colours are as in Andrej Bauer's notes on realizability:
    % https://github.com/andrejbauer/notes-on-realizability
    \colorlet{ShadeOfPurple}{blue!5!white}
    \colorlet{ShadeOfYellow}{yellow!25!white}
    \colorlet{ShadeOfGreen} {green!5!white}
    \colorlet{ShadeOfBrown} {brown!10!white}
    % But we also shade proofs
    \colorlet{ShadeOfGray}  {gray!10!white}
}
% If we don't want to have shaded environments, then we use a closing symbol
% \lozenge to mark the end of remarks, definitions and examples.
{%
    \declaretheoremstyle[
        spaceabove=6pt,
        spacebelow=6pt,
        bodyfont=\normalfont,
        qed=\(\lozenge\)
    ]{definitionwithbox}
    \declaretheoremstyle[
        headfont=\itshape,
        bodyfont=\normalfont,
        qed=\(\lozenge\)
    ]{remarkwithbox}
}
\ifthenelse{\boolean{shade-envs}}{%
    \declaretheorem[sibling=equation]{theorem}
    \declaretheorem[sibling=theorem]{lemma,proposition,corollary}
    \declaretheorem[sibling=theorem,style=definition]{definition}
    \declaretheorem[sibling=theorem,style=definition]{example}
    \declaretheorem[sibling=theorem,style=definition]{notation}
    \declaretheorem[sibling=theorem,style=remark]{remark}
    % Now we set the shading using the tcolorbox package.
    %
    % The related thmtools' option "shaded" and the package mdframed seem to have
    % issues: the former does not allow for page breaks in shaded environments and
    % the latter puts double spacing between two shaded environments.
    %
    % Since tcolorbox puts stuff inside a minipage or \parbox (according to this
    % stackexchange answer: https://tex.stackexchange.com/a/250170), new
    % paragraphs aren't indented. We can fix this by grabbing the parindent
    % value and passing it to tcbset.
    \newlength{\normalparindent}
    \AtBeginDocument{\setlength{\normalparindent}{\parindent}}
    \tcbset{shadedenv/.style={
        colback={#1},
        frame hidden,
        enhanced,
        breakable,
        boxsep=0pt,
        left=2mm,
        right=2mm,
        % LaTeX thinks this is too wide (as becomes clear from the many "Overfull
        % \hbox" warnings, but optically it looks spot on.
        add to width=1.1mm,
        enlarge left by=-0.6mm,
        before upper={\setlength{\parindent}{\normalparindent}}
    }}
    \newcommand{\setenvcolor}[2]{%
        \tcolorboxenvironment{#1}{shadedenv={#2}}
        \addtotheorempreheadhook[#1]{
            \tikzcdset{background color=#2}
            \renewcommand{\bgcolour}{#2}
        }
    }
    \setenvcolor{theorem}{ShadeOfPurple}
    \setenvcolor{lemma}{ShadeOfPurple}
    \setenvcolor{proposition}{ShadeOfPurple}
    \setenvcolor{corollary}{ShadeOfPurple}
    \setenvcolor{definition}{ShadeOfYellow}
    \setenvcolor{notation}{ShadeOfYellow}
    \setenvcolor{example}{ShadeOfGreen}
    \setenvcolor{remark}{ShadeOfBrown}
    \setenvcolor{proof}{ShadeOfGray}
}{
    % Use closing symbols if we don't have colours
    \declaretheorem[sibling=equation]{theorem}
    \declaretheorem[sibling=theorem]{lemma,proposition,corollary}
    \declaretheorem[sibling=theorem,style=definitionwithbox]{definition}
    \declaretheorem[sibling=theorem,style=definitionwithbox]{notation}
    \declaretheorem[sibling=theorem,style=definitionwithbox]{example}
    \declaretheorem[sibling=theorem,style=remarkwithbox]{remark}
}
\declaretheorem[sibling=theorem,style=remark,numbered=no]{claim}

% Note that proofs will still have the \qed symbol at the end, even when shaded,
% because we prefer to keep up the tradition.