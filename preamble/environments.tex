% Setting up the coloured environments
%
\newbool{shade-envs}
% This can be used to toggle the coloured environments on or off.
\setboolean{shade-envs}{true}
%%
\ifthenelse{\boolean{shade-envs}}{%
    % Colours are as in Andrej Bauer's notes on realizability:
    % https://github.com/andrejbauer/notes-on-realizability
    \colorlet{ShadeOfBlue}{blue!5!white}
    \colorlet{ShadeOfYellow}{yellow!25!white}
    \colorlet{ShadeOfGreen}{green!5!white}
    \colorlet{ShadeOfRed}{red!5!white}
    \definecolor{ShadeOfDarkGreen}{HTML}{0ac416}
    \definecolor{ShadeOfDarkYellow}{HTML}{e8d100}
    \definecolor{ShadeOfDarkRed}{HTML}{b00000}
    \definecolor{ShadeOfDarkBlue}{HTML}{0212a8}
    \colorlet{ShadeOfBrown} {brown!10!white}
    % But we also shade proofs
    \colorlet{ShadeOfGray}  {gray!10!white}
    \declaretheoremstyle[
        headfont=\fontfamily{lmss}\selectfont\bfseries,
        bodyfont=\fontfamily{lmss}\selectfont
    ]{newtheorem}
    \declaretheoremstyle[
        headfont=\fontfamily{lmss}\selectfont\bfseries,
        bodyfont=\fontfamily{lmss}\selectfont
    ]{newdefinition}
    \declaretheoremstyle[
        headfont=\fontfamily{lmss}\selectfont\bfseries,
        bodyfont=\fontfamily{lmss}\selectfont
    ]{newremark}
}
% If we don't want to have shaded environments, then we use a closing symbol
% \lozenge to mark the end of remarks, definitions and examples.
{%
    \declaretheoremstyle[
        spaceabove=6pt,
        spacebelow=6pt,
        bodyfont=\normalfont,
        qed=\(\lozenge\)
    ]{definitionwithbox}
    \declaretheoremstyle[
        headfont=\itshape,
        bodyfont=\normalfont,
        qed=\(\lozenge\)
    ]{remarkwithbox}
}
\ifthenelse{\boolean{shade-envs}}{%
    \declaretheorem[sibling=equation, numberwithin=chapter, style=newtheorem]{theorem}
    \declaretheorem[sibling=theorem,style=newtheorem]{lemma,proposition,corollary}
    \declaretheorem[sibling=theorem,style=newdefinition]{definition}
    \declaretheorem[sibling=theorem,style=newdefinition]{example}
    \declaretheorem[sibling=theorem,style=newdefinition]{notation}
    \declaretheorem[sibling=theorem,style=newremark]{remark}
    % Now we set the shading using the tcolorbox package.
    %
    % The related thmtools' option "shaded" and the package mdframed seem to have
    % issues: the former does not allow for page breaks in shaded environments and
    % the latter puts double spacing between two shaded environments.
    %
    % Since tcolorbox puts stuff inside a minipage or \parbox (according to this
    % stackexchange answer: https://tex.stackexchange.com/a/250170), new
    % paragraphs aren't indented. We can fix this by grabbing the parindent
    % value and passing it to tcbset.
    \newlength{\normalparindent}
    \AtBeginDocument{\setlength{\normalparindent}{\parindent}}
    \tcbset{shadedenv/.style 2 args={
                colback={#1},
                frame hidden,
                sharp corners,
                boxrule=0pt,
                borderline west={1mm}{0pt}{#2},
                enhanced,
                breakable,
                boxsep=0pt,
                top=3mm,
                bottom=3mm,
                left=4mm,
                right=3mm,
                % LaTeX thinks this is too wide (as becomes clear from the many "Overfull
                % \hbox" warnings, but optically it looks spot on.
                add to width=1.1mm,
                enlarge left by=-0.0mm,
                before upper={\setlength{\parindent}{\normalparindent}}
            }}
    \newcommand{\setenvcolor}[3]{%
        \tcolorboxenvironment{#1}{shadedenv={#2}{#3}}
    \addtotheorempreheadhook[#1]{
        \tikzcdset{background color=#2}
        \renewcommand{\bgcolour}{#2}
    }
}
\setenvcolor{theorem}{ShadeOfBlue}{ShadeOfDarkBlue}
\setenvcolor{lemma}{ShadeOfBlue}{ShadeOfDarkBlue}
\setenvcolor{proposition}{ShadeOfBlue}{ShadeOfDarkBlue}
\setenvcolor{corollary}{ShadeOfBlue}{ShadeOfDarkBlue}
\setenvcolor{definition}{ShadeOfYellow}{ShadeOfDarkYellow}
\setenvcolor{notation}{ShadeOfYellow}{ShadeOfDarkYellow}
\setenvcolor{example}{ShadeOfGreen}{ShadeOfDarkGreen}
\setenvcolor{remark}{ShadeOfRed}{ShadeOfDarkRed}
\setenvcolor{proof}{ShadeOfGray}{gray}
}{
% Use closing symbols if we don't have colours
\declaretheorem[sibling=equation]{theorem}
\declaretheorem[sibling=theorem]{lemma,proposition,corollary}
\declaretheorem[sibling=theorem,style=definitionwithbox]{definition}
\declaretheorem[sibling=theorem,style=definitionwithbox]{notation}
\declaretheorem[sibling=theorem,style=definitionwithbox]{example}
\declaretheorem[sibling=theorem,style=remarkwithbox]{remark}
}
\declaretheorem[sibling=theorem,style=remark,numbered=no]{claim}

% Note that proofs will still have the \qed symbol at the end, even when shaded,
% because we prefer to keep up the tradition.