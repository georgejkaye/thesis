\begin{figure}
    \begin{gather*}
        \iltikzfig{strings/symmetric/naturality-lhs}[box1=f,box2=g,colour=white,dom1=A,cod1=B,dom2=C,cod2=D]
        =
        \iltikzfig{strings/symmetric/naturality-rhs}[box1=f,box2=g,colour=white,dom1=A,cod1=B,dom2=C,cod2=D]
        \quad
        \iltikzfig{strings/symmetric/hexagon-lhs}[obj1=A,obj2=B,obj3=C,colour=white]
        =
        \iltikzfig{strings/symmetric/symmetry}[obj1=A,obj2=B \tensor C,colour=white]
        \\[0.5em]
        \iltikzfig{strings/symmetric/unit-l-lhs}[colour=white,obj=A,unit=I]
        =
        \iltikzfig{strings/category/identity}[obj=A,colour=white]
        \quad
        \iltikzfig{strings/symmetric/unit-r-lhs}[colour=white,obj=A,unit=I]
        =
        \iltikzfig{strings/category/identity}[obj=A,colour=white]
        \quad
        \iltikzfig{strings/symmetric/self-inverse-lhs}[colour=white,obj1=A,obj2=B]
        =
        \iltikzfig{strings/category/identity}[obj=A \tensor B,colour=white]
    \end{gather*}
    \caption{
        Equations of a symmetric monoidal category in string diagram notation
    }
    \label{fig:smc-equations-strings}
\end{figure}