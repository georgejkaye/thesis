\documentclass{article}

\usepackage{geometry}[a4paper, margin=1in]

\title{Foundations of Digital Circuits\\Report on Corrections}
\author{George Kaye}
\date{}

\begin{document}
\maketitle

Throughout the thesis many small typos, layout and typesetting issues have been
fixed; an index and table of symbols has also been added.
The report below details the less trivial corrections made.

\section*{Chapter 2: A crash course in category theory}

\begin{itemize}
    \item pg.37: Add formal definitions of PROP morphism, category of PROPs
    \item pg.39: Added labels (tightening, sliding, yanking, vanishing,
          superposing) to STMC axioms
\end{itemize}

\section*{Chapter 4: Denotational semantics}

\begin{itemize}
    \item pgs.57-59: Restructured how order-theoretical concepts are introduced
          \begin{itemize}
              \item pg.57, Definition 4.7: Explicitly define lower and upper
                    bounds
              \item pg.57: Introduce notation for joins and meets
              \item pg.58, Lemma 4.11: Illustrate how \(\top\) and \(\bot\) in a
                    finite lattice are obtained
          \end{itemize}
    \item pg.61, Definition 4.24: Reworded causality condition
    \item pg.62, Definition 4.26: Removed causality hypothesis from functional
          stream derivative definition
    \item pgs.62-63, Lemmas 4.31-4.33: Explicitly stated that tensor product is
          Cartesian product
    \item pgs.64-66: Discuss directed subsets and Scott-continuity to expand on
          the Kleene fixed-point theorem and the computation of least fixed point
    \item pg.68: Add theorem for proving the least fixed point satisfies the
          trace equations, in particular the sliding equation
    \item pg.70, Example 4.54: Fix Mealy coalgebraic viewpoint of stream functions
    \item pg.71, Definition 4.58: Changed monotone Mealy machine to have
          explicitly set poset structure on the states
    \item pg.83, Definition 4.90, Example 4.91, Definition 4.92: Restructured
          monotone completion definition, added example and provided types
          for application to Monotone Mealy encoding
    \item pg.84, Definition 4.94: Restated chosen state order to operate on
          \emph{accessible} states
    \item pg.87, Definition 4.102: Removed \(1\) from symbols required for
          Boolean completeness
\end{itemize}

\section*{Chapter 5: Operational semantics}

\begin{itemize}
    \item pg.100: Explained how \(\overline{s}\) is defined
    \item pg.103: Added remark about Mendler/Shiple/Berry notion of monotone circuits
    \item pg.108, Notation 5.36: Moved waveform definition to more appropriate place
    \item pg.108: Add explanation before Corollary 5.37
\end{itemize}

\section*{Chapter 6: Algebraic semantics}

\begin{itemize}
    \item pg.117, Definition 6.10: Fixed notation for states definition
    \item pg.118: Restructure narrative surrounding encoding equations
    \item pg.119, Proposition 6.14: Reworded proof
    \item pg.125, Theorem 6.27: Restructured proof
\end{itemize}

\section*{Chapter 8: String diagrams as hypergraphs}

\begin{itemize}
    \item Changed instances of `nodes' to `vertices'
    \item pgs.153-154, Example 8.23: Fixed example
    \item pg.156, Example 8.30: Fixed example
    \item pgs.165-166, Theorem 8.62: Removed reference to colours in monochromatic proof
\end{itemize}

\section*{Chapter 9: Graph rewriting}

\begin{itemize}
    \item pg.189, Definition 9.2: Used symmetric monoidal definition of term rewriting first
    \item pg.199, Proposition 9.29: \(q^{-1}\) is a set of vertices in \(i+j+\tilde{G}\);
          the identification condition here enforces that a context \(C\) is valid
          if and only if the map \(i+j \to C\) merges only vertices in this set
\end{itemize}

\section*{Chapter 10: Applications of graph rewriting}

\begin{itemize}
    \item pg.212, Theorem 10.8: Added Cartesian category to hypothesis
    \item pg.213: Added explanation about red wire
\end{itemize}

\end{document}